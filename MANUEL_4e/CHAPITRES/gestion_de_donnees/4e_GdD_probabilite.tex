\chapter{Probabilité}
{https://sacado.xyz/qcm/parcours_show_course/0/117120}
{ 

 \begin{CpsCol}
 \textbf{Les savoir-faire} 
 \begin{itemize}
 \item Utiliser le vocabulaire des probabilités : expérience aléatoire, issues, événement, probabilité, événement certain, événement impossible, événement contraire.
 \item Reconnaître des événements contraires et s’en sert pour calculer des probabilités.
 \item Calculer des probabilités.
 \item Savoir que la probabilité d'un événement est un nombre compris entre 0 et 1.
 \item Exprimer des probabilités sous diverses formes.
 
 \end{itemize}
 \end{CpsCol}
}
%
%
%\begin{pageHistoire} 
% 
%En 1585, dans son ouvrage \textbf{La Disme}, Simon Stevin (1548 - 1620) ingénieur et mathématicien flamand, propose une écriture des nombres qui permet de simplifier les calculs (quelquefois très lourds en écriture fractionnaire).\\
%
%Il est considéré comme un précurseur de l'écriture décimale.
% 
%
%\end{pageHistoire} 



%%%%%%%%%%%%%%%%%%%%%%%%%%%%%%%%%%%%%%%%%%%%%%%%%%%%%%%%%%%%%%%%%%%%%%%%%%%%%%%%%%%%
%%%%%%%%%%        Cours             %%%%%%%%%%%%%%%%%%%%%%%%%%%%%%%%%%%%%%%%%%%%%%%%
%%%%%%%%%%%%%%%%%%%%%%%%%%%%%%%%%%%%%%%%%%%%%%%%%%%%%%%%%%%%%%%%%%%%%%%%%%%%%%%%%%%%
\begin{pageCours} 

\section{Vocabulaire}



\begin{DefT}{Expérience aléatoire. Univers. Issue}
Une \textbf{expérience aléatoire}\index{Expérience aléatoire!Probabilité| see{Probabilité}} ou \textbf{épreuve aléatoire!Probabilité| see{Probabilité}}\index{Épreuve aléatoire} est une expérience qui est soumise au hasard. On connait les issues possibles sans savoir laquelle sera réalisée.

Une \textbf{issue} \index{Issue!Probabilité| see{Probabilité}}est le résultat d'une expérience aléatoire.

On appelle \textbf{univers}\index{Univers}, noté $\Omega$ \index{$\Omega$}, l'ensemble de toutes les issues possibles.
\end{DefT}

\begin{ExT}{Lancer de dé}
Je lance un dé équilibré et je note le numéro obtenu sur la face sortie. 
\begin{description}
\item L'\textit{expérience aléatoire} est le Lancer du dé cubique.
\item Une \textit{issue}\index{Issue!Probabilité| see{Probabilité}}  ou \textit{éventualités}\index{Éventualité!Probabilité| see{Probabilité}}  est le numéro obtenu sur la face sortie. Ici, il y a 6 issues possibles.
\item L'\textit{univers} est l'ensemble qui contient les nombres 1; 2; 3; 4; 5; 6. On le note $\Omega = \left\lbrace  1;2;3;4;5;6\right\rbrace $.
\end{description}
\end{ExT}

 
\section{Notion d'évènements}

\begin{DefT}{Évènement}
Un \textbf{évènement} est un ensemble d'issues (ou éventualités)\index{Évènement!Probabilité| see{Probabilité}}.
\end{DefT}


\begin{Rq}
On décrit un évènement par une action.
\end{Rq}


\begin{ExT}{Exemples d'évènements}
Je lance un dé équilibré et je note la face obtenue. 
\begin{description}
\item Obtenir un nombre pair.
\item Obtenir le 6.
\item Obtenir un nombre plus petit que 4.
\end{description}
\end{ExT}

\end{pageCours}


%%%%%%%%%%%%%%%%%%%%%%%%%%%%%%%%%%%%%%%%%%%%%%%%%%%%%%%%%%%%%%%%%%%%%%%%%%%%%%%%%%%%
%%%%%%%%%%   Application directe    %%%%%%%%%%%%%%%%%%%%%%%%%%%%%%%%%%%%%%%%%%%%%%%%
%%%%%%%%%%%%%%%%%%%%%%%%%%%%%%%%%%%%%%%%%%%%%%%%%%%%%%%%%%%%%%%%%%%%%%%%%%%%%%%%%%%%
\begin{pageAD} 

\Sf{Connaître le vocabulaire probabiliste}

\ExoCad{Communiquer.}
 
On lance une pièce de monnaie équilibrée et non truquée.

\begin{enumerate}
\item Quelle est l'expérience aléatoire ? \point{1}
\item Quelles sont les issues ?\point{1}
\item Quel est l'univers ?\point{1}
\end{enumerate}


\ExoCad{Communiquer.}

On considère un jeu de 52 cartes. On en tire une carte au hasard. 
\begin{enumerate}
\item Quelle est l'expérience aléatoire ?\point{1}
\item Donner une issue.\point{1}
\item Combien l'univers compte-t-il d'issues ?\point{1}
\end{enumerate}


\ExoCad{Communiquer.}

Le bingo est un jeu où il faut deviner 6 nombres tirés au hasard et sans remise parmi 1 et 49 sans se soucier de l'ordre.
\begin{enumerate}
\item Quelle est l'expérience aléatoire ?\point{1}
\item Donner une issue.\point{1}
\item Quel est l'univers ?\point{1}
\end{enumerate}


 


\Sf{Décrire un événement} 

 
\ExoCad{Représenter. Communiquer.}

Dans son armoire, Anis a 3 pantalons : un vert, un bleu et un rouge. Il a aussi 4 chemises : une vert, deux bleues et une rose. Il choisit au hasard un pantalon et une chemise. On s'intéresse aux événements B : "Anis est habillé tout en bleu" et V : "Anis est habillé tout en vert".
\begin{enumerate}
\item Décris l'expérience aléatoire ? \point{1}
\item Décris l'évènement V ? \point{1}
\item Décris l'évènement B ? \point{1}
\end{enumerate}




\end{pageAD}  




\begin{pageCours}

\begin{DefT}{Évènements particuliers}
\begin{itemize}[leftmargin=*]
\item Un \textbf{évènement élémentaire}\index{Évènement élémentaire!Probabilité| see{Probabilité}} est un ensemble qui contient une seule issue. 
\item Un \textbf{évènement impossible}\index{Évènement impossible!Probabilité| see{Probabilité}} dont on est sûr qu'il ne peut pas se produire.  
\item Un \textbf{évènement certain}\index{Évènement certain!Probabilité| see{Probabilité}}  dont on est sûr qu'il va se produire.  
\end{itemize}

\end{DefT}

\begin{ExT}{Exemples d'évènements}
Je lance un dé équilibré et je note la face obtenue. 

\begin{itemize}[leftmargin=*]
\item L'évènement A : "Obtenir le nombre 5" est un évènement élémentaire.  
\item L'évènement B : "Obtenir le nombre 0" est un évènement impossible.
\item L'évènement C : "Obtenir un nombre compris entre 1 et 6" est un événement certain.
\end{itemize}

\end{ExT}

\section{Probabilité d'un évènement}

\begin{DefT}{Probabilité d'un évènement}
Lorsqu'une expérience est répétée un grand nombre de fois, on assimile la fréquence d'apparition d'un évènement $A$ à sa probabilité\index{Probabilité} et on note la $p(A)$.
\end{DefT}
 
\begin{Pps}
\begin{itemize}[leftmargin=*]
\item La probabilité d'un évènement est la somme des probabilités des évènements élémentaires qui le composent.  
\item La somme des probabilités de tous les évènements élémentaires qui composent l'univers est égale à 1.  
\item La probabilité d'un \textbf{évènement impossible} est égale à 0.  
\item La probabilité d'un \textbf{évènement certain} est égale à 1.  
\end{itemize}
\end{Pps}

\begin{Ex}
Rosi lance un dé truqué dont la probabilité de chaque face est proportionnelle au nombre de la face.
Elle s'intéresse à l'événement F : "Obtenir une face paire".  

  L'évènement $F$ est $\lbrace 2;4;6 \rbrace$, constitué des évènements élémentaires $\lbrace 2 \rbrace$ , $\lbrace 4 \rbrace$ , $\lbrace 6 \rbrace$ 
 

La probabilité de $F$ est donc $p(F) = p\left( \lbrace 2 \rbrace\right) + p\left(\lbrace 4 \rbrace\right) + p\left(\lbrace 6 \rbrace \right)  =\dfrac{2}{21} + \dfrac{4}{21} + \dfrac{6}{21} = \dfrac{12}{21}$


\end{Ex}


\end{pageCours} 
%%%%%%%%%%%%%%%%%%%%%%%%%%%%%%%%%%%%%%%%%%%%%%%%%%%%%%%%%%%%%%%%%%%%%%%%%%%%%%%%%%%%
%%%%%%%%%%   Application directe    %%%%%%%%%%%%%%%%%%%%%%%%%%%%%%%%%%%%%%%%%%%%%%%%
%%%%%%%%%%%%%%%%%%%%%%%%%%%%%%%%%%%%%%%%%%%%%%%%%%%%%%%%%%%%%%%%%%%%%%%%%%%%%%%%%%%%
\begin{pageAD} 

\Sf{Décrire un évènement}

\ExoCad{Chercher.}
 
 
Un jeu de 54 cartes n'est pas truqué. On tire aléatoirement une carte.
\begin{enumerate}
\item Que contient l'évènement R : "Obtenir un roi" ?\point{2}
\item Déterminer un évènement élémentaire. \point{1}
\item Déterminer un évènement impossible.\point{1}
\end{enumerate}



\Sf{Calculer une probabilité}

\ExoCad{Chercher.}
 
 
Un jeu de 32 cartes n'est pas truqué. On titre aléatoirement une carte.
\begin{enumerate}
\item Déterminer la probabilité de tirer un trèfle.
\item Déterminer la probabilité de tirer un roi.
\item Déterminer la probabilité de tirer le roi de cœur.
\end{enumerate}
 

 
 

\ExoCad{Chercher. Communiquer.}

L'expérience consiste à choisir un élève au hasard. 

\begin{enumerate}
\item Complète le tableau ci-desous :

\begin{tabular}{|c|c|c|c|}
\hline 
 & Sportif & Non sportif &  \\ 
\hline 
Filles & $35$ &   & $55$ \\ 
\hline 
Garçons & $25$ & $20$ &  \\ 
\hline 
 &  &  & $100$ \\ 
\hline 
\end{tabular} 


\item Quelle est la probabilité que ce soit un sportif ? \point{1}
\item Quelle est la probabilité que ce soit une fille sportive ?\point{1}
\item Quelle est la probabilité que ce soit un garçon non sportif ?\point{1}

\end{enumerate}

 

\end{pageAD} %%%%%%%%%%%%%%%%%%%%%%%%%%%%%%%%%%%%%%%%%%%%%%%%%%%%%%%%%%%%%%%%%%%%%%%%%%%%%%%%%%%%
%%%%%%%%%%   Application directe    %%%%%%%%%%%%%%%%%%%%%%%%%%%%%%%%%%%%%%%%%%%%%%%%
%%%%%%%%%%%%%%%%%%%%%%%%%%%%%%%%%%%%%%%%%%%%%%%%%%%%%%%%%%%%%%%%%%%%%%%%%%%%%%%%%%%%

\begin{pageCours}

\section{Représentation d'une expérience}

\begin{DefT}{Arbre de dénombrement}
Lorsqu'une expérience est répétée un grand nombre de fois, on assimile la fréquence d'apparition d'un évènement $A$ à sa probabilité\index{Probabilité} et on note la $p(A)$.

\end{DefT}
 



\begin{ExT}{Arbre de dénombrement}

Dans une urne, il y a 2 boules vertes et 5 boules rouges.
 \begin{description}
\item Si une boule verte est tirée alors le joueur tire une autre boule.
\item Si une boule rouge est tirée alors le joueur perd la partie.
\end{description}

Représentons cette expérience par un arbre de dénombrement.
\end{ExT}
 





\begin{DefT}{Tableau à double entrée}
zz
\end{DefT}





\end{pageCours} 
%%%%%%%%%%%%%%%%%%%%%%%%%%%%%%%%%%%%%%%%%%%%%%%%%%%%%%%%%%%%%%%%%%%%%%%%%%%%%%%%%%%%
%%%%%%%%%%   Application directe    %%%%%%%%%%%%%%%%%%%%%%%%%%%%%%%%%%%%%%%%%%%%%%%%
%%%%%%%%%%%%%%%%%%%%%%%%%%%%%%%%%%%%%%%%%%%%%%%%%%%%%%%%%%%%%%%%%%%%%%%%%%%%%%%%%%%%
\begin{pageAD} 

\Sf{Reconnaître une situation de  }



\end{pageAD} 
%%%%%%%%%%%%%%%%%%%%%%%%%%%%%%%%%%%%%%%%%%%%%%%%%%%%%%%%%%%%%%%%%%%%%%%%%%%%%%%%%%%%
%%%%%%%%%%        Parcours 1    %%%%%%%%%%%%%%%%%%%%%%%%%%%%%%%%%%%%%%%%%%%%%%%%%%%%
%%%%%%%%%%%%%%%%%%%%%%%%%%%%%%%%%%%%%%%%%%%%%%%%%%%%%%%%%%%%%%%%%%%%%%%%%%%%%%%%%%%%
\begin{pageParcoursu} 

\ExoCu{Calculer.}
 

Je jette une pièce de monnaie en l'air et je m'intéresse à l'événement $F$: "La pièce montre la face Face".
\begin{enumerate}
\item Quelle est \textit{a priori} la probabilité de l'événement $F$ ?
\item Est ce toujours cette valeur ?
\end{enumerate}
 


 \ExoCu{Calculer.}


Une urne contient 1 boule rouge et 4 boules oranges. 

\begin{enumerate}[leftmargin=*]
\item Combien y a-t-il de chances de tirer une boule orange ? \point{5}
\item À quelle probabilité cela correspond-il ?\point{1}
\end{enumerate}
 
 
\end{pageParcoursu}
%%%%%%%%%%%%%%%%%%%%%%%%%%%%%%%%%%%%%%%%%%%%%%%%%%%%%%%%%%%%%%%%%%%%%%%%%%%%%%%%%%%%
%%%%%%%%%%        Parcours 2    %%%%%%%%%%%%%%%%%%%%%%%%%%%%%%%%%%%%%%%%%%%%%%%%%%%%
%%%%%%%%%%%%%%%%%%%%%%%%%%%%%%%%%%%%%%%%%%%%%%%%%%%%%%%%%%%%%%%%%%%%%%%%%%%%%%%%%%%%
\begin{pageParcoursd} 
 
 
\ExoCd{Chercher.}
 
 
Dans la classe il y a 25 élèves dont 19 sont droitiers, aucun n'est ambidextre. Je choisis au hasard un élève et je m'intéresse à la probabilité qu'il soit gaucher.
\begin{enumerate}
\item Quelle est l'expérience aléatoire ?
\item Nomme clairement l'événement auquel je m'intéresse.
\item Détermine la probabilité de cet événement.
\end{enumerate}
 
 
 
\ExoCd{Calculer.}
 
 \begin{minipage}{.6\linewidth}
 
 On considère une urne contenant des boules blanches ou grises, et numérotées ci-contre :
  \begin{itemize}[leftmargin=*]
\item Si on s'intéresse à la couleur de la boule, quelles sont les issues possibles ?\point{1}
\item Si on s'intéresse au numéro écrit sur la boule, quelles sont les issues possibles ?\point{1}
\item Donne un événement certain de se réaliser. \point{1}
\item Donne un événement impossible. \point{1}
 \end{itemize}

 \end{minipage}
 \begin{minipage}{.4\linewidth}
\includegraphics[scale=1]{FIG/boule.jpg} 

 \end{minipage}
 
\ExoCd{Calculer.}

La probabilité de gagner à un jeu est égale 0,4. Calcule la probabilité de perdre. \point{3}

\ExoCd{Calculer.}


Une urne contient 1 boule rouge et 4 boules oranges. 

\begin{enumerate}[leftmargin=*]
\item Combien y a-t-il de chances de tirer une boule orange ? \point{5}
\item À quelle probabilité cela correspond-il ?\point{1}
\end{enumerate}

\end{pageParcoursd}
%%%%%%%%%%%%%%%%%%%%%%%%%%%%%%%%%%%%%%%%%%%%%%%%%%%%%%%%%%%%%%%%%%%%%%%%%%%%%%%%%%%%
%%%%%%%%%%        Parcours 3    %%%%%%%%%%%%%%%%%%%%%%%%%%%%%%%%%%%%%%%%%%%%%%%%%%%%
%%%%%%%%%%%%%%%%%%%%%%%%%%%%%%%%%%%%%%%%%%%%%%%%%%%%%%%%%%%%%%%%%%%%%%%%%%%%%%%%%%%%
\begin{pageParcourst}

\ExoCt{Calculer.}

Dans une urne, il y a 6 boules vertes, 2 boules rouges et 1 boule noire.
On appelle 
\begin{description}
\item V l'événement :"Tirer une boule verte"
\item R l'événement :"Tirer une boule rouge"
\item N l'événement :"Tirer une boule noire"
\item B l'événement :"Tirer une boule bleu"
\end{description}


\begin{enumerate}
\item Cite un événement élémentaire. \point{1}
\item Cite un événement impossible. \point{1}
\item Calcule la probabilité de $p(V)$. \point{1}
\item Calcule la probabilité de $p(B)$. \point{1}
\end{enumerate}


\ExoCt{Calculer.}

Je lance un dé truqué dont la probabilité de chaque face est donnée dans le tableau :

\begin{tabular}{|c|c|c|c|c|c|c|c|}
\hline 
Face & 1 & 2 & 3 & 4 & 5 & 6 & somme \\ 
\hline 
Probabilité & $\dfrac{1}{21}$ & $\dfrac{2}{21}$  & $\dfrac{3}{21}$  & $\dfrac{4}{21}$  & $\dfrac{5}{21}$  & $\dfrac{6}{21}$  & $\ldots$ \\ 
\hline 
\end{tabular}

Est-il plus probable d'obtenir une face paire ou une des faces 5 ou 6 ? \point{3}

\ExoCt{Calculer.}
 
 On lance un dé et on s'intéresse à la probabilité d'obtenir la face 1.
\begin{enumerate}[leftmargin=*]	
\item Estime cette probabilité.
\item Ouvrir une feuille de calcul dans un tableur.
\begin{enumerate}
\item Avec 100 valeurs
\begin{enumerate}
\item Écrire dans la cellule A1 :"=ALEA.ENTRE.BORNES(1;6)". Que fait l'instruction ALEA.ENTRE.BORNES ?
\item Glisser copier la cellule A1 dans la plage "A1:A10" puis la plage "A1:A10" dans "A1:J10".
\item Écrire dans la cellule K1 :"=NB.SI(A1:I10;1)". Que fait l'instruction NB.SI ?
\item Calculer dans la cellule L1 la fréquence d'apparition du nombre 1 dans "A1:J10".
\item Que constates-tu par rapport à ton estimation ?
\end{enumerate}
\item Avec 1000 valeurs
\begin{enumerate}
\item Glisser copier la ligne "A1:I1" jusqu'à la ligne "A100:I100".
\item Modifier la cellule K1 pour obtenir le nombre d'apparaitre de 1 parmi ces \np{1000} valeurs.
\item Modifier la cellule L1 pour calculer la fréquence d'apparition du nombre 1 dans la plage A1:J100.
\item Que constates-tu par rapport à ton estimation ?
\end{enumerate}
\item Avec 10000 valeurs
\begin{enumerate}
\item Glisser copier la ligne "A1:I1" jusqu'à la ligne "A1000:I1000".
\item Modifier la cellule K1 pour obtenir le nombre d'apparition de 1 parmi ces \np{10000} valeurs.
\item Modifier la cellule L1 pour calculer la fréquence d'apparition du nombre 1 dans la plage A1:J1000.
\item Que constates-tu par rapport à ton estimation ?
\end{enumerate}
\end{enumerate}
\end{enumerate}


\end{pageParcourst}
%%%%%%%%%%%%%%%%%%%%%%%%%%%%%%%%%%%%%%%%%%%%%%%%%%%%%%%%%%%%%%%%%%%%%%%%%%%%%%%%%%%%
%%%%%%%%%%   Auto-evaluation    %%%%%%%%%%%%%%%%%%%%%%%%%%%%%%%%%%%%%%%%%%%%%%%%%%%%
%%%%%%%%%%%%%%%%%%%%%%%%%%%%%%%%%%%%%%%%%%%%%%%%%%%%%%%%%%%%%%%%%%%%%%%%%%%%%%%%%%%%
\begin{pageAuto}
 
\ExoAuto

 \ExoCad{Communiquer.}

 \begin{enumerate}
\item A-t-on plus de chance de tirer une boule blanche dans :
\begin{description}
\item[a.] une urne A qui contient 3 boules toutes blanches ?
\item ou
\item[b.] une urne B qui contient 500 boules blanches et une boule rouge ?
\end{description}
\item A-t-on plus de chance de tirer une boule blanche ou une boule noire dans chacune  
des urnes ci-dessous ?
\end{enumerate}


\definecolor{ttzzqq}{rgb}{0.2,0.6,0.}
\definecolor{xfqqff}{rgb}{0.4980392156862745,0.,1.}
\definecolor{qqqqff}{rgb}{0.,0.,1.}
\definecolor{ffdxqq}{rgb}{1.,0.8431372549019608,0.}
\definecolor{ffqqqq}{rgb}{1.,0.,0.}
\begin{tikzpicture}[line cap=round,line join=round,>=triangle 45,x=0.7730771794105831cm,y=0.7730771794105831cm]
\clip(-4.705366855860613,-10.542921242507836) rectangle (14.697611785206563,-1.212189475625239);
\draw [fill=black,fill opacity=1.0] (11.669758430390349,-7.515067887691629) circle (1.6781052296417425cm);
\draw [fill=black,fill opacity=1.0] (1.6284080190100414,-7.051620945627923) circle (1.6781052296417456cm);
\draw [->,line width=5.2pt] (-3.,3.) -- (9.,3.);
\draw [rotate around={5.24181994697955:(-1.93,3.84)},color=ffqqqq,fill=ffqqqq,fill opacity=0.59] (-1.93,3.84) ellipse (1.0259941224374605cm and 0.5801909002249424cm);
\draw (-2.9133720132142815,4.071105663900999) node[anchor=north west] {improbable};
\draw [rotate around={5.241819946979561:(0.09,2.1)},color=ffdxqq,fill=ffdxqq,fill opacity=0.59] (0.09,2.1) ellipse (1.0259941224374631cm and 0.5801909002249442cm);
\draw (-1.2140665589806914,2.3409037468631673) node[anchor=north west] {peu probable};
\draw [rotate around={5.241819946979548:(4.85,3.76)},color=qqqqff,fill=qqqqff,fill opacity=0.22] (4.85,3.76) ellipse (1.0259941224374611cm and 0.5801909002249431cm);
\draw (3.9456427293285734,4.0402092010967525) node[anchor=north west] {probable};
\draw [rotate around={5.241819946979548:(6.89,2.34)},color=xfqqff,fill=xfqqff,fill opacity=0.23] (6.89,2.34) ellipse (1.0259941224374616cm and 0.5801909002249431cm);
\draw (5.614051720757916,2.61897191210139) node[anchor=north west] {très probable};
\draw [color=ttzzqq](8.332940447531662,4.225587977922235) node[anchor=north west] {Certain};
\draw [->,line width=1.2pt] (8.98,3.74) -- (8.98,2.98);
\draw (-4.,-2.)-- (-4.,-10.);
\draw (-4.,-10.)-- (4.,-10.);
\draw (4.,-10.)-- (4.,-2.);
\draw (6.,-2.)-- (6.,-10.);
\draw (6.,-10.)-- (14.,-10.);
\draw (14.,-10.)-- (14.,-2.);
\draw(-2.07916751749961,-7.391482036474641) circle (1.0043214669567795cm);
\draw(7.776804117055214,-4.363628681658434) circle (1.0043214669567797cm);
\draw(7.529632414621236,-7.360585573670393) circle (1.0043214669567775cm);
\draw(11.360793802347878,-3.7765958883777406) circle (1.0043214669567797cm);
\begin{scriptsize}
\draw [color=black] (-3.,3.)-- ++(-4.5pt,0 pt) -- ++(9.0pt,0 pt) ++(-4.5pt,-4.5pt) -- ++(0 pt,9.0pt);
\draw [color=black] (3.,3.)-- ++(-4.5pt,0 pt) -- ++(9.0pt,0 pt) ++(-4.5pt,-4.5pt) -- ++(0 pt,9.0pt);
\end{scriptsize}
\end{tikzpicture}

\end{pageAuto}
%%%%%%%%%%%%%%%%%%%%%%%%%%%%%%%%%%%%%%%%%%%%%%%%%%%%%%%%%%%%%%%%%%%%%%%%%%%%%%%%%%%%
% Parcours 3
%%%%%%%%%%%%%%%%%%%%%%%%%%%%%%%%%%%%%%%%%%%%%%%%%%%%%%%%%%%%%%%%%%%%%%%%%%%%%%%%%%%%
\begin{pageBrouillon} 
 
\ligne{30}

\end{pageBrouillon}


