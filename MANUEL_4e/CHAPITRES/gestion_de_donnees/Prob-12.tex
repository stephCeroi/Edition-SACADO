
On lance un dé et on s'intéresse à la probabilité d'obtenir la face 1.
\begin{enumerate}
\item Estime cette probabilité.
\item Ouvrir une feuille de calcul dans un tableur.
\begin{enumerate}
\item Avec 100 valeurs
\begin{enumerate}
\item Écrire dans la cellule A1 :"=ALEA.ENTRE.BORNES(1;6)". Que fait l'instruction ALEA.ENTRE.BORNES ?
\item Glisser copier la cellule A1 dans la plage "A1:A10" puis la plage "A1:A10" dans "A1:J10".
\item Écrire dans la cellule K1 :"=NB.SI(A1:I10;1)". Que fait l'instruction NB.SI ?
\item Calculer dans la cellule L1 la fréquence d'apparition du nombre 1 dans "A1:J10".
\item Que constates-tu par rapport à ton estimation ?
\end{enumerate}
\item Avec 1000 valeurs
\begin{enumerate}
\item Glisser copier la ligne "A1:I1" jusqu'à la ligne "A100:I100".
\item Modifier la cellule K1 pour obtenir le nombre d'apparaitre de 1 parmi ces \np{1000} valeurs.
\item Modifier la cellule L1 pour calculer la fréquence d'apparition du nombre 1 dans la plage A1:J100.
\item Que constates-tu par rapport à ton estimation ?
\end{enumerate}
\item Avec 10000 valeurs
\begin{enumerate}
\item Glisser copier la ligne "A1:I1" jusqu'à la ligne "A1000:I1000".
\item Modifier la cellule K1 pour obtenir le nombre d'apparition de 1 parmi ces \np{10000} valeurs.
\item Modifier la cellule L1 pour calculer la fréquence d'apparition du nombre 1 dans la plage A1:J1000.
\item Que constates-tu par rapport à ton estimation ?
\end{enumerate}
\end{enumerate}
\end{enumerate}
