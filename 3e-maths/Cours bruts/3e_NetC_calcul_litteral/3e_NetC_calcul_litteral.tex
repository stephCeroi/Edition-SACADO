\documentclass[a4paper,dvipsnames]{article}

\addtolength{\hoffset}{-2.25cm}
\addtolength{\textwidth}{4.5cm}
\addtolength{\voffset}{-3.25cm}
\addtolength{\textheight}{5cm}
\setlength{\parskip}{0pt}
\setlength{\parindent}{0in}

%----------------------------------------------------------------------------------------
%	PACKAGES AND OTHER DOCUMENT CONFIGURATIONS
%----------------------------------------------------------------------------------------

%----------------------------------------------------------------------------------------
%		Generals
%----------------------------------------------------------------------------------------
\usepackage{fourier}
\usepackage{frcursive}
\usepackage[T1]{fontenc} %Accents handling
\usepackage[utf8]{inputenc} % Use UTF-8 encoding
%\usepackage{microtype} % Slightly tweak font spacing for aesthetics
\usepackage[english, francais]{babel} % Language hyphenation and typographical rules

%----------------------------------------------------------------------------------------
%		Graphics
%----------------------------------------------------------------------------------------
\usepackage{xcolor}
\usepackage{graphicx, multicol} % Enhanced support for graphics
\graphicspath{{FIG/}}
\usepackage{wrapfig}

%----------------------------------------------------------------------------------------
%		Other packages
%----------------------------------------------------------------------------------------
\usepackage{hyperref}
\hypersetup{
	colorlinks=true, %colorise les liens
	breaklinks=true, %permet le retour à la ligne dans les liens trop longs
	urlcolor= bleu3,  %couleur des hyperliens
	linkcolor= bleu3, %couleur des liens internes
	plainpages=false  %pour palier à "Bookmark problems can occur when you have duplicate page numbers, for example, if you have a page i and a page 1."
}
\usepackage{tabularx}
\newcolumntype{M}[1]{>{\arraybackslash}m{#1}} %Defines a scalable column type in tabular
\usepackage{booktabs} % Enhances quality of tables
\usepackage{diagbox} % barre en diagonale dans un tableau
\usepackage{multicol}
\usepackage[explicit]{titlesec}


%----------------------------------------------------------------------------------------
%		Headers and footers
%----------------------------------------------------------------------------------------
\usepackage{fancyhdr} % Headers and footers
\pagestyle{fancy} % All pages have headers and footers
\fancyhead{}\renewcommand{\headrulewidth}{0pt} % Blank out the default header
\renewcommand{\footrulewidth}{0pt}
\fancyfoot[L]{} % Custom footer text
\fancyfoot[C]{\href{https://sacado.xyz/}{sacado.xyz}} % Custom footer text
\fancyfoot[R]{\thepage} % Custom footer text

%----------------------------------------------------------------------------------------
%		Mathematics packages
%----------------------------------------------------------------------------------------
\usepackage{amsthm, amsmath, amssymb} % Mathematical typesetting
\usepackage{marvosym, wasysym} % More symbols
\usepackage[makeroom]{cancel}
\usepackage{xlop}
\usepackage{pgf,tikz,pgfplots}
\pgfplotsset{compat=1.15}
\usetikzlibrary{positioning}
%\usetikzlibrary{arrows}
\usepackage{pst-plot,pst-tree,pst-func, pstricks-add,pst-node,pst-text}
\usepackage{units}
\usepackage{nicefrac}
\usepackage[np]{numprint} %Séparation milliers dans un nombre

%----------------------------------------------------------------------------------------
%		New text commands
%----------------------------------------------------------------------------------------
\usepackage{calc}
\usepackage{boites}
 \renewcommand{\arraystretch}{1.6}

%%%%% Pour les imports.
\usepackage{import}

%%%%% Pour faire des boites
\usepackage[tikz]{bclogo}
\usepackage{bclogo}
\usepackage{framed}
\usepackage[skins]{tcolorbox}
\tcbuselibrary{breakable}
\tcbuselibrary{skins}
\usetikzlibrary{babel,arrows,shadows,decorations.pathmorphing,decorations.markings,patterns}

%%%%% Pour les symboles et les ensembles
\newcommand{\pp}{\leq}
\newcommand{\pg}{\geq}
%\newcommand{\euro}{\eurologo{}}
\newcommand{\R}{\mathbb{R}}
\newcommand{\N}{\mathbb{N}}
\newcommand{\D}{\mathbb{D}}
\newcommand{\Z}{\mathbb{Z}}
\newcommand{\Q}{\mathbb{Q}}
\newcommand{\C}{\mathbb{C}}

%%%%% Pour une double minipage
\newcommand{\mini}[2]{
\begin{minipage}[t]{0.48\linewidth}
#1
\end{minipage}
\hfill
\begin{minipage}[t]{0.48\linewidth}
#2
\end{minipage}
}


%\newcommand\hole[1]{\texttt{\_}}
%\newcommand{\PROP}[1]{\textbf{\underline{#1}}}
%\newcommand{\exercice}{\textcolor{OliveGreen}{Exercice : }}
%\newcommand{\correction}{\textcolor{BurntOrange}{Correction : }}
%\newcommand{\propriete}{\textbf{\underline{Propriété}} : }
%\newcommand{\prop}{\textbf{\underline{Propriété}} : }
%\newcommand{\vocabulaire}{\textbf{\underline{Vocabulaire}} : }
%\newcommand{\voca}{\textbf{\underline{Vocabulaire}} : }

\usepackage{enumitem}
\newlist{todolist}{itemize}{2} %Pour faire des QCM
\setlist[todolist]{label=$\square$} %Pour faire des QCM \begin{todolist} instead of itemize

%----------------------------------------------------------------------------------------
%		Définition de couleur pour geogebra
%----------------------------------------------------------------------------------------
\definecolor{zzttqq}{rgb}{0.6,0.2,0.} %rouge des polygones
\definecolor{qqqqff}{rgb}{0.,0.,1.}
\definecolor{xdxdff}{rgb}{0.49019607843137253,0.49019607843137253,1.}%bleu
\definecolor{qqwuqq}{rgb}{0.,0.39215686274509803,0.} %vert des angles
\definecolor{ffqqqq}{rgb}{1.,0.,0.} %rouge vif
\definecolor{uuuuuu}{rgb}{0.26666666666666666,0.26666666666666666,0.26666666666666666}
\definecolor{qqzzqq}{rgb}{0.,0.6,0.}
\definecolor{cqcqcq}{rgb}{0.7529411764705882,0.7529411764705882,0.7529411764705882} %gris
\definecolor{qqffqq}{rgb}{0.,1.,0.}
\definecolor{ffdxqq}{rgb}{1.,0.8431372549019608,0.}
\definecolor{ffffff}{rgb}{1.,1.,1.}
\definecolor{ududff}{rgb}{0.30196078431372547,0.30196078431372547,1.}

%-------------------------------------------------
%
%	EN TETE
%
%-------------------------------------------------

% Classe
\newcommand{\myClasse}   
{
    6e
}

% Discipline
\newcommand{\myDiscipline}   
{
    Mathématiques
}

% Parcours
\newcommand{\myParcours}
{
  Nombres et Calculs
}

%Titre de la séquence
\newcommand{\myTitle}
{
    \scshape\huge
\textcolor{sacado_purple}{
		Enchainements d'opérations
}
}

%----------------------------------------------------------------------------------------

%----------------------------------------------------------------------------------------
%		Définition de couleur pour les boites
%----------------------------------------------------------------------------------------
\definecolor{bleu1}{rgb}{0.54,0.79,0.95} %% Bleu
\definecolor{sapgreen}{rgb}{0.4, 0.49, 0}
\definecolor{dvzfxr}{rgb}{0.7,0.4,0.}
\definecolor{beamer}{rgb}{0.5176470588235295,0.49019607843137253,0.32941176470588235} % couleur beamer
\definecolor{preuveRbeamer}{rgb}{0.8,0.4,0}
\definecolor{sectioncolor}{rgb}{0.24,0.21,0.44}
\definecolor{subsectioncolor}{rgb}{0.1,0.21,0.61}
\definecolor{subsubsectioncolor}{rgb}{0.1,0.21,0.61}
\definecolor{info}{rgb}{0.82,0.62,0}
\definecolor{bleu2}{rgb}{0.38,0.56,0.68}
\definecolor{bleu3}{rgb}{0.24,0.34,0.40}
\definecolor{bleu4}{rgb}{0.12,0.20,0.25}
\definecolor{vert}{rgb}{0.21,0.33,0}
\definecolor{vertS}{rgb}{0.05,0.6,0.42}
\definecolor{red}{rgb}{0.78,0,0}
\definecolor{color5}{rgb}{0,0.4,0.58}
\definecolor{eduscol4B}{rgb}{0.19,0.53,0.64}
\definecolor{eduscol4P}{rgb}{0.62,0.12,0.39}

%----------------------------------------------------------------------------------------
%		Définition de couleur pour les boites SACADO
%----------------------------------------------------------------------------------------
\definecolor{sacado_blue}{RGB}{0,129,159} %% Bleu Sacado
\definecolor{sacado_green}{RGB}{59, 157, 38} %% Vert Sacado
\definecolor{sacado_yellow}{RGB}{255,180,0} %% Jaune Sacado
\definecolor{sacado_purple}{RGB}{94,68,145} %% Violet foncé Sacado
\definecolor{sacado_violet}{RGB}{153,117,224} %% Violet clair Sacado
\definecolor{sacado_orange}{RGB}{249,168,100} %% Orange Sacado
\definecolor{ill_frame}{HTML}{F0F0F0}
\definecolor{ill_back}{HTML}{F7F7F7}
\definecolor{ill_title}{HTML}{AAAAAA}


 % Compteurs pour Théorème, Définition, Exemple, Remarque, .....
\newcounter{cpttheo}
\setcounter{cpttheo}{0}
\newcounter{cptdef}
\setcounter{cptdef}{0}
\newcounter{cptmth}
\setcounter{cptmth}{0}
\newcounter{cpttitre}
\setcounter{cpttitre}{0}
 % Exercices
\newcounter{cptapp}
\setcounter{cptapp}{0}
\newcounter{cptex}
\setcounter{cptex}{0}
\newcounter{cptsr}
\setcounter{cptsr}{0}
\newcounter{cpti}
\setcounter{cpti}{0}
\newcounter{cptcor}
\setcounter{cptcor}{0}




%%%%% Pour réinitialiser numéros des paragraphes après une nouvelle partie
\makeatletter
    \@addtoreset{paragraph}{part}
\makeatother


%%%% Titres et sections

\newlength\chapnumb
\setlength\chapnumb{3cm}


% \titleformat{\part}[block] {
 % \normalfont\sffamily\color{violet}}{}{0pt} {
   % \parbox[t]{\chapnumb}{\fontsize{120}{110}\selectfont\ding{110}}
   % \parbox[b]{\dimexpr\textwidth-\chapnumb\relax}{
       % \raggedleft
       % \hfill{{\color{bleu3}\fontsize{40}{30}\selectfont#1}}\\
       % \rule{0.99\textwidth-\chapnumb\relax}{0.4pt}
 % }
% }

% \titleformat{name=\part,numberless}[block]
% {\normalfont\sffamily\color{bleu3}}{}{0pt}
% {\parbox[b]{\chapnumb}{%
  % \mbox{}}%
 % \parbox[b]{\dimexpr\textwidth-\chapnumb\relax}{%
   % \raggedleft%
   % \hfill{{\color{bleu3}\fontsize{40}{30}\selectfont#1}}\\
   % \rule{0.99\textwidth-\chapnumb\relax}{0.4pt}
 % }
% }



% \titleformat{\chapter}[block] {
 % \normalfont\sffamily\color{violet}}{}{0pt} {
   % \parbox[t]{\chapnumb}{ 
     % \fontsize{120}{110}\selectfont\thechapter}
     % \parbox[b]{\textwidth-\chapnumb}{
       % \raggedleft
       % \hfill{{\color{bleu3}\huge#1}}\\  
  % \ifthenelse{\thechapter<10}{\rule{0.99\textwidth-\chapnumb}{0.4pt}}{\rule{0.9\textwidth - \chapnumb}{0.4pt}}
       % \setcounter{cpttitre}{0}
	% \setcounter{cptapp}{0}
	% \setcounter{cptex}{0}
	% \setcounter{cptsr}{0}
	% \setcounter{cpti}{0}
	% \setcounter{cptcor}{0} 
 % }
% }

% \titleformat{name=\chapter,numberless}[block]
% {\normalfont\sffamily\color{bleu3}}{}{0pt}
% {\parbox[b]{\chapnumb}{%
  % \mbox{}}%
 % \parbox[b]{\textwidth-\chapnumb}{%
   % \raggedleft
   % \hfill{{\color{bleu3}\huge#1}}\\
   % \ifthenelse{\thechapter<10}{\rule{0.99\textwidth-\chapnumb}{0.4pt}}{ \rule{0.9\textwidth - \chapnumb}{0.4pt}}
       % \setcounter{cpttitre}{0}
	% \setcounter{cptapp}{0}
	% \setcounter{cptex}{0}
	% \setcounter{cptsr}{0}
	% \setcounter{cpti}{0}
	% \setcounter{cptcor}{0} 
 % }
% }
%
%       
%
%%%%% Personnalisation des numéros des sections
\renewcommand\thesection{\Roman{section}. }
\renewcommand\thesubsection{\hspace{1cm}\arabic{subsection}. }
\renewcommand\thesubsubsection{\hspace{2cm}\alph{subsubsection}. }

\titleformat{\section}[hang]{\color{sacado_purple}{}\normalfont\filright\huge}{}{0.4em}{\textbf{\thesection  #1}}   
% \titlespacing*{\section}{0.2pt}{0ex plus 0ex minus 0ex}{0.3em}
   
\titleformat{\subsection}[hang]{\color{sacado_purple}{}\normalfont\filright\Large}{}{0.4em}{\thesubsection
 #1}            
\titleformat{\subsubsection}[hang]{\color{sacado_purple}{}\normalfont\filright\large}{}{0.4em}{\thesubsubsection
 #1}
\titleformat{\paragraph}[hang]{\color{black}{}\normalfont\filright\normalsize}{}{0.4em}{#1}



%%%%%%%%%%%%%%%%%%%%% Cycle 4
%\newcommand{\Titre}[2]{\section*{#1 
%\ifthenelse{\equal{#2}{1}}   {\hfill{ \ding{182}  \ding{173} \ding{174} } \addcontentsline{toc}{section}{#1 \ding{182}} }%
%{%
%\ifthenelse{\equal{#2}{2}}{\hfill{ \ding{172}  \ding{183} \ding{174} } \addcontentsline{toc}{section}{#1 {\color{purple}\ding{183}}} }{%           
%\hfill{ \ding{172}  \ding{173} \ding{184} } \addcontentsline{toc}{section}{#1 {\color{orange}\ding{184}}}% 
%}%
%}%
%}
%}


%%%%%%%%%%%%%%%%%%%%% Cycle 4
\newcommand{\Titre}[2]{\section*{#1 
\ifthenelse{\equal{#2}{1}}   {\hfill{ \ding{182}  \ding{173} \ding{174} } \addcontentsline{toc}{section}{#1 \, \ding{182}} }%
{% sinon
\ifthenelse{\equal{#2}{1,5}}   {\hfill{ \ding{182}  \ding{183} \ding{174} } \addcontentsline{toc}{section}{#1 \, \ding{182} {\color{purple}\ding{183}}} }%
{% sinon
\ifthenelse{\equal{#2}{2}}   {\hfill{ \ding{172}  \ding{183} \ding{174} } \addcontentsline{toc}{section}{#1 \, {\color{purple}\ding{183}}} }
{% sinon
\ifthenelse{\equal{#2}{2,5}}   {\hfill{ \ding{172}  \ding{183} \ding{184} } \addcontentsline{toc}{section}{#1 \, {\color{purple}\ding{183}}  {\color{orange}\ding{184}}} }%
{% sinon
\hfill{ \ding{172}  \ding{173} \ding{184} } \addcontentsline{toc}{section}{#1 \,{\color{orange}\ding{184}}}% 
}%
}%
}%
}%
}%
}

%%%%%%%%%%%%% Titre
\newenvironment{titre}[2][]{%
\vspace{0.5cm}
\begin{tcolorbox}[enhanced, lifted shadow={0mm}{0mm}{0mm}{0mm}%
{black!60!white}, attach boxed title to top left={xshift=110mm, yshift*=-3mm}, coltitle=violet, colback=bleu3!25!white, boxed title style={colback=white!100}, colframe=bleu3,title=\stepcounter{cpttitre} \textbf{Fiche \thecpttitre}. #1 #2 ]}
{%
\end{tcolorbox}
\par}



%%%%%%%%%%%%% Définitions
\newenvironment{Def}[1][]{%
\medskip \begin{tcolorbox}[widget,colback=sacado_violet!0,colframe=sacado_violet!75,
adjusted title= \stepcounter{cptdef} Définition \thecptdef . {#1} ]}
{%
\end{tcolorbox}\par}


\newenvironment{DefT}[2][]{%
\medskip \begin{tcolorbox}[widget,colback=sacado_violet!0,colframe=sacado_violet!75,
adjusted title= \stepcounter{cptdef} Définition \thecptdef . {#1} \textit{#2}]}
{%
\end{tcolorbox}\par}

%%%%%%%%%%%%% Proposition
\newenvironment{Prop}[1][]{%
\medskip \begin{tcolorbox}[widget,colback=sacado_blue!0,colframe=sacado_blue!75!black,
adjusted title= \stepcounter{cpttheo} Proposition \thecpttheo . {#1} ]}
{%
\end{tcolorbox}\par}

%%%%%%%%%%%%% Propriétés
\newenvironment{Pp}[1][]{%
\medskip \begin{tcolorbox}[widget,colback=sacado_blue!0,colframe=sacado_blue!75!black,
adjusted title= \stepcounter{cpttheo} Propriété \thecpttheo . {#1}]}
{%
\end{tcolorbox}\par}

\newenvironment{PpT}[2][]{%
\medskip \begin{tcolorbox}[widget,colback=sacado_blue!0,colframe=sacado_blue!75!black,
adjusted title= \stepcounter{cpttheo} Propriété \thecpttheo . {#1} #2]}
{%
\end{tcolorbox}\par}

\newenvironment{Pps}[1][]{%
\medskip \begin{tcolorbox}[widget,colback=sacado_blue!0,colframe=sacado_blue!75!black,
adjusted title= \stepcounter{cpttheo} Propriétés \thecpttheo . {#1}]}
{%
\end{tcolorbox}\par}

%%%%%%%%%%%%% Théorèmes
\newenvironment{ThT}[2][]{% théorème avec titre
\medskip \begin{tcolorbox}[widget,colback=sacado_blue!0,colframe=sacado_blue!75!black,
adjusted title= \stepcounter{cpttheo} Théorème \thecpttheo . {#1} #2]}
{%
\end{tcolorbox}\par}

\newenvironment{Th}[1][]{%
\medskip \begin{tcolorbox}[widget,colback=sacado_blue!0,colframe=sacado_blue!75!black,
adjusted title= \stepcounter{cpttheo} Théorème \thecpttheo . {#1}]}
{%
\end{tcolorbox}\par}

%%%%%%%%%%%%% Règles
\newenvironment{Reg}[1][]{%
\medskip \begin{tcolorbox}[widget,colback=sacado_blue!0,colframe=sacado_blue!75!black,
adjusted title= \stepcounter{cpttheo} Règle \thecpttheo . {#1}]}
{%
\end{tcolorbox}\par}

%%%%%%%%%%%%% REMARQUES
\newenvironment{Rq}[1][]{%
\begin{bclogo}[couleur=sacado_orange!0, arrondi =0.15, noborder=true, couleurBarre=sacado_orange, logo = \bcinfo ]{ 
{\color{info}\normalsize{Remarque#1}}}}
{%
\end{bclogo}
\par}


\newenvironment{Rqs}[1][]{%
\begin{bclogo}[couleur=sacado_orange!0, arrondi =0.15, noborder=true, couleurBarre=sacado_orange, logo = \bcinfo ]{ 
{\color{info}\normalsize{Remarques#1}}}}
{%
\end{bclogo}
\par}

%%%%%%%%%%%%% EXEMPLES
\newenvironment{Ex}[1][]{%
\begin{bclogo}[couleur=sacado_yellow!15, arrondi =0.15, noborder=true, couleurBarre=sacado_yellow, logo = \bclampe ]{ 
\normalsize{Exemple#1}}}
{%
\end{bclogo}
\par}




%%%%%%%%%%%%% Preuve
\newenvironment{Pv}[1][]{%
\begin{tcolorbox}[breakable, enhanced,widget, colback=sacado_blue!10!white,boxrule=0pt,frame hidden,
borderline west={1mm}{0mm}{sacado_blue!75}]
\textbf{Preuve#1 : }}
{%
\end{tcolorbox}
\par}


%%%%%%%%%%%%% PreuveROC
\newenvironment{PvR}[1][]{%
\begin{tcolorbox}[breakable, enhanced,widget, colback=sacado_blue!10!white,boxrule=0pt,frame hidden,
borderline west={1mm}{0mm}{sacado_blue!75}]
\textbf{Preuve (ROC)#1 : }}
{%
\end{tcolorbox}
\par}


%%%%%%%%%%%%% Compétences
\newenvironment{Cps}[1][]{%
\vspace{0.4cm}
\begin{tcolorbox}[enhanced, lifted shadow={0mm}{0mm}{0mm}{0mm}%
{black!60!white}, attach boxed title to top left={xshift=5mm, yshift*=-3mm}, coltitle=white, colback=white, boxed title style={colback=sacado_green!100}, colframe=sacado_green!75!black,title=\textbf{Compétences associées#1}]}
{%
\end{tcolorbox}
\par}

%%%%%%%%%%%%% Compétences Collège
\newenvironment{CpsCol}[1][]{%
\vspace{0.4cm}
\begin{tcolorbox}[breakable, enhanced,widget, colback=white ,boxrule=0pt,frame hidden,
borderline west={2mm}{0mm}{bleu3}]
\textbf{#1}}
{%
\end{tcolorbox}
\par}




%%%%%%%%%%%%% Attendus
\newenvironment{Ats}[1][]{%
\vspace{0.4cm}
\begin{tcolorbox}[enhanced, lifted shadow={0mm}{0mm}{0mm}{0mm}%
{black!60!white}, attach boxed title to top left={xshift=5mm, yshift*=-3mm}, coltitle=white, colback=white, boxed title style={colback=sacado_green!100}, colframe=sacado_green!75!black,title=\textbf{Attendus du chapitre#1}]}
{%
\end{tcolorbox}
\par}

%%%%%%%%%%%%% Méthode
\newenvironment{Mt}[1][]{%
\vspace{0.4cm}
\begin{bclogo}[couleur=sacado_blue!0, arrondi =0.15, noborder=true, couleurBarre=bleu3, logo = \bccrayon ]{ 
\normalsize{{\color{bleu3}Méthode #1}}}}
{%
\end{bclogo}
\par}


%%%%%%%%%%%%% Méthode en vidéo
\newcommand{\MtV}[2]{\vspace{0.4cm} \colorbox{sacado_blue!0}{\hspace{0.2 cm}\tikz\node[rounded corners=1pt,draw] {\color{red}$\blacktriangleright$}; \quad  \href{https://youtu.be/#1?rel=0}{\raisebox{0.8mm}{{\color{red}\textbf{Méthode en vidéo : #2}}}}}}


%%%%%%%%%%%%% A voir (AV) : Lien externe + vidéo non Youtube
\newcommand{\AV}[2]{\vspace{0.4cm} \colorbox{bleu1!0}{\hspace{0.2 cm}\tikz\node[rounded corners=1pt,draw] {\color{red}$\blacktriangleright$}; \quad  \href{#1}{\raisebox{0.8mm}{{\color{red}\textbf{#2}}}}}}


%%%%%%%%%%%%% Etymologie
\newenvironment{Ety}[1][]{%
\begin{bclogo}[couleur=sacado_green!0, arrondi =0.15, noborder=true, couleurBarre=sacado_green, logo = \bcplume ]{ 
\normalsize{{\color{sacado_green}Étymologie#1}}}}
{%
\end{bclogo}
\par}


%%%%%%%%%%%%% Notation
\newenvironment{Nt}[1][]{%
\begin{bclogo}[couleur=sacado_violet!0, arrondi =0.15, noborder=true, couleurBarre=sacado_violet!75, logo = \bccrayon ]{ 
\normalsize{{\color{violet!75}Notation#1}}}}
{%
\end{bclogo}
\par}
%%%%%%%%%%%%% Histoire
%\newenvironment{His}[1][]{%
%\begin{bclogo}[couleur=brown!30, arrondi =0.15, noborder=true, couleurBarre=brown, logo = \bcvaletcoeur ]{ 
%\normalsize{{\color{brown}Histoire des mathématiques#1}}}}
%{%
%\end{bclogo}
%\par}

\newenvironment{His}[1][]{%
\vspace{0.4cm}
\begin{tcolorbox}[enhanced, lifted shadow={0mm}{0mm}{0mm}{0mm}%
{brown!60!white}, attach boxed title to top left={xshift=5mm, yshift*=-3mm}, coltitle=white, colback=white, boxed title style={colback=brown!100}, colframe=brown!75!black,title=\textbf{Histoire des mathématiques#1}]}
{%
\end{tcolorbox}
\par}

%%%%%%%%%%%%% Attention
\newenvironment{Att}[1][]{%
\begin{bclogo}[couleur=red!0, arrondi =0.15, noborder=true, couleurBarre=red, logo = \bcattention ]{ 
\normalsize{{\color{red}Attention. #1}}}}
{%
\end{bclogo}
\par}


%%%%%%%%%%%%% Conséquence
\newenvironment{Cq}[1][]{%
\textbf{Conséquence #1}}
{%
\par}

%%%%%%%%%%%%% Vocabulaire
\newenvironment{Voc}[1][]{%
\setlength{\logowidth}{10pt}
%\begin{footnotesize}
\begin{bclogo}[ noborder , couleur=white, logo =\bcbook]{#1}}
{%
\end{bclogo}
%\end{footnotesize}
\par}


%%%%%%%%%%%%% Video
\newenvironment{Vid}[1][]{%
\setlength{\logowidth}{12pt}
\begin{bclogo}[ noborder , couleur=white,barre=none, logo =\bcoeil]{#1}}
{%
\end{bclogo}
\par}


%%%%%%%%%%%%% Syntaxe
\newenvironment{Syn}[1][]{%
\begin{bclogo}[couleur=violet!0, arrondi =0.15, noborder=true, couleurBarre=violet!75, logo = \bcicosaedre ]{ 
\normalsize{{\color{violet!75}Syntaxe#1}}}}
{%
\end{bclogo}
\par}

%%%%%%%%%%%%% Auto évaluation
\newenvironment{autoeval}[1][]{%
\vspace{0.4cm}
\begin{tcolorbox}[enhanced, lifted shadow={0mm}{0mm}{0mm}{0mm}%
{black!60!white}, attach boxed title to top left={xshift=5mm, yshift*=-3mm}, coltitle=white, colback=white, boxed title style={colback=sacado_green!100}, colframe=sacado_green!75!black,title=\textbf{J'évalue mes compétences#1}]}
{%
\end{tcolorbox}
\par}


\newenvironment{autotest}[1][]{%
\vspace{0.4cm}
\begin{tcolorbox}[enhanced, lifted shadow={0mm}{0mm}{0mm}{0mm}%
{red!60!white}, attach boxed title to top left={xshift=5mm, yshift*=-3mm}, coltitle=white, colback=white, boxed title style={colback=red!100}, colframe=red!75!black,title=\textbf{Pour faire le point #1}]}
{%
\end{tcolorbox}
\par}

\newenvironment{ExOApp}[1][]{% Exercice d'application direct
\vspace{0.4cm}
\begin{tcolorbox}[enhanced, lifted shadow={0mm}{0mm}{0mm}{0mm}%
{red!60!white}, attach boxed title to top left={xshift=5mm, yshift*=-3mm}, coltitle=white, colback=white, boxed title style={colback=sacado_green!100}, colframe=sacado_green!75!black,title=\textbf{Application #1}]}
{%
\end{tcolorbox}
\par}

\newenvironment{ExOInt}[1][]{% Exercice d'application direct
\vspace{0.4cm}
\begin{tcolorbox}[enhanced, lifted shadow={0mm}{0mm}{0mm}{0mm}%
{red!60!white}, attach boxed title to top left={xshift=5mm, yshift*=-3mm}, coltitle=white, colback=white, boxed title style={colback=sacado_green!50}, colframe=sacado_green!75!black,title=\textbf{Exercice #1}]}
{%
\end{tcolorbox}
\par}

%Illustrations
\newtcolorbox{Illqr}[1]{
  enhanced,
  colback=white,
  colframe=ill_frame,
  colbacktitle=ill_back,
  coltitle=ill_title,
  title=\textbf{Illustration},
  boxrule=1pt, % épaisseur du trait à 1pt
  center,
  overlay={
    \node[anchor=south east, inner sep=0pt,xshift=-1pt,yshift=2pt,fill=white] at (frame.south east) {\fancyqr[height=1cm]{#1}};
  },
  after=\par,
  before=\vspace{0.4cm},
}

\newtcolorbox{Ill}{
  enhanced,
  colback=white,
  colframe=ill_frame,
  colbacktitle=ill_back,
  coltitle=ill_title,
  title=\textbf{Illustration},
  boxrule=1pt, % épaisseur du trait à 1pt
  center,
  after=\par,
  before=\vspace{0.4cm},
}

%%%%%%%%%%%%%% Propriétés
%\newenvironment{Pp}[1][]{%
%\medskip \begin{tcolorbox}[widget,colback=sacado_blue!0,colframe=sacado_blue!75!black,
%adjusted title= \stepcounter{cpttheo} Propriété \thecpttheo . {#1}]}
%{%
%\end{tcolorbox}\par}

%%%%% Pour réinitialiser numéros des chapitres après une nouvelle partie
% \makeatletter
    % \@addtoreset{section}{part}
% \makeatother

% \newcommand{\EPC}[3]{ % Exercice par compétence de niveau 1
% \ifthenelse{\equal{#1}{1}}
% {%condition2 vraie
% \vspace{0.4cm}
% \stepcounter{cptex}
% \tikz\node[rounded corners=0pt,draw,fill=bleu2]{\color{white}\textbf{ \thecptex}}; \quad  {\color{bleu2}\textbf{#3}}
% \input{#2}
% }% fin condition2 vraie
% {%condition2 fausse
% \vspace{0.4cm}
% \stepcounter{cptex}
% \tikz\node[rounded corners=2pt,draw,fill=eduscol4P]{\color{white}\textbf{ \thecptex}}; \quad  {\color{eduscol4P} \textbf{En temps libre.} \textbf{ #3}} 
% \input{#2}
% }% fin condition2 fausse
% } % fin de la procédure

\usepackage{hyperref}

\begin{document}

%-------------------------------
%	TITLE SECTION
%-------------------------------

\fancyhead[C]{}
\hrule\medskip % Upper rule
\begin{minipage}{0.295\textwidth} 
\raggedright
Classe \myClasse \hfill\\
\myDiscipline \hfill\\
\myParcours \hfill\\
\end{minipage}
\begin{minipage}{0.4\textwidth} 
\centering 
\scshape\huge
\textcolor{sacado_purple}{\myTitle} \\ 
\normalsize 
%\mySubTitle \\ 
\end{minipage}
\begin{minipage}{0.295\textwidth} 
\raggedleft
\href{https://sacado.xyz/}{\includegraphics[width=.2\linewidth]{sacadoA1.png}}
%\myAnnee \hfill\\
\end{minipage}
\medskip \hrule
\bigskip

%-------------------------------
%	CONTENTS
%-------------------------------

%\chapter{Calcul littéral}
%{URL du cours}
%

% {
% \begin{CpsCol}
 % \textbf{Les savoir-faire du parcours}
 % \begin{itemize}
 % \item 
 % \end{itemize}
% \end{CpsCol}

% \textbf{qrcode vers le parcours diagnostic}

% }

%\begin{pageCours}

\section{Expressions littérales}

\begin{Def}
Une \textbf{expression littérale} est une expression mathématique qui comporte une ou plusieurs \textbf{lettres}.

Ces lettres s'appellent les \textbf{variables}, elles ne représentent aucun nombre particulier et peuvent prendre plusieurs valeurs.
\end{Def}

\begin{Ex}
\begin{itemize}
\item Le périmètre d'un rectangle de longueur \textcolor{sacado_green}{$L$} et de largeur \textcolor{sacado_green}{$l$} est donné par l'expression littérale : $P=2\times\textcolor{sacado_green}{L}+2\times\textcolor{sacado_green}{l}$.
\item L'aire d'un carré de côté \textcolor{sacado_green}{$c$} est donné par l'expression littérale : $A=\textcolor{sacado_green}{c}^2$
\end{itemize}
\end{Ex}

\begin{Mt}
\textit{Traduire une situation par une expression littérale}

Pour un bouquet livré, un fleuriste facture 2€ par fleur et 26€ pour la livraison. Le prix a payer en fonction du nombre $y$ de fleurs est $2\times y+26$.
\end{Mt}

\section{Écriture simplifiée d'une expression littérale}

\begin{His}
François Viète (1540-1603) est un mathématicien français. En 1591, il publie \textit{In artem analyticem isagoge} qui représente une avancée considérable pour l'algèbre.

Le calcul littéral trouve ses bases dans le but de résoudre tout problème.

Les grandeurs cherchées sont désignées par des voyelles et les grandeurs connues par des consonnes.

Les symboles d'opérations sont officialisés.
\end{His}

\begin{Pp}
Pour marquer la \textbf{priorité de la multiplication},et ne pas le confondre avec la lettre "$x$",  le symbole « $\times$ » peut être omis dans certains cas. Cela permet également de simplifier l'écriture d'une expression.
\end{Pp}

\begin{Def}
\begin{itemize}
\item $a+a$ s'appelle le double du nombre $a$ et se note $2a$
\item $a+a+a$ s'appelle le triple du nombre $a$ et se note $3a$
\item $a\times a$ s'appelle le carré du nombre $a$ et se note $a^2$
\item $a\times a\times a$ s'appelle le cube du nombre $a$ et se note $a^3$
\end{itemize}
\end{Def}

\begin{Ex}
L'expression $A(x)$ peut être simplifiée de la façon suivante :
\[A(x)=6\times x\times 2+5\times2=\textcolor{sacado_green}{12x+10}\]
\end{Ex}

\section{Réduire et ordonner une expression littérale}

\begin{DefT}{Réduire}
\textbf{Réduire} une expression c'est \textbf{factoriser les coefficients} des termes \textbf{de même degré}.

(compter les différentes quantités de cette expression)
\end{DefT}

\begin{Ex}
On considère l'expression littérale $A(x)=-7x-9+12x+7$. Sa version réduite est :
\[A(x)=5x-2\]
\end{Ex}

\begin{DefT}{Ordonner}
\textbf{Ordonner} une expression c'est l'écrire avec ses \textbf{termes de degré décroissant}.
\end{DefT}

\begin{Ex}
On considère l'expression littérale $A(x)=-5x+3+3x^2+6x-5$. Sa version réduite est :
\[A(x)=x-2+3x^2\]
Sa version réduite ET ordonnée est :
\[A(x)=3x^2+x-2\]
\end{Ex}

\section{Substituer une valeur à une variable}

\begin{Def}
Lorsqu'on calcule une expression en donnant une une valeur à la lettre,  on dit qu'on \textbf{substitue} la lettre par la valeur.
\end{Def}

\begin{Rq}
Lorsque dans une expression littérale on substitue la lettre par une valeur, il faut penser à réécrire les symboles « $\times$ » qui ont été simplifiés.
\end{Rq}

\begin{Ex}
Soit l'expression $A=a(4b+7)$. Calculons pour $a=5$ et $b=9$ :
\[A=5\times(4\times9+7)=215\]
\end{Ex}

\section{Forme développée ou factorisée d'une expression.}

\begin{Def}
\begin{itemize}
\item On dit qu'une \textbf{expression littérale} est une \textbf{somme} lorsque la \textbf{dernière opération calculée} lorsqu'on substitue est \textbf{une addition} ou \textbf{une soustraction}.
\item On dit qu'une \textbf{expression littérale} est un \textbf{produit} lorsque la \textbf{dernière opération calculée} lorsqu'on substitue est une \textbf{multiplication}.
\end{itemize}
\end{Def}

\begin{Ex}
\mini{.45\linewidth}{
Dans l'expression :
\[A(x)=(4x-7)^2\]
La dernière opération calculé quand on substitue la variable $x$ est une \textbf{multiplication}, il s'agit donc d'un \textbf{produit}.
}{.45\linewidth}{
Dans l'expression :
\[A(x)=(4x+2)+(6x-1)\]
La dernière opération calculé quand on substitue la variable $x$ est une \textbf{addition}, il s'agit donc d'une \textbf{somme}.
}
\end{Ex}

\begin{Def}
\begin{itemize}
\item \textbf{Développer} une expression signifie l'\textbf{écrire sous la forme d'une somme}.
\item \textbf{Factoriser} une expression signifie l'\textbf{écrire sous la forme d'un produit}.
\end{itemize}
\end{Def}

\begin{Ex}
\begin{itemize}
\item On considère l'expression : $A(x)=3x^2+2x-4-7x^2+4x+2$, la dernière opération effectuée lorsqu'on substitue est \textbf{une addition ou une soustraction}, il s'agit donc d'une \textbf{somme}, l'expression est donc \textbf{développée}.
\item On considère l'expression : $A(x)=(-2x+8)(-x+4)$, la dernière opération effectuée lorsqu'on substitue est \textbf{une multiplication}, il s'agit donc d'un \textbf{produit}, l'expression est donc \textbf{factorisée}.
\end{itemize}
\end{Ex}

\section{La distributivité}

\subsection{La simple distributivité}

%\begin{Ill}
%\definecolor{ccwwff}{rgb}{0.8,0.4,1.}
%\definecolor{ffqqqq}{rgb}{1.,0.,0.}
%\definecolor{ffzzqq}{rgb}{1.,0.6,0.}
%\definecolor{ffqqff}{rgb}{1.,0.,1.}
%\definecolor{qqzzff}{rgb}{0.,0.6,1.}
%\definecolor{qqttzz}{rgb}{0.,0.2,0.6}
%\begin{tikzpicture}[line cap=round,line join=round,>=triangle 45,x=.8cm,y=.8cm]
%\clip(-1.8749087418124062,-3.213035065214023) rectangle (16.207967066451374,6.405515896628444);
%\fill[line width=2.pt,color=qqzzff,fill=qqzzff,pattern=north east lines,pattern color=qqzzff] (0.86,3.34) -- (0.86,-0.66) -- (7.86,-0.66) -- (7.86,3.34) -- cycle;
%\fill[line width=2.pt,color=ffqqff,fill=ffqqff,pattern=north east lines,pattern color=ffqqff] (7.86,3.34) -- (13.86,3.34) -- (13.86,-0.66) -- (7.86,-0.66) -- cycle;
%\fill[line width=2.pt,color=ffzzqq,fill=ffzzqq,pattern=north east lines,pattern color=ffzzqq] (0.86,-0.66) -- (0.86,-0.66) -- (7.86,-0.66) -- (7.86,-0.66) -- cycle;
%\fill[line width=2.pt,color=ffqqqq,fill=ffqqqq,pattern=north east lines,pattern color=ffqqqq] (7.86,-0.66) -- (7.86,-0.66) -- (13.86,-0.66) -- (13.86,-0.66) -- cycle;
%\fill[line width=0.pt,color=qqzzff,fill=qqzzff,fill opacity=0.4000000059604645] (0.86,3.34) -- (0.86,-0.66) -- (7.86,-0.66) -- (7.86,3.34) -- cycle;
%\draw [line width=2.pt,color=qqzzff] (0.86,3.34)-- (0.86,-0.66);
%\draw [line width=2.pt,color=qqzzff] (0.86,-0.66)-- (7.86,-0.66);
%\draw [line width=2.pt,color=qqzzff] (7.86,-0.66)-- (7.86,3.34);
%\draw [line width=2.pt,color=qqzzff] (7.86,3.34)-- (0.86,3.34);
%\draw [line width=2.pt,color=ffqqff] (7.86,3.34)-- (13.86,3.34);
%\draw [line width=2.pt,color=ffqqff] (13.86,3.34)-- (13.86,-0.66);
%\draw [line width=2.pt,color=ffqqff] (13.86,-0.66)-- (7.86,-0.66);
%\draw [line width=2.pt,color=ffqqff] (7.86,-0.66)-- (7.86,3.34);
%\draw [line width=2.pt,color=qqttzz] (0.56,-0.36)-- (0.56,3.04);
%\draw [line width=2.pt,color=qqzzff] (1.16,3.64)-- (7.56,3.64);
%\draw [line width=2.pt,color=ffqqff] (8.16,3.64)-- (13.56,3.64);
%\draw [line width=2.pt,color=qqttzz] (1.16,-0.96)-- (13.56,-0.96);
%\draw (4.109967412222888,4.251289781207348) node[anchor=north west] {$\mathbf{\textcolor{blue}{7}}$};
%\draw (10.650582066275744,4.251289781207348) node[anchor=north west] {$\mathbf{\textcolor{magenta}{6}}$};
%\draw (-0.2,2) node[anchor=north west] {$\mathbf{\textcolor{purple}{4}}$};
%\draw [color=qqttzz](-1.4474175879527424,4.7) node[anchor=north west] {$\mathbf{A_{ABCD}=\textcolor{purple}{4}\times(\textcolor{blue}{7}+\textcolor{magenta}{6})= \underline{\textcolor{blue}{  \textcolor{purple}{4}\times\textcolor{blue}{7}}} +\textcolor{purple}{4}\times\textcolor{magenta}{6}   }$};
%\draw [color=qqttzz](6.589416104608938,-0.990081065143764) node[anchor=north west] {$\mathbf{  \textcolor{blue}{7}+\textcolor{magenta}{6}}$};
%\draw [fill=qqttzz] (0.86,3.34) circle (1.0pt);
%\draw[color=qqttzz] (0.5,3.6) node {$\bold{A}$};
%\draw [fill=qqttzz] (13.86,3.34) circle (1.0pt);
%\draw[color=qqttzz] (14.3,3.6) node {$\bold{B}$};
%\draw [fill=qqttzz] (0.86,-0.66) circle (1.0pt);
%\draw[color=qqttzz] (0.5,-1.0114556228367473) node {$\bold{D}$};
%\draw [fill=qqttzz] (13.86,-0.66) circle (1.0pt);
%\draw[color=qqttzz] (14.3,-1.0114556228367473) node {$\bold{C}$};
%\draw [fill=ccwwff,shift={(0.56,-0.36)},rotate=180] (0,0) ++(0 pt,3.0pt) -- ++(2.598076211353316pt,-4.5pt)--++(-5.196152422706632pt,0 pt) -- ++(2.598076211353316pt,4.5pt);
%\draw [fill=ccwwff,shift={(0.56,3.04)}] (0,0) ++(0 pt,3.0pt) -- ++(2.598076211353316pt,-4.5pt)--++(-5.196152422706632pt,0 pt) -- ++(2.598076211353316pt,4.5pt);
%\draw [fill=qqzzff,shift={(1.16,3.64)},rotate=90] (0,0) ++(0 pt,3.0pt) -- ++(2.598076211353316pt,-4.5pt)--++(-5.196152422706632pt,0 pt) -- ++(2.598076211353316pt,4.5pt);
%\draw [fill=qqzzff,shift={(7.56,3.64)},rotate=270] (0,0) ++(0 pt,3.0pt) -- ++(2.598076211353316pt,-4.5pt)--++(-5.196152422706632pt,0 pt) -- ++(2.598076211353316pt,4.5pt);
%\draw [fill=ffqqff,shift={(8.16,3.64)},rotate=90] (0,0) ++(0 pt,3.0pt) -- ++(2.598076211353316pt,-4.5pt)--++(-5.196152422706632pt,0 pt) -- ++(2.598076211353316pt,4.5pt);
%\draw [fill=ffqqff,shift={(13.56,3.64)},rotate=270] (0,0) ++(0 pt,3.0pt) -- ++(2.598076211353316pt,-4.5pt)--++(-5.196152422706632pt,0 pt) -- ++(2.598076211353316pt,4.5pt);
%\draw [fill=qqttzz,shift={(13.56,-0.96)},rotate=270] (0,0) ++(0 pt,3.0pt) -- ++(2.598076211353316pt,-4.5pt)--++(-5.196152422706632pt,0 pt) -- ++(2.598076211353316pt,4.5pt);
%\draw [fill=qqttzz,shift={(1.16,-0.96)},rotate=90] (0,0) ++(0 pt,3.0pt) -- ++(2.598076211353316pt,-4.5pt)--++(-5.196152422706632pt,0 pt) -- ++(2.598076211353316pt,4.5pt);
%\end{tikzpicture}
%\end{Ill}

\begin{PpT}{Simple distributivité de la multiplication par rapport à l'addition}
\begin{itemize}
\item Pour tous nombres $k$, $a$ et $b$ : $k\times(a+b)=k\times a+k\times b$
\item Pour tous nombres $k$, $a$ et $b$ : $k\times(a-b)=k\times a-k\times b$
\end{itemize}
\end{PpT}

\begin{Ex}
Utilisons la distributivité pour transformer le calcul :
\[6\times (5-4)=6\times5-6\times4=30-24=6\]
\end{Ex}


\subsection{La double distributivité}

%\begin{Ill}
%\definecolor{yqqqyq}{rgb}{0.5019607843137255,0.,0.5019607843137255}
%\definecolor{ccwwff}{rgb}{0.8,0.4,1.}
%\definecolor{ffqqqq}{rgb}{1.,0.,0.}
%\definecolor{ffzzqq}{rgb}{1.,0.6,0.}
%\definecolor{qqccqq}{rgb}{0.,0.8,0.}
%\definecolor{qqzzff}{rgb}{0.,0.6,1.}
%\definecolor{qqttzz}{rgb}{0.,0.2,0.6}
%\begin{tikzpicture}[line cap=round,line join=round,>=triangle 45,x=.8cm,y=.8cm]
%\clip(-0.6351843956193828,-2.8516107260417605) rectangle (24.67229191287274,6.766940235800699);
%\fill[line width=2.pt,color=qqzzff,fill=qqzzff,pattern=north east lines,pattern color=qqzzff] (0.86,3.34) -- (0.86,1.) -- (4.86,1.) -- (4.86,3.34) -- cycle;
%\fill[line width=2.pt,color=qqccqq,fill=qqccqq,pattern=north east lines,pattern color=qqccqq] (4.86,3.34) -- (11.86,3.34) -- (11.86,1.) -- (4.86,1.) -- cycle;
%\fill[line width=2.pt,color=ffzzqq,fill=ffzzqq,pattern=north east lines,pattern color=ffzzqq] (0.86,-1.) -- (0.86,1.) -- (4.86,1.) -- (4.86,-1.) -- cycle;
%\fill[line width=2.pt,color=ffqqqq,fill=ffqqqq,pattern=north east lines,pattern color=ffqqqq] (4.86,-1.) -- (4.86,1.) -- (11.86,1.) -- (11.86,-1.) -- cycle;
%\fill[line width=0.pt,color=qqzzff,fill=qqzzff,fill opacity=0.4000000059604645] (0.86,3.34) -- (0.86,1.) -- (4.86,1.) -- (4.86,3.34) -- cycle;
%\draw [line width=2.pt,color=qqzzff] (0.86,3.34)-- (0.86,1.);
%\draw [line width=2.pt,color=qqzzff] (0.86,1.)-- (4.86,1.);
%\draw [line width=2.pt,color=qqzzff] (4.86,1.)-- (4.86,3.34);
%\draw [line width=2.pt,color=qqzzff] (4.86,3.34)-- (0.86,3.34);
%\draw [line width=2.pt,color=qqccqq] (4.86,3.34)-- (11.86,3.34);
%\draw [line width=2.pt,color=qqccqq] (11.86,3.34)-- (11.86,1.);
%\draw [line width=2.pt,color=qqccqq] (11.86,1.)-- (4.86,1.);
%\draw [line width=2.pt,color=qqccqq] (4.86,1.)-- (4.86,3.34);
%\draw [line width=2.pt,color=ffzzqq] (0.86,-1.)-- (0.86,1.);
%\draw [line width=2.pt,color=ffzzqq] (0.86,1.)-- (4.86,1.);
%\draw [line width=2.pt,color=ffzzqq] (4.86,1.)-- (4.86,-1.);
%\draw [line width=2.pt,color=ffzzqq] (4.86,-1.)-- (0.86,-1.);
%\draw [line width=2.pt,color=ffqqqq] (4.86,-1.)-- (4.86,1.);
%\draw [line width=2.pt,color=ffqqqq] (4.86,1.)-- (11.86,1.);
%\draw [line width=2.pt,color=ffqqqq] (11.86,1.)-- (11.86,-1.);
%\draw [line width=2.pt,color=ffqqqq] (11.86,-1.)-- (4.86,-1.);
%\draw [line width=2.pt,color=ccwwff] (0.36,1.5)-- (0.36,2.84);
%\draw [line width=2.pt,color=ccwwff] (0.36,0.5)-- (0.36,-0.5);
%\draw [line width=2.pt,color=qqzzff] (1.36,3.84)-- (4.36,3.84);
%\draw [line width=2.pt,color=qqccqq] (5.36,3.84)-- (11.36,3.84);
%\draw [line width=2.pt,color=qqttzz] (1.36,-1.5)-- (11.36,-1.5);
%\draw [line width=2.pt,color=yqqqyq] (12.36,-0.5)-- (12.36,2.84);
%\draw (0,0.2) node[anchor=north west] {$\mathbf{\textcolor{purple}{2}}$};
%\draw (2.613748373714065,4.2) node[anchor=north west] {$\mathbf{\textcolor{blue}{4}}$};
%\draw (8.128384258503734,4.2) node[anchor=north west] {$\mathbf{\textcolor{ForestGreen}{7}}$};
%\draw (0,2.4) node[anchor=north west] {$\mathbf{\textcolor{magenta}{5}}$};
%\draw [color=qqttzz](0.13429968132801287,5) node[anchor=north west] {$\mathbf{A_{ABCD}=(\textcolor{magenta}{5}+\textcolor{purple}{2})\times(\textcolor{blue}{4}+\textcolor{ForestGreen}{7})= \underline{\textcolor{magenta}{5}\times\textcolor{blue}{4}}} +\textcolor{magenta}{5}\times\textcolor{ForestGreen}{7}   +   \textcolor{purple}{2}\times\textcolor{blue}{4} +   \textcolor{purple}{2}\times\textcolor{ForestGreen}{7}$};
%\draw [color=qqttzz](5.606186450731715,-1.5263881490767992) node[anchor=north west] {$\mathbf{   \textcolor{blue}{4}+\textcolor{ForestGreen}{7}}$};
%\draw [color=qqttzz](12.4,1.3) node[anchor=north west] {$\mathbf{                  \textcolor{magenta}{5}+\textcolor{purple}{2}}$};
%\draw [fill=qqttzz] (0.86,3.34) circle (1.0pt);
%\draw[color=qqttzz] (.6,3.5) node {$\bold{A}$};
%\draw [fill=qqttzz] (11.86,3.34) circle (1.0pt);
%\draw[color=qqttzz] (12.2,3.5) node {$\bold{B}$};
%\draw [fill=qqttzz] (0.86,-1.) circle (1.0pt);
%\draw[color=qqttzz] (.6,-1.3) node {$\bold{D}$};
%\draw [fill=qqttzz] (11.86,-1.) circle (1.0pt);
%\draw[color=qqttzz] (12.2,-1.3) node {$\bold{C}$};
%\draw [fill=ccwwff,shift={(0.36,1.5)},rotate=180] (0,0) ++(0 pt,3.0pt) -- ++(2.598076211353316pt,-4.5pt)--++(-5.196152422706632pt,0 pt) -- ++(2.598076211353316pt,4.5pt);
%\draw [fill=ccwwff,shift={(0.36,0.5)}] (0,0) ++(0 pt,3.0pt) -- ++(2.598076211353316pt,-4.5pt)--++(-5.196152422706632pt,0 pt) -- ++(2.598076211353316pt,4.5pt);
%\draw [fill=ccwwff,shift={(0.36,-0.5)},rotate=180] (0,0) ++(0 pt,3.0pt) -- ++(2.598076211353316pt,-4.5pt)--++(-5.196152422706632pt,0 pt) -- ++(2.598076211353316pt,4.5pt);
%\draw [fill=ccwwff,shift={(0.36,2.84)}] (0,0) ++(0 pt,3.0pt) -- ++(2.598076211353316pt,-4.5pt)--++(-5.196152422706632pt,0 pt) -- ++(2.598076211353316pt,4.5pt);
%\draw [fill=qqzzff,shift={(1.36,3.84)},rotate=90] (0,0) ++(0 pt,3.0pt) -- ++(2.598076211353316pt,-4.5pt)--++(-5.196152422706632pt,0 pt) -- ++(2.598076211353316pt,4.5pt);
%\draw [fill=qqzzff,shift={(4.36,3.84)},rotate=270] (0,0) ++(0 pt,3.0pt) -- ++(2.598076211353316pt,-4.5pt)--++(-5.196152422706632pt,0 pt) -- ++(2.598076211353316pt,4.5pt);
%\draw [fill=qqccqq,shift={(5.36,3.84)},rotate=90] (0,0) ++(0 pt,3.0pt) -- ++(2.598076211353316pt,-4.5pt)--++(-5.196152422706632pt,0 pt) -- ++(2.598076211353316pt,4.5pt);
%\draw [fill=qqccqq,shift={(11.36,3.84)},rotate=270] (0,0) ++(0 pt,3.0pt) -- ++(2.598076211353316pt,-4.5pt)--++(-5.196152422706632pt,0 pt) -- ++(2.598076211353316pt,4.5pt);
%\draw [fill=yqqqyq,shift={(12.36,2.84)}] (0,0) ++(0 pt,3.0pt) -- ++(2.598076211353316pt,-4.5pt)--++(-5.196152422706632pt,0 pt) -- ++(2.598076211353316pt,4.5pt);
%\draw [fill=yqqqyq,shift={(12.36,-0.5)},rotate=180] (0,0) ++(0 pt,3.0pt) -- ++(2.598076211353316pt,-4.5pt)--++(-5.196152422706632pt,0 pt) -- ++(2.598076211353316pt,4.5pt);
%\draw [fill=qqttzz,shift={(11.36,-1.5)},rotate=270] (0,0) ++(0 pt,3.0pt) -- ++(2.598076211353316pt,-4.5pt)--++(-5.196152422706632pt,0 pt) -- ++(2.598076211353316pt,4.5pt);
%\draw [fill=qqttzz,shift={(1.36,-1.5)},rotate=90] (0,0) ++(0 pt,3.0pt) -- ++(2.598076211353316pt,-4.5pt)--++(-5.196152422706632pt,0 pt) -- ++(2.598076211353316pt,4.5pt);
%\end{tikzpicture}
%\end{Ill}

\begin{PpT}{Double distributivité de la multiplication par rapport à l'addition}
Pour tous nombres $a$, $b$, $c$ et $d$ : $(a+b)\times(c+d)=a\times c+a\times d+b\times c+b\times d$
\end{PpT}

\begin{Ex}
Nous pouvons utiliser la double distributivité pour calculer mentalement :
\[A=11\times15=(10+1)\times(10+5)=10\times10+10\times5+1\times10+1\times5\]
donc :
\[A=100+50+10+5=165\]
\end{Ex}

\begin{Ill}
\mini{.45\linewidth}{
\definecolor{ccwwff}{rgb}{0.8,0.4,1.}
\definecolor{ffqqqq}{rgb}{1.,0.,0.}
\definecolor{ffzzqq}{rgb}{1.,0.6,0.}
\definecolor{ffqqff}{rgb}{1.,0.,1.}
\definecolor{qqzzff}{rgb}{0.,0.6,1.}
\definecolor{qqttzz}{rgb}{0.,0.2,0.6}
\begin{tikzpicture}[line cap=round,line join=round,>=triangle 45,x=.4cm,y=.4cm]
\clip(-1.8749087418124062,-3.213035065214023) rectangle (16.207967066451374,6.405515896628444);
\fill[line width=2.pt,color=qqzzff,fill=qqzzff,pattern=north east lines,pattern color=qqzzff] (0.86,3.34) -- (0.86,-0.66) -- (7.86,-0.66) -- (7.86,3.34) -- cycle;
\fill[line width=2.pt,color=ffqqff,fill=ffqqff,pattern=north east lines,pattern color=ffqqff] (7.86,3.34) -- (13.86,3.34) -- (13.86,-0.66) -- (7.86,-0.66) -- cycle;
\fill[line width=2.pt,color=ffzzqq,fill=ffzzqq,pattern=north east lines,pattern color=ffzzqq] (0.86,-0.66) -- (0.86,-0.66) -- (7.86,-0.66) -- (7.86,-0.66) -- cycle;
\fill[line width=2.pt,color=ffqqqq,fill=ffqqqq,pattern=north east lines,pattern color=ffqqqq] (7.86,-0.66) -- (7.86,-0.66) -- (13.86,-0.66) -- (13.86,-0.66) -- cycle;
\fill[line width=0.pt,color=qqzzff,fill=qqzzff,fill opacity=0.4000000059604645] (0.86,3.34) -- (0.86,-0.66) -- (7.86,-0.66) -- (7.86,3.34) -- cycle;
\draw [line width=2.pt,color=qqzzff] (0.86,3.34)-- (0.86,-0.66);
\draw [line width=2.pt,color=qqzzff] (0.86,-0.66)-- (7.86,-0.66);
\draw [line width=2.pt,color=qqzzff] (7.86,-0.66)-- (7.86,3.34);
\draw [line width=2.pt,color=qqzzff] (7.86,3.34)-- (0.86,3.34);
\draw [line width=2.pt,color=ffqqff] (7.86,3.34)-- (13.86,3.34);
\draw [line width=2.pt,color=ffqqff] (13.86,3.34)-- (13.86,-0.66);
\draw [line width=2.pt,color=ffqqff] (13.86,-0.66)-- (7.86,-0.66);
\draw [line width=2.pt,color=ffqqff] (7.86,-0.66)-- (7.86,3.34);
\draw [line width=2.pt,color=qqttzz] (0.56,-0.36)-- (0.56,3.04);
\draw [line width=2.pt,color=qqzzff] (1.16,3.64)-- (7.56,3.64);
\draw [line width=2.pt,color=ffqqff] (8.16,3.64)-- (13.56,3.64);
\draw [line width=2.pt,color=qqttzz] (1.16,-0.96)-- (13.56,-0.96);
\draw (4.109967412222888,4.251289781207348) node[anchor=north west] {$\mathbf{\textcolor{blue}{7}}$};
\draw (10.650582066275744,4.251289781207348) node[anchor=north west] {$\mathbf{\textcolor{magenta}{6}}$};
\draw (-0.2,2) node[anchor=north west] {$\mathbf{\textcolor{purple}{4}}$};
\draw [color=qqttzz](-1.4474175879527424,4.7) node[anchor=north west] {$\mathbf{A_{ABCD}=\textcolor{purple}{4}\times(\textcolor{blue}{7}+\textcolor{magenta}{6})= \underline{\textcolor{blue}{  \textcolor{purple}{4}\times\textcolor{blue}{7}}} +\textcolor{purple}{4}\times\textcolor{magenta}{6}   }$};
\draw [color=qqttzz](6.589416104608938,-0.990081065143764) node[anchor=north west] {$\mathbf{  \textcolor{blue}{7}+\textcolor{magenta}{6}}$};
\draw [fill=qqttzz] (0.86,3.34) circle (1.0pt);
\draw[color=qqttzz] (0.5,3.6) node {$\bold{A}$};
\draw [fill=qqttzz] (13.86,3.34) circle (1.0pt);
\draw[color=qqttzz] (14.3,3.6) node {$\bold{B}$};
\draw [fill=qqttzz] (0.86,-0.66) circle (1.0pt);
\draw[color=qqttzz] (0.5,-1.0114556228367473) node {$\bold{D}$};
\draw [fill=qqttzz] (13.86,-0.66) circle (1.0pt);
\draw[color=qqttzz] (14.3,-1.0114556228367473) node {$\bold{C}$};
\draw [fill=ccwwff,shift={(0.56,-0.36)},rotate=180] (0,0) ++(0 pt,3.0pt) -- ++(2.598076211353316pt,-4.5pt)--++(-5.196152422706632pt,0 pt) -- ++(2.598076211353316pt,4.5pt);
\draw [fill=ccwwff,shift={(0.56,3.04)}] (0,0) ++(0 pt,3.0pt) -- ++(2.598076211353316pt,-4.5pt)--++(-5.196152422706632pt,0 pt) -- ++(2.598076211353316pt,4.5pt);
\draw [fill=qqzzff,shift={(1.16,3.64)},rotate=90] (0,0) ++(0 pt,3.0pt) -- ++(2.598076211353316pt,-4.5pt)--++(-5.196152422706632pt,0 pt) -- ++(2.598076211353316pt,4.5pt);
\draw [fill=qqzzff,shift={(7.56,3.64)},rotate=270] (0,0) ++(0 pt,3.0pt) -- ++(2.598076211353316pt,-4.5pt)--++(-5.196152422706632pt,0 pt) -- ++(2.598076211353316pt,4.5pt);
\draw [fill=ffqqff,shift={(8.16,3.64)},rotate=90] (0,0) ++(0 pt,3.0pt) -- ++(2.598076211353316pt,-4.5pt)--++(-5.196152422706632pt,0 pt) -- ++(2.598076211353316pt,4.5pt);
\draw [fill=ffqqff,shift={(13.56,3.64)},rotate=270] (0,0) ++(0 pt,3.0pt) -- ++(2.598076211353316pt,-4.5pt)--++(-5.196152422706632pt,0 pt) -- ++(2.598076211353316pt,4.5pt);
\draw [fill=qqttzz,shift={(13.56,-0.96)},rotate=270] (0,0) ++(0 pt,3.0pt) -- ++(2.598076211353316pt,-4.5pt)--++(-5.196152422706632pt,0 pt) -- ++(2.598076211353316pt,4.5pt);
\draw [fill=qqttzz,shift={(1.16,-0.96)},rotate=90] (0,0) ++(0 pt,3.0pt) -- ++(2.598076211353316pt,-4.5pt)--++(-5.196152422706632pt,0 pt) -- ++(2.598076211353316pt,4.5pt);
\end{tikzpicture}
}{.45\linewidth}{
\definecolor{yqqqyq}{rgb}{0.5019607843137255,0.,0.5019607843137255}
\definecolor{ccwwff}{rgb}{0.8,0.4,1.}
\definecolor{ffqqqq}{rgb}{1.,0.,0.}
\definecolor{ffzzqq}{rgb}{1.,0.6,0.}
\definecolor{qqccqq}{rgb}{0.,0.8,0.}
\definecolor{qqzzff}{rgb}{0.,0.6,1.}
\definecolor{qqttzz}{rgb}{0.,0.2,0.6}
\begin{tikzpicture}[line cap=round,line join=round,>=triangle 45,x=.4cm,y=.4cm]
\clip(-0.6351843956193828,-2.8516107260417605) rectangle (24.67229191287274,6.766940235800699);
\fill[line width=2.pt,color=qqzzff,fill=qqzzff,pattern=north east lines,pattern color=qqzzff] (0.86,3.34) -- (0.86,1.) -- (4.86,1.) -- (4.86,3.34) -- cycle;
\fill[line width=2.pt,color=qqccqq,fill=qqccqq,pattern=north east lines,pattern color=qqccqq] (4.86,3.34) -- (11.86,3.34) -- (11.86,1.) -- (4.86,1.) -- cycle;
\fill[line width=2.pt,color=ffzzqq,fill=ffzzqq,pattern=north east lines,pattern color=ffzzqq] (0.86,-1.) -- (0.86,1.) -- (4.86,1.) -- (4.86,-1.) -- cycle;
\fill[line width=2.pt,color=ffqqqq,fill=ffqqqq,pattern=north east lines,pattern color=ffqqqq] (4.86,-1.) -- (4.86,1.) -- (11.86,1.) -- (11.86,-1.) -- cycle;
\fill[line width=0.pt,color=qqzzff,fill=qqzzff,fill opacity=0.4000000059604645] (0.86,3.34) -- (0.86,1.) -- (4.86,1.) -- (4.86,3.34) -- cycle;
\draw [line width=2.pt,color=qqzzff] (0.86,3.34)-- (0.86,1.);
\draw [line width=2.pt,color=qqzzff] (0.86,1.)-- (4.86,1.);
\draw [line width=2.pt,color=qqzzff] (4.86,1.)-- (4.86,3.34);
\draw [line width=2.pt,color=qqzzff] (4.86,3.34)-- (0.86,3.34);
\draw [line width=2.pt,color=qqccqq] (4.86,3.34)-- (11.86,3.34);
\draw [line width=2.pt,color=qqccqq] (11.86,3.34)-- (11.86,1.);
\draw [line width=2.pt,color=qqccqq] (11.86,1.)-- (4.86,1.);
\draw [line width=2.pt,color=qqccqq] (4.86,1.)-- (4.86,3.34);
\draw [line width=2.pt,color=ffzzqq] (0.86,-1.)-- (0.86,1.);
\draw [line width=2.pt,color=ffzzqq] (0.86,1.)-- (4.86,1.);
\draw [line width=2.pt,color=ffzzqq] (4.86,1.)-- (4.86,-1.);
\draw [line width=2.pt,color=ffzzqq] (4.86,-1.)-- (0.86,-1.);
\draw [line width=2.pt,color=ffqqqq] (4.86,-1.)-- (4.86,1.);
\draw [line width=2.pt,color=ffqqqq] (4.86,1.)-- (11.86,1.);
\draw [line width=2.pt,color=ffqqqq] (11.86,1.)-- (11.86,-1.);
\draw [line width=2.pt,color=ffqqqq] (11.86,-1.)-- (4.86,-1.);
\draw [line width=2.pt,color=ccwwff] (0.36,1.5)-- (0.36,2.84);
\draw [line width=2.pt,color=ccwwff] (0.36,0.5)-- (0.36,-0.5);
\draw [line width=2.pt,color=qqzzff] (1.36,3.84)-- (4.36,3.84);
\draw [line width=2.pt,color=qqccqq] (5.36,3.84)-- (11.36,3.84);
\draw [line width=2.pt,color=qqttzz] (1.36,-1.5)-- (11.36,-1.5);
\draw [line width=2.pt,color=yqqqyq] (12.36,-0.5)-- (12.36,2.84);
\draw (0,0.2) node[anchor=north west] {$\mathbf{\textcolor{purple}{2}}$};
\draw (2.613748373714065,4.2) node[anchor=north west] {$\mathbf{\textcolor{blue}{4}}$};
\draw (8.128384258503734,4.2) node[anchor=north west] {$\mathbf{\textcolor{ForestGreen}{7}}$};
\draw (0,2.4) node[anchor=north west] {$\mathbf{\textcolor{magenta}{5}}$};
\draw [color=qqttzz](0.13429968132801287,5) node[anchor=north west] {$\mathbf{A_{ABCD}=(\textcolor{magenta}{5}+\textcolor{purple}{2})\times(\textcolor{blue}{4}+\textcolor{ForestGreen}{7})= \underline{\textcolor{magenta}{5}\times\textcolor{blue}{4}}} +\textcolor{magenta}{5}\times\textcolor{ForestGreen}{7}   +   \textcolor{purple}{2}\times\textcolor{blue}{4} +   \textcolor{purple}{2}\times\textcolor{ForestGreen}{7}$};
\draw [color=qqttzz](5.606186450731715,-1.5263881490767992) node[anchor=north west] {$\mathbf{   \textcolor{blue}{4}+\textcolor{ForestGreen}{7}}$};
\draw [color=qqttzz](12.4,1.3) node[anchor=north west] {$\mathbf{                  \textcolor{magenta}{5}+\textcolor{purple}{2}}$};
\draw [fill=qqttzz] (0.86,3.34) circle (1.0pt);
\draw[color=qqttzz] (.6,3.5) node {$\bold{A}$};
\draw [fill=qqttzz] (11.86,3.34) circle (1.0pt);
\draw[color=qqttzz] (12.2,3.5) node {$\bold{B}$};
\draw [fill=qqttzz] (0.86,-1.) circle (1.0pt);
\draw[color=qqttzz] (.6,-1.3) node {$\bold{D}$};
\draw [fill=qqttzz] (11.86,-1.) circle (1.0pt);
\draw[color=qqttzz] (12.2,-1.3) node {$\bold{C}$};
\draw [fill=ccwwff,shift={(0.36,1.5)},rotate=180] (0,0) ++(0 pt,3.0pt) -- ++(2.598076211353316pt,-4.5pt)--++(-5.196152422706632pt,0 pt) -- ++(2.598076211353316pt,4.5pt);
\draw [fill=ccwwff,shift={(0.36,0.5)}] (0,0) ++(0 pt,3.0pt) -- ++(2.598076211353316pt,-4.5pt)--++(-5.196152422706632pt,0 pt) -- ++(2.598076211353316pt,4.5pt);
\draw [fill=ccwwff,shift={(0.36,-0.5)},rotate=180] (0,0) ++(0 pt,3.0pt) -- ++(2.598076211353316pt,-4.5pt)--++(-5.196152422706632pt,0 pt) -- ++(2.598076211353316pt,4.5pt);
\draw [fill=ccwwff,shift={(0.36,2.84)}] (0,0) ++(0 pt,3.0pt) -- ++(2.598076211353316pt,-4.5pt)--++(-5.196152422706632pt,0 pt) -- ++(2.598076211353316pt,4.5pt);
\draw [fill=qqzzff,shift={(1.36,3.84)},rotate=90] (0,0) ++(0 pt,3.0pt) -- ++(2.598076211353316pt,-4.5pt)--++(-5.196152422706632pt,0 pt) -- ++(2.598076211353316pt,4.5pt);
\draw [fill=qqzzff,shift={(4.36,3.84)},rotate=270] (0,0) ++(0 pt,3.0pt) -- ++(2.598076211353316pt,-4.5pt)--++(-5.196152422706632pt,0 pt) -- ++(2.598076211353316pt,4.5pt);
\draw [fill=qqccqq,shift={(5.36,3.84)},rotate=90] (0,0) ++(0 pt,3.0pt) -- ++(2.598076211353316pt,-4.5pt)--++(-5.196152422706632pt,0 pt) -- ++(2.598076211353316pt,4.5pt);
\draw [fill=qqccqq,shift={(11.36,3.84)},rotate=270] (0,0) ++(0 pt,3.0pt) -- ++(2.598076211353316pt,-4.5pt)--++(-5.196152422706632pt,0 pt) -- ++(2.598076211353316pt,4.5pt);
\draw [fill=yqqqyq,shift={(12.36,2.84)}] (0,0) ++(0 pt,3.0pt) -- ++(2.598076211353316pt,-4.5pt)--++(-5.196152422706632pt,0 pt) -- ++(2.598076211353316pt,4.5pt);
\draw [fill=yqqqyq,shift={(12.36,-0.5)},rotate=180] (0,0) ++(0 pt,3.0pt) -- ++(2.598076211353316pt,-4.5pt)--++(-5.196152422706632pt,0 pt) -- ++(2.598076211353316pt,4.5pt);
\draw [fill=qqttzz,shift={(11.36,-1.5)},rotate=270] (0,0) ++(0 pt,3.0pt) -- ++(2.598076211353316pt,-4.5pt)--++(-5.196152422706632pt,0 pt) -- ++(2.598076211353316pt,4.5pt);
\draw [fill=qqttzz,shift={(1.36,-1.5)},rotate=90] (0,0) ++(0 pt,3.0pt) -- ++(2.598076211353316pt,-4.5pt)--++(-5.196152422706632pt,0 pt) -- ++(2.598076211353316pt,4.5pt);
\end{tikzpicture}
}
\end{Ill}

\section{Opposé d'une expression}

\begin{Def}
\textbf{Deux expressions} littérales sont dites \textbf{opposées} lorsque leur \textbf{somme est égale à 0}.
\end{Def}

\begin{Ex}
Déterminons l'opposé de l'expression $A(x)=10x+7$.
\[(10x+7)+(-10x-7)=0\]
Donc l'opposé de l'expression $A(x)$ est $B(x)=-10x-7$.
\end{Ex}

\begin{Pp}
L'opposée d'une expression $A(x)$ se note $-A(x)$ et vérifie : $-A(x)=(-1)\times A(x)$
\end{Pp}

\begin{Ex}
Déterminons l'opposé de l'expression : $A(x)=7x+9$.
\[-A(x)=-(7x+9)=(-1)\times(7x+9)\]
donc $-A(x)=-7x-9$ donc l'opposé de $A(x)$ est $-A(x)=-7x-9$ 
\end{Ex}

\section{Développer une expression littérale avec la simple distributivité}

\section{Développer une expression littérale avec la double distributivité}

\section{Factoriser une expression littérale avec la simple distributivité}

\section{$a^2-b^2=(a+b)(a-b)$}

\section{Exercice bilan}

\section{Les savoir-faire du parcours}

\begin{CpsCol}
\textbf{Les savoir-faire du parcours}
\begin{itemize}
\item Savoir exprimer en fonction de....
\item Savoir traduire un programme de calcul par une expression littérale.
\item Savoir traduire une situation par une expression littérale.
\item Savoir substituer la variable par un nombre dans une expression littérale.  
\item Savoir simplifier l'écriture d'une expression littérale.
\item Savoir réduire et ordonner une expression littérale.      
\item Savoir développer une expression avec la simple distributivité.
\item Savoir développer une expression avec la double distributivité.     
\item Savoir factoriser une expression avec la simple distributivité.
\item Savoir développer une expression avec l'égalité $a^2-b^2=(a+b)(a-b)$.
\item Savoir factoriser une expression avec l'égalité $a^2-b^2=(a+b)(a-b)$.   
\end{itemize}
\end{CpsCol}

\end{document}

%\end{pageCours}
