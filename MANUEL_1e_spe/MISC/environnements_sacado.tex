%----------------------------------------
%
%   Définitions des environnements "pageCours" et "pageExos"
%
%----------------------------------------

\newcounter{cpt}
\newcounter{exo}
\newcounter{cptr}

\newcommand{\titreChap}{Titre de chapitre à définir}

\renewcommand{\chapter}[3]{
  \stepcounter{chapter}
  \setcounter{exo}{0}
  \setcounter{cpt}{0}
  
%\cleardoublepage  % pour commencer à droite
{\Huge \hfill Chapitre \Roman{chapter}.\\
  \bigskip
  #1\\
  \bigskip {\begin{center}
  \fancyqr[image={\includegraphics[scale=.6]{sacadoA1.png}},image padding=.5,height=5cm]{#2}
  \end{center}}  {\normalsize #3}}
\renewcommand{\titreChap}{#1}

%\ifthenelse{\equal{#2}{}}{}{\par
%  \bigskip\bigskip
%  #2}
\newpage
}

\newenvironment{pageHistoire}{
\lhead{%
\tikz[remember picture,overlay] {%
\fill [fond!70]
([yshift=0.14cm] current page header area.south west -| current page.north west)
rectangle
(current page.north east);
%    \fill [fond!70]
%    ([yshift=-45pt]current page footer area.south west)
%    rectangle
%    (current page footer area.north east)
%    ;
}
\raisebox{0.5em}{\textcolor{texte}{\Huge \bf HISTOIRE}}
\hfill \raisebox{0.5em}{\textcolor{texte}{\Large \titreChap}}}
\begin{leftbar}}{\end{leftbar}\newpage\lhead{}}




\newenvironment{pageCours}{
\lhead{%
\tikz[remember picture,overlay] {%
\fill [fond!70]
([yshift=0.14cm] current page header area.south west -| current page.north west)
rectangle
(current page.north east);
%    \fill [fond!70]
%    ([yshift=-45pt]current page footer area.south west)
%    rectangle
%    (current page footer area.north east)
%    ;
}
\raisebox{0.5em}{\textcolor{texte}{\Huge \bf COURS}}
\hfill \raisebox{0.5em}{\textcolor{texte}{\Large \titreChap}}}
\begin{leftbar}}{\end{leftbar}\newpage\lhead{}}


\newenvironment{pageAD}{
\lhead{%
\tikz[remember picture,overlay] {%
\fill [sacado_violet!70]
([yshift=0.14cm] current page header area.south west -| current page.north west)
rectangle
(current page.north east);
%    \fill [fond!70]
%    ([yshift=-45pt]current page footer area.south west)
%    rectangle
%    (current page footer area.north east)
%    ;
}
\raisebox{0.5em}{\textcolor{texte}{\Huge \bf APPLICATIONS DIRECTES}}
\hfill \raisebox{0.5em}{\textcolor{texte}{\Large \titreChap}}}
}{\newpage\lhead{}}


 

\newenvironment{pageParcoursu}{
\lhead{%
\tikz[remember picture,overlay] {%
\fill [sacado_green]
([yshift=0.14cm] current page header area.south west -| current page.north west)
rectangle
(current page.north east);
%    \fill [fond!70]
%    ([yshift=-45pt]current page footer area.south west)
%    rectangle
%    (current page footer area.north east)
%    ;
}
\raisebox{0.5em}{\textcolor{texte}{\Huge \bf PARCOURS 1}}
\hfill \raisebox{0.5em}{\textcolor{texte}{\Large \titreChap}}}
}{\newpage\lhead{}}



\newenvironment{pageParcoursd}{
\lhead{%
\tikz[remember picture,overlay] {%
\fill [sacado_blue]
([yshift=0.14cm] current page header area.south west -| current page.north west)
rectangle
(current page.north east);
%    \fill [fond!70]
%    ([yshift=-45pt]current page footer area.south west)
%    rectangle
%    (current page footer area.north east)
%    ;
}
\raisebox{0.5em}{\textcolor{texte}{\Huge \bf PARCOURS 2}}
\hfill \raisebox{0.5em}{\textcolor{texte}{\Large \titreChap}}}
}{\newpage\lhead{}}


\newenvironment{pageParcourst}{
\lhead{%
\tikz[remember picture,overlay] {%
\fill [sacado_red]
([yshift=0.14cm] current page header area.south west -| current page.north west)
rectangle
(current page.north east);
%    \fill [fond!70]
%    ([yshift=-45pt]current page footer area.south west)
%    rectangle
%    (current page footer area.north east)
%    ;
}
\raisebox{0.5em}{\textcolor{texte}{\Huge \bf PARCOURS 3}}
\hfill \raisebox{0.5em}{\textcolor{texte}{\Large \titreChap}}}
}{\newpage\lhead{}}




\newenvironment{pageBrouillon}{
\lhead{%
\tikz[remember picture,overlay] {%
\fill [sacado_gray!70]
([yshift=0.14cm] current page header area.south west -| current page.north west)
rectangle
(current page.north east);
%    \fill [fond!70]
%    ([yshift=-45pt]current page footer area.south west)
%    rectangle
%    (current page footer area.north east)
%    ;
}
\raisebox{0.5em}{\textcolor{texte}{\Huge \bf BROUILLON}}
\hfill \raisebox{0.5em}{\textcolor{texte}{\Large \titreChap}}}
}{\newpage\lhead{}}



\newenvironment{pageRituels}{
\lhead{%
\tikz[remember picture,overlay] {%
\fill [fond!70]
([yshift=0.14cm] current page header area.south west -| current page.north west)
rectangle
(current page.north east);
%    \fill [fond!70]
%    ([yshift=-45pt]current page footer area.south west)
%    rectangle
%    (current page footer area.north east)
%    ;
}
\raisebox{0.5em}{\textcolor{texte}{\Huge \bf RITUELS}}
\hfill \raisebox{0.5em}{\textcolor{texte}{\Large \titreChap}}}
}{\newpage\lhead{}}



\newenvironment{pageAuto}{
\lhead{%
\tikz[remember picture,overlay] {%
\fill [sacado_orange!70]
([yshift=0cm] current page header area.south west -| current page.north west)
rectangle
(current page.north east);
%    \fill [fond!70]
%    ([yshift=-45pt]current page footer area.south west)
%    rectangle
%    (current page footer area.north east)
%    ;
}
\raisebox{0.5em}{\textcolor{texte}{\Huge \bf AUTOÉVALUATION}}
\hfill \raisebox{0.5em}{\textcolor{texte}{\Large \titreChap}}}
}{\newpage\lhead{}}







\fancyfoot[L]{\colorbox{fond!70}{\color{texte}\thepage}}
\fancyfoot[C]{}


\newcommand{\titresec}[2]{\phantom{.}\begin{textblock}{1}[0,1](-1.24,0.25)\colorbox{fond!70}{%
\makebox[0.8cm]{\raisebox{0.05cm}[0.6cm][0.15cm]{\color{texte}\LARGE\bf #1}}}\end{textblock}{\LARGE\bf #2}\\\bigskip}

\renewcommand{\thesection}{\arabic{section}}
\titleformat{\section}{}{%
\hspace{-1.15cm}\colorbox{fond!70}{%
\makebox[0.8cm]{\raisebox{0.05cm}[0.6cm][0.15cm]{\color{texte}\LARGE\bf \thesection}}}}{1em}{\bf \LARGE #1}
  
\renewcommand{\thesubsection}{\arabic{subsection}}
            
\titleformat{\subsection}
{%\begin{textblock}{1}[0,1](-1,0.42) toto
  %\end{textblock}
%\reversemarginpar\marginnote[\rule{0.8cm}{0.8cm}]{}[0pt]  \color{red}\normalfont\Large\bfseries}
}{\hspace{-0.83em}
\colorbox{fond!70}{\makebox[0.6cm]{\raisebox{0cm}[1em][0.2em]\normalfont\large\bfseries\color{texte}\thesubsection}}}{1em}{\bf \large #1}



\renewenvironment{leftbar}[1][\hsize]
{%
    \def\FrameCommand
    {%
        {\color{black}\vrule width 0.5pt}%
        \hspace{4pt}%must no space.
        \fboxsep=\FrameSep%\colorbox{yellow}%
    }%
    \MakeFramed{\hsize#1\advance\hsize-\width\FrameRestore}%
}
{\endMakeFramed}



\makeatletter
\newenvironment{TraitV}[1]{%
% #1 couleur du trait (par défaut CouleurA)
% #2 largeur du trait
% #3 distance entre le trait et le texte
\def\FrameCommand{{\color{#1}\vrule width 2pt}
\hspace{1em}}\MakeFramed {\advance\hsize-\width}}%
{\endMakeFramed}
\makeatother

%----------------------------------------
%
%   Définitions des environnements de Définitions, propriétés...
%
%----------------------------------------

%%%%%%%%%%%%% Définitions
\newenvironment{Def}{%
\medskip \begin{tcolorbox}[widget,colback=sacado_violet!15,colframe=sacado_violet!75!black,
title= \stepcounter{cpt} Définition \thecpt. ]}{%
\end{tcolorbox}\par}


\newenvironment{DefT}[1]{%
\medskip \begin{tcolorbox}[widget,colback=sacado_violet!15,colframe=sacado_violet!75!black,
title= \stepcounter{cpt} Définition \thecpt : #1.]}
{%
\end{tcolorbox}\par}


%%%%%%%%%%%%% Proposition
\newenvironment{Prop}{%
\medskip \begin{tcolorbox}[widget,colback=sacado_blue!15,colframe=sacado_blue!75!black,
title= \stepcounter{cpt} Proposition \thecpt.]}
{%
\end{tcolorbox}\par}


%%%%%%%%%%%%% Propriétés
\newenvironment{Pp}{%
\medskip \begin{tcolorbox}[widget,colback=sacado_blue!15,colframe=sacado_blue!75!black,
title= \stepcounter{cpt} Propriété \thecpt.]}
{%
\end{tcolorbox}\par}

\newenvironment{PpT}[1]{%
\medskip \begin{tcolorbox}[widget,colback=sacado_blue!15,colframe=sacado_blue!75!black,
title= \stepcounter{cpt} Propriété \thecpt : #1. ]}
{%
\end{tcolorbox}\par}

\newenvironment{Pps}{%
\medskip \begin{tcolorbox}[widget,colback=sacado_blue!15,colframe=sacado_blue!75!black,
title= \stepcounter{cpt} Propriétés \thecpt.]}
{%
\end{tcolorbox}\par}


%%%%%%%%%%%%% Propriétés
\newenvironment{Cq}{%
\medskip \begin{tcolorbox}[widget,colback=white,colframe=sacado_blue!75!black,
title= \stepcounter{cpt} Conséquence \thecpt.]}
{%
\end{tcolorbox}\par}



%%%%%%%%%%%%% Théorèmes
\newenvironment{ThT}[1]{% théorème avec titre
\medskip \begin{tcolorbox}[widget,colback=sacado_blue!15,colframe=sacado_blue!75!black,
title= \stepcounter{cpt} Théorème \thecpt : #1.]}
{%
\end{tcolorbox}\par}

\newenvironment{Th}{%
\medskip \begin{tcolorbox}[widget,colback=sacado_blue!15,colframe=sacado_blue!75!black,
title= \stepcounter{cpt} Théorème \thecpt.]}
{%
\end{tcolorbox}\par}


%%%%%%%%%%%%% Règles
\newenvironment{Reg}{%
\medskip \begin{tcolorbox}[widget,colback=sacado_blue!15,colframe=sacado_blue!75!black,
title= \stepcounter{cpt} Règle \thecpt.]}
{%
\end{tcolorbox}\par}

%%%%%%%%%%%%% Représentations
\newenvironment{Rep}{%
\medskip \begin{tcolorbox}[widget,colback=white,colframe=sacado_violet!75!white,
title= \stepcounter{cpt} Représentation \thecpt.]}
{%
\end{tcolorbox}\par}

 
%%%%%%%%%%%%% REMARQUES
\newenvironment{Rq}{%
\medskip \begin{tcolorbox}[widget,colback=sacado_orange!15,colframe=sacado_orange!75!black,
title= \stepcounter{cpt} Remarque \thecpt.]}
{%
\end{tcolorbox}\par}

\newenvironment{Rqs}{%
\medskip \begin{tcolorbox}[widget,colback=sacado_orange!15,colframe=sacado_orange!75!black,
title= \stepcounter{cpt} Remarques \thecpt.]}
{%
\end{tcolorbox}\par}


%%%%%%%%%%%%% EXEMPLES
\newenvironment{Ex}{%
\medskip \begin{tcolorbox}[widget,colback=white,colframe=sacado_violet!75!white,
title= \stepcounter{cpt} Exemple \thecpt.]}
{%
\end{tcolorbox}\par}

\newenvironment{Exs}{%
\medskip \begin{tcolorbox}[widget,colback=sacado_yellow!15,colframe=sacado_violet!75!white,
title= \stepcounter{cpt} Exemples \thecpt.]}
{%
\end{tcolorbox}\par}

%%%%%%%%%%%%% EXEMPLES
\newenvironment{ExT}[1]{%
\medskip \begin{tcolorbox}[widget,colback=white,colframe=sacado_violet!75!white,
title= \stepcounter{cpt} Exemple \thecpt   : #1.]}
{%
\end{tcolorbox}\par}




%%%%%%%%%%%%% Preuve
\newenvironment{Pv}[1][]{%
\begin{tcolorbox}[breakable, enhanced,widget, colback=sacado_blue!10!white,boxrule=0pt,frame hidden,
borderline west={1mm}{0mm}{sacado_blue!75}]
\textbf{Preuve#1 : }}
{%
\end{tcolorbox}
\par}


%%%%%%%%%%%%% PreuveROC
\newenvironment{PvR}[1][]{%
\begin{tcolorbox}[breakable, enhanced,widget, colback=sacado_blue!10!white,boxrule=0pt,frame hidden,
borderline west={1mm}{0mm}{sacado_blue!75}]
\textbf{Preuve (ROC)#1 : }}
{%
\end{tcolorbox}
\par}


%%%%%%%%%%%%% Compétences
\newenvironment{Cps}[1][]{%
\vspace{0.4cm}
\begin{tcolorbox}[enhanced, lifted shadow={0mm}{0mm}{0mm}{0mm}%
{black!60!white}, attach boxed title to top left={xshift=5mm, yshift*=-3mm}, coltitle=white, colback=white, boxed title style={colback=sacado_green!100}, colframe=sacado_green!75!black,title=\textbf{Compétences associées#1}]}
{%
\end{tcolorbox}
\par}


%%%%%%%%%%%%% Compétences Collège
\newenvironment{CpsCol}[1][]{%
\vspace{0.4cm}
\begin{tcolorbox}[breakable, enhanced,widget, colback=white ,boxrule=0pt,frame hidden,
borderline west={2mm}{0mm}{bleu3}]
\textbf{#1}}
{%
\end{tcolorbox}
\par}


%%%%%%%%%%%%% Attendus
\newenvironment{Ats}[1][]{%
\vspace{0.4cm}
\begin{tcolorbox}[enhanced, lifted shadow={0mm}{0mm}{0mm}{0mm}%
{black!60!white}, attach boxed title to top left={xshift=5mm, yshift*=-3mm}, coltitle=white, colback=white, boxed title style={colback=sacado_green!100}, colframe=sacado_green!75!black,title=\textbf{Attendus du chapitre#1}]}
{%
\end{tcolorbox}
\par}

%%%%%%%%%%%%% Rituel
\newenvironment{Rit}{%
\medskip \begin{tcolorbox}[widget,colback=white!15,colframe=sacado_violet!75!black,
title= \stepcounter{cptr} Rituel \thecptr. ]}{%
\end{tcolorbox}\par}


%%%%%%%%%%%%% Méthode
\newenvironment{Mt}{%
\medskip \begin{tcolorbox}[widget,colback=white!15,colframe=sacado_violet!75!black,
title= \stepcounter{cptr} Méthode \thecptr. ]}{%
\end{tcolorbox}\par}


%%%%%%%%%%%%% Méthode en vidéo
\newcommand{\MtV}[2]{\vspace{0.4cm} \colorbox{sacado_blue!0}{\hspace{0.2 cm}\tikz\node[rounded corners=1pt,draw] {\color{red}$\blacktriangleright$}; \quad  \href{https://youtu.be/#1?rel=0}{\raisebox{0.8mm}{{\color{red}\textbf{Méthode en vidéo : #2}}}}}}


%%%%%%%%%%%%% A voir (AV) : Lien externe + vidéo non Youtube
\newcommand{\AV}[2]{\vspace{0.4cm} \colorbox{bleu1!0}{\hspace{0.2 cm}\tikz\node[rounded corners=1pt,draw] {\color{red}$\blacktriangleright$}; \quad  \href{#1}{\raisebox{0.8mm}{{\color{red}\textbf{#2}}}}}}


%%%%%%%%%%%%% Etymologie
\newenvironment{Ety}[1][]{%
\begin{bclogo}[couleur=sacado_orange!0, arrondi =0.15, noborder=true, couleurBarre=sacado_orange, logo = \bcplume ]{ 
\normalsize{{\color{sacado_orange}Étymologie#1}}}}
{%
\end{bclogo}
\par}


%%%%%%%%%%%%% Notation
\newenvironment{Nt}[1][]{%
\begin{bclogo}[couleur=sacado_violet!0, arrondi =0.15, noborder=true, couleurBarre=sacado_violet!75, logo = \bccrayon ]{ 
\normalsize{{\color{violet!75}Notation#1}}}}
{%
\end{bclogo}
\par}


%%%%%%%%%%%%% Histoire
\newenvironment{His}[1][]{%
\vspace{0.4cm}
\begin{tcolorbox}[enhanced, lifted shadow={0mm}{0mm}{0mm}{0mm}%
{brown!60!white}, attach boxed title to top left={xshift=5mm, yshift*=-3mm}, coltitle=white, colback=white, boxed title style={colback=brown!100}, colframe=brown!75!black,title=\textbf{Histoire des mathématiques#1}]}
{%
\end{tcolorbox}
\par}


%%%%%%%%%%%%% Attention
\newenvironment{Att}[1][]{%
\begin{bclogo}[couleur=sacado_red!0, arrondi =0.15, noborder=true, couleurBarre=red, logo = \bcattention ]{ 
\normalsize{{\color{red}Attention. #1}}}}
{%
\end{bclogo}
\par}


 


%%%%%%%%%%%%% Vocabulaire
\newenvironment{Voc}[1][]{%
\setlength{\logowidth}{10pt}
%\begin{footnotesize}
\begin{bclogo}[ noborder , couleur=white, logo =\bcbook]{#1}}
{%
\end{bclogo}
%\end{footnotesize}
\par}


%%%%%%%%%%%%% Video
\newenvironment{Vid}[1][]{%
\setlength{\logowidth}{12pt}
\begin{bclogo}[ noborder , couleur=white,barre=none, logo =\bcoeil]{#1}}
{%
\end{bclogo}
\par}


%%%%%%%%%%%%% Syntaxe
\newenvironment{Syn}[1][]{%
\begin{bclogo}[couleur=sacado_violet!0, arrondi =0.15, noborder=true, couleurBarre=violet!75, logo = \bcicosaedre ]{ 
\normalsize{{\color{violet!75}Syntaxe#1}}}}
{%
\end{bclogo}
\par}

%%%%%%%%%%%%% Auto évaluation
\newenvironment{autoeval}[1][]{%
\vspace{0.4cm}
\begin{tcolorbox}[enhanced, lifted shadow={0mm}{0mm}{0mm}{0mm}%
{black!60!white}, attach boxed title to top left={xshift=5mm, yshift*=-3mm}, coltitle=white, colback=white, boxed title style={colback=sacado_green!100}, colframe=sacado_green!75!black,title=\textbf{J'évalue mes compétences#1}]}
{%
\end{tcolorbox}
\par}


\newenvironment{autotest}[1][]{%
\vspace{0.4cm}
\begin{tcolorbox}[enhanced, lifted shadow={0mm}{0mm}{0mm}{0mm}%
{red!60!white}, attach boxed title to top left={xshift=5mm, yshift*=-3mm}, coltitle=white, colback=white, boxed title style={colback=red!100}, colframe=red!75!black,title=\textbf{Pour faire le point #1}]}
{%
\end{tcolorbox}
\par}

\newenvironment{ExOApp}[1][]{% Exercice d'application direct
\vspace{0.4cm}
\begin{tcolorbox}[enhanced, lifted shadow={0mm}{0mm}{0mm}{0mm}%
{red!60!white}, attach boxed title to top left={xshift=5mm, yshift*=-3mm}, coltitle=white, colback=white, boxed title style={colback=sacado_green!100}, colframe=sacado_green!75!black,title=\textbf{Application #1}]}
{%
\end{tcolorbox}
\par}

\newenvironment{ExOInt}[1][]{% Exercice d'application direct
\vspace{0.4cm}
\begin{tcolorbox}[enhanced, lifted shadow={0mm}{0mm}{0mm}{0mm}%
{red!60!white}, attach boxed title to top left={xshift=5mm, yshift*=-3mm}, coltitle=white, colback=white, boxed title style={colback=sacado_green!50}, colframe=sacado_green!75!black,title=\textbf{Exercice #1}]}
{%
\end{tcolorbox}
\par}

\newenvironment{Exo}{%
\medskip \begin{tcolorbox}[widget,colback=white,colframe=black!75!black,
title=\stepcounter{exo} Exercice \theexo.]}
{%
\end{tcolorbox}\par}





%Illustrations
\newtcolorbox{Illqr}[1]{
  enhanced,
  colback=white,
  colframe=ill_frame,
  colbacktitle=ill_back,
  coltitle=ill_title,
  title=\textbf{Illustration},
  boxrule=1pt, % épaisseur du trait à 1pt
  center,
  overlay={
    \node[anchor=south east, inner sep=0pt,xshift=-1pt,yshift=2pt,fill=white] at (frame.south east) {\fancyqr[height=1cm]{#1}};
  },
  after=\par,
  before=\vspace{0.4cm},
}

\newtcolorbox{Ill}{
  enhanced,
  colback=white,
  colframe=ill_frame,
  colbacktitle=ill_back,
  coltitle=ill_title,
  title=\textbf{Illustration},
  boxrule=1pt, % épaisseur du trait à 1pt
  center,
  after=\par,
  before=\vspace{0.4cm},
}



\newcommand{\ExoCad}[1]{ % Exercice par compétence application directe
\vspace{0.4cm}
\stepcounter{exo}
\tikz\node[rounded corners=2pt,draw,fill=sacado_violet]{\color{white}\textbf{ \theexo}}; \quad  {\color{sacado_violet} \hfill \textbf{#1}  } 
} % fin de la procédure


\newcommand{\ExoCu}[1]{ % Exercice par compétence de niveau 1
\vspace{0.4cm}
\stepcounter{exo}
\tikz\node[rounded corners=2pt,draw,fill=sacado_green]{\color{white}\textbf{ \theexo}}; \quad  {\color{sacado_green} \hfill \textbf{#1}  } 
} % fin de la procédure


\newcommand{\ExoCd}[1]{ % Exercice par compétence de niveau 2
\vspace{0.4cm}
\stepcounter{exo}
\tikz\node[rounded corners=2pt,draw,fill=sacado_blue]{\color{white}\textbf{ \theexo}}; \quad  {\color{sacado_blue} \hfill \textbf{#1}  } 
} % fin de la procédure


\newcommand{\ExoCt}[1]{ % Exercice par compétence de niveau 3
\vspace{0.4cm}
\stepcounter{exo}
\tikz\node[rounded corners=2pt,draw,fill=sacado_red]{\color{white}\textbf{ \theexo}}; \quad  {\color{sacado_red} \hfill \textbf{#1}  } 
} % fin de la procédure


\newcommand{\Calcu}[1]{ % Exercice avec calculatrice
\vspace{0.4cm}
\stepcounter{exo}
\tikz\node[rounded corners=2pt,draw,fill=sacado_red]{\color{white}\textbf{ \theexo}}; \; 
\includegraphics[scale=0.5]{MISC/calculator.png} 
 {\color{sacado_red} \hfill \textbf{#1}  } 
} 
% fin de la procédure

\newcommand{\NoCalcu}[1]{ % Exercice avec calculatrice
\vspace{0.4cm}
\stepcounter{exo}
\tikz\node[rounded corners=2pt,draw,fill=sacado_blue]{\color{white}\textbf{ \theexo}}; \;
\includegraphics[scale=0.5]{MISC/no_calculator.png} 
 {\color{sacado_blue} \hfill \textbf{#1}  } 
} 
% fin de la procédure




\newcommand{\ExoAuto} { % Exercice avec calculatrice
\vspace{0.4cm}
\stepcounter{exo}
\tikz\node[rounded corners=2pt,draw,fill=sacado_orange]{\color{white}\textbf{ \theexo}}; \;} 
% fin de la procédure








\newcommand{\Sf}[1]{ 
{\color{fond!70}{\Large \textbf{#1}}  } 
} 

\newcommand{\Sfe}[1]{ 
{\color{sacado_blue}{\Large \textbf{#1}}  } 
} 
% fin de la procédure



%%%%%%%%%%%%% Pointillés ou ligne
\newcommand{\point}[1]{\vspace{0.1cm}\multido{}{#1}{ \dotfill \medskip \endgraf}}
\newcommand{\ligne}[1]{\vspace{0.1cm}\multido{}{#1}{ {\color{cqcqcq}\hrulefill} \medskip \endgraf}}