%----------------------------------------------------------------------------------------
%	PACKAGES AND OTHER DOCUMENT CONFIGURATIONS
%----------------------------------------------------------------------------------------

%----------------------------------------------------------------------------------------
%		Géometrie de la page
%----------------------------------------------------------------------------------------
\documentclass[dvipsnames,french,10pt]{book}

\usepackage[
paperheight=29.7cm, %hauteur du papier
paperwidth=21cm, %largeur du papier
left=1cm, %marge de gauche
right=1cm, %marge de droite
top=1.5cm, %marge du haut
bottom=1cm, %marge du bas
%marginparsep=0pt, %distance entre le texte et les notes de marges 
reversemp, %inverse l'emplacement de la marge
headheight=20.60pt %hauteur du header
%showframe, %permet d'afficher le cadre défini ci-dessus
%bindingoffset=1cm %permet d'ajouter le décalage dû au reliage
]{geometry} %Redéfinition de la taille des pages
\raggedbottom


%----------------------------------------------------------------------------------------
%		Generals
%----------------------------------------------------------------------------------------
%\usepackage{fourier} %!! A changer plus tard !!
\usepackage[scaled]{uarial}
\renewcommand*\familydefault{\sfdefault} %% Only if the base font of the document is to be sans serif
\usepackage{frcursive}
\usepackage[T1]{fontenc} %Accents handling
\usepackage[utf8]{inputenc} % Use UTF-8 encoding
%\usepackage{microtype} % Slightly tweak font spacing for aesthetics
\usepackage[english, francais]{babel} % Language hyphenation and typographical rules
\usepackage{marginnote}


%----------------------------------------------------------------------------------------
%		Graphics
%----------------------------------------------------------------------------------------
\usepackage{xcolor}
\usepackage{graphicx, multicol} % Enhanced support for graphics
\graphicspath{FIG/}
\usepackage{wrapfig}
\usepackage{colortbl}
\usepackage[framemethod=tikz]{mdframed}
%\usepackage{xsavebox}
% Il faudrait utiliser xsavebox à l'avenir pour réduire la taille du pdf

%----------------------------------------------------------------------------------------
%		Other packages
%----------------------------------------------------------------------------------------
\usepackage{hyperref}
\hypersetup{
	colorlinks=true, %colorise les liens
	breaklinks=true, %permet le retour à la ligne dans les liens trop longs
	urlcolor= sacado_violet,  %couleur des hyperliens et des QR codes
	linkcolor= sacado_violet, %couleur des liens internes
	plainpages=false  %pour palier à "Bookmark problems can occur when you have duplicate page numbers, for example, if you have a page i and a page 1."
}
\usepackage{tabularx}
\newcolumntype{M}[1]{>{\arraybackslash}m{#1}} %Defines a scalable column type in tabular
\usepackage{booktabs} % Enhances quality of tables
\usepackage{diagbox} % barre en diagonale dans un tableau
\usepackage{multicol}
\usepackage[explicit]{titlesec}
\usepackage{xr}
\usepackage{xspace}
\usepackage{array}
\usepackage{listings}
\usepackage{fancyvrb} %verbatim
\usepackage{stmaryrd}
\usepackage{float}



% Python style for highlighting
\lstdefinestyle{mystyle}{
    backgroundcolor=\color{white},   
    commentstyle=\color{sacado_green},
    keywordstyle=\color{sacado_red},
    numberstyle=\tiny\color{sacado_orange},
    stringstyle=\color{sacado_blue},
    basicstyle=\ttfamily\footnotesize,
    breakatwhitespace=false,         
    breaklines=true,                 
    captionpos=b,                    
    keepspaces=false,                 
    numbers=left,                    
    numbersep=5pt,                  
    showspaces=false,                
    showstringspaces=false,
    showtabs=false,                  
    tabsize=4
}

\lstset{style=mystyle}

%----------------------------------------------------------------------------------------
%		Headers and footers
%----------------------------------------------------------------------------------------

\pagestyle{empty}
\usepackage{fancyhdr}
\pagestyle{fancy}
\renewcommand{\headrulewidth}{0pt} % pas de filet sous le header

%----------------------------------------------------------------------------------------
%		Mathematics packages
%----------------------------------------------------------------------------------------
\usepackage{amsthm, amsmath, amssymb, mathrsfs} % Mathematical typesetting
\usepackage{marvosym, wasysym} % More symbols
\usepackage[makeroom]{cancel}
\usepackage{xlop}
\usepackage{pgf,tikz,pgfplots}
\pgfplotsset{compat=1.16}
\usepackage{pgf-pie}
\usetikzlibrary{positioning}
\usetikzlibrary{arrows}
\usepackage{pst-plot,pst-tree,pst-func, pstricks-add,pst-node,pst-text}
%\usepackage{units}
\usepackage{nicefrac}
\usepackage[np]{numprint} %Séparation milliers dans un nombre \np{12345} donne 12 345
\usepackage{multido}
\newcommand{\RNum}[1]{\uppercase\expandafter{\romannumeral #1\relax}}

%----------------------------------------------------------------------------------------
%		New text commands
%----------------------------------------------------------------------------------------
\usepackage{calc}
\usepackage{boites}
 \renewcommand{\arraystretch}{1.6}

%%%%% Pour les imports.
\usepackage{import}

%%%%% Pour faire des boites
\usepackage[tikz]{bclogo}
\usepackage{bclogo}
\usepackage{framed}
\usepackage[skins]{tcolorbox}
\tcbuselibrary{breakable}
\tcbuselibrary{skins}
\usetikzlibrary{quotes,babel,arrows.meta,shadows,decorations.pathmorphing,decorations.markings,patterns}
\usepackage{tikzpagenodes}
\usetikzlibrary{plotmarks}



%%%%% Pour une double minipage
\newcommand{\mini}[4]{
\begin{minipage}[c]{#1}
#2
\end{minipage}
\hfill
\begin{minipage}[c]{#3}
#4
\end{minipage}
}


\usepackage{enumitem}
\newlist{todolist}{itemize}{2} %Pour faire des QCM
\setlist[todolist]{label=$\square$} %Pour faire des QCM \begin{todolist} instead of itemize
\renewcommand{\FrenchLabelItem}{\textbullet} %bullet dans les items


%----------------------------------------------------------------------------------------
%		Définition de couleurs pour ...
%----------------------------------------------------------------------------------------

%GEOGEBRA

\definecolor{zzttqq}{rgb}{0.6,0.2,0.} %rouge des polygones
\definecolor{qqqqff}{rgb}{0.,0.,1.}
\definecolor{xdxdff}{rgb}{0.49019607843137253,0.49019607843137253,1.}%bleu
\definecolor{qqwuqq}{rgb}{0.,0.39215686274509803,0.} %vert des angles
\definecolor{ffqqqq}{rgb}{1.,0.,0.} %rouge vif
\definecolor{uuuuuu}{rgb}{0.26666666666666666,0.26666666666666666,0.26666666666666666}
\definecolor{qqzzqq}{rgb}{0.,0.6,0.}
\definecolor{cqcqcq}{rgb}{0.7529411764705882,0.7529411764705882,0.7529411764705882} %gris
\definecolor{qqffqq}{rgb}{0.,1.,0.}
\definecolor{ffdxqq}{rgb}{1.,0.8431372549019608,0.}
\definecolor{ffffff}{rgb}{1.,1.,1.}
\definecolor{ududff}{rgb}{0.30196078431372547,0.30196078431372547,1.}
\definecolor{ffqqff}{rgb}{1.,0.,1.}
\definecolor{ffxfqq}{rgb}{1,0.4980392156862745,0}
\definecolor{ffffqq}{rgb}{1,1,0}
\definecolor{qqttzz}{rgb}{0,0.2,0.6}
\definecolor{qqccqq}{rgb}{0,0.8,0}
\definecolor{qqzzff}{rgb}{0,0.6,1}
\definecolor{qqwwzz}{rgb}{0,0.4,0.6}
\definecolor{eqeqeq}{rgb}{0.8784313725490196,0.8784313725490196,0.8784313725490196}

%SACADO

\definecolor{fond}{HTML}{5D4391}  %couleur des entetes etc.  violet sacado
\definecolor{sacado_purple}{HTML}{5D4391} %% Violet foncé Sacado
\definecolor{sacado_violet}{HTML}{9274C7} %% Violet clair Sacado
\definecolor{texte}{HTML}{FFFFFF} % couleur du texte des entetes etc.
\definecolor{sacado_blue_light}{HTML}{0093CA} %% Bleu Sacado
\definecolor{sacado_blue}{HTML}{0960B5} %% Bleu Sacado
\definecolor{sacado_green}{HTML}{00B999} %% Vert Sacado
\definecolor{sacado_green_dark}{HTML}{4D8075} %% Vert Sacado foncé
\definecolor{sacado_yellow}{HTML}{F9F871} %% Jaune Sacado
\definecolor{sacado_orange}{HTML}{FF8B69} %% Orange Sacado
\definecolor{sacado_red}{HTML}{9F1E17} %% Rouge Sacado
\definecolor{sacado_gray}{HTML}{7B7485} %% Gris Sacado
%BOITES 

\definecolor{bleu1}{rgb}{0.54,0.79,0.95} %% Bleu
\definecolor{sapgreen}{rgb}{0.4, 0.49, 0}
\definecolor{dvzfxr}{rgb}{0.7,0.4,0.}
\definecolor{beamer}{rgb}{0.5176470588235295,0.49019607843137253,0.32941176470588235} % couleur beamer
\definecolor{preuveRbeamer}{rgb}{0.8,0.4,0}
\definecolor{sectioncolor}{rgb}{0.24,0.21,0.44}
\definecolor{subsectioncolor}{rgb}{0.1,0.21,0.61}
\definecolor{subsubsectioncolor}{rgb}{0.1,0.21,0.61}
\definecolor{info}{rgb}{0.82,0.62,0}
\definecolor{bleu2}{rgb}{0.38,0.56,0.68}
\definecolor{bleu3}{rgb}{0.24,0.34,0.40}
\definecolor{bleu4}{rgb}{0.12,0.20,0.25}
\definecolor{vert}{rgb}{0.21,0.33,0}
\definecolor{vertS}{rgb}{0.05,0.6,0.42}
\definecolor{red}{rgb}{0.78,0,0}
\definecolor{color5}{rgb}{0,0.4,0.58}
\definecolor{eduscol4B}{rgb}{0.19,0.53,0.64}
\definecolor{eduscol4P}{rgb}{0.62,0.12,0.39}
\definecolor{ill_frame}{HTML}{F0F0F0} %Boite illustration contour
\definecolor{ill_back}{HTML}{F7F7F7}  %Boite illustration background
\definecolor{ill_title}{HTML}{AAAAAA} %Boite illustration titre

%----------------------------------------------------------------------------------------
%		QR codes
%----------------------------------------------------------------------------------------

\usepackage[
final %Pour la compilation finale
%draft %Pour le travail sur les documents
]{qrcode}
\usepackage{fontawesome}
\usepackage{fancyqr}
\FancyQrLoad{flat}
\fancyqrset{
%image=\scalebox{.8}{\includegraphics[scale=1]{sacadoA1.png}},image padding=.5,
l color=sacado_green,r color=sacado_blue}
\newcommand{\qr}[2]{\centering \fancyqr{https://sacado.xyz/qcm/show_course_from_qrcode/#1}

\vspace{.2cm}

#2} %\qr{id} Pour obtenir un qrcode en indiquant seulement l'id de l'exercice





\newcommand{\miniqr}[3]{
\begin{minipage}[c]{.8\linewidth}
#1
\end{minipage}
\hfill
\fbox{
\begin{minipage}[c]{.18\linewidth}
\begin{center}
\fancyqr{https://sacado.xyz/qcm/show\_course\_from\_qrcode/#2}

\vspace{.2cm}

#3
\end{center}
\end{minipage}
}
}

%practice/frombook/<int:ide>/ pour accéder à un exercice depuis le livre.

\usepackage{pythontex}
\begin{pycode}
import qrcode
def qr(data):
     fic=r"QRcodes/qr"+data+'.png'
     urlcourte=r"sacado.xyz/"+data
     urllongue=r"https://"+urlcourte
     qr = qrcode.QRCode(version=1,
error_correction=qrcode.constants.ERROR_CORRECT_L,box_size=2, border=0)
     qr.add_data(urlcourte)
     qr.make(fit=True)
     qr_image = qr.make_image(fill_color="black", back_color="white")
     qr_image.save(fic)
     return r"""\parbox{3.5cm}{\begin{center}
\includegraphics{"""+fic+r"}\\{\scriptsize\tt "+urlcourte+r"}\end{center}}"
\end{pycode}

%\renewcommand{\qr}[1]{\py{qr("#1")}} % compilation complete
%utiliser :
%  pdflatex --shell-escape MANUEL_6e_V1.tex ; pythontex MANUEL_6e_V1.tex --interpreter python:python3 ; pdflatex --shell-escape % %MANUEL_6e_V1.tex

%  draft
\renewcommand{\qr}[1]{
\parbox{5cm}{\begin{center}
       \includegraphics{QRcodes/qrDummy.png}
       %\\{\tt dummy}
\end{center}}
}

\renewcommand{\miniqr}[1]{
       \includegraphics[height=1cm]{QRcodes/qrDummy.png}
       %\\{\tt dummy}}
}


\usepackage[absolute]{textpos}
\newcommand{\qrHautDroite}[1]{
\setlength{\TPHorizModule}{1cm}
\setlength{\TPVertModule}{1cm}
\begin{textblock}{3.51}(12.5,1.5){\qr{#1}}
\end{textblock}
}

 






\usepackage{makeidx}
\makeindex
