%----------------------------------------------------------------------------------------
%	PACKAGES AND OTHER DOCUMENT CONFIGURATIONS
%----------------------------------------------------------------------------------------

%----------------------------------------------------------------------------------------
%		Géometrie de la page
%----------------------------------------------------------------------------------------
\documentclass[dvipsnames,french,10pt]{book}

\usepackage[
paperheight=29.7cm, %hauteur du papier
paperwidth=21cm, %largeur du papier
left=1cm, %marge de gauche
right=1cm, %marge de droite
top=1.5cm, %marge du haut
bottom=1cm, %marge du bas
%marginparsep=0pt, %distance entre le texte et les notes de marges 
reversemp, %inverse l'emplacement de la marge
headheight=20.60pt %hauteur du header
%showframe, %permet d'afficher le cadre défini ci-dessus
%bindingoffset=1cm %permet d'ajouter le décalage dû au reliage
]{geometry} %Redéfinition de la taille des pages
\raggedbottom


%----------------------------------------------------------------------------------------
%		Generals
%----------------------------------------------------------------------------------------
%\usepackage{fourier} %!! A changer plus tard !!
\usepackage[scaled]{uarial}
\renewcommand*\familydefault{\sfdefault} %% Only if the base font of the document is to be sans serif
\usepackage{frcursive}
\usepackage[T1]{fontenc} %Accents handling
\usepackage[utf8]{inputenc} % Use UTF-8 encoding
%\usepackage{microtype} % Slightly tweak font spacing for aesthetics
\usepackage[english, francais]{babel} % Language hyphenation and typographical rules
\usepackage{marginnote}


%----------------------------------------------------------------------------------------
%		Graphics
%----------------------------------------------------------------------------------------
\usepackage{xcolor}
\usepackage{graphicx, multicol} % Enhanced support for graphics
\graphicspath{FIG/}
\usepackage{wrapfig}
\usepackage{colortbl}
\usepackage[framemethod=tikz]{mdframed}
%\usepackage{xsavebox}
% Il faudrait utiliser xsavebox à l'avenir pour réduire la taille du pdf

%----------------------------------------------------------------------------------------
%		Other packages
%----------------------------------------------------------------------------------------
\usepackage{hyperref}
\hypersetup{
	colorlinks=true, %colorise les liens
	breaklinks=true, %permet le retour à la ligne dans les liens trop longs
	urlcolor= sacado_violet,  %couleur des hyperliens et des QR codes
	linkcolor= sacado_violet, %couleur des liens internes
	plainpages=false  %pour palier à "Bookmark problems can occur when you have duplicate page numbers, for example, if you have a page i and a page 1."
}
\usepackage{tabularx}
\newcolumntype{M}[1]{>{\arraybackslash}m{#1}} %Defines a scalable column type in tabular
\usepackage{booktabs} % Enhances quality of tables
\usepackage{diagbox} % barre en diagonale dans un tableau
\usepackage{multicol}
\usepackage[explicit]{titlesec}
\usepackage{xr}
\usepackage{xspace}
\usepackage{array}
\usepackage{listings}
\usepackage{fancyvrb} %verbatim
\usepackage{stmaryrd}
\usepackage{float}



% Python style for highlighting
\lstdefinestyle{mystyle}{
    backgroundcolor=\color{white},   
    commentstyle=\color{sacado_green},
    keywordstyle=\color{sacado_red},
    numberstyle=\tiny\color{sacado_orange},
    stringstyle=\color{sacado_blue},
    basicstyle=\ttfamily\footnotesize,
    breakatwhitespace=false,         
    breaklines=true,                 
    captionpos=b,                    
    keepspaces=false,                 
    numbers=left,                    
    numbersep=5pt,                  
    showspaces=false,                
    showstringspaces=false,
    showtabs=false,                  
    tabsize=4
}

\lstset{style=mystyle}

%----------------------------------------------------------------------------------------
%		Headers and footers
%----------------------------------------------------------------------------------------

\pagestyle{empty}
\usepackage{fancyhdr}
\pagestyle{fancy}
\renewcommand{\headrulewidth}{0pt} % pas de filet sous le header

%----------------------------------------------------------------------------------------
%		Mathematics packages
%----------------------------------------------------------------------------------------
\usepackage{amsthm, amsmath, amssymb, mathrsfs} % Mathematical typesetting
\usepackage{marvosym, wasysym} % More symbols
\usepackage[makeroom]{cancel}
\usepackage{xlop}
\usepackage{pgf,tikz,pgfplots}
\pgfplotsset{compat=1.16}
\usepackage{pgf-pie}
\usetikzlibrary{positioning}
\usetikzlibrary{arrows}
\usepackage{pst-plot,pst-tree,pst-func, pstricks-add,pst-node,pst-text}
%\usepackage{units}
\usepackage{nicefrac}
\usepackage[np]{numprint} %Séparation milliers dans un nombre \np{12345} donne 12 345
\usepackage{multido}
\newcommand{\RNum}[1]{\uppercase\expandafter{\romannumeral #1\relax}}

%----------------------------------------------------------------------------------------
%		New text commands
%----------------------------------------------------------------------------------------
\usepackage{calc}
\usepackage{boites}
 \renewcommand{\arraystretch}{1.6}

%%%%% Pour les imports.
\usepackage{import}

%%%%% Pour faire des boites
\usepackage[tikz]{bclogo}
\usepackage{bclogo}
\usepackage{framed}
\usepackage[skins]{tcolorbox}
\tcbuselibrary{breakable}
\tcbuselibrary{skins}
\usetikzlibrary{quotes,babel,arrows.meta,shadows,decorations.pathmorphing,decorations.markings,patterns}
\usepackage{tikzpagenodes}
\usetikzlibrary{plotmarks}



%%%%% Pour une double minipage
\newcommand{\mini}[4]{
\begin{minipage}[c]{#1}
#2
\end{minipage}
\hfill
\begin{minipage}[c]{#3}
#4
\end{minipage}
}


\usepackage{enumitem}
\newlist{todolist}{itemize}{2} %Pour faire des QCM
\setlist[todolist]{label=$\square$} %Pour faire des QCM \begin{todolist} instead of itemize
\renewcommand{\FrenchLabelItem}{\textbullet} %bullet dans les items


%----------------------------------------------------------------------------------------
%		Définition de couleurs pour ...
%----------------------------------------------------------------------------------------

%GEOGEBRA

\definecolor{zzttqq}{rgb}{0.6,0.2,0.} %rouge des polygones
\definecolor{qqqqff}{rgb}{0.,0.,1.}
\definecolor{xdxdff}{rgb}{0.49019607843137253,0.49019607843137253,1.}%bleu
\definecolor{qqwuqq}{rgb}{0.,0.39215686274509803,0.} %vert des angles
\definecolor{ffqqqq}{rgb}{1.,0.,0.} %rouge vif
\definecolor{uuuuuu}{rgb}{0.26666666666666666,0.26666666666666666,0.26666666666666666}
\definecolor{qqzzqq}{rgb}{0.,0.6,0.}
\definecolor{cqcqcq}{rgb}{0.7529411764705882,0.7529411764705882,0.7529411764705882} %gris
\definecolor{qqffqq}{rgb}{0.,1.,0.}
\definecolor{ffdxqq}{rgb}{1.,0.8431372549019608,0.}
\definecolor{ffffff}{rgb}{1.,1.,1.}
\definecolor{ududff}{rgb}{0.30196078431372547,0.30196078431372547,1.}
\definecolor{ffqqff}{rgb}{1.,0.,1.}
\definecolor{ffxfqq}{rgb}{1,0.4980392156862745,0}
\definecolor{ffffqq}{rgb}{1,1,0}
\definecolor{qqttzz}{rgb}{0,0.2,0.6}
\definecolor{qqccqq}{rgb}{0,0.8,0}
\definecolor{qqzzff}{rgb}{0,0.6,1}
\definecolor{qqwwzz}{rgb}{0,0.4,0.6}
\definecolor{eqeqeq}{rgb}{0.8784313725490196,0.8784313725490196,0.8784313725490196}

%SACADO

\definecolor{fond}{HTML}{5D4391}  %couleur des entetes etc.  violet sacado
\definecolor{sacado_purple}{HTML}{5D4391} %% Violet foncé Sacado
\definecolor{sacado_violet}{HTML}{9274C7} %% Violet clair Sacado
\definecolor{texte}{HTML}{FFFFFF} % couleur du texte des entetes etc.
\definecolor{sacado_blue_light}{HTML}{0093CA} %% Bleu Sacado
\definecolor{sacado_blue}{HTML}{0960B5} %% Bleu Sacado
\definecolor{sacado_green}{HTML}{00B999} %% Vert Sacado
\definecolor{sacado_green_dark}{HTML}{4D8075} %% Vert Sacado foncé
\definecolor{sacado_yellow}{HTML}{F9F871} %% Jaune Sacado
\definecolor{sacado_orange}{HTML}{FF8B69} %% Orange Sacado
\definecolor{sacado_red}{HTML}{9F1E17} %% Rouge Sacado
\definecolor{sacado_gray}{HTML}{7B7485} %% Gris Sacado
%BOITES 

\definecolor{bleu1}{rgb}{0.54,0.79,0.95} %% Bleu
\definecolor{sapgreen}{rgb}{0.4, 0.49, 0}
\definecolor{dvzfxr}{rgb}{0.7,0.4,0.}
\definecolor{beamer}{rgb}{0.5176470588235295,0.49019607843137253,0.32941176470588235} % couleur beamer
\definecolor{preuveRbeamer}{rgb}{0.8,0.4,0}
\definecolor{sectioncolor}{rgb}{0.24,0.21,0.44}
\definecolor{subsectioncolor}{rgb}{0.1,0.21,0.61}
\definecolor{subsubsectioncolor}{rgb}{0.1,0.21,0.61}
\definecolor{info}{rgb}{0.82,0.62,0}
\definecolor{bleu2}{rgb}{0.38,0.56,0.68}
\definecolor{bleu3}{rgb}{0.24,0.34,0.40}
\definecolor{bleu4}{rgb}{0.12,0.20,0.25}
\definecolor{vert}{rgb}{0.21,0.33,0}
\definecolor{vertS}{rgb}{0.05,0.6,0.42}
\definecolor{red}{rgb}{0.78,0,0}
\definecolor{color5}{rgb}{0,0.4,0.58}
\definecolor{eduscol4B}{rgb}{0.19,0.53,0.64}
\definecolor{eduscol4P}{rgb}{0.62,0.12,0.39}
\definecolor{ill_frame}{HTML}{F0F0F0} %Boite illustration contour
\definecolor{ill_back}{HTML}{F7F7F7}  %Boite illustration background
\definecolor{ill_title}{HTML}{AAAAAA} %Boite illustration titre

%----------------------------------------------------------------------------------------
%		QR codes
%----------------------------------------------------------------------------------------

\usepackage[
final %Pour la compilation finale
%draft %Pour le travail sur les documents
]{qrcode}
\usepackage{fontawesome}
\usepackage{fancyqr}
\FancyQrLoad{flat}
\fancyqrset{
%image=\scalebox{.8}{\includegraphics[scale=1]{sacadoA1.png}},image padding=.5,
l color=sacado_green,r color=sacado_blue}
\newcommand{\qr}[2]{\centering \fancyqr{https://sacado.xyz/qcm/show_course_from_qrcode/#1}

\vspace{.2cm}

#2} %\qr{id} Pour obtenir un qrcode en indiquant seulement l'id de l'exercice





\newcommand{\miniqr}[3]{
\begin{minipage}[c]{.8\linewidth}
#1
\end{minipage}
\hfill
\fbox{
\begin{minipage}[c]{.18\linewidth}
\begin{center}
\fancyqr{https://sacado.xyz/qcm/show\_course\_from\_qrcode/#2}

\vspace{.2cm}

#3
\end{center}
\end{minipage}
}
}

%practice/frombook/<int:ide>/ pour accéder à un exercice depuis le livre.

\usepackage{pythontex}
\begin{pycode}
import qrcode
def qr(data):
     fic=r"QRcodes/qr"+data+'.png'
     urlcourte=r"sacado.xyz/"+data
     urllongue=r"https://"+urlcourte
     qr = qrcode.QRCode(version=1,
error_correction=qrcode.constants.ERROR_CORRECT_L,box_size=2, border=0)
     qr.add_data(urlcourte)
     qr.make(fit=True)
     qr_image = qr.make_image(fill_color="black", back_color="white")
     qr_image.save(fic)
     return r"""\parbox{3.5cm}{\begin{center}
\includegraphics{"""+fic+r"}\\{\scriptsize\tt "+urlcourte+r"}\end{center}}"
\end{pycode}

%\renewcommand{\qr}[1]{\py{qr("#1")}} % compilation complete
%utiliser :
%  pdflatex --shell-escape MANUEL_6e_V1.tex ; pythontex MANUEL_6e_V1.tex --interpreter python:python3 ; pdflatex --shell-escape % %MANUEL_6e_V1.tex

%  draft
\renewcommand{\qr}[1]{
\parbox{5cm}{\begin{center}
       \includegraphics{QRcodes/qrDummy.png}
       %\\{\tt dummy}
\end{center}}
}

\renewcommand{\miniqr}[1]{
       \includegraphics[height=1cm]{QRcodes/qrDummy.png}
       %\\{\tt dummy}}
}


\usepackage[absolute]{textpos}
\newcommand{\qrHautDroite}[1]{
\setlength{\TPHorizModule}{1cm}
\setlength{\TPVertModule}{1cm}
\begin{textblock}{3.51}(12.5,1.5){\qr{#1}}
\end{textblock}
}

 






\usepackage{makeidx}
\makeindex

%----------------------------------------
%
%   Définitions des environnements "pageCours" et "pageExos"
%
%----------------------------------------

\newcounter{cpt}
\newcounter{exo}
\newcounter{cptr}

\newcommand{\titreChap}{Titre de chapitre à définir}

\renewcommand{\chapter}[3]{
  \stepcounter{chapter}
  \setcounter{exo}{0}
  \setcounter{cpt}{0}
  
%\cleardoublepage  % pour commencer à droite
{\Huge \hfill Chapitre \Roman{chapter}.\\
  \bigskip
  #1\\
  \bigskip {\begin{center}
  \fancyqr[image={\includegraphics[scale=.6]{sacadoA1.png}},image padding=.5,height=5cm]{#2}
  \end{center}}  {\normalsize #3}}
\renewcommand{\titreChap}{#1}

%\ifthenelse{\equal{#2}{}}{}{\par
%  \bigskip\bigskip
%  #2}
\newpage
}

\renewenvironment{leftbar}[1][\hsize]
{%
    \def\FrameCommand
    {%
        {\color{black}\vrule width 0.5pt}%
        \hspace{4pt}%must no space.
        \fboxsep=\FrameSep%\colorbox{yellow}%
    }%
    \MakeFramed{\hsize#1\advance\hsize-\width\FrameRestore}%
}
{\endMakeFramed}


\newcommand{\headerGeneral}[3]{ % intitulé, couleur, qrcode
\begin{tikzpicture}[remember picture,overlay]
\coordinate(NO) at (-2,0);
\coordinate(SW) at (22,1);
\coordinate(titre) at (0,0.2);
\coordinate(qr) at (16.85,0.);
\shade[left color=#2 , right color=#2 ] (NO) rectangle (SW);
\draw (titre) node[color=texte, anchor=west]{ {\large \bf  #1} \quad\quad \bf {\small \titreChap} };
\draw (qr) node {\qr{#3}};
\end{tikzpicture}
}

\newenvironment{pageCours}{\lhead{%
\pagecolor{white!100}
\headerGeneral{COURS}{fond!70}{p/1234}
}\begin{leftbar}}{\end{leftbar}\newpage}

\newenvironment{pageAD}{\lhead{%
\pagecolor{sacado_violet!6}
\headerGeneral{APPLICATIONS DIRECTES}{sacado_violet!70}{p/1234}
} }{ \newpage}

\newenvironment{pageParcoursu}{\lhead{%
\pagecolor{sacado_green!6} 
\headerGeneral{PARCOURS 1}{sacado_green}{p/1234}
} }{ \newpage}


\newenvironment{pageParcoursd}{\lhead{%
\pagecolor{sacado_blue!6}
\headerGeneral{PARCOURS 2}{sacado_blue_light}{p/1234}
} }{ \newpage}

\newenvironment{pageParcourst}{\lhead{%
\pagecolor{sacado_red!6}
\headerGeneral{PARCOURS 3}{sacado_red}{p/1234}
} }{ \newpage}

\newenvironment{pageBrouillon}{\lhead{%
\pagecolor{sacado_gray!6}
\headerGeneral{BROUILLON}{sacado_gray}{p/1234}
} }{ \newpage}

\newenvironment{pageRituels}{\lhead{%
\pagecolor{fond!6}
\headerGeneral{RITUELS}{fond!70}{p/1234}
} }{ \newpage}

\newenvironment{pageAuto}{\lhead{%
\pagecolor{sacado_orange!6}
\headerGeneral{AUTOÉVALUATION}{sacado_orange}{p/1234}
} }{ \newpage}

\newenvironment{pageHistoire}{\lhead{%
\pagecolor{olive!6}
\headerGeneral{HISTOIRE}{olive}{p/1234}
} }{ \newpage}



\newenvironment{pageExercices}{\lhead{%
\pagecolor{white!100}
\headerGeneral{ACTIVITÉS}{fond}{p/1234}
}\begin{leftbar}}{\end{leftbar}\newpage}



\fancyfoot[L]{\colorbox{fond!70}{\color{texte}\thepage}}
\fancyfoot[C]{}


\newcommand{\titresec}[2]{\phantom{.}\begin{textblock}{1}[0,1](-1.24,0.25)\colorbox{fond!70}{%
\makebox[0.8cm]{\raisebox{0.05cm}[0.6cm][0.15cm]{\color{texte}\LARGE\bf #1}}}\end{textblock}{\LARGE\bf #2}\\\bigskip}

\renewcommand{\thesection}{\arabic{section}}
\titleformat{\section}{}{%
\hspace{-1.15cm}\colorbox{fond!70}{%
\makebox[0.8cm]{\raisebox{0.05cm}[0.6cm][0.15cm]{\color{texte}\LARGE\bf \thesection}}}}{1em}{\bf \LARGE #1}
  
\renewcommand{\thesubsection}{\arabic{subsection}}
            
\titleformat{\subsection}
{%\begin{textblock}{1}[0,1](-1,0.42) toto
  %\end{textblock}
%\reversemarginpar\marginnote[\rule{0.8cm}{0.8cm}]{}[0pt]  \color{red}\normalfont\Large\bfseries}
}{\hspace{-0.83em}
\colorbox{fond!70}{\makebox[0.6cm]{\raisebox{0cm}[1em][0.2em]\normalfont\large\bfseries\color{texte}\thesubsection}}}{1em}{\bf \large #1}




\makeatletter
\newenvironment{TraitV}[1]{%
% #1 couleur du trait (par défaut CouleurA)
% #2 largeur du trait
% #3 distance entre le trait et le texte
\def\FrameCommand{{\color{#1}\vrule width 2pt}
\hspace{1em}}\MakeFramed {\advance\hsize-\width}}%
{\endMakeFramed}
\makeatother

%----------------------------------------
%
%   Définitions des environnements de Définitions, propriétés...
%
%----------------------------------------

%%%%%%%%%%%%% Définitions
\newenvironment{Def}{%
\medskip \begin{tcolorbox}[widget,colback=sacado_violet!15,colframe=sacado_violet!75!black,
title= \stepcounter{cpt} Définition \thecpt. ]}{%
\end{tcolorbox}\par}


\newenvironment{DefT}[1]{%
\medskip \begin{tcolorbox}[widget,colback=sacado_violet!15,colframe=sacado_violet!75!black,
title= \stepcounter{cpt} Définition \thecpt : #1.]}
{%
\end{tcolorbox}\par}


%%%%%%%%%%%%% Proposition
\newenvironment{Prop}{%
\medskip \begin{tcolorbox}[widget,colback=sacado_blue!15,colframe=sacado_blue!75!black,
title= \stepcounter{cpt} Proposition \thecpt.]}
{%
\end{tcolorbox}\par}


%%%%%%%%%%%%% Propriétés
\newenvironment{Pp}{%
\medskip \begin{tcolorbox}[widget,colback=white!100,colframe=sacado_violet!75!black,
title= \stepcounter{cpt} Propriété \thecpt.]}
{%
\end{tcolorbox}\par}

\newenvironment{PpT}[1]{%
\medskip \begin{tcolorbox}[widget,colback=white!100,colframe=sacado_violet!75!black,
title= \stepcounter{cpt} Propriété \thecpt : #1. ]}
{%
\end{tcolorbox}\par}

\newenvironment{Pps}{%
\medskip \begin{tcolorbox}[widget,colback=white!100,colframe=sacado_violet!75!black,
title= \stepcounter{cpt} Propriétés \thecpt.]}
{%
\end{tcolorbox}\par}


%%%%%%%%%%%%% Conséquence
\newenvironment{Cq}{%
\medskip \begin{tcolorbox}[widget,colback=white,colframe=sacado_blue,
title= \stepcounter{cpt} Conséquence \thecpt.]}
{%
\end{tcolorbox}\par}



%%%%%%%%%%%%% Théorèmes
\newenvironment{ThT}[1]{% théorème avec titre
\medskip \begin{tcolorbox}[widget,colback=white!100,colframe=sacado_violet!75!black,
title= \stepcounter{cpt} Théorème \thecpt : #1.]}
{%
\end{tcolorbox}\par}

\newenvironment{Th}{%
\medskip \begin{tcolorbox}[widget,colback=white!100,colframe=sacado_violet!75!black,
title= \stepcounter{cpt} Théorème \thecpt.]}
{%
\end{tcolorbox}\par}


%%%%%%%%%%%%% Règles
\newenvironment{Reg}{%
\medskip \begin{tcolorbox}[widget,colback=sacado_blue!15,colframe=sacado_blue,
title= \stepcounter{cpt} Règle \thecpt.]}
{%
\end{tcolorbox}\par}

%%%%%%%%%%%%% Représentations
\newenvironment{Rep}{%
\medskip \begin{tcolorbox}[widget,colback=white,colframe=sacado_violet!75!white,
title= \stepcounter{cpt} Représentation \thecpt.]}
{%
\end{tcolorbox}\par}

 
%%%%%%%%%%%%% REMARQUES
\newenvironment{Rq}{%
\medskip \begin{tcolorbox}[widget,colback=sacado_orange!15,colframe=sacado_orange,
title= \stepcounter{cpt} Remarque \thecpt.]}
{%
\end{tcolorbox}\par}

\newenvironment{Rqs}{%
\medskip \begin{tcolorbox}[widget,colback=sacado_orange!15,colframe=sacado_orange,
title= \stepcounter{cpt} Remarques \thecpt.]}
{%
\end{tcolorbox}\par}


%%%%%%%%%%%%% EXEMPLES
\newenvironment{Ex}{%
\medskip \begin{tcolorbox}[widget,colback=white,colframe=sacado_blue_light,
title= \stepcounter{cpt} Exemple \thecpt.]}
{%
\end{tcolorbox}\par}

\newenvironment{Exs}{%
\medskip \begin{tcolorbox}[widget,colback=white!15,colframe=sacado_blue_light,
title= \stepcounter{cpt} Exemples \thecpt.]}
{%
\end{tcolorbox}\par}

 
\newenvironment{ExT}[1]{%
\medskip \begin{tcolorbox}[widget,colback=white,colframe=sacado_blue_light,
title= \stepcounter{cpt} Exemple \thecpt   : #1.]}
{%
\end{tcolorbox}\par}

 
\newenvironment{ExCor}{%
\medskip \begin{tcolorbox}[widget,colback=white,colframe=sacado_blue ,
title= \stepcounter{cpt} Exercice commenté \thecpt.]}
{%
\end{tcolorbox}\par}

\newenvironment{ExQr}[1]{%
\medskip \begin{tcolorbox}[widget,colback=white,colframe=sacado_blue_light ,
title= \stepcounter{cpt} Exemple  \thecpt. \hfill {\color{sacado_blue}https://sacado.xyz/a/#1} ]
\begin{minipage}{1.5cm}
\miniqr{#1}
\end{minipage}
\begin{minipage}{0.8\linewidth}
}
{%
\end{minipage}
\end{tcolorbox}\par}


\newenvironment{OuQr}[1]{%
\medskip \begin{tcolorbox}[widget,colback=white,colframe=sacado_orange ,
title= \stepcounter{cpt} Outil \thecpt. \hfill {\color{sacado_orange}https://sacado.xyz/a/#1} ]
\begin{minipage}{1.5cm}
\miniqr{#1}
\end{minipage}
\begin{minipage}{0.8\linewidth}
 \color{sacado_orange!90!black}
}
{%
 
\end{minipage}
\end{tcolorbox}\par}



\newenvironment{MeQr}[1]{%
\medskip \begin{tcolorbox}[widget,colback=white,colframe=sacado_blue,
title= \stepcounter{cpt} Méthode \thecpt. \hfill {\color{sacado_blue}https://sacado.xyz/a/#1} ]
\begin{minipage}{1.5cm}
\miniqr{#1}
\end{minipage}
\begin{minipage}{0.8\linewidth}
}
{%
\end{minipage}
\end{tcolorbox}\par}




%%%%%%%%%%%%% Logique
\newenvironment{Log}{%
\medskip \begin{tcolorbox}[widget,colback=sacado_blue!10,colframe=sacado_blue,
title= \stepcounter{cpt} Logique mathématique \thecpt.]}
{%
\end{tcolorbox}\par}
%%%%%%%%%%%%% Logique avec paramètre
\newenvironment{LogT}[1]{%
\medskip \begin{tcolorbox}[widget,colback=sacado_blue!10,colframe=sacado_blue,
title= \stepcounter{cpt} Logique mathématique \thecpt. #1]}
{%
\end{tcolorbox}\par}

%%%%%%%%%%%%% Preuve
\newenvironment{Pv}[1][]{%
\begin{tcolorbox}[breakable, enhanced,widget, colback=sacado_blue!10!white,boxrule=0pt,frame hidden,
borderline west={1mm}{0mm}{sacado_blue!75}]
\textbf{Preuve#1 : }}
{%
\end{tcolorbox}
\par}


%%%%%%%%%%%%% PreuveROC
\newenvironment{PvR}[1][]{%
\begin{tcolorbox}[breakable, enhanced,widget, colback=sacado_blue!10!white,boxrule=0pt,frame hidden,
borderline west={1mm}{0mm}{sacado_blue!75}]
\textbf{Preuve (ROC)#1 : }}
{%
\end{tcolorbox}
\par}


%%%%%%%%%%%%% DemoExigible
\newenvironment{DemoE}{%
\medskip \begin{tcolorbox}[widget,colback=sacado_blue!10,colframe=sacado_blue,
title= \stepcounter{cpt} Démonstration exigible \thecpt. ]}
{%
\end{tcolorbox}\par}





%%%%%%%%%%%%% Compétences
\newenvironment{Cps}[1][]{%
\vspace{0.4cm}
\begin{tcolorbox}[enhanced, lifted shadow={0mm}{0mm}{0mm}{0mm}%
{black!60!white}, attach boxed title to top left={xshift=5mm, yshift*=-3mm}, coltitle=white, colback=white, boxed title style={colback=sacado_green!100}, colframe=sacado_green!75!black,title=\textbf{Compétences associées#1}]}
{%
\end{tcolorbox}
\par}

%%%%%%%%%%%%% Chapitres connexes
\newenvironment{CCon}[1][]{%
\vspace{0.4cm}
\begin{tcolorbox}[breakable, enhanced,widget, colback=white ,boxrule=0pt,frame hidden,
borderline west={2mm}{0mm}{sacado_violet}]
\textbf{#1}}
{%
\end{tcolorbox}
\par}
%%%%%%%%%%%%% Compétences Collège
\newenvironment{CpsCol}[1][]{%
\vspace{0.4cm}
\begin{tcolorbox}[breakable, enhanced,widget, colback=white ,boxrule=0pt,frame hidden,
borderline west={2mm}{0mm}{sacado_violet}]
\textbf{#1}}
{%
\end{tcolorbox}
\par}


 

%%%%%%%%%%%%% Rituel
\newenvironment{Rit}{%
\medskip \begin{tcolorbox}[widget,colback=white!15,colframe=sacado_violet!75!black,
title= \stepcounter{cpt} Rituel \thecpt. ]}{%
\end{tcolorbox}\par}


%%%%%%%%%%%%% Méthode
\newenvironment{Mt}{%
\medskip \begin{tcolorbox}[widget,colback=white!15,colframe=sacado_violet!75!black,
title= \stepcounter{cpt} Méthode \thecpt. ]}{%
\end{tcolorbox}\par}

%%%%%%%%%%%%% Méthode
\newenvironment{MtT}[1]{%
\medskip \begin{tcolorbox}[widget,colback=white!15,colframe=sacado_violet!75!black,
title= \stepcounter{cpt} Méthode \thecpt. #1 ]}{%
\end{tcolorbox}\par}


%%%%%%%%%%%%% VocU
\newenvironment{VocU}[1]{%
\medskip \begin{tcolorbox}[widget,colback=white!15,colframe=sacado_violet!75,
title= \stepcounter{cpt} Vocabulaire \thecpt. #1 ]}{%
\end{tcolorbox}\par}


%%%%%%%%%%%%% Notation
\newenvironment{Nt}[1]{%
\medskip \begin{tcolorbox}[widget,colback=white!5,colframe=sacado_red!75,
title= \stepcounter{cptr} Notation \thecptr. #1 ]}{%
\end{tcolorbox}\par}

%%%%%%%%%%%%% Ety
\newenvironment{Ety}[1]{%
\medskip \begin{tcolorbox}[widget,colback=white!15,colframe=sacado_violet!75,
title= \stepcounter{cpt} Étymologie \thecpt. #1 ]}{%
\end{tcolorbox}\par}


%%%%%%%%%%%%% His
\newenvironment{His}[1]{%
\begin{tcolorbox}[right=5mm, enhanced, lifted shadow={0mm}{0mm}{0mm}{0mm}%
{sacado_green_dark!90!white}, attach boxed title to top left={xshift=0.3cm, yshift*=-2mm}, coltitle=sacado_green_dark, colback=sacado_green!10!white, boxed title style={colback=white}, colframe=sacado_green_dark,title= Les mathématiciennes et mathématiciens ]
}{%
\end{tcolorbox}\par}


%%%%%%%%%%%%% Attention
\newenvironment{Att}[1]{%
\medskip \begin{tcolorbox}[widget,colback=sacado_red!5,colframe=sacado_red!95!white,
title= \stepcounter{cpt} Notation \thecpt. #1 ]}{%
\end{tcolorbox}\par}



%%%%%%%%%%%%%%%%%%%%%%%%%%%%%%%%%%%%%%%%%%%%%%%%%%%%%%%%%%%%%%%%%%%%%%%%%%%%%%%%%%%%%%%%%%%%%%%%%%%%%%%%%%%%%%%%%%%%%%
%%%%%%%%%%%%%%%%%%%%%%%%%%%%%%%%%%%%%%%%%%%%%%%%%%%%%%%%%%%%%%%%%%%%%%%%%%%%%%%%%%%%%%%%%%%%%%%%%%%%%%%%%%%%%%%%%%%%%%
%%%%%%%%%%%%%%%%  Exercices                                            %%%%%%%%%%%%%%%%%%%%%%%%%%%%%%%%%%%%%%%%%%%%%%%
%%%%%%%%%%%%%%%%%%%%%%%%%%%%%%%%%%%%%%%%%%%%%%%%%%%%%%%%%%%%%%%%%%%%%%%%%%%%%%%%%%%%%%%%%%%%%%%%%%%%%%%%%%%%%%%%%%%%%%
%%%%%%%%%%%%%%%%%%%%%%%%%%%%%%%%%%%%%%%%%%%%%%%%%%%%%%%%%%%%%%%%%%%%%%%%%%%%%%%%%%%%%%%%%%%%%%%%%%%%%%%%%%%%%%%%%%%%%%
 
 
 
%%%%%%%%%%%%% ExoCad 7 paramètres : Compétences, qrcode , calculatrice, python, scratch, tableur, annales
\newenvironment{ExoCad}[7]{% code avant
\tcbset{top=-0.2cm }
\stepcounter{exo}

\begin{tcolorbox}[right=-5mm, enhanced, lifted shadow={0mm}{0mm}{0mm}{0mm}%
{black!60!white}, attach boxed title to top right={xshift=-0.3cm, yshift*=-2mm}, coltitle=sacado_violet!85!black, colback=white!100!white, boxed title style={colback=white}, colframe=sacado_violet!100!black,title= {\footnotesize  #1}  ]
 
\hspace{-1.3cm} 
\begin{minipage}[t]{0.7cm}

 \begin{tikzpicture}
 	\node[fill=sacado_violet,minimum width=0.7cm]{\textcolor{white}{\bf {\Large \theexo}}};
 \end{tikzpicture}


%%%%%%%%%%%%%%%%%%%%%%%% Condition pour la calculatrice
 \ifthenelse{\equal{#3}{1}}{
 \begin{tikzpicture}
 	\node[minimum width=0.7cm]{\includegraphics[scale=0.5]{MISC/calculator.png} };
 \end{tikzpicture} 
 }{
 \ifthenelse{\equal{#3}{2}}{
 \begin{tikzpicture}
 	\node[minimum width=0.7cm]{\includegraphics[scale=0.5]{MISC/no_calculator.png} };
 \end{tikzpicture} 
 }{}
 }

\end{minipage}
\hfill
\begin{minipage}[t]{17.3cm}
} 
{ 
\end{minipage}%code  après
\hfill
\begin{minipage}[t]{1cm}

\begin{center}
\colorbox{sacado_violet}{\includegraphics[height=1cm]{qrcodes/qrDummy.png}}
\colorbox{white}{ {\footnotesize /b/ABCD} }
\end{center}

\end{minipage}

\end{tcolorbox}
\par 
}

 
%%%%%%%%%%%%% ExoCu 7 paramètres : Compétences, qrcode , calculatrice, python, scratch, tableur, annales
\newenvironment{ExoCu}[7]{% code avant
\tcbset{top=-0.2cm }
\stepcounter{exo}

\begin{tcolorbox}[right=-5mm, enhanced, lifted shadow={0mm}{0mm}{0mm}{0mm}%
{black!60!white}, attach boxed title to top right={xshift=-0.3cm, yshift*=-2mm}, coltitle=sacado_green!85!black, colback=white!100!white, boxed title style={colback=white}, colframe=sacado_green!100!black,title= {\footnotesize  #1}  ]
 
\hspace{-1.3cm} 
\begin{minipage}[t]{0.7cm}

 \begin{tikzpicture}
 	\node[fill=sacado_green,minimum width=0.7cm]{\textcolor{white}{\bf {\Large \theexo}}};
 \end{tikzpicture}


%%%%%%%%%%%%%%%%%%%%%%%% Condition pour la calculatrice
 \ifthenelse{\equal{#3}{1}}{
 \begin{tikzpicture}
 	\node[minimum width=0.7cm]{\includegraphics[scale=0.5]{MISC/calculator.png} };
 \end{tikzpicture} 
 }{
 \ifthenelse{\equal{#3}{2}}{
 \begin{tikzpicture}
 	\node[minimum width=0.7cm]{\includegraphics[scale=0.5]{MISC/no_calculator.png} };
 \end{tikzpicture} 
 }{}
 }

\end{minipage}
\hfill
\begin{minipage}[t]{17.3cm}
} 
{ 
\end{minipage}%code  après
\hfill
\begin{minipage}[t]{1cm}

 
\begin{center}
\colorbox{sacado_green}{\includegraphics[height=1cm]{qrcodes/qrDummy.png}}
\colorbox{white}{ {\footnotesize /b/ABCD}  }
\end{center}
 

\end{minipage}

\end{tcolorbox}
\par
}  


 %%%%%%%%%%%%% ExoCd 7 paramètres : Compétences, qrcode , calculatrice, python, scratch, tableur, annales
\newenvironment{ExoCd}[7]{% code avant
\tcbset{top=-0.2cm }
\stepcounter{exo}

\begin{tcolorbox}[right=-5mm, enhanced, lifted shadow={0mm}{0mm}{0mm}{0mm}%
{black!60!white}, attach boxed title to top right={xshift=-0.3cm, yshift*=-2mm}, coltitle=sacado_blue!85!black, colback=white!100!white, boxed title style={colback=white}, colframe=sacado_blue!100!black,title= {\footnotesize  #1}  ]
 
\hspace{-1.3cm} 
\begin{minipage}[t]{0.7cm}

 \begin{tikzpicture}
 	\node[fill=sacado_blue,minimum width=0.7cm]{\textcolor{white}{\bf {\Large \theexo}}};
 \end{tikzpicture}


%%%%%%%%%%%%%%%%%%%%%%%% Condition pour la calculatrice
 \ifthenelse{\equal{#3}{1}}{
 \begin{tikzpicture}
 	\node[minimum width=0.7cm]{\includegraphics[scale=0.5]{MISC/calculator.png} };
 \end{tikzpicture} 
 }{
 \ifthenelse{\equal{#3}{2}}{
 \begin{tikzpicture}
 	\node[minimum width=0.7cm]{\includegraphics[scale=0.5]{MISC/no_calculator.png} };
 \end{tikzpicture} 
 }{}
 }

\end{minipage}
\hfill
\begin{minipage}[t]{17.3cm}
} 
{ 
\end{minipage}%code  après
\hfill
\begin{minipage}[t]{1cm}

\begin{center}
\colorbox{sacado_blue}{\includegraphics[height=1cm]{qrcodes/qrDummy.png}}
\colorbox{white}{ {\footnotesize /b/ABCD} }
\end{center}

\end{minipage}

\end{tcolorbox}
\par
}  


%%%%%%%%%%%%% ExoCt 7 paramètres : Compétences, qrcode , calculatrice, python, scratch, tableur, annales
\newenvironment{ExoCt}[7]{% code avant
\tcbset{top=-0.2cm }
\stepcounter{exo}

\begin{tcolorbox}[right=-5mm, enhanced, lifted shadow={0mm}{0mm}{0mm}{0mm}%
{black!60!white}, attach boxed title to top right={xshift=-0.3cm, yshift*=-2mm}, coltitle=sacado_red!85!black, colback=white!100!white, boxed title style={colback=white}, colframe=sacado_red!100!black,title= {\footnotesize  #1}  ]
 
\hspace{-1.3cm} 
\begin{minipage}[t]{0.7cm}

 \begin{tikzpicture}
 	\node[fill=sacado_red,minimum width=0.7cm]{\textcolor{white}{\bf {\Large \theexo}}};
 \end{tikzpicture}


%%%%%%%%%%%%%%%%%%%%%%%% Condition pour la calculatrice
 \ifthenelse{\equal{#3}{1}}{
 \begin{tikzpicture}
 	\node[minimum width=0.7cm]{\includegraphics[scale=0.5]{MISC/calculator.png} };
 \end{tikzpicture} 
 }{
 \ifthenelse{\equal{#3}{2}}{
 \begin{tikzpicture}
 	\node[minimum width=0.7cm]{\includegraphics[scale=0.5]{MISC/no_calculator.png} };
 \end{tikzpicture} 
 }{}
 }

\end{minipage}
\hfill
\begin{minipage}[t]{17.3cm}
} 
{ 
\end{minipage}%code  après
\hfill
\begin{minipage}[t]{1cm}

\begin{center}
\colorbox{sacado_red}{\includegraphics[height=1cm]{qrcodes/qrDummy.png}}
\colorbox{white}{ {\footnotesize /b/ABCD} }
\end{center}

\end{minipage}

\end{tcolorbox}
\par
}  

 
 
 

%%%%%%%%%%%%% ExoCu 7 paramètres : compétences , qrcode , calculatrice, python, scratch, tableur, annales
\newenvironment{ExoAuto}[7]{% code avant
\tcbset{top=-0.2cm }
\stepcounter{exo}

\begin{tcolorbox}[right=-5mm, enhanced, lifted shadow={0mm}{0mm}{0mm}{0mm}%
{black!60!white}, attach boxed title to top right={xshift=-0.3cm, yshift*=-2mm}, coltitle=sacado_orange!85!black, colback=white!100!white, boxed title style={colback=white}, colframe=sacado_orange!100!black,title= {\footnotesize  #1}  ]
 
\hspace{-1.3cm} 
\begin{minipage}[t]{0.7cm}

 \begin{tikzpicture}
 	\node[fill=sacado_orange,minimum width=0.7cm]{\textcolor{white}{\bf {\Large \theexo}}};
 \end{tikzpicture}


%%%%%%%%%%%%%%%%%%%%%%%% Condition pour la calculatrice
 \ifthenelse{\equal{#3}{1}}{
 \begin{tikzpicture}
 	\node[minimum width=0.7cm]{\includegraphics[scale=0.5]{MISC/calculator.png} };
 \end{tikzpicture} 
 }{
 \ifthenelse{\equal{#3}{2}}{
 \begin{tikzpicture}
 	\node[minimum width=0.7cm]{\includegraphics[scale=0.5]{MISC/no_calculator.png} };
 \end{tikzpicture} 
 }{}
 }

\end{minipage}
\hfill
\begin{minipage}[t]{17.3cm}
} 
{ 
\end{minipage}%code  après
\hfill
\begin{minipage}[t]{1cm}

\begin{center}
\colorbox{sacado_orange}{\includegraphics[height=1cm]{qrcodes/qrDummy.png}}
\colorbox{white}{  {\footnotesize /b/ABCD}  }
\end{center}

\end{minipage}

\end{tcolorbox}
\par
}  

 

%%%%%%%%%%%%% ExoDec 7 paramètres : compétences , qrcode , calculatrice, python, scratch, tableur, annales
 
\newenvironment{ExoDec}[6]{% code avant
\tcbset{top=-0.2cm }
\stepcounter{exo}

\begin{tcolorbox}[right=-5mm, enhanced, lifted shadow={0mm}{0mm}{0mm}{0mm}%
{black!60!white}, attach boxed title to top right={xshift=-0.3cm, yshift*=-2mm}, coltitle=sacado_violet!85!black, colback=white!100!white, boxed title style={colback=white}, colframe=sacado_violet!100!black,title= {\footnotesize  #1}  ]
 
\hspace{-1.3cm} 
\begin{minipage}[t]{1cm}

 \begin{tikzpicture}
 	\node[fill=sacado_violet,minimum width=0.7cm]{\textcolor{white}{\bf {\Large \theexo}}};
 \end{tikzpicture}


%%%%%%%%%%%%%%%%%%%%%%%% Condition pour la calculatrice
 \ifthenelse{\equal{#3}{1}}{
 \begin{tikzpicture}
 	\node[minimum width=0.7cm]{\includegraphics[scale=0.5]{MISC/calculator.png} };
 \end{tikzpicture} 
 }{
 \ifthenelse{\equal{#3}{2}}{
 \begin{tikzpicture}
 	\node[minimum width=0.7cm]{\includegraphics[scale=0.5]{MISC/no_calculator.png} };
 \end{tikzpicture} 
 }{}
 }

\end{minipage}
\begin{minipage}[t]{17.3cm}
} 
{ 
\end{minipage}
\end{tcolorbox}
\par
}  
%%%%%%%%%%%%%%%%%%%%%%%%%%%%%%%%%%%%%%%%%%%%%%%%%%%%%%%%%%%%%%%%%%%%%%%%%%%%%%%%%%%%%%%%%%%%%%%%%%%%%%%%%%%%%%%%%%%%%%
%%%%%%%%%%%%%%%%%%%%%%%%%%%%%%%%%%%%%%%%%%%%%%%%%%%%%%%%%%%%%%%%%%%%%%%%%%%%%%%%%%%%%%%%%%%%%%%%%%%%%%%%%%%%%%%%%%%%%%
%%%%%%%%%%%%%%%%  Exercices sans qrcode                                %%%%%%%%%%%%%%%%%%%%%%%%%%%%%%%%%%%%%%%%%%%%%%%
%%%%%%%%%%%%%%%%%%%%%%%%%%%%%%%%%%%%%%%%%%%%%%%%%%%%%%%%%%%%%%%%%%%%%%%%%%%%%%%%%%%%%%%%%%%%%%%%%%%%%%%%%%%%%%%%%%%%%%
%%%%%%%%%%%%%%%%%%%%%%%%%%%%%%%%%%%%%%%%%%%%%%%%%%%%%%%%%%%%%%%%%%%%%%%%%%%%%%%%%%%%%%%%%%%%%%%%%%%%%%%%%%%%%%%%%%%%%%
 
 
 
%%%%%%%%%%%%% ExoCad 7 paramètres : Compétences , calculatrice, python, scratch, tableur, annales
\newenvironment{ExoCadN}[6]{% code avant
\tcbset{top=-0.2cm }
\stepcounter{exo}

\begin{tcolorbox}[right=-5mm, enhanced, lifted shadow={0mm}{0mm}{0mm}{0mm}%
{black!60!white}, attach boxed title to top right={xshift=-0.3cm, yshift*=-2mm}, coltitle=sacado_violet!85!black, colback=white!100!white, boxed title style={colback=white}, colframe=sacado_violet!100!black,title= {\footnotesize  #1}  ]
 
\hspace{-1.3cm} 
\begin{minipage}[t]{1cm}

 \begin{tikzpicture}
 	\node[fill=sacado_violet,minimum width=0.7cm]{\textcolor{white}{\bf {\Large \theexo}}};
 \end{tikzpicture}


%%%%%%%%%%%%%%%%%%%%%%%% Condition pour la calculatrice
 \ifthenelse{\equal{#3}{1}}{
 \begin{tikzpicture}
 	\node[minimum width=0.7cm]{\includegraphics[scale=0.5]{MISC/calculator.png} };
 \end{tikzpicture} 
 }{
 \ifthenelse{\equal{#3}{2}}{
 \begin{tikzpicture}
 	\node[minimum width=0.7cm]{\includegraphics[scale=0.5]{MISC/no_calculator.png} };
 \end{tikzpicture} 
 }{}
 }

\end{minipage}
\begin{minipage}[t]{17.3cm}
} 
{ 
\end{minipage}%code  après
\end{tcolorbox}
\par 
}

 
%%%%%%%%%%%%% ExoCu 7 paramètres : Compétences , calculatrice, python, scratch, tableur, annales
\newenvironment{ExoCuN}[6]{% code avant
\tcbset{top=-0.2cm }
\stepcounter{exo}

\begin{tcolorbox}[right=-5mm, enhanced, lifted shadow={0mm}{0mm}{0mm}{0mm}%
{black!60!white}, attach boxed title to top right={xshift=-0.3cm, yshift*=-2mm}, coltitle=sacado_green!85!black, colback=white!100!white, boxed title style={colback=white}, colframe=sacado_green!100!black,title= {\footnotesize  #1}  ]
 
\hspace{-1.3cm} 
\begin{minipage}[t]{1cm}

 \begin{tikzpicture}
 	\node[fill=sacado_green,minimum width=0.7cm]{\textcolor{white}{\bf {\Large \theexo}}};
 \end{tikzpicture}


%%%%%%%%%%%%%%%%%%%%%%%% Condition pour la calculatrice
 \ifthenelse{\equal{#2}{1}}{
 \begin{tikzpicture}
 	\node[minimum width=0.7cm]{\includegraphics[scale=0.5]{MISC/calculator.png} };
 \end{tikzpicture} 
 }{
 \ifthenelse{\equal{#2}{2}}{
 \begin{tikzpicture}
 	\node[minimum width=0.7cm]{\includegraphics[scale=0.5]{MISC/no_calculator.png} };
 \end{tikzpicture} 
 }{}
 }

\end{minipage}
\begin{minipage}[t]{17.3cm}
} 
{ 
\end{minipage}
\end{tcolorbox}
\par
}  


 %%%%%%%%%%%%% ExoCd 6 paramètres : Compétences, calculatrice, python, scratch, tableur, annales
\newenvironment{ExoCdN}[6]{% code avant
\tcbset{top=-0.2cm }
\stepcounter{exo}

\begin{tcolorbox}[right=-5mm, enhanced, lifted shadow={0mm}{0mm}{0mm}{0mm}%
{black!60!white}, attach boxed title to top right={xshift=-0.3cm, yshift*=-2mm}, coltitle=sacado_blue!85!black, colback=white!100!white, boxed title style={colback=white}, colframe=sacado_blue!100!black,title= {\footnotesize  #1}  ]
 
\hspace{-1.3cm} 
\begin{minipage}[t]{1cm}

 \begin{tikzpicture}
 	\node[fill=sacado_blue,minimum width=0.7cm]{\textcolor{white}{\bf {\Large \theexo}}};
 \end{tikzpicture}


%%%%%%%%%%%%%%%%%%%%%%%% Condition pour la calculatrice
 \ifthenelse{\equal{#2}{1}}{
 \begin{tikzpicture}
 	\node[minimum width=0.7cm]{\includegraphics[scale=0.5]{MISC/calculator.png} };
 \end{tikzpicture} 
 }{
 \ifthenelse{\equal{#2}{2}}{
 \begin{tikzpicture}
 	\node[minimum width=0.7cm]{\includegraphics[scale=0.5]{MISC/no_calculator.png} };
 \end{tikzpicture} 
 }{}
 }

\end{minipage}
\begin{minipage}[t]{17.3cm}
} 
{ 
\end{minipage}%code  après
\end{tcolorbox}
\par
}  


%%%%%%%%%%%%% ExoCt 6 paramètres : Compétences,   calculatrice, python, scratch, tableur, annales
\newenvironment{ExoCtN}[6]{% code avant
\tcbset{top=-0.2cm }
\stepcounter{exo}

\begin{tcolorbox}[right=-5mm, enhanced, lifted shadow={0mm}{0mm}{0mm}{0mm}%
{black!60!white}, attach boxed title to top right={xshift=-0.3cm, yshift*=-2mm}, coltitle=sacado_red!85!black, colback=white!100!white, boxed title style={colback=white}, colframe=sacado_red!100!black,title= {\footnotesize  #1}  ]
 
\hspace{-1.3cm} 
\begin{minipage}[t]{1cm}

 \begin{tikzpicture}
 	\node[fill=sacado_red,minimum width=0.7cm]{\textcolor{white}{\bf {\Large \theexo}}};
 \end{tikzpicture}


%%%%%%%%%%%%%%%%%%%%%%%% Condition pour la calculatrice
 \ifthenelse{\equal{#2}{1}}{
 \begin{tikzpicture}
 	\node[minimum width=0.7cm]{\includegraphics[scale=0.5]{MISC/calculator.png} };
 \end{tikzpicture} 
 }{
 \ifthenelse{\equal{#2}{2}}{
 \begin{tikzpicture}
 	\node[minimum width=0.7cm]{\includegraphics[scale=0.5]{MISC/no_calculator.png} };
 \end{tikzpicture} 
 }{}
 }

\end{minipage}
\begin{minipage}[t]{17.3cm}
} 
{ 
\end{minipage}%code  après
\end{tcolorbox}
\par
}  

 
 
 

%%%%%%%%%%%%% ExoCu 7 paramètres : compétences , qrcode , calculatrice, python, scratch, tableur, annales
\newenvironment{ExoAutoN}[6]{% code avant
\tcbset{top=-0.2cm }
\stepcounter{exo}

\begin{tcolorbox}[right=5mm, enhanced, lifted shadow={0mm}{0mm}{0mm}{0mm}%
{black!60!white}, attach boxed title to top right={xshift=-0.3cm, yshift*=-2mm}, coltitle=sacado_orange!85!black, colback=white!100!white, boxed title style={colback=white}, colframe=sacado_orange!100!black,title= {\footnotesize  #1}  ]
 
\hspace{-1.4cm} 
\begin{minipage}[t]{0.7cm}

 \begin{tikzpicture}
 	\node[fill=sacado_orange,minimum width=0.7cm]{\textcolor{white}{\bf {\Large \theexo}}};
 \end{tikzpicture}


%%%%%%%%%%%%%%%%%%%%%%%% Condition pour la calculatrice
 \ifthenelse{\equal{#2}{1}}{
 \begin{tikzpicture}
 	\node[minimum width=0.7cm]{\includegraphics[scale=0.5]{MISC/calculator.png} };
 \end{tikzpicture} 
 }{
 \ifthenelse{\equal{#2}{2}}{
 \begin{tikzpicture}
 	\node[minimum width=0.7cm]{\includegraphics[scale=0.5]{MISC/no_calculator.png} };
 \end{tikzpicture} 
 }{}
 }

\end{minipage}
\hfill
\begin{minipage}[t]{17.3cm}
} 
{ 
\end{minipage}%code  après
\end{tcolorbox}
\par
}  
%%%%%%%%%%%%%%%%%%%%%%%%%%%%%%%%%%%%%%%%%%%%%%%%%%%%%%%%%%%%%%%%%%%%%%%%%%%%%%%%%%%%%%%%%%%%%%%%%%%%%%%%%%%%%%%%%%%%%%
%%%%%%%%%%%%%%%%%%%%%%%%%%%%%%%%%%%%%%%%%%%%%%%%%%%%%%%%%%%%%%%%%%%%%%%%%%%%%%%%%%%%%%%%%%%%%%%%%%%%%%%%%%%%%%%%%%%%%%
%%%%%%%%%%%%%%%%  Exercices   sans contours                            %%%%%%%%%%%%%%%%%%%%%%%%%%%%%%%%%%%%%%%%%%%%%%%
%%%%%%%%%%%%%%%%%%%%%%%%%%%%%%%%%%%%%%%%%%%%%%%%%%%%%%%%%%%%%%%%%%%%%%%%%%%%%%%%%%%%%%%%%%%%%%%%%%%%%%%%%%%%%%%%%%%%%%
%%%%%%%%%%%%%%%%%%%%%%%%%%%%%%%%%%%%%%%%%%%%%%%%%%%%%%%%%%%%%%%%%%%%%%%%%%%%%%%%%%%%%%%%%%%%%%%%%%%%%%%%%%%%%%%%%%%%%%


\newcommand{\Sf}[1]{ \vspace{0.1cm}
{\color{fond}{\Large \textbf{#1}}  } 
} 

\newcommand{\Sfe}[1]{ \vspace{0.1cm}
{\color{sacado_blue}{\Large \textbf{#1}}  } 
} 
% fin de la procédure



%%%%%%%%%%%%% Pointillés ou ligne
\newcommand{\point}[1]{\vspace{0.1cm}\multido{}{#1}{ \dotfill \medskip \endgraf}}
\newcommand{\ligne}[1]{\vspace{0.1cm}\multido{}{#1}{ {\color{cqcqcq}\hrulefill} \medskip \endgraf}}
%----------------------------------------
%
%   Macros et opérateurs
%
%----------------------------------------

\newcommand{\second}{2\up{d}\xspace}
\newcommand{\seconde}{2\up{de}\xspace}
\newcommand{\R}{\mathbb R}
\newcommand{\Rp}{\R_+}
\newcommand{\Rpe}{\R_+^*}
\newcommand{\Rm}{\R_-}
\newcommand{\Rme}{\R_-^*}
\newcommand{\N}{\mathbb N}
\newcommand{\D}{\mathbb D}
\newcommand{\Q}{\mathbb Q}
\newcommand{\Z}{\mathbb Z}
\newcommand{\C}{\mathbb C}
\newcommand{\grs}{\mathfrak S}
\newcommand{\IN}[1]{\llbracket 1,#1\rrbracket}
\newcommand{\card}{\text{Card}\,}
\usepackage{mathrsfs}
\newcommand{\parties}{\mathscr P}
\renewcommand{\epsilon}{\varepsilon}
\newcommand{\rmd}{\text{d}}
\newcommand{\diff}{\mathrm D}
\newcommand{\Id}{{\rm Id}}
\newcommand{\e}{{\rm e}}
\newcommand{\I}{{\rm i}}
\newcommand{\J}{{\rm j}}
\newcommand{\ro}{\circ}
\newcommand{\exu}{\exists\,!\,}
\newcommand{\telq}{\,\, \mid \,\,}
\newcommand{\para}{\raisebox{0.1em}{\text{\footnotesize /\hspace{-0.1em}/}}}   
\newcommand{\vect}[1]{\overrightarrow{#1}}
\newcommand{\scal}[2]{\left(\, #1 \mid #2 \, \right)}
\newcommand{\ortho}[1]{{#1}^\perp}
\newcommand{\veci}{\vec{\text{\it \i}}}
\newcommand{\vecj}{\vec{\text{\it \j}}}
\newcommand{\rep}{$(O;\veci,\vecj,\vec{k})$\xspace}
\newcommand{\Oijk}{$(O, \veci,\vecj,\vec{k})$\xspace}
\newcommand{\rond}{repère orthonormal direct}
\newcommand{\bond}{base orthonormale directe}
\newcommand{\eq}{\Longleftrightarrow}
\newcommand{\implique}{\Longrightarrow}
\newcommand{\noneq}{\ \ \ /\hspace{-1.45em}\eq}
\newcommand{\tend}{\longrightarrow}
\newcommand{\egx}[2]{\underset{#1 \tend #2}=}
\newcommand{\asso}{\longmapsto}
\newcommand{\vers}{\longrightarrow}
\newcommand{\eqn}{~\underset{n \rightarrow \infty}{\sim}~}
\newcommand{\eqx}[2]{~\underset{#1 \rightarrow #2}{\sim}~}

\newcommand{\egn}{~\underset{n \rightarrow \infty}{=}~}
\renewcommand{\descriptionlabel}{\hspace{\labelsep}$\bullet$}
\renewcommand{\bar}{\overline}
\DeclareMathOperator{\ash}{{Argsh}}
\DeclareMathOperator{\cotan}{{cotan}}
\DeclareMathOperator{\ach}{{Argch}}
\DeclareMathOperator{\ath}{{Argth}}
\DeclareMathOperator{\sh}{{sh}}
\DeclareMathOperator{\ch}{{ch}}
\DeclareMathOperator{\Mat}{{Mat}}
\DeclareMathOperator{\Vect}{{Vect}}
\DeclareMathOperator{\trace}{{tr}}
\newcommand{\tr}{{}^{\mathrm t}}
\newcommand{\divi}{~\big|~}
\newcommand{\ndivi}{~\not{\big|}~}
\newcommand{\et}{\wedge}
\newcommand{\ou}{\vee}
\renewcommand{\det}{\operatorname{\text{dét}}}
\DeclareMathOperator{\grad}{{grad}}
\renewcommand{\arcsin}{{\mathop{\mathrm{Arcsin}}}}
\renewcommand{\arccos}{{\mathop{\mathrm{Arccos}}}}
\renewcommand{\arctan}{{\mathop{\mathrm{Arctan}}}}
\renewcommand{\tanh}{{\mathop{\mathrm{th}}}}
\newcommand{\pgcd}{\mathop{\mathrm{pgcd}}}
\newcommand{\ppcm}{\mathop{\mathrm{ppcm}}}
\newcommand{\fonc}[4]{\left\{\begin{tabular}{ccc}$#1$ & $\vers$ & $#2$ \\
$#3$ & $\asso$ & $#4$ \end{tabular}\right.}
\renewcommand{\geq}{\geqslant}
\renewcommand{\leq}{\leqslant}
\renewcommand{\Re}{\text{\rm Re}}
\renewcommand{\Im}{\text{\rm Im}}
\renewcommand{\ker}{\mathop{\mathrm{Ker}}}
\newcommand{\Lin}{\mathcal L}
\newcommand{\GO}{\mathcal O}
\newcommand{\GSO}{\mathcal{SO}}
\newcommand{\GL}{\mathcal{GL}}
\renewcommand{\emptyset}{\varnothing}
%\newcommand{\arc}[1]{\overset{\frown}{#1}}
\newcommand{\rg}{\mathop{\mathrm{rg}}}
\newcommand{\ds}{\displaystyle}
\newcommand{\co}[3]{\begin{pmatrix}#1 \\ #2 \\ #3\end{pmatrix}}
\newcommand{\demi}{\frac 1 2}
\newcommand{\limi}[2]{\underset{#1 \rightarrow #2}\lim}

\title{Mathématiques 2nde  : le livre sacado}
\author{L'équipe SACADO}

\begin{document}

%\maketitle

\chapter{Intervalle de $\mathbb{R}$}{4}
{URL du parcours}
{
 \begin{CpsCol}
	\textbf{Les savoir-faire du parcours}
 	\begin{itemize}
 		\item Représenter les ensembles de nombres sur la droite graduée
 		\item Identifier les intervalles de $\R$
 		\item Lire et écrire des propositions contenant les connecteurs « $et$ », « $ou$ » ;
 	\end{itemize}
 \end{CpsCol}

\begin{His}
\textbf{Georg Cantor} est un mathématicien allemand, né le 3 mars 1845 à Saint-Pétersbourg et mort le $6$ janvier $1918$ à Halle.\\ 
Il est connu pour être le créateur de la théorie des ensembles.\\ 
Il prouva que les nombres réels sont "plus nombreux" que les entiers naturels. En fait, le théorème de Cantor implique\\ 
l'existence d'une "infinité d'infinis". Il définit les nombres cardinaux, les nombres ordinaux et leur arithmétique. \\ 
Une partie du travail de Cantor consista à dire que l'infini a plusieurs taille. Et qu'il y a des infini plus grand que d'autres. \\ 
Cantor rejoint ainsi le monde de la philosophie, comme \textbf{Pascal} 3 siècle plus tôt. \end{His}

%\begin{ExoDec}{Chercher.}{1234}{1}{0}{0}{0}
%L'\textbf{E}urope est composée de 27 pays dont la \textbf{F}rance, l'\textbf{I}talie et la \textbf{G}rèce. Les capitales respectives sont \textbf{P}aris, \textbf{R}ome et \textbf{A}thènes. 
%\begin{enumerate}
%\item Un habitant de Rome est-il en Italie ? 
%\item Un habitant de Grèce est-il une habitant de Rome ? 
%\item Un habitant d'Italie est-il en Europe ?
%\item Si je ne suis pas en Europe, puis-je être en France ?
%\item Construire un diagramme de Venn qui illustre cette situation.
%\end{enumerate}
%
%{\small \textit{Pour faciliter l'écriture, on pourra utilisera l'initiale des noms respectifs : $E$, $F$, $I$, $G$, $P$, $R$ et $A$.}}
%\end{ExoDec}

\begin{ExoDec}{Chercher.}{1234}{2}{0}{0}{0}
 
L'INSEE estime qu'un couple avec deux enfants appartient à la classe moyenne quand les revenus du foyer sont situés dans l'intervalle $[3253;5609]$. Monsieur Twicks gagne $2731$ euros et madame Twicks gagne $2952$ euros et ont deux enfants Mary et Paul.

 La famille appartient-elle à la classe moyenne ?

%Le couple gagne donc $s_{global}=2731+2952 = 5683$ euros. On compare cette valeur à l'intervalle de la classe moyenne $[3253;5609]$.
%
%$5683>5609 $ donc $5683 \not \in [3253;5609]$. Le couple n'appartient donc pas à la classe moyenne.
 
 \end{ExoDec}
}

%%%%%%%%%%%%%%%%%%%%%%%%%%%%%%%%%%%%%%%%%%%%%%%%%%%%%%%%%%%%%%%%%%%
%%%% Page de cours 1
%%%%%%%%%%%%%%%%%%%%%%%%%%%%%%%%%%%%%%%%%%%%%%%%%%%%%%%%%%%%%%%%%%%

\begin{pageCours} % Début page de cours 1

\newgeometry{left=2cm,right=.8cm,top=1.5cm} %Ne pas toucher cette ligne

\section{Intervalles de $\R$ }

\begin{DefT}{Intervalle fermé. Intervalle ouvert}\index{Intervalle}
Un \textbf{intervalle fermé} de $\R$ est un sous-ensemble borné de $\R$, c'est à dire un ensemble de nombres compris entre deux valeurs réelles.
 
Un \textbf{intervalle ouvert} de $\R$ est un sous-ensemble de $\R$ dont les bornes ne sont pas incluses dans l'ensemble, c'est à dire un ensemble de nombres compris entre deux valeurs réelles non comprises.
\end{DefT}

\begin{Rep}
\begin{enumerate}
\item On a représenté sur la droite des nombres réels tous les nombres réels $x$ tels que $-1 \leq x \leq 3$.

\begin{center}
\definecolor{ffxfqq}{rgb}{1.,0.4980392156862745,0.}
\begin{tikzpicture}[line cap=round,line join=round,>=triangle 45,x=1.0cm,y=1.0cm]
\draw[->,color=black] (-4.390839866186475,0.) -- (7.64974334956303,0.);
\foreach \x in {-4.,-3.,-2.,-1.,1.,2.,3.,4.,5.,6.,7.}
\draw[shift={(\x,0)},color=black] (0pt,2pt) -- (0pt,-2pt) node[below] {\footnotesize $\x$};
\draw[color=black] (0pt,-10pt) node[right] {\footnotesize $0$};
\clip(-4.390839866186475,-0.5880295569511441) rectangle (7.64974334956303,0.5863275715079787);
\draw [line width=2.4pt,color=ffxfqq] (-1.,0.)-- (3.,0.);
\draw [color=ffxfqq](-1.16,0.35) node[anchor=north west] {\Large{[}};
\draw [color=ffxfqq](2.85,0.35) node[anchor=north west] {\Large{]}};
\end{tikzpicture}
 \end{center} 
 
Cet intervalle est noté $[-1;3]$ et on écrit alors $x \in [-1;3]$. Cet intervalle est dit \textbf{fermé}.
 \item On a représenté sur la droite des nombres réels tous les nombres réels $x$ tels que $-1 < x < 3$.

\begin{center}
\definecolor{ffxfqq}{rgb}{1.,0.4980392156862745,0.}
\begin{tikzpicture}[line cap=round,line join=round,>=triangle 45,x=1.0cm,y=1.0cm]
\draw[->,color=black] (-4.390839866186475,0.) -- (7.64974334956303,0.);
\foreach \x in {-4.,-3.,-2.,-1.,1.,2.,3.,4.,5.,6.,7.}
\draw[shift={(\x,0)},color=black] (0pt,2pt) -- (0pt,-2pt) node[below] {\footnotesize $\x$};
\draw[color=black] (0pt,-10pt) node[right] {\footnotesize $0$};
\clip(-4.390839866186475,-0.5295569511441) rectangle (7.64974334956303,0.563275715079787);
\draw [line width=2.4pt,color=ffxfqq] (-1.,0.)-- (3.,0.);
\draw [color=ffxfqq](-1.2,0.35) node[anchor=north west] {\Large{]}};
\draw [color=ffxfqq](2.85,0.35) node[anchor=north west] {\Large{[}};
\end{tikzpicture}
 \end{center}
Cet intervalle ouvert est noté $]-1;3[$ et on écrit alors $x \in ]-1;3[$. Cet intervalle est dit \textbf{ouvert}.

 \item On a représenté sur la droite des nombres réels tous les nombres réels $x$ tels que $x \geq -1$.


\begin{center}
\definecolor{ffxfqq}{rgb}{1.,0.4980392156862745,0.}
\begin{tikzpicture}[line cap=round,line join=round,>=triangle 45,x=1.0cm,y=1.0cm]
\draw[->,color=black] (-4.390839866186475,0.) -- (7.64974334956303,0.);
\foreach \x in {-4.,-3.,-2.,-1.,0,1.,2.,3.,4.,5.,6.,7.}
\draw[shift={(\x,0)},color=black] (0pt,2pt) -- (0pt,-2pt) node[below] {\footnotesize $\x$};
\clip(-4.390839866186475,-0.5880295569511441) rectangle (7.64974334956303,0.53275715079787);
\draw [line width=2.4pt,color=ffxfqq] (-1.,0.)-- (8.,0.);
\draw [color=ffxfqq](-1.2,0.35) node[anchor=north west] {\Large{[}};
\end{tikzpicture}
 \end{center} 
Cet ensemble est noté $[-1 ; +\infty[$ et on écrit alors $x \in [-1 ; +\infty[$. Cet intervalle est semi-ouvert à droite.


 \item On a représenté sur la droite des nombres réels tous les nombres réels $x$ tels que $x \geq -1$.


\begin{center}
\definecolor{ffxfqq}{rgb}{1.,0.4980392156862745,0.}
\begin{tikzpicture}[line cap=round,line join=round,>=triangle 45,x=1.0cm,y=1.0cm]
\draw[->,color=black] (-4.390839866186475,0.) -- (7.64974334956303,0.);
\foreach \x in {-4.,-3.,-2.,-1.,0,1.,2.,3.,4.,5.,6.,7.}
\draw[shift={(\x,0)},color=black] (0pt,2pt) -- (0pt,-2pt) node[below] {\footnotesize $\x$};
\clip(-4.390839866186475,-0.5880295569511441) rectangle (7.64974334956303,0.53275715079787);
\draw [line width=2.4pt,color=ffxfqq] (-5.,0.)-- (2.,0.);
\draw [color=ffxfqq](1.8,0.35) node[anchor=north west] {\Large{]}};
\end{tikzpicture}
 \end{center} 
Cet ensemble est noté $]-\infty;2]$ et on écrit alors $x \in ]-\infty;2]$. Cet intervalle est semi-ouvert à gauche.



\end{enumerate}
\end{Rep}

\begin{Rqs}
\begin{enumerate}
\item  $+ \infty$ se lit "plus l’infini". L'ensemble des nombres réels $\R$ est l'intervalle $]-\infty ; +\infty[ = \R$.
\item Un intervalle est une partie de $\R$ "sans trou", en "un seul morceau".
\item $+\infty$ et $-\infty$ ne sont pas des nombres. Ce ne sont que des notations (ce qui explique qu'ils soient toujours exclus).
\item Les intervalles correspondants aux quatre premières lignes du tableau sont dits bornés.
\item  Plus généralement, les différents types d'intervalles sont donnés dans le tableau ci-dessous (où $a$ et $b$ représentent deux réels, avec $a < b$).
\end{enumerate}
\end{Rqs}

\end{pageCours} % Fin page de cours 1

%%%%%%%%%%%%%%%%%%%%%%%%%%%%%%%%%%%%%%%%%%%%%%%%%%%%%%%%%%%%%%%%%%%
%%%% Application direct 1
%%%%%%%%%%%%%%%%%%%%%%%%%%%%%%%%%%%%%%%%%%%%%%%%%%%%%%%%%%%%%%%%%%%

\begin{pageAD}  % Début page d'exercice d'application direct 1
\restoregeometry %Ne pas toucher cette ligne

\Sf{Connaitre et représenter les intervalles}
 

  
\begin{ExoCad}{Représenter.}{1234}{0}{0}{0}{0}{0}

Soit $x$ un nombre. Écrire sous forme d'intervalle les inégalités suivantes :


\begin{enumerate}[leftmargin=*]
\begin{minipage}{0.49\linewidth}
	\item $-2 \leq x \leq 3$  \point{1}
	\item $-\pi \leq x <  \pi$  \point{1}
	\item Soient $a$ et $b$ deux réels, $a \leq x \leq  b$  \point{1}
\end{minipage}
\hfill
\begin{minipage}{0.49\linewidth}
	\item $ x \leq -2$  \point{1}
	\item $0 \geq x $  \point{1}
	\item Soient $a$ un réel, $x <  a$  \point{1}
\end{minipage}
\end{enumerate} 
\end{ExoCad}

  
\begin{ExoCad}{Représenter.}{1234}{0}{0}{0}{0}{0}

Soit $x$ un nombre. Écrire sous forme d'inégalités  les  intervalles suivants :

\begin{enumerate}[leftmargin=*]
\begin{minipage}{0.3\linewidth}
	\item $x \in [-3;2]$  \point{1}
	\item $x \in ]-\infty;3]$  \point{1}
\end{minipage}
\hfill
\begin{minipage}{0.3\linewidth}
	\item $x \in ]0;1]$  \point{1}
	\item $x \in ]1;+\infty[$  \point{1}
\end{minipage}
\hfill
\begin{minipage}{0.3\linewidth}
	\item $x \in ]-5;8[$  \point{1}
	\item $x \in [a;b]$  \point{1}
\end{minipage}
\end{enumerate} 
\end{ExoCad}



\begin{ExoCad}{Représenter.}{1234}{0}{0}{0}{0}{0}
\begin{enumerate}[leftmargin=*]
	\item Écrire l'ensemble des réels strictement supérieurs à $-2$ et inférieurs à $4$  \point{1}
	\item Écrire l'ensemble des réels supérieurs à $2$ et inférieurs à $6$  \point{1}
	\item Écrire l'ensemble des réels inférieurs à $-1$  \point{1}
\end{enumerate}
\end{ExoCad}



\begin{ExoCad}{Représenter.}{1234}{0}{0}{0}{0}{0}

Recopier et compléter le tableau.

\begin{tabular}{|c|c|c|}
\hline 
Intervalle & Inégalité & Représentation  sur la droite graduée  \\ 
\hline 
$x\in \left[ -6 ; \dfrac{2}{7}\right]$ & $-6  \leq x \leq  \dfrac{2}{7} $  &   \\ 
\hline 
 & $-3 \leq x <5$ &    \\ 
\hline 
$x\in \left[ 5 ; 8 \right[ $  &  &     \\ 
\hline 
 &  & 
 \definecolor{ffdxqq}{rgb}{1.,0.8431372549019608,0.}
\definecolor{ffxfqq}{rgb}{1.,0.4980392156862745,0.}
\begin{tikzpicture}[line cap=round,line join=round,>=triangle 45,x=1.0cm,y=1.0cm]
\draw[->,color=black] (-5.174092090680384,0.) -- (2.566282833730012,0.);
\foreach \x in {-5.,-4.,-3.,-2.,-1.,1.,2.}
\draw[shift={(\x,0)},color=black] (0pt,2pt) -- (0pt,-2pt) node[below] {\footnotesize $\x$};
\draw[color=black] (0pt,-10pt) node[right] {\footnotesize $0$};
\clip(-5.174092090680384,-0.4115875953650586) rectangle (2.566282833730012,0.4791698364123281);
\draw [line width=2.4pt,color=ffxfqq] (-4.,0.)-- (1.,0.);
\draw [color=ffdxqq](-4.2,0.35) node[anchor=north west] {\Large{]}};
\draw [color=ffdxqq](0.88 ,0.35) node[anchor=north west] {\Large{]}};
\end{tikzpicture} 
 \\ 
\hline 
\end{tabular} 
 
\end{ExoCad}



\begin{ExoCad}{Représenter.}{1234}{0}{0}{0}{0}{0}

Représenter sur une droite graduée les intervalles donnés. Soit $x$ un réel,
\begin{enumerate}[leftmargin=*]
	\item $-10< x < 3$  \point{1}
	\item $ x \in [0;\pi]$  \point{1}
	\item $ x \geq \dfrac{1}{3}$  \point{1}
	\item l'ensemble des réels strictement supérieurs à $-1$ et inférieurs à $3$  \point{1}
\end{enumerate}
\end{ExoCad}


 
 
 
\begin{ExoCad}{Raisonner. Communiquer.}{1234}{0}{0}{0}{0}{0}

Complète par  $\in$ ou $\notin$.

\begin{enumerate}
\begin{minipage}{0.32\linewidth}
\item $4 \ldots \ldots [4;5]$ \vspace{0.1cm}
\item $1,5 \ldots \ldots [1;3]$\vspace{0.1cm}
\item $-5,9 \ldots \ldots ]-\infty;-6]$
\end{minipage}
\hfill
\begin{minipage}{0.32\linewidth}
\item $3 \ldots \ldots [0;3[$\vspace{0.1cm}
\item $\dfrac{2}{3} \ldots \ldots [2;3]$\vspace{0.1cm}
\item $\dfrac{1}{2} \ldots \ldots ]-1;1[$
\end{minipage}
\hfill
\begin{minipage}{0.32\linewidth}
\item $\dfrac{5}{3} \ldots \ldots \left[ \dfrac{5}{6}; \dfrac{11}{6}  \right]$ 
\item $\sqrt{5} \ldots \ldots \left[ 1;3 \right]$\vspace{0.1cm}
\item $\sqrt{24} \ldots \ldots \left[0; 4\right]$ 
\end{minipage}

\end{enumerate}
 
\end{ExoCad}
 
\end{pageAD} % Fin page d'exercice d'application direct 1

%%%%%%%%%%%%%%%%%%%%%%%%%%%%%%%%%%%%%%%%%%%%%%%%%%%%%%%%%%%%%%%%%%%
%%%% Page de cours 2
%%%%%%%%%%%%%%%%%%%%%%%%%%%%%%%%%%%%%%%%%%%%%%%%%%%%%%%%%%%%%%%%%%%

\begin{pageCours} % Début page de cours 2

\newgeometry{left=2cm,right=.8cm,top=1.5cm} %Ne pas toucher cette ligne

\section{Opérations d'ensembles de nombres et logique mathématique}
 
\begin{minipage}[t]{0.69\linewidth}

\begin{DefT}{Inclusion}\index{Ensemble!Inclusion}

\begin{minipage}{0.58\linewidth}
Un ensemble $A$ \textbf{est inclus dans} un ensemble $B$ lorsque \textbf{tous} les éléments de $A$ sont contenus dans $B$. On note $A \subset B$. \point{3}
\end{minipage}
\hfill
\begin{minipage}{0.38\linewidth}

\definecolor{ffttww}{rgb}{1.,0.2,0.4}
\definecolor{xdxdff}{rgb}{0.49019607843137253,0.49019607843137253,1.}
\begin{tikzpicture}[line cap=round,line join=round,>=triangle 45,x=1.0cm,y=1.0cm]
\clip(1.32,0.48) rectangle (6.42,3.82);
\draw [rotate around={-1.123302714075422:(3.77,2.23)},line width=2.pt,color=xdxdff,fill=xdxdff,fill opacity=0.30000001192092896] (3.77,2.23) ellipse (2.1702715668569548cm and 1.5389212695611632cm);
\draw [rotate around={-61.97549946792974:(3.77,2.32)},line width=2.pt,color=ffttww,fill=ffttww,fill opacity=0.3499999940395355] (3.77,2.32) ellipse (1.1237882510756807cm and 0.8772685069325904cm);
\draw (4.76,3.48) node[anchor=north west] {$B$};
\draw (3.6,3.24) node[anchor=north west] {$A$};
\end{tikzpicture}
\end{minipage}
\end{DefT}
\end{minipage}
\begin{minipage}[t]{0.28\linewidth}
\begin{Rq}
\begin{description}[leftmargin=*]
\item Un ensemble \textbf{est inclus dans} un ensemble. $A$ et $B$ deux ensembles : $A \subset B$.
\item Un élément \textbf{appartient à} un ensemble. $x$ un élément et $A$ un ensemble : $x \in B$.
\end{description}
\end{Rq}
\end{minipage}


\begin{minipage}[t]{0.55\linewidth}
\begin{LogT}{Le contre exemple}

$\Z$ est-il inclus dans $\N$ ? Autrement dit, tous les éléments de $\Z$ sont -ils contenus dans $\N$ ?\\ On cherche  un seul élément de $\Z$ qui n'appartient pas à $\N$ : $-2 \in \Z$ mais $-2 \not\in \N$ donc $\Z \not\subset \N$. 

$-2$ est appelé un \textbf{contre exemple}\index{Contre exemple!Logique}. 
 

\end{LogT}
\end{minipage}
\begin{minipage}[t]{0.43\linewidth}
 
\begin{LogT}{La contra-posée} 
Soit $A$ et $B$ deux ensembles tels que $A \subset B$. 

Ces deux propositions disent la même idée :

Si $x \in A$ alors $x \in B \Longleftrightarrow$ Si $x \not\in B$ alors $x \not\in A$.

$\sqrt{3} \not\in \Q \Rightarrow \sqrt{3} \not\in \Z$
\end{LogT}

\end{minipage}

\begin{DefT}{Complémentaire}\index{Ensemble!Complémentaire}

\begin{minipage}{0.68\linewidth}
Soit $\Omega$ un ensemble contenant un ensemble $A$. On appelle \textbf{complémentaire} de $A$ dans $\Omega$, \textbf{tous} les éléments de $\Omega$ qui n'appartiennent pas à $A$. \point{2}
Si $\Omega=\lbrace{0;1;2;3;4;5;6\rbrace}$ et $A=\lbrace{3;4;6\rbrace}$ alors $\Omega\setminus A=\lbrace{0;1;2;5\rbrace}$
\end{minipage}
\begin{minipage}{0.28\linewidth}

\definecolor{ffffff}{rgb}{1.,1.,1.}
\definecolor{xfqqff}{rgb}{0.4980392156862745,0.,1.}
\definecolor{ffttww}{rgb}{1.,0.2,0.4}
\definecolor{xdxdff}{rgb}{0.49019607843137253,0.49019607843137253,1.}
\definecolor{ududff}{rgb}{0.30196078431372547,0.30196078431372547,1.}
\begin{tikzpicture}[line cap=round,line join=round,>=triangle 45,x=1.0cm,y=1.0cm]
\clip(1.32,0.5) rectangle (5.9,3.96);
\draw [rotate around={0.916654256385289:(3.49,2.28)},line width=2.pt,color=xdxdff,fill=xdxdff,fill opacity=0.5] (3.49,2.28) ellipse (2.0420114277198222cm and 1.6145930357022946cm);
\draw [rotate around={-61.97549946792974:(2.99,2.34)},line width=2.pt,color=ffttww,fill=ffttww,fill opacity=1.0] (2.99,2.34) ellipse (1.1237882510756807cm and 0.8772685069325904cm);
\draw [color=xfqqff](5.36,3.78) node[anchor=north west] {$B$};
\draw [color=ffffff](2.32,3.3) node[anchor=north west] {$A$};
\draw (3.8,3.48) node[anchor=north west] {$B\setminus A$};
\end{tikzpicture}

\end{minipage}
\end{DefT}
 


 
\begin{minipage}[t]{0.68\linewidth}
\begin{DefT}{Intersection}\index{Ensemble!Intersection}

\begin{minipage}{0.58\linewidth}
L'\textbf{intersection} de deux ensembles $A$ et $B$ est l'ensemble $A \cap B$ qui contient \textbf{tous} les éléments communs aux deux ensembles. 

On lit $A$ "inter" $B$. 

L'ensemble $A \cap B$ est l'ensemble qui réunit les éléments de $A$ \textbf{et} de $B$ pris une seule fois. \point{2}
Si $A=\lbrace{0;1;2;3;4;5;6\rbrace}$ et $B=\lbrace{3;4;7;8\rbrace}$ alors $A \cap B=\lbrace{3;4 \rbrace}$
\end{minipage}
\hfill
\begin{minipage}{0.38\linewidth}

\definecolor{ffttww}{rgb}{1.,0.2,0.4}
\definecolor{xdxdff}{rgb}{0.49019607843137253,0.49019607843137253,1.}
\begin{tikzpicture}[line cap=round,line join=round,>=triangle 45,x=1.0cm,y=1.0cm]
\clip(-0.06,0.86) rectangle (4.72,4.1);
\draw [rotate around={-2.7263109939062526:(3.08,2.22)},line width=2.pt,color=xdxdff,fill=xdxdff,fill opacity=0.30000001192092896] (3.08,2.22) ellipse (1.44092369450198cm and 1.1700688412983384cm);
\draw [rotate around={-61.97549946792966:(1.29,2.28)},line width=2.pt,color=ffttww,fill=ffttww,fill opacity=0.3499999940395355] (1.29,2.28) ellipse (1.123788251075678cm and 0.8772685069325883cm);
\draw (4.16,3.52) node[anchor=north west] {$B$};
\draw (0.28,3.98) node[anchor=north west] {$A$};
\draw [->,line width=1.pt] (1.84,3.38) -- (1.88,2.16);
\draw (1.24,3.98) node[anchor=north west] {$A\cap B$};
\end{tikzpicture}

\end{minipage}

\end{DefT}

\end{minipage}
\begin{minipage}[t]{0.28\linewidth}


\begin{Rq}\index{Ensembles disjoints}
Deux ensembles sont \textbf{disjoints} lorsque $A \cap B = \oslash$. $\oslash$ est l'ensemble vide.\index{Ensemble!vide}
\vspace{0.2cm}

Si $C=\lbrace{0;1;2;3\rbrace}$ 

et $D=\lbrace{4;5;6\rbrace}$ 

alors $A \cap B= \oslash $

\definecolor{xdxdff}{rgb}{0.49019607843137253,0.49019607843137253,1.}
\definecolor{ffttww}{rgb}{1.,0.2,0.4}
\begin{tikzpicture}[line cap=round,line join=round,>=triangle 45,x=1.0cm,y=1.0cm]
\clip(-5.398001213875861,3.010943854292776) rectangle (-0.9359834713983067,4.7475129216353915);
\draw [rotate around={9.950626687951598:(-4.276467024550424,3.8551093731398804)},line width=2.pt,color=ffttww,fill=ffttww,fill opacity=0.15000000596046448] (-4.276467024550424,3.8551093731398804) ellipse (0.8703197956159994cm and 0.5200054996195858cm);
\draw [rotate around={-0.9240453527727059:(-2.0695771681358472,3.7706928212551696)},line width=2.pt,color=xdxdff,fill=xdxdff,fill opacity=0.25] (-2.0695771681358472,3.7706928212551696) ellipse (0.9314625917474598cm and 0.5553715874828972cm);
\draw (-4.867382887743395,4.10984671409655) node[anchor=north west] {$1$};
\draw (-4.360883576435132,3.892775580678723) node[anchor=north west] {$2$};
\draw (-4.553835695028756,4.35103686233858) node[anchor=north west] {$3$};
\draw (-4.095574413368899,4.182203758569159) node[anchor=north west] {$0$};
\draw (-2.624314509092516,4.158084743744956) node[anchor=north west] {$5$};
\draw (-1.7801489902454115,4.158084743744956) node[anchor=north west] {$6$};
\draw (-2.3590053460262834,3.823942476909302) node[anchor=north west] {$4$};
\end{tikzpicture}

\end{Rq}
\end{minipage}


\begin{DefT}{Réunion}\index{Ensemble!Réunion}

\begin{minipage}{0.68\linewidth}
La \textbf{réunion} de deux ensembles est l'ensemble $A \cup B$ qui contient \textbf{tous} les éléments de $A$ \textbf{ou} de $B$ pris une seule fois. 

On lit $A$ "union" $B$. \point{2}

Si $A=\lbrace{0;1;2;3;4;5;6\rbrace}$ et $B=\lbrace{3;4;7;8\rbrace}$ alors $A \cup B=\lbrace{0;1;2;3;4;5;6;7;8 \rbrace}$
\end{minipage}
\hfill
\begin{minipage}{0.28\linewidth}

\definecolor{ffqqqq}{rgb}{1.,0.,0.}
\definecolor{xfqqff}{rgb}{0.4980392156862745,0.,1.}
\definecolor{ffttww}{rgb}{1.,0.2,0.4}
\definecolor{xdxdff}{rgb}{0.49019607843137253,0.49019607843137253,1.}
\begin{tikzpicture}[line cap=round,line join=round,>=triangle 45,x=1.0cm,y=1.0cm]
\clip(0.06,0.8) rectangle (4.82,4.12);
\draw [rotate around={-2.7263109939062526:(3.08,2.22)},line width=2.pt,color=xdxdff,fill=xdxdff,fill opacity=0.5] (3.08,2.22) ellipse (1.44092369450198cm and 1.1700688412983384cm);
\draw [rotate around={-61.97549946792966:(1.29,2.28)},line width=2.pt,color=ffttww,fill=ffttww,fill opacity=0.550000011920929] (1.29,2.28) ellipse (1.123788251075678cm and 0.8772685069325883cm);
\draw [color=xfqqff](4.16,3.52) node[anchor=north west] {$B$};
\draw [color=ffqqqq](0.28,3.98) node[anchor=north west] {$A$};
\draw (1.36,3.96) node[anchor=north west] {$A\cup B$};
\draw [rotate around={-2.7263109939062526:(3.08,2.22)},line width=1.pt,fill=black,pattern=north east lines,pattern color=black] (3.08,2.22) ellipse (1.44092369450198cm and 1.1700688412983384cm);
\draw [rotate around={-61.97549946792966:(1.29,2.28)},line width=1.pt,fill=black,pattern=north east lines,pattern color=black] (1.29,2.28) ellipse (1.123788251075678cm and 0.8772685069325883cm);
\end{tikzpicture}


\end{minipage}

\end{DefT}

\end{pageCours} % Fin page de cours 2

%%%%%%%%%%%%%%%%%%%%%%%%%%%%%%%%%%%%%%%%%%%%%%%%%%%%%%%%%%%%%%%%%%%
%%%% Application direct 2
%%%%%%%%%%%%%%%%%%%%%%%%%%%%%%%%%%%%%%%%%%%%%%%%%%%%%%%%%%%%%%%%%%%

\begin{pageAD}  % Début page d'exercice d'application direct 2
\restoregeometry %Ne pas toucher cette ligne

\Sf{Opérer avec les ensembles}
 
  
\begin{ExoCad}{Communiquer.}{1234}{2}{0}{0}{0}{0}

Compléter par $\subset$  ou $\not\subset$. Justifier l'utilisation de  $\not\subset$.

\begin{enumerate}
\begin{minipage}{0.49\linewidth}
\item $\Z \ldots \ldots \R$ \point{1}
\item $\Q \ldots \ldots \Z$ \point{1}
\item $\Z \ldots \ldots \N$ \point{1}
\end{minipage}
\hfill
\begin{minipage}{0.49\linewidth}
\item $\Z \ldots \ldots \D$ \point{1}
\item $\D \ldots \ldots \Q$ \point{1}
\item $\Q \ldots \ldots \N$ \point{1}
\end{minipage}
\end{enumerate}
 
\end{ExoCad}

\begin{ExoCad}{Représenter.}{1234}{0}{0}{0}{0}{0}

On propose dans chaque cas deux ensembles $A$ et $B$. Lequel est-il inclus dans l'autre ? \vspace{0.2cm} 

\begin{enumerate}

\begin{minipage}{0.48\linewidth}
\item $A=[-1,1;3]$ et $B=]-2,9;6]$ \point{1}
\item $A=[0,7;0,8]$ et $B=[0,5;+\infty[$\point{1}
\end{minipage}
\hfill
\begin{minipage}{0.48\linewidth}
\item $A=]1;2[$ et $B=[1;2]$\point{1}
\item $A=\Q$ et $B=\Z$\point{1}
\end{minipage}

\end{enumerate}
 
\end{ExoCad}




\begin{ExoCad}{Raisonner.}{1234}{0}{0}{0}{0}{0}

Déterminer les complémentaires de l'ensemble $A$ par rapport $\Omega$. 
\begin{enumerate}
\item $\Omega = \lbrace 1;2;3;4;5;6\rbrace$ et $A = \lbrace 1;2 \rbrace$. $\Omega \setminus A = $\point{1}
\item $\Omega = \lbrace -4;-2;-1;0;1;2;\rbrace$ et $A = \lbrace -2;1;2 \rbrace$.  $\Omega \setminus A = $\point{1}
\item $\Omega = \R$ et $A = \Q$. $\Omega \setminus A = $\point{1}
\item $\Omega = \R$ et $A = \R^-$. $\Omega \setminus A = $\point{1}
\end{enumerate} 
 \end{ExoCad}


\begin{ExoCad}{Représenter.}{1234}{0}{0}{0}{0}{0}

Déterminer les intersections des ensembles $A$ et $B$ suivants.   $A \cap B = $ se lit $A$ inter $B$.
\begin{enumerate}
\item $A = \lbrace 1;2;3;4;5;6\rbrace$ et $B = \lbrace 1;2 \rbrace$. $A \cap B = $\point{1}
\item $A = \lbrace -4;-2;-1;0;1;2;\rbrace$ et $B = \lbrace -2;1;2 \rbrace$.  $A \cap B = $\point{1}
\item $A = \N$ et $B = \R$. $A \cap B = $ \point{1}
\item $A = [-4;3[$ et $B =[-2;7]$. $A \cap B = $ \point{1}
\item $A = [-2;1]$ et $B =[2;3]$. $A \cap B = $ \point{1}
\item $A = \N$ et $B =]-\infty;5]$. $A \cap B = $ \point{1}
\end{enumerate}
 
\end{ExoCad}









\begin{ExoCad}{Représenter.}{e/1234}{0}{0}{0}{0}{0}

Déterminer les réunions des ensembles $A$ et $B$ suivants. 
\begin{enumerate}
\item $A = \lbrace 1;2;3;4;5;6\rbrace$, $B = \lbrace 1;2 \rbrace$. $A \cup B = $\point{1}
\item $A = \lbrace -4;-2;-1;0;1;2;\rbrace$,$B = \lbrace -2;1;2 \rbrace$.  $A \cup B = $\point{1}
\item $A=\Z$, $B =\R$. $A \cup B = $ \point{1}
\item $A=\left\lbrace 1;2;8;6  \right\rbrace $, $B =\left\lbrace 0;2;4;8  \right\rbrace $. $A \cup B = $ \point{1}
\item $A=[-2;1]$, $B =[2;3]$. $A \cup B = $ \point{1}
\item $A=[0;+\infty[$, $B =]-\infty;5]$. $A \cup B = $ \point{1}
\end{enumerate} 
\end{ExoCad}

\end{pageAD} % Fin page d'exercice d'application direct 2

%%%%%%%%%%%%%%%%%%%%%%%%%%%%%%%%%%%%%%%%%%%%%%%%%%%%%%%%%%%%%%%%%%%
%%%% Parcours niveau 1
%%%%%%%%%%%%%%%%%%%%%%%%%%%%%%%%%%%%%%%%%%%%%%%%%%%%%%%%%%%%%%%%%%%

\begin{pageParcoursu} % Début du parcours niveau 1

%Premier exo du parcours 1
\begin{ExoCu}{Communiquer.}{1234}{2}{0}{0}{0}{0}

Johan visite Vienne, la capitale de l'Autriche. L' Autriche est un pays Européen.
\begin{enumerate}[leftmargin=*]
\item Arrivé en Autriche, Johan est-il à Vienne ? \point{1}
\item A Vienne, Johan est-il en Autriche ?\point{1}
\item Peut-on dire que Johan est en Europe ?\point{1}
\item Lettres $A$, $E$ et $V$ sont les initiales respectives de Autriche, Europe et Vienne. Compléter avec les lettres $A$, $E$ et $V$ : $\ldots\ldots \subset \ldots\ldots \subset \ldots\ldots $
\end{enumerate}
\end{ExoCu}

%Deuxième exo du parcours 1
\begin{ExoCu}{Communiquer.}{1234}{2}{0}{0}{0}{0}

Recopie et complète par $\subset$, $\in$, $\not\subset$ ou $\notin$.

\begin{enumerate}
\begin{minipage}{0.32\linewidth}
\item $\N \ldots \ldots \R$
\item $-5 \ldots \ldots \Z$
\item $\Z \ldots \ldots \N$
\end{minipage}
\hfill
\begin{minipage}{0.32\linewidth}
\item $\left\lbrace 0;1;2 \right\rbrace \ldots \ldots \N$
\item $]-\infty;1] \ldots \ldots \R$
\item $]0;+\infty[ \ldots \ldots \Z$
\end{minipage}
\hfill
\begin{minipage}{0.32\linewidth}
\item $\sqrt{3} \ldots \ldots \N$
\item $-1,5 \ldots \ldots \Z$
\item $\Q \ldots \ldots \Z$
\end{minipage}
\end{enumerate}
 
\end{ExoCu}

%Troisième exo du parcours 1
\begin{ExoCu}{Communiquer.}{1234}{2}{0}{0}{0}{0}

Déterminer les intersections des ensembles suivants.
\begin{enumerate}
\item $A = \Z$ et $B = \Q$. $A \cap B = $ \point{1}
\item $A = ]-2;3[$ et $B =[-1;5]$. $A \cap B = $ \point{1}
\item $A = [-4;3]$ et $B =[2;6]$. $A \cap B = $ \point{1}
\item $A = ]-6;1[$ et $B =[0;5]$. $A \cap B = $ \point{1}
\end{enumerate}
\end{ExoCu}

%Quatrième exo du parcours 1
\begin{ExoCu}{Chercher.communiquer.}{2}{0}{0}{0}{0}
 
Dans chaque cas, trouver, lorsque cela est possible, un nombre $x$ qui remplit les critères suivants :
\begin{enumerate}
\item $x \in \Q$ et $x \not\in \Z$ \point{1}
\item $x \in \R$ et $x \not\in \N$ \point{1}
\end{enumerate}
\end{ExoCu}

%Cinquième exo du parcours 1
\begin{ExoCu}{Représenter. Raisonner. Communiquer.}{1234}{2}{0}{0}{0}{0}
Déterminer, dans chaque cas, la réunion des ensembles suivants. On écrira : $A \cup B = $ où $A$ et $B$ sont les ensembles ci-dessous.
\begin{enumerate}
\item $A=\left\lbrace 1;3;5;7  \right\rbrace $ et $B=\left\lbrace 0;2;4;5;7;8  \right\rbrace $\point{1}
\item $A=[-3;4]$ et $B=[2;6]$\point{1}
\item $A=[0;+\infty[$ et $B=]-\infty;5]$\point{1}
\end{enumerate}
On pourra représenter les intervalles sur une droite graduée tracée à main levée.\end{ExoCu}

%Sixième exo du parcours 1
\begin{ExoCu}{Modéliser.}{1234}{0}{0}{0}{0}{0}
Déterminer l'ensemble des valeurs de $x$ dans chaque cas.
\begin{enumerate}
\item $x < -4$ et $x \geq 10$\point{1}
\item $x \leq 6$ et $x \leq 3$\point{1}
\item $x \leq 6$ ou $x \geq 3$\point{1}
\end{enumerate} 
\end{ExoCu}


\end{pageParcoursu} % Fin du parcours niveau 1
 
%%%%%%%%%%%%%%%%%%%%%%%%%%%%%%%%%%%%%%%%%%%%%%%%%%%%%%%%%%%%%%%%%%%
%%%% Parcours Niveau 2
%%%%%%%%%%%%%%%%%%%%%%%%%%%%%%%%%%%%%%%%%%%%%%%%%%%%%%%%%%%%%%%%%%%

\begin{pageParcoursd} % Début du parcours niveau 2

%Premier exo du parcours 2
\begin{ExoCd}{Représenter. Raisonner.}{1234}{0}{0}{0}{0}{0}

Déterminer l'intervalle des valeurs de $x$ dans chaque cas.

\begin{enumerate}
\begin{minipage}{0.5\linewidth}
%
Déterminer l'ensemble des valeurs de $x$ dans chaque cas.
\begin{enumerate}
\item On jette un dé à 6 face et on regarde la face obtenue. Soit $x$ le numéro de la face. 
\item $[-1,1;3]$ et $[2,9;6]$
\item $x > -4$ et $x \leq 10$
\item $x \leq -3$ et $x \leq 5$
\item $x \leq 5$ ou $x \geq 2$
\end{enumerate} 

\item $[-1,1;3]$ et $[2,9;6]$
\item $x > -4$ et $x \leq 10$
\item $x \leq -3$ et $x \leq 5$


\end{minipage}
\begin{minipage}{0.5\linewidth}
%
Soit $x$ un réel.Écrire sous forme d'intervalle ou de réunion d'intervalles le plus grand ensemble auquel appartient $x$.


\begin{enumerate}
	\item $x \geq 1$ ou $x<3$.  
	\item $x \geq 1$ et $x<3$.  
\end{enumerate}
 
\item $x \geq 1$ ou $x<3$.  
\item $x \geq 1$ et $x<3$.  
\item $x \leq 5$ ou $x \geq 2$
\end{minipage}
\end{enumerate}

\end{ExoCd}

%Deuxième exo du parcours 2
\begin{ExoCd}{Représenter.}{1234}{0}{0}{0}{0}{0}

Dans chaque cas, trouver, lorsque cela est possible, un nombre $x$ qui remplit les conditions suivantes :
 
\begin{enumerate}[leftmargin=*]
\item $x \not\in D$ et $x \in \R$  \point{1}
\item $x \in \Q$ et $x \not\in \Z$  \point{1}
\item $x \not\in \N$ et $x \in \Z$  \point{1}
\item $x \in \D$ ou $x \in \Q$  \point{1}
\item $x \not\in \N$ ou $x \in \Z$  \point{1}
\end{enumerate}
 
 \end{ExoCd}

%Troisième exo du parcours 2
\begin{ExoCd}{Représenter. Raisonner. Communiquer.}{1234}{0}{0}{0}{0}{0}

On propose dans chaque cas deux ensembles $A$ et $B$. Lequel est inclus dans l'autre ?  

\textit{{\small On pourra représenter chaque intervalle sur une droite graduée.}}

\begin{minipage}{0.48\linewidth}

 
\begin{enumerate}
\item $A = \left[ -\frac{11}{10};\frac{29}{10}\right]$ et $B=\left[-\frac{3}{2};3 \right]$
\item $A =\left[ \frac{1}{2}; +\infty \right[$ et $B=[0,7;0,8]$.
\item $A =[1;2]$ et $B=]1;2[$. 
\end{enumerate}

\end{minipage}
\hfill
\begin{minipage}{0.48\linewidth}
 
\begin{enumerate}

\item

\begin{tikzpicture}[line cap=round,line join=round,>=triangle 45,x=1.0cm,y=1.0cm]
\draw [->,line width=1.pt,domain=0.34:6.36] plot(\x,{(-14.-0.*\x)/7.});
\end{tikzpicture}
\item

\begin{tikzpicture}[line cap=round,line join=round,>=triangle 45,x=1.0cm,y=1.0cm]
\draw [->,line width=1.pt,domain=0.34:6.36] plot(\x,{(-14.-0.*\x)/7.});
\end{tikzpicture}
\item

\begin{tikzpicture}[line cap=round,line join=round,>=triangle 45,x=1.0cm,y=1.0cm]
\draw [->,line width=1.pt,domain=0.34:6.36] plot(\x,{(-14.-0.*\x)/7.});
\end{tikzpicture}
\end{enumerate}

\end{minipage} 
\end{ExoCd}

%Quatrième exo du parcours 2
\begin{ExoCd}{Représenter. Raisonner. Communiquer.}{1234}{0}{0}{0}{0}{0}

Déterminer les intersections des ensembles suivants. On écrira : $A \cap B = $ où $A$ et $B$ sont les ensembles ci-dessous.
 

\textit{{\small On pourra représenter chaque intervalle sur une droite graduée.}}



\begin{minipage}{0.48\linewidth}

\begin{enumerate}
\item $\Z$ et $\Q$
\item $[-5;2[$ et $[0;7]$
\item $[-1;4]$ et $[-3;-1]$
\item $\N$ et $]-\infty;5]$
\item $[-5;0[$ et $[0;3]$
\end{enumerate}

\end{minipage}
\hfill
\begin{minipage}{0.48\linewidth}
 
\begin{enumerate}
\item

\begin{tikzpicture}[line cap=round,line join=round,>=triangle 45,x=1.0cm,y=1.0cm]
\draw [->,line width=1.pt,domain=0.34:6.36] plot(\x,{(-14.-0.*\x)/7.});
\end{tikzpicture}
\item

\begin{tikzpicture}[line cap=round,line join=round,>=triangle 45,x=1.0cm,y=1.0cm]
\draw [->,line width=1.pt,domain=0.34:6.36] plot(\x,{(-14.-0.*\x)/7.});
\end{tikzpicture}
\item

\begin{tikzpicture}[line cap=round,line join=round,>=triangle 45,x=1.0cm,y=1.0cm]
\draw [->,line width=1.pt,domain=0.34:6.36] plot(\x,{(-14.-0.*\x)/7.});
\end{tikzpicture}
\item

\begin{tikzpicture}[line cap=round,line join=round,>=triangle 45,x=1.0cm,y=1.0cm]
\draw [->,line width=1.pt,domain=0.34:6.36] plot(\x,{(-14.-0.*\x)/7.});
\end{tikzpicture}
\item

\begin{tikzpicture}[line cap=round,line join=round,>=triangle 45,x=1.0cm,y=1.0cm]
\draw [->,line width=1.pt,domain=0.34:6.36] plot(\x,{(-14.-0.*\x)/7.});
\end{tikzpicture}
\end{enumerate}

\end{minipage} 
\end{ExoCd}

%Cinquième exo du parcours 2
\begin{ExoCd}{Chercher.}{1234}{0}{0}{0}{0}{0}

On donne le programme en Python ci dessous. 
 
\begin{lstlisting}
def is_in(x,a,b):
    if x > a and x < b :
    	test = "{} is in  ]{};{}[".format(x,a,b) 
    else :
        test = "{} is not in ]{};{}[".format(x,a,b) 
    return test    
x=int(input("Entrer un nombre  :")) 
a=int(input("Entrer la borne inf :"))
b=int(input("Entrer la borne sup :"))    
print(is_in(x,a,b))
\end{lstlisting}
 


\begin{enumerate}
\item Que fait ce programme ? Vous pouvez aussi ouvrir l'éditeur Python : \url{https://sacado.xyz/tool/show/18}
\item Modifier ce programme pour qu'il teste si un nombre $x$ appartient à l'intervalle $[a;b]$.
\end{enumerate} 
\end{ExoCd}

\end{pageParcoursd} % Fin du parcours niveau 2

%%%%%%%%%%%%%%%%%%%%%%%%%%%%%%%%%%%%%%%%%%%%%%%%%%%%%%%%%%%%%%%%%%%%
%%%%% Parcours Niveau 3
%%%%%%%%%%%%%%%%%%%%%%%%%%%%%%%%%%%%%%%%%%%%%%%%%%%%%%%%%%%%%%%%%%%%

\begin{pageParcourst} % Début du parcours niveau 3

% Premier exo du parcours 3
\begin{ExoCtN}{Représenter.}{1}{1}{0}{0}{0}

Démontrer que si $p^2$ est impair alors $p$ est impair.

 
\end{ExoCtN}

% Deuxième exo du parcours 3
\begin{DefT}{Nombre décimal périodique}
Le nombre $a_0,\underline{a_1a_2a_3}$ est un nombre décimal périodique de période $a_1a_2a_3$. Les chiffres $a_1$, $a_2$, $a_3$ se répètent indéfiniment.
\end{DefT}

\begin{ExoCtN}{Représenter. Calculer.}{1234}{0}{0}{0}{0} 
 
Démontrer que $0,\underline{9}=1$.   \point{6}
 
\end{ExoCtN}

% Troisième exo du parcours 3
\begin{ExoCtN}{Représenter. Calculer.}{1234}{1}{0}{0}{0}

\begin{enumerate}
\item On considère le nombre $\frac{19}{11}$.

\begin{enumerate}
\item Donner le développement décimal de $\frac{19}{11}$ avec 8 chiffres significatifs. $\frac{19}{11}$ semble-t-il décimal ?
\item On dit que $\frac{19}{11}$ a une écriture périodique.
Préciser sa période (série de chiffres qui se répète à l'infini dans le développement décimal).
\end{enumerate}
\item On considère le nombre $x=0,13131313....$ dont le développement décimal a pour période 13.
\begin{enumerate}
\item Démontrer que $100x = 13 + x$. 
\item  En déduire une écriture fractionnaire de $x$. Quelle est la nature du nombre $x$ ?
\end{enumerate}
\end{enumerate} 
 
\end{ExoCtN}

% Quatrième exo du parcours 3
\begin{ExoCtN}{Représenter. Calculer.}{2}{1}{0}{0} 

\begin{enumerate}
\item Démontrer que $0,\underline{12}$ est un nombre rationnel à préciser.
\item Démontrer que $x=3,\underline{412}$ est un nombre rationnel. 
\end{enumerate}
\end{ExoCtN} 

% Cinquième exo du parcours 3
\begin{ExoCt}{Raisonner.}{1234}{2}{0}{0}{0}{0}
 
Représenter graphiquement dans le plan muni d'un repère orthonormal 
 
\begin{enumerate}
\item l'ensemble des points $M(x;y)$ tes que  $$1 < x < 3 \text{ et} 0 \leq y <4$$
\item l'ensemble des points $M(x;y)$ tes que  $$1 \leq  x \leq  5 \text{ et } -2 \leq y   \leq  1$$
\end{enumerate}
 
\end{ExoCt}

% Sixième exo du parcours 3
\begin{ExoCt}{Représenter.}{1234}{2}{0}{0}{0}{0}
Représenter graphiquement, dans le plan muni d'un repère orthonormal, l'ensemble des points $M(x;y)$ tes que  $$1 \leq  2x+1 \leq  5 \text{ et }  -2 \leq 3y + 4  \leq  13$$
\end{ExoCt} 
 
\end{pageParcourst} % Fin du parcours niveau 3

%%%%%%%%%%%%%%%%%%%%%%%%%%%%%%%%%%%%%%%%%%%%%%%%%%%%%%%%%%%%%%%%%%%
%%%%  Autoevaluation/exos ouverts
%%%%%%%%%%%%%%%%%%%%%%%%%%%%%%%%%%%%%%%%%%%%%%%%%%%%%%%%%%%%%%%%%%%

\begin{pageAuto} % Début de la page d'exos ouverts

%Premier exercice : "ce que je peux avoir en éval"
\begin{ExoAutoN}{Raisonner.}{2}{0}{0}{0}{0}

Compléter avec $\in$, $\not\in$, $\subset$ ou $\not\subset$.

 \begin{tabular}{ccc}

$\frac{2}{10}..........\Z$ & $-\sqrt{25}..........\Z$ & $\frac{\sqrt{3}}{4}..........\Q$ \\ 

$\pi..........\R$  & $-\frac{5}{3}..........\Q$  &  $\sqrt{11}..........\R$ \\ 

\end{tabular} 

\end{ExoAutoN}

%Deuxième exercice : "ce que je peux avoir en éval"
\begin{ExoAutoN}{Raisonner.}{2}{0}{0}{0}{0}
 

Déterminer, dans chaque cas, l'intersection puis la réunion des ensembles suivants. 

\begin{enumerate}
\item $A=\left\lbrace 1;3;5;7  \right\rbrace $ et $B=\left\lbrace 0;2;4;5;7;8  \right\rbrace $\point{2}
\item $A=[-3;4]$ et $B=[2;6]$\point{2}
\item $A=[0;+\infty[$ et $B=]-\infty;5]$\point{2}
\end{enumerate}
On pourra représenter les intervalles sur une droite graduée tracée à main levée.
\end{ExoAutoN}

%Troisième exercice : "ce que je peux avoir en éval"
\begin{ExoAutoN}{Représenter. Raisonner.}{2}{0}{0}{0}{0}

 Recopier et compléter le tableau.

\begin{tabular}{|c|c|c|}
\hline 
Intervalle & Inégalité & Représentation   \\ 
\hline 
$x\in \left[ -2 ; 6\right]$ & $-2  \leq x \leq  6 $  &  \\ 
\hline 
 & $1 \leq x <3$ &     \\ 
\hline 
$x\in \left[ -6 ; 6 \right[ $  &  &  \\ 
\hline 
 &  & \definecolor{ffdxqq}{rgb}{1.,0.8431372549019608,0.}
\definecolor{ffxfqq}{rgb}{1.,0.4980392156862745,0.}
\begin{tikzpicture}[line cap=round,line join=round,>=triangle 45,x=1.0cm,y=1.0cm]
\draw[->,color=black] (-5.174092090680384,0.) -- (2.566282833730012,0.);
\foreach \x in {-5.,-4.,-3.,-2.,0,-1.,1.,2.}
\draw[shift={(\x,0)},color=black] (0pt,2pt) -- (0pt,-2pt) node[below] {\footnotesize $\x$};
\clip(-5.174092090680384,-0.4115875953650586) rectangle (2.566282833730012,0.4791698364123281);
\draw [line width=2.4pt,color=ffxfqq] (-4.,0.)-- (1.,0.);
\draw [color=ffxfqq](0.8,0.35) node[anchor=north west] {\Large{]}};
\draw [color=ffxfqq](-4.2,0.35) node[anchor=north west] {\Large{]}};
\end{tikzpicture}    \\ 
\hline 
\end{tabular} 
\end{ExoAutoN}

%%%%%%%%%%%%%%%%%%%%%%%%%%%%%%%%%%%%%%%%%%%%%%%%%%%%%%%%%%%%%%%%%%%
%%%%%%%%%%%%%%%%%%%%%%%%%%%%%%%%%%%%%%%%%%%%%%%%%%%%%%%%%%%%%%%%%%%

%Problème ouvert
\begin{ExoAutoN}{Raisonner.}{2}{0}{0}{0}{0}

Démontrer que $0,\underline{485}$ est un nombre rationnel à préciser.

\end{ExoAutoN}

\end{pageAuto} % Fin de la page d'exos ouverts

%%%%%%%%%%%%%%%%%%%%%%%%%%%%%%%%%%%%%%%%%%%%%%%%%%%%%%%%%%%%%%%%%%%
%%% Page algorithmique
%%%%%%%%%%%%%%%%%%%%%%%%%%%%%%%%%%%%%%%%%%%%%%%%%%%%%%%%%%%%%%%%%%%

\begin{pageAlgo} % Début de la page d'exos d'algorithmique

%Exercice d'algorithmique
\begin{ExoAlgoN}{Compétence.}{1}{0}{0}{0}{0}

\end{ExoAlgoN}

%Exercice d'algorithmique
\begin{ExoAlgo}{Compétence.}{1}{0}{0}{0}{0}

\end{ExoAlgo}

\end{pageAlgo} % Fin de la page d'exos d'algorithmique

%%%%%%%%%%%%%%%%%%%%%%%%%%%%%%%%%%%%%%%%%%%%%%%%%%%%%%%%%%%%%%%%%%%
%%%%  Page(s) blanche(s)
%%%%%%%%%%%%%%%%%%%%%%%%%%%%%%%%%%%%%%%%%%%%%%%%%%%%%%%%%%%%%%%%%%%

\begin{pageBrouillon}

\end{pageBrouillon}

\end{document}
