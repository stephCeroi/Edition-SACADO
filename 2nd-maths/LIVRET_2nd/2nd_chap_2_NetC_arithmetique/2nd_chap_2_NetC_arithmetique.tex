%----------------------------------------------------------------------------------------
%	PACKAGES AND OTHER DOCUMENT CONFIGURATIONS
%----------------------------------------------------------------------------------------

%----------------------------------------------------------------------------------------
%		Géometrie de la page
%----------------------------------------------------------------------------------------
\documentclass[dvipsnames,french,10pt]{book}

\usepackage[
paperheight=29.7cm, %hauteur du papier
paperwidth=21cm, %largeur du papier
left=1cm, %marge de gauche
right=1cm, %marge de droite
top=1.5cm, %marge du haut
bottom=1cm, %marge du bas
%marginparsep=0pt, %distance entre le texte et les notes de marges 
reversemp, %inverse l'emplacement de la marge
headheight=20.60pt %hauteur du header
%showframe, %permet d'afficher le cadre défini ci-dessus
%bindingoffset=1cm %permet d'ajouter le décalage dû au reliage
]{geometry} %Redéfinition de la taille des pages
\raggedbottom


%----------------------------------------------------------------------------------------
%		Generals
%----------------------------------------------------------------------------------------
%\usepackage{fourier} %!! A changer plus tard !!
\usepackage[scaled]{uarial}
\renewcommand*\familydefault{\sfdefault} %% Only if the base font of the document is to be sans serif
\usepackage{frcursive}
\usepackage[T1]{fontenc} %Accents handling
\usepackage[utf8]{inputenc} % Use UTF-8 encoding
%\usepackage{microtype} % Slightly tweak font spacing for aesthetics
\usepackage[english, francais]{babel} % Language hyphenation and typographical rules
\usepackage{marginnote}


%----------------------------------------------------------------------------------------
%		Graphics
%----------------------------------------------------------------------------------------
\usepackage{xcolor}
\usepackage{graphicx, multicol} % Enhanced support for graphics
\graphicspath{FIG/}
\usepackage{wrapfig}
\usepackage{colortbl}
\usepackage[framemethod=tikz]{mdframed}
%\usepackage{xsavebox}
% Il faudrait utiliser xsavebox à l'avenir pour réduire la taille du pdf

%----------------------------------------------------------------------------------------
%		Other packages
%----------------------------------------------------------------------------------------
\usepackage{hyperref}
\hypersetup{
	colorlinks=true, %colorise les liens
	breaklinks=true, %permet le retour à la ligne dans les liens trop longs
	urlcolor= sacado_violet,  %couleur des hyperliens et des QR codes
	linkcolor= sacado_violet, %couleur des liens internes
	plainpages=false  %pour palier à "Bookmark problems can occur when you have duplicate page numbers, for example, if you have a page i and a page 1."
}
\usepackage{tabularx}
\newcolumntype{M}[1]{>{\arraybackslash}m{#1}} %Defines a scalable column type in tabular
\usepackage{booktabs} % Enhances quality of tables
\usepackage{diagbox} % barre en diagonale dans un tableau
\usepackage{multicol}
\usepackage[explicit]{titlesec}
\usepackage{xr}
\usepackage{xspace}
\usepackage{array}
\usepackage{listings}
\usepackage{fancyvrb} %verbatim
\usepackage{stmaryrd}
\usepackage{float}



% Python style for highlighting
\lstdefinestyle{mystyle}{
    backgroundcolor=\color{white},   
    commentstyle=\color{sacado_green},
    keywordstyle=\color{sacado_red},
    numberstyle=\tiny\color{sacado_orange},
    stringstyle=\color{sacado_blue},
    basicstyle=\ttfamily\footnotesize,
    breakatwhitespace=false,         
    breaklines=true,                 
    captionpos=b,                    
    keepspaces=false,                 
    numbers=left,                    
    numbersep=5pt,                  
    showspaces=false,                
    showstringspaces=false,
    showtabs=false,                  
    tabsize=4
}

\lstset{style=mystyle}

%----------------------------------------------------------------------------------------
%		Headers and footers
%----------------------------------------------------------------------------------------

\pagestyle{empty}
\usepackage{fancyhdr}
\pagestyle{fancy}
\renewcommand{\headrulewidth}{0pt} % pas de filet sous le header

%----------------------------------------------------------------------------------------
%		Mathematics packages
%----------------------------------------------------------------------------------------
\usepackage{amsthm, amsmath, amssymb, mathrsfs} % Mathematical typesetting
\usepackage{marvosym, wasysym} % More symbols
\usepackage[makeroom]{cancel}
\usepackage{xlop}
\usepackage{pgf,tikz,pgfplots}
\pgfplotsset{compat=1.16}
\usepackage{pgf-pie}
\usetikzlibrary{positioning}
\usetikzlibrary{arrows}
\usepackage{pst-plot,pst-tree,pst-func, pstricks-add,pst-node,pst-text}
%\usepackage{units}
\usepackage{nicefrac}
\usepackage[np]{numprint} %Séparation milliers dans un nombre \np{12345} donne 12 345
\usepackage{multido}
\newcommand{\RNum}[1]{\uppercase\expandafter{\romannumeral #1\relax}}

%----------------------------------------------------------------------------------------
%		New text commands
%----------------------------------------------------------------------------------------
\usepackage{calc}
\usepackage{boites}
 \renewcommand{\arraystretch}{1.6}

%%%%% Pour les imports.
\usepackage{import}

%%%%% Pour faire des boites
\usepackage[tikz]{bclogo}
\usepackage{bclogo}
\usepackage{framed}
\usepackage[skins]{tcolorbox}
\tcbuselibrary{breakable}
\tcbuselibrary{skins}
\usetikzlibrary{quotes,babel,arrows.meta,shadows,decorations.pathmorphing,decorations.markings,patterns}
\usepackage{tikzpagenodes}
\usetikzlibrary{plotmarks}



%%%%% Pour une double minipage
\newcommand{\mini}[4]{
\begin{minipage}[c]{#1}
#2
\end{minipage}
\hfill
\begin{minipage}[c]{#3}
#4
\end{minipage}
}


\usepackage{enumitem}
\newlist{todolist}{itemize}{2} %Pour faire des QCM
\setlist[todolist]{label=$\square$} %Pour faire des QCM \begin{todolist} instead of itemize
\renewcommand{\FrenchLabelItem}{\textbullet} %bullet dans les items


%----------------------------------------------------------------------------------------
%		Définition de couleurs pour ...
%----------------------------------------------------------------------------------------

%GEOGEBRA

\definecolor{zzttqq}{rgb}{0.6,0.2,0.} %rouge des polygones
\definecolor{qqqqff}{rgb}{0.,0.,1.}
\definecolor{xdxdff}{rgb}{0.49019607843137253,0.49019607843137253,1.}%bleu
\definecolor{qqwuqq}{rgb}{0.,0.39215686274509803,0.} %vert des angles
\definecolor{ffqqqq}{rgb}{1.,0.,0.} %rouge vif
\definecolor{uuuuuu}{rgb}{0.26666666666666666,0.26666666666666666,0.26666666666666666}
\definecolor{qqzzqq}{rgb}{0.,0.6,0.}
\definecolor{cqcqcq}{rgb}{0.7529411764705882,0.7529411764705882,0.7529411764705882} %gris
\definecolor{qqffqq}{rgb}{0.,1.,0.}
\definecolor{ffdxqq}{rgb}{1.,0.8431372549019608,0.}
\definecolor{ffffff}{rgb}{1.,1.,1.}
\definecolor{ududff}{rgb}{0.30196078431372547,0.30196078431372547,1.}
\definecolor{ffqqff}{rgb}{1.,0.,1.}
\definecolor{ffxfqq}{rgb}{1,0.4980392156862745,0}
\definecolor{ffffqq}{rgb}{1,1,0}
\definecolor{qqttzz}{rgb}{0,0.2,0.6}
\definecolor{qqccqq}{rgb}{0,0.8,0}
\definecolor{qqzzff}{rgb}{0,0.6,1}
\definecolor{qqwwzz}{rgb}{0,0.4,0.6}
\definecolor{eqeqeq}{rgb}{0.8784313725490196,0.8784313725490196,0.8784313725490196}

%SACADO

\definecolor{fond}{HTML}{5D4391}  %couleur des entetes etc.  violet sacado
\definecolor{sacado_purple}{HTML}{5D4391} %% Violet foncé Sacado
\definecolor{sacado_violet}{HTML}{9274C7} %% Violet clair Sacado
\definecolor{texte}{HTML}{FFFFFF} % couleur du texte des entetes etc.
\definecolor{sacado_blue_light}{HTML}{0093CA} %% Bleu Sacado
\definecolor{sacado_blue}{HTML}{0960B5} %% Bleu Sacado
\definecolor{sacado_green}{HTML}{00B999} %% Vert Sacado
\definecolor{sacado_green_dark}{HTML}{4D8075} %% Vert Sacado foncé
\definecolor{sacado_yellow}{HTML}{F9F871} %% Jaune Sacado
\definecolor{sacado_orange}{HTML}{FF8B69} %% Orange Sacado
\definecolor{sacado_red}{HTML}{9F1E17} %% Rouge Sacado
\definecolor{sacado_gray}{HTML}{7B7485} %% Gris Sacado
%BOITES 

\definecolor{bleu1}{rgb}{0.54,0.79,0.95} %% Bleu
\definecolor{sapgreen}{rgb}{0.4, 0.49, 0}
\definecolor{dvzfxr}{rgb}{0.7,0.4,0.}
\definecolor{beamer}{rgb}{0.5176470588235295,0.49019607843137253,0.32941176470588235} % couleur beamer
\definecolor{preuveRbeamer}{rgb}{0.8,0.4,0}
\definecolor{sectioncolor}{rgb}{0.24,0.21,0.44}
\definecolor{subsectioncolor}{rgb}{0.1,0.21,0.61}
\definecolor{subsubsectioncolor}{rgb}{0.1,0.21,0.61}
\definecolor{info}{rgb}{0.82,0.62,0}
\definecolor{bleu2}{rgb}{0.38,0.56,0.68}
\definecolor{bleu3}{rgb}{0.24,0.34,0.40}
\definecolor{bleu4}{rgb}{0.12,0.20,0.25}
\definecolor{vert}{rgb}{0.21,0.33,0}
\definecolor{vertS}{rgb}{0.05,0.6,0.42}
\definecolor{red}{rgb}{0.78,0,0}
\definecolor{color5}{rgb}{0,0.4,0.58}
\definecolor{eduscol4B}{rgb}{0.19,0.53,0.64}
\definecolor{eduscol4P}{rgb}{0.62,0.12,0.39}
\definecolor{ill_frame}{HTML}{F0F0F0} %Boite illustration contour
\definecolor{ill_back}{HTML}{F7F7F7}  %Boite illustration background
\definecolor{ill_title}{HTML}{AAAAAA} %Boite illustration titre

%----------------------------------------------------------------------------------------
%		QR codes
%----------------------------------------------------------------------------------------

\usepackage[
final %Pour la compilation finale
%draft %Pour le travail sur les documents
]{qrcode}
\usepackage{fontawesome}
\usepackage{fancyqr}
\FancyQrLoad{flat}
\fancyqrset{
%image=\scalebox{.8}{\includegraphics[scale=1]{sacadoA1.png}},image padding=.5,
l color=sacado_green,r color=sacado_blue}
\newcommand{\qr}[2]{\centering \fancyqr{https://sacado.xyz/qcm/show_course_from_qrcode/#1}

\vspace{.2cm}

#2} %\qr{id} Pour obtenir un qrcode en indiquant seulement l'id de l'exercice





\newcommand{\miniqr}[3]{
\begin{minipage}[c]{.8\linewidth}
#1
\end{minipage}
\hfill
\fbox{
\begin{minipage}[c]{.18\linewidth}
\begin{center}
\fancyqr{https://sacado.xyz/qcm/show\_course\_from\_qrcode/#2}

\vspace{.2cm}

#3
\end{center}
\end{minipage}
}
}

%practice/frombook/<int:ide>/ pour accéder à un exercice depuis le livre.

\usepackage{pythontex}
\begin{pycode}
import qrcode
def qr(data):
     fic=r"QRcodes/qr"+data+'.png'
     urlcourte=r"sacado.xyz/"+data
     urllongue=r"https://"+urlcourte
     qr = qrcode.QRCode(version=1,
error_correction=qrcode.constants.ERROR_CORRECT_L,box_size=2, border=0)
     qr.add_data(urlcourte)
     qr.make(fit=True)
     qr_image = qr.make_image(fill_color="black", back_color="white")
     qr_image.save(fic)
     return r"""\parbox{3.5cm}{\begin{center}
\includegraphics{"""+fic+r"}\\{\scriptsize\tt "+urlcourte+r"}\end{center}}"
\end{pycode}

%\renewcommand{\qr}[1]{\py{qr("#1")}} % compilation complete
%utiliser :
%  pdflatex --shell-escape MANUEL_6e_V1.tex ; pythontex MANUEL_6e_V1.tex --interpreter python:python3 ; pdflatex --shell-escape % %MANUEL_6e_V1.tex

%  draft
\renewcommand{\qr}[1]{
\parbox{5cm}{\begin{center}
       \includegraphics{QRcodes/qrDummy.png}
       %\\{\tt dummy}
\end{center}}
}

\renewcommand{\miniqr}[1]{
       \includegraphics[height=1cm]{QRcodes/qrDummy.png}
       %\\{\tt dummy}}
}


\usepackage[absolute]{textpos}
\newcommand{\qrHautDroite}[1]{
\setlength{\TPHorizModule}{1cm}
\setlength{\TPVertModule}{1cm}
\begin{textblock}{3.51}(12.5,1.5){\qr{#1}}
\end{textblock}
}

 






\usepackage{makeidx}
\makeindex

%----------------------------------------
%
%   Définitions des environnements "pageCours" et "pageExos"
%
%----------------------------------------

\newcounter{cpt}
\newcounter{exo}
\newcounter{cptr}

\newcommand{\titreChap}{Titre de chapitre à définir}

\renewcommand{\chapter}[3]{
  \stepcounter{chapter}
  \setcounter{exo}{0}
  \setcounter{cpt}{0}
  
%\cleardoublepage  % pour commencer à droite
{\Huge \hfill Chapitre \Roman{chapter}.\\
  \bigskip
  #1\\
  \bigskip {\begin{center}
  \fancyqr[image={\includegraphics[scale=.6]{sacadoA1.png}},image padding=.5,height=5cm]{#2}
  \end{center}}  {\normalsize #3}}
\renewcommand{\titreChap}{#1}

%\ifthenelse{\equal{#2}{}}{}{\par
%  \bigskip\bigskip
%  #2}
\newpage
}

\renewenvironment{leftbar}[1][\hsize]
{%
    \def\FrameCommand
    {%
        {\color{black}\vrule width 0.5pt}%
        \hspace{4pt}%must no space.
        \fboxsep=\FrameSep%\colorbox{yellow}%
    }%
    \MakeFramed{\hsize#1\advance\hsize-\width\FrameRestore}%
}
{\endMakeFramed}


\newcommand{\headerGeneral}[3]{ % intitulé, couleur, qrcode
\begin{tikzpicture}[remember picture,overlay]
\coordinate(NO) at (-2,0);
\coordinate(SW) at (22,1);
\coordinate(titre) at (0,0.2);
\coordinate(qr) at (16.85,0.);
\shade[left color=#2 , right color=#2 ] (NO) rectangle (SW);
\draw (titre) node[color=texte, anchor=west]{ {\large \bf  #1} \quad\quad \bf {\small \titreChap} };
\draw (qr) node {\qr{#3}};
\end{tikzpicture}
}

\newenvironment{pageCours}{\lhead{%
\pagecolor{white!100}
\headerGeneral{COURS}{fond!70}{p/1234}
}\begin{leftbar}}{\end{leftbar}\newpage}

\newenvironment{pageAD}{\lhead{%
\pagecolor{sacado_violet!6}
\headerGeneral{APPLICATIONS DIRECTES}{sacado_violet!70}{p/1234}
} }{ \newpage}

\newenvironment{pageParcoursu}{\lhead{%
\pagecolor{sacado_green!6} 
\headerGeneral{PARCOURS 1}{sacado_green}{p/1234}
} }{ \newpage}


\newenvironment{pageParcoursd}{\lhead{%
\pagecolor{sacado_blue!6}
\headerGeneral{PARCOURS 2}{sacado_blue_light}{p/1234}
} }{ \newpage}

\newenvironment{pageParcourst}{\lhead{%
\pagecolor{sacado_red!6}
\headerGeneral{PARCOURS 3}{sacado_red}{p/1234}
} }{ \newpage}

\newenvironment{pageBrouillon}{\lhead{%
\pagecolor{sacado_gray!6}
\headerGeneral{BROUILLON}{sacado_gray}{p/1234}
} }{ \newpage}

\newenvironment{pageRituels}{\lhead{%
\pagecolor{fond!6}
\headerGeneral{RITUELS}{fond!70}{p/1234}
} }{ \newpage}

\newenvironment{pageAuto}{\lhead{%
\pagecolor{sacado_orange!6}
\headerGeneral{AUTOÉVALUATION}{sacado_orange}{p/1234}
} }{ \newpage}

\newenvironment{pageHistoire}{\lhead{%
\pagecolor{olive!6}
\headerGeneral{HISTOIRE}{olive}{p/1234}
} }{ \newpage}



\newenvironment{pageExercices}{\lhead{%
\pagecolor{white!100}
\headerGeneral{ACTIVITÉS}{fond}{p/1234}
}\begin{leftbar}}{\end{leftbar}\newpage}



\fancyfoot[L]{\colorbox{fond!70}{\color{texte}\thepage}}
\fancyfoot[C]{}


\newcommand{\titresec}[2]{\phantom{.}\begin{textblock}{1}[0,1](-1.24,0.25)\colorbox{fond!70}{%
\makebox[0.8cm]{\raisebox{0.05cm}[0.6cm][0.15cm]{\color{texte}\LARGE\bf #1}}}\end{textblock}{\LARGE\bf #2}\\\bigskip}

\renewcommand{\thesection}{\arabic{section}}
\titleformat{\section}{}{%
\hspace{-1.15cm}\colorbox{fond!70}{%
\makebox[0.8cm]{\raisebox{0.05cm}[0.6cm][0.15cm]{\color{texte}\LARGE\bf \thesection}}}}{1em}{\bf \LARGE #1}
  
\renewcommand{\thesubsection}{\arabic{subsection}}
            
\titleformat{\subsection}
{%\begin{textblock}{1}[0,1](-1,0.42) toto
  %\end{textblock}
%\reversemarginpar\marginnote[\rule{0.8cm}{0.8cm}]{}[0pt]  \color{red}\normalfont\Large\bfseries}
}{\hspace{-0.83em}
\colorbox{fond!70}{\makebox[0.6cm]{\raisebox{0cm}[1em][0.2em]\normalfont\large\bfseries\color{texte}\thesubsection}}}{1em}{\bf \large #1}




\makeatletter
\newenvironment{TraitV}[1]{%
% #1 couleur du trait (par défaut CouleurA)
% #2 largeur du trait
% #3 distance entre le trait et le texte
\def\FrameCommand{{\color{#1}\vrule width 2pt}
\hspace{1em}}\MakeFramed {\advance\hsize-\width}}%
{\endMakeFramed}
\makeatother

%----------------------------------------
%
%   Définitions des environnements de Définitions, propriétés...
%
%----------------------------------------

%%%%%%%%%%%%% Définitions
\newenvironment{Def}{%
\medskip \begin{tcolorbox}[widget,colback=sacado_violet!15,colframe=sacado_violet!75!black,
title= \stepcounter{cpt} Définition \thecpt. ]}{%
\end{tcolorbox}\par}


\newenvironment{DefT}[1]{%
\medskip \begin{tcolorbox}[widget,colback=sacado_violet!15,colframe=sacado_violet!75!black,
title= \stepcounter{cpt} Définition \thecpt : #1.]}
{%
\end{tcolorbox}\par}


%%%%%%%%%%%%% Proposition
\newenvironment{Prop}{%
\medskip \begin{tcolorbox}[widget,colback=sacado_blue!15,colframe=sacado_blue!75!black,
title= \stepcounter{cpt} Proposition \thecpt.]}
{%
\end{tcolorbox}\par}


%%%%%%%%%%%%% Propriétés
\newenvironment{Pp}{%
\medskip \begin{tcolorbox}[widget,colback=white!100,colframe=sacado_violet!75!black,
title= \stepcounter{cpt} Propriété \thecpt.]}
{%
\end{tcolorbox}\par}

\newenvironment{PpT}[1]{%
\medskip \begin{tcolorbox}[widget,colback=white!100,colframe=sacado_violet!75!black,
title= \stepcounter{cpt} Propriété \thecpt : #1. ]}
{%
\end{tcolorbox}\par}

\newenvironment{Pps}{%
\medskip \begin{tcolorbox}[widget,colback=white!100,colframe=sacado_violet!75!black,
title= \stepcounter{cpt} Propriétés \thecpt.]}
{%
\end{tcolorbox}\par}


%%%%%%%%%%%%% Conséquence
\newenvironment{Cq}{%
\medskip \begin{tcolorbox}[widget,colback=white,colframe=sacado_blue,
title= \stepcounter{cpt} Conséquence \thecpt.]}
{%
\end{tcolorbox}\par}



%%%%%%%%%%%%% Théorèmes
\newenvironment{ThT}[1]{% théorème avec titre
\medskip \begin{tcolorbox}[widget,colback=white!100,colframe=sacado_violet!75!black,
title= \stepcounter{cpt} Théorème \thecpt : #1.]}
{%
\end{tcolorbox}\par}

\newenvironment{Th}{%
\medskip \begin{tcolorbox}[widget,colback=white!100,colframe=sacado_violet!75!black,
title= \stepcounter{cpt} Théorème \thecpt.]}
{%
\end{tcolorbox}\par}


%%%%%%%%%%%%% Règles
\newenvironment{Reg}{%
\medskip \begin{tcolorbox}[widget,colback=sacado_blue!15,colframe=sacado_blue,
title= \stepcounter{cpt} Règle \thecpt.]}
{%
\end{tcolorbox}\par}

%%%%%%%%%%%%% Représentations
\newenvironment{Rep}{%
\medskip \begin{tcolorbox}[widget,colback=white,colframe=sacado_violet!75!white,
title= \stepcounter{cpt} Représentation \thecpt.]}
{%
\end{tcolorbox}\par}

 
%%%%%%%%%%%%% REMARQUES
\newenvironment{Rq}{%
\medskip \begin{tcolorbox}[widget,colback=sacado_orange!15,colframe=sacado_orange,
title= \stepcounter{cpt} Remarque \thecpt.]}
{%
\end{tcolorbox}\par}

\newenvironment{Rqs}{%
\medskip \begin{tcolorbox}[widget,colback=sacado_orange!15,colframe=sacado_orange,
title= \stepcounter{cpt} Remarques \thecpt.]}
{%
\end{tcolorbox}\par}


%%%%%%%%%%%%% EXEMPLES
\newenvironment{Ex}{%
\medskip \begin{tcolorbox}[widget,colback=white,colframe=sacado_blue_light,
title= \stepcounter{cpt} Exemple \thecpt.]}
{%
\end{tcolorbox}\par}

\newenvironment{Exs}{%
\medskip \begin{tcolorbox}[widget,colback=white!15,colframe=sacado_blue_light,
title= \stepcounter{cpt} Exemples \thecpt.]}
{%
\end{tcolorbox}\par}

 
\newenvironment{ExT}[1]{%
\medskip \begin{tcolorbox}[widget,colback=white,colframe=sacado_blue_light,
title= \stepcounter{cpt} Exemple \thecpt   : #1.]}
{%
\end{tcolorbox}\par}

 
\newenvironment{ExCor}{%
\medskip \begin{tcolorbox}[widget,colback=white,colframe=sacado_blue ,
title= \stepcounter{cpt} Exercice commenté \thecpt.]}
{%
\end{tcolorbox}\par}

\newenvironment{ExQr}[1]{%
\medskip \begin{tcolorbox}[widget,colback=white,colframe=sacado_blue_light ,
title= \stepcounter{cpt} Exemple  \thecpt. \hfill {\color{sacado_blue}https://sacado.xyz/a/#1} ]
\begin{minipage}{1.5cm}
\miniqr{#1}
\end{minipage}
\begin{minipage}{0.8\linewidth}
}
{%
\end{minipage}
\end{tcolorbox}\par}


\newenvironment{OuQr}[1]{%
\medskip \begin{tcolorbox}[widget,colback=white,colframe=sacado_orange ,
title= \stepcounter{cpt} Outil \thecpt. \hfill {\color{sacado_orange}https://sacado.xyz/a/#1} ]
\begin{minipage}{1.5cm}
\miniqr{#1}
\end{minipage}
\begin{minipage}{0.8\linewidth}
 \color{sacado_orange!90!black}
}
{%
 
\end{minipage}
\end{tcolorbox}\par}



\newenvironment{MeQr}[1]{%
\medskip \begin{tcolorbox}[widget,colback=white,colframe=sacado_blue,
title= \stepcounter{cpt} Méthode \thecpt. \hfill {\color{sacado_blue}https://sacado.xyz/a/#1} ]
\begin{minipage}{1.5cm}
\miniqr{#1}
\end{minipage}
\begin{minipage}{0.8\linewidth}
}
{%
\end{minipage}
\end{tcolorbox}\par}




%%%%%%%%%%%%% Logique
\newenvironment{Log}{%
\medskip \begin{tcolorbox}[widget,colback=sacado_blue!10,colframe=sacado_blue,
title= \stepcounter{cpt} Logique mathématique \thecpt.]}
{%
\end{tcolorbox}\par}
%%%%%%%%%%%%% Logique avec paramètre
\newenvironment{LogT}[1]{%
\medskip \begin{tcolorbox}[widget,colback=sacado_blue!10,colframe=sacado_blue,
title= \stepcounter{cpt} Logique mathématique \thecpt. #1]}
{%
\end{tcolorbox}\par}

%%%%%%%%%%%%% Preuve
\newenvironment{Pv}[1][]{%
\begin{tcolorbox}[breakable, enhanced,widget, colback=sacado_blue!10!white,boxrule=0pt,frame hidden,
borderline west={1mm}{0mm}{sacado_blue!75}]
\textbf{Preuve#1 : }}
{%
\end{tcolorbox}
\par}


%%%%%%%%%%%%% PreuveROC
\newenvironment{PvR}[1][]{%
\begin{tcolorbox}[breakable, enhanced,widget, colback=sacado_blue!10!white,boxrule=0pt,frame hidden,
borderline west={1mm}{0mm}{sacado_blue!75}]
\textbf{Preuve (ROC)#1 : }}
{%
\end{tcolorbox}
\par}


%%%%%%%%%%%%% DemoExigible
\newenvironment{DemoE}{%
\medskip \begin{tcolorbox}[widget,colback=sacado_blue!10,colframe=sacado_blue,
title= \stepcounter{cpt} Démonstration exigible \thecpt. ]}
{%
\end{tcolorbox}\par}





%%%%%%%%%%%%% Compétences
\newenvironment{Cps}[1][]{%
\vspace{0.4cm}
\begin{tcolorbox}[enhanced, lifted shadow={0mm}{0mm}{0mm}{0mm}%
{black!60!white}, attach boxed title to top left={xshift=5mm, yshift*=-3mm}, coltitle=white, colback=white, boxed title style={colback=sacado_green!100}, colframe=sacado_green!75!black,title=\textbf{Compétences associées#1}]}
{%
\end{tcolorbox}
\par}

%%%%%%%%%%%%% Chapitres connexes
\newenvironment{CCon}[1][]{%
\vspace{0.4cm}
\begin{tcolorbox}[breakable, enhanced,widget, colback=white ,boxrule=0pt,frame hidden,
borderline west={2mm}{0mm}{sacado_violet}]
\textbf{#1}}
{%
\end{tcolorbox}
\par}
%%%%%%%%%%%%% Compétences Collège
\newenvironment{CpsCol}[1][]{%
\vspace{0.4cm}
\begin{tcolorbox}[breakable, enhanced,widget, colback=white ,boxrule=0pt,frame hidden,
borderline west={2mm}{0mm}{sacado_violet}]
\textbf{#1}}
{%
\end{tcolorbox}
\par}


 

%%%%%%%%%%%%% Rituel
\newenvironment{Rit}{%
\medskip \begin{tcolorbox}[widget,colback=white!15,colframe=sacado_violet!75!black,
title= \stepcounter{cpt} Rituel \thecpt. ]}{%
\end{tcolorbox}\par}


%%%%%%%%%%%%% Méthode
\newenvironment{Mt}{%
\medskip \begin{tcolorbox}[widget,colback=white!15,colframe=sacado_violet!75!black,
title= \stepcounter{cpt} Méthode \thecpt. ]}{%
\end{tcolorbox}\par}

%%%%%%%%%%%%% Méthode
\newenvironment{MtT}[1]{%
\medskip \begin{tcolorbox}[widget,colback=white!15,colframe=sacado_violet!75!black,
title= \stepcounter{cpt} Méthode \thecpt. #1 ]}{%
\end{tcolorbox}\par}


%%%%%%%%%%%%% VocU
\newenvironment{VocU}[1]{%
\medskip \begin{tcolorbox}[widget,colback=white!15,colframe=sacado_violet!75,
title= \stepcounter{cpt} Vocabulaire \thecpt. #1 ]}{%
\end{tcolorbox}\par}


%%%%%%%%%%%%% Notation
\newenvironment{Nt}[1]{%
\medskip \begin{tcolorbox}[widget,colback=white!5,colframe=sacado_red!75,
title= \stepcounter{cptr} Notation \thecptr. #1 ]}{%
\end{tcolorbox}\par}

%%%%%%%%%%%%% Ety
\newenvironment{Ety}[1]{%
\medskip \begin{tcolorbox}[widget,colback=white!15,colframe=sacado_violet!75,
title= \stepcounter{cpt} Étymologie \thecpt. #1 ]}{%
\end{tcolorbox}\par}


%%%%%%%%%%%%% His
\newenvironment{His}[1]{%
\begin{tcolorbox}[right=5mm, enhanced, lifted shadow={0mm}{0mm}{0mm}{0mm}%
{sacado_green_dark!90!white}, attach boxed title to top left={xshift=0.3cm, yshift*=-2mm}, coltitle=sacado_green_dark, colback=sacado_green!10!white, boxed title style={colback=white}, colframe=sacado_green_dark,title= Les mathématiciennes et mathématiciens ]
}{%
\end{tcolorbox}\par}


%%%%%%%%%%%%% Attention
\newenvironment{Att}[1]{%
\medskip \begin{tcolorbox}[widget,colback=sacado_red!5,colframe=sacado_red!95!white,
title= \stepcounter{cpt} Notation \thecpt. #1 ]}{%
\end{tcolorbox}\par}



%%%%%%%%%%%%%%%%%%%%%%%%%%%%%%%%%%%%%%%%%%%%%%%%%%%%%%%%%%%%%%%%%%%%%%%%%%%%%%%%%%%%%%%%%%%%%%%%%%%%%%%%%%%%%%%%%%%%%%
%%%%%%%%%%%%%%%%%%%%%%%%%%%%%%%%%%%%%%%%%%%%%%%%%%%%%%%%%%%%%%%%%%%%%%%%%%%%%%%%%%%%%%%%%%%%%%%%%%%%%%%%%%%%%%%%%%%%%%
%%%%%%%%%%%%%%%%  Exercices                                            %%%%%%%%%%%%%%%%%%%%%%%%%%%%%%%%%%%%%%%%%%%%%%%
%%%%%%%%%%%%%%%%%%%%%%%%%%%%%%%%%%%%%%%%%%%%%%%%%%%%%%%%%%%%%%%%%%%%%%%%%%%%%%%%%%%%%%%%%%%%%%%%%%%%%%%%%%%%%%%%%%%%%%
%%%%%%%%%%%%%%%%%%%%%%%%%%%%%%%%%%%%%%%%%%%%%%%%%%%%%%%%%%%%%%%%%%%%%%%%%%%%%%%%%%%%%%%%%%%%%%%%%%%%%%%%%%%%%%%%%%%%%%
 
 
 
%%%%%%%%%%%%% ExoCad 7 paramètres : Compétences, qrcode , calculatrice, python, scratch, tableur, annales
\newenvironment{ExoCad}[7]{% code avant
\tcbset{top=-0.2cm }
\stepcounter{exo}

\begin{tcolorbox}[right=-5mm, enhanced, lifted shadow={0mm}{0mm}{0mm}{0mm}%
{black!60!white}, attach boxed title to top right={xshift=-0.3cm, yshift*=-2mm}, coltitle=sacado_violet!85!black, colback=white!100!white, boxed title style={colback=white}, colframe=sacado_violet!100!black,title= {\footnotesize  #1}  ]
 
\hspace{-1.3cm} 
\begin{minipage}[t]{0.7cm}

 \begin{tikzpicture}
 	\node[fill=sacado_violet,minimum width=0.7cm]{\textcolor{white}{\bf {\Large \theexo}}};
 \end{tikzpicture}


%%%%%%%%%%%%%%%%%%%%%%%% Condition pour la calculatrice
 \ifthenelse{\equal{#3}{1}}{
 \begin{tikzpicture}
 	\node[minimum width=0.7cm]{\includegraphics[scale=0.5]{MISC/calculator.png} };
 \end{tikzpicture} 
 }{
 \ifthenelse{\equal{#3}{2}}{
 \begin{tikzpicture}
 	\node[minimum width=0.7cm]{\includegraphics[scale=0.5]{MISC/no_calculator.png} };
 \end{tikzpicture} 
 }{}
 }

\end{minipage}
\hfill
\begin{minipage}[t]{17.3cm}
} 
{ 
\end{minipage}%code  après
\hfill
\begin{minipage}[t]{1cm}

\begin{center}
\colorbox{sacado_violet}{\includegraphics[height=1cm]{qrcodes/qrDummy.png}}
\colorbox{white}{ {\footnotesize /b/ABCD} }
\end{center}

\end{minipage}

\end{tcolorbox}
\par 
}

 
%%%%%%%%%%%%% ExoCu 7 paramètres : Compétences, qrcode , calculatrice, python, scratch, tableur, annales
\newenvironment{ExoCu}[7]{% code avant
\tcbset{top=-0.2cm }
\stepcounter{exo}

\begin{tcolorbox}[right=-5mm, enhanced, lifted shadow={0mm}{0mm}{0mm}{0mm}%
{black!60!white}, attach boxed title to top right={xshift=-0.3cm, yshift*=-2mm}, coltitle=sacado_green!85!black, colback=white!100!white, boxed title style={colback=white}, colframe=sacado_green!100!black,title= {\footnotesize  #1}  ]
 
\hspace{-1.3cm} 
\begin{minipage}[t]{0.7cm}

 \begin{tikzpicture}
 	\node[fill=sacado_green,minimum width=0.7cm]{\textcolor{white}{\bf {\Large \theexo}}};
 \end{tikzpicture}


%%%%%%%%%%%%%%%%%%%%%%%% Condition pour la calculatrice
 \ifthenelse{\equal{#3}{1}}{
 \begin{tikzpicture}
 	\node[minimum width=0.7cm]{\includegraphics[scale=0.5]{MISC/calculator.png} };
 \end{tikzpicture} 
 }{
 \ifthenelse{\equal{#3}{2}}{
 \begin{tikzpicture}
 	\node[minimum width=0.7cm]{\includegraphics[scale=0.5]{MISC/no_calculator.png} };
 \end{tikzpicture} 
 }{}
 }

\end{minipage}
\hfill
\begin{minipage}[t]{17.3cm}
} 
{ 
\end{minipage}%code  après
\hfill
\begin{minipage}[t]{1cm}

 
\begin{center}
\colorbox{sacado_green}{\includegraphics[height=1cm]{qrcodes/qrDummy.png}}
\colorbox{white}{ {\footnotesize /b/ABCD}  }
\end{center}
 

\end{minipage}

\end{tcolorbox}
\par
}  


 %%%%%%%%%%%%% ExoCd 7 paramètres : Compétences, qrcode , calculatrice, python, scratch, tableur, annales
\newenvironment{ExoCd}[7]{% code avant
\tcbset{top=-0.2cm }
\stepcounter{exo}

\begin{tcolorbox}[right=-5mm, enhanced, lifted shadow={0mm}{0mm}{0mm}{0mm}%
{black!60!white}, attach boxed title to top right={xshift=-0.3cm, yshift*=-2mm}, coltitle=sacado_blue!85!black, colback=white!100!white, boxed title style={colback=white}, colframe=sacado_blue!100!black,title= {\footnotesize  #1}  ]
 
\hspace{-1.3cm} 
\begin{minipage}[t]{0.7cm}

 \begin{tikzpicture}
 	\node[fill=sacado_blue,minimum width=0.7cm]{\textcolor{white}{\bf {\Large \theexo}}};
 \end{tikzpicture}


%%%%%%%%%%%%%%%%%%%%%%%% Condition pour la calculatrice
 \ifthenelse{\equal{#3}{1}}{
 \begin{tikzpicture}
 	\node[minimum width=0.7cm]{\includegraphics[scale=0.5]{MISC/calculator.png} };
 \end{tikzpicture} 
 }{
 \ifthenelse{\equal{#3}{2}}{
 \begin{tikzpicture}
 	\node[minimum width=0.7cm]{\includegraphics[scale=0.5]{MISC/no_calculator.png} };
 \end{tikzpicture} 
 }{}
 }

\end{minipage}
\hfill
\begin{minipage}[t]{17.3cm}
} 
{ 
\end{minipage}%code  après
\hfill
\begin{minipage}[t]{1cm}

\begin{center}
\colorbox{sacado_blue}{\includegraphics[height=1cm]{qrcodes/qrDummy.png}}
\colorbox{white}{ {\footnotesize /b/ABCD} }
\end{center}

\end{minipage}

\end{tcolorbox}
\par
}  


%%%%%%%%%%%%% ExoCt 7 paramètres : Compétences, qrcode , calculatrice, python, scratch, tableur, annales
\newenvironment{ExoCt}[7]{% code avant
\tcbset{top=-0.2cm }
\stepcounter{exo}

\begin{tcolorbox}[right=-5mm, enhanced, lifted shadow={0mm}{0mm}{0mm}{0mm}%
{black!60!white}, attach boxed title to top right={xshift=-0.3cm, yshift*=-2mm}, coltitle=sacado_red!85!black, colback=white!100!white, boxed title style={colback=white}, colframe=sacado_red!100!black,title= {\footnotesize  #1}  ]
 
\hspace{-1.3cm} 
\begin{minipage}[t]{0.7cm}

 \begin{tikzpicture}
 	\node[fill=sacado_red,minimum width=0.7cm]{\textcolor{white}{\bf {\Large \theexo}}};
 \end{tikzpicture}


%%%%%%%%%%%%%%%%%%%%%%%% Condition pour la calculatrice
 \ifthenelse{\equal{#3}{1}}{
 \begin{tikzpicture}
 	\node[minimum width=0.7cm]{\includegraphics[scale=0.5]{MISC/calculator.png} };
 \end{tikzpicture} 
 }{
 \ifthenelse{\equal{#3}{2}}{
 \begin{tikzpicture}
 	\node[minimum width=0.7cm]{\includegraphics[scale=0.5]{MISC/no_calculator.png} };
 \end{tikzpicture} 
 }{}
 }

\end{minipage}
\hfill
\begin{minipage}[t]{17.3cm}
} 
{ 
\end{minipage}%code  après
\hfill
\begin{minipage}[t]{1cm}

\begin{center}
\colorbox{sacado_red}{\includegraphics[height=1cm]{qrcodes/qrDummy.png}}
\colorbox{white}{ {\footnotesize /b/ABCD} }
\end{center}

\end{minipage}

\end{tcolorbox}
\par
}  

 
 
 

%%%%%%%%%%%%% ExoCu 7 paramètres : compétences , qrcode , calculatrice, python, scratch, tableur, annales
\newenvironment{ExoAuto}[7]{% code avant
\tcbset{top=-0.2cm }
\stepcounter{exo}

\begin{tcolorbox}[right=-5mm, enhanced, lifted shadow={0mm}{0mm}{0mm}{0mm}%
{black!60!white}, attach boxed title to top right={xshift=-0.3cm, yshift*=-2mm}, coltitle=sacado_orange!85!black, colback=white!100!white, boxed title style={colback=white}, colframe=sacado_orange!100!black,title= {\footnotesize  #1}  ]
 
\hspace{-1.3cm} 
\begin{minipage}[t]{0.7cm}

 \begin{tikzpicture}
 	\node[fill=sacado_orange,minimum width=0.7cm]{\textcolor{white}{\bf {\Large \theexo}}};
 \end{tikzpicture}


%%%%%%%%%%%%%%%%%%%%%%%% Condition pour la calculatrice
 \ifthenelse{\equal{#3}{1}}{
 \begin{tikzpicture}
 	\node[minimum width=0.7cm]{\includegraphics[scale=0.5]{MISC/calculator.png} };
 \end{tikzpicture} 
 }{
 \ifthenelse{\equal{#3}{2}}{
 \begin{tikzpicture}
 	\node[minimum width=0.7cm]{\includegraphics[scale=0.5]{MISC/no_calculator.png} };
 \end{tikzpicture} 
 }{}
 }

\end{minipage}
\hfill
\begin{minipage}[t]{17.3cm}
} 
{ 
\end{minipage}%code  après
\hfill
\begin{minipage}[t]{1cm}

\begin{center}
\colorbox{sacado_orange}{\includegraphics[height=1cm]{qrcodes/qrDummy.png}}
\colorbox{white}{  {\footnotesize /b/ABCD}  }
\end{center}

\end{minipage}

\end{tcolorbox}
\par
}  

 

%%%%%%%%%%%%% ExoDec 7 paramètres : compétences , qrcode , calculatrice, python, scratch, tableur, annales
 
\newenvironment{ExoDec}[6]{% code avant
\tcbset{top=-0.2cm }
\stepcounter{exo}

\begin{tcolorbox}[right=-5mm, enhanced, lifted shadow={0mm}{0mm}{0mm}{0mm}%
{black!60!white}, attach boxed title to top right={xshift=-0.3cm, yshift*=-2mm}, coltitle=sacado_violet!85!black, colback=white!100!white, boxed title style={colback=white}, colframe=sacado_violet!100!black,title= {\footnotesize  #1}  ]
 
\hspace{-1.3cm} 
\begin{minipage}[t]{1cm}

 \begin{tikzpicture}
 	\node[fill=sacado_violet,minimum width=0.7cm]{\textcolor{white}{\bf {\Large \theexo}}};
 \end{tikzpicture}


%%%%%%%%%%%%%%%%%%%%%%%% Condition pour la calculatrice
 \ifthenelse{\equal{#3}{1}}{
 \begin{tikzpicture}
 	\node[minimum width=0.7cm]{\includegraphics[scale=0.5]{MISC/calculator.png} };
 \end{tikzpicture} 
 }{
 \ifthenelse{\equal{#3}{2}}{
 \begin{tikzpicture}
 	\node[minimum width=0.7cm]{\includegraphics[scale=0.5]{MISC/no_calculator.png} };
 \end{tikzpicture} 
 }{}
 }

\end{minipage}
\begin{minipage}[t]{17.3cm}
} 
{ 
\end{minipage}
\end{tcolorbox}
\par
}  
%%%%%%%%%%%%%%%%%%%%%%%%%%%%%%%%%%%%%%%%%%%%%%%%%%%%%%%%%%%%%%%%%%%%%%%%%%%%%%%%%%%%%%%%%%%%%%%%%%%%%%%%%%%%%%%%%%%%%%
%%%%%%%%%%%%%%%%%%%%%%%%%%%%%%%%%%%%%%%%%%%%%%%%%%%%%%%%%%%%%%%%%%%%%%%%%%%%%%%%%%%%%%%%%%%%%%%%%%%%%%%%%%%%%%%%%%%%%%
%%%%%%%%%%%%%%%%  Exercices sans qrcode                                %%%%%%%%%%%%%%%%%%%%%%%%%%%%%%%%%%%%%%%%%%%%%%%
%%%%%%%%%%%%%%%%%%%%%%%%%%%%%%%%%%%%%%%%%%%%%%%%%%%%%%%%%%%%%%%%%%%%%%%%%%%%%%%%%%%%%%%%%%%%%%%%%%%%%%%%%%%%%%%%%%%%%%
%%%%%%%%%%%%%%%%%%%%%%%%%%%%%%%%%%%%%%%%%%%%%%%%%%%%%%%%%%%%%%%%%%%%%%%%%%%%%%%%%%%%%%%%%%%%%%%%%%%%%%%%%%%%%%%%%%%%%%
 
 
 
%%%%%%%%%%%%% ExoCad 7 paramètres : Compétences , calculatrice, python, scratch, tableur, annales
\newenvironment{ExoCadN}[6]{% code avant
\tcbset{top=-0.2cm }
\stepcounter{exo}

\begin{tcolorbox}[right=-5mm, enhanced, lifted shadow={0mm}{0mm}{0mm}{0mm}%
{black!60!white}, attach boxed title to top right={xshift=-0.3cm, yshift*=-2mm}, coltitle=sacado_violet!85!black, colback=white!100!white, boxed title style={colback=white}, colframe=sacado_violet!100!black,title= {\footnotesize  #1}  ]
 
\hspace{-1.3cm} 
\begin{minipage}[t]{1cm}

 \begin{tikzpicture}
 	\node[fill=sacado_violet,minimum width=0.7cm]{\textcolor{white}{\bf {\Large \theexo}}};
 \end{tikzpicture}


%%%%%%%%%%%%%%%%%%%%%%%% Condition pour la calculatrice
 \ifthenelse{\equal{#3}{1}}{
 \begin{tikzpicture}
 	\node[minimum width=0.7cm]{\includegraphics[scale=0.5]{MISC/calculator.png} };
 \end{tikzpicture} 
 }{
 \ifthenelse{\equal{#3}{2}}{
 \begin{tikzpicture}
 	\node[minimum width=0.7cm]{\includegraphics[scale=0.5]{MISC/no_calculator.png} };
 \end{tikzpicture} 
 }{}
 }

\end{minipage}
\begin{minipage}[t]{17.3cm}
} 
{ 
\end{minipage}%code  après
\end{tcolorbox}
\par 
}

 
%%%%%%%%%%%%% ExoCu 7 paramètres : Compétences , calculatrice, python, scratch, tableur, annales
\newenvironment{ExoCuN}[6]{% code avant
\tcbset{top=-0.2cm }
\stepcounter{exo}

\begin{tcolorbox}[right=-5mm, enhanced, lifted shadow={0mm}{0mm}{0mm}{0mm}%
{black!60!white}, attach boxed title to top right={xshift=-0.3cm, yshift*=-2mm}, coltitle=sacado_green!85!black, colback=white!100!white, boxed title style={colback=white}, colframe=sacado_green!100!black,title= {\footnotesize  #1}  ]
 
\hspace{-1.3cm} 
\begin{minipage}[t]{1cm}

 \begin{tikzpicture}
 	\node[fill=sacado_green,minimum width=0.7cm]{\textcolor{white}{\bf {\Large \theexo}}};
 \end{tikzpicture}


%%%%%%%%%%%%%%%%%%%%%%%% Condition pour la calculatrice
 \ifthenelse{\equal{#2}{1}}{
 \begin{tikzpicture}
 	\node[minimum width=0.7cm]{\includegraphics[scale=0.5]{MISC/calculator.png} };
 \end{tikzpicture} 
 }{
 \ifthenelse{\equal{#2}{2}}{
 \begin{tikzpicture}
 	\node[minimum width=0.7cm]{\includegraphics[scale=0.5]{MISC/no_calculator.png} };
 \end{tikzpicture} 
 }{}
 }

\end{minipage}
\begin{minipage}[t]{17.3cm}
} 
{ 
\end{minipage}
\end{tcolorbox}
\par
}  


 %%%%%%%%%%%%% ExoCd 6 paramètres : Compétences, calculatrice, python, scratch, tableur, annales
\newenvironment{ExoCdN}[6]{% code avant
\tcbset{top=-0.2cm }
\stepcounter{exo}

\begin{tcolorbox}[right=-5mm, enhanced, lifted shadow={0mm}{0mm}{0mm}{0mm}%
{black!60!white}, attach boxed title to top right={xshift=-0.3cm, yshift*=-2mm}, coltitle=sacado_blue!85!black, colback=white!100!white, boxed title style={colback=white}, colframe=sacado_blue!100!black,title= {\footnotesize  #1}  ]
 
\hspace{-1.3cm} 
\begin{minipage}[t]{1cm}

 \begin{tikzpicture}
 	\node[fill=sacado_blue,minimum width=0.7cm]{\textcolor{white}{\bf {\Large \theexo}}};
 \end{tikzpicture}


%%%%%%%%%%%%%%%%%%%%%%%% Condition pour la calculatrice
 \ifthenelse{\equal{#2}{1}}{
 \begin{tikzpicture}
 	\node[minimum width=0.7cm]{\includegraphics[scale=0.5]{MISC/calculator.png} };
 \end{tikzpicture} 
 }{
 \ifthenelse{\equal{#2}{2}}{
 \begin{tikzpicture}
 	\node[minimum width=0.7cm]{\includegraphics[scale=0.5]{MISC/no_calculator.png} };
 \end{tikzpicture} 
 }{}
 }

\end{minipage}
\begin{minipage}[t]{17.3cm}
} 
{ 
\end{minipage}%code  après
\end{tcolorbox}
\par
}  


%%%%%%%%%%%%% ExoCt 6 paramètres : Compétences,   calculatrice, python, scratch, tableur, annales
\newenvironment{ExoCtN}[6]{% code avant
\tcbset{top=-0.2cm }
\stepcounter{exo}

\begin{tcolorbox}[right=-5mm, enhanced, lifted shadow={0mm}{0mm}{0mm}{0mm}%
{black!60!white}, attach boxed title to top right={xshift=-0.3cm, yshift*=-2mm}, coltitle=sacado_red!85!black, colback=white!100!white, boxed title style={colback=white}, colframe=sacado_red!100!black,title= {\footnotesize  #1}  ]
 
\hspace{-1.3cm} 
\begin{minipage}[t]{1cm}

 \begin{tikzpicture}
 	\node[fill=sacado_red,minimum width=0.7cm]{\textcolor{white}{\bf {\Large \theexo}}};
 \end{tikzpicture}


%%%%%%%%%%%%%%%%%%%%%%%% Condition pour la calculatrice
 \ifthenelse{\equal{#2}{1}}{
 \begin{tikzpicture}
 	\node[minimum width=0.7cm]{\includegraphics[scale=0.5]{MISC/calculator.png} };
 \end{tikzpicture} 
 }{
 \ifthenelse{\equal{#2}{2}}{
 \begin{tikzpicture}
 	\node[minimum width=0.7cm]{\includegraphics[scale=0.5]{MISC/no_calculator.png} };
 \end{tikzpicture} 
 }{}
 }

\end{minipage}
\begin{minipage}[t]{17.3cm}
} 
{ 
\end{minipage}%code  après
\end{tcolorbox}
\par
}  

 
 
 

%%%%%%%%%%%%% ExoCu 7 paramètres : compétences , qrcode , calculatrice, python, scratch, tableur, annales
\newenvironment{ExoAutoN}[6]{% code avant
\tcbset{top=-0.2cm }
\stepcounter{exo}

\begin{tcolorbox}[right=5mm, enhanced, lifted shadow={0mm}{0mm}{0mm}{0mm}%
{black!60!white}, attach boxed title to top right={xshift=-0.3cm, yshift*=-2mm}, coltitle=sacado_orange!85!black, colback=white!100!white, boxed title style={colback=white}, colframe=sacado_orange!100!black,title= {\footnotesize  #1}  ]
 
\hspace{-1.4cm} 
\begin{minipage}[t]{0.7cm}

 \begin{tikzpicture}
 	\node[fill=sacado_orange,minimum width=0.7cm]{\textcolor{white}{\bf {\Large \theexo}}};
 \end{tikzpicture}


%%%%%%%%%%%%%%%%%%%%%%%% Condition pour la calculatrice
 \ifthenelse{\equal{#2}{1}}{
 \begin{tikzpicture}
 	\node[minimum width=0.7cm]{\includegraphics[scale=0.5]{MISC/calculator.png} };
 \end{tikzpicture} 
 }{
 \ifthenelse{\equal{#2}{2}}{
 \begin{tikzpicture}
 	\node[minimum width=0.7cm]{\includegraphics[scale=0.5]{MISC/no_calculator.png} };
 \end{tikzpicture} 
 }{}
 }

\end{minipage}
\hfill
\begin{minipage}[t]{17.3cm}
} 
{ 
\end{minipage}%code  après
\end{tcolorbox}
\par
}  
%%%%%%%%%%%%%%%%%%%%%%%%%%%%%%%%%%%%%%%%%%%%%%%%%%%%%%%%%%%%%%%%%%%%%%%%%%%%%%%%%%%%%%%%%%%%%%%%%%%%%%%%%%%%%%%%%%%%%%
%%%%%%%%%%%%%%%%%%%%%%%%%%%%%%%%%%%%%%%%%%%%%%%%%%%%%%%%%%%%%%%%%%%%%%%%%%%%%%%%%%%%%%%%%%%%%%%%%%%%%%%%%%%%%%%%%%%%%%
%%%%%%%%%%%%%%%%  Exercices   sans contours                            %%%%%%%%%%%%%%%%%%%%%%%%%%%%%%%%%%%%%%%%%%%%%%%
%%%%%%%%%%%%%%%%%%%%%%%%%%%%%%%%%%%%%%%%%%%%%%%%%%%%%%%%%%%%%%%%%%%%%%%%%%%%%%%%%%%%%%%%%%%%%%%%%%%%%%%%%%%%%%%%%%%%%%
%%%%%%%%%%%%%%%%%%%%%%%%%%%%%%%%%%%%%%%%%%%%%%%%%%%%%%%%%%%%%%%%%%%%%%%%%%%%%%%%%%%%%%%%%%%%%%%%%%%%%%%%%%%%%%%%%%%%%%


\newcommand{\Sf}[1]{ \vspace{0.1cm}
{\color{fond}{\Large \textbf{#1}}  } 
} 

\newcommand{\Sfe}[1]{ \vspace{0.1cm}
{\color{sacado_blue}{\Large \textbf{#1}}  } 
} 
% fin de la procédure



%%%%%%%%%%%%% Pointillés ou ligne
\newcommand{\point}[1]{\vspace{0.1cm}\multido{}{#1}{ \dotfill \medskip \endgraf}}
\newcommand{\ligne}[1]{\vspace{0.1cm}\multido{}{#1}{ {\color{cqcqcq}\hrulefill} \medskip \endgraf}}
%----------------------------------------
%
%   Macros et opérateurs
%
%----------------------------------------

\newcommand{\second}{2\up{d}\xspace}
\newcommand{\seconde}{2\up{de}\xspace}
\newcommand{\R}{\mathbb R}
\newcommand{\Rp}{\R_+}
\newcommand{\Rpe}{\R_+^*}
\newcommand{\Rm}{\R_-}
\newcommand{\Rme}{\R_-^*}
\newcommand{\N}{\mathbb N}
\newcommand{\D}{\mathbb D}
\newcommand{\Q}{\mathbb Q}
\newcommand{\Z}{\mathbb Z}
\newcommand{\C}{\mathbb C}
\newcommand{\grs}{\mathfrak S}
\newcommand{\IN}[1]{\llbracket 1,#1\rrbracket}
\newcommand{\card}{\text{Card}\,}
\usepackage{mathrsfs}
\newcommand{\parties}{\mathscr P}
\renewcommand{\epsilon}{\varepsilon}
\newcommand{\rmd}{\text{d}}
\newcommand{\diff}{\mathrm D}
\newcommand{\Id}{{\rm Id}}
\newcommand{\e}{{\rm e}}
\newcommand{\I}{{\rm i}}
\newcommand{\J}{{\rm j}}
\newcommand{\ro}{\circ}
\newcommand{\exu}{\exists\,!\,}
\newcommand{\telq}{\,\, \mid \,\,}
\newcommand{\para}{\raisebox{0.1em}{\text{\footnotesize /\hspace{-0.1em}/}}}   
\newcommand{\vect}[1]{\overrightarrow{#1}}
\newcommand{\scal}[2]{\left(\, #1 \mid #2 \, \right)}
\newcommand{\ortho}[1]{{#1}^\perp}
\newcommand{\veci}{\vec{\text{\it \i}}}
\newcommand{\vecj}{\vec{\text{\it \j}}}
\newcommand{\rep}{$(O;\veci,\vecj,\vec{k})$\xspace}
\newcommand{\Oijk}{$(O, \veci,\vecj,\vec{k})$\xspace}
\newcommand{\rond}{repère orthonormal direct}
\newcommand{\bond}{base orthonormale directe}
\newcommand{\eq}{\Longleftrightarrow}
\newcommand{\implique}{\Longrightarrow}
\newcommand{\noneq}{\ \ \ /\hspace{-1.45em}\eq}
\newcommand{\tend}{\longrightarrow}
\newcommand{\egx}[2]{\underset{#1 \tend #2}=}
\newcommand{\asso}{\longmapsto}
\newcommand{\vers}{\longrightarrow}
\newcommand{\eqn}{~\underset{n \rightarrow \infty}{\sim}~}
\newcommand{\eqx}[2]{~\underset{#1 \rightarrow #2}{\sim}~}

\newcommand{\egn}{~\underset{n \rightarrow \infty}{=}~}
\renewcommand{\descriptionlabel}{\hspace{\labelsep}$\bullet$}
\renewcommand{\bar}{\overline}
\DeclareMathOperator{\ash}{{Argsh}}
\DeclareMathOperator{\cotan}{{cotan}}
\DeclareMathOperator{\ach}{{Argch}}
\DeclareMathOperator{\ath}{{Argth}}
\DeclareMathOperator{\sh}{{sh}}
\DeclareMathOperator{\ch}{{ch}}
\DeclareMathOperator{\Mat}{{Mat}}
\DeclareMathOperator{\Vect}{{Vect}}
\DeclareMathOperator{\trace}{{tr}}
\newcommand{\tr}{{}^{\mathrm t}}
\newcommand{\divi}{~\big|~}
\newcommand{\ndivi}{~\not{\big|}~}
\newcommand{\et}{\wedge}
\newcommand{\ou}{\vee}
\renewcommand{\det}{\operatorname{\text{dét}}}
\DeclareMathOperator{\grad}{{grad}}
\renewcommand{\arcsin}{{\mathop{\mathrm{Arcsin}}}}
\renewcommand{\arccos}{{\mathop{\mathrm{Arccos}}}}
\renewcommand{\arctan}{{\mathop{\mathrm{Arctan}}}}
\renewcommand{\tanh}{{\mathop{\mathrm{th}}}}
\newcommand{\pgcd}{\mathop{\mathrm{pgcd}}}
\newcommand{\ppcm}{\mathop{\mathrm{ppcm}}}
\newcommand{\fonc}[4]{\left\{\begin{tabular}{ccc}$#1$ & $\vers$ & $#2$ \\
$#3$ & $\asso$ & $#4$ \end{tabular}\right.}
\renewcommand{\geq}{\geqslant}
\renewcommand{\leq}{\leqslant}
\renewcommand{\Re}{\text{\rm Re}}
\renewcommand{\Im}{\text{\rm Im}}
\renewcommand{\ker}{\mathop{\mathrm{Ker}}}
\newcommand{\Lin}{\mathcal L}
\newcommand{\GO}{\mathcal O}
\newcommand{\GSO}{\mathcal{SO}}
\newcommand{\GL}{\mathcal{GL}}
\renewcommand{\emptyset}{\varnothing}
%\newcommand{\arc}[1]{\overset{\frown}{#1}}
\newcommand{\rg}{\mathop{\mathrm{rg}}}
\newcommand{\ds}{\displaystyle}
\newcommand{\co}[3]{\begin{pmatrix}#1 \\ #2 \\ #3\end{pmatrix}}
\newcommand{\demi}{\frac 1 2}
\newcommand{\limi}[2]{\underset{#1 \rightarrow #2}\lim}

\title{Mathématiques 2nde  : le livre sacado}
\author{L'équipe SACADO}

\begin{document}


\chapter{Arithmétique}{2}{https://sacado.xyz/qcm/parcours_show_course/0/117129}


 \begin{CpsCol}
\textbf{Les savoir-faire du chapitre}
 \begin{itemize}
\item Modéliser et résoudre des problèmes mobilisant les notions de multiple, de diviseur, de nombre pair, de nombre impair, de nombre premier.
\item Présenter les résultats fractionnaires sous forme irréductible.
 \end{itemize}
 \end{CpsCol}

 \begin{His}
   
  \hfill{\em Dieu a fait les nombres entiers, tout le reste est l'œuvre de l'Homme}

  \hfill{Leopold Kronecker}

     \bigskip

     
\begin{minipage}[b]{0.6\textwidth}{L'arithmétique est la branche des mathématiques qui étudie
  les nombres entiers et les opérations $+$, $-$, $\times$, $\div$,...
  Comme le relève la citation en exergue, les concepts de l'arithmétique
  sont élémentaires et fondamentaux.

  Mais cela ne signifie pas que
  les problèmes de l'arithmétique sont simples. Par exemple,
  les nombres premiers sont source de nombreux problèmes
  non résolus à ce jour, en particulier l'{\em hypothèse de Riemann}
  qui porte sur la {\em fonction zêta} et qui est reliée à la
  répartition des nombres premiers.

  L'importance de l'arithmétique dépasse les mathématiques : la plupart des
  cryptosystèmes (algorithmes qui permettent la sécurité des
  communications sur internet) sont fondés sur l'arithmétique.
  }
\end{minipage}
\begin{minipage}[b]{0.4\textwidth}
\begin{center}    
  \includegraphics[width=5.5cm]{FIG/zeta7b.png}

  Une représentation de la fonction zêta de Riemann

\end{center}
\end{minipage}

\end{His}

 

\begin{ExoDec}{Chercher. Calculer.}{1234}{1}{0}{0}{0}

Pour fêter les 25 ans de sa boutique, un chocolatier souhaite offrir aux premiers clients de la journée une boîte contenant des truffes au chocolat.
Il a confectionné 300 truffes : 125 truffes parfumées au café et 175 truffes enrobées de noix de coco, et toutes les boîtes devront être identiques : elles
devront contenir le même nombre de truffes café, et le même nombre
de truffes noix de coco. Toutes les truffes devront être utilisées.

Combien, au minimum,  y aura-t-il de truffes de chaque sorte dans chaque boîte ?
 
\end{ExoDec}

\newpage

\begin{pageCours}


\section{Les entiers naturels et entiers relatifs}

\begin{DefLTQ}{Entiers naturels et relatifs}{}

\begin{enumerate}
\item On appelle \textbf{entiers naturels} les nombres : $0$ ; $1$ ; $2$ ; $3$ ; $\ldots$  Leur ensemble est noté $\N$.\index{Ensemble de nombres! Entiers naturels $\N$}, on a donc : $\N =  \lbrace 0 ; 1 ; 2 ; 3 \ldots \rbrace $
 
\item  On appelle \textbf{entiers relatifs} ou simplement \textbf{entiers} les nombres entiers naturels et leurs opposés. Leur ensemble est noté $\Z$ \index{Ensemble de nombres! Entiers $\Z$}(d'après le mot allemand Zahl qui signifie chiffre, nombre).
On a donc : $\Z = \lbrace \ldots -3 ; -2 ; -1 ; 0 ; 1 ; 2 ; 3  \cdots  \rbrace$
Parfois, on dit abusivement que les nombres entiers sont les nombres sans partie décimale.
\end{enumerate}
\end{DefLTQ}

 
\section{Multiples et diviseurs}

\begin{minipage}{0.5\linewidth}

\begin{DefTQ}{Multiple et diviseur}{a/3885}\index{Multiple!Nombres}

  Soit $d$ un nombre entier. Le nombre $m$ est dit \textbf{multiple} de $d$ s'il existe un entier $q \in \Z$ tel que $m=qd$. Dans ce cas, on dit aussi
  que $d$ est un \textbf{diviseur} de $m$. Le nombre $q$ est donc aussi un diviseur de $m$.
\end{DefTQ}
\end{minipage}
\begin{minipage}{0.5\linewidth}
\begin{ExTQ}{}{b/23} 
  $35=5 \times 7$ où $7 \in \Z$ donc $35$ est un multiple de $5$
  et $5$ est un diviseur de $35$. Comme $5 \in \Z$, on peut aussi
  dire que $35$ est un multiple de $7$ et que $7$ est un diviseur de
  $35$.
  
  1 est un diviseur de tous les entiers, et tous les entiers
  divisent 0.
\end{ExTQ}
\end{minipage}


\begin{minipage}{0.5\linewidth}
\begin{DefTQ}{Nombres pairs et impairs}{a/3887}\index{Nombres pairs et impairs}

  Un \textbf{nombre pair} est un nombre entier divisible par 2, autrement
  dit un nombre entier $a$ est pair lorsqu'il existe $n \in \Z$
  tel que $a=2n$.
 % . \\  Soit $n$ un nombre pair, $n=2\times k$ avec $k\in\Z$.

  Un entier $a$ est un \textbf{nombre impair} lorsqu'il existe $n \in \Z$
  tel que $a=2n+1$.
\end{DefTQ}
\end{minipage}
\begin{minipage}{0.5\linewidth}
\begin{Exs} 
$46 = 2 \times 23$ et $23\in\Z$ donc $46$ est un nombre pair.

  $15= 2 \times 7,5$. Comme $7,5 \not\in\Z$, $15$ n'est pas pair. Par contre,
  $15=2 \times 7+1$ avec $7 \in \Z$, donc $15$ est impair.
\end{Exs}
\end{minipage}


%\begin{minipage}{0.5\linewidth}
%\begin{DefT}{Diviseur}\index{Diviseur!Nombres}%
%
%Soit $n$ un nombre entier. Le nombre $q$ est dit \textbf{diviseur} de $n$ s'il %existe un entier $k \in \Z$ tel que $n=kq$.
%\end{DefT}
%\end{minipage}
%\begin{minipage}{0.5\linewidth}
%\begin{Ex} 
%
%$48 = 6 \times 8$ où $6 \in \Z$  donc $8$ est un diviseur de $48$ et aussi,  $6%$ est diviseur de $48$.
%\end{Ex}
%\end{minipage}


\section{Nombres premiers}

\begin{minipage}{0.5\linewidth}
\begin{DefTQ}{Nombre premier}{a/3889}\index{Nombre premier}

Un \textbf{nombre premier} est un nombre entier naturel qui a exactement deux diviseurs positifs (qui sont alors 1 et lui-même) 

\end{DefTQ}
\end{minipage}
\begin{minipage}{0.5\linewidth}
\begin{Ex} 

  $19$ est un nombre premier :  il n'est divisible que par 1 et lui-même.

  $18$ n'est pas premier : il est divisible par 1 et 18, mais aussi par 2
  par exemple.
  
  $1$ n'est pas premier, car il n'a qu'un seul diviseur.  
\end{Ex}
\end{minipage}



\begin{minipage}{0.5\linewidth}
  \begin{ThTQ}{Décomposition en facteurs premiers}{a/3893}\label{theo:DFP}
    Tout entier naturel supérieur ou égal à 2 admet une décomposition en facteurs premiers. Cette décomposition est unique (à l'ordre des facteurs près).
    
  \end{ThTQ}
\end{minipage}
\begin{minipage}{0.5\linewidth}
\begin{Exs}
Les décompositions en facteurs premiers de $8$, $15$ et $19$ sont respectivement  
$8=2^3$ ; $15=3 \times 5$ ; $19=19$.
  
\end{Exs}
\end{minipage}




\section{Nombres premiers entre eux}

\begin{minipage}{0.5\linewidth}
  \begin{DefTQ}{Nombres premiers entre eux}{}\index{Premiers entre eux}

    Deux nombres entiers $a$ et $b$ sont \textbf{premiers entre eux}
    lorsque leur seul diviseur positif commun est $1$.
  \end{DefTQ}
\end{minipage}
\begin{minipage}{0.5\linewidth}
\begin{Ex}
  Les diviseurs positifs de $8$ sont $1$ ; $2$ ; $4$ et $8$. Ceux de $15$ sont
  $1$ ; $3$ ; $5$ et $15$. Le seul diviseur commun est $1$ ; donc \textbf{$8$ et $15$ sont premiers entre eux} (Pourtant, ils ne sont pas premiers)
\end{Ex}
\end{minipage}


\begin{minipage}{0.5\linewidth}
  \begin{DefTQ}{Fraction irréductible}{a/4956}\index{Fraction irréductible}
    Une fraction est \textbf{irréductible} lorsque son numérateur
    et son dénominateur sont premiers entre eux.
  \end{DefTQ}
\end{minipage}
\begin{minipage}{0.5\linewidth}
\begin{Ex}
  Les fractions $\dfrac{8}{15}$ et $\dfrac{15}8$ sont irréductibles, puisque
  $8$ et $15$ sont premiers entre eux.

  La fraction $\dfrac{10}{12}$ n'est pas irréductible : $10$ et $12$ ne sont
  pas premiers entre eux : ils admettent un facteur commun différent de 1, comme 2 par exemple.
\end{Ex}
\begin{Mt}
  Pour rendre irréductible une fraction, on peut décomposer en facteurs premiers
  son numérateur et son dénominateur, puis simplifier tous les facteurs
  communs.
\end{Mt}
\end{minipage}




\section{Logique}

\begin{minipage}{0.5\linewidth}
\begin{DefTQ}{Proposition universelle}{}\index{Proposition universelle}

  Une \textbf{proposition universelle} est une proposition qui porte sur tous
  les éléments d'un ensemble. 

\end{DefTQ}
\end{minipage}
\begin{minipage}{0.5\linewidth}
\begin{Ex} 
  Le théorème %\ref{theo:DFP}
  8 de décomposition en facteurs premiers est une proposition
  universelle (et elle est vraie).
  
%Quels que soient trois nombres $a$, $b$ et  $c$, $a(b+c)=ab+ac$. La distributivité est une proposition universelle.

\end{Ex}
\end{minipage}

\begin{minipage}{0.5\linewidth}
\begin{DefTQ}{Contre-exemple}{}\index{Contre-exemple}

Un \textbf{contre-exemple} est un cas particulier qui vient contredire une proposition universelle. 
\end{DefTQ}


\begin{Mt} 
  Pour démontrer qu'une proposition universelle est fausse, il suffit
  de trouver un contre-exemple. Par contre, démontrer qu'une proposition
  universelle est vraie nécessite une démonstration générale qui traite tous
  les cas.
\end{Mt}


%Considérons une proposition universelle : « Tous les nombres sont pairs ». Pour démontrer que cette proposition est fausse, il suffit de démontrer qu'un seul nombre n'est pas pair. La contradiction vient sur le mot Tous .

%$3=2\times1,5$ et $1,5 \not\in\mathbb Z$ donc $3$ n'est pas pair. Il existe des nombres impairs, la proposition universelle initiale est fausse. 

\end{minipage}
\begin{minipage}{0.5\linewidth}


\begin{Ex}
  La proposition universelle « tout entier naturel dont le dernier chiffre est
  7 est premier » est fausse, et pour le justifier, il suffit
  de dire que 27 est un contre-exemple.

  La proposition universelle « tout nombre premier supérieur à 3 est impair »
  est vraie, en voici une démonstration : soit $p$ un nombre
  premier supérieur à 3, alors les seuls diviseurs de $p$ sont 1 et $p$
  qui est différent de 2, donc $p$ n'a pas 2 comme diviseur, il est donc
  impair.
\end{Ex}
%Considérons une proposition universelle : « tous les nombres sont pairs ». Pour démontrer que cette proposition est fausse, il suffit de démontrer qu'un seul nombre n'est pas pair. La contradiction vient sur le mot \textbf{tous}.

%$3=2\times1,5$ et $1,5 \not\in\Z$ donc $3$ n'est pas pair. \textbf{Il existe
%  au moins un nombre impair}, la proposition universelle initiale est fausse. 

%\end{Mt}
\end{minipage}




\end{pageCours} 
\begin{pageAD} 
 

\Sf{Multiples et les diviseurs ; nombres premiers}

\begin{ExoCad}{Calculer.}{b/37}{0}{0}{0}{0}{0}
\begin{enumerate}
\item Déterminer les multiples de 4, compris entre 0 et 40 : \point{1}
\item Déterminer les multiples de 6, compris entre 0 et 40 : \point{1}
\item Déterminer tous les multiples communs 4 et de 6, compris entre 0 et 40 :\
\point{1}
\end{enumerate}

\end{ExoCad}



\begin{ExoCad}{Calculer.}{b/292}{0}{0}{0}{0}{0}
\begin{enumerate}
\item Donner tous les nombres premiers inférieurs à 20 \point{1}
\item $51$ est-il un nombre premier ? Justifier. 
\point{1}
\end{enumerate}
\end{ExoCad}


\begin{ExoCad}{Calculer.}{b/291}{0}{0}{0}{0}{0}

Décomposer 24 en produit de facteurs premiers.

\point{3}

\end{ExoCad}



\begin{ExoCad}{Calculer.}{b/290}{0}{0}{0}{0}{0}

Écrire la fraction $\dfrac{735}{840}$ sous forme irréductible : 

\point{3}

\end{ExoCad}

\Sf{Propriétés universelles, contre-exemples}

\begin{ExoCad}{Calculer, raisonner.}{b/293}{0}{0}{0}{0}{0}

Soit $a$ un entier. Démontrer que la somme de deux multiples de $a$ est un multiple de $a$. \point{3}

\end{ExoCad}


\begin{ExoCad}{Raisonner.}{b/295}{0}{0}{0}{0}{0}
Vrai ou faux : quel que soit l'entier naturel $n$, $2n+3$ est un nombre premier. Justifier.\point{2}
\end{ExoCad}


 
\begin{ExoCad}{Raisonner.}{b/296}{0}{0}{0}{0}{0}

Les propositions suivantes sont-elles vraies ou fausses ?
\begin{enumerate}[leftmargin=*]
\item La différence de deux nombres entiers naturels est un entier naturel. \point{1}
\item Le quotient de deux nombres décimaux non nuls est un nombre décimal. \point{1} 
\item Il existe deux nombres premiers distincts dont le quotient est un
  entier relatif. \point{1}
\item Il existe deux nombres premiers distincts dont le quotient est un
  nombre décimal. \point{1}
\end{enumerate} 
 
 \end{ExoCad}
 
\end{pageAD}


%%%%%%%%%%%%%%%%%%%%%%%%%%%%%%%%%%%%%%%%%%%%%%%%%%%%%%%%%%%%%%%%%%%
%%%%  Niveau 1
%%%%%%%%%%%%%%%%%%%%%%%%%%%%%%%%%%%%%%%%%%%%%%%%%%%%%%%%%%%%%%%%%%%
\begin{pageParcoursu} 



 
%%%%%%%%%%%%%%%%%%%%%%%%%%%
\begin{ExoCu}{Représenter. Raisonner.}{b/297}{2}{0}{0}{0}{0}

\begin{enumerate}
\item Décomposer $186$ et $155$ en produit de facteurs premiers. \point{3}
\item Déterminer le PGCD (plus grand diviseur commun)
  de $186$ et $155$. \point{2}
\item Un chocolatier a fabriqué $186$ pralines et $155$ chocolats
  qu'il répartit dans des colis. 
Les colis sont constitués ainsi :
\begin{description}
\item Le nombre de pralines est le même dans chaque colis.
\item Le nombre de chocolats est le même dans chaque colis.
\item Tous les chocolats et toutes les pralines sont utilisés.
\end{description}
\begin{enumerate}
\item Quel nombre maximal de colis pourra-t-il réaliser ?  \point{3}
\item Combien y aura-t-il de chocolats et de pralines dans chaque colis ?   \point{3}
\end{enumerate}
\end{enumerate}

\bidon{
$186=2 \times 3 \times 31$ et $155=5 \times 31$.
En ne conservant que les facteurs communs des décompositions de $186$ et $155$, on obtient que le pgcd de $186$ et $155$ est $31$.
Soit $n$ un nombre maximal de colis. Les $186$ chocolats sont équitablement répartis sur les $n$ colis, donc $n$ diviser $186$. De même, en considérant les pralines, on obtient que $n$ divise $155$. Donc $n$ est un diviseur commun de $186$ et $155$, et comme $n$ est le plus grand possible, on en déduit que $n$ est le pgcd de ces deux nombres, à savoir $31$. 
Il y a aura alors $\dfrac{186}{31}=6$ chocolats et $\dfrac{155}{31}=5$ pralines par colis.
}

\end{ExoCu}


\begin{ExoCu}{Raisonner.}{b/298}{1}{0}{0}{0}{0}
Simplifier la fraction $\dfrac{2310}{2730}$ pour la rendre irréductible. \point{2}
\end{ExoCu}

\bidon{
On décompose le numérateur et le dénominateur en facteurs premiers :
$2310=2 \times 3 \times 5 \times 7 \times 11$ et
$2730=2 \times 3 \times 5 \times 7 \times 13$.
On a donc $\dfrac{2310}{2730}=\dfrac{2 \times 3 \times 5 \times 7 \times 11}{
 2 \times 3 \times 5 \times 7 \times 13}=\dfrac{11}{13}$
}

\begin{ExoCu}{Raisonner.}{b/299}{1}{0}{0}{0}{0}
Démontrer que le produit de deux entiers consécutifs est pair. \point{2}
\end{ExoCu}

\bidon{
Si on appalle $n$ le premier des deux nombres consécutifs, le deuxième
est $n+1$, et le produit des deux entiers est $n(n+1)$.
\begin{itemize}
\item Si $n$ est pair,
il existe un entier $k$ tel que $n=2k$, alors $n(n+1)=2k(n+1)$ qui
et $k(n+1)$ est un entier, donc $n(n+1)$ est pair. 
\item Si $n$ est impair, alors il existe un entier $k$ tel que $n=2k+1$,
  donc $n(n+1)=n(2k+2)=2n(k+1)$ et $n(k+1)$ est un entier, donc
  $n(n+1)$ est aussi pair dans ce cas.
\end{itemize}
Dans tous les cas, $n(n+1)$ est pair.
}

\begin{ExoCu}{Chercher.}{b/300}{1}{0}{0}{0}{0}
  \begin{enumerate}
  \item Proposer deux entiers non premiers entre eux. \point{1}
  \item Proposer deux entiers non premiers, premiers entre eux.\point{1}
  \end{enumerate}
\end{ExoCu}

\bidon{
\begin{enumerate}
\item  On peut prendre par exemple 8 et 6, qui admette un facteur 2 en commun.
\item  On peut prendre par exemple 6 et 35, ou 14 et 15, ou 10 et 21, ....
\end{enumerate}
}

\begin{ExoCu}{Chercher. Raisonner.}{b/301}{1}{0}{0}{0}{0}
La somme de deux nombres premiers est-il un nombre premier ? Justifier. \point{2}
\end{ExoCu}

\bidon{
Non, par exemple $3$ et $5$ sont premiers,
mais leur somme $3+5=8$ n'est pas un nombre premier.
}

\end{pageParcoursu} 
 
%%%%%%%%%%%%%%%%%%%%%%%%%%%%%%%%%%%%%%%%%%%%%%%%%%%%%%%%%%%%%%%%%%%
%%%%  Niveau 2
%%%%%%%%%%%%%%%%%%%%%%%%%%%%%%%%%%%%%%%%%%%%%%%%%%%%%%%%%%%%%%%%%%%
\begin{pageParcoursd} 

 
\begin{ExoCd}{Représenter.}{b/302}{2}{0}{0}{0}{0}
On veut démontrer que la proposition universelle $\mathcal P$ suivante : « La somme de deux nombres impairs est un nombre pair » est vraie.

\begin{enumerate}
\item Calculer $a=5+7$. Peut-on en déduire que la proposition $P$
  est vraie ?
  \point{2}
\item Soient $n$ et $m$ deux nombres impairs. Il existe donc deux entiers
  relatifs $k$ et $q$ tels que $n=2k+1$ et $m=2q+1$.
  Calculer $n+m$ en fonction de $k$ et $q$. \point{1}
\item En déduire que la somme $n+m$ est un nombre pair. \point{1}
\end{enumerate}
\end{ExoCd}


\bidon{
\begin{enumerate}
\item On a $5+7=12$, on obtient bien que la somme des deux nombres impairs 5 et 7 est un
  nombre pair. Mais ce résultat n'est qu'un cas particulier, il y a d'autres nombres impairs
  dont on pourrait calculer la somme, et donc ce résultat ne prouve pas que la proposition
  est vraie.
\item On a $n+m=2k+1+2q+1=2k+2q+2=2(k+p+1)$,
\item $n+m=2(k+p+1)$ et $k+p+1$ est un entier, donc $n+m$ est pair.
\end{enumerate}
}

\bidon{
On veut démontrer que la proposition universelle $\mathcal P$ suivante : « La somme de deux nombres impairs est un nombre pair » est vraie.

Calculer $a=5+7$. Peut-on en déduire que la proposition $P$ est vraie ?
Soient $n$ et $m$ deux nombres impairs. Il existe donc deux entiers relatifs $k$ et $q$ tels que $n=2k+1$ et $m=2q+1$.
Calculer $n+m$ en fonction de $k$ et $q$.
En déduire que la somme $n+m$ est un nombre pair.


On a $5+7=12$, on obtient bien que la somme des deux nombres impairs 5 et 7 est un nombre pair. Mais ce résultat n'est qu'un cas particulier, il y a d'autres nombres impairs dont on pourrait calculer la somme, et donc ce résultat ne prouve pas que la proposition est vraie.
On a $n+m=2k+1+2q+1=2k+2q+2=2(k+p+1)$,
$n+m=2(k+p+1)$ et $k+p+1$ est un entier, donc $n+m$ est pair.
}

 
\begin{ExoCd}{Représenter.}{b/303}{1}{1}{0}{0}{0}

\begin{enumerate}
\item En utilisant les décompositions en facteurs premiers,
  démontrer que les entiers qui sont divisibles par 2 et 3
  sont divisibles par 6. \point{5}

\item Est-il vrai que tout entier divisible par 4 et 6 est
  divisible par 24 ? \point{1}
\end{enumerate}

 
\end{ExoCd}


\bidon{
  $p$ étant pair, il existe un entier $k$ tel que $p=2k$. Alors $p^2=(2k)^2=4k^2=2(2k^2)$,
  et $2k^2$ est pair, donc $p^2$ est pair.
}
  
 %%%%%%%%%%%%%%%%%%%%%%%%%%%%%%%%%%%%%%%%%%%%%%%%%%%%%%%%%%%%%%%%%%%
\begin{ExoCd}{Raisonner.}{b/304}{1}{0}{0}{0}{0}

Démontrer que tout nombre entier $n$ multiple de $9$ est un multiple de $3$.  \point{6}

\end{ExoCd}

\bidon{
  $n$ étant un multiple de 9, il existe un entier $k$ tel que $n=9k$.
Donc $n=3\times 3k$, or $3k$ est un entier, donc $n$ est divisible par 3.
}


\begin{ExoCd}{Raisonner.}{b/305}{1}{0}{0}{0}{0}
Montrer que la somme de trois entiers consécutifs est toujours un multiple de 3. \point{6}
\end{ExoCd}

\bidon{
  Soit $n$ le premier des trois nombres, les deux autres sont donc $n+1$ et $n+2$, et la somme des trois est $n+n+1+n+2=3n+3$. Or $3n+3=3(n+1)$ et $n+1$ est un entier, donc $3n+3$ est un multiple de 3.
}


\end{pageParcoursd}
 
%
%%%%%%%%%%%%%%%%%%%%%%%%%%%%%%%%%%%%%%%%%%%%%%%%%%%%%%%%%%%%%%%%%%%%
%%%%%  Niveau 3
%%%%%%%%%%%%%%%%%%%%%%%%%%%%%%%%%%%%%%%%%%%%%%%%%%%%%%%%%%%%%%%%%%%%
\begin{pageParcourst}




\bidon{
%% \begin{ExoCt}{Raisonner.}[1234}{1}{0}{0}{0}{0}
%% On donne le programme en Python ci dessous. 
 
%% \begin{lstlisting}[language=Python] 
%% def is_divisible(x,y):
%%     if x%y == 0 :
%%     	test = "{} est divisible par {}".format(x,y) 
%%     else :	
%%         test = "{} n'est pas divisible par {}".format(x,y)
%%     return test    
    
%% n=int(input("Entrer un nombre n :"))
    
%% print(is_divisible(n,4))
%% \end{lstlisting}
 
%% \begin{enumerate}
%% \item Que fait ce programme ? On pourra tester le programme avec l'éditeur : \url{https://sacado.xyz/tool/show/18}  \point{3}
%% \item Modifier le programme pour qu'il teste si un nombre $a$ divise $n$.   
%% \end{enumerate}
 
\end{ExoCt}
}



\begin{ExoCt}{Chercher.}{b/306}{2}{0}{0}{0}{0} 
Je suis un nombre à trois chiffres non nuls. Je suis divisible par 94. Changez l'ordre de mes chiffres d'une certaine manière, et je deviens divisible par 49.
Qui suis-je ?   \point{5}
\end{ExoCt}

\bidon{
On dresse la liste des multiples de 94 ayant trois chiffres non nuls :
188, 282, 376, 564, 658, 752, 846

De même pour les multiples de 49 :
147,196, 245, 294, 343, 392, 441, 539, 588, 637, 686, 735, 784, 833, 882, 931

On cherche alors le nombre de la première liste qui a les mêmes chiffres
qu'un nombre de la deuxième liste. On obtient une seule solution,
376, qui a les mêmes chiffres que 637. 
}
   %%%%%%%%%%%%%%%%%%%%%%%%%%%%%%%%%%%%%%%%%%%%%%%%%%%%%%%%%%%%%%%%%%%
\begin{ExoCt}{Raisonner.}{b/310}{2}{1}{0}{0}{0}
 
 
Dans un pays où le système fiduciaire (les pièces et les billets) n'est constitué que de pièces de 3 et de 5, il s'agit d'aider les habitants en créant un algorithme  qui donne le nombre minimal de pièces nécessaires à tout achat d'un montant entier supérieur ou égal à 8. 

Pour tester l'algorithme, on peut utiliser l'éditeur Python : \url{https://sacado.xyz/tool/show/18}

\hfill{{\footnotesize Source : d’après PISA, items libérés}}
 
\end{ExoCt}


\bidon{
# Pour utiliser au total le minimum de pièces, il faut utiiser le 
# maximum de pièces de 5 et le minimum de pièces de 3.
# En appelant p3 le nombre de pièces de 3 utilisées, on teste
# successivement p3=0, p3=1, p3=2,... et on s'arrête dès que #  ce qu'il reste à payer est divisible par 5.


n=int(input("Rentrer le montant à payer (un entier > 8) : "))
p3=0 #le nombre de pièces de 3 utilisées.
while (n-3*p3) % 5 != 0 : # il reste  payer n-3*p3, ce nombre est
                          #divisible par 5 le reste de la division euclidienne par 5 est nul
    p3=p3+1

p5=(n-p3)//5  #le nombre de pièce de 5
print("Le nombre minimum de pièces est", p3+p5, "réparties en", p3, "pièces de 3 et ",p5,"pièces de 5") 


Pour utiliser au total le minimum de pièces, il faut utiliser le 
maximum de pièces de 5 et le minimum de pièces de 3.
En appelant {\tt p3} le nombre de pièces de 3 utilisées, on fixe
successivement {\tt p3}=0, {\tt p3}=1, {\tt p3}=2,... et on s'arrête dès que 
 ce qu'il reste à payer est divisible par 5.

Voici le programme correspondant en python :
\begin{lstlisting}
n=int(input("Rentrer le montant à payer (un entier > 8) : "))
p3=0 #le nombre de pièces de 3 utilisées.
while (n-3*p3) % 5 != 0 : # il reste  payer n-3*p3, ce nombre est
                          #divisible par 5 le reste de la division euclidienne par 5 est nul
    p3=p3+1

p5=(n-p3)//5  #le nombre de pièce de 5
print("Le nombre minimum de pièces est", p3+p5, "réparties en", p3, "pièces de 3 et ",p5,"pièces de 5") 
\end{lstlisting}
}

%%%%%%%%%%%%%%%%%%%%%%%%%%%%%%%%%%%%%%%%%%%%%%%%%%%%%%%%%%%%%%%%%%%
 
\end{pageParcourst}



%
%%%%%%%%%%%%%%%%%%%%%%%%%%%%%%%%%%%%%%%%%%%%%%%%%%%%%%%%%%%%%%%%%%%%
%%%%%  Brouillon
%%%%%%%%%%%%%%%%%%%%%%%%%%%%%%%%%%%%%%%%%%%%%%%%%%%%%%%%%%%%%%%%%%%%


%%%%%%%%%%%%%%%%%%%%%%%%%%%%%%%%%%%%%%%%%%%%%%%%%%%%%%%%%%%%%%%%%%%
%%%%  Auto
%%%%%%%%%%%%%%%%%%%%%%%%%%%%%%%%%%%%%%%%%%%%%%%%%%%%%%%%%%%%%%%%%%%


%%%%%%%%%%%%%%%%%%%%%%%%%%%%%%%%%%%%%%%%%%%%%%%%%%%%%%%%%%%%%%%%%%%
\begin{pageAuto} 

 
%%%%%%%%%%%%%%%%%%%%%%%%%%%%%%%%%%%%%%%%%%%%%%%%%%%%%%%%%%%%%%%%%%%
\begin{ExoAuto}{Raisonner.}{b/307}{1}{0}{0}{0}{0}
Simplifier le nombre $a=\dfrac{60}{126}$ pour la rendre irréductible. \point{4}
\end{ExoAuto}
%%%%%%%%%%%%%%%%%%%%%%%%%%%%%%%%%%%%%%%%%%%%%%%%%%%%%%%%%%%%%%%%%%%
%%%%%%%%%%%%%%%%%%%%%%%%%%%%%%%%%%%%%%%%%%%%%%%%%%%%%%%%%%%%%%%%%%%
\begin{ExoAuto}{Raisonner.}{b/308}{1}{0}{0}{0}{0}
Simplifier le nombre $b=\dfrac{12a+4}{8}$. \point{2}
\end{ExoAuto}
%%%%%%%%%%%%%%%%%%%%%%%%%%%%%%%%%%%%%%%%%%%%%%%%%%%%%%%%%%%%%%%%%%%
\begin{ExoAuto}{Raisonner.}{b/314}{2}{0}{0}{0}{0}
Le produit de deux nombres impairs est-il impair ? \point{4}
\end{ExoAuto}


%%%%%%%%%%%%%%%%%%%%%%%%%%%%%%%%%%%%%%%%%%%%%%%%%%%%%%%%%%%%%%%%%%%
\begin{ExoAuto}{Raisonner.}{b/309}{2}{0}{0}{0}{0}
Soit $n$ un entier.

Démontrer que la différence de deux multiples de $n$ est un multiple de $n$. \point{6}
\end{ExoAuto}

%%%%%%%%%%%%%%%%%%%%%%%%%%%%%%%%%%%%%%%%%%%%%%%%%%%%%%%%%%%%%%%%%%%
\begin{ExoAuto}{Raisonner.}{b/311}{2}{0}{0}{0}{0}
  Pour déterminer le PGCD de deux entiers naturel $a$ et $b$ (avec $b \neq 0$),
  on effectue la division euclidienne de $a$ par $b$. On appelle $r_0$ le reste. \\
Puis on divise $b$ par $r_0$ et on appelle $r_1$ le reste. \\
On divise alors $r_0$ par $r_1$ et on appelle $r_2$ le reste.\\ 
On divise alors $r_1$ par $r_2$ et on appelle $r_3$ le reste. Et ainsi de suite,
jusqu'à obtenir un reste nul. 
Le PGCD de $a$ et de $b$ est alors le dernier reste non nul (ou $b$
si le premier reste est déjà nul).
On appelle ce procédé « la méthode par divisions successives » ou
« l'algorithme d'Euclide ». 

\begin{enumerate}[leftmargin=*]

\item Déterminer à l'aide de ce procédé le PGCD de $912$ et de $\np{1104}$. \point{3}
\item 
%\begin{enumerate}[leftmargin=*]
% \item Un tapissier achète $2 622$ clous tête plate et $2 530$ clous tête ronde pour la fabrication de fauteuils identiques : ch Après la fabrication, il ne lui reste plus aucun clous. Quel est le plus grand nombre de fauteuil que le tapissier peut réaliser ? 
% \point{3}
  %\item Dans ce cas, quelle sera le nombre de chaque type de clou par fauteuil ? \point{2}
  Un carreleur doit carreler une pièce rectangulaire de $912$cm par
  $\np{1104}$cm
  en utilisant des carreaux carrés. Le travail sera grandement facilité
  si :
  \begin{itemize}
  \item il n'a pas de découpe à faire : il disposera un nombre entier
    de carreaux sur la longueur et sur la largeur de la pièce ;
  \item il utilise le moins possible de carreaux, donc les carreaux
    sont les plus grands possibles.
    
  \end{itemize}
  \begin{enumerate}
  \item Quel est le côté $c$ des carreaux qui répond à ces deux contraintes ?
  \item Combien de carreaux seront disposés en longueur ? en largeur ?
    Combien de carreaux seront utilisés au total ?
  \end{enumerate}
\end{enumerate}
\end{ExoAuto}

\end{pageAuto}





\begin{pageAlgo}
\begin{ExoCt}{Raisonner.}{1234}{2}{1}{0}{0}{0}
% division par soustractions successives
\begin{enumerate}
\item On pose $a=87$ et $b=12$. Soustraire $b$ à $a$ autant de fois
  que possible, tant que le résultat reste positif. Combien
  de soustractions ont été faites ? Quelle est la dernière valeur obtenue ?
  \point{2}
\item Effectuer la division euclidienne de $87$ par $12$ ; comparer le quotient
  et le reste avec les deux valeurs obtenues à la question précédente.
  \point{3}
\item Écrire un programme python qui lit deux entiers naturels
  $a$ et $b$ (avec $b \neq 0$), qui soustrait $b$ à $a$ tant que
  le résultat est positif, et affiche le nombre de soustractions
  effectuées (que l'on note $q$) et la dernière valeur obtenue (que
  l'on note $r$).
  \point{7}
\item En python le quotient et le reste de la division euclidienne
  de {\tt a} par {\tt b} sont donnés par les expressions
  {\tt a // b} et {\tt a \% b} respectivement. Compléter le programme
  précédent pour qu'il affiche aussi le quotient et le reste
  de {\tt a} par {\tt b}. Constater sur quelques exemples que $q$ et $r$
  sont bien le quotient et le reste de {\tt a} par {\tt b}.
  \point{2}
%\item Justifier que les valeurs de $q$ et $r$ données
%  par le programme de la question 2 sont toujours le quotient et
%  le reste de la division euclidienne de {\tt a} par {\tt b}.
\end{enumerate}
\end{ExoCt}


%% \bidon{
%%   \begin{enumerate}
%%   \item On obtient $87-12=75$, $75-12=63$, $63-12=51$, $51-12=39$,
%%     $39-12=27$, $27-12=15$, $15-12=3$ et on s'arrête car
%%     la soustraction suivante, $3-12$, donnerait un nombre négatif.
%%     On a effectué $7$ divisions et la denière valeur est $3$.
%%  \item 
%% %   \begin{lstlisting}
%% def soustractions(a,b):
%%     q=0
%%     while a-b>=0 :
%%         a = a - b
%%         q = q + 1
%%     return q,a
%% a=int(input("Entrer l'entier naturel a : "))
%% b=int(input("Entrer l'entier naturel b : "))
%% q,r=soustractions(a,b)
%% %x   \end{lstlisting}
%% }


\begin{ExoCt}{Raisonner.}{b/312}{2}{1}{0}{0}{0}
 
 \begin{minipage}{0.5\linewidth} 
 
\textbf{Le crible d'Eratosthène}

L'algorithme procède par élimination : il s'agit de rayer d'une table d'entiers tous les multiples d'un entier $n$ (autres que lui-même), et d'entourer
tous les autres. 

En supprimant tous ces multiples, à la fin il ne restera que les entiers qui ne sont multiples d'aucun entier à part 1 et eux-mêmes, et qui sont donc les nombres premiers.

On commence par entourer deux, puis on raye tous
les multiples de 2 à partir de 4. On entoure alors le premier
nombre non rayé ni entouré, qui est 3, et on raye puis les multiples de 3
sauf 3. Puis on entoure le premier nombre non rayé ni entouré, qui est
5, et on raye tous les multiples de 5 sauf 5... On répète l'opération
jusqu'à ce que tous les entiers soient rayés ou entourés. 

\begin{enumerate}

\item Exécuter le crible sur la table ci-contre.
\item Quel est le résultat de ce crible ? \point{3}

\item Écrire un code en Python du crible d'Eratosthène : \url{https://sacado.xyz/tool/show/18}

\end{enumerate}

\end{minipage}
\begin{minipage}{0.5\linewidth}

 \begin{tabular}{|c|c|c|c|c|c|c|c|c|c|}
 \hline 
 &  & 2 & 3 & 4 & 5 & 6 & 7 & 8 & 9 \\ 
 \hline 
 10&11 & 12 & 13 & 14 & 15 & 16 & 17 & 18 & 19 \\
 \hline 
 20&21 & 22 & 23 & 24 & 25 & 26 & 27 & 28 & 29 \\
 \hline 
 30&31 & 32 & 33 & 34 & 35 & 36 & 37 & 38 & 39 \\
 \hline 
 40&41 & 42 & 43 & 44 & 45 & 46 & 47 & 48 & 49 \\
 \hline 
 50&51 & 52 & 53 & 54 & 55 & 56 & 57 & 58 & 59 \\
 \hline 
 60&61 & 62 & 63 & 64 & 65 & 66 & 67 & 68 & 69 \\
 \hline 
 70&71 & 72 & 73 & 74 & 75 & 76 & 77 & 78 & 79 \\
 \hline 
 80&81 & 82 & 83 & 84 & 85 & 86 & 87 & 88 & 89 \\
 \hline 
 90&91 & 92 & 93 & 94 & 95 & 96 & 97 & 98 & 99 \\
 \hline 
 \end{tabular}  
 
 \end{minipage}


 
\end{ExoCt}




\end{document}
