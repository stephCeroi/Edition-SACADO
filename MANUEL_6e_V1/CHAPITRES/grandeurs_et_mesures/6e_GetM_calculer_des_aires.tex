%\documentclass[a4paper,dvipsnames]{article}
%
%\addtolength{\hoffset}{-2.25cm}
%\addtolength{\textwidth}{4.5cm}
%\addtolength{\voffset}{-3.25cm}
%\addtolength{\textheight}{5cm}
%\setlength{\parskip}{0pt}
%\setlength{\parindent}{0in}
%
%%----------------------------------------------------------------------------------------
%	PACKAGES AND OTHER DOCUMENT CONFIGURATIONS
%----------------------------------------------------------------------------------------

%----------------------------------------------------------------------------------------
%		Generals
%----------------------------------------------------------------------------------------
\usepackage{fourier}
\usepackage{frcursive}
\usepackage[T1]{fontenc} %Accents handling
\usepackage[utf8]{inputenc} % Use UTF-8 encoding
%\usepackage{microtype} % Slightly tweak font spacing for aesthetics
\usepackage[english, francais]{babel} % Language hyphenation and typographical rules

%----------------------------------------------------------------------------------------
%		Graphics
%----------------------------------------------------------------------------------------
\usepackage{xcolor}
\usepackage{graphicx, multicol} % Enhanced support for graphics
\graphicspath{{FIG/}}
\usepackage{wrapfig}

%----------------------------------------------------------------------------------------
%		Other packages
%----------------------------------------------------------------------------------------
\usepackage{hyperref}
\hypersetup{
	colorlinks=true, %colorise les liens
	breaklinks=true, %permet le retour à la ligne dans les liens trop longs
	urlcolor= bleu3,  %couleur des hyperliens
	linkcolor= bleu3, %couleur des liens internes
	plainpages=false  %pour palier à "Bookmark problems can occur when you have duplicate page numbers, for example, if you have a page i and a page 1."
}
\usepackage{tabularx}
\newcolumntype{M}[1]{>{\arraybackslash}m{#1}} %Defines a scalable column type in tabular
\usepackage{booktabs} % Enhances quality of tables
\usepackage{diagbox} % barre en diagonale dans un tableau
\usepackage{multicol}
\usepackage[explicit]{titlesec}


%----------------------------------------------------------------------------------------
%		Headers and footers
%----------------------------------------------------------------------------------------
\usepackage{fancyhdr} % Headers and footers
\pagestyle{fancy} % All pages have headers and footers
\fancyhead{}\renewcommand{\headrulewidth}{0pt} % Blank out the default header
\renewcommand{\footrulewidth}{0pt}
\fancyfoot[L]{} % Custom footer text
\fancyfoot[C]{\href{https://sacado.xyz/}{sacado.xyz}} % Custom footer text
\fancyfoot[R]{\thepage} % Custom footer text

%----------------------------------------------------------------------------------------
%		Mathematics packages
%----------------------------------------------------------------------------------------
\usepackage{amsthm, amsmath, amssymb} % Mathematical typesetting
\usepackage{marvosym, wasysym} % More symbols
\usepackage[makeroom]{cancel}
\usepackage{xlop}
\usepackage{pgf,tikz,pgfplots}
\pgfplotsset{compat=1.15}
\usetikzlibrary{positioning}
%\usetikzlibrary{arrows}
\usepackage{pst-plot,pst-tree,pst-func, pstricks-add,pst-node,pst-text}
\usepackage{units}
\usepackage{nicefrac}
\usepackage[np]{numprint} %Séparation milliers dans un nombre

%----------------------------------------------------------------------------------------
%		New text commands
%----------------------------------------------------------------------------------------
\usepackage{calc}
\usepackage{boites}
 \renewcommand{\arraystretch}{1.6}

%%%%% Pour les imports.
\usepackage{import}

%%%%% Pour faire des boites
\usepackage[tikz]{bclogo}
\usepackage{bclogo}
\usepackage{framed}
\usepackage[skins]{tcolorbox}
\tcbuselibrary{breakable}
\tcbuselibrary{skins}
\usetikzlibrary{babel,arrows,shadows,decorations.pathmorphing,decorations.markings,patterns}

%%%%% Pour les symboles et les ensembles
\newcommand{\pp}{\leq}
\newcommand{\pg}{\geq}
%\newcommand{\euro}{\eurologo{}}
\newcommand{\R}{\mathbb{R}}
\newcommand{\N}{\mathbb{N}}
\newcommand{\D}{\mathbb{D}}
\newcommand{\Z}{\mathbb{Z}}
\newcommand{\Q}{\mathbb{Q}}
\newcommand{\C}{\mathbb{C}}

%%%%% Pour une double minipage
\newcommand{\mini}[2]{
\begin{minipage}[t]{0.48\linewidth}
#1
\end{minipage}
\hfill
\begin{minipage}[t]{0.48\linewidth}
#2
\end{minipage}
}


%\newcommand\hole[1]{\texttt{\_}}
%\newcommand{\PROP}[1]{\textbf{\underline{#1}}}
%\newcommand{\exercice}{\textcolor{OliveGreen}{Exercice : }}
%\newcommand{\correction}{\textcolor{BurntOrange}{Correction : }}
%\newcommand{\propriete}{\textbf{\underline{Propriété}} : }
%\newcommand{\prop}{\textbf{\underline{Propriété}} : }
%\newcommand{\vocabulaire}{\textbf{\underline{Vocabulaire}} : }
%\newcommand{\voca}{\textbf{\underline{Vocabulaire}} : }

\usepackage{enumitem}
\newlist{todolist}{itemize}{2} %Pour faire des QCM
\setlist[todolist]{label=$\square$} %Pour faire des QCM \begin{todolist} instead of itemize

%----------------------------------------------------------------------------------------
%		Définition de couleur pour geogebra
%----------------------------------------------------------------------------------------
\definecolor{zzttqq}{rgb}{0.6,0.2,0.} %rouge des polygones
\definecolor{qqqqff}{rgb}{0.,0.,1.}
\definecolor{xdxdff}{rgb}{0.49019607843137253,0.49019607843137253,1.}%bleu
\definecolor{qqwuqq}{rgb}{0.,0.39215686274509803,0.} %vert des angles
\definecolor{ffqqqq}{rgb}{1.,0.,0.} %rouge vif
\definecolor{uuuuuu}{rgb}{0.26666666666666666,0.26666666666666666,0.26666666666666666}
\definecolor{qqzzqq}{rgb}{0.,0.6,0.}
\definecolor{cqcqcq}{rgb}{0.7529411764705882,0.7529411764705882,0.7529411764705882} %gris
\definecolor{qqffqq}{rgb}{0.,1.,0.}
\definecolor{ffdxqq}{rgb}{1.,0.8431372549019608,0.}
\definecolor{ffffff}{rgb}{1.,1.,1.}
\definecolor{ududff}{rgb}{0.30196078431372547,0.30196078431372547,1.}

%-------------------------------------------------
%
%	EN TETE
%
%-------------------------------------------------

% Classe
\newcommand{\myClasse}   
{
    6e
}

% Discipline
\newcommand{\myDiscipline}   
{
    Mathématiques
}

% Parcours
\newcommand{\myParcours}
{
  Nombres et Calculs
}

%Titre de la séquence
\newcommand{\myTitle}
{
    \scshape\huge
\textcolor{sacado_purple}{
		Enchainements d'opérations
}
}

%----------------------------------------------------------------------------------------

%%----------------------------------------------------------------------------------------
%		Définition de couleur pour les boites
%----------------------------------------------------------------------------------------
\definecolor{bleu1}{rgb}{0.54,0.79,0.95} %% Bleu
\definecolor{sapgreen}{rgb}{0.4, 0.49, 0}
\definecolor{dvzfxr}{rgb}{0.7,0.4,0.}
\definecolor{beamer}{rgb}{0.5176470588235295,0.49019607843137253,0.32941176470588235} % couleur beamer
\definecolor{preuveRbeamer}{rgb}{0.8,0.4,0}
\definecolor{sectioncolor}{rgb}{0.24,0.21,0.44}
\definecolor{subsectioncolor}{rgb}{0.1,0.21,0.61}
\definecolor{subsubsectioncolor}{rgb}{0.1,0.21,0.61}
\definecolor{info}{rgb}{0.82,0.62,0}
\definecolor{bleu2}{rgb}{0.38,0.56,0.68}
\definecolor{bleu3}{rgb}{0.24,0.34,0.40}
\definecolor{bleu4}{rgb}{0.12,0.20,0.25}
\definecolor{vert}{rgb}{0.21,0.33,0}
\definecolor{vertS}{rgb}{0.05,0.6,0.42}
\definecolor{red}{rgb}{0.78,0,0}
\definecolor{color5}{rgb}{0,0.4,0.58}
\definecolor{eduscol4B}{rgb}{0.19,0.53,0.64}
\definecolor{eduscol4P}{rgb}{0.62,0.12,0.39}

%----------------------------------------------------------------------------------------
%		Définition de couleur pour les boites SACADO
%----------------------------------------------------------------------------------------
\definecolor{sacado_blue}{RGB}{0,129,159} %% Bleu Sacado
\definecolor{sacado_green}{RGB}{59, 157, 38} %% Vert Sacado
\definecolor{sacado_yellow}{RGB}{255,180,0} %% Jaune Sacado
\definecolor{sacado_purple}{RGB}{94,68,145} %% Violet foncé Sacado
\definecolor{sacado_violet}{RGB}{153,117,224} %% Violet clair Sacado
\definecolor{sacado_orange}{RGB}{249,168,100} %% Orange Sacado
\definecolor{ill_frame}{HTML}{F0F0F0}
\definecolor{ill_back}{HTML}{F7F7F7}
\definecolor{ill_title}{HTML}{AAAAAA}


 % Compteurs pour Théorème, Définition, Exemple, Remarque, .....
\newcounter{cpttheo}
\setcounter{cpttheo}{0}
\newcounter{cptdef}
\setcounter{cptdef}{0}
\newcounter{cptmth}
\setcounter{cptmth}{0}
\newcounter{cpttitre}
\setcounter{cpttitre}{0}
 % Exercices
\newcounter{cptapp}
\setcounter{cptapp}{0}
\newcounter{cptex}
\setcounter{cptex}{0}
\newcounter{cptsr}
\setcounter{cptsr}{0}
\newcounter{cpti}
\setcounter{cpti}{0}
\newcounter{cptcor}
\setcounter{cptcor}{0}




%%%%% Pour réinitialiser numéros des paragraphes après une nouvelle partie
\makeatletter
    \@addtoreset{paragraph}{part}
\makeatother


%%%% Titres et sections

\newlength\chapnumb
\setlength\chapnumb{3cm}


% \titleformat{\part}[block] {
 % \normalfont\sffamily\color{violet}}{}{0pt} {
   % \parbox[t]{\chapnumb}{\fontsize{120}{110}\selectfont\ding{110}}
   % \parbox[b]{\dimexpr\textwidth-\chapnumb\relax}{
       % \raggedleft
       % \hfill{{\color{bleu3}\fontsize{40}{30}\selectfont#1}}\\
       % \rule{0.99\textwidth-\chapnumb\relax}{0.4pt}
 % }
% }

% \titleformat{name=\part,numberless}[block]
% {\normalfont\sffamily\color{bleu3}}{}{0pt}
% {\parbox[b]{\chapnumb}{%
  % \mbox{}}%
 % \parbox[b]{\dimexpr\textwidth-\chapnumb\relax}{%
   % \raggedleft%
   % \hfill{{\color{bleu3}\fontsize{40}{30}\selectfont#1}}\\
   % \rule{0.99\textwidth-\chapnumb\relax}{0.4pt}
 % }
% }



% \titleformat{\chapter}[block] {
 % \normalfont\sffamily\color{violet}}{}{0pt} {
   % \parbox[t]{\chapnumb}{ 
     % \fontsize{120}{110}\selectfont\thechapter}
     % \parbox[b]{\textwidth-\chapnumb}{
       % \raggedleft
       % \hfill{{\color{bleu3}\huge#1}}\\  
  % \ifthenelse{\thechapter<10}{\rule{0.99\textwidth-\chapnumb}{0.4pt}}{\rule{0.9\textwidth - \chapnumb}{0.4pt}}
       % \setcounter{cpttitre}{0}
	% \setcounter{cptapp}{0}
	% \setcounter{cptex}{0}
	% \setcounter{cptsr}{0}
	% \setcounter{cpti}{0}
	% \setcounter{cptcor}{0} 
 % }
% }

% \titleformat{name=\chapter,numberless}[block]
% {\normalfont\sffamily\color{bleu3}}{}{0pt}
% {\parbox[b]{\chapnumb}{%
  % \mbox{}}%
 % \parbox[b]{\textwidth-\chapnumb}{%
   % \raggedleft
   % \hfill{{\color{bleu3}\huge#1}}\\
   % \ifthenelse{\thechapter<10}{\rule{0.99\textwidth-\chapnumb}{0.4pt}}{ \rule{0.9\textwidth - \chapnumb}{0.4pt}}
       % \setcounter{cpttitre}{0}
	% \setcounter{cptapp}{0}
	% \setcounter{cptex}{0}
	% \setcounter{cptsr}{0}
	% \setcounter{cpti}{0}
	% \setcounter{cptcor}{0} 
 % }
% }
%
%       
%
%%%%% Personnalisation des numéros des sections
\renewcommand\thesection{\Roman{section}. }
\renewcommand\thesubsection{\hspace{1cm}\arabic{subsection}. }
\renewcommand\thesubsubsection{\hspace{2cm}\alph{subsubsection}. }

\titleformat{\section}[hang]{\color{sacado_purple}{}\normalfont\filright\huge}{}{0.4em}{\textbf{\thesection  #1}}   
% \titlespacing*{\section}{0.2pt}{0ex plus 0ex minus 0ex}{0.3em}
   
\titleformat{\subsection}[hang]{\color{sacado_purple}{}\normalfont\filright\Large}{}{0.4em}{\thesubsection
 #1}            
\titleformat{\subsubsection}[hang]{\color{sacado_purple}{}\normalfont\filright\large}{}{0.4em}{\thesubsubsection
 #1}
\titleformat{\paragraph}[hang]{\color{black}{}\normalfont\filright\normalsize}{}{0.4em}{#1}



%%%%%%%%%%%%%%%%%%%%% Cycle 4
%\newcommand{\Titre}[2]{\section*{#1 
%\ifthenelse{\equal{#2}{1}}   {\hfill{ \ding{182}  \ding{173} \ding{174} } \addcontentsline{toc}{section}{#1 \ding{182}} }%
%{%
%\ifthenelse{\equal{#2}{2}}{\hfill{ \ding{172}  \ding{183} \ding{174} } \addcontentsline{toc}{section}{#1 {\color{purple}\ding{183}}} }{%           
%\hfill{ \ding{172}  \ding{173} \ding{184} } \addcontentsline{toc}{section}{#1 {\color{orange}\ding{184}}}% 
%}%
%}%
%}
%}


%%%%%%%%%%%%%%%%%%%%% Cycle 4
\newcommand{\Titre}[2]{\section*{#1 
\ifthenelse{\equal{#2}{1}}   {\hfill{ \ding{182}  \ding{173} \ding{174} } \addcontentsline{toc}{section}{#1 \, \ding{182}} }%
{% sinon
\ifthenelse{\equal{#2}{1,5}}   {\hfill{ \ding{182}  \ding{183} \ding{174} } \addcontentsline{toc}{section}{#1 \, \ding{182} {\color{purple}\ding{183}}} }%
{% sinon
\ifthenelse{\equal{#2}{2}}   {\hfill{ \ding{172}  \ding{183} \ding{174} } \addcontentsline{toc}{section}{#1 \, {\color{purple}\ding{183}}} }
{% sinon
\ifthenelse{\equal{#2}{2,5}}   {\hfill{ \ding{172}  \ding{183} \ding{184} } \addcontentsline{toc}{section}{#1 \, {\color{purple}\ding{183}}  {\color{orange}\ding{184}}} }%
{% sinon
\hfill{ \ding{172}  \ding{173} \ding{184} } \addcontentsline{toc}{section}{#1 \,{\color{orange}\ding{184}}}% 
}%
}%
}%
}%
}%
}

%%%%%%%%%%%%% Titre
\newenvironment{titre}[2][]{%
\vspace{0.5cm}
\begin{tcolorbox}[enhanced, lifted shadow={0mm}{0mm}{0mm}{0mm}%
{black!60!white}, attach boxed title to top left={xshift=110mm, yshift*=-3mm}, coltitle=violet, colback=bleu3!25!white, boxed title style={colback=white!100}, colframe=bleu3,title=\stepcounter{cpttitre} \textbf{Fiche \thecpttitre}. #1 #2 ]}
{%
\end{tcolorbox}
\par}



%%%%%%%%%%%%% Définitions
\newenvironment{Def}[1][]{%
\medskip \begin{tcolorbox}[widget,colback=sacado_violet!0,colframe=sacado_violet!75,
adjusted title= \stepcounter{cptdef} Définition \thecptdef . {#1} ]}
{%
\end{tcolorbox}\par}


\newenvironment{DefT}[2][]{%
\medskip \begin{tcolorbox}[widget,colback=sacado_violet!0,colframe=sacado_violet!75,
adjusted title= \stepcounter{cptdef} Définition \thecptdef . {#1} \textit{#2}]}
{%
\end{tcolorbox}\par}

%%%%%%%%%%%%% Proposition
\newenvironment{Prop}[1][]{%
\medskip \begin{tcolorbox}[widget,colback=sacado_blue!0,colframe=sacado_blue!75!black,
adjusted title= \stepcounter{cpttheo} Proposition \thecpttheo . {#1} ]}
{%
\end{tcolorbox}\par}

%%%%%%%%%%%%% Propriétés
\newenvironment{Pp}[1][]{%
\medskip \begin{tcolorbox}[widget,colback=sacado_blue!0,colframe=sacado_blue!75!black,
adjusted title= \stepcounter{cpttheo} Propriété \thecpttheo . {#1}]}
{%
\end{tcolorbox}\par}

\newenvironment{PpT}[2][]{%
\medskip \begin{tcolorbox}[widget,colback=sacado_blue!0,colframe=sacado_blue!75!black,
adjusted title= \stepcounter{cpttheo} Propriété \thecpttheo . {#1} #2]}
{%
\end{tcolorbox}\par}

\newenvironment{Pps}[1][]{%
\medskip \begin{tcolorbox}[widget,colback=sacado_blue!0,colframe=sacado_blue!75!black,
adjusted title= \stepcounter{cpttheo} Propriétés \thecpttheo . {#1}]}
{%
\end{tcolorbox}\par}

%%%%%%%%%%%%% Théorèmes
\newenvironment{ThT}[2][]{% théorème avec titre
\medskip \begin{tcolorbox}[widget,colback=sacado_blue!0,colframe=sacado_blue!75!black,
adjusted title= \stepcounter{cpttheo} Théorème \thecpttheo . {#1} #2]}
{%
\end{tcolorbox}\par}

\newenvironment{Th}[1][]{%
\medskip \begin{tcolorbox}[widget,colback=sacado_blue!0,colframe=sacado_blue!75!black,
adjusted title= \stepcounter{cpttheo} Théorème \thecpttheo . {#1}]}
{%
\end{tcolorbox}\par}

%%%%%%%%%%%%% Règles
\newenvironment{Reg}[1][]{%
\medskip \begin{tcolorbox}[widget,colback=sacado_blue!0,colframe=sacado_blue!75!black,
adjusted title= \stepcounter{cpttheo} Règle \thecpttheo . {#1}]}
{%
\end{tcolorbox}\par}

%%%%%%%%%%%%% REMARQUES
\newenvironment{Rq}[1][]{%
\begin{bclogo}[couleur=sacado_orange!0, arrondi =0.15, noborder=true, couleurBarre=sacado_orange, logo = \bcinfo ]{ 
{\color{info}\normalsize{Remarque#1}}}}
{%
\end{bclogo}
\par}


\newenvironment{Rqs}[1][]{%
\begin{bclogo}[couleur=sacado_orange!0, arrondi =0.15, noborder=true, couleurBarre=sacado_orange, logo = \bcinfo ]{ 
{\color{info}\normalsize{Remarques#1}}}}
{%
\end{bclogo}
\par}

%%%%%%%%%%%%% EXEMPLES
\newenvironment{Ex}[1][]{%
\begin{bclogo}[couleur=sacado_yellow!15, arrondi =0.15, noborder=true, couleurBarre=sacado_yellow, logo = \bclampe ]{ 
\normalsize{Exemple#1}}}
{%
\end{bclogo}
\par}




%%%%%%%%%%%%% Preuve
\newenvironment{Pv}[1][]{%
\begin{tcolorbox}[breakable, enhanced,widget, colback=sacado_blue!10!white,boxrule=0pt,frame hidden,
borderline west={1mm}{0mm}{sacado_blue!75}]
\textbf{Preuve#1 : }}
{%
\end{tcolorbox}
\par}


%%%%%%%%%%%%% PreuveROC
\newenvironment{PvR}[1][]{%
\begin{tcolorbox}[breakable, enhanced,widget, colback=sacado_blue!10!white,boxrule=0pt,frame hidden,
borderline west={1mm}{0mm}{sacado_blue!75}]
\textbf{Preuve (ROC)#1 : }}
{%
\end{tcolorbox}
\par}


%%%%%%%%%%%%% Compétences
\newenvironment{Cps}[1][]{%
\vspace{0.4cm}
\begin{tcolorbox}[enhanced, lifted shadow={0mm}{0mm}{0mm}{0mm}%
{black!60!white}, attach boxed title to top left={xshift=5mm, yshift*=-3mm}, coltitle=white, colback=white, boxed title style={colback=sacado_green!100}, colframe=sacado_green!75!black,title=\textbf{Compétences associées#1}]}
{%
\end{tcolorbox}
\par}

%%%%%%%%%%%%% Compétences Collège
\newenvironment{CpsCol}[1][]{%
\vspace{0.4cm}
\begin{tcolorbox}[breakable, enhanced,widget, colback=white ,boxrule=0pt,frame hidden,
borderline west={2mm}{0mm}{bleu3}]
\textbf{#1}}
{%
\end{tcolorbox}
\par}




%%%%%%%%%%%%% Attendus
\newenvironment{Ats}[1][]{%
\vspace{0.4cm}
\begin{tcolorbox}[enhanced, lifted shadow={0mm}{0mm}{0mm}{0mm}%
{black!60!white}, attach boxed title to top left={xshift=5mm, yshift*=-3mm}, coltitle=white, colback=white, boxed title style={colback=sacado_green!100}, colframe=sacado_green!75!black,title=\textbf{Attendus du chapitre#1}]}
{%
\end{tcolorbox}
\par}

%%%%%%%%%%%%% Méthode
\newenvironment{Mt}[1][]{%
\vspace{0.4cm}
\begin{bclogo}[couleur=sacado_blue!0, arrondi =0.15, noborder=true, couleurBarre=bleu3, logo = \bccrayon ]{ 
\normalsize{{\color{bleu3}Méthode #1}}}}
{%
\end{bclogo}
\par}


%%%%%%%%%%%%% Méthode en vidéo
\newcommand{\MtV}[2]{\vspace{0.4cm} \colorbox{sacado_blue!0}{\hspace{0.2 cm}\tikz\node[rounded corners=1pt,draw] {\color{red}$\blacktriangleright$}; \quad  \href{https://youtu.be/#1?rel=0}{\raisebox{0.8mm}{{\color{red}\textbf{Méthode en vidéo : #2}}}}}}


%%%%%%%%%%%%% A voir (AV) : Lien externe + vidéo non Youtube
\newcommand{\AV}[2]{\vspace{0.4cm} \colorbox{bleu1!0}{\hspace{0.2 cm}\tikz\node[rounded corners=1pt,draw] {\color{red}$\blacktriangleright$}; \quad  \href{#1}{\raisebox{0.8mm}{{\color{red}\textbf{#2}}}}}}


%%%%%%%%%%%%% Etymologie
\newenvironment{Ety}[1][]{%
\begin{bclogo}[couleur=sacado_green!0, arrondi =0.15, noborder=true, couleurBarre=sacado_green, logo = \bcplume ]{ 
\normalsize{{\color{sacado_green}Étymologie#1}}}}
{%
\end{bclogo}
\par}


%%%%%%%%%%%%% Notation
\newenvironment{Nt}[1][]{%
\begin{bclogo}[couleur=sacado_violet!0, arrondi =0.15, noborder=true, couleurBarre=sacado_violet!75, logo = \bccrayon ]{ 
\normalsize{{\color{violet!75}Notation#1}}}}
{%
\end{bclogo}
\par}
%%%%%%%%%%%%% Histoire
%\newenvironment{His}[1][]{%
%\begin{bclogo}[couleur=brown!30, arrondi =0.15, noborder=true, couleurBarre=brown, logo = \bcvaletcoeur ]{ 
%\normalsize{{\color{brown}Histoire des mathématiques#1}}}}
%{%
%\end{bclogo}
%\par}

\newenvironment{His}[1][]{%
\vspace{0.4cm}
\begin{tcolorbox}[enhanced, lifted shadow={0mm}{0mm}{0mm}{0mm}%
{brown!60!white}, attach boxed title to top left={xshift=5mm, yshift*=-3mm}, coltitle=white, colback=white, boxed title style={colback=brown!100}, colframe=brown!75!black,title=\textbf{Histoire des mathématiques#1}]}
{%
\end{tcolorbox}
\par}

%%%%%%%%%%%%% Attention
\newenvironment{Att}[1][]{%
\begin{bclogo}[couleur=red!0, arrondi =0.15, noborder=true, couleurBarre=red, logo = \bcattention ]{ 
\normalsize{{\color{red}Attention. #1}}}}
{%
\end{bclogo}
\par}


%%%%%%%%%%%%% Conséquence
\newenvironment{Cq}[1][]{%
\textbf{Conséquence #1}}
{%
\par}

%%%%%%%%%%%%% Vocabulaire
\newenvironment{Voc}[1][]{%
\setlength{\logowidth}{10pt}
%\begin{footnotesize}
\begin{bclogo}[ noborder , couleur=white, logo =\bcbook]{#1}}
{%
\end{bclogo}
%\end{footnotesize}
\par}


%%%%%%%%%%%%% Video
\newenvironment{Vid}[1][]{%
\setlength{\logowidth}{12pt}
\begin{bclogo}[ noborder , couleur=white,barre=none, logo =\bcoeil]{#1}}
{%
\end{bclogo}
\par}


%%%%%%%%%%%%% Syntaxe
\newenvironment{Syn}[1][]{%
\begin{bclogo}[couleur=violet!0, arrondi =0.15, noborder=true, couleurBarre=violet!75, logo = \bcicosaedre ]{ 
\normalsize{{\color{violet!75}Syntaxe#1}}}}
{%
\end{bclogo}
\par}

%%%%%%%%%%%%% Auto évaluation
\newenvironment{autoeval}[1][]{%
\vspace{0.4cm}
\begin{tcolorbox}[enhanced, lifted shadow={0mm}{0mm}{0mm}{0mm}%
{black!60!white}, attach boxed title to top left={xshift=5mm, yshift*=-3mm}, coltitle=white, colback=white, boxed title style={colback=sacado_green!100}, colframe=sacado_green!75!black,title=\textbf{J'évalue mes compétences#1}]}
{%
\end{tcolorbox}
\par}


\newenvironment{autotest}[1][]{%
\vspace{0.4cm}
\begin{tcolorbox}[enhanced, lifted shadow={0mm}{0mm}{0mm}{0mm}%
{red!60!white}, attach boxed title to top left={xshift=5mm, yshift*=-3mm}, coltitle=white, colback=white, boxed title style={colback=red!100}, colframe=red!75!black,title=\textbf{Pour faire le point #1}]}
{%
\end{tcolorbox}
\par}

\newenvironment{ExOApp}[1][]{% Exercice d'application direct
\vspace{0.4cm}
\begin{tcolorbox}[enhanced, lifted shadow={0mm}{0mm}{0mm}{0mm}%
{red!60!white}, attach boxed title to top left={xshift=5mm, yshift*=-3mm}, coltitle=white, colback=white, boxed title style={colback=sacado_green!100}, colframe=sacado_green!75!black,title=\textbf{Application #1}]}
{%
\end{tcolorbox}
\par}

\newenvironment{ExOInt}[1][]{% Exercice d'application direct
\vspace{0.4cm}
\begin{tcolorbox}[enhanced, lifted shadow={0mm}{0mm}{0mm}{0mm}%
{red!60!white}, attach boxed title to top left={xshift=5mm, yshift*=-3mm}, coltitle=white, colback=white, boxed title style={colback=sacado_green!50}, colframe=sacado_green!75!black,title=\textbf{Exercice #1}]}
{%
\end{tcolorbox}
\par}

%Illustrations
\newtcolorbox{Illqr}[1]{
  enhanced,
  colback=white,
  colframe=ill_frame,
  colbacktitle=ill_back,
  coltitle=ill_title,
  title=\textbf{Illustration},
  boxrule=1pt, % épaisseur du trait à 1pt
  center,
  overlay={
    \node[anchor=south east, inner sep=0pt,xshift=-1pt,yshift=2pt,fill=white] at (frame.south east) {\fancyqr[height=1cm]{#1}};
  },
  after=\par,
  before=\vspace{0.4cm},
}

\newtcolorbox{Ill}{
  enhanced,
  colback=white,
  colframe=ill_frame,
  colbacktitle=ill_back,
  coltitle=ill_title,
  title=\textbf{Illustration},
  boxrule=1pt, % épaisseur du trait à 1pt
  center,
  after=\par,
  before=\vspace{0.4cm},
}

%%%%%%%%%%%%%% Propriétés
%\newenvironment{Pp}[1][]{%
%\medskip \begin{tcolorbox}[widget,colback=sacado_blue!0,colframe=sacado_blue!75!black,
%adjusted title= \stepcounter{cpttheo} Propriété \thecpttheo . {#1}]}
%{%
%\end{tcolorbox}\par}

%%%%% Pour réinitialiser numéros des chapitres après une nouvelle partie
% \makeatletter
    % \@addtoreset{section}{part}
% \makeatother

% \newcommand{\EPC}[3]{ % Exercice par compétence de niveau 1
% \ifthenelse{\equal{#1}{1}}
% {%condition2 vraie
% \vspace{0.4cm}
% \stepcounter{cptex}
% \tikz\node[rounded corners=0pt,draw,fill=bleu2]{\color{white}\textbf{ \thecptex}}; \quad  {\color{bleu2}\textbf{#3}}
% \input{#2}
% }% fin condition2 vraie
% {%condition2 fausse
% \vspace{0.4cm}
% \stepcounter{cptex}
% \tikz\node[rounded corners=2pt,draw,fill=eduscol4P]{\color{white}\textbf{ \thecptex}}; \quad  {\color{eduscol4P} \textbf{En temps libre.} \textbf{ #3}} 
% \input{#2}
% }% fin condition2 fausse
% } % fin de la procédure
%
%\usepackage{hyperref}
%
%\begin{document}
%
%%-------------------------------
%%	TITLE SECTION
%%-------------------------------
%
%\fancyhead[C]{}
%\hrule\medskip % Upper rule
%\begin{minipage}{0.295\textwidth} 
%\raggedright
%Classe \myClasse \hfill\\
%\myDiscipline \hfill\\
%\myParcours \hfill\\
%\end{minipage}
%\begin{minipage}{0.4\textwidth} 
%\centering 
%\scshape\huge
%\textcolor{sacado_purple}{\myTitle} \\ 
%\normalsize 
%%\mySubTitle \\ 
%\end{minipage}
%\begin{minipage}{0.295\textwidth} 
%\raggedleft
%\href{https://sacado.xyz/}{\includegraphics[width=.2\linewidth]{sacadoA1.png}}
%%\myAnnee \hfill\\
%\end{minipage}
%\medskip \hrule
%\bigskip

%-------------------------------
%	CONTENTS
%-------------------------------

\chapter{Calculer des aires}
{https://sacado.xyz/qcm/parcours_show_course/0/117135}


\begin{pageCours} 

\section{Aire d'une figure}

\begin{Def}
L'\textbf{aire} d'une figure est la \textbf{mesure} de sa \textbf{surface intérieure} dans une \textbf{unité d'aire} donnée.
\end{Def}

\begin{Ex}
L'aire des \textcolor{zzttqq}{figures oranges} peut être exprimée comme somme de \textcolor{qqwuqq}{carrés verts}, entiers ou non, valant $1\,u$ (une unité).
\begin{center}
\begin{tikzpicture}[line cap=round,line join=round,>=triangle 45,x=1.0cm,y=1.0cm]
\draw [color=cqcqcq,, xstep=1.0cm,ystep=1.0cm] (-0.18164835357409106,-0.17948573521827438) grid (9.210839706161456,2.2136687567143394);
\clip(-0.18164835357409106,-0.17948573521827438) rectangle (9.210839706161456,2.2136687567143394);
\fill[line width=2.pt,color=qqwuqq,fill=qqwuqq,fill opacity=0.10000000149011612] (0.,0.) -- (1.,0.) -- (1.,1.) -- (0.,1.) -- cycle;
\fill[line width=2.pt,color=zzttqq,fill=zzttqq,fill opacity=0.10000000149011612] (2.,1.) -- (2.,0.) -- (4.,0.) -- (4.,1.) -- cycle;
\fill[line width=2.pt,color=zzttqq,fill=zzttqq,fill opacity=0.10000000149011612] (5.,0.) -- (5.,2.) -- (8.,2.) -- (9.,1.) -- (9.,0.) -- cycle;
\draw [line width=2.pt,color=qqwuqq] (0.,0.)-- (1.,0.);
\draw [line width=2.pt,color=qqwuqq] (1.,0.)-- (1.,1.);
\draw [line width=2.pt,color=qqwuqq] (1.,1.)-- (0.,1.);
\draw [line width=2.pt,color=qqwuqq] (0.,1.)-- (0.,0.);
\draw [line width=2.pt,color=zzttqq] (2.,1.)-- (2.,0.);
\draw [line width=2.pt,color=zzttqq] (2.,0.)-- (4.,0.);
\draw [line width=2.pt,color=zzttqq] (4.,0.)-- (4.,1.);
\draw [line width=2.pt,color=zzttqq] (4.,1.)-- (2.,1.);
\draw [line width=2.pt,color=zzttqq] (5.,0.)-- (5.,2.);
\draw [line width=2.pt,color=zzttqq] (5.,2.)-- (8.,2.);
\draw [line width=2.pt,color=zzttqq] (8.,2.)-- (9.,1.);
\draw [line width=2.pt,color=zzttqq] (9.,1.)-- (9.,0.);
\draw [line width=2.pt,color=zzttqq] (9.,0.)-- (5.,0.);
%\begin{scriptsize}
\draw[color=qqwuqq] (0.5,0.5) node {$1\,u$};
\draw[color=zzttqq] (3,0.5) node {$2\,u$};
\draw[color=zzttqq] (7,1) node {$7,5\,u$};
%\end{scriptsize}
\end{tikzpicture}
\end{center}
\end{Ex}

%\begin{ExOApp}[]
%Déterminer l'aire des figures suivantes en fonction de l'unité proposée :
%\begin{center}
%\begin{tikzpicture}[line cap=round,line join=round,>=triangle 45,x=1.0cm,y=1.0cm]
%\draw [color=cqcqcq,, xstep=1.0cm,ystep=1.0cm] (-0.2202476195730043,-0.19235215721791207) grid (12.234448876076325,4.246563432657098);
%\clip(-0.2202476195730043,-0.19235215721791207) rectangle (12.234448876076325,4.246563432657098);
%\fill[line width=2.pt,color=qqwuqq,fill=qqwuqq,fill opacity=0.10000000149011612] (0.,0.) -- (1.,0.) -- (1.,1.) -- (0.,1.) -- cycle;
%\fill[line width=2.pt,color=qqwuqq,fill=qqwuqq,fill opacity=0.10000000149011612] (2.,0.) -- (2.,4.) -- (9.,4.) -- cycle;
%\fill[line width=2.pt,color=qqwuqq,fill=qqwuqq,fill opacity=0.10000000149011612] (6.012866421999638,0.003580915971741372) -- (8.012866421999638,2.0035809159717415) -- (9.012866421999638,2.0035809159717415) -- (7.012866421999638,0.003580915971741372) -- cycle;
%\fill[line width=2.pt,color=qqwuqq,fill=qqwuqq,fill opacity=0.10000000149011612] (10.,3.) -- (11.,4.) -- (12.,3.) -- (12.,1.) -- (11.,0.) -- (10.,1.) -- cycle;
%\draw [line width=2.pt,color=qqwuqq] (0.,0.)-- (1.,0.);
%\draw [line width=2.pt,color=qqwuqq] (1.,0.)-- (1.,1.);
%\draw [line width=2.pt,color=qqwuqq] (1.,1.)-- (0.,1.);
%\draw [line width=2.pt,color=qqwuqq] (0.,1.)-- (0.,0.);
%\draw [line width=2.pt,color=qqwuqq] (2.,0.)-- (2.,4.);
%\draw [line width=2.pt,color=qqwuqq] (2.,4.)-- (9.,4.);
%\draw [line width=2.pt,color=qqwuqq] (9.,4.)-- (2.,0.);
%\draw [line width=2.pt,color=qqwuqq] (6.012866421999638,0.003580915971741372)-- (8.012866421999638,2.0035809159717415);
%\draw [line width=2.pt,color=qqwuqq] (8.012866421999638,2.0035809159717415)-- (9.012866421999638,2.0035809159717415);
%\draw [line width=2.pt,color=qqwuqq] (9.012866421999638,2.0035809159717415)-- (7.012866421999638,0.003580915971741372);
%\draw [line width=2.pt,color=qqwuqq] (7.012866421999638,0.003580915971741372)-- (6.012866421999638,0.003580915971741372);
%\draw [line width=2.pt,color=qqwuqq] (10.,3.)-- (11.,4.);
%\draw [line width=2.pt,color=qqwuqq] (11.,4.)-- (12.,3.);
%\draw [line width=2.pt,color=qqwuqq] (12.,3.)-- (12.,1.);
%\draw [line width=2.pt,color=qqwuqq] (12.,1.)-- (11.,0.);
%\draw [line width=2.pt,color=qqwuqq] (11.,0.)-- (10.,1.);
%\draw [line width=2.pt,color=qqwuqq] (10.,1.)-- (10.,3.);
%%\begin{scriptsize}
%\draw[color=qqwuqq] (0.5,0.5) node {$unité$};
%\draw[color=qqwuqq] (4,2.5) node {$Fig.\,1$};
%\draw[color=qqwuqq] (7.5,1) node {$Fig.\,2$};
%\draw[color=qqwuqq] (11,2) node {$Fig.\,3$};
%%\end{scriptsize}
%\end{tikzpicture}
%\end{center}
%\end{ExOApp}

\section{Encadrer une aire}

\fancyqr{https://sacado.xyz/qcm/parcours_show_course/0/117135}


\section{Périmètre et aire}

\begin{Rqs}
\begin{itemize}
\item Deux figures ayant la même aire n'ont pas nécessairement le même périmètre.
\item Deux figures ayant le même périmètre n'ont pas nécessairement la même aire.
\end{itemize}
\end{Rqs}

\begin{Ex}
\[\textcolor{qqwuqq}{\mathcal{P}_1=20\,u.l.}\hspace{.5cm}\textcolor{qqwuqq}{\mathcal{A}_1=14\,u.a.}\hspace{1cm}\textcolor{qqqqff}{\mathcal{P}_2=20\,u.l.}\hspace{.5cm}\textcolor{qqqqff}{\mathcal{A}_2=14\,u.a.}\]
\begin{center}
\begin{tikzpicture}[line cap=round,line join=round,>=triangle 45,x=.9cm,y=.9cm]
\draw [color=cqcqcq,, xstep=.9cm,ystep=.9cm] (-3.02,-4.96) grid (11.,4.06);
\clip(-3.02,-4.96) rectangle (11.,4.06);
\fill[line width=2.pt,color=qqwuqq,fill=qqwuqq,fill opacity=0.10000000149011612] (1.,-1.) -- (-2.,-1.) -- (-2.,0.) -- (-1.,0.) -- (-1.,1.) -- (-2.,1.) -- (-2.,2.) -- (-1.,2.) -- (-1.,3.) -- (1.,3.) -- (1.,2.) -- (3.,2.) -- (3.,0.) -- (1.,0.) -- cycle;
\fill[line width=2.pt,color=qqqqff,fill=qqqqff,fill opacity=0.10000000149011612] (5.,-1.) -- (5.,0.) -- (7.,0.) -- (7.,2.) -- (9.,2.) -- (9.,0.) -- (10.,0.) -- (10.,-2.) -- (9.,-2.) -- (9.,-3.) -- (7.,-3.) -- (7.,-1.) -- cycle;
\fill[line width=2.pt,color=ffqqff,fill=ffqqff,fill opacity=1.0] (-1.,-3.) -- (0.,-3.) -- (0.,-4.) -- (-1.,-4.) -- cycle;
\draw [line width=2.pt,color=qqwuqq] (1.,-1.)-- (-2.,-1.);
\draw [line width=2.pt,color=qqwuqq] (-2.,-1.)-- (-2.,0.);
\draw [line width=2.pt,color=qqwuqq] (-2.,0.)-- (-1.,0.);
\draw [line width=2.pt,color=qqwuqq] (-1.,0.)-- (-1.,1.);
\draw [line width=2.pt,color=qqwuqq] (-1.,1.)-- (-2.,1.);
\draw [line width=2.pt,color=qqwuqq] (-2.,1.)-- (-2.,2.);
\draw [line width=2.pt,color=qqwuqq] (-2.,2.)-- (-1.,2.);
\draw [line width=2.pt,color=qqwuqq] (-1.,2.)-- (-1.,3.);
\draw [line width=2.pt,color=qqwuqq] (-1.,3.)-- (1.,3.);
\draw [line width=2.pt,color=qqwuqq] (1.,3.)-- (1.,2.);
\draw [line width=2.pt,color=qqwuqq] (1.,2.)-- (3.,2.);
\draw [line width=2.pt,color=qqwuqq] (3.,2.)-- (3.,0.);
\draw [line width=2.pt,color=qqwuqq] (3.,0.)-- (1.,0.);
\draw [line width=2.pt,color=qqwuqq] (1.,0.)-- (1.,-1.);
\draw [line width=2.pt,color=qqqqff] (5.,-1.)-- (5.,0.);
\draw [line width=2.pt,color=qqqqff] (5.,0.)-- (7.,0.);
\draw [line width=2.pt,color=qqqqff] (7.,0.)-- (7.,2.);
\draw [line width=2.pt,color=qqqqff] (7.,2.)-- (9.,2.);
\draw [line width=2.pt,color=qqqqff] (9.,2.)-- (9.,0.);
\draw [line width=2.pt,color=qqqqff] (9.,0.)-- (10.,0.);
\draw [line width=2.pt,color=qqqqff] (10.,0.)-- (10.,-2.);
\draw [line width=2.pt,color=qqqqff] (10.,-2.)-- (9.,-2.);
\draw [line width=2.pt,color=qqqqff] (9.,-2.)-- (9.,-3.);
\draw [line width=2.pt,color=qqqqff] (9.,-3.)-- (7.,-3.);
\draw [line width=2.pt,color=qqqqff] (7.,-3.)-- (7.,-1.);
\draw [line width=2.pt,color=qqqqff] (7.,-1.)-- (5.,-1.);
\draw [line width=3.6pt,color=ffqqff] (-1.,-2.)-- (0.,-2.);
\draw [line width=2.pt,color=ffqqff] (-1.,-3.)-- (0.,-3.);
\draw [line width=2.pt,color=ffqqff] (0.,-3.)-- (0.,-4.);
\draw [line width=2.pt,color=ffqqff] (0.,-4.)-- (-1.,-4.);
\draw [line width=2.pt,color=ffqqff] (-1.,-4.)-- (-1.,-3.);
\draw [color=ffqqff](-0.86,-2.32) node[anchor=north west] {1 u.l.};
\draw [color=ffqqff](-0.94,-4.28) node[anchor=north west] {1 u.a.};
\begin{scriptsize}
\draw[color=qqwuqq] (0.14,1.19) node {$Fig.1$};
\draw[color=qqqqff] (8.12,-0.49) node {$Fig.2$};
\end{scriptsize}
\end{tikzpicture}
\end{center}
\end{Ex}

\section{Unités d'aires usuelles}

\begin{Def}
L'unité de mesure des aires est le \textbf{mètre carré}, on le note $m^2$, c'est l'aire d'un carré de $1\,m$ de côté.

\begin{center}
\begin{tikzpicture}[line cap=round,line join=round,>=triangle 45,x=2.0cm,y=2.0cm]
\clip(-0.19063259065814545,-0.2571668280229305) rectangle (1.1623070595760396,1.1049452944162321);
\fill[line width=2.pt,color=zzttqq,fill=zzttqq,fill opacity=0.10000000149011612] (0.,0.) -- (1.,0.) -- (1.,1.) -- (0.,1.) -- cycle;
\draw [line width=1.pt,color=zzttqq] (0.,0.)-- (1.,0.);
\draw [line width=1.pt,color=zzttqq] (0.5,0.010319031230599716) -- (0.5,-0.010319031230599716);
\draw [line width=1.pt,color=zzttqq] (1.,0.)-- (1.,1.);
\draw [line width=1.pt,color=zzttqq] (0.9896809687694005,0.5) -- (1.0103190312305999,0.5);
\draw [line width=1.pt,color=zzttqq] (1.,1.)-- (0.,1.);
\draw [line width=1.pt,color=zzttqq] (0.5,0.9896809687694005) -- (0.5,1.0103190312305999);
\draw [line width=1.pt,color=zzttqq] (0.,1.)-- (0.,0.);
\draw [line width=1.pt,color=zzttqq] (0.010319031230599716,0.5) -- (-0.010319031230599716,0.5);
\begin{scriptsize}
\draw [fill=xdxdff] (0.,0.) circle (2.0pt);
\draw [fill=xdxdff] (1.,0.) circle (2.0pt);
\draw [fill=xdxdff] (1.,1.) circle (2.0pt);
\draw [fill=xdxdff] (0.,1.) circle (2.0pt);
\draw[color=zzttqq] (0.5,0.5) node {$1\,m^2$};
\draw[color=zzttqq] (0.5,-0.1) node {$1\,m$};
\end{scriptsize}
\end{tikzpicture}
\end{center}
\end{Def}

\begin{Rq}
De la même manière nous pouvons définir :
\begin{itemize}
\item $1\,dm^2$ est l'aire d'un carré de $1\,dm$ de côté.
\item $1\,cm^2$ est l'aire d'un carré de $1\,cm$ de côté.
\item $1\,mm^2$ est l'aire d'un carré de $1\,mm$ de côté.
\end{itemize}
\end{Rq}

\begin{Voc}
Pour mesurer la superficie des terrains, on utilise l'\textbf{are} ($a$) et l'\textbf{hectare} ($ha$) :
\begin{itemize}
\item $1\,a=1\,dam^2=100\,m^2$
\item $1\,ha=1\,hm^2=10\,000\,m^2$
\end{itemize}
\end{Voc}

\section{Convertir des unités d'aire}

\begin{Pp}
Dans un carré de $1 \,cm$ de côté, on peut construire $10\times10=100$ carrés de $1\, mm$ de côté.
Donc $1 \,cm^2 = 100 mm^2$

\begin{center}
\begin{tikzpicture}[line cap=round,line join=round,>=triangle 45,x=1.0cm,y=1.0cm]
\clip(-0.7008474435890936,-1.001194059753229) rectangle (12.772217041258656,10.391470761992991);
\fill[line width=2.pt,color=zzttqq,fill=zzttqq,fill opacity=0.10000000149011612] (0.,0.) -- (1.,0.) -- (1.,1.) -- (0.,1.) -- cycle;
\fill[line width=2.pt,color=zzttqq,fill=zzttqq,fill opacity=0.10000000149011612] (2.,0.) -- (3.,0.) -- (3.,1.) -- (2.,1.) -- cycle;
\draw [line width=2.pt,color=zzttqq] (0.,0.)-- (1.,0.);
%\draw [line width=1.pt,color=zzttqq] (0.5,0.08915998556149216) -- (0.5,-0.08915998556149216);
\draw [line width=2.pt,color=zzttqq] (1.,0.)-- (1.,1.);
%\draw [line width=1.pt,color=zzttqq] (0.910840014438507,0.5) -- (1.089159985561492,0.5);
\draw [line width=2.pt,color=zzttqq] (1.,1.)-- (0.,1.);
%\draw [line width=1.pt,color=zzttqq] (0.5,0.9108400144385084) -- (0.5,1.0891599855614926);
\draw [line width=2.pt,color=zzttqq] (0.,1.)-- (0.,0.);
%\draw [line width=1.pt,color=zzttqq] (0.08915998556149246,0.5) -- (-0.08915998556149246,0.5);
\draw [line width=.5pt,color=zzttqq] (3.,10.)-- (3.,0.);
\draw [line width=.5pt,color=zzttqq] (4.,10.)-- (4.,0.);
\draw [line width=.5pt,color=zzttqq] (5.,10.)-- (5.,0.);
\draw [line width=.5pt,color=zzttqq] (6.,10.)-- (6.,0.);
\draw [line width=.5pt,color=zzttqq] (7.,10.)-- (7.,0.);
\draw [line width=.5pt,color=zzttqq] (8.,10.)-- (8.,0.);
\draw [line width=.5pt,color=zzttqq] (9.,10.)-- (9.,0.);
\draw [line width=.5pt,color=zzttqq] (10.,10.)-- (10.,0.);
\draw [line width=.5pt,color=zzttqq] (11.,10.)-- (11.,0.);
\draw [line width=.5pt,color=zzttqq] (2.,1.)-- (12.,1.);
\draw [line width=.5pt,color=zzttqq] (2.,2.)-- (12.,2.);
\draw [line width=.5pt,color=zzttqq] (2.,3.)-- (12.,3.);
\draw [line width=.5pt,color=zzttqq] (2.,4.)-- (12.,4.);
\draw [line width=.5pt,color=zzttqq] (2.,5.)-- (12.,5.);
\draw [line width=.5pt,color=zzttqq] (2.,6.)-- (12.,6.);
\draw [line width=.5pt,color=zzttqq] (2.,7.)-- (12.,7.);
\draw [line width=.5pt,color=zzttqq] (2.,8.)-- (12.,8.);
\draw [line width=.5pt,color=zzttqq] (2.,9.)-- (12.,9.);
\draw [line width=2.pt,color=zzttqq] (2.,0.)-- (3.,0.);
\draw [line width=2.pt,color=zzttqq] (3.,0.)-- (3.,1.);
\draw [line width=2.pt,color=zzttqq] (3.,1.)-- (2.,1.);
\draw [line width=2.pt,color=zzttqq] (2.,1.)-- (2.,0.);
\draw [line width=2.pt,color=zzttqq] (2.,10.)-- (2.,0.);
\draw [line width=2.pt,color=zzttqq] (2.,0.)-- (12.,0.);
\draw [line width=2.pt,color=zzttqq] (12.,0.)-- (12.,10.);
\draw [line width=2.pt,color=zzttqq] (12.,10.)-- (2.,10.);
%\begin{scriptsize}
\draw [fill=xdxdff] (0.,0.) circle (1.5pt);
\draw [fill=xdxdff] (1.,0.) circle (1.5pt);
\draw [fill=xdxdff] (1.,1.) circle (1.5pt);
\draw [fill=xdxdff] (0.,1.) circle (1.5pt);
\draw[color=zzttqq] (0.5,0.5) node {$1\,cm^2$};
\draw[color=zzttqq] (0.5,-0.4) node {$1\,cm$};
\draw [fill=xdxdff] (2.,0.) circle (1.5pt);
\draw [fill=xdxdff] (12.,0.) circle (1.5pt);
\draw [fill=xdxdff] (12.,10.) circle (1.5pt);
\draw [fill=xdxdff] (2.,10.) circle (1.5pt);
\draw[color=zzttqq] (7,5) node {$1\,dm^2$};
\draw[color=zzttqq] (7,-0.4) node {$1\,dm$};
%\end{scriptsize}
\end{tikzpicture}
\end{center}
\end{Pp}

\begin{Mt}
Nous pouvons nous aider d'un tableau de conversion.

\begin{center}
\begin{tabular}{M{.5cm}|M{.5cm}||M{.5cm}|M{.5cm}||M{.5cm}|M{.5cm}||M{.5cm}|M{.5cm}||M{.5cm}|M{.5cm}||M{.5cm}|M{.5cm}||M{.5cm}|M{.5cm}}
\multicolumn{2}{c}{$km^2$} & \multicolumn{2}{c}{$hm^2$} & \multicolumn{2}{c}{$dam^2$} & \multicolumn{2}{c}{$m^2$} & \multicolumn{2}{c}{$dm^2$} & \multicolumn{2}{c}{$cm^2$} & \multicolumn{2}{c}{$mm^2$} \\\hline
 & & & & & & 2 & 5 & & & & & &  \\\hline
  & & & & & & 2 & 5 & 0 & 0 & & & & \\\hline
\end{tabular}
\end{center}

Ainsi : $25\,m^2=2\,500\,dm^2$.
\end{Mt}

\fancyqr{Lien vers le tableau de conversion en ligne}

%\begin{ExOApp}[]
%Effectuer les conversions suivantes :
%\[28\,m^2=...\,cm^2\hspace{.5cm}4,32\,dm^2=...\,m^2\hspace{.5cm}1\,cm^2=...\,mm^2\hspace{.5cm}3,3\,dm^2=...\,mm^2\hspace{.5cm}2,1\,dm^2=...\,dam^2\]
%\end{ExOApp}

\section{Calculs d'aire de surfaces particulières}

\subsection{Aire d'un carré}

\begin{Pp}
L'aire d'un carré est donnée par la formule :
\[\mathcal{A}_{carré}=côté\times côté\]
\[\mathcal{A}_{carré}=c\times c\]
\begin{center}
\begin{tikzpicture}[line cap=round,line join=round,>=triangle 45,x=0.5cm,y=0.5cm]
\clip(-0.5258160132035038,-0.9097206258865002) rectangle (6.482416380185364,6.303964834409533);
\fill[line width=2.pt,color=zzttqq,fill=zzttqq,fill opacity=0.10000000149011612] (0.,6.) -- (6.,6.) -- (6.,0.) -- (0.,0.) -- cycle;
\draw[line width=2.pt,color=zzttqq] (0.,5.515742477247899) -- (0.48425752275210066,5.515742477247899) -- (0.4842575227521006,6.) -- (0.,6.) -- cycle; 
\draw[line width=2.pt,color=zzttqq] (5.515742477247899,6.) -- (5.515742477247899,5.515742477247899) -- (6.,5.515742477247899) -- (6.,6.) -- cycle; 
\draw[line width=2.pt,color=zzttqq] (6.,0.4842575227521006) -- (5.515742477247899,0.48425752275210066) -- (5.515742477247899,0.) -- (6.,0.) -- cycle; 
\draw[line width=2.pt,color=zzttqq] (0.4842575227521006,0.) -- (0.48425752275210066,0.4842575227521006) -- (0.,0.4842575227521006) -- (0.,0.) -- cycle; 
\draw [line width=1.2pt,color=zzttqq] (0.,6.)-- (6.,6.);
\draw [line width=1.2pt,color=zzttqq] (2.954343762909519,6.114140592726204) -- (2.954343762909519,5.885859407273798);
\draw [line width=1.2pt,color=zzttqq] (3.045656237090481,6.114140592726204) -- (3.045656237090481,5.885859407273798);
\draw [line width=1.2pt,color=zzttqq] (6.,6.)-- (6.,0.);
\draw [line width=1.2pt,color=zzttqq] (6.114140592726202,3.0456562370904816) -- (5.885859407273796,3.0456562370904816);
\draw [line width=1.2pt,color=zzttqq] (6.114140592726202,2.9543437629095193) -- (5.885859407273796,2.9543437629095193);
\draw [line width=1.2pt,color=zzttqq] (6.,0.)-- (0.,0.);
\draw [line width=1.2pt,color=zzttqq] (3.045656237090481,-0.11414059272620306) -- (3.045656237090481,0.11414059272620306);
\draw [line width=1.2pt,color=zzttqq] (2.954343762909519,-0.11414059272620306) -- (2.954343762909519,0.11414059272620306);
\draw [line width=1.2pt,color=zzttqq] (0.,0.)-- (0.,6.);
\draw [line width=1.2pt,color=zzttqq] (-0.11414059272620307,2.9543437629095193) -- (0.11414059272620307,2.9543437629095193);
\draw [line width=1.2pt,color=zzttqq] (-0.11414059272620307,3.0456562370904816) -- (0.11414059272620307,3.0456562370904816);
%\begin{scriptsize}
\draw[color=zzttqq] (3,-0.7) node {$c$};
%\end{scriptsize}
\end{tikzpicture}
\end{center}
\end{Pp}

%\begin{ExOApp}[]
%\begin{enumerate}
%\item Un carré a un côté de $3\,cm$. Calculer son aire.
%\item Un carré a une aire de $64\,dm^2$. Calculer la longueur de son côté.
%\end{enumerate}
%\end{ExOApp}

\subsection{Aire d'un rectangle}

\begin{Pp}
L'aire d'un rectangle est donnée par la formule :
\[\mathcal{A}_{rectangle}=largeur\times Longueur\]
\[\mathcal{A}_{rectangle}=l\times L\]
\begin{center}
\begin{tikzpicture}[line cap=round,line join=round,>=triangle 45,x=0.3cm,y=0.3cm]
\clip(-1.3248001622869252,-1.2521424040651092) rectangle (11.55025869722878,6.92032403513103);
\fill[line width=2.pt,color=zzttqq,fill=zzttqq,fill opacity=0.10000000149011612] (0.,6.) -- (10.,6.) -- (10.,0.) -- (0.,0.) -- cycle;
\draw[line width=2.pt,color=zzttqq] (0.,5.515742477247899) -- (0.48425752275210066,5.515742477247899) -- (0.4842575227521006,6.) -- (0.,6.) -- cycle; 
\draw[line width=2.pt,color=zzttqq] (9.5157424772479,6.) -- (9.5157424772479,5.515742477247899) -- (10.,5.515742477247899) -- (10.,6.) -- cycle; 
\draw[line width=2.pt,color=zzttqq] (10.,0.4842575227521006) -- (9.5157424772479,0.48425752275210066) -- (9.5157424772479,0.) -- (10.,0.) -- cycle; 
\draw[line width=2.pt,color=zzttqq] (0.4842575227521006,0.) -- (0.48425752275210066,0.4842575227521006) -- (0.,0.4842575227521006) -- (0.,0.) -- cycle; 
\draw [line width=1.pt,color=zzttqq] (0.,6.)-- (10.,6.);
\draw [line width=1.pt,color=zzttqq] (5.,6.380468642420676) -- (5.,5.619531357579324);
\draw [line width=1.pt,color=zzttqq] (10.,6.)-- (10.,0.);
\draw [line width=1.pt,color=zzttqq] (10.114140592726201,3.0456562370904816) -- (9.885859407273795,3.0456562370904816);
\draw [line width=1.pt,color=zzttqq] (10.114140592726201,2.9543437629095193) -- (9.885859407273795,2.9543437629095193);
\draw [line width=1.pt,color=zzttqq] (10.,0.)-- (0.,0.);
\draw [line width=1.pt,color=zzttqq] (5.,-0.3804686424206773) -- (5.,0.3804686424206773);
\draw [line width=1.pt,color=zzttqq] (0.,0.)-- (0.,6.);
\draw [line width=1.pt,color=zzttqq] (-0.11414059272620307,2.9543437629095193) -- (0.11414059272620307,2.9543437629095193);
\draw [line width=1.pt,color=zzttqq] (-0.11414059272620307,3.0456562370904816) -- (0.11414059272620307,3.0456562370904816);
%\begin{scriptsize}
%\draw [fill=xdxdff] (0.,6.) circle (2.0pt);
%\draw[color=xdxdff] (-0.5029878946582631,6.475175723498838) node {$A$};
%\draw [fill=xdxdff] (10.,6.) circle (2.0pt);
%\draw[color=xdxdff] (10.271884058695306,6.475175723498838) node {$B$};
%\draw [fill=xdxdff] (10.,0.) circle (2.0pt);
%\draw[color=xdxdff] (10.340368414331028,-0.1678067731661803) node {$C$};
%\draw [fill=xdxdff] (0.,0.) circle (2.0pt);
%\draw[color=xdxdff] (-0.5029878946582631,-0.23629112880190217) node {$D$};
\draw[color=zzttqq] (5,-0.7) node {$L$};
\draw[color=zzttqq] (-0.7,3) node {$l$};
%\end{scriptsize}
\end{tikzpicture}
\end{center}
\end{Pp}

\begin{Pv}
L'aire du rectangle correspond au nombre de carreaux verts unitaires rentrant à l'intérieur. Pour la calculer, on multiplie le nombre d'unité rentrant dans la largeur par le nombre d'unité rentrant dans la longueur. Ici :
\[3\times5=15\,u\] 
\begin{center}
\begin{tikzpicture}[line cap=round,line join=round,>=triangle 45,x=1.0cm,y=1.0cm]
\draw [color=cqcqcq,, xstep=2.0cm,ystep=2.0cm] (-0.7997534357463911,-0.7727519146150565) grid (10.865415140871562,6.851839679495307);
\clip(-0.7997534357463911,-0.7727519146150565) rectangle (10.865415140871562,6.851839679495307);
\fill[line width=2.pt,color=zzttqq,fill=zzttqq,fill opacity=0.10000000149011612] (0.,6.) -- (10.,6.) -- (10.,0.) -- (0.,0.) -- cycle;
\draw[line width=2.pt,color=zzttqq] (0.,5.515742477247899) -- (0.48425752275210066,5.515742477247899) -- (0.4842575227521006,6.) -- (0.,6.) -- cycle; 
\draw[line width=2.pt,color=zzttqq] (9.5157424772479,6.) -- (9.5157424772479,5.515742477247899) -- (10.,5.515742477247899) -- (10.,6.) -- cycle; 
\draw[line width=2.pt,color=zzttqq] (10.,0.4842575227521006) -- (9.5157424772479,0.48425752275210066) -- (9.5157424772479,0.) -- (10.,0.) -- cycle; 
\draw[line width=2.pt,color=zzttqq] (0.4842575227521006,0.) -- (0.48425752275210066,0.4842575227521006) -- (0.,0.4842575227521006) -- (0.,0.) -- cycle; 
\fill[line width=2.pt,color=qqwuqq,fill=qqwuqq,fill opacity=0.10000000149011612] (0.,0.) -- (0.,2.) -- (2.,2.) -- (2.,0.) -- cycle;
\draw [line width=1.pt,color=zzttqq] (0.,6.)-- (10.,6.);
\draw [line width=1.pt,color=zzttqq] (5.,6.114140592726204) -- (5.,5.885859407273798);
\draw [line width=1.pt,color=zzttqq] (10.,6.)-- (10.,0.);
\draw [line width=1.pt,color=zzttqq] (10.114140592726201,3.0456562370904816) -- (9.885859407273795,3.0456562370904816);
\draw [line width=1.pt,color=zzttqq] (10.114140592726201,2.9543437629095193) -- (9.885859407273795,2.9543437629095193);
\draw [line width=1.pt,color=zzttqq] (10.,0.)-- (0.,0.);
\draw [line width=1.pt,color=zzttqq] (5.,-0.11414059272620306) -- (5.,0.11414059272620306);
\draw [line width=1.pt,color=zzttqq] (0.,0.)-- (0.,6.);
\draw [line width=1.pt,color=zzttqq] (-0.11414059272620307,2.9543437629095193) -- (0.11414059272620307,2.9543437629095193);
\draw [line width=1.pt,color=zzttqq] (-0.11414059272620307,3.0456562370904816) -- (0.11414059272620307,3.0456562370904816);
\draw [line width=1.pt,color=qqwuqq] (0.,0.)-- (0.,2.);
\draw [line width=1.pt,color=qqwuqq] (0.,2.)-- (2.,2.);
\draw [line width=1.pt,color=qqwuqq] (2.,2.)-- (2.,0.);
\draw [line width=1.pt,color=qqwuqq] (2.,0.)-- (0.,0.);
\draw [line width=1.pt,color=qqwuqq] (2.,2.)-- (2.,6.);
\draw [line width=1.pt,color=qqwuqq] (0.,4.)-- (2.,4.);
\draw [line width=1.pt,color=qqwuqq] (2.,2.)-- (10.,2.);
\draw [line width=1.pt,color=qqwuqq] (4.,2.)-- (4.,0.);
\draw [line width=1.pt,color=qqwuqq] (6.,2.)-- (6.,0.);
\draw [line width=1.pt,color=qqwuqq] (8.,0.)-- (8.,2.);
%\begin{scriptsize}
\draw [fill=xdxdff] (0.,6.) circle (2pt);
\draw[color=xdxdff] (-0.5029878946582631,6.429519486408356) node {$A$};
\draw [fill=xdxdff] (10.,6.) circle (2pt);
\draw[color=xdxdff] (10.271884058695306,6.429519486408356) node {$B$};
\draw [fill=xdxdff] (10.,0.) circle (2pt);
\draw[color=xdxdff] (10.340368414331028,-0.21346301025666153) node {$C$};
\draw [fill=xdxdff] (0.,0.) circle (2pt);
\draw[color=xdxdff] (-0.5029878946582631,-0.2819473658923834) node {$D$};
\draw[color=qqwuqq] (1,1) node {$unité$};
\draw[color=qqwuqq] (1,3) node {$2$};
\draw[color=qqwuqq] (1,5) node {$3$};
\draw[color=qqwuqq] (3,1) node {$2$};
\draw[color=qqwuqq] (5,1) node {$3$};\draw[color=qqwuqq] (7,1) node {$4$};\draw[color=qqwuqq] (9,1) node {$5$};
%\end{scriptsize}
\end{tikzpicture}
\end{center}
\end{Pv}

%\begin{ExOApp}[]
%\begin{enumerate}
%\item Un rectangle a une largeur de $6\,km$ et une aire de $90\,km^2$. Calculer sa longueur.
%\item Un rectangle a une longueur de $17\,dam$ et une largeur de $9\,dam$. Calculer son aire.
%\item Un rectangle a une longueur de $15\,m$ et une aire de $135\,m^2$. Calculer sa largeur.
%\end{enumerate}
%\end{ExOApp}

\subsection{Aire d'un triangle}

\begin{PpT}{Aire d'un triangle rectangle}
L'aire d'un triangle rectangle est donnée par la formule :
\[\mathcal{A}_{triangle\,rectangle}=\frac{a\times b}{2}\]
\begin{center}
\begin{tikzpicture}[line cap=round,line join=round,>=triangle 45,x=0.6cm,y=0.6cm]
\clip(-1.0736908582892786,-1.0695174557031843) rectangle (11.185008800504932,6.897495916585789);
\draw[line width=2.pt,color=zzttqq] (0.4842575227521006,0.) -- (0.48425752275210066,0.4842575227521006) -- (0.,0.4842575227521006) -- (0.,0.) -- cycle; 
\fill[line width=2.pt,color=zzttqq,fill=zzttqq,fill opacity=0.10000000149011612] (0.,0.) -- (0.,6.) -- (10.,0.) -- cycle;
%\draw [line width=2.pt,dash pattern=on 1pt off 1pt] (0.,6.)-- (10.,6.);
%\draw [line width=2.pt,dash pattern=on 1pt off 1pt] (10.,6.)-- (10.,0.);
\draw [line width=2.pt,color=zzttqq] (0.,0.)-- (0.,6.);
\draw [line width=2.pt,color=zzttqq] (0.,6.)-- (10.,0.);
\draw [line width=2.pt,color=zzttqq] (10.,0.)-- (0.,0.);
%\begin{scriptsize}
\draw [fill=xdxdff] (0.,6.) circle (2.0pt);
\draw[color=xdxdff] (-0.5029878946582631,6.475175723498838) node {$A$};
\draw [fill=xdxdff] (10.,0.) circle (2.0pt);
\draw[color=xdxdff] (10.340368414331028,-0.1678067731661803) node {$B$};
\draw [fill=xdxdff] (0.,0.) circle (2.0pt);
\draw[color=xdxdff] (-0.5029878946582631,-0.23629112880190217) node {$C$};
\draw[color=zzttqq] (0.41013684715136145,2.913989230441302) node {$b$};
\draw[color=zzttqq] (4.7703074892923185,0.5626930202815192) node {$a$};
%\end{scriptsize}
\end{tikzpicture}
\end{center}
\end{PpT}

\begin{Pv}
Il s'agit de l'aire d'un \textbf{demi-rectangle}, donc la moitié de l'aire du rectangle dont une des diagonales est $[AB]$.
\begin{center}
\begin{tikzpicture}[line cap=round,line join=round,>=triangle 45,x=0.6cm,y=0.6cm]
\clip(-1.0736908582892786,-1.0695174557031843) rectangle (11.185008800504932,6.897495916585789);
\draw[line width=2.pt,color=zzttqq] (0.4842575227521006,0.) -- (0.48425752275210066,0.4842575227521006) -- (0.,0.4842575227521006) -- (0.,0.) -- cycle; 
\fill[line width=2.pt,color=zzttqq,fill=zzttqq,fill opacity=0.10000000149011612] (0.,0.) -- (0.,6.) -- (10.,0.) -- cycle;
\draw [line width=1.pt,dash pattern=on 1pt off 2pt] (0.,6.)-- (10.,6.);
\draw [line width=1.pt,dash pattern=on 1pt off 2pt] (10.,6.)-- (10.,0.);
\draw [line width=2.pt,color=zzttqq] (0.,0.)-- (0.,6.);
\draw [line width=2.pt,color=zzttqq] (0.,6.)-- (10.,0.);
\draw [line width=2.pt,color=zzttqq] (10.,0.)-- (0.,0.);
%\begin{scriptsize}
\draw [fill=xdxdff] (0.,6.) circle (2.0pt);
\draw[color=xdxdff] (-0.5029878946582631,6.475175723498838) node {$A$};
\draw [fill=xdxdff] (10.,0.) circle (2.0pt);
\draw[color=xdxdff] (10.340368414331028,-0.1678067731661803) node {$B$};
\draw [fill=xdxdff] (0.,0.) circle (2.0pt);
\draw[color=xdxdff] (-0.5029878946582631,-0.23629112880190217) node {$C$};
\draw[color=zzttqq] (0.41013684715136145,2.913989230441302) node {$b$};
\draw[color=zzttqq] (4.7703074892923185,0.5626930202815192) node {$a$};
%\end{scriptsize}
\end{tikzpicture}
\end{center}
\end{Pv}

\begin{PpT}{Aire d'un triangle quelconque}
L'aire d'un triangle quelconque est donné par la formule :
\[\mathcal{A}_{triangle}=\frac{base\times hauteur}{2}\]
\[\mathcal{A}_{triangle}=\frac{b\times h}{2}\]
Où la \textcolor{ffqqqq}{hauteur (h)} désigne une perpendiculaire à un côté du triangle passant par le sommet opposé et la \textcolor{zzttqq}{base (b)} désigne le côté que la \textcolor{ffqqqq}{hauteur} coupe.
\begin{center}
\begin{tikzpicture}[line cap=round,line join=round,>=triangle 45,x=0.6cm,y=0.6cm]
\clip(-1.598737584829813,-0.5444707291626504) rectangle (10.979555733597763,7.6051675914882475);
\fill[line width=2.pt,color=zzttqq,fill=zzttqq,fill opacity=0.10000000149011612] (0.,0.) -- (4.,6.) -- (10.,2.) -- cycle;
\draw[line width=2.pt,color=qqwuqq,fill=qqwuqq,fill opacity=0.10000000149011612] (4.9050292862262665,1.474853568868666) -- (4.430175717357601,1.3798828550949327) -- (4.5251464311313345,0.9050292862262668) -- (5.,1.) -- cycle; 
\draw [line width=2.pt,color=zzttqq] (0.,0.)-- (4.,6.);
\draw [line width=2.pt,color=zzttqq] (4.,6.)-- (10.,2.);
\draw [line width=2.pt,color=zzttqq] (10.,2.)-- (0.,0.);
\draw [line width=2.pt,color=ffqqqq] (4.,6.)-- (5.,1.);
%\draw [line width=2.pt,dash pattern=on 2pt off 2pt] (0.,0.)-- (-1.,5.);
%\draw [line width=2.pt,dash pattern=on 2pt off 2pt] (-1.,5.)-- (9.,7.);
%\draw [line width=2.pt,dash pattern=on 2pt off 2pt] (9.,7.)-- (10.,2.);
%\begin{scriptsize}
\draw [fill=xdxdff] (0.,0.) circle (2.0pt);
\draw[color=xdxdff] (-0.6171284873844668,0.24309936064815066) node {$A$};
\draw [fill=xdxdff] (4.,6.) circle (2.5pt);
\draw[color=xdxdff] (4.15394828857082,6.429519486408356) node {$B$};
\draw [fill=xdxdff] (10.,2.) circle (2.5pt);
\draw[color=xdxdff] (10.157743465969101,2.4117706224460087) node {$C$};
\draw[color=zzttqq] (5.135557386016167,0.38006807191959435) node {$b$};
\draw [fill=uuuuuu] (5.,1.) circle (2.0pt);
\draw[color=ffqqqq] (4.085463932935099,3.3020672457103926) node {$h$};
%\end{scriptsize}
\end{tikzpicture}
\end{center}
\end{PpT}

\begin{Rqs}
\begin{itemize}
\item Deux triangles de même hauteur et de même base ont la même aire.
\item L'aire d'un triangle ne dépend pas du côté choisi.
\end{itemize}
\end{Rqs}

%\begin{ExOApp}[]
%Le triangle $ABC$ a une hauteur de $10\,mm$ et une aire de $85\,mm^2$. Calculer la longueur de sa base.
%\end{ExOApp}

\subsection{Aire d'un disque}

\begin{Def}
Un \textbf{disque} est un cercle contenant une surface.
\begin{center}
\begin{tikzpicture}[line cap=round,line join=round,>=triangle 45,x=1.0cm,y=1.0cm]
\clip(-3.2258782096036835,-1.3387575571899715) rectangle (3.220423513612537,1.6953257957763235);
\draw [line width=2.pt,color=zzttqq,fill=zzttqq,fill opacity=0.10000000149011612] (2.,0.) circle (1.cm);
\draw [line width=2.pt] (-2.,0.) circle (1.cm);
\begin{scriptsize}
\draw [fill=xdxdff] (-2.,0.) circle (2.5pt);
\draw[color=black] (-2,1.3) node {$cercle$};
\draw [fill=xdxdff] (2.,0.) circle (2.5pt);
\draw[color=zzttqq] (2,1.3) node {$disque$};
\end{scriptsize}
\end{tikzpicture}
\end{center}
\end{Def}

\begin{Pp}
L'aire d'un disque est donnée par la formule :
\[\mathcal{A}_{disque}=\pi \times rayon \times rayon=\pi \times rayon^2\]
\[\mathcal{A}_{disque}=\pi \times r^2\]
\begin{center}
\begin{tikzpicture}[line cap=round,line join=round,>=triangle 45,x=0.5cm,y=0.5cm]
%\clip(-0.4116754204773013,-4.356766526217832) rectangle (10.705618311054875,6.3952773085904955);
\draw [line width=2.pt,color=zzttqq,fill=zzttqq,fill opacity=0.1] (5.021416793289964,1.0306694504589518) circle (2.363592350903993cm);
\draw [line width=2.pt,color=qqwuqq] (5.021416793289964,1.0306694504589518)-- (9.221790605614236,3.19934071225681);
%\begin{scriptsize}
\draw [fill=ududff] (5.021416793289964,1.0306694504589518) circle (2.5pt);
\draw[color=ududff] (4.450713829658948,1.3160209322744594) node {$A$};
\draw [fill=ududff] (9.221790605614236,3.19934071225681) circle (2.5pt);
\draw[color=ududff] (9.381587435430921,3.621660905343761) node {$B$};
\draw[color=qqwuqq] (7.349884884904506,1.7497551846340311) node {$r$};
%\end{scriptsize}
\end{tikzpicture}
\end{center}
\end{Pp}

\begin{Rq}
$r^2$ se prononce "$r$ au carré" et signifie qu'il faut prendre le nombre $r\times r$.
\end{Rq}

\begin{Pv}
La preuve de la formule de l'aire du disque a été apportée par le grand mathématicien Archimède. Il a découpé le disque en secteurs égaux puis les a rassemblés pour former une sorte de rectangle dont la largeur est le rayon et la longueur est la moitié du périmètre :
\begin{center}
\begin{tikzpicture}[line cap=round,line join=round,>=triangle 45,x=1.0cm,y=1.0cm]
\clip(-5.06354915382803,-6.147772259251827) rectangle (4.735026060140306,2.4059020054008484);
\draw [line width=2.pt,color=zzttqq,fill=zzttqq,fill opacity=0.10999999940395355] (0.,0.) circle (2.cm);
\draw [line width=2.pt,color=zzttqq] (0.,2.)-- (0.,-2.);
\draw [line width=2.pt,color=zzttqq] (-2.,0.)-- (2.,0.);
\draw [line width=2.pt,color=zzttqq] (-1.4142135623730951,1.4142135623730951)-- (1.414213562373095,-1.414213562373095);
\draw [line width=2.pt,color=zzttqq] (1.414213562373095,1.414213562373095)-- (-1.4142135623730951,-1.4142135623730951);
\draw [shift={(-2.84,-5.4)},line width=2.pt,color=zzttqq,fill=zzttqq,fill opacity=0.10999999940395355]  (0,0) --  plot[domain=1.1955255039389394:1.9809236673363877,variable=\t]({1.*2.128097742116184*cos(\t r)+0.*2.128097742116184*sin(\t r)},{0.*2.128097742116184*cos(\t r)+1.*2.128097742116184*sin(\t r)}) -- cycle ;
\draw [shift={(-2.06,-3.42)},line width=2.pt,color=zzttqq,fill=zzttqq,fill opacity=0.10999999940395355]  (0,0) --  plot[domain=4.3371181575287325:5.122516320926181,variable=\t]({1.*2.128097742116184*cos(\t r)+0.*2.128097742116184*sin(\t r)},{0.*2.128097742116184*cos(\t r)+1.*2.128097742116184*sin(\t r)}) -- cycle ;
\draw [shift={(-1.2114718625761427,-5.3716147160748715)},line width=2.pt,color=zzttqq,fill=zzttqq,fill opacity=0.10999999940395355]  (0,0) --  plot[domain=1.1955255039389394:1.9809236673363877,variable=\t]({1.*2.128097742116184*cos(\t r)+0.*2.128097742116184*sin(\t r)},{0.*2.128097742116184*cos(\t r)+1.*2.128097742116184*sin(\t r)}) -- cycle ;
\draw [shift={(-0.4314718625761429,-3.391614716074871)},line width=2.pt,color=zzttqq,fill=zzttqq,fill opacity=0.10999999940395355]  (0,0) --  plot[domain=4.3371181575287325:5.122516320926181,variable=\t]({1.*2.128097742116184*cos(\t r)+0.*2.128097742116184*sin(\t r)},{0.*2.128097742116184*cos(\t r)+1.*2.128097742116184*sin(\t r)}) -- cycle ;
\draw [shift={(0.41705627484771446,-5.343229432149743)},line width=2.pt,color=zzttqq,fill=zzttqq,fill opacity=0.10999999940395355]  (0,0) --  plot[domain=1.1955255039389394:1.9809236673363877,variable=\t]({1.*2.128097742116184*cos(\t r)+0.*2.128097742116184*sin(\t r)},{0.*2.128097742116184*cos(\t r)+1.*2.128097742116184*sin(\t r)}) -- cycle ;
\draw [shift={(1.1970562748477143,-3.3632294321497422)},line width=2.pt,color=zzttqq,fill=zzttqq,fill opacity=0.10999999940395355]  (0,0) --  plot[domain=4.3371181575287325:5.122516320926181,variable=\t]({1.*2.128097742116184*cos(\t r)+0.*2.128097742116184*sin(\t r)},{0.*2.128097742116184*cos(\t r)+1.*2.128097742116184*sin(\t r)}) -- cycle ;
\draw [shift={(2.0455844122715714,-5.314844148224614)},line width=2.pt,color=zzttqq,fill=zzttqq,fill opacity=0.10999999940395355]  (0,0) --  plot[domain=1.1955255039389392:1.9809236673363877,variable=\t]({1.*2.1280977421161844*cos(\t r)+0.*2.1280977421161844*sin(\t r)},{0.*2.1280977421161844*cos(\t r)+1.*2.1280977421161844*sin(\t r)}) -- cycle ;
\draw [shift={(2.8255844122715716,-3.3348441482246134)},line width=2.pt,color=zzttqq,fill=zzttqq,fill opacity=0.10999999940395355]  (0,0) --  plot[domain=4.337118157528732:5.122516320926181,variable=\t]({1.*2.1280977421161844*cos(\t r)+0.*2.1280977421161844*sin(\t r)},{0.*2.1280977421161844*cos(\t r)+1.*2.1280977421161844*sin(\t r)}) -- cycle ;
%\draw [line width=2.pt] (-3.688528137423857,-3.448385283925129)-- (-2.84,-5.4);
%\draw [line width=2.pt] (-3.688528137423857,-3.448385283925129)-- (2.8722786296747205,-3.3229307497861025);
%\begin{scriptsize}
%\draw [fill=zzttqq] (0.,0.) circle (2.0pt);
%\draw [fill=xdxdff] (-3.688528137423857,-3.448385283925129) circle (2.5pt);
%\draw [fill=xdxdff] (-2.84,-5.4) circle (2.5pt);
\draw[color=black] (-3.9592015374996326,-4.531408929898445) node {$r$};
%\draw [fill=xdxdff] (2.8722786296747205,-3.3229307497861025) circle (2.5pt);
\draw[color=black] (-0.12410345170464818,-2.724294648633796) node {$\frac{\mathcal{P}}{2}$};
%\end{scriptsize}
\end{tikzpicture}
\end{center}
En augmentant le nombre de secteurs la forme ressemble de plus en plus à un rectangle et pour un nombre infini de secteur elle devient un rectangle. La formule de l'aire du rectangle nous donne :
\[\mathcal{A}_{disque}=r\times \frac{\mathcal{P}}{2}=r\times \frac{2\times\pi\times r}{2}=\pi \times r \times r\]
\end{Pv}

%\begin{ExOApp}[]
%Déterminer l'aire du disque de centre O et de rayon $4\,hm$.
%\end{ExOApp}

\section{Formulaire}

\begin{Pp}
\begin{center}
\begin{tikzpicture}[line cap=round,line join=round,>=triangle 45,x=.8cm,y=.8cm]
\clip(-1.1508968623201985,-5.535831819898127) rectangle (14.729158042124956,2.4283294543676694);
\fill[line width=2.pt,color=zzttqq,fill=zzttqq,fill opacity=0.10000000149011612] (0.,0.) -- (0.,2.) -- (2.,2.) -- (2.,0.) -- cycle;
\fill[line width=2.pt,color=zzttqq,fill=zzttqq,fill opacity=0.10000000149011612] (4.,2.) -- (4.,0.) -- (8.,0.) -- (8.,2.) -- cycle;
\fill[line width=2.pt,color=zzttqq,fill=zzttqq,fill opacity=0.10000000149011612] (0.,-2.) -- (0.,-4.) -- (3.,-4.) -- cycle;
\fill[line width=2.pt,color=zzttqq,fill=zzttqq,fill opacity=0.10000000149011612] (6.,-2.) -- (5.,-4.) -- (9.,-3.) -- cycle;
\draw [line width=2.pt,color=zzttqq,fill=zzttqq,fill opacity=0.14000000059604645] (12.,-1.) circle (1.54cm);
\draw[line width=2.pt,color=zzttqq,fill=zzttqq,fill opacity=0.10000000149011612] (0.34130378445434223,-4.) -- (0.3413037844543423,-3.6586962155456577) -- (0.,-3.6586962155456577) -- (0.,-4.) -- cycle; 
\draw[line width=2.pt,color=zzttqq,fill=zzttqq,fill opacity=0.10000000149011612] (0.,1.6586962155456577) -- (0.3413037844543423,1.6586962155456577) -- (0.34130378445434223,2.) -- (0.,2.) -- cycle; 
\draw[line width=2.pt,color=zzttqq,fill=zzttqq,fill opacity=0.10000000149011612] (1.6586962155456577,2.) -- (1.6586962155456577,1.6586962155456577) -- (2.,1.6586962155456577) -- (2.,2.) -- cycle; 
\draw[line width=2.pt,color=zzttqq,fill=zzttqq,fill opacity=0.10000000149011612] (2.,0.34130378445434223) -- (1.6586962155456577,0.3413037844543423) -- (1.6586962155456577,0.) -- (2.,0.) -- cycle; 
\draw[line width=2.pt,color=zzttqq,fill=zzttqq,fill opacity=0.10000000149011612] (0.34130378445434223,0.) -- (0.3413037844543423,0.34130378445434223) -- (0.,0.34130378445434223) -- (0.,0.) -- cycle; 
\draw[line width=2.pt,color=zzttqq,fill=zzttqq,fill opacity=0.10000000149011612] (4.,1.6586962155456577) -- (4.341303784454342,1.6586962155456577) -- (4.341303784454342,2.) -- (4.,2.) -- cycle; 
\draw[line width=2.pt,color=zzttqq,fill=zzttqq,fill opacity=0.10000000149011612] (7.658696215545658,2.) -- (7.658696215545658,1.6586962155456577) -- (8.,1.6586962155456577) -- (8.,2.) -- cycle; 
\draw[line width=2.pt,color=zzttqq,fill=zzttqq,fill opacity=0.10000000149011612] (8.,0.34130378445434223) -- (7.658696215545658,0.3413037844543423) -- (7.658696215545658,0.) -- (8.,0.) -- cycle; 
\draw[line width=2.pt,color=zzttqq,fill=zzttqq,fill opacity=0.10000000149011612] (4.341303784454342,0.) -- (4.341303784454342,0.34130378445434223) -- (4.,0.34130378445434223) -- (4.,0.) -- cycle; 
\draw[line width=2.pt,color=zzttqq,fill=zzttqq,fill opacity=0.10000000149011612] (6.7428780126419525,-3.564280496839512) -- (6.660099685952052,-3.233167190079913) -- (6.328986379192454,-3.315945516769813) -- (6.411764705882353,-3.6470588235294117) -- cycle; 
\draw [line width=2.pt,color=zzttqq] (0.,0.)-- (0.,2.);
\draw [line width=2.pt,color=zzttqq] (-0.09653528817291887,1.) -- (0.09653528817291887,1.);
\draw [line width=2.pt,color=zzttqq] (0.,2.)-- (2.,2.);
\draw [line width=2.pt,color=zzttqq] (1.,2.0965352881729187) -- (1.,1.903464711827081);
\draw [line width=2.pt,color=zzttqq] (2.,2.)-- (2.,0.);
\draw [line width=2.pt,color=zzttqq] (2.0965352881729196,1.) -- (1.9034647118270818,1.);
\draw [line width=2.pt,color=zzttqq] (2.,0.)-- (0.,0.);
\draw [line width=2.pt,color=zzttqq] (1.,-0.09653528817291875) -- (1.,0.09653528817291875);
\draw [line width=2.pt,color=zzttqq] (4.,2.)-- (4.,0.);
\draw [line width=2.pt,color=zzttqq] (4.096535288172918,1.) -- (3.9034647118270804,1.);
\draw [line width=2.pt,color=zzttqq] (4.,0.)-- (8.,0.);
\draw [line width=2.pt,color=zzttqq] (5.959776963261284,0.09653528817291875) -- (5.959776963261284,-0.09653528817291875);
\draw [line width=2.pt,color=zzttqq] (6.040223036738716,0.09653528817291875) -- (6.040223036738716,-0.09653528817291875);
\draw [line width=2.pt,color=zzttqq] (8.,0.)-- (8.,2.);
\draw [line width=2.pt,color=zzttqq] (7.903464711827081,1.) -- (8.09653528817292,1.);
\draw [line width=2.pt,color=zzttqq] (8.,2.)-- (4.,2.);
\draw [line width=2.pt,color=zzttqq] (6.040223036738716,1.903464711827081) -- (6.040223036738716,2.0965352881729187);
\draw [line width=2.pt,color=zzttqq] (5.959776963261284,1.903464711827081) -- (5.959776963261284,2.0965352881729187);
\draw [line width=2.pt,color=zzttqq] (0.,-2.)-- (0.,-4.);
\draw [line width=2.pt,color=zzttqq] (0.,-4.)-- (3.,-4.);
\draw [line width=2.pt,color=zzttqq] (3.,-4.)-- (0.,-2.);
\draw [line width=2.pt,color=zzttqq] (6.,-2.)-- (5.,-4.);
\draw [line width=2.pt,color=zzttqq] (5.,-4.)-- (9.,-3.);
\draw [line width=2.pt,color=zzttqq] (9.,-3.)-- (6.,-2.);
\draw [line width=2.pt] (12.,-1.)-- (10.46,-1.);
\draw (-0.45906063041427997,-0.35510468795148753) node[anchor=north west] {$\mathcal{A}_{carré}=c\times c=c^2$};
\draw (4.769934145618826,-0.3390154732560011) node[anchor=north west] {$\mathcal{A}_{rectangle}=l\times L$};
\draw (-0.2659900540684422,-4.0234456385224) node[anchor=north west] {$\mathcal{A}_{triangle\;rectangle}=\frac{b\times h}{2}$};
\draw (5.220432157092447,-3.991267209131427) node[anchor=north west] {$\mathcal{A}_{triangle}=\frac{b\times h}{2}$};
\draw (10.079374995129363,-2.823683960015078) node[anchor=north west] {$\mathcal{A}_{disque}=\pi\times r\times r$};
\draw (11.079374995129363,-3.423683960015078) node[anchor=north west] {$=\pi\times r^2$};
\draw [line width=2.pt] (6.411764705882353,-3.6470588235294117)-- (6.,-2.);
%\begin{scriptsize}
\draw[color=zzttqq] (1.0372363362659625,0.4091330100841192) node {$c$};
\draw[color=zzttqq] (4.43206063701361,1.0527015979035774) node {$l$};
\draw[color=zzttqq] (5.944446818389339,0.5217575129525244) node {$L$};
\draw[color=zzttqq] (0.26495403088261155,-2.9856912906635236) node {$h$};
\draw[color=zzttqq] (1.3107529860892326,-3.6) node {$b$};
\draw[color=zzttqq] (7.143093313203082,-3.8) node {$b$};
\draw[color=black] (11.269976882595364,-0.6044875157315277) node {$r$};
\draw[color=black] (6.491480118035879,-2.6156393526673347) node {$h$};
%\end{scriptsize}
\end{tikzpicture}
\end{center}
\end{Pp}

%\section{Les savoir-faire du parcours}
%
%\begin{CpsCol}
%\begin{itemize}
%\item Savoir exprimer l'aire d'une figure en fonction d'une unité d'aire.
%\item Savoir convertir des unités d'aire.
%\item Savoir calculer l'aire d'un carré.
%\item Savoir calculer l'aire d'un rectangle.
%\item Savoir calculer l'aire d'un triangle.
%\item Savoir calculer l'aire d'un disque.
%\item Savoir résoudre un problème d'aire.
%\end{itemize}
%\end{CpsCol}

% \end{document}

\end{pageCours} 
