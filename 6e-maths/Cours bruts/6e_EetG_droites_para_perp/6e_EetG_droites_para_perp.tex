\documentclass[a4paper,dvipsnames]{article}

\addtolength{\hoffset}{-2.25cm}
\addtolength{\textwidth}{4.5cm}
\addtolength{\voffset}{-3.25cm}
\addtolength{\textheight}{5cm}
\setlength{\parskip}{0pt}
\setlength{\parindent}{0in}

%----------------------------------------------------------------------------------------
%	PACKAGES AND OTHER DOCUMENT CONFIGURATIONS
%----------------------------------------------------------------------------------------

%----------------------------------------------------------------------------------------
%		Generals
%----------------------------------------------------------------------------------------
\usepackage{fourier}
\usepackage{frcursive}
\usepackage[T1]{fontenc} %Accents handling
\usepackage[utf8]{inputenc} % Use UTF-8 encoding
%\usepackage{microtype} % Slightly tweak font spacing for aesthetics
\usepackage[english, francais]{babel} % Language hyphenation and typographical rules

%----------------------------------------------------------------------------------------
%		Graphics
%----------------------------------------------------------------------------------------
\usepackage{xcolor}
\usepackage{graphicx, multicol} % Enhanced support for graphics
\graphicspath{{FIG/}}
\usepackage{wrapfig}

%----------------------------------------------------------------------------------------
%		Other packages
%----------------------------------------------------------------------------------------
\usepackage{hyperref}
\hypersetup{
	colorlinks=true, %colorise les liens
	breaklinks=true, %permet le retour à la ligne dans les liens trop longs
	urlcolor= bleu3,  %couleur des hyperliens
	linkcolor= bleu3, %couleur des liens internes
	plainpages=false  %pour palier à "Bookmark problems can occur when you have duplicate page numbers, for example, if you have a page i and a page 1."
}
\usepackage{tabularx}
\newcolumntype{M}[1]{>{\arraybackslash}m{#1}} %Defines a scalable column type in tabular
\usepackage{booktabs} % Enhances quality of tables
\usepackage{diagbox} % barre en diagonale dans un tableau
\usepackage{multicol}
\usepackage[explicit]{titlesec}


%----------------------------------------------------------------------------------------
%		Headers and footers
%----------------------------------------------------------------------------------------
\usepackage{fancyhdr} % Headers and footers
\pagestyle{fancy} % All pages have headers and footers
\fancyhead{}\renewcommand{\headrulewidth}{0pt} % Blank out the default header
\renewcommand{\footrulewidth}{0pt}
\fancyfoot[L]{} % Custom footer text
\fancyfoot[C]{\href{https://sacado.xyz/}{sacado.xyz}} % Custom footer text
\fancyfoot[R]{\thepage} % Custom footer text

%----------------------------------------------------------------------------------------
%		Mathematics packages
%----------------------------------------------------------------------------------------
\usepackage{amsthm, amsmath, amssymb} % Mathematical typesetting
\usepackage{marvosym, wasysym} % More symbols
\usepackage[makeroom]{cancel}
\usepackage{xlop}
\usepackage{pgf,tikz,pgfplots}
\pgfplotsset{compat=1.15}
\usetikzlibrary{positioning}
%\usetikzlibrary{arrows}
\usepackage{pst-plot,pst-tree,pst-func, pstricks-add,pst-node,pst-text}
\usepackage{units}
\usepackage{nicefrac}
\usepackage[np]{numprint} %Séparation milliers dans un nombre

%----------------------------------------------------------------------------------------
%		New text commands
%----------------------------------------------------------------------------------------
\usepackage{calc}
\usepackage{boites}
 \renewcommand{\arraystretch}{1.6}

%%%%% Pour les imports.
\usepackage{import}

%%%%% Pour faire des boites
\usepackage[tikz]{bclogo}
\usepackage{bclogo}
\usepackage{framed}
\usepackage[skins]{tcolorbox}
\tcbuselibrary{breakable}
\tcbuselibrary{skins}
\usetikzlibrary{babel,arrows,shadows,decorations.pathmorphing,decorations.markings,patterns}

%%%%% Pour les symboles et les ensembles
\newcommand{\pp}{\leq}
\newcommand{\pg}{\geq}
%\newcommand{\euro}{\eurologo{}}
\newcommand{\R}{\mathbb{R}}
\newcommand{\N}{\mathbb{N}}
\newcommand{\D}{\mathbb{D}}
\newcommand{\Z}{\mathbb{Z}}
\newcommand{\Q}{\mathbb{Q}}
\newcommand{\C}{\mathbb{C}}

%%%%% Pour une double minipage
\newcommand{\mini}[2]{
\begin{minipage}[t]{0.48\linewidth}
#1
\end{minipage}
\hfill
\begin{minipage}[t]{0.48\linewidth}
#2
\end{minipage}
}


%\newcommand\hole[1]{\texttt{\_}}
%\newcommand{\PROP}[1]{\textbf{\underline{#1}}}
%\newcommand{\exercice}{\textcolor{OliveGreen}{Exercice : }}
%\newcommand{\correction}{\textcolor{BurntOrange}{Correction : }}
%\newcommand{\propriete}{\textbf{\underline{Propriété}} : }
%\newcommand{\prop}{\textbf{\underline{Propriété}} : }
%\newcommand{\vocabulaire}{\textbf{\underline{Vocabulaire}} : }
%\newcommand{\voca}{\textbf{\underline{Vocabulaire}} : }

\usepackage{enumitem}
\newlist{todolist}{itemize}{2} %Pour faire des QCM
\setlist[todolist]{label=$\square$} %Pour faire des QCM \begin{todolist} instead of itemize

%----------------------------------------------------------------------------------------
%		Définition de couleur pour geogebra
%----------------------------------------------------------------------------------------
\definecolor{zzttqq}{rgb}{0.6,0.2,0.} %rouge des polygones
\definecolor{qqqqff}{rgb}{0.,0.,1.}
\definecolor{xdxdff}{rgb}{0.49019607843137253,0.49019607843137253,1.}%bleu
\definecolor{qqwuqq}{rgb}{0.,0.39215686274509803,0.} %vert des angles
\definecolor{ffqqqq}{rgb}{1.,0.,0.} %rouge vif
\definecolor{uuuuuu}{rgb}{0.26666666666666666,0.26666666666666666,0.26666666666666666}
\definecolor{qqzzqq}{rgb}{0.,0.6,0.}
\definecolor{cqcqcq}{rgb}{0.7529411764705882,0.7529411764705882,0.7529411764705882} %gris
\definecolor{qqffqq}{rgb}{0.,1.,0.}
\definecolor{ffdxqq}{rgb}{1.,0.8431372549019608,0.}
\definecolor{ffffff}{rgb}{1.,1.,1.}
\definecolor{ududff}{rgb}{0.30196078431372547,0.30196078431372547,1.}

%-------------------------------------------------
%
%	EN TETE
%
%-------------------------------------------------

% Classe
\newcommand{\myClasse}   
{
    6e
}

% Discipline
\newcommand{\myDiscipline}   
{
    Mathématiques
}

% Parcours
\newcommand{\myParcours}
{
  Nombres et Calculs
}

%Titre de la séquence
\newcommand{\myTitle}
{
    \scshape\huge
\textcolor{sacado_purple}{
		Enchainements d'opérations
}
}

%----------------------------------------------------------------------------------------

%----------------------------------------------------------------------------------------
%		Définition de couleur pour les boites
%----------------------------------------------------------------------------------------
\definecolor{bleu1}{rgb}{0.54,0.79,0.95} %% Bleu
\definecolor{sapgreen}{rgb}{0.4, 0.49, 0}
\definecolor{dvzfxr}{rgb}{0.7,0.4,0.}
\definecolor{beamer}{rgb}{0.5176470588235295,0.49019607843137253,0.32941176470588235} % couleur beamer
\definecolor{preuveRbeamer}{rgb}{0.8,0.4,0}
\definecolor{sectioncolor}{rgb}{0.24,0.21,0.44}
\definecolor{subsectioncolor}{rgb}{0.1,0.21,0.61}
\definecolor{subsubsectioncolor}{rgb}{0.1,0.21,0.61}
\definecolor{info}{rgb}{0.82,0.62,0}
\definecolor{bleu2}{rgb}{0.38,0.56,0.68}
\definecolor{bleu3}{rgb}{0.24,0.34,0.40}
\definecolor{bleu4}{rgb}{0.12,0.20,0.25}
\definecolor{vert}{rgb}{0.21,0.33,0}
\definecolor{vertS}{rgb}{0.05,0.6,0.42}
\definecolor{red}{rgb}{0.78,0,0}
\definecolor{color5}{rgb}{0,0.4,0.58}
\definecolor{eduscol4B}{rgb}{0.19,0.53,0.64}
\definecolor{eduscol4P}{rgb}{0.62,0.12,0.39}

%----------------------------------------------------------------------------------------
%		Définition de couleur pour les boites SACADO
%----------------------------------------------------------------------------------------
\definecolor{sacado_blue}{RGB}{0,129,159} %% Bleu Sacado
\definecolor{sacado_green}{RGB}{59, 157, 38} %% Vert Sacado
\definecolor{sacado_yellow}{RGB}{255,180,0} %% Jaune Sacado
\definecolor{sacado_purple}{RGB}{94,68,145} %% Violet foncé Sacado
\definecolor{sacado_violet}{RGB}{153,117,224} %% Violet clair Sacado
\definecolor{sacado_orange}{RGB}{249,168,100} %% Orange Sacado
\definecolor{ill_frame}{HTML}{F0F0F0}
\definecolor{ill_back}{HTML}{F7F7F7}
\definecolor{ill_title}{HTML}{AAAAAA}


 % Compteurs pour Théorème, Définition, Exemple, Remarque, .....
\newcounter{cpttheo}
\setcounter{cpttheo}{0}
\newcounter{cptdef}
\setcounter{cptdef}{0}
\newcounter{cptmth}
\setcounter{cptmth}{0}
\newcounter{cpttitre}
\setcounter{cpttitre}{0}
 % Exercices
\newcounter{cptapp}
\setcounter{cptapp}{0}
\newcounter{cptex}
\setcounter{cptex}{0}
\newcounter{cptsr}
\setcounter{cptsr}{0}
\newcounter{cpti}
\setcounter{cpti}{0}
\newcounter{cptcor}
\setcounter{cptcor}{0}




%%%%% Pour réinitialiser numéros des paragraphes après une nouvelle partie
\makeatletter
    \@addtoreset{paragraph}{part}
\makeatother


%%%% Titres et sections

\newlength\chapnumb
\setlength\chapnumb{3cm}


% \titleformat{\part}[block] {
 % \normalfont\sffamily\color{violet}}{}{0pt} {
   % \parbox[t]{\chapnumb}{\fontsize{120}{110}\selectfont\ding{110}}
   % \parbox[b]{\dimexpr\textwidth-\chapnumb\relax}{
       % \raggedleft
       % \hfill{{\color{bleu3}\fontsize{40}{30}\selectfont#1}}\\
       % \rule{0.99\textwidth-\chapnumb\relax}{0.4pt}
 % }
% }

% \titleformat{name=\part,numberless}[block]
% {\normalfont\sffamily\color{bleu3}}{}{0pt}
% {\parbox[b]{\chapnumb}{%
  % \mbox{}}%
 % \parbox[b]{\dimexpr\textwidth-\chapnumb\relax}{%
   % \raggedleft%
   % \hfill{{\color{bleu3}\fontsize{40}{30}\selectfont#1}}\\
   % \rule{0.99\textwidth-\chapnumb\relax}{0.4pt}
 % }
% }



% \titleformat{\chapter}[block] {
 % \normalfont\sffamily\color{violet}}{}{0pt} {
   % \parbox[t]{\chapnumb}{ 
     % \fontsize{120}{110}\selectfont\thechapter}
     % \parbox[b]{\textwidth-\chapnumb}{
       % \raggedleft
       % \hfill{{\color{bleu3}\huge#1}}\\  
  % \ifthenelse{\thechapter<10}{\rule{0.99\textwidth-\chapnumb}{0.4pt}}{\rule{0.9\textwidth - \chapnumb}{0.4pt}}
       % \setcounter{cpttitre}{0}
	% \setcounter{cptapp}{0}
	% \setcounter{cptex}{0}
	% \setcounter{cptsr}{0}
	% \setcounter{cpti}{0}
	% \setcounter{cptcor}{0} 
 % }
% }

% \titleformat{name=\chapter,numberless}[block]
% {\normalfont\sffamily\color{bleu3}}{}{0pt}
% {\parbox[b]{\chapnumb}{%
  % \mbox{}}%
 % \parbox[b]{\textwidth-\chapnumb}{%
   % \raggedleft
   % \hfill{{\color{bleu3}\huge#1}}\\
   % \ifthenelse{\thechapter<10}{\rule{0.99\textwidth-\chapnumb}{0.4pt}}{ \rule{0.9\textwidth - \chapnumb}{0.4pt}}
       % \setcounter{cpttitre}{0}
	% \setcounter{cptapp}{0}
	% \setcounter{cptex}{0}
	% \setcounter{cptsr}{0}
	% \setcounter{cpti}{0}
	% \setcounter{cptcor}{0} 
 % }
% }
%
%       
%
%%%%% Personnalisation des numéros des sections
\renewcommand\thesection{\Roman{section}. }
\renewcommand\thesubsection{\hspace{1cm}\arabic{subsection}. }
\renewcommand\thesubsubsection{\hspace{2cm}\alph{subsubsection}. }

\titleformat{\section}[hang]{\color{sacado_purple}{}\normalfont\filright\huge}{}{0.4em}{\textbf{\thesection  #1}}   
% \titlespacing*{\section}{0.2pt}{0ex plus 0ex minus 0ex}{0.3em}
   
\titleformat{\subsection}[hang]{\color{sacado_purple}{}\normalfont\filright\Large}{}{0.4em}{\thesubsection
 #1}            
\titleformat{\subsubsection}[hang]{\color{sacado_purple}{}\normalfont\filright\large}{}{0.4em}{\thesubsubsection
 #1}
\titleformat{\paragraph}[hang]{\color{black}{}\normalfont\filright\normalsize}{}{0.4em}{#1}



%%%%%%%%%%%%%%%%%%%%% Cycle 4
%\newcommand{\Titre}[2]{\section*{#1 
%\ifthenelse{\equal{#2}{1}}   {\hfill{ \ding{182}  \ding{173} \ding{174} } \addcontentsline{toc}{section}{#1 \ding{182}} }%
%{%
%\ifthenelse{\equal{#2}{2}}{\hfill{ \ding{172}  \ding{183} \ding{174} } \addcontentsline{toc}{section}{#1 {\color{purple}\ding{183}}} }{%           
%\hfill{ \ding{172}  \ding{173} \ding{184} } \addcontentsline{toc}{section}{#1 {\color{orange}\ding{184}}}% 
%}%
%}%
%}
%}


%%%%%%%%%%%%%%%%%%%%% Cycle 4
\newcommand{\Titre}[2]{\section*{#1 
\ifthenelse{\equal{#2}{1}}   {\hfill{ \ding{182}  \ding{173} \ding{174} } \addcontentsline{toc}{section}{#1 \, \ding{182}} }%
{% sinon
\ifthenelse{\equal{#2}{1,5}}   {\hfill{ \ding{182}  \ding{183} \ding{174} } \addcontentsline{toc}{section}{#1 \, \ding{182} {\color{purple}\ding{183}}} }%
{% sinon
\ifthenelse{\equal{#2}{2}}   {\hfill{ \ding{172}  \ding{183} \ding{174} } \addcontentsline{toc}{section}{#1 \, {\color{purple}\ding{183}}} }
{% sinon
\ifthenelse{\equal{#2}{2,5}}   {\hfill{ \ding{172}  \ding{183} \ding{184} } \addcontentsline{toc}{section}{#1 \, {\color{purple}\ding{183}}  {\color{orange}\ding{184}}} }%
{% sinon
\hfill{ \ding{172}  \ding{173} \ding{184} } \addcontentsline{toc}{section}{#1 \,{\color{orange}\ding{184}}}% 
}%
}%
}%
}%
}%
}

%%%%%%%%%%%%% Titre
\newenvironment{titre}[2][]{%
\vspace{0.5cm}
\begin{tcolorbox}[enhanced, lifted shadow={0mm}{0mm}{0mm}{0mm}%
{black!60!white}, attach boxed title to top left={xshift=110mm, yshift*=-3mm}, coltitle=violet, colback=bleu3!25!white, boxed title style={colback=white!100}, colframe=bleu3,title=\stepcounter{cpttitre} \textbf{Fiche \thecpttitre}. #1 #2 ]}
{%
\end{tcolorbox}
\par}



%%%%%%%%%%%%% Définitions
\newenvironment{Def}[1][]{%
\medskip \begin{tcolorbox}[widget,colback=sacado_violet!0,colframe=sacado_violet!75,
adjusted title= \stepcounter{cptdef} Définition \thecptdef . {#1} ]}
{%
\end{tcolorbox}\par}


\newenvironment{DefT}[2][]{%
\medskip \begin{tcolorbox}[widget,colback=sacado_violet!0,colframe=sacado_violet!75,
adjusted title= \stepcounter{cptdef} Définition \thecptdef . {#1} \textit{#2}]}
{%
\end{tcolorbox}\par}

%%%%%%%%%%%%% Proposition
\newenvironment{Prop}[1][]{%
\medskip \begin{tcolorbox}[widget,colback=sacado_blue!0,colframe=sacado_blue!75!black,
adjusted title= \stepcounter{cpttheo} Proposition \thecpttheo . {#1} ]}
{%
\end{tcolorbox}\par}

%%%%%%%%%%%%% Propriétés
\newenvironment{Pp}[1][]{%
\medskip \begin{tcolorbox}[widget,colback=sacado_blue!0,colframe=sacado_blue!75!black,
adjusted title= \stepcounter{cpttheo} Propriété \thecpttheo . {#1}]}
{%
\end{tcolorbox}\par}

\newenvironment{PpT}[2][]{%
\medskip \begin{tcolorbox}[widget,colback=sacado_blue!0,colframe=sacado_blue!75!black,
adjusted title= \stepcounter{cpttheo} Propriété \thecpttheo . {#1} #2]}
{%
\end{tcolorbox}\par}

\newenvironment{Pps}[1][]{%
\medskip \begin{tcolorbox}[widget,colback=sacado_blue!0,colframe=sacado_blue!75!black,
adjusted title= \stepcounter{cpttheo} Propriétés \thecpttheo . {#1}]}
{%
\end{tcolorbox}\par}

%%%%%%%%%%%%% Théorèmes
\newenvironment{ThT}[2][]{% théorème avec titre
\medskip \begin{tcolorbox}[widget,colback=sacado_blue!0,colframe=sacado_blue!75!black,
adjusted title= \stepcounter{cpttheo} Théorème \thecpttheo . {#1} #2]}
{%
\end{tcolorbox}\par}

\newenvironment{Th}[1][]{%
\medskip \begin{tcolorbox}[widget,colback=sacado_blue!0,colframe=sacado_blue!75!black,
adjusted title= \stepcounter{cpttheo} Théorème \thecpttheo . {#1}]}
{%
\end{tcolorbox}\par}

%%%%%%%%%%%%% Règles
\newenvironment{Reg}[1][]{%
\medskip \begin{tcolorbox}[widget,colback=sacado_blue!0,colframe=sacado_blue!75!black,
adjusted title= \stepcounter{cpttheo} Règle \thecpttheo . {#1}]}
{%
\end{tcolorbox}\par}

%%%%%%%%%%%%% REMARQUES
\newenvironment{Rq}[1][]{%
\begin{bclogo}[couleur=sacado_orange!0, arrondi =0.15, noborder=true, couleurBarre=sacado_orange, logo = \bcinfo ]{ 
{\color{info}\normalsize{Remarque#1}}}}
{%
\end{bclogo}
\par}


\newenvironment{Rqs}[1][]{%
\begin{bclogo}[couleur=sacado_orange!0, arrondi =0.15, noborder=true, couleurBarre=sacado_orange, logo = \bcinfo ]{ 
{\color{info}\normalsize{Remarques#1}}}}
{%
\end{bclogo}
\par}

%%%%%%%%%%%%% EXEMPLES
\newenvironment{Ex}[1][]{%
\begin{bclogo}[couleur=sacado_yellow!15, arrondi =0.15, noborder=true, couleurBarre=sacado_yellow, logo = \bclampe ]{ 
\normalsize{Exemple#1}}}
{%
\end{bclogo}
\par}




%%%%%%%%%%%%% Preuve
\newenvironment{Pv}[1][]{%
\begin{tcolorbox}[breakable, enhanced,widget, colback=sacado_blue!10!white,boxrule=0pt,frame hidden,
borderline west={1mm}{0mm}{sacado_blue!75}]
\textbf{Preuve#1 : }}
{%
\end{tcolorbox}
\par}


%%%%%%%%%%%%% PreuveROC
\newenvironment{PvR}[1][]{%
\begin{tcolorbox}[breakable, enhanced,widget, colback=sacado_blue!10!white,boxrule=0pt,frame hidden,
borderline west={1mm}{0mm}{sacado_blue!75}]
\textbf{Preuve (ROC)#1 : }}
{%
\end{tcolorbox}
\par}


%%%%%%%%%%%%% Compétences
\newenvironment{Cps}[1][]{%
\vspace{0.4cm}
\begin{tcolorbox}[enhanced, lifted shadow={0mm}{0mm}{0mm}{0mm}%
{black!60!white}, attach boxed title to top left={xshift=5mm, yshift*=-3mm}, coltitle=white, colback=white, boxed title style={colback=sacado_green!100}, colframe=sacado_green!75!black,title=\textbf{Compétences associées#1}]}
{%
\end{tcolorbox}
\par}

%%%%%%%%%%%%% Compétences Collège
\newenvironment{CpsCol}[1][]{%
\vspace{0.4cm}
\begin{tcolorbox}[breakable, enhanced,widget, colback=white ,boxrule=0pt,frame hidden,
borderline west={2mm}{0mm}{bleu3}]
\textbf{#1}}
{%
\end{tcolorbox}
\par}




%%%%%%%%%%%%% Attendus
\newenvironment{Ats}[1][]{%
\vspace{0.4cm}
\begin{tcolorbox}[enhanced, lifted shadow={0mm}{0mm}{0mm}{0mm}%
{black!60!white}, attach boxed title to top left={xshift=5mm, yshift*=-3mm}, coltitle=white, colback=white, boxed title style={colback=sacado_green!100}, colframe=sacado_green!75!black,title=\textbf{Attendus du chapitre#1}]}
{%
\end{tcolorbox}
\par}

%%%%%%%%%%%%% Méthode
\newenvironment{Mt}[1][]{%
\vspace{0.4cm}
\begin{bclogo}[couleur=sacado_blue!0, arrondi =0.15, noborder=true, couleurBarre=bleu3, logo = \bccrayon ]{ 
\normalsize{{\color{bleu3}Méthode #1}}}}
{%
\end{bclogo}
\par}


%%%%%%%%%%%%% Méthode en vidéo
\newcommand{\MtV}[2]{\vspace{0.4cm} \colorbox{sacado_blue!0}{\hspace{0.2 cm}\tikz\node[rounded corners=1pt,draw] {\color{red}$\blacktriangleright$}; \quad  \href{https://youtu.be/#1?rel=0}{\raisebox{0.8mm}{{\color{red}\textbf{Méthode en vidéo : #2}}}}}}


%%%%%%%%%%%%% A voir (AV) : Lien externe + vidéo non Youtube
\newcommand{\AV}[2]{\vspace{0.4cm} \colorbox{bleu1!0}{\hspace{0.2 cm}\tikz\node[rounded corners=1pt,draw] {\color{red}$\blacktriangleright$}; \quad  \href{#1}{\raisebox{0.8mm}{{\color{red}\textbf{#2}}}}}}


%%%%%%%%%%%%% Etymologie
\newenvironment{Ety}[1][]{%
\begin{bclogo}[couleur=sacado_green!0, arrondi =0.15, noborder=true, couleurBarre=sacado_green, logo = \bcplume ]{ 
\normalsize{{\color{sacado_green}Étymologie#1}}}}
{%
\end{bclogo}
\par}


%%%%%%%%%%%%% Notation
\newenvironment{Nt}[1][]{%
\begin{bclogo}[couleur=sacado_violet!0, arrondi =0.15, noborder=true, couleurBarre=sacado_violet!75, logo = \bccrayon ]{ 
\normalsize{{\color{violet!75}Notation#1}}}}
{%
\end{bclogo}
\par}
%%%%%%%%%%%%% Histoire
%\newenvironment{His}[1][]{%
%\begin{bclogo}[couleur=brown!30, arrondi =0.15, noborder=true, couleurBarre=brown, logo = \bcvaletcoeur ]{ 
%\normalsize{{\color{brown}Histoire des mathématiques#1}}}}
%{%
%\end{bclogo}
%\par}

\newenvironment{His}[1][]{%
\vspace{0.4cm}
\begin{tcolorbox}[enhanced, lifted shadow={0mm}{0mm}{0mm}{0mm}%
{brown!60!white}, attach boxed title to top left={xshift=5mm, yshift*=-3mm}, coltitle=white, colback=white, boxed title style={colback=brown!100}, colframe=brown!75!black,title=\textbf{Histoire des mathématiques#1}]}
{%
\end{tcolorbox}
\par}

%%%%%%%%%%%%% Attention
\newenvironment{Att}[1][]{%
\begin{bclogo}[couleur=red!0, arrondi =0.15, noborder=true, couleurBarre=red, logo = \bcattention ]{ 
\normalsize{{\color{red}Attention. #1}}}}
{%
\end{bclogo}
\par}


%%%%%%%%%%%%% Conséquence
\newenvironment{Cq}[1][]{%
\textbf{Conséquence #1}}
{%
\par}

%%%%%%%%%%%%% Vocabulaire
\newenvironment{Voc}[1][]{%
\setlength{\logowidth}{10pt}
%\begin{footnotesize}
\begin{bclogo}[ noborder , couleur=white, logo =\bcbook]{#1}}
{%
\end{bclogo}
%\end{footnotesize}
\par}


%%%%%%%%%%%%% Video
\newenvironment{Vid}[1][]{%
\setlength{\logowidth}{12pt}
\begin{bclogo}[ noborder , couleur=white,barre=none, logo =\bcoeil]{#1}}
{%
\end{bclogo}
\par}


%%%%%%%%%%%%% Syntaxe
\newenvironment{Syn}[1][]{%
\begin{bclogo}[couleur=violet!0, arrondi =0.15, noborder=true, couleurBarre=violet!75, logo = \bcicosaedre ]{ 
\normalsize{{\color{violet!75}Syntaxe#1}}}}
{%
\end{bclogo}
\par}

%%%%%%%%%%%%% Auto évaluation
\newenvironment{autoeval}[1][]{%
\vspace{0.4cm}
\begin{tcolorbox}[enhanced, lifted shadow={0mm}{0mm}{0mm}{0mm}%
{black!60!white}, attach boxed title to top left={xshift=5mm, yshift*=-3mm}, coltitle=white, colback=white, boxed title style={colback=sacado_green!100}, colframe=sacado_green!75!black,title=\textbf{J'évalue mes compétences#1}]}
{%
\end{tcolorbox}
\par}


\newenvironment{autotest}[1][]{%
\vspace{0.4cm}
\begin{tcolorbox}[enhanced, lifted shadow={0mm}{0mm}{0mm}{0mm}%
{red!60!white}, attach boxed title to top left={xshift=5mm, yshift*=-3mm}, coltitle=white, colback=white, boxed title style={colback=red!100}, colframe=red!75!black,title=\textbf{Pour faire le point #1}]}
{%
\end{tcolorbox}
\par}

\newenvironment{ExOApp}[1][]{% Exercice d'application direct
\vspace{0.4cm}
\begin{tcolorbox}[enhanced, lifted shadow={0mm}{0mm}{0mm}{0mm}%
{red!60!white}, attach boxed title to top left={xshift=5mm, yshift*=-3mm}, coltitle=white, colback=white, boxed title style={colback=sacado_green!100}, colframe=sacado_green!75!black,title=\textbf{Application #1}]}
{%
\end{tcolorbox}
\par}

\newenvironment{ExOInt}[1][]{% Exercice d'application direct
\vspace{0.4cm}
\begin{tcolorbox}[enhanced, lifted shadow={0mm}{0mm}{0mm}{0mm}%
{red!60!white}, attach boxed title to top left={xshift=5mm, yshift*=-3mm}, coltitle=white, colback=white, boxed title style={colback=sacado_green!50}, colframe=sacado_green!75!black,title=\textbf{Exercice #1}]}
{%
\end{tcolorbox}
\par}

%Illustrations
\newtcolorbox{Illqr}[1]{
  enhanced,
  colback=white,
  colframe=ill_frame,
  colbacktitle=ill_back,
  coltitle=ill_title,
  title=\textbf{Illustration},
  boxrule=1pt, % épaisseur du trait à 1pt
  center,
  overlay={
    \node[anchor=south east, inner sep=0pt,xshift=-1pt,yshift=2pt,fill=white] at (frame.south east) {\fancyqr[height=1cm]{#1}};
  },
  after=\par,
  before=\vspace{0.4cm},
}

\newtcolorbox{Ill}{
  enhanced,
  colback=white,
  colframe=ill_frame,
  colbacktitle=ill_back,
  coltitle=ill_title,
  title=\textbf{Illustration},
  boxrule=1pt, % épaisseur du trait à 1pt
  center,
  after=\par,
  before=\vspace{0.4cm},
}

%%%%%%%%%%%%%% Propriétés
%\newenvironment{Pp}[1][]{%
%\medskip \begin{tcolorbox}[widget,colback=sacado_blue!0,colframe=sacado_blue!75!black,
%adjusted title= \stepcounter{cpttheo} Propriété \thecpttheo . {#1}]}
%{%
%\end{tcolorbox}\par}

%%%%% Pour réinitialiser numéros des chapitres après une nouvelle partie
% \makeatletter
    % \@addtoreset{section}{part}
% \makeatother

% \newcommand{\EPC}[3]{ % Exercice par compétence de niveau 1
% \ifthenelse{\equal{#1}{1}}
% {%condition2 vraie
% \vspace{0.4cm}
% \stepcounter{cptex}
% \tikz\node[rounded corners=0pt,draw,fill=bleu2]{\color{white}\textbf{ \thecptex}}; \quad  {\color{bleu2}\textbf{#3}}
% \input{#2}
% }% fin condition2 vraie
% {%condition2 fausse
% \vspace{0.4cm}
% \stepcounter{cptex}
% \tikz\node[rounded corners=2pt,draw,fill=eduscol4P]{\color{white}\textbf{ \thecptex}}; \quad  {\color{eduscol4P} \textbf{En temps libre.} \textbf{ #3}} 
% \input{#2}
% }% fin condition2 fausse
% } % fin de la procédure

\usepackage{hyperref}

\begin{document}

%-------------------------------
%	TITLE SECTION
%-------------------------------

\fancyhead[C]{}
\hrule\medskip % Upper rule
\begin{minipage}{0.295\textwidth} 
\raggedright
Classe \myClasse \hfill\\
\myDiscipline \hfill\\
\myParcours \hfill\\
\end{minipage}
\begin{minipage}{0.4\textwidth} 
\centering 
\scshape\huge
\textcolor{sacado_purple}{\myTitle} \\ 
\normalsize 
%\mySubTitle \\ 
\end{minipage}
\begin{minipage}{0.295\textwidth} 
\raggedleft
\href{https://sacado.xyz/}{\includegraphics[width=.2\linewidth]{sacadoA1.png}}
%\myAnnee \hfill\\
\end{minipage}
\medskip \hrule
\bigskip

%-------------------------------
%	CONTENTS
%-------------------------------

%\chapter{Droites parallèles et perpendiculaires}
%{https://sacado.xyz/qcm/parcours_show_course/0/117127}
%

%\begin{His}
%La géométrie étudiée au collège est la \textit{géométrie euclidienne} du savant grec \textbf{Euclide} vivant à Alexandrie au IIIe siècle avant J.C. Il a écrit 13 livres appelés 'Les éléments' dans lesquels il va démontrer la plupart des propositions géométriques à partir d'une petite série de postulats ou axiomes (points de départs).\\
%
%\begin{Def}
%Un postulat est une proposition qu'on demande d'admettre au commencement d'une discussion.
%\end{Def}
%
%Les cinq postulats de la géométrie euclidienne sont les suivants :
%\begin{itemize}
%\item \textbf{Postulat 1} : Par deux points distincts, il passe une droite et une seule.
%\item \textbf{Postulat 2} : Tout segment est prolongeable en une droite.
%\item \textbf{Postulat 3} : Deux points distincts étant donnés, il passe un cercle et un seul de centre le premier point et passant par le second.
%\item \textbf{Postulat 4} : Tous les angles droits sont égaux entre eux.
%\item \textbf{Postulat 5} : Il n'existe qu'une seule parallèle à une droite donnée passant par un point extérieur à cette droite.
%\end{itemize}
%C'est à partir de ces cinq postulats qu'on peut construire toute la géométrie.
%\end{His}

%\begin{pageCours}

\section{Droites sécantes}

\mini{.58\linewidth}
{
\begin{Def}
Lorsque deux droites \textcolor{orange}{$(d_1)$} et \textcolor{qqqqff}{$(d_2)$} se coupent en un point $A$ on dit qu'elles sont sécantes en $A$.
On dit que $A$ est le point d'intersection des droites \textcolor{orange}{$(d_1)$} et \textcolor{qqqqff}{$(d_2)$}.
\end{Def}
}
{.38\linewidth}
{
\begin{Ill}
\begin{center}
\begin{tikzpicture}[line cap=round,line join=round,>=triangle 45,x=0.5cm,y=0.5cm]
\clip(3,-3) rectangle (14,3.52);
\draw [line width=2.pt,color=qqqqff,domain=2.24:15.28] plot(\x,{(--25.75-3.3*\x)/8.9});
\draw [line width=2.pt,color=orange,domain=2.24:15.28] plot(\x,{(-57.3296--6.36*\x)/11.92});
\draw [color=orange](4.7,-.6) node[anchor=north west] {$(d_1)$};
\draw [color=qqqqff](5.18,1.9) node[anchor=north west] {$(d_2)$};
\draw (8.28,0.7) node[anchor=north west] {$A$};
\end{tikzpicture}
\end{center}
\end{Ill}
}

\mini{.48\linewidth}{\qr{}{Determiner deux droites sécantes en un point donné}}{.48\linewidth}{\qr{}{Déterminer le point d'intersection de deux droites}}

\section{Droites perpendiculaires}

\mini{.58\linewidth}{
\begin{Def}
Deux droites sont \textbf{perpendiculaires} lorsqu'elles se coupent en formant un \textbf{angle droit}.
\end{Def}
\begin{Nt}
Lorsque deux droites $(d_1)$ et $(d_2)$ sont perpendiculaires,  on note $(d_1)\perp(d_2)$.
\end{Nt}
}
{.38\linewidth}{
\begin{Ill}
\begin{center}
\begin{tikzpicture}[line cap=round,line join=round,>=triangle 45,x=0.5cm,y=0.5cm]
%\clip(1.98,-3.92) rectangle (13.46,4.02);
\clip(2.24,-4.04) rectangle (15.28,3.52);
\draw[line width=2.pt,color=xdxdff,fill=xdxdff,fill opacity=0.10000000149011612] (7.567363216384809,-0.29103294643190725) -- (7.193047343953587,-0.49075181796400164) -- (7.392766215485682,-0.8650676903952232) -- (7.767082087916903,-0.6653488188631289) -- cycle; 
\draw [line width=2.pt,color=orange,domain=1.98:13.46] plot(\x,{(-57.3296--6.36*\x)/11.92});
\draw [color=orange](4.52,-1.06) node[anchor=north west] {$(d_1)$};
\draw [color=qqqqff](4.64,2.1) node[anchor=north west] {$(d_2)$};
\draw (8.,0.70) node[anchor=north west] {$A$};
\draw [line width=2.pt,color=qqqqff,domain=1.98:13.46] plot(\x,{(-88.352--11.92*\x)/-6.36});
\end{tikzpicture}
\end{center}
\end{Ill}}

\mini{.68\linewidth}{
\begin{Rqs}
\begin{itemize}
\item Deux droites perpendiculaires sont sécantes.
\item Deux droites sécantes ne sont pas forcément perpendiculaires.
\end{itemize}
\end{Rqs}}{.28\linewidth}{\qr{•}{Reconnaître des droites perpendiculaires}}

\mini{.68\linewidth}
{\begin{Pp}
Soit $(d)$ une droite et $A$ un point.
Il existe une \textbf{unique droite} passant par $A$
et perpendiculaire à $(d)$.
\end{Pp}}
{.28\linewidth}{\begin{Mt}
\qr{}{Construire la droite perpendiculaire à une droite passant par un point donné.}
\end{Mt}}

\section{Droites parallèles}


\mini{.58\linewidth}{
\begin{Def}
Deux droites sont \textbf{parallèles} lorsqu'elles \textbf{ne se coupent pas}.
\end{Def}
\begin{Nt}
Lorsque deux droites $(d_1)$ et $(d_2)$ sont parallèles,  on note $(d_1)//(d_2)$.
\end{Nt}
}
{.38\linewidth}{
\begin{Ill}
\begin{center}
\begin{tikzpicture}[line cap=round,line join=round,>=triangle 45,x=0.5cm,y=0.5cm]
%\clip(1.,-3.5) rectangle (12.46,3.32);
\clip(2.24,-4.04) rectangle (15.28,3.52);
\draw [line width=2.pt,color=orange,domain=1.:12.46] plot(\x,{(-57.3296--6.36*\x)/11.92});
\draw [color=orange](4.52,-.8) node[anchor=north west] {$(d_1)$};
\draw [color=qqqqff](4.64,2.1) node[anchor=north west] {$(d_2)$};
\draw [line width=2.pt,color=qqqqff,domain=1.:12.46] plot(\x,{(-22.8992--6.36*\x)/11.92});
\end{tikzpicture}
\end{center}
\end{Ill}}

\mini{.68\linewidth}{
\begin{Rq}
Deux droites sont soit \textbf{parallèles} soit \textbf{sécantes}.
\end{Rq}}{.28\linewidth}{\qr{•}{Reconnaître des droites parallèles ou sécantes}}

\mini{.68\linewidth}
{\begin{Pp}
Soit $(d)$ une droite et $A$ un point.
Il existe une \textbf{unique droite} passant par $A$
et parallèle à $(d)$.
\end{Pp}}
{.28\linewidth}{\begin{Mt}
\qr{}{Construire la parallèle à une droite passant par un point donné.}
\end{Mt}}

\section{Propriétés}

\mini{.58\linewidth}{
\begin{Pp}
\textbf{Si} deux droites sont parallèles à une même droite, \textbf{alors} elles sont parallèles entre elles.
\end{Pp}
\begin{Nt}
Si  $(d_1)//(d_2)$ et  $(d_1)//(d_3)$ alors on peut affirmer que : $(d_2)//(d_3)$.
\end{Nt}
}
{.38\linewidth}{
\begin{Ill}
\begin{center}
\begin{tikzpicture}[line cap=round,line join=round,>=triangle 45,x=0.5cm,y=0.5cm]
%\clip(0.78,-3.94) rectangle (12.6,2.88);
\clip(2.24,-4.04) rectangle (15.28,3.52);
\draw [line width=2.pt,color=orange,domain=0.78:12.6] plot(\x,{(-57.3296--6.36*\x)/11.92});
\draw [color=orange](4.52,-1.06) node[anchor=north west] {$(d_1)$};
\draw [color=qqqqff](4.64,1.92) node[anchor=north west] {$(d_2)$};
\draw [line width=2.pt,color=qqqqff,domain=0.78:12.6] plot(\x,{(-22.8992--6.36*\x)/11.92});
\draw [line width=2.pt,domain=0.78:12.6] plot(\x,{(-69.9816--6.36*\x)/11.92});
\draw (8.64,-1.62) node[anchor=north west] {$(d_3)$};
\end{tikzpicture}
\end{center}
\end{Ill}}

\mini{.58\linewidth}{
\begin{Pp}
\textbf{Si} deux droites sont perpendiculaires à une même droite, \textbf{alors} elles sont parallèles entre elles.
\end{Pp}
\begin{Nt}
Si $(d_2)\perp(d_1)$ et $(d_3)\perp(d_1)$ alors on peut affirmer que : $(d_2)//(d_3)$.
\end{Nt}
}
{.38\linewidth}{
\begin{Ill}
\begin{center}
\begin{tikzpicture}[line cap=round,line join=round,>=triangle 45,x=0.5cm,y=0.5cm]
\clip(0.64,-4.06) rectangle (11.66,2.7);
\draw[line width=2.pt,color=xdxdff,fill=xdxdff,fill opacity=0.10000000149011612] (4.670102723253892,1.1145231636719009) -- (4.295234489709246,0.9158429998932381) -- (4.493914653487909,0.5409747663485918) -- (4.868782887032555,0.7396549301272544) -- cycle; 
\draw[line width=2.pt,color=xdxdff,fill=xdxdff,fill opacity=0.10000000149011612] (5.897183681954806,-1.2007239282166144) -- (5.5223154484101595,-1.399404091995277) -- (5.720995612188823,-1.7742723255399235) -- (6.095863845733469,-1.575592161761261) -- cycle; 
\draw [line width=2.pt,color=orange,domain=0.64:11.66] plot(\x,{(-9.6128--1.06*\x)/2.});
\draw [color=orange](4,-1.06) node[anchor=north west] {$(d_1)$};
\draw [color=qqqqff](4.64,2.2) node[anchor=north west] {$(d_2)$};
\draw [line width=2.pt,color=qqqqff,domain=0.64:11.66] plot(\x,{(-3.6816--1.06*\x)/2.});
\draw (6.92,-2.26) node[anchor=north west] {$(d_3)$};
\draw [line width=2.pt,domain=0.64:11.66] plot(\x,{(-10.5216--2.*\x)/-1.06});
\end{tikzpicture}
\end{center}
\end{Ill}}

\mini{.58\linewidth}{
\begin{Pp}
\textbf{Si} deux droites sont parallèles, toute perpendiculaire à l'une est \textbf{alors} perpendiculaire à l'autre.
\end{Pp}
\begin{Nt}
Si $(d_1)//(d_2)$  et  $(d_1)\perp(d_3)$ alors on peut affirmer que : $(d_2)\perp(d_3)$.
\end{Nt}
}
{.38\linewidth}{
\begin{Ill}
\begin{center}
\begin{tikzpicture}[line cap=round,line join=round,>=triangle 45,x=0.5cm,y=0.5cm]
\clip(0.64,-4.06) rectangle (11.66,2.7);
\draw[line width=2.pt,color=xdxdff,fill=xdxdff,fill opacity=0.10000000149011612] (4.670102723253892,1.1145231636719009) -- (4.295234489709246,0.9158429998932381) -- (4.493914653487909,0.5409747663485918) -- (4.868782887032555,0.7396549301272544) -- cycle; 
\draw[line width=2.pt,color=xdxdff,fill=xdxdff,fill opacity=0.10000000149011612] (5.897183681954806,-1.2007239282166144) -- (5.5223154484101595,-1.399404091995277) -- (5.720995612188823,-1.7742723255399235) -- (6.095863845733469,-1.575592161761261) -- cycle; 
\draw [line width=2.pt,color=orange,domain=0.64:11.66] plot(\x,{(-9.6128--1.06*\x)/2.});
\draw [color=orange](4,-1.06) node[anchor=north west] {$(d_1)$};
\draw [color=qqqqff](4.64,2.2) node[anchor=north west] {$(d_2)$};
\draw [line width=2.pt,color=qqqqff,domain=0.64:11.66] plot(\x,{(-3.6816--1.06*\x)/2.});
\draw (6.92,-2.26) node[anchor=north west] {$(d_3)$};
\draw [line width=2.pt,domain=0.64:11.66] plot(\x,{(-10.5216--2.*\x)/-1.06});
\end{tikzpicture}
\end{center}
\end{Ill}}

\section{La médiatrice d'un segment}

\mini{.58\linewidth}{
\begin{Def}
La \textbf{médiatrice} d'une segment est la droite passant par le \textbf{milieu} de ce segment \textbf{perpendiculairement}.
\end{Def}
\begin{Nt}
$(d)$ est la médiatrice de $[AB]$ signifie :
\begin{itemize}
\item $(d)$ passe par le milieu de $[AB]$.
\item $(d)$ et $(AB)$ sont perpendiculaires.
\end{itemize}
\end{Nt}
}
{.38\linewidth}{
\begin{Ill}
\begin{center}
\begin{tikzpicture}[line cap=round,line join=round,>=triangle 45,x=0.5cm,y=0.5cm]
\clip(-1.86,-3.04) rectangle (9.78,5.28);
\draw[line width=2.pt,color=xdxdff,fill=xdxdff,fill opacity=0.10000000149011612] (3.282146544641254,1.1068490608270765) -- (2.965297483814177,1.3889956054683306) -- (2.683150939172923,1.072146544641254) -- (3.,0.79) -- cycle; 
\draw [line width=2.pt,color=qqqqff,domain=-1.86:9.78] plot(\x,{(-9.6454--4.2*\x)/3.74});
\draw [line width=2.pt,color=orange] (0.9,2.66)-- (3.,0.79);
\draw [line width=1.pt,color=orange] (2.109606188003346,1.9042368956187326) -- (1.790393811996652,1.5457631043812687);
\draw [line width=2.pt,color=orange] (3.,0.79)-- (5.1,-1.08);
\draw [line width=1.pt,color=orange] (4.209606188003345,0.034236895618731764) -- (3.890393811996651,-0.32423689561873187);
\draw [color=qqqqff](4,4) node[anchor=north west] {$(d)$};
\draw [line width=2.pt,color=orange] (0.9,2.66)-- ++(-2.5pt,-2.5pt) -- ++(5.0pt,5.0pt);% ++(-5.0pt,0) -- ++(5.0pt,-5.0pt);
\draw[color=orange] (1.3,3.3) node {$A$};
\draw [line width=2.pt,color=orange] (5.1,-1.08)-- ++(-2.5pt,-2.5pt) -- ++(5.0pt,5.0pt);% ++(-5.0pt,0) -- ++(5.0pt,-5.0pt);
\draw[color=orange] (5.24,-0.2) node {$B$};
\end{tikzpicture}
\end{center}
\end{Ill}}

\begin{Mt}
\qr{}{Construire la médiatrice d'un segment avec une équerre.}
\end{Mt}

\mini{.58\linewidth}{
\begin{Pp}
Tous les \textbf{points de la médiatrice} d'un segment sont à \textbf{égale distance} des \textbf{extrémités} de ce segment.
\end{Pp}
\begin{Nt}
Si $(d)$ est la médiatrice de $[AB]$ alors :
\begin{itemize}
\item Si $M\in(d)$ alors $MA=MB$.
\item Si $MA=MB$ alors $M\in(d)$. 
\end{itemize}
\end{Nt}
}
{.38\linewidth}{
\begin{Illqr}{Manipuler le point M}
\begin{center}
\begin{tikzpicture}[line cap=round,line join=round,>=triangle 45,x=0.5cm,y=0.5cm]
\clip(-1.86,-3.04) rectangle (9.78,5.28);
\draw[line width=2.pt,color=xdxdff,fill=xdxdff,fill opacity=0.10000000149011612] (3.282146544641254,1.1068490608270765) -- (2.965297483814177,1.3889956054683306) -- (2.683150939172923,1.072146544641254) -- (3.,0.79) -- cycle; 
\draw [line width=2.pt,color=qqqqff,domain=-1.86:9.78] plot(\x,{(-9.6454--4.2*\x)/3.74});
\draw [line width=2.pt,color=orange] (0.9,2.66)-- (3.,0.79);
\draw [line width=1.pt,color=orange] (2.109606188003346,1.9042368956187326) -- (1.790393811996652,1.5457631043812687);
\draw [line width=2.pt,color=orange] (3.,0.79)-- (5.1,-1.08);
\draw [line width=1.pt,color=orange] (4.209606188003345,0.034236895618731764) -- (3.890393811996651,-0.32423689561873187);
\draw [color=qqqqff](4,4) node[anchor=north west] {$(d)$};
\draw [line width=2.pt] (1.1139414941380312,-1.328033616208628)-- (0.9,2.66);
\draw [line width=1.pt] (0.8898214896297482,0.6096267239312342) -- (1.1294768878442067,0.6224832439374486);
\draw [line width=1.pt] (0.8844646062938268,0.709483139853925) -- (1.1241200045082855,0.7223396598601394);
\draw [line width=2.pt] (1.1139414941380312,-1.328033616208628)-- (5.1,-1.08);
\draw [line width=1.pt] (3.0496146469026013,-1.0873537129383606) -- (3.064519886121612,-1.326890419611522);
\draw [line width=1.pt] (3.1494216080164192,-1.0811431965971057) -- (3.1643268472354302,-1.3206799032702672);
\draw [line width=2.pt,color=orange] (0.9,2.66)-- ++(-2.5pt,-2.5pt) -- ++(5.0pt,5.0pt);% ++(-5.0pt,0) -- ++(5.0pt,-5.0pt);
\draw[color=orange] (1.3,3.3) node {$A$};
\draw [line width=2.pt,color=orange] (5.1,-1.08)-- ++(-2.5pt,-2.5pt) -- ++(5.0pt,5.0pt);% ++(-5.0pt,0) -- ++(5.0pt,-5.0pt);
\draw[color=orange] (5.24,-0.2) node {$B$};
%\draw [color=black] (1.1139414941380312,-1.328033616208628)-- ++(-2.5pt,-2.5pt) -- ++(5.0pt,5.0pt) ++(-5.0pt,0) -- ++(5.0pt,-5.0pt);
\draw[color=black] (0.42,-1.01) node {$M$};
\end{tikzpicture}
\end{center}
\end{Illqr}}

\begin{Mt}
\qr{}{Construire la médiatrice d'un segment avec un compas.}
\end{Mt}

\begin{Rq}
Tu peux construire un angle droit sans équerre (avec un compas et une règle) !!!!
\end{Rq}

\section{Distance d'un point à une droite}

\begin{Def}
La \textbf{distance} d'un \textbf{point} à une \textbf{droite} est la longueur du \textbf{plus petit} segment reliant ce point à l'un des points de la droite.
\end{Def}


\mini{.58\linewidth}{
\begin{Def}
La distance d’un point $A$ à une droite $(d)$ est la longueur du segment reliant le point $A$ au pied de la perpendiculaire à $(d)$ passant par ce même point $A$.
\end{Def}}
{.38\linewidth}{
\begin{Illqr}{Déterminer la distance entre un point et une droite}
\begin{center}
\begin{tikzpicture}[line cap=round,line join=round,>=triangle 45,x=0.5cm,y=0.5cm]
\clip(-5.72,-0.86) rectangle (5.86,6.92);
\draw[line width=2.pt,color=xdxdff,fill=xdxdff,fill opacity=0.10000000149011612] (-0.020116391374110087,2.7030211271476134) -- (-0.32777806381733166,2.4108836675340957) -- (-0.035640604203813986,2.103221995090874) -- (0.2720210682394075,2.395359454704392) -- cycle; 
\draw [line width=2.pt,color=qqqqff,domain=-5.72:5.86] plot(\x,{(--9.3176--4.14*\x)/4.36});
\draw [color=qqqqff](2.26,3.88) node[anchor=north west] {$(d)$};
\draw [line width=1.pt,dash pattern=on 1pt off 3pt,color=black!40] (-2.6,5.42)-- (-1.9175072200767929,0.3163119515784584);
\draw [line width=1.pt,dash pattern=on 1pt off 3pt,color=black!40] (-2.6,5.42)-- (0.2720210682394075,2.395359454704392);
\draw [color=black] (-2.6,5.42)-- ++(-2.5pt,-2.5pt) -- ++(5.0pt,5.0pt) ++(-5.0pt,0) -- ++(5.0pt,-5.0pt);
\draw[color=black] (-2.46,5.79) node {$A$};
\draw[color=black!40] (-4,2.59) node {$5,15\,cm$};
\draw[color=black!40] (0.5,4.35) node {$4,17\,cm$};
\end{tikzpicture}
\end{center}
\end{Illqr}}




\section{Exercice bilan}

\qr{•}{Exercice de construction bilan}

\section{Les savoir-faire du parcours}

\begin{CpsCol}
\begin{itemize}
\item Savoir déterminer le point d'intersection de deux droites
\item Savoir déterminer deux droites ayant un point d'intersection donné
\item Savoir reconnaitre des droites sécantes.
\item Savoir construire un angle droit
\item Savoir construire deux droites perpendiculaires.
\item Savoir reconnaitre des droites perpendiculaires.
\item Savoir construire la perpendiculaire à une droite passant par un point.
\item Savoir construire deux droites parallèles.
\item Savoir reconnaitre des droites parallèles.
\item Savoir construire la parallèle à une droite passant par un point.
\item Savoir reconnaître la médiatrice d'un segment. 
\item Savoir construire la médiatrice d'un segment. 
\item Savoir déterminer la distance d'un point à une droite.
\end{itemize}
\end{CpsCol}

\end{document}

%\end{pageCours}
