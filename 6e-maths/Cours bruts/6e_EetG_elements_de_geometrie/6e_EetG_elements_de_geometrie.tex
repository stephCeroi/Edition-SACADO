\documentclass[a4paper,dvipsnames]{article}

\addtolength{\hoffset}{-2.25cm}
\addtolength{\textwidth}{4.5cm}
\addtolength{\voffset}{-3.25cm}
\addtolength{\textheight}{5cm}
\setlength{\parskip}{0pt}
\setlength{\parindent}{0in}

%----------------------------------------------------------------------------------------
%	PACKAGES AND OTHER DOCUMENT CONFIGURATIONS
%----------------------------------------------------------------------------------------

%----------------------------------------------------------------------------------------
%		Generals
%----------------------------------------------------------------------------------------
\usepackage{fourier}
\usepackage{frcursive}
\usepackage[T1]{fontenc} %Accents handling
\usepackage[utf8]{inputenc} % Use UTF-8 encoding
%\usepackage{microtype} % Slightly tweak font spacing for aesthetics
\usepackage[english, francais]{babel} % Language hyphenation and typographical rules

%----------------------------------------------------------------------------------------
%		Graphics
%----------------------------------------------------------------------------------------
\usepackage{xcolor}
\usepackage{graphicx, multicol} % Enhanced support for graphics
\graphicspath{{FIG/}}
\usepackage{wrapfig}

%----------------------------------------------------------------------------------------
%		Other packages
%----------------------------------------------------------------------------------------
\usepackage{hyperref}
\hypersetup{
	colorlinks=true, %colorise les liens
	breaklinks=true, %permet le retour à la ligne dans les liens trop longs
	urlcolor= bleu3,  %couleur des hyperliens
	linkcolor= bleu3, %couleur des liens internes
	plainpages=false  %pour palier à "Bookmark problems can occur when you have duplicate page numbers, for example, if you have a page i and a page 1."
}
\usepackage{tabularx}
\newcolumntype{M}[1]{>{\arraybackslash}m{#1}} %Defines a scalable column type in tabular
\usepackage{booktabs} % Enhances quality of tables
\usepackage{diagbox} % barre en diagonale dans un tableau
\usepackage{multicol}
\usepackage[explicit]{titlesec}


%----------------------------------------------------------------------------------------
%		Headers and footers
%----------------------------------------------------------------------------------------
\usepackage{fancyhdr} % Headers and footers
\pagestyle{fancy} % All pages have headers and footers
\fancyhead{}\renewcommand{\headrulewidth}{0pt} % Blank out the default header
\renewcommand{\footrulewidth}{0pt}
\fancyfoot[L]{} % Custom footer text
\fancyfoot[C]{\href{https://sacado.xyz/}{sacado.xyz}} % Custom footer text
\fancyfoot[R]{\thepage} % Custom footer text

%----------------------------------------------------------------------------------------
%		Mathematics packages
%----------------------------------------------------------------------------------------
\usepackage{amsthm, amsmath, amssymb} % Mathematical typesetting
\usepackage{marvosym, wasysym} % More symbols
\usepackage[makeroom]{cancel}
\usepackage{xlop}
\usepackage{pgf,tikz,pgfplots}
\pgfplotsset{compat=1.15}
\usetikzlibrary{positioning}
%\usetikzlibrary{arrows}
\usepackage{pst-plot,pst-tree,pst-func, pstricks-add,pst-node,pst-text}
\usepackage{units}
\usepackage{nicefrac}
\usepackage[np]{numprint} %Séparation milliers dans un nombre

%----------------------------------------------------------------------------------------
%		New text commands
%----------------------------------------------------------------------------------------
\usepackage{calc}
\usepackage{boites}
 \renewcommand{\arraystretch}{1.6}

%%%%% Pour les imports.
\usepackage{import}

%%%%% Pour faire des boites
\usepackage[tikz]{bclogo}
\usepackage{bclogo}
\usepackage{framed}
\usepackage[skins]{tcolorbox}
\tcbuselibrary{breakable}
\tcbuselibrary{skins}
\usetikzlibrary{babel,arrows,shadows,decorations.pathmorphing,decorations.markings,patterns}

%%%%% Pour les symboles et les ensembles
\newcommand{\pp}{\leq}
\newcommand{\pg}{\geq}
%\newcommand{\euro}{\eurologo{}}
\newcommand{\R}{\mathbb{R}}
\newcommand{\N}{\mathbb{N}}
\newcommand{\D}{\mathbb{D}}
\newcommand{\Z}{\mathbb{Z}}
\newcommand{\Q}{\mathbb{Q}}
\newcommand{\C}{\mathbb{C}}

%%%%% Pour une double minipage
\newcommand{\mini}[2]{
\begin{minipage}[t]{0.48\linewidth}
#1
\end{minipage}
\hfill
\begin{minipage}[t]{0.48\linewidth}
#2
\end{minipage}
}


%\newcommand\hole[1]{\texttt{\_}}
%\newcommand{\PROP}[1]{\textbf{\underline{#1}}}
%\newcommand{\exercice}{\textcolor{OliveGreen}{Exercice : }}
%\newcommand{\correction}{\textcolor{BurntOrange}{Correction : }}
%\newcommand{\propriete}{\textbf{\underline{Propriété}} : }
%\newcommand{\prop}{\textbf{\underline{Propriété}} : }
%\newcommand{\vocabulaire}{\textbf{\underline{Vocabulaire}} : }
%\newcommand{\voca}{\textbf{\underline{Vocabulaire}} : }

\usepackage{enumitem}
\newlist{todolist}{itemize}{2} %Pour faire des QCM
\setlist[todolist]{label=$\square$} %Pour faire des QCM \begin{todolist} instead of itemize

%----------------------------------------------------------------------------------------
%		Définition de couleur pour geogebra
%----------------------------------------------------------------------------------------
\definecolor{zzttqq}{rgb}{0.6,0.2,0.} %rouge des polygones
\definecolor{qqqqff}{rgb}{0.,0.,1.}
\definecolor{xdxdff}{rgb}{0.49019607843137253,0.49019607843137253,1.}%bleu
\definecolor{qqwuqq}{rgb}{0.,0.39215686274509803,0.} %vert des angles
\definecolor{ffqqqq}{rgb}{1.,0.,0.} %rouge vif
\definecolor{uuuuuu}{rgb}{0.26666666666666666,0.26666666666666666,0.26666666666666666}
\definecolor{qqzzqq}{rgb}{0.,0.6,0.}
\definecolor{cqcqcq}{rgb}{0.7529411764705882,0.7529411764705882,0.7529411764705882} %gris
\definecolor{qqffqq}{rgb}{0.,1.,0.}
\definecolor{ffdxqq}{rgb}{1.,0.8431372549019608,0.}
\definecolor{ffffff}{rgb}{1.,1.,1.}
\definecolor{ududff}{rgb}{0.30196078431372547,0.30196078431372547,1.}

%-------------------------------------------------
%
%	EN TETE
%
%-------------------------------------------------

% Classe
\newcommand{\myClasse}   
{
    6e
}

% Discipline
\newcommand{\myDiscipline}   
{
    Mathématiques
}

% Parcours
\newcommand{\myParcours}
{
  Nombres et Calculs
}

%Titre de la séquence
\newcommand{\myTitle}
{
    \scshape\huge
\textcolor{sacado_purple}{
		Enchainements d'opérations
}
}

%----------------------------------------------------------------------------------------

%----------------------------------------------------------------------------------------
%		Définition de couleur pour les boites
%----------------------------------------------------------------------------------------
\definecolor{bleu1}{rgb}{0.54,0.79,0.95} %% Bleu
\definecolor{sapgreen}{rgb}{0.4, 0.49, 0}
\definecolor{dvzfxr}{rgb}{0.7,0.4,0.}
\definecolor{beamer}{rgb}{0.5176470588235295,0.49019607843137253,0.32941176470588235} % couleur beamer
\definecolor{preuveRbeamer}{rgb}{0.8,0.4,0}
\definecolor{sectioncolor}{rgb}{0.24,0.21,0.44}
\definecolor{subsectioncolor}{rgb}{0.1,0.21,0.61}
\definecolor{subsubsectioncolor}{rgb}{0.1,0.21,0.61}
\definecolor{info}{rgb}{0.82,0.62,0}
\definecolor{bleu2}{rgb}{0.38,0.56,0.68}
\definecolor{bleu3}{rgb}{0.24,0.34,0.40}
\definecolor{bleu4}{rgb}{0.12,0.20,0.25}
\definecolor{vert}{rgb}{0.21,0.33,0}
\definecolor{vertS}{rgb}{0.05,0.6,0.42}
\definecolor{red}{rgb}{0.78,0,0}
\definecolor{color5}{rgb}{0,0.4,0.58}
\definecolor{eduscol4B}{rgb}{0.19,0.53,0.64}
\definecolor{eduscol4P}{rgb}{0.62,0.12,0.39}

%----------------------------------------------------------------------------------------
%		Définition de couleur pour les boites SACADO
%----------------------------------------------------------------------------------------
\definecolor{sacado_blue}{RGB}{0,129,159} %% Bleu Sacado
\definecolor{sacado_green}{RGB}{59, 157, 38} %% Vert Sacado
\definecolor{sacado_yellow}{RGB}{255,180,0} %% Jaune Sacado
\definecolor{sacado_purple}{RGB}{94,68,145} %% Violet foncé Sacado
\definecolor{sacado_violet}{RGB}{153,117,224} %% Violet clair Sacado
\definecolor{sacado_orange}{RGB}{249,168,100} %% Orange Sacado
\definecolor{ill_frame}{HTML}{F0F0F0}
\definecolor{ill_back}{HTML}{F7F7F7}
\definecolor{ill_title}{HTML}{AAAAAA}


 % Compteurs pour Théorème, Définition, Exemple, Remarque, .....
\newcounter{cpttheo}
\setcounter{cpttheo}{0}
\newcounter{cptdef}
\setcounter{cptdef}{0}
\newcounter{cptmth}
\setcounter{cptmth}{0}
\newcounter{cpttitre}
\setcounter{cpttitre}{0}
 % Exercices
\newcounter{cptapp}
\setcounter{cptapp}{0}
\newcounter{cptex}
\setcounter{cptex}{0}
\newcounter{cptsr}
\setcounter{cptsr}{0}
\newcounter{cpti}
\setcounter{cpti}{0}
\newcounter{cptcor}
\setcounter{cptcor}{0}




%%%%% Pour réinitialiser numéros des paragraphes après une nouvelle partie
\makeatletter
    \@addtoreset{paragraph}{part}
\makeatother


%%%% Titres et sections

\newlength\chapnumb
\setlength\chapnumb{3cm}


% \titleformat{\part}[block] {
 % \normalfont\sffamily\color{violet}}{}{0pt} {
   % \parbox[t]{\chapnumb}{\fontsize{120}{110}\selectfont\ding{110}}
   % \parbox[b]{\dimexpr\textwidth-\chapnumb\relax}{
       % \raggedleft
       % \hfill{{\color{bleu3}\fontsize{40}{30}\selectfont#1}}\\
       % \rule{0.99\textwidth-\chapnumb\relax}{0.4pt}
 % }
% }

% \titleformat{name=\part,numberless}[block]
% {\normalfont\sffamily\color{bleu3}}{}{0pt}
% {\parbox[b]{\chapnumb}{%
  % \mbox{}}%
 % \parbox[b]{\dimexpr\textwidth-\chapnumb\relax}{%
   % \raggedleft%
   % \hfill{{\color{bleu3}\fontsize{40}{30}\selectfont#1}}\\
   % \rule{0.99\textwidth-\chapnumb\relax}{0.4pt}
 % }
% }



% \titleformat{\chapter}[block] {
 % \normalfont\sffamily\color{violet}}{}{0pt} {
   % \parbox[t]{\chapnumb}{ 
     % \fontsize{120}{110}\selectfont\thechapter}
     % \parbox[b]{\textwidth-\chapnumb}{
       % \raggedleft
       % \hfill{{\color{bleu3}\huge#1}}\\  
  % \ifthenelse{\thechapter<10}{\rule{0.99\textwidth-\chapnumb}{0.4pt}}{\rule{0.9\textwidth - \chapnumb}{0.4pt}}
       % \setcounter{cpttitre}{0}
	% \setcounter{cptapp}{0}
	% \setcounter{cptex}{0}
	% \setcounter{cptsr}{0}
	% \setcounter{cpti}{0}
	% \setcounter{cptcor}{0} 
 % }
% }

% \titleformat{name=\chapter,numberless}[block]
% {\normalfont\sffamily\color{bleu3}}{}{0pt}
% {\parbox[b]{\chapnumb}{%
  % \mbox{}}%
 % \parbox[b]{\textwidth-\chapnumb}{%
   % \raggedleft
   % \hfill{{\color{bleu3}\huge#1}}\\
   % \ifthenelse{\thechapter<10}{\rule{0.99\textwidth-\chapnumb}{0.4pt}}{ \rule{0.9\textwidth - \chapnumb}{0.4pt}}
       % \setcounter{cpttitre}{0}
	% \setcounter{cptapp}{0}
	% \setcounter{cptex}{0}
	% \setcounter{cptsr}{0}
	% \setcounter{cpti}{0}
	% \setcounter{cptcor}{0} 
 % }
% }
%
%       
%
%%%%% Personnalisation des numéros des sections
\renewcommand\thesection{\Roman{section}. }
\renewcommand\thesubsection{\hspace{1cm}\arabic{subsection}. }
\renewcommand\thesubsubsection{\hspace{2cm}\alph{subsubsection}. }

\titleformat{\section}[hang]{\color{sacado_purple}{}\normalfont\filright\huge}{}{0.4em}{\textbf{\thesection  #1}}   
% \titlespacing*{\section}{0.2pt}{0ex plus 0ex minus 0ex}{0.3em}
   
\titleformat{\subsection}[hang]{\color{sacado_purple}{}\normalfont\filright\Large}{}{0.4em}{\thesubsection
 #1}            
\titleformat{\subsubsection}[hang]{\color{sacado_purple}{}\normalfont\filright\large}{}{0.4em}{\thesubsubsection
 #1}
\titleformat{\paragraph}[hang]{\color{black}{}\normalfont\filright\normalsize}{}{0.4em}{#1}



%%%%%%%%%%%%%%%%%%%%% Cycle 4
%\newcommand{\Titre}[2]{\section*{#1 
%\ifthenelse{\equal{#2}{1}}   {\hfill{ \ding{182}  \ding{173} \ding{174} } \addcontentsline{toc}{section}{#1 \ding{182}} }%
%{%
%\ifthenelse{\equal{#2}{2}}{\hfill{ \ding{172}  \ding{183} \ding{174} } \addcontentsline{toc}{section}{#1 {\color{purple}\ding{183}}} }{%           
%\hfill{ \ding{172}  \ding{173} \ding{184} } \addcontentsline{toc}{section}{#1 {\color{orange}\ding{184}}}% 
%}%
%}%
%}
%}


%%%%%%%%%%%%%%%%%%%%% Cycle 4
\newcommand{\Titre}[2]{\section*{#1 
\ifthenelse{\equal{#2}{1}}   {\hfill{ \ding{182}  \ding{173} \ding{174} } \addcontentsline{toc}{section}{#1 \, \ding{182}} }%
{% sinon
\ifthenelse{\equal{#2}{1,5}}   {\hfill{ \ding{182}  \ding{183} \ding{174} } \addcontentsline{toc}{section}{#1 \, \ding{182} {\color{purple}\ding{183}}} }%
{% sinon
\ifthenelse{\equal{#2}{2}}   {\hfill{ \ding{172}  \ding{183} \ding{174} } \addcontentsline{toc}{section}{#1 \, {\color{purple}\ding{183}}} }
{% sinon
\ifthenelse{\equal{#2}{2,5}}   {\hfill{ \ding{172}  \ding{183} \ding{184} } \addcontentsline{toc}{section}{#1 \, {\color{purple}\ding{183}}  {\color{orange}\ding{184}}} }%
{% sinon
\hfill{ \ding{172}  \ding{173} \ding{184} } \addcontentsline{toc}{section}{#1 \,{\color{orange}\ding{184}}}% 
}%
}%
}%
}%
}%
}

%%%%%%%%%%%%% Titre
\newenvironment{titre}[2][]{%
\vspace{0.5cm}
\begin{tcolorbox}[enhanced, lifted shadow={0mm}{0mm}{0mm}{0mm}%
{black!60!white}, attach boxed title to top left={xshift=110mm, yshift*=-3mm}, coltitle=violet, colback=bleu3!25!white, boxed title style={colback=white!100}, colframe=bleu3,title=\stepcounter{cpttitre} \textbf{Fiche \thecpttitre}. #1 #2 ]}
{%
\end{tcolorbox}
\par}



%%%%%%%%%%%%% Définitions
\newenvironment{Def}[1][]{%
\medskip \begin{tcolorbox}[widget,colback=sacado_violet!0,colframe=sacado_violet!75,
adjusted title= \stepcounter{cptdef} Définition \thecptdef . {#1} ]}
{%
\end{tcolorbox}\par}


\newenvironment{DefT}[2][]{%
\medskip \begin{tcolorbox}[widget,colback=sacado_violet!0,colframe=sacado_violet!75,
adjusted title= \stepcounter{cptdef} Définition \thecptdef . {#1} \textit{#2}]}
{%
\end{tcolorbox}\par}

%%%%%%%%%%%%% Proposition
\newenvironment{Prop}[1][]{%
\medskip \begin{tcolorbox}[widget,colback=sacado_blue!0,colframe=sacado_blue!75!black,
adjusted title= \stepcounter{cpttheo} Proposition \thecpttheo . {#1} ]}
{%
\end{tcolorbox}\par}

%%%%%%%%%%%%% Propriétés
\newenvironment{Pp}[1][]{%
\medskip \begin{tcolorbox}[widget,colback=sacado_blue!0,colframe=sacado_blue!75!black,
adjusted title= \stepcounter{cpttheo} Propriété \thecpttheo . {#1}]}
{%
\end{tcolorbox}\par}

\newenvironment{PpT}[2][]{%
\medskip \begin{tcolorbox}[widget,colback=sacado_blue!0,colframe=sacado_blue!75!black,
adjusted title= \stepcounter{cpttheo} Propriété \thecpttheo . {#1} #2]}
{%
\end{tcolorbox}\par}

\newenvironment{Pps}[1][]{%
\medskip \begin{tcolorbox}[widget,colback=sacado_blue!0,colframe=sacado_blue!75!black,
adjusted title= \stepcounter{cpttheo} Propriétés \thecpttheo . {#1}]}
{%
\end{tcolorbox}\par}

%%%%%%%%%%%%% Théorèmes
\newenvironment{ThT}[2][]{% théorème avec titre
\medskip \begin{tcolorbox}[widget,colback=sacado_blue!0,colframe=sacado_blue!75!black,
adjusted title= \stepcounter{cpttheo} Théorème \thecpttheo . {#1} #2]}
{%
\end{tcolorbox}\par}

\newenvironment{Th}[1][]{%
\medskip \begin{tcolorbox}[widget,colback=sacado_blue!0,colframe=sacado_blue!75!black,
adjusted title= \stepcounter{cpttheo} Théorème \thecpttheo . {#1}]}
{%
\end{tcolorbox}\par}

%%%%%%%%%%%%% Règles
\newenvironment{Reg}[1][]{%
\medskip \begin{tcolorbox}[widget,colback=sacado_blue!0,colframe=sacado_blue!75!black,
adjusted title= \stepcounter{cpttheo} Règle \thecpttheo . {#1}]}
{%
\end{tcolorbox}\par}

%%%%%%%%%%%%% REMARQUES
\newenvironment{Rq}[1][]{%
\begin{bclogo}[couleur=sacado_orange!0, arrondi =0.15, noborder=true, couleurBarre=sacado_orange, logo = \bcinfo ]{ 
{\color{info}\normalsize{Remarque#1}}}}
{%
\end{bclogo}
\par}


\newenvironment{Rqs}[1][]{%
\begin{bclogo}[couleur=sacado_orange!0, arrondi =0.15, noborder=true, couleurBarre=sacado_orange, logo = \bcinfo ]{ 
{\color{info}\normalsize{Remarques#1}}}}
{%
\end{bclogo}
\par}

%%%%%%%%%%%%% EXEMPLES
\newenvironment{Ex}[1][]{%
\begin{bclogo}[couleur=sacado_yellow!15, arrondi =0.15, noborder=true, couleurBarre=sacado_yellow, logo = \bclampe ]{ 
\normalsize{Exemple#1}}}
{%
\end{bclogo}
\par}




%%%%%%%%%%%%% Preuve
\newenvironment{Pv}[1][]{%
\begin{tcolorbox}[breakable, enhanced,widget, colback=sacado_blue!10!white,boxrule=0pt,frame hidden,
borderline west={1mm}{0mm}{sacado_blue!75}]
\textbf{Preuve#1 : }}
{%
\end{tcolorbox}
\par}


%%%%%%%%%%%%% PreuveROC
\newenvironment{PvR}[1][]{%
\begin{tcolorbox}[breakable, enhanced,widget, colback=sacado_blue!10!white,boxrule=0pt,frame hidden,
borderline west={1mm}{0mm}{sacado_blue!75}]
\textbf{Preuve (ROC)#1 : }}
{%
\end{tcolorbox}
\par}


%%%%%%%%%%%%% Compétences
\newenvironment{Cps}[1][]{%
\vspace{0.4cm}
\begin{tcolorbox}[enhanced, lifted shadow={0mm}{0mm}{0mm}{0mm}%
{black!60!white}, attach boxed title to top left={xshift=5mm, yshift*=-3mm}, coltitle=white, colback=white, boxed title style={colback=sacado_green!100}, colframe=sacado_green!75!black,title=\textbf{Compétences associées#1}]}
{%
\end{tcolorbox}
\par}

%%%%%%%%%%%%% Compétences Collège
\newenvironment{CpsCol}[1][]{%
\vspace{0.4cm}
\begin{tcolorbox}[breakable, enhanced,widget, colback=white ,boxrule=0pt,frame hidden,
borderline west={2mm}{0mm}{bleu3}]
\textbf{#1}}
{%
\end{tcolorbox}
\par}




%%%%%%%%%%%%% Attendus
\newenvironment{Ats}[1][]{%
\vspace{0.4cm}
\begin{tcolorbox}[enhanced, lifted shadow={0mm}{0mm}{0mm}{0mm}%
{black!60!white}, attach boxed title to top left={xshift=5mm, yshift*=-3mm}, coltitle=white, colback=white, boxed title style={colback=sacado_green!100}, colframe=sacado_green!75!black,title=\textbf{Attendus du chapitre#1}]}
{%
\end{tcolorbox}
\par}

%%%%%%%%%%%%% Méthode
\newenvironment{Mt}[1][]{%
\vspace{0.4cm}
\begin{bclogo}[couleur=sacado_blue!0, arrondi =0.15, noborder=true, couleurBarre=bleu3, logo = \bccrayon ]{ 
\normalsize{{\color{bleu3}Méthode #1}}}}
{%
\end{bclogo}
\par}


%%%%%%%%%%%%% Méthode en vidéo
\newcommand{\MtV}[2]{\vspace{0.4cm} \colorbox{sacado_blue!0}{\hspace{0.2 cm}\tikz\node[rounded corners=1pt,draw] {\color{red}$\blacktriangleright$}; \quad  \href{https://youtu.be/#1?rel=0}{\raisebox{0.8mm}{{\color{red}\textbf{Méthode en vidéo : #2}}}}}}


%%%%%%%%%%%%% A voir (AV) : Lien externe + vidéo non Youtube
\newcommand{\AV}[2]{\vspace{0.4cm} \colorbox{bleu1!0}{\hspace{0.2 cm}\tikz\node[rounded corners=1pt,draw] {\color{red}$\blacktriangleright$}; \quad  \href{#1}{\raisebox{0.8mm}{{\color{red}\textbf{#2}}}}}}


%%%%%%%%%%%%% Etymologie
\newenvironment{Ety}[1][]{%
\begin{bclogo}[couleur=sacado_green!0, arrondi =0.15, noborder=true, couleurBarre=sacado_green, logo = \bcplume ]{ 
\normalsize{{\color{sacado_green}Étymologie#1}}}}
{%
\end{bclogo}
\par}


%%%%%%%%%%%%% Notation
\newenvironment{Nt}[1][]{%
\begin{bclogo}[couleur=sacado_violet!0, arrondi =0.15, noborder=true, couleurBarre=sacado_violet!75, logo = \bccrayon ]{ 
\normalsize{{\color{violet!75}Notation#1}}}}
{%
\end{bclogo}
\par}
%%%%%%%%%%%%% Histoire
%\newenvironment{His}[1][]{%
%\begin{bclogo}[couleur=brown!30, arrondi =0.15, noborder=true, couleurBarre=brown, logo = \bcvaletcoeur ]{ 
%\normalsize{{\color{brown}Histoire des mathématiques#1}}}}
%{%
%\end{bclogo}
%\par}

\newenvironment{His}[1][]{%
\vspace{0.4cm}
\begin{tcolorbox}[enhanced, lifted shadow={0mm}{0mm}{0mm}{0mm}%
{brown!60!white}, attach boxed title to top left={xshift=5mm, yshift*=-3mm}, coltitle=white, colback=white, boxed title style={colback=brown!100}, colframe=brown!75!black,title=\textbf{Histoire des mathématiques#1}]}
{%
\end{tcolorbox}
\par}

%%%%%%%%%%%%% Attention
\newenvironment{Att}[1][]{%
\begin{bclogo}[couleur=red!0, arrondi =0.15, noborder=true, couleurBarre=red, logo = \bcattention ]{ 
\normalsize{{\color{red}Attention. #1}}}}
{%
\end{bclogo}
\par}


%%%%%%%%%%%%% Conséquence
\newenvironment{Cq}[1][]{%
\textbf{Conséquence #1}}
{%
\par}

%%%%%%%%%%%%% Vocabulaire
\newenvironment{Voc}[1][]{%
\setlength{\logowidth}{10pt}
%\begin{footnotesize}
\begin{bclogo}[ noborder , couleur=white, logo =\bcbook]{#1}}
{%
\end{bclogo}
%\end{footnotesize}
\par}


%%%%%%%%%%%%% Video
\newenvironment{Vid}[1][]{%
\setlength{\logowidth}{12pt}
\begin{bclogo}[ noborder , couleur=white,barre=none, logo =\bcoeil]{#1}}
{%
\end{bclogo}
\par}


%%%%%%%%%%%%% Syntaxe
\newenvironment{Syn}[1][]{%
\begin{bclogo}[couleur=violet!0, arrondi =0.15, noborder=true, couleurBarre=violet!75, logo = \bcicosaedre ]{ 
\normalsize{{\color{violet!75}Syntaxe#1}}}}
{%
\end{bclogo}
\par}

%%%%%%%%%%%%% Auto évaluation
\newenvironment{autoeval}[1][]{%
\vspace{0.4cm}
\begin{tcolorbox}[enhanced, lifted shadow={0mm}{0mm}{0mm}{0mm}%
{black!60!white}, attach boxed title to top left={xshift=5mm, yshift*=-3mm}, coltitle=white, colback=white, boxed title style={colback=sacado_green!100}, colframe=sacado_green!75!black,title=\textbf{J'évalue mes compétences#1}]}
{%
\end{tcolorbox}
\par}


\newenvironment{autotest}[1][]{%
\vspace{0.4cm}
\begin{tcolorbox}[enhanced, lifted shadow={0mm}{0mm}{0mm}{0mm}%
{red!60!white}, attach boxed title to top left={xshift=5mm, yshift*=-3mm}, coltitle=white, colback=white, boxed title style={colback=red!100}, colframe=red!75!black,title=\textbf{Pour faire le point #1}]}
{%
\end{tcolorbox}
\par}

\newenvironment{ExOApp}[1][]{% Exercice d'application direct
\vspace{0.4cm}
\begin{tcolorbox}[enhanced, lifted shadow={0mm}{0mm}{0mm}{0mm}%
{red!60!white}, attach boxed title to top left={xshift=5mm, yshift*=-3mm}, coltitle=white, colback=white, boxed title style={colback=sacado_green!100}, colframe=sacado_green!75!black,title=\textbf{Application #1}]}
{%
\end{tcolorbox}
\par}

\newenvironment{ExOInt}[1][]{% Exercice d'application direct
\vspace{0.4cm}
\begin{tcolorbox}[enhanced, lifted shadow={0mm}{0mm}{0mm}{0mm}%
{red!60!white}, attach boxed title to top left={xshift=5mm, yshift*=-3mm}, coltitle=white, colback=white, boxed title style={colback=sacado_green!50}, colframe=sacado_green!75!black,title=\textbf{Exercice #1}]}
{%
\end{tcolorbox}
\par}

%Illustrations
\newtcolorbox{Illqr}[1]{
  enhanced,
  colback=white,
  colframe=ill_frame,
  colbacktitle=ill_back,
  coltitle=ill_title,
  title=\textbf{Illustration},
  boxrule=1pt, % épaisseur du trait à 1pt
  center,
  overlay={
    \node[anchor=south east, inner sep=0pt,xshift=-1pt,yshift=2pt,fill=white] at (frame.south east) {\fancyqr[height=1cm]{#1}};
  },
  after=\par,
  before=\vspace{0.4cm},
}

\newtcolorbox{Ill}{
  enhanced,
  colback=white,
  colframe=ill_frame,
  colbacktitle=ill_back,
  coltitle=ill_title,
  title=\textbf{Illustration},
  boxrule=1pt, % épaisseur du trait à 1pt
  center,
  after=\par,
  before=\vspace{0.4cm},
}

%%%%%%%%%%%%%% Propriétés
%\newenvironment{Pp}[1][]{%
%\medskip \begin{tcolorbox}[widget,colback=sacado_blue!0,colframe=sacado_blue!75!black,
%adjusted title= \stepcounter{cpttheo} Propriété \thecpttheo . {#1}]}
%{%
%\end{tcolorbox}\par}

%%%%% Pour réinitialiser numéros des chapitres après une nouvelle partie
% \makeatletter
    % \@addtoreset{section}{part}
% \makeatother

% \newcommand{\EPC}[3]{ % Exercice par compétence de niveau 1
% \ifthenelse{\equal{#1}{1}}
% {%condition2 vraie
% \vspace{0.4cm}
% \stepcounter{cptex}
% \tikz\node[rounded corners=0pt,draw,fill=bleu2]{\color{white}\textbf{ \thecptex}}; \quad  {\color{bleu2}\textbf{#3}}
% \input{#2}
% }% fin condition2 vraie
% {%condition2 fausse
% \vspace{0.4cm}
% \stepcounter{cptex}
% \tikz\node[rounded corners=2pt,draw,fill=eduscol4P]{\color{white}\textbf{ \thecptex}}; \quad  {\color{eduscol4P} \textbf{En temps libre.} \textbf{ #3}} 
% \input{#2}
% }% fin condition2 fausse
% } % fin de la procédure

\usepackage{hyperref}

\begin{document}

%-------------------------------
%	TITLE SECTION
%-------------------------------

\fancyhead[C]{}
\hrule\medskip % Upper rule
\begin{minipage}{0.295\textwidth} 
\raggedright
Classe \myClasse \hfill\\
\myDiscipline \hfill\\
\myParcours \hfill\\
\end{minipage}
\begin{minipage}{0.4\textwidth} 
\centering 
\scshape\huge
\textcolor{sacado_purple}{\myTitle} \\ 
\normalsize 
%\mySubTitle \\ 
\end{minipage}
\begin{minipage}{0.295\textwidth} 
\raggedleft
\href{https://sacado.xyz/}{\includegraphics[width=.2\linewidth]{sacadoA1.png}}
%\myAnnee \hfill\\
\end{minipage}
\medskip \hrule
\bigskip

%-------------------------------
%	CONTENTS
%-------------------------------

\section{Illusions d'optique}

Quelques fois il arrive que nos yeux nous trompent.

\subsection{Les segments sont-ils de même longueur ?}

\begin{center}
\begin{tikzpicture}[line cap=round,line join=round,>=triangle 45,x=0.5cm,y=0.5cm]
\clip(-6.335217913319962,-6.880345876695259) rectangle (12.328090050851376,2.7730892771864735);
\draw [line width=1.pt] (0.,0.)-- (5.,0.);
\draw [line width=1.pt] (0.,-3.)-- (5.,-3.);
\draw [line width=1.pt] (5.,0.)-- (6.,1.);
\draw [line width=1.pt] (5.,0.)-- (6.,-1.);
\draw [line width=1.pt] (0.,0.)-- (-1.,1.);
\draw [line width=1.pt] (0.,0.)-- (-1.,-1.);
\draw [line width=1.pt] (0.,-3.)-- (1.,-2.);
\draw [line width=1.pt] (0.,-3.)-- (1.,-4.);
\draw [line width=1.pt] (5.,-3.)-- (4.,-2.);
\draw [line width=1.pt] (5.,-3.)-- (4.,-4.);
\end{tikzpicture}
\end{center}

\subsection{Les droites violettes sont-elles parallèles ?}

\begin{center}
\definecolor{xfqqff}{rgb}{0.4980392156862745,0.,1.}
\begin{tikzpicture}[line cap=round,line join=round,>=triangle 45,x=1.0cm,y=1.0cm]
\clip(-6.367032765272981,-3.2831809059084223) rectangle (7.49507762775438,3.930960866483946);
\draw [line width=1.pt] (0.,-3.2831809059084223) -- (0.,3.930960866483946);
\draw [line width=1.pt,domain=-6.367032765272981:7.49507762775438] plot(\x,{(-0.-0.*\x)/2.});
\draw [line width=1.pt,domain=-6.367032765272981:7.49507762775438] plot(\x,{(-0.--0.17364817766693033*\x)/0.984807753012208});
\draw [line width=1.pt,domain=-6.367032765272981:7.49507762775438] plot(\x,{(-0.--0.34202014332566866*\x)/0.9396926207859083});
\draw [line width=1.pt,domain=-6.367032765272981:7.49507762775438] plot(\x,{(-0.--0.5*\x)/0.8660254037844386});
\draw [line width=1.pt,domain=-6.367032765272981:7.49507762775438] plot(\x,{(-0.--0.6427876096865391*\x)/0.766044443118978});
\draw [line width=1.pt,domain=-6.367032765272981:7.49507762775438] plot(\x,{(-0.--0.7660444431189779*\x)/0.6427876096865393});
\draw [line width=1.pt,domain=-6.367032765272981:7.49507762775438] plot(\x,{(-0.--0.8660254037844385*\x)/0.5});
\draw [line width=1.pt,domain=-6.367032765272981:7.49507762775438] plot(\x,{(-0.-0.9396926207859082*\x)/-0.34202014332566877});
\draw [line width=1.pt,domain=-6.367032765272981:7.49507762775438] plot(\x,{(-0.-0.9848077530122078*\x)/-0.17364817766693044});
\draw [line width=1.pt,domain=-6.367032765272981:7.49507762775438] plot(\x,{(-0.-0.984807753012208*\x)/0.17364817766693033});
\draw [line width=1.pt,domain=-6.367032765272981:7.49507762775438] plot(\x,{(-0.-0.9396926207859083*\x)/0.34202014332566866});
\draw [line width=1.pt,domain=-6.367032765272981:7.49507762775438] plot(\x,{(-0.-0.8660254037844386*\x)/0.5});
\draw [line width=1.pt,domain=-6.367032765272981:7.49507762775438] plot(\x,{(-0.-0.766044443118978*\x)/0.6427876096865391});
\draw [line width=1.pt,domain=-6.367032765272981:7.49507762775438] plot(\x,{(-0.-0.6427876096865393*\x)/0.7660444431189779});
\draw [line width=1.pt,domain=-6.367032765272981:7.49507762775438] plot(\x,{(-0.-0.5*\x)/0.8660254037844385});
\draw [line width=1.pt,domain=-6.367032765272981:7.49507762775438] plot(\x,{(-0.-0.34202014332566877*\x)/0.9396926207859082});
\draw [line width=1.pt,domain=-6.367032765272981:7.49507762775438] plot(\x,{(-0.-0.17364817766693044*\x)/0.9848077530122078});
\draw [line width=1.pt,color=sacado_violet,domain=-6.367032765272981:7.49507762775438] plot(\x,{(--1.5-0.*\x)/1.});
\draw [line width=1.pt,color=sacado_violet,domain=-6.367032765272981:7.49507762775438] plot(\x,{(-1.5-0.*\x)/1.});
\end{tikzpicture}
\end{center}

\section{La géométrie Euclidienne}

\begin{His}
La géométrie étudiée au collège est la \textbf{géométrie euclidienne}. Ce nom vient du savant grec \textbf{Euclide} vivant à Alexandrie au IIIe siècle avant J.C. On le surnomme le père de la géométrie. Son œuvre \textcolor{sacado_orange}{Les Éléments} (recueil de 13 livres) comprend une collection de \textcolor{sacado_orange}{définitions} et de \textcolor{sacado_orange}{postulats}, ainsi que des \textcolor{sacado_orange}{propriétés} et des \textcolor{sacado_orange}{théorèmes} et leurs démonstrations.

\begin{Def}
\begin{itemize}
\item \textcolor{sacado_orange}{Les Éléments} : Les Éléments (Euclide) vers 300 av. J.C.

Les livres I à IV traitent de géométrie plane.

Les livres V à X font intervenir les proportions.

Les livres XI à XIII traitent de la géométrie dans l'espace.


\item \textcolor{sacado_orange}{Définition} : Énoncé qui précise la nature d'un objet mathématique.
\begin{Ex}
\begin{itemize}
\item Un triangle équilatéral est un triangle dont les trois côtés sont de même longueur.
\item Un nombre entier est un nombre qui peut être écrit sans virgule.
\end{itemize}
\end{Ex}
\item \textcolor{sacado_orange}{Postulat} : Énoncé considéré comme admis sans démonstration. Les postulats servent de base pour le monde de la géométrie.
\begin{Ex}
\begin{itemize}
\item Par deux points passent une et une seule droite.
\item Deux droites non parallèles se croisent en un point et un seul.
\item Il existe qu'une seule droite passant par un point et parallèle à une autre droite.
\end{itemize}
\end{Ex}
\item \textcolor{sacado_orange}{Propriétés} : Énoncé concernant un objet qui est démontré. On l'utilise pour prouver des affirmations.
\begin{Ex}
\begin{itemize}
\item Les diagonales d'un rectangle sont de même longueur.
\item La somme de deux nombres entier est un nombre entier.
\end{itemize}
\end{Ex}
\item \textcolor{sacado_orange}{Théorème} : Énoncé qui est démontré par un raisonnement logique. On l'utilise pour prouver des affirmations.
\begin{Ex}
Le théorème de M. Pythagore dont tu as sûrement déjà entendu parler et que tu verras en quatrième.
\end{Ex}
\end{itemize}
\end{Def}
\end{His}

\begin{Def}
Le monde de la géométrie est un ensemble de points, tous les éléments de géométrie sont composés de points.
\end{Def}

\section{Éléments de géométrie}

\subsection{Droites}

\begin{Def}
Une droite est un ensemble de point qui sont alignés.
\end{Def}

\begin{Ex}
\begin{center}
\begin{tikzpicture}[line cap=round,line join=round,>=triangle 45,x=1.0cm,y=1.0cm]
\clip(-2.3462809917355365,-0.851900826446281) rectangle (5.785950413223146,3.2142148760330564);
\draw [line width=1.pt,domain=-2.3462809917355365:5.785950413223146] plot(\x,{(--13.382241650160516-2.760330578512396*\x)/4.958677685950416});
\draw (-0.544628099173552,2.437355371900825) node[anchor=north west] {d};
\begin{scriptsize}
\draw[color=black] (-6.5942148760330594,6.115041322314047) node {$d$};
\end{scriptsize}
\end{tikzpicture}
\end{center}
\end{Ex}

\begin{Rq}
\begin{itemize}
\item Une droite est illimitée.
\item Une droite n'a pas de longueur.
\end{itemize}
\end{Rq}

\begin{Pp}
Par \textbf{un point} donné, il passe \textbf{une infinité} de droites.
\end{Pp}

\begin{Ex}
\begin{center}
\begin{tikzpicture}[line cap=round,line join=round,>=triangle 45,x=1.0cm,y=1.0cm]
\clip(-2.1809917355371895,-1.0502479338842965) rectangle (5.075206611570254,3.3629752066115697);
\draw [line width=1.pt,domain=-2.1809917355371895:5.075206611570254] plot(\x,{(--13.382241650160516-2.760330578512396*\x)/4.958677685950416});
\draw (-0.544628099173552,2.4373553719008263) node[anchor=north west] {d};
\draw [line width=1.pt,domain=-2.1809917355371895:5.075206611570254] plot(\x,{(--0.5658151688996957--4.542932352179724*\x)/5.197879725633495});
\draw [line width=1.pt,domain=-2.1809917355371895:5.075206611570254] plot(\x,{(-10.158031348004188--5.402436484411128*\x)/-0.2236078776722943});
\draw [line width=1.pt,domain=-2.1809917355371895:5.075206611570254] plot(\x,{(-5.844775417312083-0.5314478131095313*\x)/-4.02526077023428});
\draw [line width=1.pt,domain=-2.1809917355371895:5.075206611570254] plot(\x,{(--8.739713482677567-3.060373432944241*\x)/1.89209460166655});
%\begin{scriptsize}
\draw[color=black] (-6.5942148760330594,6.115041322314048) node {$d$};
\draw [fill=xdxdff] (1.8102764198483703,1.6910315257334436) circle (2.5pt);
\draw[color=xdxdff] (1.9181818181818215,1.9993388429752064) node {$A$};
\draw [fill=ududff] (-3.3876033057851243,-2.8519008264462804) circle (2.5pt);
\draw[color=ududff] (-3.271900826446281,-2.5461157024793373) node {$D$};
\draw [fill=ududff] (2.0338842975206646,-3.711404958677684) circle (2.5pt);
\draw[color=ududff] (2.1495867768595076,-3.405619834710742) node {$E$};
\draw [fill=ududff] (5.8355371900826505,2.222479338842975) circle (2.5pt);
\draw[color=ududff] (5.9512396694214935,2.5282644628099167) node {$F$};
\draw [fill=ududff] (-0.0818181818181798,4.751404958677685) circle (2.5pt);
\draw[color=ududff] (0.03388429752066324,5.0571900826446265) node {$G$};
%\end{scriptsize}
\end{tikzpicture}
\end{center}
\end{Ex}

\begin{Pp}
Par \textbf{deux points} distincts il ne passe qu'\textbf{une seule droite}.
\end{Pp}

\begin{Ex}
\begin{center}
\begin{tikzpicture}[line cap=round,line join=round,>=triangle 45,x=1.0cm,y=1.0cm]
\clip(-2.858677685950413,0.205950413223141) rectangle (4.8272727272727325,4.668760330578511);
\draw [line width=1.pt,domain=-2.858677685950413:4.8272727272727325] plot(\x,{(--13.382241650160516-2.760330578512396*\x)/4.958677685950416});
\draw (-0.544628099173552,2.4373553719008263) node[anchor=north west] {d};
%\begin{scriptsize}
\draw [fill=ududff] (-1.5198347107438006,3.5447933884297504) circle (2.5pt);
\draw[color=ududff] (-1.4041322314049576,3.850578512396693) node {$A$};
\draw [fill=ududff] (3.438842975206616,0.7844628099173547) circle (2.5pt);
\draw[color=ududff] (3.5545454545454587,1.0902479338842976) node {$B$};
\draw[color=black] (-6.5942148760330594,6.115041322314048) node {$d$};
%\end{scriptsize}
\end{tikzpicture}
\end{center}
\end{Ex}

\begin{Rq}
On dit qu'une droite est entièrement définie par deux points distincts.
\end{Rq}

\begin{Nt}
On peut nommer une droite :
\begin{itemize}
\item par une lettre : $(d),\,(d'),\,(d_1)$
\item par deux points de cette droite : $(AB),\,(AC),\,(GH)$
\end{itemize}
\end{Nt}

\begin{Ex}
Dans l'exemple précédent la droite peut-être nommé $(AB)$ ou $(d)$.
\end{Ex}

\begin{DefT}{Appartenance d'un point à une droite}
Lorsqu'une droite $(d)$ passe par un point $M$, on dit que $M$ \textit{appartient à} $(d)$ et on le note $M \in (d)$. $M \notin (d)$ signifie que le point $M$ n'appartient pas à la droite $(d)$.
\end{DefT}

\begin{Rq}
Les affirmations suivantes sont équivalentes :
\begin{itemize}
\item $C\in (AB)$
\item les points $A$, $B$ et $C$ sont alignés.
\end{itemize}
Dans ce cas la droite $(AB)$ peut s'appeler aussi $(AC)$ ou $(BC)$.
\end{Rq}

\begin{ExOApp}[]
Enoncer les points appartenant à la droite suivante :
\begin{center}
\begin{tikzpicture}[line cap=round,line join=round,>=triangle 45,x=2.0cm,y=2.0cm]
\clip(1.1827938840801815,-0.9503807246093496) rectangle (6.553184844277072,2.516910526391547);
\draw [line width=1.pt,domain=1.1827938840801815:6.553184844277072] plot(\x,{(-6.625462741616014--2.4958677685950406*\x)/4.049586776859507});
%\begin{scriptsize}
\draw [color=qqqqff] (1.8355371900826478,-0.5047933884297513)-- ++(-4.0pt,-4.0pt) -- ++(8.0pt,8.0pt) ++(-8.0pt,0) -- ++(8.0pt,-8.0pt);
\draw[color=qqqqff] (1.9257848664375183,-0.18783734797945317) node {$A$};
\draw [color=qqqqff] (5.885123966942155,1.991074380165289)-- ++(-4.0pt,-4.0pt) -- ++(8.0pt,8.0pt) ++(-8.0pt,0) -- ++(8.0pt,-8.0pt);
\draw[color=qqqqff] (5.979647945615268,2.314869118908412) node {$B$};
\draw [color=qqqqff] (3.621098597018986,0.5956954787024403)-- ++(-4.0pt,-4.0pt) -- ++(8.0pt,8.0pt) ++(-8.0pt,0) -- ++(8.0pt,-8.0pt);
\draw[color=qqqqff] (3.711570209998135,0.9201316607990289) node {$C$};
\draw [color=qqqqff] (3.4719008264462854,-0.05851239669421425)-- ++(-4.0pt,-4.0pt) -- ++(8.0pt,8.0pt) ++(-8.0pt,0) -- ++(8.0pt,-8.0pt);
\draw[color=qqqqff] (3.568185985332684,0.25535025553193963) node {$D$};
\draw [color=qqqqff] (3.918181818181823,1.5778512396694215)-- ++(-4.0pt,-4.0pt) -- ++(8.0pt,8.0pt) ++(-8.0pt,0) -- ++(8.0pt,-8.0pt);
\draw[color=qqqqff] (4.011373588844077,1.8977513744271013) node {$E$};
\draw [color=qqqqff] (5.3892561983471134,0.52)-- ++(-4.0pt,-4.0pt) -- ++(8.0pt,8.0pt) ++(-8.0pt,0) -- ++(8.0pt,-8.0pt);
\draw[color=qqqqff] (5.48432062404371,0.8419220837087831) node {$F$};
\draw [color=qqqqff] (5.025619834710749,1.6109090909090908)-- ++(-4.0pt,-4.0pt) -- ++(8.0pt,8.0pt) ++(-8.0pt,0) -- ++(8.0pt,-8.0pt);
\draw[color=qqqqff] (5.119342597622562,1.9238212334571831) node {$G$};
\draw [color=qqqqff] (2.7214240228394804,0.041202169718336956)-- ++(-4.0pt,-4.0pt) -- ++(8.0pt,8.0pt) ++(-8.0pt,0) -- ++(8.0pt,-8.0pt);
\draw[color=qqqqff] (2.8121600734603063,0.3596296916522673) node {$H$};
\draw [color=qqqqff] (2.0338842975206646,0.9497520661157027)-- ++(-4.0pt,-4.0pt) -- ++(8.0pt,8.0pt) ++(-8.0pt,0) -- ++(8.0pt,-8.0pt);
\draw[color=qqqqff] (2.1213088091631334,1.272074757705135) node {$I$};
\draw [color=qqqqff] (4.562809917355377,0.5695867768595045)-- ++(-4.0pt,-4.0pt) -- ++(8.0pt,8.0pt) ++(-8.0pt,0) -- ++(8.0pt,-8.0pt);
\draw[color=qqqqff] (4.650085135081087,0.8940618017689469) node {$J$};
%\end{scriptsize}
\end{tikzpicture}
\end{center}
\end{ExOApp}

\subsection{Demi-droites}

\begin{Def}
Un point sur une droite partage cette droite en deux \textbf{demi-droites}.

Ce point s'appelle l'\textbf{origine} de chacune des demi-droites.
\end{Def}

\begin{Rq}
Une demi-droite est illimitée, elle n'a pas de longueur.
\end{Rq}

\begin{Nt}
Pour différencier une droite et une demi-droite, on utilise une notation différente :

La demi-droite d'origine $A$ passant par $B$ se note $[AB)$.
\end{Nt}

\begin{Ex}
La demi-droite $[AB)$ :
\begin{center}
\begin{tikzpicture}[line cap=round,line join=round,>=triangle 45,x=1.0cm,y=1.0cm]
\clip(2.251658104313543,-1.185009455880087) rectangle (8.87340229795437,2.8558186937826124);
\draw [line width=1.pt,domain=2.955544298125757:8.87340229795437] plot(\x,{(-4.322472350712872--1.3556326695642604*\x)/3.506396039546028});
%\begin{scriptsize}
\draw [fill=ududff] (2.955544298125757,-0.09007537661664596) circle (2.5pt);
\draw[color=ududff] (3.0467888047310443,0.1510708194116119) node {$A$};
\draw [fill=ududff] (6.461940337671785,1.2655572929476144) circle (2.5pt);
\draw[color=ududff] (6.553184844277072,1.5067034889758724) node {$B$};
%\end{scriptsize}
\end{tikzpicture}
\end{center}
\end{Ex}

\begin{ExOApp}[]
Nommer les demi-droites présentent sur cette figure.
\begin{center}
\begin{tikzpicture}[line cap=round,line join=round,>=triangle 45,x=2.0cm,y=2.0cm]
\clip(2.3168327518887484,-2.358153112233774) rectangle (7.374385403724655,2.0346181343350316);
\draw [line width=1.pt,domain=2.955544298125757:7.374385403724655] plot(\x,{(-4.322472350712872--1.3556326695642604*\x)/3.506396039546028});
\draw [line width=1.pt,domain=2.3168327518887484:6.461940337671785] plot(\x,{(--9.090410710854771-1.850959991135817*\x)/-2.2680777356171333});
\draw [line width=1.pt,domain=2.3168327518887484:4.050478377389201] plot(\x,{(--11.435217734799942-2.9849988589443814*\x)/-0.5344321101166809});
%\begin{scriptsize}
\draw [fill=ududff] (2.955544298125757,-0.09007537661664596) circle (2.5pt);
\draw[color=ududff] (3.0467888047310443,0.1510708194116119) node {$A$};
\draw [fill=ududff] (6.461940337671785,1.2655572929476144) circle (2.5pt);
\draw[color=ududff] (6.553184844277072,1.5067034889758724) node {$B$};
\draw [fill=ududff] (4.193862602054652,-0.5854026981882027) circle (2.5pt);
\draw[color=ududff] (4.285107108659938,-0.3442565021599448) node {$C$};
\draw [fill=ududff] (4.050478377389201,1.2264525044024917) circle (2.5pt);
\draw[color=ududff] (4.141722883994487,1.4675987004307494) node {$D$};
\draw [fill=ududff] (3.51604626727252,-1.7585463545418896) circle (2.5pt);
\draw[color=ududff] (3.607290773877807,-1.5174001585136316) node {$E$};
\draw [fill=qqqqff] (3.6460415689221093,-1.03247503557223) circle (2.5pt);
\draw[color=qqqqff] (3.881024293693668,-1.1654570616075255) node {$H$};
%\end{scriptsize}
\end{tikzpicture}
\end{center}
\end{ExOApp}

\subsection{Segments}

\begin{Def}
Une portion de droite délimitée par deux points s'appelle un \textbf{segment}.

Ces points s'appellent \textbf{les extrémités} du segment.
\end{Def}

\begin{Nt}
Le segment qui a pour extrémités les points $A$ et $B$ se note $[AB]$.
\end{Nt}

\begin{Ex}
Le segment $[AB]$ :
\begin{center}
\begin{tikzpicture}[line cap=round,line join=round,>=triangle 45,x=1.0cm,y=1.0cm]
\clip(2.1734485272232975,-0.8982410065491858) rectangle (7.426525121784819,2.3213865836659324);
\draw [line width=1.pt] (2.955544298125757,-0.09007537661664596)-- (6.461940337671785,1.2655572929476144);
%\begin{scriptsize}
\draw [fill=ududff] (2.955544298125757,-0.09007537661664596) circle (2.5pt);
\draw[color=ududff] (3.0467888047310443,0.1510708194116119) node {$A$};
\draw [fill=ududff] (6.461940337671785,1.2655572929476144) circle (2.5pt);
\draw[color=ududff] (6.553184844277072,1.5067034889758724) node {$B$};
%\end{scriptsize}
\end{tikzpicture}
\end{center}
\end{Ex}

\begin{Def}
La \textbf{longueur} du segment $[AB]$ est la distance entre les points $A$ et $B$, elle se note $AB$.
\end{Def}

\begin{Ex}
Dans cet exemple $AB=3,76$ :

\begin{tikzpicture}[line cap=round,line join=round,>=triangle 45,x=1.0cm,y=1.0cm]
\clip(2.447182047039158,-0.41594861449267007) rectangle (6.996372447788466,1.9955133457899086);
\draw [line width=1.pt] (2.838229932490388,0.17062321368417335)-- (6.344625972036417,1.526255883248433);
\draw [line width=1.pt,dash pattern=on 1pt off 1pt,color=zzttqq] (2.955544298125757,-0.09007537661664596)-- (6.461940337671785,1.2655572929476144);
\begin{scriptsize}
\draw [fill=ududff] (2.838229932490388,0.17062321368417335) circle (2.5pt);
\draw[color=ududff] (2.9294744390956753,0.4117694097124312) node {$A$};
\draw [fill=ududff] (6.344625972036417,1.526255883248433) circle (2.5pt);
\draw[color=ududff] (6.435870478641703,1.7674020792766916) node {$B$};
\draw[color=zzttqq] (4.845609077806702,0.3596296916522673) node {$3.76$};
\end{scriptsize}
\end{tikzpicture}
\end{Ex}

\begin{Att}
Il ne faut pas confondre :
\begin{itemize}
\item $[AB]$ qui est un segment, un ensemble de point.
\item $AB$ qui est un nombre, la longueur du segment $[AB]$.
\end{itemize}
\end{Att}

\begin{Def}
On dit qu'un point $M$ est le milieu d'un segment $[AB]$ lorsque :
\begin{itemize}
\item $M$ appartient à $[AB]$
\item $M$ est à la même distance de $A$ que de $B$
\end{itemize}
\end{Def}

\begin{Rq}
Traduit en langage mathématique :

$M$ est le milieu de $[AB]$ $\Leftrightarrow M\in(AB)$ et $MA=MB$.
\end{Rq}

\begin{Ex}
$M$ est le milieu du segment $[AB]$ car il appartient à $[AB]$ et $AM=BM=1,88$ :

\begin{tikzpicture}[line cap=round,line join=round,>=triangle 45,x=1.0cm,y=1.0cm]
\clip(2.525391624129404,-0.16828495370689175) rectangle (6.748708787002687,2.0476530638500723);
\draw [line width=1.pt] (2.838229932490388,0.17062321368417335)-- (6.344625972036417,1.526255883248433);
\draw [line width=1.pt] (2.838229932490388,0.17062321368417335)-- (4.591427952263403,0.8484395484663032);
\draw [line width=1.pt] (4.591427952263403,0.8484395484663032)-- (6.344625972036417,1.526255883248433);
%\begin{scriptsize}
\draw [fill=ududff] (2.838229932490388,0.17062321368417335) circle (2.5pt);
\draw[color=ududff] (2.9294744390956753,0.4117694097124312) node {$A$};
\draw [fill=ududff] (6.344625972036417,1.526255883248433) circle (2.5pt);
\draw[color=ududff] (6.435870478641703,1.7674020792766916) node {$B$};
\draw [fill=xdxdff] (4.591427952263403,0.8484395484663032) circle (2.5pt);
\draw[color=xdxdff] (4.689189923626209,1.0895857444945614) node {$M$};
\draw[color=black] (3.946198941268873,0.43783926874251317) node {$1.88$};
\draw[color=black] (5.692879496284366,1.1156556035246434) node {$1.88$};
%\end{scriptsize}
\end{tikzpicture}
\end{Ex}

\begin{Nt}
Lorsque sur une figure, deux segments ont la même longueur, on code cette propriété pour en informer les observateurs :

\begin{tikzpicture}[line cap=round,line join=round,>=triangle 45,x=1.0cm,y=1.0cm]
\clip(2.525391624129404,-0.16828495370689175) rectangle (6.748708787002687,2.0476530638500723);
\draw [line width=1.pt] (2.838229932490388,0.17062321368417335)-- (6.344625972036417,1.526255883248433);
\draw [line width=1.pt] (2.838229932490388,0.17062321368417335)-- (4.591427952263403,0.8484395484663032);
\draw [line width=1.pt] (3.656231375675976,0.5707277600286093) -- (3.712636900737872,0.42483270001274537);
\draw [line width=1.pt] (3.7170209840159183,0.5942300621377316) -- (3.7734265090778143,0.44833500212186767);
\draw [line width=1.pt] (4.591427952263403,0.8484395484663032)-- (6.344625972036417,1.526255883248433);
\draw [line width=1.pt] (5.40942939544899,1.2485440948107394) -- (5.465834920510886,1.1026490347948756);
\draw [line width=1.pt] (5.470219003788932,1.2720463969198617) -- (5.526624528850828,1.126151336903998);
%\begin{scriptsize}
\draw [fill=ududff] (2.838229932490388,0.17062321368417335) circle (2.5pt);
\draw[color=ududff] (2.9294744390956753,0.4117694097124312) node {$A$};
\draw [fill=ududff] (6.344625972036417,1.526255883248433) circle (2.5pt);
\draw[color=ududff] (6.435870478641703,1.7674020792766916) node {$B$};
\draw [fill=xdxdff] (4.591427952263403,0.8484395484663032) circle (2.5pt);
\draw[color=xdxdff] (4.689189923626209,1.0895857444945614) node {$M$};
%\end{scriptsize}
\end{tikzpicture}
\end{Nt}

\section{Conjecture ou affirmation}

\begin{Def}
Un \textbf{conjecture} est une impression qui n'est pas démontrée, on n'est pas certain que ce soit vrai.
\end{Def}

\begin{Rq}
Pour énoncer une conjecture on utilise des mots qui montrent que nous ne sommes pas certains que ce qui est dit est vrai :
\begin{itemize}
\item Le triangle \textbf{semble} rectangle.
\item Les droites \textbf{ont l'air} parallèles.
\item J'\textbf{ai l'impression que} le quadrilatère est un carré.
\end{itemize}
\end{Rq}

\begin{Reg}
Dans un exercice de mathématiques sont considérés comme vrais :
\begin{itemize}
\item Les données de l'énoncé.
\item Les propriétés codées sur la figure donnée.
\item Les résultats qui sont démontrés.
\end{itemize}
\end{Reg}

\begin{Mt}
Différencier conjecture et affirmation :
\begin{center}
\begin{tabular}{c|c}
\textbf{Conjecture} & \textbf{Affirmation} \\\hline
\begin{tikzpicture}[line cap=round,line join=round,>=triangle 45,x=1.0cm,y=1.0cm]
\clip(-2.8256198347107437,-1.628760330578512) rectangle (3.901652892561988,2.834049586776858);
%\draw[line width=1.pt,color=qqwuqq,fill=qqwuqq,fill opacity=0.10000000149011612] (0.8086387280175087,0.3563276562140491) -- (0.6623110718034596,0.6749663842315579) -- (0.34367234378595085,0.5286387280175088) -- (0.49,0.21) -- cycle; 
\draw [line width=1.pt] (-1.84,-0.86)-- (2.82,1.28);
\draw [line width=1.pt,domain=-2.8256198347107437:3.901652892561988] plot(\x,{(-2.7328--4.66*\x)/-2.14});
\draw [line width=1.pt] (-1.84,-0.86)-- (0.49,0.21);
%\draw [line width=1.pt] (-0.7539396454152754,-0.25212023753208024) -- (-0.6711642230280046,-0.43236952179594923);
%\draw [line width=1.pt] (-0.6788357769719975,-0.21763047820405126) -- (-0.5960603545847267,-0.3978797624679203);
%\draw [line width=1.pt] (0.49,0.21)-- (2.82,1.28);
%\draw [line width=1.pt] (1.5760603545847258,0.8178797624679205) -- (1.6588357769719946,0.6376304782040515);
%\draw [line width=1.pt] (1.6511642230280037,0.8523695217959495) -- (1.7339396454152725,0.6721202375320805);
\draw (-0.7595041322314033,2.0241322314049577) node[anchor=north west] {d};
%\begin{scriptsize}
\draw [fill=ududff] (-1.84,-0.86) circle (2.5pt);
\draw[color=ududff] (-1.7181818181818171,-0.5461157024793389) node {$A$};
\draw [fill=ududff] (2.82,1.28) circle (2.5pt);
\draw[color=ududff] (2.942975206611574,1.586115702479338) node {$B$};
\draw[color=black] (-1.7842975206611562,5.91669421487603) node {$d$};
\draw [fill=uuuuuu] (0.49,0.21) circle (2.0pt);
\draw [fill=xdxdff] (-1.5477169977790624,4.647271593294594) circle (2.5pt);
%\end{scriptsize}
\end{tikzpicture}
 &
\begin{tikzpicture}[line cap=round,line join=round,>=triangle 45,x=1.0cm,y=1.0cm]
\clip(-2.8256198347107437,-1.628760330578512) rectangle (3.901652892561988,2.834049586776858);
\draw[line width=1.pt,color=qqwuqq,fill=qqwuqq,fill opacity=0.10000000149011612] (0.8086387280175087,0.3563276562140491) -- (0.6623110718034596,0.6749663842315579) -- (0.34367234378595085,0.5286387280175088) -- (0.49,0.21) -- cycle; 
\draw [line width=1.pt] (-1.84,-0.86)-- (2.82,1.28);
\draw [line width=1.pt,domain=-2.8256198347107437:3.901652892561988] plot(\x,{(-2.7328--4.66*\x)/-2.14});
\draw [line width=1.pt] (-1.84,-0.86)-- (0.49,0.21);
\draw [line width=1.pt] (-0.7539396454152754,-0.25212023753208024) -- (-0.6711642230280046,-0.43236952179594923);
\draw [line width=1.pt] (-0.6788357769719975,-0.21763047820405126) -- (-0.5960603545847267,-0.3978797624679203);
\draw [line width=1.pt] (0.49,0.21)-- (2.82,1.28);
\draw [line width=1.pt] (1.5760603545847258,0.8178797624679205) -- (1.6588357769719946,0.6376304782040515);
\draw [line width=1.pt] (1.6511642230280037,0.8523695217959495) -- (1.7339396454152725,0.6721202375320805);
\draw (-0.7595041322314033,2.0241322314049577) node[anchor=north west] {d};
%\begin{scriptsize}
\draw [fill=ududff] (-1.84,-0.86) circle (2.5pt);
\draw[color=ududff] (-1.7181818181818171,-0.5461157024793389) node {$A$};
\draw [fill=ududff] (2.82,1.28) circle (2.5pt);
\draw[color=ududff] (2.942975206611574,1.586115702479338) node {$B$};
\draw[color=black] (-1.7842975206611562,5.91669421487603) node {$d$};
\draw [fill=uuuuuu] (0.49,0.21) circle (2.0pt);
\draw [fill=xdxdff] (-1.5477169977790624,4.647271593294594) circle (2.5pt);
%\end{scriptsize}
\end{tikzpicture} 
  \\\hline
Je pense que la droite (d) est la médiatrice du segment [AB] & La droite (d) est la médiatrice de [AB] \\
\end{tabular}
\end{center}
\end{Mt}

\section{Notations en géométrie}

\begin{center}
\begin{tabular}{M{5cm}|M{2cm}|M{5cm}}
\textbf{Définition} & \textbf{Notation} & \textbf{Représentation} \\\hline\hline
Le \textbf{point} $A$. & $A$ &
\begin{tikzpicture}[line cap=round,line join=round,>=triangle 45,x=1.0cm,y=1.0cm]
\clip(1.6259814875915755,0.027238989018722733) rectangle (4.206897531569693,2.178002359000482);
%\begin{scriptsize}
\draw [color=ududff] (2.8903696505505523,1.1221730682821638)-- ++(-2.5pt,-2.5pt) -- ++(5.0pt,5.0pt) ++(-5.0pt,0) -- ++(5.0pt,-5.0pt);
\draw[color=ududff] (2.981614157155839,1.3633192643104217) node {$A$};
%\end{scriptsize}
\end{tikzpicture}
\\\hline
La \textbf{droite} passant par $A$ et $B$. & $(AB)$ &
\begin{tikzpicture}[line cap=round,line join=round,>=triangle 45,x=1.0cm,y=1.0cm]
\clip(1.4043876858358786,-0.8591362180040629) rectangle (5.002028231987193,1.9564085572447856);
\draw [line width=1.pt,domain=1.4043876858358786:5.002028231987193] plot(\x,{(--4.9896391167547565-1.2252833744138507*\x)/1.2904580219890582});
%\begin{scriptsize}
\draw [color=ududff] (2.8903696505505523,1.1221730682821638)-- ++(-2.5pt,-2.5pt) -- ++(5.0pt,5.0pt) ++(-5.0pt,0) -- ++(5.0pt,-5.0pt);
\draw[color=ududff] (2.981614157155839,1.3633192643104217) node {$A$};
\draw [color=ududff] (4.1808276725396105,-0.10311030613168692)-- ++(-2.5pt,-2.5pt) -- ++(5.0pt,5.0pt) ++(-5.0pt,0) -- ++(5.0pt,-5.0pt);
\draw[color=ududff] (4.272072179144898,0.13803588989657092) node {$B$};
%\end{scriptsize}
\end{tikzpicture}
\\\hline
La \textbf{demi-droite} d'\textbf{origine} $A$ passant par $B$. & $[AB)$ &
\begin{tikzpicture}[line cap=round,line join=round,>=triangle 45,x=1.0cm,y=1.0cm]
\clip(1.4043876858358786,-0.8591362180040629) rectangle (5.002028231987193,1.9564085572447856);
\draw [line width=1.pt,domain=2.8903696505505523:5.1845172451977675] plot(\x,{(--4.9896391167547565-1.2252833744138507*\x)/1.2904580219890582});
%\begin{scriptsize}
\draw [color=ududff] (2.8903696505505523,1.1221730682821638)-- ++(-2.5pt,-2.5pt) -- ++(5.0pt,5.0pt) ++(-5.0pt,0) -- ++(5.0pt,-5.0pt);
\draw[color=ududff] (2.981614157155839,1.3633192643104217) node {$A$};
\draw [color=ududff] (4.1808276725396105,-0.10311030613168692)-- ++(-2.5pt,-2.5pt) -- ++(5.0pt,5.0pt) ++(-5.0pt,0) -- ++(5.0pt,-5.0pt);
\draw[color=ududff] (4.272072179144898,0.13803588989657092) node {$B$};
%\end{scriptsize}
\end{tikzpicture}
\\\hline
Le \textbf{segment} ayant pour \textbf{extrémités} les points $A$ et $B$. & $[AB]$ &
\begin{tikzpicture}[line cap=round,line join=round,>=triangle 45,x=1.0cm,y=1.0cm]
\clip(1.4043876858358786,-0.8591362180040629) rectangle (5.002028231987193,1.9564085572447856);
\draw [line width=1.pt] (2.8903696505505523,1.1221730682821638)-- (4.1808276725396105,-0.10311030613168692);
%\begin{scriptsize}
\draw [color=ududff] (2.8903696505505523,1.1221730682821638)-- ++(-2.5pt,-2.5pt) -- ++(5.0pt,5.0pt) ++(-5.0pt,0) -- ++(5.0pt,-5.0pt);
\draw[color=ududff] (2.981614157155839,1.3633192643104217) node {$A$};
\draw [color=ududff] (4.1808276725396105,-0.10311030613168692)-- ++(-2.5pt,-2.5pt) -- ++(5.0pt,5.0pt) ++(-5.0pt,0) -- ++(5.0pt,-5.0pt);
\draw[color=ududff] (4.272072179144898,0.13803588989657092) node {$B$};
%\end{scriptsize}
\end{tikzpicture}
\\
\end{tabular}
\end{center}

\begin{center}
\begin{tabular}{M{5cm}|M{2cm}|M{5cm}}
\textbf{Définition} & \textbf{Notation} & \textbf{Représentation} \\\hline\hline
Le point $A$ \textbf{appartient à} la droite $(d)$ & $A\in(d)$ &
\begin{tikzpicture}[line cap=round,line join=round,>=triangle 45,x=1.0cm,y=1.0cm]
\clip(0.6483617739635007,-0.7418218523686942) rectangle (4.910783725381907,1.8912339096695807);
\draw [line width=1.pt,domain=0.6483617739635007:4.910783725381907] plot(\x,{(-1.2811669811193174--1.134038867808564*\x)/3.649780264211479});
\draw (1.3261781087456326,0.7832649008910987) node[anchor=north west] {d};
%\begin{scriptsize}
\draw [color=ududff] (3.5516665938001113,0.7525277641175045)-- ++(-2.5pt,-2.5pt) -- ++(5.0pt,5.0pt) ++(-5.0pt,0) -- ++(5.0pt,-5.0pt);
\draw[color=ududff] (3.64639556242293,0.9983412378892746) node {$A$};
%\end{scriptsize}
\end{tikzpicture}
\\\hline
Le point $A$ \textbf{n'appartient pas à} la droite $(d)$ & $A\notin(d)$ &
\begin{tikzpicture}[line cap=round,line join=round,>=triangle 45,x=1.0cm,y=1.0cm]
\clip(0.6483617739635007,-0.7418218523686942) rectangle (4.910783725381907,1.8912339096695807);
\draw [line width=1.pt,domain=0.5962220559033368:4.637050205566045] plot(\x,{(-1.2811669811193174--1.134038867808564*\x)/3.649780264211479});
\draw (1.3261781087456326,0.7832649008910987) node[anchor=north west] {d};
%\begin{scriptsize}
\draw [color=ududff] (3.489976408242438,0.23579786125937818)-- ++(-2.5pt,-2.5pt) -- ++(5.0pt,5.0pt) ++(-5.0pt,0) -- ++(5.0pt,-5.0pt);
\draw[color=ududff] (3.5812209148477248,0.47694405728763606) node {$A$};
%\end{scriptsize}
\end{tikzpicture}
\\\hline
La distance entre les points $A$ et $B$ & $AB$ &
\begin{tikzpicture}[line cap=round,line join=round,>=triangle 45,x=1.0cm,y=1.0cm]
\clip(0.9351302232944027,-1.4457080461809064) rectangle (4.454561192355472,1.1091381387671229);
\draw [line width=1.pt] (3.489976408242438,0.23579786125937818)-- (1.5738417695314115,-0.32470410788738335);
\draw (2.0952389501330515,-0.2595294603121785) node[anchor=north west] {$AB=2\,cm$};
%\begin{scriptsize}
\draw [color=ududff] (3.489976408242438,0.23579786125937818)-- ++(-2.5pt,-2.5pt) -- ++(5.0pt,5.0pt) ++(-5.0pt,0) -- ++(5.0pt,-5.0pt);
\draw[color=ududff] (3.5812209148477248,0.47694405728763606) node {$A$};
\draw [color=ududff] (1.5738417695314115,-0.32470410788738335)-- ++(-2.5pt,-2.5pt) -- ++(5.0pt,5.0pt) ++(-5.0pt,0) -- ++(5.0pt,-5.0pt);
\draw[color=ududff] (1.6650862761366985,-0.08355791185912549) node {$B$};
%\end{scriptsize}
\end{tikzpicture}
\\
\end{tabular}
\end{center}

\section{Les savoir-faire du parcours}

\begin{CpsCol}
\begin{itemize}
\item Savoir déterminer si des points sont alignés
\item Savoir déterminer si des points appartiennent à une droite
\item Savoir construire une droite.       
\item Savoir déterminer une demi-droite.
\item Savoir construire une demi-droite.   
\item Savoir déterminer si des points appartiennent à une demi-droit.
\item Savoir construire un segment.
\item Savoir coder des segments de même longueurs.
\item Savoir déterminer si un point est le milieu d'un segment.
\item Savoir utiliser les notations de géométrie.
\item Savoir interpréter les codages d'une figure.
\item Savoir coder une figure.
\item Savoir différencier conjecture et affirmation.
\end{itemize}
\end{CpsCol}

\end{document}
