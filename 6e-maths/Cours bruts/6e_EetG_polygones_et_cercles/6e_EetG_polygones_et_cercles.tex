\documentclass[a4paper,dvipsnames]{article}

\addtolength{\hoffset}{-2.25cm}
\addtolength{\textwidth}{4.5cm}
\addtolength{\voffset}{-3.25cm}
\addtolength{\textheight}{5cm}
\setlength{\parskip}{0pt}
\setlength{\parindent}{0in}

%----------------------------------------------------------------------------------------
%	PACKAGES AND OTHER DOCUMENT CONFIGURATIONS
%----------------------------------------------------------------------------------------

%----------------------------------------------------------------------------------------
%		Generals
%----------------------------------------------------------------------------------------
\usepackage{fourier}
\usepackage{frcursive}
\usepackage[T1]{fontenc} %Accents handling
\usepackage[utf8]{inputenc} % Use UTF-8 encoding
%\usepackage{microtype} % Slightly tweak font spacing for aesthetics
\usepackage[english, francais]{babel} % Language hyphenation and typographical rules

%----------------------------------------------------------------------------------------
%		Graphics
%----------------------------------------------------------------------------------------
\usepackage{xcolor}
\usepackage{graphicx, multicol} % Enhanced support for graphics
\graphicspath{{FIG/}}
\usepackage{wrapfig}

%----------------------------------------------------------------------------------------
%		Other packages
%----------------------------------------------------------------------------------------
\usepackage{hyperref}
\hypersetup{
	colorlinks=true, %colorise les liens
	breaklinks=true, %permet le retour à la ligne dans les liens trop longs
	urlcolor= bleu3,  %couleur des hyperliens
	linkcolor= bleu3, %couleur des liens internes
	plainpages=false  %pour palier à "Bookmark problems can occur when you have duplicate page numbers, for example, if you have a page i and a page 1."
}
\usepackage{tabularx}
\newcolumntype{M}[1]{>{\arraybackslash}m{#1}} %Defines a scalable column type in tabular
\usepackage{booktabs} % Enhances quality of tables
\usepackage{diagbox} % barre en diagonale dans un tableau
\usepackage{multicol}
\usepackage[explicit]{titlesec}


%----------------------------------------------------------------------------------------
%		Headers and footers
%----------------------------------------------------------------------------------------
\usepackage{fancyhdr} % Headers and footers
\pagestyle{fancy} % All pages have headers and footers
\fancyhead{}\renewcommand{\headrulewidth}{0pt} % Blank out the default header
\renewcommand{\footrulewidth}{0pt}
\fancyfoot[L]{} % Custom footer text
\fancyfoot[C]{\href{https://sacado.xyz/}{sacado.xyz}} % Custom footer text
\fancyfoot[R]{\thepage} % Custom footer text

%----------------------------------------------------------------------------------------
%		Mathematics packages
%----------------------------------------------------------------------------------------
\usepackage{amsthm, amsmath, amssymb} % Mathematical typesetting
\usepackage{marvosym, wasysym} % More symbols
\usepackage[makeroom]{cancel}
\usepackage{xlop}
\usepackage{pgf,tikz,pgfplots}
\pgfplotsset{compat=1.15}
\usetikzlibrary{positioning}
%\usetikzlibrary{arrows}
\usepackage{pst-plot,pst-tree,pst-func, pstricks-add,pst-node,pst-text}
\usepackage{units}
\usepackage{nicefrac}
\usepackage[np]{numprint} %Séparation milliers dans un nombre

%----------------------------------------------------------------------------------------
%		New text commands
%----------------------------------------------------------------------------------------
\usepackage{calc}
\usepackage{boites}
 \renewcommand{\arraystretch}{1.6}

%%%%% Pour les imports.
\usepackage{import}

%%%%% Pour faire des boites
\usepackage[tikz]{bclogo}
\usepackage{bclogo}
\usepackage{framed}
\usepackage[skins]{tcolorbox}
\tcbuselibrary{breakable}
\tcbuselibrary{skins}
\usetikzlibrary{babel,arrows,shadows,decorations.pathmorphing,decorations.markings,patterns}

%%%%% Pour les symboles et les ensembles
\newcommand{\pp}{\leq}
\newcommand{\pg}{\geq}
%\newcommand{\euro}{\eurologo{}}
\newcommand{\R}{\mathbb{R}}
\newcommand{\N}{\mathbb{N}}
\newcommand{\D}{\mathbb{D}}
\newcommand{\Z}{\mathbb{Z}}
\newcommand{\Q}{\mathbb{Q}}
\newcommand{\C}{\mathbb{C}}

%%%%% Pour une double minipage
\newcommand{\mini}[2]{
\begin{minipage}[t]{0.48\linewidth}
#1
\end{minipage}
\hfill
\begin{minipage}[t]{0.48\linewidth}
#2
\end{minipage}
}


%\newcommand\hole[1]{\texttt{\_}}
%\newcommand{\PROP}[1]{\textbf{\underline{#1}}}
%\newcommand{\exercice}{\textcolor{OliveGreen}{Exercice : }}
%\newcommand{\correction}{\textcolor{BurntOrange}{Correction : }}
%\newcommand{\propriete}{\textbf{\underline{Propriété}} : }
%\newcommand{\prop}{\textbf{\underline{Propriété}} : }
%\newcommand{\vocabulaire}{\textbf{\underline{Vocabulaire}} : }
%\newcommand{\voca}{\textbf{\underline{Vocabulaire}} : }

\usepackage{enumitem}
\newlist{todolist}{itemize}{2} %Pour faire des QCM
\setlist[todolist]{label=$\square$} %Pour faire des QCM \begin{todolist} instead of itemize

%----------------------------------------------------------------------------------------
%		Définition de couleur pour geogebra
%----------------------------------------------------------------------------------------
\definecolor{zzttqq}{rgb}{0.6,0.2,0.} %rouge des polygones
\definecolor{qqqqff}{rgb}{0.,0.,1.}
\definecolor{xdxdff}{rgb}{0.49019607843137253,0.49019607843137253,1.}%bleu
\definecolor{qqwuqq}{rgb}{0.,0.39215686274509803,0.} %vert des angles
\definecolor{ffqqqq}{rgb}{1.,0.,0.} %rouge vif
\definecolor{uuuuuu}{rgb}{0.26666666666666666,0.26666666666666666,0.26666666666666666}
\definecolor{qqzzqq}{rgb}{0.,0.6,0.}
\definecolor{cqcqcq}{rgb}{0.7529411764705882,0.7529411764705882,0.7529411764705882} %gris
\definecolor{qqffqq}{rgb}{0.,1.,0.}
\definecolor{ffdxqq}{rgb}{1.,0.8431372549019608,0.}
\definecolor{ffffff}{rgb}{1.,1.,1.}
\definecolor{ududff}{rgb}{0.30196078431372547,0.30196078431372547,1.}

%-------------------------------------------------
%
%	EN TETE
%
%-------------------------------------------------

% Classe
\newcommand{\myClasse}   
{
    6e
}

% Discipline
\newcommand{\myDiscipline}   
{
    Mathématiques
}

% Parcours
\newcommand{\myParcours}
{
  Nombres et Calculs
}

%Titre de la séquence
\newcommand{\myTitle}
{
    \scshape\huge
\textcolor{sacado_purple}{
		Enchainements d'opérations
}
}

%----------------------------------------------------------------------------------------

%----------------------------------------------------------------------------------------
%		Définition de couleur pour les boites
%----------------------------------------------------------------------------------------
\definecolor{bleu1}{rgb}{0.54,0.79,0.95} %% Bleu
\definecolor{sapgreen}{rgb}{0.4, 0.49, 0}
\definecolor{dvzfxr}{rgb}{0.7,0.4,0.}
\definecolor{beamer}{rgb}{0.5176470588235295,0.49019607843137253,0.32941176470588235} % couleur beamer
\definecolor{preuveRbeamer}{rgb}{0.8,0.4,0}
\definecolor{sectioncolor}{rgb}{0.24,0.21,0.44}
\definecolor{subsectioncolor}{rgb}{0.1,0.21,0.61}
\definecolor{subsubsectioncolor}{rgb}{0.1,0.21,0.61}
\definecolor{info}{rgb}{0.82,0.62,0}
\definecolor{bleu2}{rgb}{0.38,0.56,0.68}
\definecolor{bleu3}{rgb}{0.24,0.34,0.40}
\definecolor{bleu4}{rgb}{0.12,0.20,0.25}
\definecolor{vert}{rgb}{0.21,0.33,0}
\definecolor{vertS}{rgb}{0.05,0.6,0.42}
\definecolor{red}{rgb}{0.78,0,0}
\definecolor{color5}{rgb}{0,0.4,0.58}
\definecolor{eduscol4B}{rgb}{0.19,0.53,0.64}
\definecolor{eduscol4P}{rgb}{0.62,0.12,0.39}

%----------------------------------------------------------------------------------------
%		Définition de couleur pour les boites SACADO
%----------------------------------------------------------------------------------------
\definecolor{sacado_blue}{RGB}{0,129,159} %% Bleu Sacado
\definecolor{sacado_green}{RGB}{59, 157, 38} %% Vert Sacado
\definecolor{sacado_yellow}{RGB}{255,180,0} %% Jaune Sacado
\definecolor{sacado_purple}{RGB}{94,68,145} %% Violet foncé Sacado
\definecolor{sacado_violet}{RGB}{153,117,224} %% Violet clair Sacado
\definecolor{sacado_orange}{RGB}{249,168,100} %% Orange Sacado
\definecolor{ill_frame}{HTML}{F0F0F0}
\definecolor{ill_back}{HTML}{F7F7F7}
\definecolor{ill_title}{HTML}{AAAAAA}


 % Compteurs pour Théorème, Définition, Exemple, Remarque, .....
\newcounter{cpttheo}
\setcounter{cpttheo}{0}
\newcounter{cptdef}
\setcounter{cptdef}{0}
\newcounter{cptmth}
\setcounter{cptmth}{0}
\newcounter{cpttitre}
\setcounter{cpttitre}{0}
 % Exercices
\newcounter{cptapp}
\setcounter{cptapp}{0}
\newcounter{cptex}
\setcounter{cptex}{0}
\newcounter{cptsr}
\setcounter{cptsr}{0}
\newcounter{cpti}
\setcounter{cpti}{0}
\newcounter{cptcor}
\setcounter{cptcor}{0}




%%%%% Pour réinitialiser numéros des paragraphes après une nouvelle partie
\makeatletter
    \@addtoreset{paragraph}{part}
\makeatother


%%%% Titres et sections

\newlength\chapnumb
\setlength\chapnumb{3cm}


% \titleformat{\part}[block] {
 % \normalfont\sffamily\color{violet}}{}{0pt} {
   % \parbox[t]{\chapnumb}{\fontsize{120}{110}\selectfont\ding{110}}
   % \parbox[b]{\dimexpr\textwidth-\chapnumb\relax}{
       % \raggedleft
       % \hfill{{\color{bleu3}\fontsize{40}{30}\selectfont#1}}\\
       % \rule{0.99\textwidth-\chapnumb\relax}{0.4pt}
 % }
% }

% \titleformat{name=\part,numberless}[block]
% {\normalfont\sffamily\color{bleu3}}{}{0pt}
% {\parbox[b]{\chapnumb}{%
  % \mbox{}}%
 % \parbox[b]{\dimexpr\textwidth-\chapnumb\relax}{%
   % \raggedleft%
   % \hfill{{\color{bleu3}\fontsize{40}{30}\selectfont#1}}\\
   % \rule{0.99\textwidth-\chapnumb\relax}{0.4pt}
 % }
% }



% \titleformat{\chapter}[block] {
 % \normalfont\sffamily\color{violet}}{}{0pt} {
   % \parbox[t]{\chapnumb}{ 
     % \fontsize{120}{110}\selectfont\thechapter}
     % \parbox[b]{\textwidth-\chapnumb}{
       % \raggedleft
       % \hfill{{\color{bleu3}\huge#1}}\\  
  % \ifthenelse{\thechapter<10}{\rule{0.99\textwidth-\chapnumb}{0.4pt}}{\rule{0.9\textwidth - \chapnumb}{0.4pt}}
       % \setcounter{cpttitre}{0}
	% \setcounter{cptapp}{0}
	% \setcounter{cptex}{0}
	% \setcounter{cptsr}{0}
	% \setcounter{cpti}{0}
	% \setcounter{cptcor}{0} 
 % }
% }

% \titleformat{name=\chapter,numberless}[block]
% {\normalfont\sffamily\color{bleu3}}{}{0pt}
% {\parbox[b]{\chapnumb}{%
  % \mbox{}}%
 % \parbox[b]{\textwidth-\chapnumb}{%
   % \raggedleft
   % \hfill{{\color{bleu3}\huge#1}}\\
   % \ifthenelse{\thechapter<10}{\rule{0.99\textwidth-\chapnumb}{0.4pt}}{ \rule{0.9\textwidth - \chapnumb}{0.4pt}}
       % \setcounter{cpttitre}{0}
	% \setcounter{cptapp}{0}
	% \setcounter{cptex}{0}
	% \setcounter{cptsr}{0}
	% \setcounter{cpti}{0}
	% \setcounter{cptcor}{0} 
 % }
% }
%
%       
%
%%%%% Personnalisation des numéros des sections
\renewcommand\thesection{\Roman{section}. }
\renewcommand\thesubsection{\hspace{1cm}\arabic{subsection}. }
\renewcommand\thesubsubsection{\hspace{2cm}\alph{subsubsection}. }

\titleformat{\section}[hang]{\color{sacado_purple}{}\normalfont\filright\huge}{}{0.4em}{\textbf{\thesection  #1}}   
% \titlespacing*{\section}{0.2pt}{0ex plus 0ex minus 0ex}{0.3em}
   
\titleformat{\subsection}[hang]{\color{sacado_purple}{}\normalfont\filright\Large}{}{0.4em}{\thesubsection
 #1}            
\titleformat{\subsubsection}[hang]{\color{sacado_purple}{}\normalfont\filright\large}{}{0.4em}{\thesubsubsection
 #1}
\titleformat{\paragraph}[hang]{\color{black}{}\normalfont\filright\normalsize}{}{0.4em}{#1}



%%%%%%%%%%%%%%%%%%%%% Cycle 4
%\newcommand{\Titre}[2]{\section*{#1 
%\ifthenelse{\equal{#2}{1}}   {\hfill{ \ding{182}  \ding{173} \ding{174} } \addcontentsline{toc}{section}{#1 \ding{182}} }%
%{%
%\ifthenelse{\equal{#2}{2}}{\hfill{ \ding{172}  \ding{183} \ding{174} } \addcontentsline{toc}{section}{#1 {\color{purple}\ding{183}}} }{%           
%\hfill{ \ding{172}  \ding{173} \ding{184} } \addcontentsline{toc}{section}{#1 {\color{orange}\ding{184}}}% 
%}%
%}%
%}
%}


%%%%%%%%%%%%%%%%%%%%% Cycle 4
\newcommand{\Titre}[2]{\section*{#1 
\ifthenelse{\equal{#2}{1}}   {\hfill{ \ding{182}  \ding{173} \ding{174} } \addcontentsline{toc}{section}{#1 \, \ding{182}} }%
{% sinon
\ifthenelse{\equal{#2}{1,5}}   {\hfill{ \ding{182}  \ding{183} \ding{174} } \addcontentsline{toc}{section}{#1 \, \ding{182} {\color{purple}\ding{183}}} }%
{% sinon
\ifthenelse{\equal{#2}{2}}   {\hfill{ \ding{172}  \ding{183} \ding{174} } \addcontentsline{toc}{section}{#1 \, {\color{purple}\ding{183}}} }
{% sinon
\ifthenelse{\equal{#2}{2,5}}   {\hfill{ \ding{172}  \ding{183} \ding{184} } \addcontentsline{toc}{section}{#1 \, {\color{purple}\ding{183}}  {\color{orange}\ding{184}}} }%
{% sinon
\hfill{ \ding{172}  \ding{173} \ding{184} } \addcontentsline{toc}{section}{#1 \,{\color{orange}\ding{184}}}% 
}%
}%
}%
}%
}%
}

%%%%%%%%%%%%% Titre
\newenvironment{titre}[2][]{%
\vspace{0.5cm}
\begin{tcolorbox}[enhanced, lifted shadow={0mm}{0mm}{0mm}{0mm}%
{black!60!white}, attach boxed title to top left={xshift=110mm, yshift*=-3mm}, coltitle=violet, colback=bleu3!25!white, boxed title style={colback=white!100}, colframe=bleu3,title=\stepcounter{cpttitre} \textbf{Fiche \thecpttitre}. #1 #2 ]}
{%
\end{tcolorbox}
\par}



%%%%%%%%%%%%% Définitions
\newenvironment{Def}[1][]{%
\medskip \begin{tcolorbox}[widget,colback=sacado_violet!0,colframe=sacado_violet!75,
adjusted title= \stepcounter{cptdef} Définition \thecptdef . {#1} ]}
{%
\end{tcolorbox}\par}


\newenvironment{DefT}[2][]{%
\medskip \begin{tcolorbox}[widget,colback=sacado_violet!0,colframe=sacado_violet!75,
adjusted title= \stepcounter{cptdef} Définition \thecptdef . {#1} \textit{#2}]}
{%
\end{tcolorbox}\par}

%%%%%%%%%%%%% Proposition
\newenvironment{Prop}[1][]{%
\medskip \begin{tcolorbox}[widget,colback=sacado_blue!0,colframe=sacado_blue!75!black,
adjusted title= \stepcounter{cpttheo} Proposition \thecpttheo . {#1} ]}
{%
\end{tcolorbox}\par}

%%%%%%%%%%%%% Propriétés
\newenvironment{Pp}[1][]{%
\medskip \begin{tcolorbox}[widget,colback=sacado_blue!0,colframe=sacado_blue!75!black,
adjusted title= \stepcounter{cpttheo} Propriété \thecpttheo . {#1}]}
{%
\end{tcolorbox}\par}

\newenvironment{PpT}[2][]{%
\medskip \begin{tcolorbox}[widget,colback=sacado_blue!0,colframe=sacado_blue!75!black,
adjusted title= \stepcounter{cpttheo} Propriété \thecpttheo . {#1} #2]}
{%
\end{tcolorbox}\par}

\newenvironment{Pps}[1][]{%
\medskip \begin{tcolorbox}[widget,colback=sacado_blue!0,colframe=sacado_blue!75!black,
adjusted title= \stepcounter{cpttheo} Propriétés \thecpttheo . {#1}]}
{%
\end{tcolorbox}\par}

%%%%%%%%%%%%% Théorèmes
\newenvironment{ThT}[2][]{% théorème avec titre
\medskip \begin{tcolorbox}[widget,colback=sacado_blue!0,colframe=sacado_blue!75!black,
adjusted title= \stepcounter{cpttheo} Théorème \thecpttheo . {#1} #2]}
{%
\end{tcolorbox}\par}

\newenvironment{Th}[1][]{%
\medskip \begin{tcolorbox}[widget,colback=sacado_blue!0,colframe=sacado_blue!75!black,
adjusted title= \stepcounter{cpttheo} Théorème \thecpttheo . {#1}]}
{%
\end{tcolorbox}\par}

%%%%%%%%%%%%% Règles
\newenvironment{Reg}[1][]{%
\medskip \begin{tcolorbox}[widget,colback=sacado_blue!0,colframe=sacado_blue!75!black,
adjusted title= \stepcounter{cpttheo} Règle \thecpttheo . {#1}]}
{%
\end{tcolorbox}\par}

%%%%%%%%%%%%% REMARQUES
\newenvironment{Rq}[1][]{%
\begin{bclogo}[couleur=sacado_orange!0, arrondi =0.15, noborder=true, couleurBarre=sacado_orange, logo = \bcinfo ]{ 
{\color{info}\normalsize{Remarque#1}}}}
{%
\end{bclogo}
\par}


\newenvironment{Rqs}[1][]{%
\begin{bclogo}[couleur=sacado_orange!0, arrondi =0.15, noborder=true, couleurBarre=sacado_orange, logo = \bcinfo ]{ 
{\color{info}\normalsize{Remarques#1}}}}
{%
\end{bclogo}
\par}

%%%%%%%%%%%%% EXEMPLES
\newenvironment{Ex}[1][]{%
\begin{bclogo}[couleur=sacado_yellow!15, arrondi =0.15, noborder=true, couleurBarre=sacado_yellow, logo = \bclampe ]{ 
\normalsize{Exemple#1}}}
{%
\end{bclogo}
\par}




%%%%%%%%%%%%% Preuve
\newenvironment{Pv}[1][]{%
\begin{tcolorbox}[breakable, enhanced,widget, colback=sacado_blue!10!white,boxrule=0pt,frame hidden,
borderline west={1mm}{0mm}{sacado_blue!75}]
\textbf{Preuve#1 : }}
{%
\end{tcolorbox}
\par}


%%%%%%%%%%%%% PreuveROC
\newenvironment{PvR}[1][]{%
\begin{tcolorbox}[breakable, enhanced,widget, colback=sacado_blue!10!white,boxrule=0pt,frame hidden,
borderline west={1mm}{0mm}{sacado_blue!75}]
\textbf{Preuve (ROC)#1 : }}
{%
\end{tcolorbox}
\par}


%%%%%%%%%%%%% Compétences
\newenvironment{Cps}[1][]{%
\vspace{0.4cm}
\begin{tcolorbox}[enhanced, lifted shadow={0mm}{0mm}{0mm}{0mm}%
{black!60!white}, attach boxed title to top left={xshift=5mm, yshift*=-3mm}, coltitle=white, colback=white, boxed title style={colback=sacado_green!100}, colframe=sacado_green!75!black,title=\textbf{Compétences associées#1}]}
{%
\end{tcolorbox}
\par}

%%%%%%%%%%%%% Compétences Collège
\newenvironment{CpsCol}[1][]{%
\vspace{0.4cm}
\begin{tcolorbox}[breakable, enhanced,widget, colback=white ,boxrule=0pt,frame hidden,
borderline west={2mm}{0mm}{bleu3}]
\textbf{#1}}
{%
\end{tcolorbox}
\par}




%%%%%%%%%%%%% Attendus
\newenvironment{Ats}[1][]{%
\vspace{0.4cm}
\begin{tcolorbox}[enhanced, lifted shadow={0mm}{0mm}{0mm}{0mm}%
{black!60!white}, attach boxed title to top left={xshift=5mm, yshift*=-3mm}, coltitle=white, colback=white, boxed title style={colback=sacado_green!100}, colframe=sacado_green!75!black,title=\textbf{Attendus du chapitre#1}]}
{%
\end{tcolorbox}
\par}

%%%%%%%%%%%%% Méthode
\newenvironment{Mt}[1][]{%
\vspace{0.4cm}
\begin{bclogo}[couleur=sacado_blue!0, arrondi =0.15, noborder=true, couleurBarre=bleu3, logo = \bccrayon ]{ 
\normalsize{{\color{bleu3}Méthode #1}}}}
{%
\end{bclogo}
\par}


%%%%%%%%%%%%% Méthode en vidéo
\newcommand{\MtV}[2]{\vspace{0.4cm} \colorbox{sacado_blue!0}{\hspace{0.2 cm}\tikz\node[rounded corners=1pt,draw] {\color{red}$\blacktriangleright$}; \quad  \href{https://youtu.be/#1?rel=0}{\raisebox{0.8mm}{{\color{red}\textbf{Méthode en vidéo : #2}}}}}}


%%%%%%%%%%%%% A voir (AV) : Lien externe + vidéo non Youtube
\newcommand{\AV}[2]{\vspace{0.4cm} \colorbox{bleu1!0}{\hspace{0.2 cm}\tikz\node[rounded corners=1pt,draw] {\color{red}$\blacktriangleright$}; \quad  \href{#1}{\raisebox{0.8mm}{{\color{red}\textbf{#2}}}}}}


%%%%%%%%%%%%% Etymologie
\newenvironment{Ety}[1][]{%
\begin{bclogo}[couleur=sacado_green!0, arrondi =0.15, noborder=true, couleurBarre=sacado_green, logo = \bcplume ]{ 
\normalsize{{\color{sacado_green}Étymologie#1}}}}
{%
\end{bclogo}
\par}


%%%%%%%%%%%%% Notation
\newenvironment{Nt}[1][]{%
\begin{bclogo}[couleur=sacado_violet!0, arrondi =0.15, noborder=true, couleurBarre=sacado_violet!75, logo = \bccrayon ]{ 
\normalsize{{\color{violet!75}Notation#1}}}}
{%
\end{bclogo}
\par}
%%%%%%%%%%%%% Histoire
%\newenvironment{His}[1][]{%
%\begin{bclogo}[couleur=brown!30, arrondi =0.15, noborder=true, couleurBarre=brown, logo = \bcvaletcoeur ]{ 
%\normalsize{{\color{brown}Histoire des mathématiques#1}}}}
%{%
%\end{bclogo}
%\par}

\newenvironment{His}[1][]{%
\vspace{0.4cm}
\begin{tcolorbox}[enhanced, lifted shadow={0mm}{0mm}{0mm}{0mm}%
{brown!60!white}, attach boxed title to top left={xshift=5mm, yshift*=-3mm}, coltitle=white, colback=white, boxed title style={colback=brown!100}, colframe=brown!75!black,title=\textbf{Histoire des mathématiques#1}]}
{%
\end{tcolorbox}
\par}

%%%%%%%%%%%%% Attention
\newenvironment{Att}[1][]{%
\begin{bclogo}[couleur=red!0, arrondi =0.15, noborder=true, couleurBarre=red, logo = \bcattention ]{ 
\normalsize{{\color{red}Attention. #1}}}}
{%
\end{bclogo}
\par}


%%%%%%%%%%%%% Conséquence
\newenvironment{Cq}[1][]{%
\textbf{Conséquence #1}}
{%
\par}

%%%%%%%%%%%%% Vocabulaire
\newenvironment{Voc}[1][]{%
\setlength{\logowidth}{10pt}
%\begin{footnotesize}
\begin{bclogo}[ noborder , couleur=white, logo =\bcbook]{#1}}
{%
\end{bclogo}
%\end{footnotesize}
\par}


%%%%%%%%%%%%% Video
\newenvironment{Vid}[1][]{%
\setlength{\logowidth}{12pt}
\begin{bclogo}[ noborder , couleur=white,barre=none, logo =\bcoeil]{#1}}
{%
\end{bclogo}
\par}


%%%%%%%%%%%%% Syntaxe
\newenvironment{Syn}[1][]{%
\begin{bclogo}[couleur=violet!0, arrondi =0.15, noborder=true, couleurBarre=violet!75, logo = \bcicosaedre ]{ 
\normalsize{{\color{violet!75}Syntaxe#1}}}}
{%
\end{bclogo}
\par}

%%%%%%%%%%%%% Auto évaluation
\newenvironment{autoeval}[1][]{%
\vspace{0.4cm}
\begin{tcolorbox}[enhanced, lifted shadow={0mm}{0mm}{0mm}{0mm}%
{black!60!white}, attach boxed title to top left={xshift=5mm, yshift*=-3mm}, coltitle=white, colback=white, boxed title style={colback=sacado_green!100}, colframe=sacado_green!75!black,title=\textbf{J'évalue mes compétences#1}]}
{%
\end{tcolorbox}
\par}


\newenvironment{autotest}[1][]{%
\vspace{0.4cm}
\begin{tcolorbox}[enhanced, lifted shadow={0mm}{0mm}{0mm}{0mm}%
{red!60!white}, attach boxed title to top left={xshift=5mm, yshift*=-3mm}, coltitle=white, colback=white, boxed title style={colback=red!100}, colframe=red!75!black,title=\textbf{Pour faire le point #1}]}
{%
\end{tcolorbox}
\par}

\newenvironment{ExOApp}[1][]{% Exercice d'application direct
\vspace{0.4cm}
\begin{tcolorbox}[enhanced, lifted shadow={0mm}{0mm}{0mm}{0mm}%
{red!60!white}, attach boxed title to top left={xshift=5mm, yshift*=-3mm}, coltitle=white, colback=white, boxed title style={colback=sacado_green!100}, colframe=sacado_green!75!black,title=\textbf{Application #1}]}
{%
\end{tcolorbox}
\par}

\newenvironment{ExOInt}[1][]{% Exercice d'application direct
\vspace{0.4cm}
\begin{tcolorbox}[enhanced, lifted shadow={0mm}{0mm}{0mm}{0mm}%
{red!60!white}, attach boxed title to top left={xshift=5mm, yshift*=-3mm}, coltitle=white, colback=white, boxed title style={colback=sacado_green!50}, colframe=sacado_green!75!black,title=\textbf{Exercice #1}]}
{%
\end{tcolorbox}
\par}

%Illustrations
\newtcolorbox{Illqr}[1]{
  enhanced,
  colback=white,
  colframe=ill_frame,
  colbacktitle=ill_back,
  coltitle=ill_title,
  title=\textbf{Illustration},
  boxrule=1pt, % épaisseur du trait à 1pt
  center,
  overlay={
    \node[anchor=south east, inner sep=0pt,xshift=-1pt,yshift=2pt,fill=white] at (frame.south east) {\fancyqr[height=1cm]{#1}};
  },
  after=\par,
  before=\vspace{0.4cm},
}

\newtcolorbox{Ill}{
  enhanced,
  colback=white,
  colframe=ill_frame,
  colbacktitle=ill_back,
  coltitle=ill_title,
  title=\textbf{Illustration},
  boxrule=1pt, % épaisseur du trait à 1pt
  center,
  after=\par,
  before=\vspace{0.4cm},
}

%%%%%%%%%%%%%% Propriétés
%\newenvironment{Pp}[1][]{%
%\medskip \begin{tcolorbox}[widget,colback=sacado_blue!0,colframe=sacado_blue!75!black,
%adjusted title= \stepcounter{cpttheo} Propriété \thecpttheo . {#1}]}
%{%
%\end{tcolorbox}\par}

%%%%% Pour réinitialiser numéros des chapitres après une nouvelle partie
% \makeatletter
    % \@addtoreset{section}{part}
% \makeatother

% \newcommand{\EPC}[3]{ % Exercice par compétence de niveau 1
% \ifthenelse{\equal{#1}{1}}
% {%condition2 vraie
% \vspace{0.4cm}
% \stepcounter{cptex}
% \tikz\node[rounded corners=0pt,draw,fill=bleu2]{\color{white}\textbf{ \thecptex}}; \quad  {\color{bleu2}\textbf{#3}}
% \input{#2}
% }% fin condition2 vraie
% {%condition2 fausse
% \vspace{0.4cm}
% \stepcounter{cptex}
% \tikz\node[rounded corners=2pt,draw,fill=eduscol4P]{\color{white}\textbf{ \thecptex}}; \quad  {\color{eduscol4P} \textbf{En temps libre.} \textbf{ #3}} 
% \input{#2}
% }% fin condition2 fausse
% } % fin de la procédure

\usepackage{hyperref}

\begin{document}

%-------------------------------
%	TITLE SECTION
%-------------------------------

\fancyhead[C]{}
\hrule\medskip % Upper rule
\begin{minipage}{0.295\textwidth} 
\raggedright
Classe \myClasse \hfill\\
\myDiscipline \hfill\\
\myParcours \hfill\\
\end{minipage}
\begin{minipage}{0.4\textwidth} 
\centering 
\scshape\huge
\textcolor{sacado_purple}{\myTitle} \\ 
\normalsize 
%\mySubTitle \\ 
\end{minipage}
\begin{minipage}{0.295\textwidth} 
\raggedleft
\href{https://sacado.xyz/}{\includegraphics[width=.2\linewidth]{sacadoA1.png}}
%\myAnnee \hfill\\
\end{minipage}
\medskip \hrule
\bigskip

%-------------------------------
%	CONTENTS
%-------------------------------

%\chapter{Polygones et cercles}
%{https://sacado.xyz/qcm/parcours_show_course/0/117131}
%
%\begin{pageCours}

\section{Polygones}

\mini{.48\linewidth}{\begin{Def}
Un \textbf{polygone} est une \textbf{figure} géométrique \textbf{fermée} dont les \textbf{côtés} sont des \textbf{segments} de droites.
\end{Def}}
{.48\linewidth}{
\begin{Ill}
Le polygone $ABCDE$ a $5$ sommets et $5$ côtés.

\begin{tikzpicture}[line cap=round,line join=round,>=triangle 45,x=.8cm,y=.8cm]
\clip(-2,-2) rectangle (5,4);
\fill[line width=2.pt,color=zzttqq,fill=zzttqq,fill opacity=0.10000000149011612] (-1.063311344808564,1.6089901265883788) -- (1.4923210474495676,2.7760130109143466) -- (2.1127635935469176,0.9442302557697901) -- (4.2695400633138965,0.) -- (2.467302191316832,-1.330725746587158) -- cycle;
\draw [line width=2.pt,color=zzttqq] (-1.063311344808564,1.6089901265883788)-- (1.4923210474495676,2.7760130109143466);
\draw [line width=2.pt,color=zzttqq] (1.4923210474495676,2.7760130109143466)-- (2.1127635935469176,0.9442302557697901);
\draw [line width=2.pt,color=zzttqq] (2.1127635935469176,0.9442302557697901)-- (4.2695400633138965,0.);
\draw [line width=2.pt,color=zzttqq] (4.2695400633138965,0.)-- (2.467302191316832,-1.330725746587158);
\draw [line width=2.pt,color=zzttqq] (2.467302191316832,-1.330725746587158)-- (-1.063311344808564,1.6089901265883788);
\draw[color=ududff] (-0.9599042537923393,1.8822802957026874) node {$A$};
\draw[color=ududff] (1.5957281384657929,3.049303180028654) node {$B$};
\draw[color=ududff] (2.2161706845631426,1.2175204248840987) node {$C$};
\draw[color=xdxdff] (4.372947154330122,0.272084164164328) node {$D$};
\draw[color=ududff] (2.570709282333057,-1.0574355774728494) node {$E$};
\end{tikzpicture}
\end{Ill}}

\mini{.58\linewidth}{\begin{DefT}{Polygones réguliers}
Un polygone est dit \textbf{régulier} lorsque tous ses côtés sont de \textbf{même longueur} et tous ses \textbf{angles} ont la \textbf{même} mesure.
\end{DefT}
\begin{Rq}
Il existe un \textbf{cercle} qui passe par \textbf{tous les sommets} d'un polygone régulier.
\end{Rq}}
{.38\linewidth}{\begin{Illqr}{Appliquette poly régulier}
\begin{center}
\begin{tikzpicture}[line cap=round,line join=round,>=triangle 45,x=0.8cm,y=0.8cm]
\clip(-3,-3) rectangle (3,3);
\fill[line width=2.pt,color=zzttqq,fill=zzttqq,fill opacity=0.10000000149011612] (2.,0.) -- (-1.,1.7320508075688772) -- (-1.,-1.7320508075688774) -- cycle;
\draw [shift={(2.,0.)},line width=2.pt,color=xdxdff,fill=xdxdff,fill opacity=0.10000000149011612] (0,0) -- (150.:0.5323591126142355) arc (150.:210.:0.5323591126142355) -- cycle;
\draw [shift={(-1.,-1.7320508075688774)},line width=2.pt,color=xdxdff,fill=xdxdff,fill opacity=0.10000000149011612] (0,0) -- (30.:0.5323591126142355) arc (30.:90.:0.5323591126142355) -- cycle;
\draw [shift={(-1.,1.7320508075688772)},line width=2.pt,color=xdxdff,fill=xdxdff,fill opacity=0.10000000149011612] (0,0) -- (-90.:0.5323591126142355) arc (-90.:-30.:0.5323591126142355) -- cycle;
\draw [line width=2.pt] (0.,0.) circle (1.6cm);
\draw [line width=2.pt,color=zzttqq] (2.,0.)-- (-1.,1.7320508075688772);
\draw [line width=2.pt,color=zzttqq] (-1.,1.7320508075688772)-- (-1.,-1.7320508075688774);
\draw [line width=2.pt,color=zzttqq] (-1.,-1.7320508075688774)-- (2.,0.);
\draw [line width=1.pt,dash pattern=on 1pt off 3pt] (2.,0.)-- (0.,0.);
\draw [line width=1.pt,dash pattern=on 1pt off 3pt] (-1.,-1.7320508075688774)-- (0.,0.);
\draw [line width=1.pt,dash pattern=on 1pt off 3pt] (-1.,1.7320508075688772)-- (0.,0.);
\draw[color=xdxdff] (2.073110611619774,0.19681571921764635) node {$A$};
\draw[color=ududff] (-0.9293947835245144,1.9323064263400531) node {$B$};
\draw[color=qqqqff] (-0.9293947835245144,-1.5386749879047603) node {$C$};
\end{tikzpicture}
\end{center}
\end{Illqr}}

\section{Triangles}

\mini{.38\linewidth}{\begin{Def}
Un \textbf{triangle} est un polygone ayant \textbf{3 côtés}.

Un triangle a 3 sommets.
\end{Def}}
{.68\linewidth}{
\begin{Ill}
\centering
\begin{tikzpicture}[line cap=round,line join=round,>=triangle 45,x=0.8cm,y=0.8cm]
\clip(-3,-3) rectangle (3,3);
\fill[line width=2.pt,color=zzttqq,fill=zzttqq,fill opacity=0.10000000149011612] (-2.270939747312388,-0.20245361524303004) -- (-0.23732793712600822,2.12927929800732) -- (2.,-2.) -- cycle;
\draw [shift={(2.,-2.)},line width=2.pt,color=violet,fill=violet,fill opacity=0.10000000149011612] (0,0) -- (118.44974022885543:0.47912320135281194) arc (118.44974022885543:157.1748246920994:0.47912320135281194) -- cycle;
\draw [shift={(-2.270939747312388,-0.20245361524303004)},line width=2.pt,color=xdxdff,fill=xdxdff,fill opacity=0.10000000149011612] (0,0) -- (-22.82517530790062:0.47912320135281194) arc (-22.82517530790062:48.9068163841034:0.47912320135281194) -- cycle;
\draw [shift={(-0.23732793712600822,2.12927929800732)},line width=2.pt,color=qqwuqq,fill=qqwuqq,fill opacity=0.10000000149011612] (0,0) -- (-131.09318361589658:0.47912320135281194) arc (-131.09318361589658:-61.55025977114456:0.47912320135281194) -- cycle;
\draw [line width=2.pt,color=zzttqq] (-2.270939747312388,-0.20245361524303004)-- (-0.23732793712600822,2.12927929800732);
\draw [line width=2.pt,color=zzttqq] (-0.23732793712600822,2.12927929800732)-- (2.,-2.);
\draw [line width=2.pt,color=zzttqq] (2.,-2.)-- (-2.270939747312388,-0.20245361524303004);
\draw[color=ududff] (-2.196409471546395,-0.005480743575763017) node {$A$};
\draw[color=ududff] (-0.16279766136001528,2.326252169674587) node {$B$};
\draw[color=ududff] (2.073110611619774,-1.8048545442118777) node {$C$};
\end{tikzpicture}
\end{Ill}}

\subsection{Triangles particuliers}

\mini{.48\linewidth}{
\begin{DefT}{Triangle équilatéral}
Un triangle \textbf{équilatéral} est un triangle dont les trois côtés ont la même mesure.
\end{DefT}
\begin{Pp}
Dans un triangle équilatéral les \textbf{trois} angles ont la même mesure : 60°.
\end{Pp}
}{.48\linewidth}{
\begin{Ill}
\centering
\begin{tikzpicture}[line cap=round,line join=round,>=triangle 45,x=0.8cm,y=0.8cm]
\clip(-3,-3) rectangle (3,3);
\fill[line width=2.pt,color=zzttqq,fill=zzttqq,fill opacity=0.10000000149011612] (-1.3552820736159028,-1.5972344902923261) -- (2.04116906486292,-0.5112219005592864) -- (-0.5975709959150303,1.8871847732094331) -- cycle;
\draw [shift={(2.04116906486292,-0.5112219005592864)},line width=2.pt,color=xdxdff,fill=xdxdff,fill opacity=0.10000000149011612] (0,0) -- (137.73163652585896:0.5323591126142355) arc (137.73163652585896:197.731636525859:0.5323591126142355) -- cycle;
\draw [shift={(-1.3552820736159028,-1.5972344902923261)},line width=2.pt,color=xdxdff,fill=xdxdff,fill opacity=0.10000000149011612] (0,0) -- (17.731636525858974:0.5323591126142355) arc (17.731636525858974:77.73163652585897:0.5323591126142355) -- cycle;
\draw [shift={(-0.5975709959150303,1.8871847732094331)},line width=2.pt,color=xdxdff,fill=xdxdff,fill opacity=0.10000000149011612] (0,0) -- (-102.26836347414104:0.5323591126142355) arc (-102.26836347414104:-42.26836347414104:0.5323591126142355) -- cycle;
\draw [line width=2.pt,color=zzttqq] (-1.3552820736159028,-1.5972344902923261)-- (2.04116906486292,-0.5112219005592864);
\draw [line width=2.pt,color=zzttqq] (2.04116906486292,-0.5112219005592864)-- (-0.5975709959150303,1.8871847732094331);
\draw [line width=2.pt,color=zzttqq] (-0.5975709959150303,1.8871847732094331)-- (-1.3552820736159028,-1.5972344902923261);
\draw[color=ududff] (-1.674697541184444,-1.3896144363727745) node {$A$};
\draw[color=ududff] (2.115699340628913,-0.31424902889201944) node {$B$};
\draw[color=qqqqff] (-0.5248018579376954,2.081366977872039) node {$C$};
\draw[color=xdxdff] (1.1787473024278583,-0.38877930465801236) node {$60\textrm{\degre}$};
\end{tikzpicture}
\end{Ill}
}

\mini{.48\linewidth}{
\begin{DefT}{Triangle isocèle}
Un triangle \textbf{isocèle} est un triangle dont deux côtés ont la même mesure.
\end{DefT}
\begin{Pp}
Un triangle isocèle posséde \textbf{deux} angles de \textbf{même mesure}.
\end{Pp}
}{.48\linewidth}{
\begin{Ill}
\centering
\begin{tikzpicture}[line cap=round,line join=round,>=triangle 45,x=0.8cm,y=0.8cm]
\clip(-3,-3) rectangle (3,3);
\fill[line width=2.pt,color=zzttqq,fill=zzttqq,fill opacity=0.10000000149011612] (-1.5256369896524582,-1.171347200200938) -- (1.9027556955832186,-2.35318443020454) -- (1.479155321328516,1.9816251921209547) -- cycle;
\draw [shift={(1.9027556955832186,-2.35318443020454)},line width=2.pt,color=xdxdff,fill=xdxdff,fill opacity=0.10000000149011612] (0,0) -- (95.58125955604939:0.31941546756854133) arc (95.58125955604939:160.97987786167886:0.31941546756854133) -- cycle;
\draw [shift={(-1.5256369896524582,-1.171347200200938)},line width=2.pt,color=xdxdff,fill=xdxdff,fill opacity=0.10000000149011612] (0,0) -- (-19.02012213832115:0.31941546756854133) arc (-19.02012213832115:46.378496167308306:0.31941546756854133) -- cycle;
\draw [shift={(1.479155321328516,1.9816251921209547)},line width=2.pt,color=qqwuqq,fill=qqwuqq,fill opacity=0.10000000149011612] (0,0) -- (-133.6215038326917:0.31941546756854133) arc (-133.6215038326917:-84.41874044395061:0.31941546756854133) -- cycle;
\draw [line width=2.pt,color=zzttqq] (-1.5256369896524582,-1.171347200200938)-- (1.9027556955832186,-2.35318443020454);
\draw [line width=2.pt,color=zzttqq] (1.9027556955832186,-2.35318443020454)-- (1.479155321328516,1.9816251921209547);
\draw [line width=2.pt,color=zzttqq] (1.479155321328516,1.9816251921209547)-- (-1.5256369896524582,-1.171347200200938);
\draw[color=ududff] (-1.8450524572209994,-0.9637271462813863) node {$A$};
\draw[color=ududff] (1.9772859713492117,-2.156211558537273) node {$B$};
\draw[color=xdxdff] (1.551398681257823,2.177191618142601) node {$C$};
\end{tikzpicture}
\end{Ill}
}

\mini{.48\linewidth}{
\begin{DefT}{Triangle rectangle}
Un triangle \textbf{rectangle} est un triangle dont un \textbf{angle est droit} (sa mesure est de 90°).
\end{DefT}
\begin{Def}
Dans un triangle  rectangle, le côté opposé à l'angle droite s'appelle l' \textbf{hypoténuse} du triangle.
\end{Def}
}{.48\linewidth}{
\begin{Ill}
\centering
\begin{tikzpicture}[line cap=round,line join=round,>=triangle 45,x=0.8cm,y=0.8cm]
\clip(-3,-3) rectangle (3,3);
\fill[line width=2.pt,color=zzttqq,fill=zzttqq,fill opacity=0.10000000149011612] (-1.5256369896524582,-1.171347200200938) -- (2.2328183454040444,-0.6496352698389876) -- (1.8498215704497243,2.109504761974792) -- cycle;
\draw[line width=2.pt,color=xdxdff,fill=xdxdff,fill opacity=0.10000000149011612] (2.1862372898057174,-0.3140615427734898) -- (1.85066356274022,-0.3606425983718165) -- (1.8972446183385467,-0.6962163254373143) -- (2.2328183454040444,-0.6496352698389876) -- cycle; 
\draw [line width=2.pt,color=zzttqq] (-1.5256369896524582,-1.171347200200938)-- (2.2328183454040444,-0.6496352698389876);
\draw [line width=2.pt,color=zzttqq] (2.2328183454040444,-0.6496352698389876)-- (1.8498215704497243,2.109504761974792);
\draw [line width=2.pt,color=zzttqq] (1.8498215704497243,2.109504761974792)-- (-1.5256369896524582,-1.171347200200938);
\draw[color=ududff] (-1.8450524572209994,-0.9637271462813863) node {$A$};
\draw[color=ududff] (2.307348621170038,-0.45266239817172055) node {$B$};
\draw[color=xdxdff] (1.9240500600877881,2.3049578051700172) node {$C$};
\draw[color=xdxdff] (1.6046345925192467,-0.2610131176305959) node {$90\textrm{\degre}$};
\end{tikzpicture}
\end{Ill}
}

\subsection{Construire un triangle}

\begin{Mt}
\qr{•}{Construire un triangle dont on connait la longueur des trois côtés}
\end{Mt}

\section{Quadrilatères}

\mini{.38\linewidth}{\begin{Def}
Un quadrilatère est un polygone qui a 4 côtés.

$ABCD$ est un quadrilatère.

Les segments $[AC]$ et $[BD]$ sont les diagonales de $ABCD$.
\end{Def}}
{.68\linewidth}{
\begin{Illqr}{Quadrilatères avec une particularité}
\centering
\begin{tikzpicture}[line cap=round,line join=round,>=triangle 45,x=0.8cm,y=0.8cm]
\clip(-3,-3) rectangle (3,3);
\fill[line width=2.pt,color=zzttqq,fill=zzttqq,fill opacity=0.10000000149011612] (-1.5575785364093122,-1.171347200200938) -- (-1.0997496995610698,1.4265652693565296) -- (2.3818788969360307,1.8205110126910635) -- (2.296701438917753,-0.22374797974759944) -- cycle;
\draw [line width=2.pt,color=zzttqq] (-1.5575785364093122,-1.171347200200938)-- (-1.0997496995610698,1.4265652693565296);
\draw [line width=2.pt,color=zzttqq] (-1.0997496995610698,1.4265652693565296)-- (2.3818788969360307,1.8205110126910635);
\draw [line width=2.pt,color=zzttqq] (2.3818788969360307,1.8205110126910635)-- (2.296701438917753,-0.22374797974759944);
\draw [line width=2.pt,color=zzttqq] (2.296701438917753,-0.22374797974759944)-- (-1.5575785364093122,-1.171347200200938);
\draw [line width=1.pt,dash pattern=on 1pt off 3pt] (-1.5575785364093122,-1.171347200200938)-- (2.3818788969360307,1.8205110126910635);
\draw [line width=1.pt,dash pattern=on 1pt off 3pt] (-1.0997496995610698,1.4265652693565296)-- (2.296701438917753,-0.22374797974759944);
\draw[color=ududff] (-1.4830482606433193,-0.974374328533671) node {$A$};
\draw[color=ududff] (-1.0252194237950767,1.6235381410237966) node {$B$};
\draw[color=ududff] (2.4564091727020236,2.0174838843583305) node {$C$};
\draw[color=ududff] (2.371231714683746,-0.026775108080332427) node {$D$};
\end{tikzpicture}
\end{Illqr}
\begin{flushright}
Scan le QR code pour les particularités qu'on peut trouver sur les quadrilatères.
\end{flushright}
}

\subsection{Quadrilatères particuliers}

\mini{.48\linewidth}{
\begin{DefT}{Losange}
Un \textbf{losange} est un quadrilatère qui a ses \textbf{quatre côtés de même longueur}.
\end{DefT}
\begin{Pp}
Les \textbf{diagonales} d'un losange sont \textbf{perpendiculaires}.
\end{Pp}
}{.48\linewidth}{
\begin{Ill}
\centering
\begin{tikzpicture}[line cap=round,line join=round,>=triangle 45,x=0.8cm,y=0.8cm]
\clip(-3,-3) rectangle (3,3);
\fill[line width=2.pt,color=zzttqq,fill=zzttqq,fill opacity=0.10000000149011612] (-1.,1.) -- (2.,2.) -- (1.,-1.) -- (-2.,-2.) -- cycle;
\draw[line width=2.pt,color=xdxdff,fill=xdxdff,fill opacity=0.10000000149011612] (0.15970773378427064,0.1597077337842706) -- (0.,0.3194154675685413) -- (-0.1597077337842706,0.15970773378427064) -- (0.,0.) -- cycle; 
\draw [line width=1.pt,dash pattern=on 1pt off 3pt] (-1.,1.)-- (1.,-1.);
\draw [line width=1.pt,dash pattern=on 1pt off 3pt] (-2.,-2.)-- (2.,2.);
\draw [line width=2.pt,color=zzttqq] (-1.,1.)-- (2.,2.);
\draw [line width=2.pt,color=zzttqq] (0.45454638211680987,1.5521874871992187) -- (0.4949495980129798,1.430977839510711);
\draw [line width=2.pt,color=zzttqq] (0.5050504019870204,1.569022160489289) -- (0.5454536178831904,1.4478125128007813);
\draw [line width=2.pt,color=zzttqq] (2.,2.)-- (1.,-1.);
\draw [line width=2.pt,color=zzttqq] (1.5690221604892896,0.5050504019870207) -- (1.4478125128007822,0.5454536178831906);
\draw [line width=2.pt,color=zzttqq] (1.5521874871992185,0.454546382116809) -- (1.4309778395107111,0.49494959801297883);
\draw [line width=2.pt,color=zzttqq] (1.,-1.)-- (-2.,-2.);
\draw [line width=2.pt,color=zzttqq] (-0.45454638211680987,-1.5521874871992187) -- (-0.4949495980129798,-1.4309778395107113);
\draw [line width=2.pt,color=zzttqq] (-0.5050504019870204,-1.5690221604892898) -- (-0.5454536178831904,-1.4478125128007824);
\draw [line width=2.pt,color=zzttqq] (-2.,-2.)-- (-1.,1.);
\draw [line width=2.pt,color=zzttqq] (-1.5690221604892896,-0.5050504019870207) -- (-1.4478125128007822,-0.5454536178831906);
\draw [line width=2.pt,color=zzttqq] (-1.5521874871992185,-0.454546382116809) -- (-1.4309778395107111,-0.49494959801297883);
\draw[color=ududff] (-0.9293947835245144,1.1976508509324084) node {$A$};
\draw[color=ududff] (1.0722754799050112,-0.8040194124971158) node {$B$};
\draw[color=ududff] (-1.930229915239277,-1.8048545442118777) node {$C$};
\draw[color=ududff] (2.073110611619774,2.1984859826471705) node {$D$};
\end{tikzpicture}
\end{Ill}
}

\mini{.48\linewidth}{
\begin{DefT}{Rectangle}
Un \textbf{rectangle} est un quadrilatère qui possède \textbf{quatre angles droits}.
\end{DefT}
\begin{Pp}
Les \textbf{diagonales} d'un rectangle ont la \textbf{même longueur}.
\end{Pp}
}{.48\linewidth}{
\begin{Ill}
\centering
\begin{tikzpicture}[line cap=round,line join=round,>=triangle 45,x=0.8cm,y=0.8cm]
\clip(-3,-3) rectangle (3,3);
\fill[line width=2.pt,color=zzttqq,fill=zzttqq,fill opacity=0.10000000149011612] (-2.,1.) -- (2.,1.) -- (2.,-1.) -- (-2.,-1.) -- cycle;
\draw[line width=2.pt,color=xdxdff,fill=xdxdff,fill opacity=0.10000000149011612] (1.7741391568664127,1.) -- (1.7741391568664127,0.7741391568664128) -- (2.,0.7741391568664128) -- (2.,1.) -- cycle; 
\draw[line width=2.pt,color=xdxdff,fill=xdxdff,fill opacity=0.10000000149011612] (2.,-0.7741391568664128) -- (1.7741391568664127,-0.7741391568664127) -- (1.7741391568664127,-1.) -- (2.,-1.) -- cycle; 
\draw[line width=2.pt,color=xdxdff,fill=xdxdff,fill opacity=0.10000000149011612] (-1.7741391568664127,-1.) -- (-1.7741391568664127,-0.7741391568664128) -- (-2.,-0.7741391568664128) -- (-2.,-1.) -- cycle; 
\draw[line width=2.pt,color=xdxdff,fill=xdxdff,fill opacity=0.10000000149011612] (-2.,0.7741391568664128) -- (-1.7741391568664127,0.7741391568664128) -- (-1.7741391568664127,1.) -- (-2.,1.) -- cycle; 
\draw [line width=2.pt,color=zzttqq] (-2.,1.)-- (2.,1.);
\draw [line width=2.pt,color=zzttqq] (2.,1.)-- (2.,-1.);
\draw [line width=2.pt,color=zzttqq] (2.,-1.)-- (-2.,-1.);
\draw [line width=2.pt,color=zzttqq] (-2.,-1.)-- (-2.,1.);
\draw [line width=1.pt,dash pattern=on 1pt off 3pt] (-2.,1.)-- (2.,-1.);
\draw [line width=1.pt,dash pattern=on 1pt off 3pt] (-2.,-1.)-- (2.,1.);
\draw[color=ududff] (-1.930229915239277,1.1976508509324084) node {$A$};
\draw[color=ududff] (2.073110611619774,1.1976508509324084) node {$B$};
\draw[color=ududff] (2.073110611619774,-0.8040194124971158) node {$C$};
\draw[color=ududff] (-1.930229915239277,-0.8040194124971158) node {$D$};
\end{tikzpicture}
\end{Ill}
}

\mini{.48\linewidth}{
\begin{DefT}{Carré}
Un \textbf{carré} est un quadrilatère qui possède \textbf{quatre angles droits} et qui a ses \textbf{quatre côtés de même longueur}.
\end{DefT}
\begin{Pp}
Les \textbf{diagonales} d'un carré sont \textbf{perpendiculaires} et de \textbf{même longueur}.
\end{Pp}
}{.48\linewidth}{
\begin{Ill}
\centering
\begin{tikzpicture}[line cap=round,line join=round,>=triangle 45,x=0.8cm,y=0.8cm]
\clip(-3,-3) rectangle (3,3);
\fill[line width=2.pt,color=zzttqq,fill=zzttqq,fill opacity=0.10000000149011612] (-0.39703567091027886,-2.0444161448882836) -- (2.0092275181060657,-0.18115925073846062) -- (0.14597062395624327,2.2251039382778837) -- (-2.2602925650601016,0.3618470441280611) -- cycle;
\draw[line width=2.pt,color=xdxdff,fill=xdxdff,fill opacity=0.10000000149011612] (-0.09213762139290618,2.0407280845783213) -- (0.0922382323066562,1.8026198392291717) -- (0.33034647765580566,1.9869956929287342) -- (0.14597062395624327,2.2251039382778837) -- cycle; 
\draw[line width=2.pt,color=xdxdff,fill=xdxdff,fill opacity=0.10000000149011612] (1.8248516644065031,0.05694899461068881) -- (1.5867434190573537,-0.12742685908887366) -- (1.7711192727569163,-0.36553510443802306) -- (2.0092275181060657,-0.18115925073846062) -- cycle; 
\draw[line width=2.pt,color=xdxdff,fill=xdxdff,fill opacity=0.10000000149011612] (-0.15892742556112943,-1.860040291188721) -- (-0.34330327926069193,-1.6219320458395716) -- (-0.5814115246098414,-1.8063078995391342) -- (-0.39703567091027886,-2.0444161448882836) -- cycle; 
\draw[line width=2.pt,color=xdxdff,fill=xdxdff,fill opacity=0.10000000149011612] (-2.0759167113605392,0.1237387987789117) -- (-1.8378084660113896,0.30811465247847414) -- (-2.022184319710952,0.5462228978276236) -- (-2.2602925650601016,0.3618470441280611) -- cycle; 
\draw[line width=2.pt,color=xdxdff,fill=xdxdff,fill opacity=0.10000000149011612] (-0.09228730228532271,0.3517425966530293) -- (-0.35368600224355173,0.3849878178447245) -- (-0.3869312234352469,0.12358911788649543) -- (-0.12553252347701782,0.09034389669480025) -- cycle; 
\draw [line width=2.pt,color=zzttqq] (-0.39703567091027886,-2.0444161448882836)-- (2.0092275181060657,-0.18115925073846062);
\draw [line width=2.pt,color=zzttqq] (0.7459379042986362,-1.0785739883852803) -- (0.8241619541614476,-1.1795947613509685);
\draw [line width=2.pt,color=zzttqq] (0.7880298930343392,-1.0459806342757758) -- (0.8662539428971506,-1.147001407241464);
\draw [line width=2.pt,color=zzttqq] (2.0092275181060657,-0.18115925073846062)-- (0.14597062395624327,2.2251039382778837);
\draw [line width=2.pt,color=zzttqq] (1.0433853616030633,0.9618143244704542) -- (1.1444061345687517,1.0400383743332657);
\draw [line width=2.pt,color=zzttqq] (1.0107920074935577,1.0039063132061572) -- (1.1118127804592461,1.0821303630689687);
\draw [line width=2.pt,color=zzttqq] (0.14597062395624327,2.2251039382778837)-- (-2.2602925650601016,0.3618470441280611);
\draw [line width=2.pt,color=zzttqq] (-0.9970029512526712,1.2592617817748812) -- (-1.0752270011154828,1.3602825547405692);
\draw [line width=2.pt,color=zzttqq] (-1.0390949399883742,1.2266684276653763) -- (-1.1173189898511857,1.3276892006310645);
\draw [line width=2.pt,color=zzttqq] (-2.2602925650601016,0.3618470441280611)-- (-0.39703567091027886,-2.0444161448882836);
\draw [line width=2.pt,color=zzttqq] (-1.2944504085570996,-0.7811265310808538) -- (-1.395471181522788,-0.8593505809436652);
\draw [line width=2.pt,color=zzttqq] (-1.261857054447594,-0.8232185198165568) -- (-1.3628778274132824,-0.9014425696793682);
\draw [line width=1.pt,dash pattern=on 1pt off 3pt] (-2.2602925650601016,0.3618470441280611)-- (2.0092275181060657,-0.18115925073846062);
\draw [line width=1.pt,dash pattern=on 1pt off 3pt] (0.14597062395624327,2.2251039382778837)-- (-0.39703567091027886,-2.0444161448882836);
\draw[color=ududff] (-0.32250539514428594,-1.8474432732210166) node {$A$};
\draw[color=ududff] (2.0837577938720586,0.01581362092880638) node {$B$};
\draw[color=qqqqff] (0.2205008997222343,2.4220768099451493) node {$C$};
\draw[color=qqqqff] (-2.18576228929411,0.5588199157953262) node {$D$};
\end{tikzpicture}
\end{Ill}
}

\begin{Rqs}
\begin{itemize}
\item Un carré est un \textbf{rectangle particulier}, c'est un rectangle qui a ses 4 côtés de même longueur.
\item Un carré est un \textbf{losange particulier}, c'est un losange qui a  4 angles droits.
\end{itemize}
\end{Rqs}

\section{Cercles}

\mini{.58\linewidth}{\begin{Def}
Le cercle de centre $O$ et de rayon $r$ est l'ensemble des points situés à une distance de $O$ égale à $r$.
\end{Def}
\begin{Nt}
\[M\in C(O;r)\Leftrightarrow OM=r\]
\end{Nt}
}
{.38\linewidth}{
\begin{Illqr}{Animation cercle}
\centering
\begin{tikzpicture}[line cap=round,line join=round,>=triangle 45,x=0.8cm,y=0.8cm]
\clip(-3,-3) rectangle (3,3);
\draw [line width=2.pt] (0.,0.) circle (2);
\draw [line width=1.pt,dash pattern=on 1pt off 3pt] (0.,0.)-- (2.,0.);
\draw [color=qqqqff] (0.,0.)-- ++(-2.5pt,-2.5pt) -- ++(5.0pt,5.0pt) ++(-5.0pt,0) -- ++(5.0pt,-5.0pt);
\draw[color=xdxdff] (0.07144034819024836,.3) node {$O$};
\draw [color=qqqqff] (2.,0.)-- ++(-2.5pt,0 pt) -- ++(5.0pt,0 pt);% ++(-2.5pt,-2.5pt) -- ++(0 pt,5.0pt);
\draw[color=xdxdff] (2.2,0.19681571921764635) node {$M$};
\draw[color=black] (-0.9719835125336532,1.5809494120146579) node {$C$};
\draw[color=black] (1.0296867508958722,-0.06936383708947125) node {$r$};
\end{tikzpicture}
\end{Illqr}
\begin{flushright}
Scan le QR code pour voir l'animation.
\end{flushright}
}

\begin{Voc}
\mini{.58\linewidth}{
\begin{itemize}
    \item Les points $A$, $B$, $C$, $D$ et $M$ appartiennent au cercle $\mathcal{C}$ de centre $O$.
    \item Le segment $[OM]$ est aussi appelé le \textcolor{qqqqff}{\textbf{rayon}} du cercle $\mathcal{C}$.
    \item Le segment $[CD]$ est une \textcolor{xdxdff}{\textbf{corde}} du cercle $\mathcal{C}$.
    \item La corde $[AB]$ est un \textcolor{qqwuqq}{\textbf{diamètre}} du cercle $\mathcal{C}$ car le centre O est le \textbf{milieu} de la corde.
    \item Le diamètre $[AB]$ est le double du rayon $[OM]$ : $AB=2\times OM$
    \item La portion colorée en rouge du cercle $\mathcal{C}$ est l'\textcolor{ffqqqq}{\textbf{arc de cercle}} \t{CD}
\end{itemize}
}{.38\linewidth}{
\begin{Ill}
\centering
\begin{tikzpicture}[line cap=round,line join=round,>=triangle 45,x=0.8cm,y=0.8cm]
\clip(-3,-3) rectangle (3,3);
\draw [line width=2.pt] (0.,0.) circle (2);
\draw [line width=2.pt,color=qqqqff] (0.,0.)-- (2.,0.);
\draw [line width=2.pt,color=qqwuqq] (-1.5844956505091685,1.2203989239250939)-- (0.,0.);
\draw [line width=2.pt,color=qqwuqq] (-0.7743544134251696,0.677052966073721) -- (-0.8523172720063054,0.5758304822601795);
\draw [line width=2.pt,color=qqwuqq] (-0.7321783785028615,0.6445684416649147) -- (-0.8101412370839972,0.5433459578513732);
\draw [line width=2.pt,color=qqwuqq] (0.,0.)-- (1.5844956505091685,-1.2203989239250939);
\draw [line width=2.pt,color=qqwuqq] (0.8101412370839972,-0.5433459578513732) -- (0.7321783785028615,-0.6445684416649147);
\draw [line width=2.pt,color=qqwuqq] (0.8523172720063054,-0.5758304822601795) -- (0.7743544134251696,-0.677052966073721);
\draw [line width=2.pt,color=xdxdff] (-1.9584617055069056,0.4054969149857767)-- (-0.7472818915155957,-1.855146833706937);
\draw [shift={(0.,0.)},line width=2.pt,color=ffqqqq]  plot[domain=2.9374287928705773:4.32945781065304,variable=\t]({1.*2.*cos(\t r)+0.*2.*sin(\t r)},{0.*2.*cos(\t r)+1.*2.*sin(\t r)});
\draw [color=qqqqff] (0.,0.)-- ++(-2.5pt,-2.5pt) -- ++(5.0pt,5.0pt) ++(-5.0pt,0) -- ++(5.0pt,-5.0pt);
\draw[color=qqqqff] (0.07144034819024836,0.19681571921764635) node {$O$};
\draw [color=qqqqff] (2.,0.)-- ++(-2.5pt,0 pt) -- ++(5.0pt,0 pt) ++(-2.5pt,-2.5pt) -- ++(0 pt,5.0pt);
\draw[color=qqqqff] (2.073110611619774,0.19681571921764635) node {$M$};
\draw[color=qqqqff] (1.0296867508958722,-0.06936383708947125) node {$r$};
\draw [color=qqqqff] (-1.5844956505091685,1.2203989239250939)-- ++(-2.5pt,-2.5pt) -- ++(5.0pt,5.0pt) ++(-5.0pt,0) -- ++(5.0pt,-5.0pt);
\draw[color=qqqqff] (-1.5149898074001733,1.4212416782303874) node {$A$};
\draw [color=qqqqff] (1.5844956505091685,-1.2203989239250939)-- ++(-2.5pt,-2.5pt) -- ++(5.0pt,5.0pt) ++(-5.0pt,0) -- ++(5.0pt,-5.0pt);
\draw[color=qqqqff] (1.6578705037806702,-1.0276102397950946) node {$B$};
\draw [color=qqqqff] (-1.9584617055069056,0.4054969149857767)-- ++(-2.5pt,0 pt) -- ++(5.0pt,0 pt) ++(-2.5pt,-2.5pt) -- ++(0 pt,5.0pt);
\draw[color=qqqqff] (-1.8876411862301383,0.6014086448044651) node {$C$};
\draw [color=qqqqff] (-0.7472818915155957,-1.855146833706937)-- ++(-2.5pt,0 pt) -- ++(5.0pt,0 pt) ++(-2.5pt,-2.5pt) -- ++(0 pt,5.0pt);
\draw[color=qqqqff] (-0.6738624094696813,-1.655793992679892) node {$D$};
\end{tikzpicture}
\end{Ill}
}
\end{Voc}

\section{Les savoir-faire du parcours}

\begin{CpsCol}
\begin{itemize}
\item Savoir décrire un polygone.
\item Savoir reconnaître un polygone.
\item Savoir coder une figure.                  
\item Savoir reconnaître un polygone.
\item Savoir construire un polygone.
\item Savoir construire un cercle.
\item Savoir comparer des distances avec un cercle.
\end{itemize}
\end{CpsCol}


\end{document}

%\end{pageCours}
