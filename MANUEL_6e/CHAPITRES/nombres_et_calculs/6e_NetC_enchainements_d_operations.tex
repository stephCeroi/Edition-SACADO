\chapter{Enchainements d'opérations}
{https://sacado.xyz/qcm/parcours_show_course/0/117122}
{

 \begin{CpsCol}
\textbf{Les savoir-faire du parcours}
 \begin{itemize}
 \item Savoir utiliser le vocabulaire des opérations.
 \item Savoir traduire une expression numérique par une phrase.
 \item Savoir calculer en respectant les règles de priorités.
 \item Savoir calculer en utilisant la simple distributivité.
 \item Savoir résoudre un problème numérique.
 \end{itemize}
 \end{CpsCol}
}

\begin{pageCours} 

\section{Vocabulaire des opérations}

\begin{DefT}{Somme. Différence. Produit. Quotient}
\begin{itemize}[leftmargin=*]
\item La \textbf{somme}\index{Somme} de deux nombres, appelés \textbf{termes}\index{Termes}, est le résultat de l'addition de ces nombres.
\item La \textbf{différence}\index{Différence} de deux nombres, appelés \textbf{termes}\index{Termes}, est le résultat de la soustraction de ces nombres.
\item Le \textbf{produit}\index{Produit} de deux nombres, appelés \textbf{facteurs}\index{Facteurs}, est le résultat de la multiplication de ces nombres.
\item Le \textbf{quotient}\index{Quotient} de deux nombres est le résultat de la division de ces nombres.
\end{itemize}
\end{DefT}

\begin{Ex}

\begin{itemize}[leftmargin=*]
\item Calculer "la somme de $13$ et $8$" revient à faire l'opération : $13+8=21$. $13$ et $8$ sont les termes et la somme est $21$.
\item Calculer "le produit de $5$ et $7$" revient à faire l'opération : $5 \times 78=35$. $5$ et $7$ sont les facteurs et le produit est $35$.
\item La phrase : "$A$ est le quotient de $340$ par la somme de $12$ et de $5$" revient à faire l'opération : $A=340\div(12+5)=20$
\end{itemize}
\end{Ex}

\section{Règles de priorité}

\begin{Reg}
Pour effectuer un calcul avec plusieurs opérations, on suit les règles de priorité des opérations.
\begin{enumerate}[leftmargin=*]
\item Les opérations s'effectuent de gauche à droite.
\item Les opérations entre parenthèses sont prioritaires.
\item En l'absence de parenthèses, La multiplication est prioritaire sur l'addition et la soustraction.
\end{enumerate}
\end{Reg}

\begin{Ex}

\begin{enumerate}[leftmargin=*]
\begin{minipage}{0.5\linewidth}

\item $10-5+2=5+2=7$ 
\item $10\times5+2=52$
\end{minipage}
\begin{minipage}{0.5\linewidth}
\item $10\times(5+2)=70$
\item $A=14-\textcolor{red}{(}2,5+3,5\textcolor{red}{)}=14-\textcolor{red}{6}=8$
\end{minipage}
\end{enumerate}

\end{Ex}

\section{Distributivité}

\begin{DefT}{distributivité}
L'égalité $\textcolor{red}{3}\times(\textcolor{sacado_blue}{10}+\textcolor{orange}{11})=\textcolor{red}{3}\times\textcolor{sacado_blue}{10}+\textcolor{red}{3}\times\textcolor{orange}{11}$ s'appelle la \textbf{distributivité}\index{Distributivité} de la multiplication par rapport à l'addition.
\end{DefT}

\begin{Mt}
On peut utiliser la \textbf{distributivité} pour calculer astucieusement des produits en décomposant un des facteurs :
 
$A=25\times\textcolor{sacado_blue}{19} = 25 \times \textcolor{sacado_blue}{(20-1)} $
 
on utilise la distributivité : $A=25\times20-25\times1=500-25=475$
\end{Mt}

\begin{Mt}
On peut utiliser la \textbf{distributivité} pour calculer astucieusement des produits en décomposant un des facteurs :
 
\definecolor{xfqqff}{rgb}{0.4980392156862745,0.,1.}
\definecolor{ffqqff}{rgb}{1.,0.,1.}
\begin{tikzpicture}[line cap=round,line join=round,>=triangle 45,x=1.0cm,y=1.0cm]
\clip(-0.64,-0.08) rectangle (10.46,1.54);
\draw (-0.16,0.6) node[anchor=north west] {$15 \times {\color{sacado_red}12} = 15 \times ({\color{sacado_red}10+2 }) =   {\color{ffqqff} 15 \times10 }  + {\color{xfqqff} 15 \times 2 }={\color{ffqqff}150}+{\color{xfqqff}30}=180 $};

\draw [shift={(2.3,0.57)},line width=1.pt,color=ffqqff]  plot[domain=0.:3.141592653589793,variable=\t]({1.*0.37*cos(\t r)+0.*0.37*sin(\t r)},{0.*0.37*cos(\t r)+1.*0.37*sin(\t r)});
\draw [shift={(2.7,0.57)},line width=1.pt,color=xfqqff]  plot[domain=0.0:3.141592653589793,variable=\t]({1.*0.7011419257183239*cos(\t r)+0.*0.7011419257183239*sin(\t r)},{0.*0.7011419257183239*cos(\t r)+1.*0.7011419257183239*sin(\t r)});

\end{tikzpicture}

\end{Mt}

\end{pageCours} 


\begin{pageAD} 
 

\Sf{Connaitre le vocabulaire des opérations}
 
  
\begin{ExoCad}{Communiquer.}{1234}{2}{0}{0}{0}{0}

Écrire avec des nombres et des opérations les phrases proposées.

\begin{itemize}[leftmargin=*]
\item La somme de $5$ et $6$ égale à $11$ : \point{1}
\item $42$ est le produit des facteurs $6$ et $7$ : \point{1}
\item $102$ est la somme du produit de $9$ par $11$ et du quotient de $6$ par $2$ : \point{1}
\end{itemize}
 
\end{ExoCad}

\Sf{Connaitre les règles de priorités}

\begin{ExoCad}{Calculer.}{1234}{0}{0}{0}{0}{0}

 Calcule les opérations suivantes :
 
\begin{itemize}[leftmargin=*]
\item $6 -2+4=$ \point{1}
\item $23 - 6 - 2 \times 5=$   \point{1}
\item $12 \times 2 -  11 \times 2 =$   \point{1}
\item $12 + 15 \div 3=$   \point{1}
\end{itemize}
 
\end{ExoCad}


\Sf{Utiliser la distributivité}

\begin{ExoCad}{Représenter. Calculer.}{1234}{0}{0}{0}{0}{0}

Calcule les expressions suivantes en utilisant la distributivité.

\begin{itemize}[leftmargin=*]
\item $32 \times 9= (\ldots \ldots + \ldots \ldots) \times 9  = \ldots \ldots \times 9 + \ldots \ldots \times 9 = \ldots \ldots   + \ldots \ldots  = \ldots \ldots $ \vspace{0.1cm}
\item  $14 \times 9 = 14 \times (\ldots \ldots - \ldots \ldots)= 14\times \ldots \ldots - 14 \times \ldots \ldots = \ldots \ldots   - \ldots \ldots  = \ldots \ldots $ \vspace{0.1cm}
\item $31 \times 80 = (\ldots \ldots + \ldots \ldots) \times 80  = \ldots \ldots \times 80 + \ldots \ldots \times 80 = \ldots \ldots   + \ldots \ldots  = \ldots \ldots $ \vspace{0.1cm}
\item  $15 \times 18 = 15 \times (\ldots \ldots - \ldots \ldots)= 15\times \ldots \ldots - 15 \times \ldots \ldots = \ldots \ldots   - \ldots \ldots  = \ldots \ldots $ \vspace{0.1cm}
\end{itemize}
 
\end{ExoCad}

\begin{ExoCad}{Calculer.}{1234}{0}{0}{0}{0}{0}

Calcule les expressions suivantes en utilisant la distributivité.

\begin{itemize}[leftmargin=*]
\item  $3 \times 6 + 3 \times 4=3 \times (\ldots \ldots + \ldots \ldots)=$ \point{1}
\item $32 \times 9 + 9 \times 8=9 \times (\ldots \ldots + \ldots \ldots)=$ \point{1}
\item  $26 \times 17 - 26 \times 7=26 \times (\ldots \ldots - \ldots \ldots)=$ \point{1}
\item  $5 \times 101 - 5=5 \times (\ldots \ldots - \ldots \ldots)=$ \point{1}
\end{itemize}
 
\end{ExoCad}


\Sf{Résoudre un problème}

\begin{ExoCad}{Calculer.}{1234}{1}{0}{0}{0}{0}

Apolline possède des poules qui pondent $1\,224$ œufs par jour. 
Les œufs sont réparties dans des boites de 6. 

Combien de boites sont-elles remplies chaque jour ? 
\point{3}
\end{ExoCad}

\end{pageAD}
 


%%%%%%%%%%%%%%%%%%%%%%%%%%%%%%%%%%%%%%%%%%%%%%%%%%%%%%%%%%%%%%%%%%%
%%%%  Niveau 1
%%%%%%%%%%%%%%%%%%%%%%%%%%%%%%%%%%%%%%%%%%%%%%%%%%%%%%%%%%%%%%%%%%%
\begin{pageParcoursu} 

 %%%%%%%%%%%%%%%%%%%%%%%%%%%
\begin{ExoCu}{Calculer.}{1234}{1}{0}{0}{0}{0}

 Dans une bibliothèque, il y a $5600$ livres. $3267$ sont des romans, $1234$ sont des BD et les autres sont des documentaires.
 Quel est le nombre de livres documentaires ? \point{3}
\end{ExoCu}
 


%%%%%%%%%%%%%%%%%%%%%%%%%%%
\begin{ExoCu}{Représenter.}{1234}{2}{0}{0}{0}{0}

Clara a dans sa tirelire 124,17 euros. Elle achète un cadeau à sa mère pour une valeur de 51,30 euros.

Quelle somme reste-elle à Clara après avoir acheté le cadeau ? \point{3}

\end{ExoCu}
%%%%%%%%%%%%%%%%%%%%%%%%%%%
\begin{ExoCu}{Représenter.}{1234}{2}{0}{0}{0}{0}

\end{ExoCu}


%%%%%%%%%%%%%%%%%%%%%%%%%%%
\begin{ExoCu}{Raisonner.}{1234}{2}{0}{0}{0}{0}

\end{ExoCu}

%%%%%%%%%%%%%%%%%%%%%%%%%%%
\begin{ExoCu}{Représenter.}{1234}{2}{0}{0}{0}{0}


\end{ExoCu}


\end{pageParcoursu}

  
%%%%%%%%%%%%%%%%%%%%%%%%%%%%%%%%%%%%%%%%%%%%%%%%%%%%%%%%%%%%%%%%%%%
%%%%  Niveau 2
%%%%%%%%%%%%%%%%%%%%%%%%%%%%%%%%%%%%%%%%%%%%%%%%%%%%%%%%%%%%%%%%%%%



\begin{pageParcoursd} 
 
%%%%%%%%%%%%%%%%%%%%%%%%%%%%%%%%%%%%%%%%%%%%%%%%%%%%%%%%%%%%%%%%%%%
\begin{ExoCd}{Représenter.}{1234}{2}{0}{0}{0}{0}


 
\end{ExoCd}

 
%%%%%%%%%%%%%%%%%%%%%%%%%%%%%%%%%%%%%%%%%%%%%%%%%%%%%%%%%%%%%%%%%%%
\begin{ExoCd}{Chercher.communiquer.}{1234}{2}{0}{0}{0}{0}



\end{ExoCd}


%%%%%%%%%%%%%%%%%%%%%%%%%%%%%%%%%%%%%%%%%%%%%%%%%%%%%%%%%%%%%%%%%%%
\begin{ExoCd}{Représenter. Raisonner.}{1234}{2}{0}{0}{0}{0}


\end{ExoCd}

 %%%%%%%%%%%%%%%%%%%%%%%%%%%%%%%%%%%%%%%%%%%%%%%%%%%%%%%%%%%%%%%%%%%
\begin{ExoCd}{Représenter. Raisonner.}{1234}{2}{0}{0}{0}{0}


\end{ExoCd}
 
%%%%%%%%%%%%%%%%%%%%%%%%%%%%%%%%%%%%%%%%%%%%%%%%%%%%%%%%%%%%%%%%%%%
\begin{ExoCd}{Représenter. Raisonner.}{1234}{2}{0}{0}{0}{0}


\end{ExoCd}
 
\end{pageParcoursd}

%%%%%%%%%%%%%%%%%%%%%%%%%%%%%%%%%%%%%%%%%%%%%%%%%%%%%%%%%%%%%%%%%%%
%%%%  Niveau 3
%%%%%%%%%%%%%%%%%%%%%%%%%%%%%%%%%%%%%%%%%%%%%%%%%%%%%%%%%%%%%%%%%%%
\begin{pageParcourst}

%%%%%%%%%%%%%%%%%%%%%%%%%%%%%%%%%%%%%%%%%%%%%%%%%%%%%%%%%%%%%%%%%%%
\begin{ExoCt}{Représenter.}{1234}{2}{0}{0}{0}{0}

 

\end{ExoCt}

%%%%%%%%%%%%%%%%%%%%%%%%%%%%%%%%%%%%%%%%%%%%%%%%%%%%%%%%%%%%%%%%%%%
\begin{ExoCt}{Représenter. Raisonner.}{1234}{2}{0}{0}{0}{0}
 
 Quel est le nombre à quatre chiffres inférieur à 5000 qui est divisible par 2, par 3, par 4, … par 10 ? 


\end{ExoCt}


%%%%%%%%%%%%%%%%%%%%%%%%%%%%%%%%%%%%%%%%%%%%%%%%%%%%%%%%%%%%%%%%%%%
\begin{ExoCt}{Raisonner.}{1234}{2}{0}{0}{0}{0}
 
Le maître apporte un sac de gommettes à un groupe de 3 élèves. Chacun en prend une poignée. Le sac est vide et les élèves comptent leurs gommettes.
\begin{itemize}
\item Gilles : j’en ai 11
\item Nadia : j’en ai 9
\item Soledad : j’en ai 17
\end{itemize}
Le maître : Voici encore un autre sac, il contient 50 gommettes. 
Vous devez vous les partager de façon à ce que chacun ait finalement le même nombre de gommettes. 
 
 \point{5}
 
\hfill{\tiny D'après IREM Lyon} 
\end{ExoCt}

%%%%%%%%%%%%%%%%%%%%%%%%%%%%%%%%%%%%%%%%%%%%%%%%%%%%%%%%%%%%%%%%%%%
\begin{ExoCt}{Chercher. Représenter.}{1234}{2}{0}{0}{0}{0}

Combien de pièces de 2 € et de billets de 5 € au minimum faut-il combiner pour obtenir la somme de 23 € ? de 54 € ? de 81 € ?

\end{ExoCt}

\begin{ExoCt}{Chercher. Représenter.}{1234}{2}{0}{0}{0}{0}

Un fermier a des poules et des lapins. En regardant tous les animaux, il voit 5 têtes et 16 pattes. Combien le fermier a-t-il de lapins et de poules ?

\end{ExoCt}

 

\end{pageParcourst}

%%%%%%%%%%%%%%%%%%%%%%%%%%%%%%%%%%%%%%%%%%%%%%%%%%%%%%%%%%%%%%%%%%%
%%%%  Brouillon
%%%%%%%%%%%%%%%%%%%%%%%%%%%%%%%%%%%%%%%%%%%%%%%%%%%%%%%%%%%%%%%%%%%


\begin{pageBrouillon} 
 
\ligne{32}



\end{pageBrouillon}

%%%%%%%%%%%%%%%%%%%%%%%%%%%%%%%%%%%%%%%%%%%%%%%%%%%%%%%%%%%%%%%%%%%
%%%%  Auto
%%%%%%%%%%%%%%%%%%%%%%%%%%%%%%%%%%%%%%%%%%%%%%%%%%%%%%%%%%%%%%%%%%%


%%%%%%%%%%%%%%%%%%%%%%%%%%%%%%%%%%%%%%%%%%%%%%%%%%%%%%%%%%%%%%%%%%%
\begin{pageAuto} 


\begin{ExoAuto}{Raisonner.}{1234}{2}{0}{0}{0}{0}

 
%%%%%%%%%%%%%%%%%%%%%%%%%%%%%%%%%%%%%%%%%%%%%%%%%%%%%%%%%%%%%%%%%%%
\end{ExoAuto}

\begin{ExoAuto}{Raisonner.}{1234}{2}{0}{0}{0}{0}
  

\end{ExoAuto}

%%%%%%%%%%%%%%%%%%%%%%%%%%%%%%%%%%%%%%%%%%%%%%%%%%%%%%%%%%%%%%%%%%%
\begin{ExoAuto}{Raisonner.}{1234}{2}{0}{0}{0}{0}

 
 

\end{ExoAuto}

 
%%%%%%%%%%%%%%%%%%%%%%%%%%%%%%%%%%%%%%%%%%%%%%%%%%%%%%%%%%%%%%%%%%%
\begin{ExoAuto}{Raisonner.}{1234}{2}{0}{0}{0}{0}

 
 

\end{ExoAuto}


\end{pageAuto}



