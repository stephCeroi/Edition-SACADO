\chapter{La division euclidienne}
{https://sacado.xyz/qcm/parcours_show_course/0/117123}
{


\begin{CpsCol}
\textbf{ Les savoir-faire du parcours }
 \begin{itemize}
 \item Savoir déterminer le reste et le quotient de deux nombres.
 \item Savoir déterminer si une égalité correspond à une division euclidienne.
 \item Savoir déterminer le reste et le quotient d'après une égalité.
 \item Savoir poser une division euclidienne.
 \item Savoir vérifier une division euclidienne posée.
 \item Savoir déterminer des multiples ou des diviseurs.
 \item Savoir utiliser les critères de divisibilité.
 \end{itemize}
\end{CpsCol}

}



\begin{pageCours} 

\section{La division euclidienne}

\begin{DefT}{La division euclidienne}\index{La division euclidienne}
Effectuer la \textbf{division euclidienne} d'un nombre entier \textcolor{blue}{a} (le \textcolor{blue}{dividende}) par un nombre entier \textcolor{red}{b} (le \textcolor{red}{diviseur}) différent de 0, c'est trouver deux nombres entiers, le \textcolor{olive}{quotient, q} et le \textcolor{OliveGreen}{reste, r}, tels que : \(\textcolor{blue}{a}=\textcolor{olive}{q}\times\textcolor{red}{b}+\textcolor{OliveGreen}{r}\) avec $0\leq \textcolor{OliveGreen}{r}<\textcolor{red}{b}$.
\end{DefT}

\begin{Ex}
La division euclidienne des nombres \textcolor{blue}{$83$} et \textcolor{red}{$12$} s'écrit : $\textcolor{blue}{83}=\textcolor{olive}{6}\times\textcolor{red}{12}+\textcolor{OliveGreen}{11}$ où 
\textcolor{blue}{$83$} est le \textcolor{blue}{\textbf{dividende}},
\textcolor{red}{$12$} est le \textcolor{red}{\textbf{quotient}},
\textcolor{olive}{$6$} est le \textcolor{olive}{\textbf{diviseur}},
\textcolor{OliveGreen}{$11$} est le \textcolor{OliveGreen}{\textbf{reste}}.
\end{Ex}

\section{Poser une division euclidienne}

Lorsque le calcul mental ne permet pas de compléter facilement l'égalité d'une division euclidienne, on peut poser la division :

\begin{Ex}
Division euclidienne de $273$ par $17$. Cette pose de l'opération s'appelle une \textbf{potence}\index{potence|Division}.
\begin{equation*}
\renewcommand{\arraystretch}{1.2}
\renewcommand{\arraycolsep}{2pt}
  \begin{array}{rrrr|rrr}
 & 2 & 7 & 3 & 1 & 7 \\
\cline{5-7}
 -& 1 & 7 &  & 1 & 6\\
\cline{2-3}
    &1 & 0 & 3 &   &   &  \\
    -& 1 & 0 & 2 &   &   &  \\
    \cline{2-4}
    &&  & 1 &   &   &  \\
  \end{array}
\end{equation*}
\end{Ex} 


\begin{Mt}
Pour vérifier une division euclidienne, il suffit de calculer l'opération : ( quotient × diviseur ) + reste. On doit retrouver le dividende.

En prenant l'exemple précédent, on peut vérifier que : $17\times16+1=273$
\end{Mt}

\section{Multiples et diviseurs}

\begin{DefT}{Multiples et diviseurs}\index{Multiples}\index{diviseurs}
Soit $a$ et $b$ deux nombres entiers positifs.

Lorsque le reste dans la division euclidienne de $a$ par $b$ est égal à $0$, il existe alors un entier $q$ tel que $a=b\times q$.

On dit que $a$ divise $b$ : que $b$ est un \textbf{diviseur} de $a$ ou que $a$ est un \textbf{multiple} de $b$.
\end{DefT}

\begin{Ex}
\(42\div6=7\), autrement dit la division euclidienne de $42$ par $6$ a pour reste $0$ : \(42=6\times7+0\).\\

\begin{itemize}
    \item 42 est \textbf{divisible} par 6 et par 7.
    \item 42 est un \textbf{multiple} de 6 et 7
    \item 6 est un \textbf{diviseur} de 42.
    \item 7 est un \textbf{diviseur} de 42.
\end{itemize}
\end{Ex} 
\end{pageCours} 

\begin{pageAD} 
 

\Sf{Connaitre le vocabulaire de la division euclidienne}
 
  
\begin{ExoCad}{Raisonner. Communiquer.}{1234}{0}{0}{0}{0}{0}

$19 = 3 \times 5 + 4$. 

Le diviseur est $\ldots\ldots\ldots\ldots\ldots\ldots$, le quotient est $\ldots\ldots\ldots\ldots\ldots\ldots$ et le reste est $\ldots\ldots\ldots\ldots\ldots\ldots$.
 
\end{ExoCad}

  
\begin{ExoCad}{Raisonner. Communiquer.}{1234}{0}{0}{0}{0}{0}

$26 = 4 \times 6 + 2$. 

Le diviseur est $\ldots\ldots\ldots\ldots\ldots\ldots$, le quotient est $\ldots\ldots\ldots\ldots\ldots\ldots$ et le reste est $\ldots\ldots\ldots\ldots\ldots\ldots$.

\end{ExoCad}


\begin{ExoCad}{Raisonner.}{1234}{0}{0}{0}{0}{0}

On sait que $4\,769 = 251 \times 19$. 

Déterminer le reste de la division de $4\,772$ par $19$. \point{3}

\end{ExoCad}




\Sf{Savoir poser une division}

\begin{ExoCad}{Calculer.}{1234}{0}{0}{0}{0}{0}

Calcule 

\begin{minipage}{0.3\linewidth}
\begin{equation*}
\renewcommand{\arraystretch}{1.2}
\renewcommand{\arraycolsep}{2pt}
  \begin{array}{rrrr|rrr}
 & 4 & 2 & 6 &  & 8 \\
\cline{5-7}
  &   &   &  &  & \\

    &  &   &   &   &   &  \\
     &   &   &   &   &   &  \\
    &&  &   &   &   &  \\
  \end{array}
\end{equation*}
\end{minipage}
\begin{minipage}{0.3\linewidth}
\begin{equation*}
\renewcommand{\arraystretch}{1.2}
\renewcommand{\arraycolsep}{2pt}
  \begin{array}{rrrr|rrr}
 & 6 & 0 & 8& 2 & 3 \\
\cline{5-7}
  &   &   &  &  & \\

    &  &   &   &   &   &  \\
     &   &   &   &   &   &  \\
    &&  &   &   &   &  \\
  \end{array}
\end{equation*}
\end{minipage}
\begin{minipage}{0.3\linewidth}
\begin{equation*}
\renewcommand{\arraystretch}{1.2}
\renewcommand{\arraycolsep}{2pt}
  \begin{array}{rrrr|rrr}
 & 2  & 3 & 7 & 1 & 5 \\
\cline{5-7}
 & &  &  &  & \\
    & &  &  &   &   &  \\
    &  &  &  &   &   &  \\
    &&  &  &   &   &  \\
  \end{array}
\end{equation*}
\end{minipage}



 
\end{ExoCad}

\begin{ExoCad}{Calculer.}{1234}{0}{0}{0}{0}{0}

Paul a effectué trois calculs sur son brouillon mais il ne se rappelle plus ses calculs. Aide-le en reliant les bons calculs. \vspace{0.1cm}

\begin{minipage}{0.3\linewidth}
\begin{itemize}
\item $363$
\item $353$
\item $354$
\end{itemize}
\end{minipage}
\begin{minipage}{0.2\linewidth}

\end{minipage}
\begin{minipage}{0.3\linewidth}
\begin{itemize}
\item $29 \times 12 + 5 $
\item $12 \times 28 + 18 $
\item $14 \times 25 + 13 $
\end{itemize}
\end{minipage}
\end{ExoCad}

\Sf{Déterminer un diviseur, un multiple}

\begin{ExoCad}{Calculer.}{1234}{0}{0}{0}{0}{0}

\begin{minipage}{0.6\linewidth}
Un chocolatier range $3\,582$ chocolats dans des boites. Un boite contient 69 chocolats. Combien de boites peut-on remplir ?  \point{3}
\end{minipage}
\begin{minipage}{0.4\linewidth}
 \vspace{2cm}
\end{minipage}

\end{ExoCad}


 

 
\end{pageAD}

\begin{pageCours}
\section{Critères de divisibilité}

\subsection{Divisibilité par 2}\index{Divisibilité par 2|Critères de divisibilité}

\begin{minipage}{0.5\linewidth}
\begin{Pp}
Un nombre est divisible par \textbf{2} s'il se termine par $0$, $2$, $4$, $6$ ou $8$.
\end{Pp}
\end{minipage}
\begin{minipage}{0.5\linewidth}
\begin{Ex}
$102\color{red}{4}$ se termine par $4$ donc $1024$ est pair donc $1024$ est divisible par $2$. en effet, $1024= 512\times2$
 \end{Ex}
\end{minipage}

\begin{DefT}{Nombre pair. Nombre impair}
Un nombre entier divisible par 2 est dit \textbf{nombre pair}\index{Nombre pair}.

Un nombre entier qui n'est pas divisible par 2 est dit \textbf{nombre impair}\index{Nombre impair}.
\end{DefT}


\subsection{Divisibilité par 3}\index{Divisibilité par 3|Critères de divisibilité}
 
\begin{minipage}{0.5\linewidth}
\begin{Pp}
Un nombre est divisible par \textbf{3} lorsque la somme de ses chiffre est divisible par 3.
\end{Pp}
\end{minipage}
\begin{minipage}{0.5\linewidth}
\begin{Ex}
Les chiffres de $2067$ sont $2$, $0$, $6$ et $7$. La somme des chiffres est : $2+0+6+7=15$, or $15$ est divisible par $3$ ($3\times5$) donc $2067$ est divisible par $3$ : \(2067=3\times689\).
 \end{Ex}
\end{minipage}

\subsection{Divisibilité par 5}

\begin{minipage}{0.5\linewidth}
\begin{Pp}
Un nombre est divisible par \textbf{5} lorsqu'on son chiffre des unités est $0$ ou $5$.
\end{Pp} 
\end{minipage}
\begin{minipage}{0.5\linewidth}
\begin{Ex}
$32\color{red}{5}$ se termine par un $5$ donc $325$ est divisible par $5$. En effet, $325 = 65\times5$
\end{Ex} 
\end{minipage}

\subsection{Divisibilité par 9}

 
\begin{minipage}{0.5\linewidth}
\begin{Pp}
Un nombre est divisible par \textbf{9} lorsque la somme de ses chiffre est divisible par 9.
\end{Pp} 
\end{minipage}
\begin{minipage}{0.5\linewidth}
\begin{Ex}
Les chiffres de $594$ sont $5$, $4$ et $9$. La somme des chiffres est : $5+4+9 = 18$, or $18$ est divisible par $9$ ($9\times2$) donc $594$ est divisible par $9$. En effet, \(594=9\times66\).
\end{Ex} 
\end{minipage}

\begin{Pp}
Tout nombre divisible par $9$ est divisible par $3$.
\end{Pp}

 
\subsection{Divisibilité par 10}


\begin{minipage}{0.5\linewidth}
\begin{Pp}
Un nombre est divisible par \textbf{10} lorsque son chiffre des unités est $0$.
\end{Pp} 
\end{minipage}
\begin{minipage}{0.5\linewidth}
\begin{Ex}
$34\color{red}{0}$ se termine  par $0$ donc $340$ est divisible par$10$. En effet, $340=34\times10$
\end{Ex} 
\end{minipage}

\begin{Pp}
Tout nombre divisible par $10$ est divisible par $5$.
\end{Pp}


\end{pageCours} 




\begin{pageAD} 
 

\Sf{Connaitre le vocabulaire des opérations}
 
  
\begin{ExoCad}{Communiquer.}{1234}{2}{0}{0}{0}{0}

 
\end{ExoCad}

\Sf{Connaitre les règles de priorités}

\begin{ExoCad}{Calculer.}{1234}{0}{0}{0}{0}{0}

 
 
\end{ExoCad}

\begin{ExoCad}{Calculer.}{1235}{0}{0}{0}{0}{0}

 
\end{ExoCad}


\Sf{Utiliser la distributivité}

\begin{ExoCad}{Représenter. Calculer.}{1236}{0}{0}{0}{0}{0}

 
 
\end{ExoCad}

\begin{ExoCad}{Calculer.}{1237}{0}{0}{0}{0}{0}

 
\end{ExoCad}

 
\end{pageAD}


%%%%%%%%%%%%%%%%%%%%%%%%%%%%%%%%%%%%%%%%%%%%%%%%%%%%%%%%%%%%%%%%%%%
%%%%  Niveau 1
%%%%%%%%%%%%%%%%%%%%%%%%%%%%%%%%%%%%%%%%%%%%%%%%%%%%%%%%%%%%%%%%%%%
\begin{pageParcoursu} 

 %%%%%%%%%%%%%%%%%%%%%%%%%%%
\begin{ExoCu}{Représenter.}{1238}{2}{0}{0}{0}{0}


\end{ExoCu}
%%%%%%%%%%%%%%%%%%%%%%%%%%%
\begin{ExoCu}{Représenter.}{1239}{2}{0}{0}{0}{0}


\end{ExoCu}
%%%%%%%%%%%%%%%%%%%%%%%%%%%
\begin{ExoCu}{Représenter.}{1234}{2}{0}{0}{0}{0}

\end{ExoCu}


%%%%%%%%%%%%%%%%%%%%%%%%%%%
\begin{ExoCu}{Raisonner.}{1234}{2}{0}{0}{0}{0}

\end{ExoCu}

%%%%%%%%%%%%%%%%%%%%%%%%%%%
\begin{ExoCu}{Représenter.}{1234}{2}{0}{0}{0}{0}


\end{ExoCu}


\end{pageParcoursu}

  
%%%%%%%%%%%%%%%%%%%%%%%%%%%%%%%%%%%%%%%%%%%%%%%%%%%%%%%%%%%%%%%%%%%
%%%%  Niveau 2
%%%%%%%%%%%%%%%%%%%%%%%%%%%%%%%%%%%%%%%%%%%%%%%%%%%%%%%%%%%%%%%%%%%



\begin{pageParcoursd} 

%%%%%%%%%%%%%%%%%%%%%%%%%%%%%%%%%%%%%%%%%%%%%%%%%%%%%%%%%%%%%%%%%%%
\begin{ExoCd}{Calculer.}{1234}{0}{0}{0}{0}{0}

 Un pack contient 6 bouteilles de 1,5 L de jus d’orange. Combien de gobelets de 20 cL, pleins à 
ras bord, peut-on espérer servir ?  

 
\end{ExoCd}

\begin{ExoCd}{Calculer.}{1234}{0}{0}{0}{0}{0}

Pour remplir 4 aquariums identiques, 128 dm3 d’eau ont été nécessaires. Quelle quantité d’eau 
faudrait-il pour remplir 10 aquariums de même volume que les précédents ?
 
\end{ExoCd}


\begin{ExoCd}{Calculer.}{1234}{0}{0}{0}{0}{0}
Myriam a dépensé 85,56 € en frais d’essence ce mois-ci. Flora a dépensé trois fois moins 
qu’elle ; à combien lui reviennent ses dépenses ? (Réponse : 85,56 € : 3 = 28,52 €.)
\end{ExoCd}


%%%%%%%%%%%%%%%%%%%%%%%%%%%%%%%%%%%%%%%%%%%%%%%%%%%%%%%%%%%%%%%%%%%
\begin{ExoCd}{Représenter. Raisonner.}{1234}{2}{0}{0}{0}{0}


\end{ExoCd}

 %%%%%%%%%%%%%%%%%%%%%%%%%%%%%%%%%%%%%%%%%%%%%%%%%%%%%%%%%%%%%%%%%%%
\begin{ExoCd}{Représenter. Raisonner.}{1234}{2}{0}{0}{0}{0}


\end{ExoCd}
 
%%%%%%%%%%%%%%%%%%%%%%%%%%%%%%%%%%%%%%%%%%%%%%%%%%%%%%%%%%%%%%%%%%%
\begin{ExoCd}{Représenter. Raisonner.}{1234}{2}{0}{0}{0}{0}


\end{ExoCd}
 
\end{pageParcoursd}

%%%%%%%%%%%%%%%%%%%%%%%%%%%%%%%%%%%%%%%%%%%%%%%%%%%%%%%%%%%%%%%%%%%
%%%%  Niveau 3
%%%%%%%%%%%%%%%%%%%%%%%%%%%%%%%%%%%%%%%%%%%%%%%%%%%%%%%%%%%%%%%%%%%
\begin{pageParcourst}

%%%%%%%%%%%%%%%%%%%%%%%%%%%%%%%%%%%%%%%%%%%%%%%%%%%%%%%%%%%%%%%%%%%
\begin{ExoCt}{Représenter.}{1234}{2}{0}{0}{0}{0}

 

\end{ExoCt}

%%%%%%%%%%%%%%%%%%%%%%%%%%%%%%%%%%%%%%%%%%%%%%%%%%%%%%%%%%%%%%%%%%%
\begin{ExoCt}{Représenter. Raisonner.}{1234}{2}{0}{0}{0}{0}
 
 


\end{ExoCt}


%%%%%%%%%%%%%%%%%%%%%%%%%%%%%%%%%%%%%%%%%%%%%%%%%%%%%%%%%%%%%%%%%%%
\begin{ExoCt}{Raisonner.}{1234}{2}{0}{0}{0}{0}
 
\end{ExoCt}

%%%%%%%%%%%%%%%%%%%%%%%%%%%%%%%%%%%%%%%%%%%%%%%%%%%%%%%%%%%%%%%%%%%
\begin{ExoCt}{Représenter.}{1234}{2}{0}{0}{0}{0}

 

\end{ExoCt}

%%%%%%%%%%%%%%%%%%%%%%%%%%%%%%%%%%%%%%%%%%%%%%%%%%%%%%%%%%%%%%%%%%%
\begin{ExoCt}{Représenter.}{1234}{2}{0}{0}{0}{0}

 

\end{ExoCt} 
 
\end{pageParcourst}

%%%%%%%%%%%%%%%%%%%%%%%%%%%%%%%%%%%%%%%%%%%%%%%%%%%%%%%%%%%%%%%%%%%
%%%%  Brouillon
%%%%%%%%%%%%%%%%%%%%%%%%%%%%%%%%%%%%%%%%%%%%%%%%%%%%%%%%%%%%%%%%%%%


\begin{pageBrouillon} 
 
\ligne{32}



\end{pageBrouillon}

%%%%%%%%%%%%%%%%%%%%%%%%%%%%%%%%%%%%%%%%%%%%%%%%%%%%%%%%%%%%%%%%%%%
%%%%  Auto
%%%%%%%%%%%%%%%%%%%%%%%%%%%%%%%%%%%%%%%%%%%%%%%%%%%%%%%%%%%%%%%%%%%


%%%%%%%%%%%%%%%%%%%%%%%%%%%%%%%%%%%%%%%%%%%%%%%%%%%%%%%%%%%%%%%%%%%
\begin{pageAuto} 


\begin{ExoAuto}{Raisonner.}{1234}{2}{0}{0}{0}{0}

 
%%%%%%%%%%%%%%%%%%%%%%%%%%%%%%%%%%%%%%%%%%%%%%%%%%%%%%%%%%%%%%%%%%%
\end{ExoAuto}

\begin{ExoAuto}{Raisonner.}{1234}{2}{0}{0}{0}{0}
  

\end{ExoAuto}

%%%%%%%%%%%%%%%%%%%%%%%%%%%%%%%%%%%%%%%%%%%%%%%%%%%%%%%%%%%%%%%%%%%
\begin{ExoAuto}{Raisonner.}{1234}{2}{0}{0}{0}{0}

 
 

\end{ExoAuto}

 
%%%%%%%%%%%%%%%%%%%%%%%%%%%%%%%%%%%%%%%%%%%%%%%%%%%%%%%%%%%%%%%%%%%
\begin{ExoAuto}{Raisonner.}{1234}{2}{0}{0}{0}{0}

 
 

\end{ExoAuto}


\end{pageAuto}
