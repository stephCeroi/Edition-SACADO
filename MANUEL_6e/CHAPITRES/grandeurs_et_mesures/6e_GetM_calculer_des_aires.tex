%-------------------------------
%	CONTENTS
%-------------------------------

\chapter{Calculer des aires}
{https://sacado.xyz/qcm/parcours_show_course/0/117135}
{



\begin{CpsCol}
\textbf{ Les savoir-faire du parcours}
\begin{itemize}
\item Savoir exprimer l'aire d'une figure en fonction d'une unité d'aire.
\item Savoir convertir des unités d'aire.
\item Savoir calculer l'aire d'un carré, d'un rectangle, d'un triangle, d'un disque.
\item Savoir résoudre un problème d'aire.
\end{itemize}
\end{CpsCol}
 

}

\begin{pageCours} 

\section{Aire d'une figure}

\begin{DefT}
L'\textbf{aire}\index{Aire} d'une figure est la \textbf{mesure} de sa \textbf{surface intérieure} dans une \textbf{unité d'aire} donnée.
\end{DefT}

\begin{Ex}
L'aire des \textcolor{zzttqq}{figures oranges} peut être exprimée comme somme de \textcolor{qqwuqq}{carrés verts}, entiers ou non, valant $1\,u$ (une unité).
\begin{center}
\begin{tikzpicture}[line cap=round,line join=round,>=triangle 45,x=.7cm,y=.7cm]
\draw [color=cqcqcq,, xstep=.7cm,ystep=.7cm] (-0.18164835357409106,-0.17948573521827438) grid (9.210839706161456,2.2136687567143394);
\clip(-0.18164835357409106,-0.17948573521827438) rectangle (9.210839706161456,2.2136687567143394);
\fill[line width=1.pt,color=qqwuqq,fill=qqwuqq,fill opacity=0.10000000149011612] (0.,0.) -- (1.,0.) -- (1.,1.) -- (0.,1.) -- cycle;
\fill[line width=1.pt,color=zzttqq,fill=zzttqq,fill opacity=0.10000000149011612] (2.,1.) -- (2.,0.) -- (4.,0.) -- (4.,1.) -- cycle;
\fill[line width=1.pt,color=zzttqq,fill=zzttqq,fill opacity=0.10000000149011612] (5.,0.) -- (5.,2.) -- (8.,2.) -- (9.,1.) -- (9.,0.) -- cycle;
\draw [line width=1.pt,color=qqwuqq] (0.,0.)-- (1.,0.);
\draw [line width=1.pt,color=qqwuqq] (1.,0.)-- (1.,1.);
\draw [line width=1.pt,color=qqwuqq] (1.,1.)-- (0.,1.);
\draw [line width=1.pt,color=qqwuqq] (0.,1.)-- (0.,0.);
\draw [line width=1.pt,color=zzttqq] (2.,1.)-- (2.,0.);
\draw [line width=1.pt,color=zzttqq] (2.,0.)-- (4.,0.);
\draw [line width=1.pt,color=zzttqq] (4.,0.)-- (4.,1.);
\draw [line width=1.pt,color=zzttqq] (4.,1.)-- (2.,1.);
\draw [line width=1.pt,color=zzttqq] (5.,0.)-- (5.,2.);
\draw [line width=1.pt,color=zzttqq] (5.,2.)-- (8.,2.);
\draw [line width=1.pt,color=zzttqq] (8.,2.)-- (9.,1.);
\draw [line width=1.pt,color=zzttqq] (9.,1.)-- (9.,0.);
\draw [line width=1.pt,color=zzttqq] (9.,0.)-- (5.,0.);
%\begin{scriptsize}
\draw[color=qqwuqq] (0.5,0.5) node {$1\,u$};
\draw[color=zzttqq] (3,0.5) node {$2\,u$};
\draw[color=zzttqq] (7,1) node {$7,5\,u$};
%\end{scriptsize}
\end{tikzpicture}
\end{center}
\end{Ex}


\begin{Rqs}
\begin{itemize}
\item Deux figures ayant la même aire n'ont pas nécessairement le même périmètre.
\item Deux figures ayant le même périmètre n'ont pas nécessairement la même aire.
\end{itemize}
\end{Rqs}

\begin{Ex}
\[\textcolor{qqwuqq}{\mathcal{P}_1=20\,u.l.}\hspace{.5cm}\textcolor{qqwuqq}{\mathcal{A}_1=14\,u.a.}\hspace{1cm}\textcolor{qqqqff}{\mathcal{P}_2=20\,u.l.}\hspace{.5cm}\textcolor{qqqqff}{\mathcal{A}_2=14\,u.a.}\]
\begin{center}
\begin{tikzpicture}[line cap=round,line join=round,>=triangle 45,x=.7cm,y=.7cm]
\draw [color=cqcqcq,, xstep=.7cm,ystep=.7cm] (-3.02,-4.96) grid (11.,4.06);
\clip(-3.02,-4.96) rectangle (11.,4.06);
\fill[line width=1.pt,color=qqwuqq,fill=qqwuqq,fill opacity=0.10000000149011612] (1.,-1.) -- (-2.,-1.) -- (-2.,0.) -- (-1.,0.) -- (-1.,1.) -- (-2.,1.) -- (-2.,2.) -- (-1.,2.) -- (-1.,3.) -- (1.,3.) -- (1.,2.) -- (3.,2.) -- (3.,0.) -- (1.,0.) -- cycle;
\fill[line width=1.pt,color=qqqqff,fill=qqqqff,fill opacity=0.10000000149011612] (5.,-1.) -- (5.,0.) -- (7.,0.) -- (7.,2.) -- (9.,2.) -- (9.,0.) -- (10.,0.) -- (10.,-2.) -- (9.,-2.) -- (9.,-3.) -- (7.,-3.) -- (7.,-1.) -- cycle;
\fill[line width=1.pt,color=ffqqff,fill=ffqqff,fill opacity=1.0] (-1.,-3.) -- (0.,-3.) -- (0.,-4.) -- (-1.,-4.) -- cycle;
\draw [line width=1.pt,color=qqwuqq] (1.,-1.)-- (-2.,-1.);
\draw [line width=1.pt,color=qqwuqq] (-2.,-1.)-- (-2.,0.);
\draw [line width=1.pt,color=qqwuqq] (-2.,0.)-- (-1.,0.);
\draw [line width=1.pt,color=qqwuqq] (-1.,0.)-- (-1.,1.);
\draw [line width=1.pt,color=qqwuqq] (-1.,1.)-- (-2.,1.);
\draw [line width=1.pt,color=qqwuqq] (-2.,1.)-- (-2.,2.);
\draw [line width=1.pt,color=qqwuqq] (-2.,2.)-- (-1.,2.);
\draw [line width=1.pt,color=qqwuqq] (-1.,2.)-- (-1.,3.);
\draw [line width=1.pt,color=qqwuqq] (-1.,3.)-- (1.,3.);
\draw [line width=1.pt,color=qqwuqq] (1.,3.)-- (1.,2.);
\draw [line width=1.pt,color=qqwuqq] (1.,2.)-- (3.,2.);
\draw [line width=1.pt,color=qqwuqq] (3.,2.)-- (3.,0.);
\draw [line width=1.pt,color=qqwuqq] (3.,0.)-- (1.,0.);
\draw [line width=1.pt,color=qqwuqq] (1.,0.)-- (1.,-1.);
\draw [line width=1.pt,color=qqqqff] (5.,-1.)-- (5.,0.);
\draw [line width=1.pt,color=qqqqff] (5.,0.)-- (7.,0.);
\draw [line width=1.pt,color=qqqqff] (7.,0.)-- (7.,2.);
\draw [line width=1.pt,color=qqqqff] (7.,2.)-- (9.,2.);
\draw [line width=1.pt,color=qqqqff] (9.,2.)-- (9.,0.);
\draw [line width=1.pt,color=qqqqff] (9.,0.)-- (10.,0.);
\draw [line width=1.pt,color=qqqqff] (10.,0.)-- (10.,-2.);
\draw [line width=1.pt,color=qqqqff] (10.,-2.)-- (9.,-2.);
\draw [line width=1.pt,color=qqqqff] (9.,-2.)-- (9.,-3.);
\draw [line width=1.pt,color=qqqqff] (9.,-3.)-- (7.,-3.);
\draw [line width=1.pt,color=qqqqff] (7.,-3.)-- (7.,-1.);
\draw [line width=1.pt,color=qqqqff] (7.,-1.)-- (5.,-1.);
\draw [line width=3.6pt,color=ffqqff] (-1.,-2.)-- (0.,-2.);
\draw [line width=1.pt,color=ffqqff] (-1.,-3.)-- (0.,-3.);
\draw [line width=1.pt,color=ffqqff] (0.,-3.)-- (0.,-4.);
\draw [line width=1.pt,color=ffqqff] (0.,-4.)-- (-1.,-4.);
\draw [line width=1.pt,color=ffqqff] (-1.,-4.)-- (-1.,-3.);
\draw [color=ffqqff](-0.86,-2.32) node[anchor=north west] {1 u.l.};
\draw [color=ffqqff](-0.94,-4.28) node[anchor=north west] {1 u.a.};
\begin{scriptsize}
\draw[color=qqwuqq] (0.14,1.19) node {$Fig.1$};
\draw[color=qqqqff] (8.12,-0.49) node {$Fig.2$};
\end{scriptsize}
\end{tikzpicture}
\end{center}
\end{Ex}



\end{pageCours} 


\begin{pageAD} 

\Sf{Exprimer l'aire d'une figure en fonction d'une unité d'aire}

\ExoCad{Chercher. Représenter.}

\definecolor{xfqqff}{rgb}{0.4980392156862745,0.,1.}
\definecolor{qqzzff}{rgb}{0.,0.6,1.}
\definecolor{qqccqq}{rgb}{0.,0.8,0.}
\definecolor{zzttqq}{rgb}{0.6,0.2,0.}
\definecolor{cqcqcq}{rgb}{0.7529411764705882,0.7529411764705882,0.7529411764705882}
\begin{tikzpicture}[line cap=round,line join=round,>=triangle 45,x=1.0cm,y=1.0cm]
\draw [color=cqcqcq,, xstep=0.5cm,ystep=0.5cm] (-3.52,0.46) grid (7.06,4.52);
\clip(-3.52,0.46) rectangle (7.06,4.52);
\fill[line width=1.pt,color=zzttqq,fill=zzttqq,fill opacity=0.10000000149011612] (-3.,4.) -- (0.,4.) -- (0.,3.) -- (2.,3.) -- (2.,1.) -- (0.,1.) -- (0.,2.) -- (-2.,2.) -- (-2.,3.) -- (-3.,3.) -- cycle;
\fill[line width=1.pt,color=qqccqq,fill=qqccqq,fill opacity=0.5] (-3.,1.5) -- (-2.5,1.5) -- (-2.5,1.) -- (-3.,1.) -- cycle;
\fill[line width=1.pt,color=qqzzff,fill=qqzzff,fill opacity=0.10000000149011612] (3.5,4.) -- (6.5,4.) -- (6.5,1.) -- (3.5,1.) -- cycle;
\draw [line width=1.pt,color=zzttqq] (-3.,4.)-- (0.,4.);
\draw [line width=1.pt,color=zzttqq] (0.,4.)-- (0.,3.);
\draw [line width=1.pt,color=zzttqq] (0.,3.)-- (2.,3.);
\draw [line width=1.pt,color=zzttqq] (2.,3.)-- (2.,1.);
\draw [line width=1.pt,color=zzttqq] (2.,1.)-- (0.,1.);
\draw [line width=1.pt,color=zzttqq] (0.,1.)-- (0.,2.);
\draw [line width=1.pt,color=zzttqq] (0.,2.)-- (-2.,2.);
\draw [line width=1.pt,color=zzttqq] (-2.,2.)-- (-2.,3.);
\draw [line width=1.pt,color=zzttqq] (-2.,3.)-- (-3.,3.);
\draw [line width=1.pt,color=zzttqq] (-3.,3.)-- (-3.,4.);
\draw [line width=1.pt,color=qqccqq] (-3.,1.5)-- (-2.5,1.5);
\draw [line width=1.pt,color=qqccqq] (-2.5,1.5)-- (-2.5,1.);
\draw [line width=1.pt,color=qqccqq] (-2.5,1.)-- (-3.,1.);
\draw [line width=1.pt,color=qqccqq] (-3.,1.)-- (-3.,1.5);
\draw [color=qqccqq](-3.24,1.16) node[anchor=north west] {$1u.a.$};
\draw [line width=1.pt,color=qqzzff] (3.5,4.)-- (6.5,4.);
\draw [line width=1.pt,color=qqzzff] (6.5,4.)-- (6.5,1.);
\draw [line width=1.pt,color=qqzzff] (6.5,1.)-- (3.5,1.);
\draw [line width=1.pt,color=qqzzff] (3.5,1.)-- (3.5,4.);
\draw [line width=1.pt,color=xfqqff] (1.5,4.)-- (2.,4.);
\draw [color=xfqqff](1.34,4.16) node[anchor=north west] {$1u.l.$};
\draw (-1.46,3.32) node[anchor=north west] {$Fig. 1$};
\draw (4.42,3.34) node[anchor=north west] {$Fig. 2$};
\end{tikzpicture}


\begin{enumerate}
\item Déterminer le périmètre de la $Fig. \,1$ : \point{1}
\item Déterminer l'aire de la $Fig. \,1$ : \point{1}
\item Déterminer le périmètre de la $Fig. \,2$ : \point{1}
\item Déterminer l'aire de la $Fig. \,2$ : \point{1}
\end{enumerate}

\ExoCad{Chercher. Représenter.}


Déterminer l'aire des figures suivantes en fonction de l'unité proposée :
\begin{center}
\begin{tikzpicture}[line cap=round,line join=round,>=triangle 45,x=1.0cm,y=1.0cm]
\draw [color=cqcqcq,, xstep=1.0cm,ystep=1.0cm] (-0.2202476195730043,-0.19235215721791207) grid (12.234448876076325,4.246563432657098);
\clip(-0.2202476195730043,-0.19235215721791207) rectangle (12.234448876076325,4.246563432657098);
\fill[line width=1.pt,color=qqwuqq,fill=qqwuqq,fill opacity=0.10000000149011612] (0.,0.) -- (1.,0.) -- (1.,1.) -- (0.,1.) -- cycle;
\fill[line width=1.pt,color=qqwuqq,fill=qqwuqq,fill opacity=0.10000000149011612] (2.,0.) -- (2.,4.) -- (9.,4.) -- cycle;
\fill[line width=1.pt,color=qqwuqq,fill=qqwuqq,fill opacity=0.10000000149011612] (6.012866421999638,0.003580915971741372) -- (8.012866421999638,2.0035809159717415) -- (9.012866421999638,2.0035809159717415) -- (7.012866421999638,0.003580915971741372) -- cycle;
\fill[line width=1.pt,color=qqwuqq,fill=qqwuqq,fill opacity=0.10000000149011612] (10.,3.) -- (11.,4.) -- (12.,3.) -- (12.,1.) -- (11.,0.) -- (10.,1.) -- cycle;
\draw [line width=1.pt,color=qqwuqq] (0.,0.)-- (1.,0.);
\draw [line width=1.pt,color=qqwuqq] (1.,0.)-- (1.,1.);
\draw [line width=1.pt,color=qqwuqq] (1.,1.)-- (0.,1.);
\draw [line width=1.pt,color=qqwuqq] (0.,1.)-- (0.,0.);
\draw [line width=1.pt,color=qqwuqq] (2.,0.)-- (2.,4.);
\draw [line width=1.pt,color=qqwuqq] (2.,4.)-- (9.,4.);
\draw [line width=1.pt,color=qqwuqq] (9.,4.)-- (2.,0.);
\draw [line width=1.pt,color=qqwuqq] (6.012866421999638,0.003580915971741372)-- (8.012866421999638,2.0035809159717415);
\draw [line width=1.pt,color=qqwuqq] (8.012866421999638,2.0035809159717415)-- (9.012866421999638,2.0035809159717415);
\draw [line width=1.pt,color=qqwuqq] (9.012866421999638,2.0035809159717415)-- (7.012866421999638,0.003580915971741372);
\draw [line width=1.pt,color=qqwuqq] (7.012866421999638,0.003580915971741372)-- (6.012866421999638,0.003580915971741372);
\draw [line width=1.pt,color=qqwuqq] (10.,3.)-- (11.,4.);
\draw [line width=1.pt,color=qqwuqq] (11.,4.)-- (12.,3.);
\draw [line width=1.pt,color=qqwuqq] (12.,3.)-- (12.,1.);
\draw [line width=1.pt,color=qqwuqq] (12.,1.)-- (11.,0.);
\draw [line width=1.pt,color=qqwuqq] (11.,0.)-- (10.,1.);
\draw [line width=1.pt,color=qqwuqq] (10.,1.)-- (10.,3.);
%\begin{scriptsize}
\draw[color=qqwuqq] (0.5,0.5) node {$1 u$};
\draw[color=qqwuqq] (4,2.5) node {$Fig.\,1$};
\draw[color=qqwuqq] (7.5,1) node {$Fig.\,2$};
\draw[color=qqwuqq] (11,2) node {$Fig.\,3$};
%\end{scriptsize}
\end{tikzpicture}
\end{center}

\begin{enumerate}
\item Déterminer le périmètre de $Fig. \,1$ : \point{1}
\item Déterminer l'aire de $Fig. \,1$ : \point{1}
\item Déterminer le périmètre de $Fig. \,2$ : \point{1}
\item Déterminer l'aire de $Fig. \,2$ : \point{1}
\item Déterminer le périmètre de $Fig. \,3$ : \point{1}
\item Déterminer l'aire de $Fig. \,3$ : \point{1}
\end{enumerate}
 
 
 
\end{pageAD} 


\begin{pageCours} 

\section{Unités d'aires usuelles}

\begin{DefT}{mètre carré}

\begin{minipage}{0.78\linewidth}

L'unité de mesure des aires est le \textbf{mètre carré}\index{mètre carré}, on le note $m^2$, c'est l'aire d'un carré de $1\,m$ de côté.

\end{minipage}
\hfill
\begin{minipage}{0.18\linewidth}

\begin{center}
\begin{tikzpicture}[line cap=round,line join=round,>=triangle 45,x=2.0cm,y=2.0cm]
\clip(-0.19063259065814545,-0.2571668280229305) rectangle (1.1623070595760396,1.1049452944162321);
\fill[line width=1.pt,color=zzttqq,fill=zzttqq,fill opacity=0.10000000149011612] (0.,0.) -- (1.,0.) -- (1.,1.) -- (0.,1.) -- cycle;
\draw [line width=1.pt,color=zzttqq] (0.,0.)-- (1.,0.);
\draw [line width=1.pt,color=zzttqq] (0.5,0.010319031230599716) -- (0.5,-0.010319031230599716);
\draw [line width=1.pt,color=zzttqq] (1.,0.)-- (1.,1.);
\draw [line width=1.pt,color=zzttqq] (0.9896809687694005,0.5) -- (1.0103190312305999,0.5);
\draw [line width=1.pt,color=zzttqq] (1.,1.)-- (0.,1.);
\draw [line width=1.pt,color=zzttqq] (0.5,0.9896809687694005) -- (0.5,1.0103190312305999);
\draw [line width=1.pt,color=zzttqq] (0.,1.)-- (0.,0.);
\draw [line width=1.pt,color=zzttqq] (0.010319031230599716,0.5) -- (-0.010319031230599716,0.5);
\begin{scriptsize}
\draw [fill=xdxdff] (0.,0.) circle (2.0pt);
\draw [fill=xdxdff] (1.,0.) circle (2.0pt);
\draw [fill=xdxdff] (1.,1.) circle (2.0pt);
\draw [fill=xdxdff] (0.,1.) circle (2.0pt);
\draw[color=zzttqq] (0.5,0.5) node {$1\,m^2$};
\draw[color=zzttqq] (0.5,-0.1) node {$1\,m$};
\end{scriptsize}
\end{tikzpicture}
\end{center}

\end{minipage}
\end{DefT}


\begin{minipage}{0.49\linewidth}
\begin{Rq}
De la même manière nous pouvons définir :
\begin{itemize}[leftmargin=*]
\item $1\,dm^2$ est l'aire d'un carré de $1\,dm$ de côté.
\item $1\,cm^2$ est l'aire d'un carré de $1\,cm$ de côté.
\item $1\,mm^2$ est l'aire d'un carré de $1\,mm$ de côté.
\end{itemize}
\end{Rq}
 
\end{minipage}
\hfill
\begin{minipage}{0.49\linewidth}



\begin{VocU}
Pour mesurer la superficie des terrains, on utilise l'\textbf{are}\index{are} ($a$) et l'\textbf{hectare}\index{hectare} ($ha$) :
\begin{itemize}[leftmargin=*]
\item $1 a$ est l'aire d'un coté de longueur $10\,m$. $1\,a=10\,m \times 10\,m = 100\,m^2$. 
\item $1 ha$ est l'aire d'un coté de longueur $100\,m$. $1\,ha=100\,m \times 100\,m =10\,000\,m^2$. 
\end{itemize}
\end{VocU}
\end{minipage} 

\section{Convertir des unités d'aire}

\begin{Pp}
Dans un carré de $1 \,cm$ de côté, on peut construire $10\times10=100$ carrés de $1\, mm$ de côté.
Donc $1 \,cm^2 = 100 mm^2$
\end{Pp}

\begin{Mt}
Ce tableau de conversion\index{Tableau de conversion u.a.} aide à convertir des unités d'aire.

\begin{center}
\begin{tabular}{|M{.5cm}|M{.5cm}||M{.5cm}|M{.5cm}||M{.5cm}|M{.5cm}||M{.5cm}|M{.5cm}||M{.5cm}|M{.5cm}||M{.5cm}|M{.5cm}||M{.5cm}|M{.5cm}|}
\multicolumn{2}{c}{$km^2$} & \multicolumn{2}{c}{$hm^2$} & \multicolumn{2}{c}{$dam^2$} & \multicolumn{2}{c}{$m^2$} & \multicolumn{2}{c}{$dm^2$} & \multicolumn{2}{c}{$cm^2$} & \multicolumn{2}{c}{$mm^2$} \\\hline
 & & & & & & 2 & 5 & & & & & &  \\\hline
  & & & & & & 2 & 5 & 0 & 0 & & & & \\\hline
\end{tabular}
\end{center}

Ainsi : $25\,m^2=2\,500\,dm^2$.
\end{Mt}
 
\end{pageCours} 
 
%%%%%%%%%%%%%%%%%%%%%%%%%%%%%%%%%%%%%%%%%%%%%%%%%%%%%%%%%%%%%%%%%%%%%%%%%%%%%%%%%%%%%%%%%%%%%%%%%%%%%%%%%%%%%%%%
%%%%%%%%%%%%% Application directe
%%%%%%%%%%%%%%%%%%%%%%%%%%%%%%%%%%%%%%%%%%%%%%%%%%%%%%%%%%%%%%%%%%%%%%%%%%%%%%%%%%%%%%%%%%%%%%%%%%%%%%%%%%%%%%%%
\begin{pageAD} 

\Sf{Représenter des unités d'aire}

\ExoCad{Représenter.}


En t'appuyant sur la figure, complète les phrases ci dessous:

\begin{enumerate}
\item $1 \,dm = \dots \ldots \ldots\,cm$.
\item Dans un carré de $1 \,dm$ de coté, je compte $\dots\ldots \ldots$ carrés de coté $1\,cm$.
\item $1 \,dm^2 = \dots \ldots\ldots \,cm^2$
\end{enumerate}



\begin{center}
\begin{tikzpicture}[line cap=round,line join=round,>=triangle 45,x=1.0cm,y=1.0cm]
\clip(-0.7008474435890936,-1.001194059753229) rectangle (12.772217041258656,10.391470761992991);
\draw [line width=1.pt,color=zzttqq] (0.,0.)-- (1.,0.);
%\draw [line width=1.pt,color=zzttqq] (0.5,0.08915998556149216) -- (0.5,-0.08915998556149216);
\draw [line width=1.pt,color=zzttqq] (1.,0.)-- (1.,1.);
%\draw [line width=1.pt,color=zzttqq] (0.910840014438507,0.5) -- (1.089159985561492,0.5);
\draw [line width=1.pt,color=zzttqq] (1.,1.)-- (0.,1.);
%\draw [line width=1.pt,color=zzttqq] (0.5,0.9108400144385084) -- (0.5,1.0891599855614926);
\draw [line width=1.pt,color=zzttqq] (0.,1.)-- (0.,0.);
%\draw [line width=1.pt,color=zzttqq] (0.08915998556149246,0.5) -- (-0.08915998556149246,0.5);
\draw [line width=.5pt,color=zzttqq] (3.,10.)-- (3.,0.);
\draw [line width=.5pt,color=zzttqq] (4.,10.)-- (4.,0.);
\draw [line width=.5pt,color=zzttqq] (5.,10.)-- (5.,0.);
\draw [line width=.5pt,color=zzttqq] (6.,10.)-- (6.,0.);
\draw [line width=.5pt,color=zzttqq] (7.,10.)-- (7.,0.);
\draw [line width=.5pt,color=zzttqq] (8.,10.)-- (8.,0.);
\draw [line width=.5pt,color=zzttqq] (9.,10.)-- (9.,0.);
\draw [line width=.5pt,color=zzttqq] (10.,10.)-- (10.,0.);
\draw [line width=.5pt,color=zzttqq] (11.,10.)-- (11.,0.);
\draw [line width=.5pt,color=zzttqq] (2.,1.)-- (12.,1.);
\draw [line width=.5pt,color=zzttqq] (2.,2.)-- (12.,2.);
\draw [line width=.5pt,color=zzttqq] (2.,3.)-- (12.,3.);
\draw [line width=.5pt,color=zzttqq] (2.,4.)-- (12.,4.);
\draw [line width=.5pt,color=zzttqq] (2.,5.)-- (12.,5.);
\draw [line width=.5pt,color=zzttqq] (2.,6.)-- (12.,6.);
\draw [line width=.5pt,color=zzttqq] (2.,7.)-- (12.,7.);
\draw [line width=.5pt,color=zzttqq] (2.,8.)-- (12.,8.);
\draw [line width=.5pt,color=zzttqq] (2.,9.)-- (12.,9.);
\draw [line width=1.pt,color=zzttqq] (2.,0.)-- (3.,0.);
\draw [line width=1.pt,color=zzttqq] (3.,0.)-- (3.,1.);
\draw [line width=1.pt,color=zzttqq] (3.,1.)-- (2.,1.);
\draw [line width=1.pt,color=zzttqq] (2.,1.)-- (2.,0.);
\draw [line width=1.pt,color=zzttqq] (2.,10.)-- (2.,0.);
\draw [line width=1.pt,color=zzttqq] (2.,0.)-- (12.,0.);
\draw [line width=1.pt,color=zzttqq] (12.,0.)-- (12.,10.);
\draw [line width=1.pt,color=zzttqq] (12.,10.)-- (2.,10.);
%\begin{scriptsize}
\draw[color=zzttqq] (0.5,0.5) node {$1\,cm^2$};
\draw[color=zzttqq] (0.5,-0.4) node {$1\,cm$};
\draw[color=zzttqq] (7,5) node {$1\,dm^2$};
\draw[color=zzttqq] (7,-0.4) node {$1\,dm$};
%\end{scriptsize}
\end{tikzpicture}
\end{center}
 
 \Sf{Convertir des unités d'aire}
 
\ExoCad{Représenter.}
 
Effectuer les conversions suivantes :
\[28\,m^2=\dots \ldots \ldots\,cm^2\hspace{.5cm}4,32\,dm^2=\dots \ldots \ldots\,m^2\hspace{.5cm}1\,cm^2=\dots \ldots \ldots\,mm^2\]

\[3,3\,dm^2=\dots \ldots \ldots\,mm^2\hspace{.5cm}2,1\,dm^2=\dots \ldots \ldots\,dam^2\hspace{.5cm}25\,dm^2=\dots \ldots \ldots\,m^2\]
 
\[1,5\,km^2=\dots \ldots \ldots\,m^2\hspace{.5cm}3,4\,ha =\dots \ldots \ldots\,m^2\hspace{.5cm}6,8\,a=\dots \ldots \ldots\,m^2\]

\end{pageAD} 



\begin{pageCours}


\begin{DefT}{Disque}

Un \textbf{disque}\index{Disque} est l'ensemble des points intérieur au cercle. Le disque est une surface.

\begin{center}
\begin{tikzpicture}[line cap=round,line join=round,>=triangle 45,x=1.0cm,y=1.0cm]
\clip(-3.2258782096036835,-1.3387575571899715) rectangle (3.220423513612537,1.6953257957763235);
\draw [line width=1.pt,color=sacado_blue,fill=sacado_blue,fill opacity=0.10000000149011612] (2.,0.) circle (1.cm);
\draw [line width=1.pt] (-2.,0.) circle (1.cm);
\begin{scriptsize}
\draw [fill=black] (-2.,0.) circle (0.5pt);
\draw[color=black] (-2,1.3) node {$cercle$};
\draw [fill=sacado_blue] (2.,0.) circle (0.5pt);
\draw[color=sacado_blue] (2,1.3) node {$disque$};
\end{scriptsize}
\end{tikzpicture}
\end{center}

\end{DefT}

\section{Calculs d'aire de surfaces particulières}

%\subsection{Aire d'un carré}
%
%\begin{Pp}
%L'aire d'un carré est donnée par la formule :
%\[\mathcal{A}_{carré}=côté\times côté\]
%\[\mathcal{A}_{carré}=c\times c\]
%\begin{center}
%\begin{tikzpicture}[line cap=round,line join=round,>=triangle 45,x=0.5cm,y=0.5cm]
%\clip(-0.5258160132035038,-0.9097206258865002) rectangle (6.482416380185364,6.303964834409533);
%\fill[line width=1.pt,color=zzttqq,fill=zzttqq,fill opacity=0.10000000149011612] (0.,6.) -- (6.,6.) -- (6.,0.) -- (0.,0.) -- cycle;
%\draw[line width=1.pt,color=zzttqq] (0.,5.515742477247899) -- (0.48425752275210066,5.515742477247899) -- (0.4842575227521006,6.) -- (0.,6.) -- cycle; 
%\draw[line width=1.pt,color=zzttqq] (5.515742477247899,6.) -- (5.515742477247899,5.515742477247899) -- (6.,5.515742477247899) -- (6.,6.) -- cycle; 
%\draw[line width=1.pt,color=zzttqq] (6.,0.4842575227521006) -- (5.515742477247899,0.48425752275210066) -- (5.515742477247899,0.) -- (6.,0.) -- cycle; 
%\draw[line width=1.pt,color=zzttqq] (0.4842575227521006,0.) -- (0.48425752275210066,0.4842575227521006) -- (0.,0.4842575227521006) -- (0.,0.) -- cycle; 
%\draw [line width=1.2pt,color=zzttqq] (0.,6.)-- (6.,6.);
%\draw [line width=1.2pt,color=zzttqq] (2.954343762909519,6.114140592726204) -- (2.954343762909519,5.885859407273798);
%\draw [line width=1.2pt,color=zzttqq] (3.045656237090481,6.114140592726204) -- (3.045656237090481,5.885859407273798);
%\draw [line width=1.2pt,color=zzttqq] (6.,6.)-- (6.,0.);
%\draw [line width=1.2pt,color=zzttqq] (6.114140592726202,3.0456562370904816) -- (5.885859407273796,3.0456562370904816);
%\draw [line width=1.2pt,color=zzttqq] (6.114140592726202,2.9543437629095193) -- (5.885859407273796,2.9543437629095193);
%\draw [line width=1.2pt,color=zzttqq] (6.,0.)-- (0.,0.);
%\draw [line width=1.2pt,color=zzttqq] (3.045656237090481,-0.11414059272620306) -- (3.045656237090481,0.11414059272620306);
%\draw [line width=1.2pt,color=zzttqq] (2.954343762909519,-0.11414059272620306) -- (2.954343762909519,0.11414059272620306);
%\draw [line width=1.2pt,color=zzttqq] (0.,0.)-- (0.,6.);
%\draw [line width=1.2pt,color=zzttqq] (-0.11414059272620307,2.9543437629095193) -- (0.11414059272620307,2.9543437629095193);
%\draw [line width=1.2pt,color=zzttqq] (-0.11414059272620307,3.0456562370904816) -- (0.11414059272620307,3.0456562370904816);
%%\begin{scriptsize}
%\draw[color=zzttqq] (3,-0.7) node {$c$};
%%\end{scriptsize}
%\end{tikzpicture}
%\end{center}
%\end{Pp}
%

%
%\subsection{Aire d'un rectangle}
%
%\begin{Pp}
%L'aire d'un rectangle est donnée par la formule :
%\[\mathcal{A}_{rectangle}=largeur\times Longueur\]
%\[\mathcal{A}_{rectangle}=l\times L\]
%\begin{center}
%\begin{tikzpicture}[line cap=round,line join=round,>=triangle 45,x=0.3cm,y=0.3cm]
%\clip(-1.3248001622869252,-1.2521424040651092) rectangle (11.55025869722878,6.92032403513103);
%\fill[line width=1.pt,color=zzttqq,fill=zzttqq,fill opacity=0.10000000149011612] (0.,6.) -- (10.,6.) -- (10.,0.) -- (0.,0.) -- cycle;
%\draw[line width=1.pt,color=zzttqq] (0.,5.515742477247899) -- (0.48425752275210066,5.515742477247899) -- (0.4842575227521006,6.) -- (0.,6.) -- cycle; 
%\draw[line width=1.pt,color=zzttqq] (9.5157424772479,6.) -- (9.5157424772479,5.515742477247899) -- (10.,5.515742477247899) -- (10.,6.) -- cycle; 
%\draw[line width=1.pt,color=zzttqq] (10.,0.4842575227521006) -- (9.5157424772479,0.48425752275210066) -- (9.5157424772479,0.) -- (10.,0.) -- cycle; 
%\draw[line width=1.pt,color=zzttqq] (0.4842575227521006,0.) -- (0.48425752275210066,0.4842575227521006) -- (0.,0.4842575227521006) -- (0.,0.) -- cycle; 
%\draw [line width=1.pt,color=zzttqq] (0.,6.)-- (10.,6.);
%\draw [line width=1.pt,color=zzttqq] (5.,6.380468642420676) -- (5.,5.619531357579324);
%\draw [line width=1.pt,color=zzttqq] (10.,6.)-- (10.,0.);
%\draw [line width=1.pt,color=zzttqq] (10.114140592726201,3.0456562370904816) -- (9.885859407273795,3.0456562370904816);
%\draw [line width=1.pt,color=zzttqq] (10.114140592726201,2.9543437629095193) -- (9.885859407273795,2.9543437629095193);
%\draw [line width=1.pt,color=zzttqq] (10.,0.)-- (0.,0.);
%\draw [line width=1.pt,color=zzttqq] (5.,-0.3804686424206773) -- (5.,0.3804686424206773);
%\draw [line width=1.pt,color=zzttqq] (0.,0.)-- (0.,6.);
%\draw [line width=1.pt,color=zzttqq] (-0.11414059272620307,2.9543437629095193) -- (0.11414059272620307,2.9543437629095193);
%\draw [line width=1.pt,color=zzttqq] (-0.11414059272620307,3.0456562370904816) -- (0.11414059272620307,3.0456562370904816);
%%\begin{scriptsize}
%%\draw [fill=xdxdff] (0.,6.) circle (2.0pt);
%%\draw[color=xdxdff] (-0.5029878946582631,6.475175723498838) node {$A$};
%%\draw [fill=xdxdff] (10.,6.) circle (2.0pt);
%%\draw[color=xdxdff] (10.271884058695306,6.475175723498838) node {$B$};
%%\draw [fill=xdxdff] (10.,0.) circle (2.0pt);
%%\draw[color=xdxdff] (10.340368414331028,-0.1678067731661803) node {$C$};
%%\draw [fill=xdxdff] (0.,0.) circle (2.0pt);
%%\draw[color=xdxdff] (-0.5029878946582631,-0.23629112880190217) node {$D$};
%\draw[color=zzttqq] (5,-0.7) node {$L$};
%\draw[color=zzttqq] (-0.7,3) node {$l$};
%%\end{scriptsize}
%\end{tikzpicture}
%\end{center}
%\end{Pp}
%
%\begin{Pv}
%L'aire du rectangle correspond au nombre de carreaux verts unitaires rentrant à l'intérieur. Pour la calculer, on multiplie le nombre d'unité rentrant dans la largeur par le nombre d'unité rentrant dans la longueur. Ici :
%\[3\times5=15\,u\] 
%\begin{center}
%\begin{tikzpicture}[line cap=round,line join=round,>=triangle 45,x=1.0cm,y=1.0cm]
%\draw [color=cqcqcq,, xstep=2.0cm,ystep=2.0cm] (-0.7997534357463911,-0.7727519146150565) grid (10.865415140871562,6.851839679495307);
%\clip(-0.7997534357463911,-0.7727519146150565) rectangle (10.865415140871562,6.851839679495307);
%\fill[line width=1.pt,color=zzttqq,fill=zzttqq,fill opacity=0.10000000149011612] (0.,6.) -- (10.,6.) -- (10.,0.) -- (0.,0.) -- cycle;
%\draw[line width=1.pt,color=zzttqq] (0.,5.515742477247899) -- (0.48425752275210066,5.515742477247899) -- (0.4842575227521006,6.) -- (0.,6.) -- cycle; 
%\draw[line width=1.pt,color=zzttqq] (9.5157424772479,6.) -- (9.5157424772479,5.515742477247899) -- (10.,5.515742477247899) -- (10.,6.) -- cycle; 
%\draw[line width=1.pt,color=zzttqq] (10.,0.4842575227521006) -- (9.5157424772479,0.48425752275210066) -- (9.5157424772479,0.) -- (10.,0.) -- cycle; 
%\draw[line width=1.pt,color=zzttqq] (0.4842575227521006,0.) -- (0.48425752275210066,0.4842575227521006) -- (0.,0.4842575227521006) -- (0.,0.) -- cycle; 
%\fill[line width=1.pt,color=qqwuqq,fill=qqwuqq,fill opacity=0.10000000149011612] (0.,0.) -- (0.,2.) -- (2.,2.) -- (2.,0.) -- cycle;
%\draw [line width=1.pt,color=zzttqq] (0.,6.)-- (10.,6.);
%\draw [line width=1.pt,color=zzttqq] (5.,6.114140592726204) -- (5.,5.885859407273798);
%\draw [line width=1.pt,color=zzttqq] (10.,6.)-- (10.,0.);
%\draw [line width=1.pt,color=zzttqq] (10.114140592726201,3.0456562370904816) -- (9.885859407273795,3.0456562370904816);
%\draw [line width=1.pt,color=zzttqq] (10.114140592726201,2.9543437629095193) -- (9.885859407273795,2.9543437629095193);
%\draw [line width=1.pt,color=zzttqq] (10.,0.)-- (0.,0.);
%\draw [line width=1.pt,color=zzttqq] (5.,-0.11414059272620306) -- (5.,0.11414059272620306);
%\draw [line width=1.pt,color=zzttqq] (0.,0.)-- (0.,6.);
%\draw [line width=1.pt,color=zzttqq] (-0.11414059272620307,2.9543437629095193) -- (0.11414059272620307,2.9543437629095193);
%\draw [line width=1.pt,color=zzttqq] (-0.11414059272620307,3.0456562370904816) -- (0.11414059272620307,3.0456562370904816);
%\draw [line width=1.pt,color=qqwuqq] (0.,0.)-- (0.,2.);
%\draw [line width=1.pt,color=qqwuqq] (0.,2.)-- (2.,2.);
%\draw [line width=1.pt,color=qqwuqq] (2.,2.)-- (2.,0.);
%\draw [line width=1.pt,color=qqwuqq] (2.,0.)-- (0.,0.);
%\draw [line width=1.pt,color=qqwuqq] (2.,2.)-- (2.,6.);
%\draw [line width=1.pt,color=qqwuqq] (0.,4.)-- (2.,4.);
%\draw [line width=1.pt,color=qqwuqq] (2.,2.)-- (10.,2.);
%\draw [line width=1.pt,color=qqwuqq] (4.,2.)-- (4.,0.);
%\draw [line width=1.pt,color=qqwuqq] (6.,2.)-- (6.,0.);
%\draw [line width=1.pt,color=qqwuqq] (8.,0.)-- (8.,2.);
%%\begin{scriptsize}
%\draw [fill=xdxdff] (0.,6.) circle (2pt);
%\draw[color=xdxdff] (-0.5029878946582631,6.429519486408356) node {$A$};
%\draw [fill=xdxdff] (10.,6.) circle (2pt);
%\draw[color=xdxdff] (10.271884058695306,6.429519486408356) node {$B$};
%\draw [fill=xdxdff] (10.,0.) circle (2pt);
%\draw[color=xdxdff] (10.340368414331028,-0.21346301025666153) node {$C$};
%\draw [fill=xdxdff] (0.,0.) circle (2pt);
%\draw[color=xdxdff] (-0.5029878946582631,-0.2819473658923834) node {$D$};
%\draw[color=qqwuqq] (1,1) node {$unité$};
%\draw[color=qqwuqq] (1,3) node {$2$};
%\draw[color=qqwuqq] (1,5) node {$3$};
%\draw[color=qqwuqq] (3,1) node {$2$};
%\draw[color=qqwuqq] (5,1) node {$3$};\draw[color=qqwuqq] (7,1) node {$4$};\draw[color=qqwuqq] (9,1) node {$5$};
%%\end{scriptsize}
%\end{tikzpicture}
%\end{center}
%\end{Pv}
%

%\subsection{Aire d'un triangle}
%
%\begin{PpT}{Aire d'un triangle rectangle}
%L'aire d'un triangle rectangle est donnée par la formule :
%\[\mathcal{A}_{triangle\,rectangle}=\frac{a\times b}{2}\]
%\begin{center}
%\begin{tikzpicture}[line cap=round,line join=round,>=triangle 45,x=0.6cm,y=0.6cm]
%\clip(-1.0736908582892786,-1.0695174557031843) rectangle (11.185008800504932,6.897495916585789);
%\draw[line width=1.pt,color=zzttqq] (0.4842575227521006,0.) -- (0.48425752275210066,0.4842575227521006) -- (0.,0.4842575227521006) -- (0.,0.) -- cycle; 
%\fill[line width=1.pt,color=zzttqq,fill=zzttqq,fill opacity=0.10000000149011612] (0.,0.) -- (0.,6.) -- (10.,0.) -- cycle;
%%\draw [line width=1.pt,dash pattern=on 1pt off 1pt] (0.,6.)-- (10.,6.);
%%\draw [line width=1.pt,dash pattern=on 1pt off 1pt] (10.,6.)-- (10.,0.);
%\draw [line width=1.pt,color=zzttqq] (0.,0.)-- (0.,6.);
%\draw [line width=1.pt,color=zzttqq] (0.,6.)-- (10.,0.);
%\draw [line width=1.pt,color=zzttqq] (10.,0.)-- (0.,0.);
%%\begin{scriptsize}
%\draw [fill=xdxdff] (0.,6.) circle (2.0pt);
%\draw[color=xdxdff] (-0.5029878946582631,6.475175723498838) node {$A$};
%\draw [fill=xdxdff] (10.,0.) circle (2.0pt);
%\draw[color=xdxdff] (10.340368414331028,-0.1678067731661803) node {$B$};
%\draw [fill=xdxdff] (0.,0.) circle (2.0pt);
%\draw[color=xdxdff] (-0.5029878946582631,-0.23629112880190217) node {$C$};
%\draw[color=zzttqq] (0.41013684715136145,2.913989230441302) node {$b$};
%\draw[color=zzttqq] (4.7703074892923185,0.5626930202815192) node {$a$};
%%\end{scriptsize}
%\end{tikzpicture}
%\end{center}
%\end{PpT}
%
%\begin{Pv}
%Il s'agit de l'aire d'un \textbf{demi-rectangle}, donc la moitié de l'aire du rectangle dont une des diagonales est $[AB]$.
%\begin{center}
%\begin{tikzpicture}[line cap=round,line join=round,>=triangle 45,x=0.6cm,y=0.6cm]
%\clip(-1.0736908582892786,-1.0695174557031843) rectangle (11.185008800504932,6.897495916585789);
%\draw[line width=1.pt,color=zzttqq] (0.4842575227521006,0.) -- (0.48425752275210066,0.4842575227521006) -- (0.,0.4842575227521006) -- (0.,0.) -- cycle; 
%\fill[line width=1.pt,color=zzttqq,fill=zzttqq,fill opacity=0.10000000149011612] (0.,0.) -- (0.,6.) -- (10.,0.) -- cycle;
%\draw [line width=1.pt,dash pattern=on 1pt off 2pt] (0.,6.)-- (10.,6.);
%\draw [line width=1.pt,dash pattern=on 1pt off 2pt] (10.,6.)-- (10.,0.);
%\draw [line width=1.pt,color=zzttqq] (0.,0.)-- (0.,6.);
%\draw [line width=1.pt,color=zzttqq] (0.,6.)-- (10.,0.);
%\draw [line width=1.pt,color=zzttqq] (10.,0.)-- (0.,0.);
%%\begin{scriptsize}
%\draw [fill=xdxdff] (0.,6.) circle (2.0pt);
%\draw[color=xdxdff] (-0.5029878946582631,6.475175723498838) node {$A$};
%\draw [fill=xdxdff] (10.,0.) circle (2.0pt);
%\draw[color=xdxdff] (10.340368414331028,-0.1678067731661803) node {$B$};
%\draw [fill=xdxdff] (0.,0.) circle (2.0pt);
%\draw[color=xdxdff] (-0.5029878946582631,-0.23629112880190217) node {$C$};
%\draw[color=zzttqq] (0.41013684715136145,2.913989230441302) node {$b$};
%\draw[color=zzttqq] (4.7703074892923185,0.5626930202815192) node {$a$};
%%\end{scriptsize}
%\end{tikzpicture}
%\end{center}
%\end{Pv}
%
%\begin{PpT}{Aire d'un triangle quelconque}
%L'aire d'un triangle quelconque est donné par la formule :
%\[\mathcal{A}_{triangle}=\frac{base\times hauteur}{2}\]
%\[\mathcal{A}_{triangle}=\frac{b\times h}{2}\]
%Où la \textcolor{ffqqqq}{hauteur (h)} désigne une perpendiculaire à un côté du triangle passant par le sommet opposé et la \textcolor{zzttqq}{base (b)} désigne le côté que la \textcolor{ffqqqq}{hauteur} coupe.
%\begin{center}
%\begin{tikzpicture}[line cap=round,line join=round,>=triangle 45,x=0.6cm,y=0.6cm]
%\clip(-1.598737584829813,-0.5444707291626504) rectangle (10.979555733597763,7.6051675914882475);
%\fill[line width=1.pt,color=zzttqq,fill=zzttqq,fill opacity=0.10000000149011612] (0.,0.) -- (4.,6.) -- (10.,2.) -- cycle;
%\draw[line width=1.pt,color=qqwuqq,fill=qqwuqq,fill opacity=0.10000000149011612] (4.9050292862262665,1.474853568868666) -- (4.430175717357601,1.3798828550949327) -- (4.5251464311313345,0.9050292862262668) -- (5.,1.) -- cycle; 
%\draw [line width=1.pt,color=zzttqq] (0.,0.)-- (4.,6.);
%\draw [line width=1.pt,color=zzttqq] (4.,6.)-- (10.,2.);
%\draw [line width=1.pt,color=zzttqq] (10.,2.)-- (0.,0.);
%\draw [line width=1.pt,color=ffqqqq] (4.,6.)-- (5.,1.);
%%\draw [line width=1.pt,dash pattern=on 2pt off 2pt] (0.,0.)-- (-1.,5.);
%%\draw [line width=1.pt,dash pattern=on 2pt off 2pt] (-1.,5.)-- (9.,7.);
%%\draw [line width=1.pt,dash pattern=on 2pt off 2pt] (9.,7.)-- (10.,2.);
%%\begin{scriptsize}
%\draw [fill=xdxdff] (0.,0.) circle (2.0pt);
%\draw[color=xdxdff] (-0.6171284873844668,0.24309936064815066) node {$A$};
%\draw [fill=xdxdff] (4.,6.) circle (2.5pt);
%\draw[color=xdxdff] (4.15394828857082,6.429519486408356) node {$B$};
%\draw [fill=xdxdff] (10.,2.) circle (2.5pt);
%\draw[color=xdxdff] (10.157743465969101,2.4117706224460087) node {$C$};
%\draw[color=zzttqq] (5.135557386016167,0.38006807191959435) node {$b$};
%\draw [fill=uuuuuu] (5.,1.) circle (2.0pt);
%\draw[color=ffqqqq] (4.085463932935099,3.3020672457103926) node {$h$};
%%\end{scriptsize}
%\end{tikzpicture}
%\end{center}
%\end{PpT}
%

%\subsection{Aire d'un disque}
%
%\begin{Def}
%Un \textbf{disque} est un cercle contenant une surface.
%\begin{center}
%\begin{tikzpicture}[line cap=round,line join=round,>=triangle 45,x=1.0cm,y=1.0cm]
%\clip(-3.2258782096036835,-1.3387575571899715) rectangle (3.220423513612537,1.6953257957763235);
%\draw [line width=1.pt,color=zzttqq,fill=zzttqq,fill opacity=0.10000000149011612] (2.,0.) circle (1.cm);
%\draw [line width=1.pt] (-2.,0.) circle (1.cm);
%\begin{scriptsize}
%\draw [fill=xdxdff] (-2.,0.) circle (2.5pt);
%\draw[color=black] (-2,1.3) node {$cercle$};
%\draw [fill=xdxdff] (2.,0.) circle (2.5pt);
%\draw[color=zzttqq] (2,1.3) node {$disque$};
%\end{scriptsize}
%\end{tikzpicture}
%\end{center}
%\end{Def}
%
%\begin{Pp}
%L'aire d'un disque est donnée par la formule :
%\[\mathcal{A}_{disque}=\pi \times rayon \times rayon=\pi \times rayon^2\]
%\[\mathcal{A}_{disque}=\pi \times r^2\]
%\begin{center}
%\begin{tikzpicture}[line cap=round,line join=round,>=triangle 45,x=0.5cm,y=0.5cm]
%%\clip(-0.4116754204773013,-4.356766526217832) rectangle (10.705618311054875,6.3952773085904955);
%\draw [line width=1.pt,color=zzttqq,fill=zzttqq,fill opacity=0.1] (5.021416793289964,1.0306694504589518) circle (2.363592350903993cm);
%\draw [line width=1.pt,color=qqwuqq] (5.021416793289964,1.0306694504589518)-- (9.221790605614236,3.19934071225681);
%%\begin{scriptsize}
%\draw [fill=ududff] (5.021416793289964,1.0306694504589518) circle (2.5pt);
%\draw[color=ududff] (4.450713829658948,1.3160209322744594) node {$A$};
%\draw [fill=ududff] (9.221790605614236,3.19934071225681) circle (2.5pt);
%\draw[color=ududff] (9.381587435430921,3.621660905343761) node {$B$};
%\draw[color=qqwuqq] (7.349884884904506,1.7497551846340311) node {$r$};
%%\end{scriptsize}
%\end{tikzpicture}
%\end{center}
%\end{Pp}
%
%\begin{Rq}
%$r^2$ se prononce "$r$ au carré" et signifie qu'il faut prendre le nombre $r\times r$.
%\end{Rq}
%
%\begin{Pv}
%La preuve de la formule de l'aire du disque a été apportée par le grand mathématicien Archimède. Il a découpé le disque en secteurs égaux puis les a rassemblés pour former une sorte de rectangle dont la largeur est le rayon et la longueur est la moitié du périmètre :
%\begin{center}
%\begin{tikzpicture}[line cap=round,line join=round,>=triangle 45,x=1.0cm,y=1.0cm]
%\clip(-5.06354915382803,-6.147772259251827) rectangle (4.735026060140306,2.4059020054008484);
%\draw [line width=1.pt,color=zzttqq,fill=zzttqq,fill opacity=0.10999999940395355] (0.,0.) circle (2.cm);
%\draw [line width=1.pt,color=zzttqq] (0.,2.)-- (0.,-2.);
%\draw [line width=1.pt,color=zzttqq] (-2.,0.)-- (2.,0.);
%\draw [line width=1.pt,color=zzttqq] (-1.4142135623730951,1.4142135623730951)-- (1.414213562373095,-1.414213562373095);
%\draw [line width=1.pt,color=zzttqq] (1.414213562373095,1.414213562373095)-- (-1.4142135623730951,-1.4142135623730951);
%\draw [shift={(-2.84,-5.4)},line width=1.pt,color=zzttqq,fill=zzttqq,fill opacity=0.10999999940395355]  (0,0) --  plot[domain=1.1955255039389394:1.9809236673363877,variable=\t]({1.*2.128097742116184*cos(\t r)+0.*2.128097742116184*sin(\t r)},{0.*2.128097742116184*cos(\t r)+1.*2.128097742116184*sin(\t r)}) -- cycle ;
%\draw [shift={(-2.06,-3.42)},line width=1.pt,color=zzttqq,fill=zzttqq,fill opacity=0.10999999940395355]  (0,0) --  plot[domain=4.3371181575287325:5.122516320926181,variable=\t]({1.*2.128097742116184*cos(\t r)+0.*2.128097742116184*sin(\t r)},{0.*2.128097742116184*cos(\t r)+1.*2.128097742116184*sin(\t r)}) -- cycle ;
%\draw [shift={(-1.2114718625761427,-5.3716147160748715)},line width=1.pt,color=zzttqq,fill=zzttqq,fill opacity=0.10999999940395355]  (0,0) --  plot[domain=1.1955255039389394:1.9809236673363877,variable=\t]({1.*2.128097742116184*cos(\t r)+0.*2.128097742116184*sin(\t r)},{0.*2.128097742116184*cos(\t r)+1.*2.128097742116184*sin(\t r)}) -- cycle ;
%\draw [shift={(-0.4314718625761429,-3.391614716074871)},line width=1.pt,color=zzttqq,fill=zzttqq,fill opacity=0.10999999940395355]  (0,0) --  plot[domain=4.3371181575287325:5.122516320926181,variable=\t]({1.*2.128097742116184*cos(\t r)+0.*2.128097742116184*sin(\t r)},{0.*2.128097742116184*cos(\t r)+1.*2.128097742116184*sin(\t r)}) -- cycle ;
%\draw [shift={(0.41705627484771446,-5.343229432149743)},line width=1.pt,color=zzttqq,fill=zzttqq,fill opacity=0.10999999940395355]  (0,0) --  plot[domain=1.1955255039389394:1.9809236673363877,variable=\t]({1.*2.128097742116184*cos(\t r)+0.*2.128097742116184*sin(\t r)},{0.*2.128097742116184*cos(\t r)+1.*2.128097742116184*sin(\t r)}) -- cycle ;
%\draw [shift={(1.1970562748477143,-3.3632294321497422)},line width=1.pt,color=zzttqq,fill=zzttqq,fill opacity=0.10999999940395355]  (0,0) --  plot[domain=4.3371181575287325:5.122516320926181,variable=\t]({1.*2.128097742116184*cos(\t r)+0.*2.128097742116184*sin(\t r)},{0.*2.128097742116184*cos(\t r)+1.*2.128097742116184*sin(\t r)}) -- cycle ;
%\draw [shift={(2.0455844122715714,-5.314844148224614)},line width=1.pt,color=zzttqq,fill=zzttqq,fill opacity=0.10999999940395355]  (0,0) --  plot[domain=1.1955255039389392:1.9809236673363877,variable=\t]({1.*2.1280977421161844*cos(\t r)+0.*2.1280977421161844*sin(\t r)},{0.*2.1280977421161844*cos(\t r)+1.*2.1280977421161844*sin(\t r)}) -- cycle ;
%\draw [shift={(2.8255844122715716,-3.3348441482246134)},line width=1.pt,color=zzttqq,fill=zzttqq,fill opacity=0.10999999940395355]  (0,0) --  plot[domain=4.337118157528732:5.122516320926181,variable=\t]({1.*2.1280977421161844*cos(\t r)+0.*2.1280977421161844*sin(\t r)},{0.*2.1280977421161844*cos(\t r)+1.*2.1280977421161844*sin(\t r)}) -- cycle ;
%%\draw [line width=1.pt] (-3.688528137423857,-3.448385283925129)-- (-2.84,-5.4);
%%\draw [line width=1.pt] (-3.688528137423857,-3.448385283925129)-- (2.8722786296747205,-3.3229307497861025);
%%\begin{scriptsize}
%%\draw [fill=zzttqq] (0.,0.) circle (2.0pt);
%%\draw [fill=xdxdff] (-3.688528137423857,-3.448385283925129) circle (2.5pt);
%%\draw [fill=xdxdff] (-2.84,-5.4) circle (2.5pt);
%\draw[color=black] (-3.9592015374996326,-4.531408929898445) node {$r$};
%%\draw [fill=xdxdff] (2.8722786296747205,-3.3229307497861025) circle (2.5pt);
%\draw[color=black] (-0.12410345170464818,-2.724294648633796) node {$\frac{\mathcal{P}}{2}$};
%%\end{scriptsize}
%\end{tikzpicture}
%\end{center}
%En augmentant le nombre de secteurs la forme ressemble de plus en plus à un rectangle et pour un nombre infini de secteur elle devient un rectangle. La formule de l'aire du rectangle nous donne :
%\[\mathcal{A}_{disque}=r\times \frac{\mathcal{P}}{2}=r\times \frac{2\times\pi\times r}{2}=\pi \times r \times r\]
%\end{Pv}
%

%\section{Formulaire}

\begin{Pp}
\begin{center}
\begin{tikzpicture}[line cap=round,line join=round,>=triangle 45,x=1.cm,y=1.cm]
\clip(-1.1508968623201985,-5.535831819898127) rectangle (14.729158042124956,2.4283294543676694);
\fill[line width=1.pt,color=sacado_blue,fill=sacado_blue,fill opacity=0.10000000149011612] (0.,0.) -- (0.,2.) -- (2.,2.) -- (2.,0.) -- cycle;
\fill[line width=1.pt,color=sacado_blue,fill=sacado_blue,fill opacity=0.10000000149011612] (4.,2.) -- (4.,0.) -- (8.,0.) -- (8.,2.) -- cycle;
\fill[line width=1.pt,color=sacado_blue,fill=sacado_blue,fill opacity=0.10000000149011612] (0.,-2.) -- (0.,-4.) -- (3.,-4.) -- cycle;
\fill[line width=1.pt,color=sacado_blue,fill=sacado_blue,fill opacity=0.10000000149011612] (6.,-2.) -- (5.,-4.) -- (9.,-3.) -- cycle;
\draw [line width=1.pt,color=sacado_blue,fill=sacado_blue,fill opacity=0.14000000059604645] (12.,-1.) circle (1.54cm);
\draw[line width=1.pt,color=sacado_blue,fill=sacado_blue,fill opacity=0.10000000149011612] (0.34130378445434223,-4.) -- (0.3413037844543423,-3.6586962155456577) -- (0.,-3.6586962155456577) -- (0.,-4.) -- cycle; 
\draw[line width=1.pt,color=sacado_blue,fill=sacado_blue,fill opacity=0.10000000149011612] (0.,1.6586962155456577) -- (0.3413037844543423,1.6586962155456577) -- (0.34130378445434223,2.) -- (0.,2.) -- cycle; 
\draw[line width=1.pt,color=sacado_blue,fill=sacado_blue,fill opacity=0.10000000149011612] (1.6586962155456577,2.) -- (1.6586962155456577,1.6586962155456577) -- (2.,1.6586962155456577) -- (2.,2.) -- cycle; 
\draw[line width=1.pt,color=sacado_blue,fill=sacado_blue,fill opacity=0.10000000149011612] (2.,0.34130378445434223) -- (1.6586962155456577,0.3413037844543423) -- (1.6586962155456577,0.) -- (2.,0.) -- cycle; 
\draw[line width=1.pt,color=sacado_blue,fill=sacado_blue,fill opacity=0.10000000149011612] (0.34130378445434223,0.) -- (0.3413037844543423,0.34130378445434223) -- (0.,0.34130378445434223) -- (0.,0.) -- cycle; 
\draw[line width=1.pt,color=sacado_blue,fill=sacado_blue,fill opacity=0.10000000149011612] (4.,1.6586962155456577) -- (4.341303784454342,1.6586962155456577) -- (4.341303784454342,2.) -- (4.,2.) -- cycle; 
\draw[line width=1.pt,color=sacado_blue,fill=sacado_blue,fill opacity=0.10000000149011612] (7.658696215545658,2.) -- (7.658696215545658,1.6586962155456577) -- (8.,1.6586962155456577) -- (8.,2.) -- cycle; 
\draw[line width=1.pt,color=sacado_blue,fill=sacado_blue,fill opacity=0.10000000149011612] (8.,0.34130378445434223) -- (7.658696215545658,0.3413037844543423) -- (7.658696215545658,0.) -- (8.,0.) -- cycle; 
\draw[line width=1.pt,color=sacado_blue,fill=sacado_blue,fill opacity=0.10000000149011612] (4.341303784454342,0.) -- (4.341303784454342,0.34130378445434223) -- (4.,0.34130378445434223) -- (4.,0.) -- cycle; 
\draw[line width=1.pt,color=sacado_blue,fill=sacado_blue,fill opacity=0.10000000149011612] (6.7428780126419525,-3.564280496839512) -- (6.660099685952052,-3.233167190079913) -- (6.328986379192454,-3.315945516769813) -- (6.411764705882353,-3.6470588235294117) -- cycle; 
\draw [line width=1.pt,color=sacado_blue] (0.,0.)-- (0.,2.);
\draw [line width=1.pt,color=sacado_blue] (-0.09653528817291887,1.) -- (0.09653528817291887,1.);
\draw [line width=1.pt,color=sacado_blue] (0.,2.)-- (2.,2.);
\draw [line width=1.pt,color=sacado_blue] (1.,2.0965352881729187) -- (1.,1.903464711827081);
\draw [line width=1.pt,color=sacado_blue] (2.,2.)-- (2.,0.);
\draw [line width=1.pt,color=sacado_blue] (2.0965352881729196,1.) -- (1.9034647118270818,1.);
\draw [line width=1.pt,color=sacado_blue] (2.,0.)-- (0.,0.);
\draw [line width=1.pt,color=sacado_blue] (1.,-0.09653528817291875) -- (1.,0.09653528817291875);
\draw [line width=1.pt,color=sacado_blue] (4.,2.)-- (4.,0.);
\draw [line width=1.pt,color=sacado_blue] (4.096535288172918,1.) -- (3.9034647118270804,1.);
\draw [line width=1.pt,color=sacado_blue] (4.,0.)-- (8.,0.);
\draw [line width=1.pt,color=sacado_blue] (5.959776963261284,0.09653528817291875) -- (5.959776963261284,-0.09653528817291875);
\draw [line width=1.pt,color=sacado_blue] (6.040223036738716,0.09653528817291875) -- (6.040223036738716,-0.09653528817291875);
\draw [line width=1.pt,color=sacado_blue] (8.,0.)-- (8.,2.);
\draw [line width=1.pt,color=sacado_blue] (7.903464711827081,1.) -- (8.09653528817292,1.);
\draw [line width=1.pt,color=sacado_blue] (8.,2.)-- (4.,2.);
\draw [line width=1.pt,color=sacado_blue] (6.040223036738716,1.903464711827081) -- (6.040223036738716,2.0965352881729187);
\draw [line width=1.pt,color=sacado_blue] (5.959776963261284,1.903464711827081) -- (5.959776963261284,2.0965352881729187);
\draw [line width=1.pt,color=sacado_blue] (0.,-2.)-- (0.,-4.);
\draw [line width=1.pt,color=sacado_blue] (0.,-4.)-- (3.,-4.);
\draw [line width=1.pt,color=sacado_blue] (3.,-4.)-- (0.,-2.);
\draw [line width=1.pt,color=sacado_blue] (6.,-2.)-- (5.,-4.);
\draw [line width=1.pt,color=sacado_blue] (5.,-4.)-- (9.,-3.);
\draw [line width=1.pt,color=sacado_blue] (9.,-3.)-- (6.,-2.);
\draw [line width=1.pt] (12.,-1.)-- (10.46,-1.);
\draw (-0.45906063041427997,-0.35510468795148753) node[anchor=north west] {$\mathcal{A}_{carre}=c\times c=c^2$};
\draw (4.769934145618826,-0.3390154732560011) node[anchor=north west] {$\mathcal{A}_{rectangle}=l\times L$};
\draw (-0.2659900540684422,-4.0234456385224) node[anchor=north west] {$\mathcal{A}_{triangle\;rectangle}=\frac{b\times h}{2}$};
\draw (5.220432157092447,-3.991267209131427) node[anchor=north west] {$\mathcal{A}_{triangle}=\frac{b\times h}{2}$};
\draw (10.079374995129363,-2.823683960015078) node[anchor=north west] {$\mathcal{A}_{disque}=\pi\times r\times r$};
\draw (11.079374995129363,-3.423683960015078) node[anchor=north west] {$=\pi\times r^2$};
\draw [line width=1.pt] (6.411764705882353,-3.6470588235294117)-- (6.,-2.);
%\begin{scriptsize}
\draw[color=sacado_blue] (1.0372363362659625,0.4091330100841192) node {$c$};
\draw[color=sacado_blue] (4.43206063701361,1.0527015979035774) node {$l$};
\draw[color=sacado_blue] (5.944446818389339,0.5217575129525244) node {$L$};
\draw[color=sacado_blue] (0.26495403088261155,-2.9856912906635236) node {$h$};
\draw[color=sacado_blue] (1.3107529860892326,-3.6) node {$b$};
\draw[color=sacado_blue] (7.143093313203082,-3.8) node {$b$};
\draw[color=black] (11.269976882595364,-0.6044875157315277) node {$r$};
\draw[color=black] (6.491480118035879,-2.6156393526673347) node {$h$};
%\end{scriptsize}
\end{tikzpicture}
\end{center}
\end{Pp}

 

\begin{Rqs}
\begin{itemize}
\item $r^2$ se prononce "$r$ au carré" et signifie qu'il faut prendre le nombre $r\times r$.
\item Deux triangles de même hauteur et de même base ont la même aire.
\item L'aire d'un triangle ne dépend pas du côté choisi.
\end{itemize}
\end{Rqs}



\end{pageCours} 

%%%%%%%%%%%%%%%%%%%%%%%%%%%%%%%%%%%%%%%%%%%%%%%%%%%%%%%%%%%%%%%%%%%%%%%%%%%%%%%%%%%%%%%%%%%%%%%%%%%%%
%%%%%%%%%% Application directe
%%%%%%%%%%%%%%%%%%%%%%%%%%%%%%%%%%%%%%%%%%%%%%%%%%%%%%%%%%%%%%%%%%%%%%%%%%%%%%%%%%%%%%%%%%%%%%%%%%%%%
\begin{pageAD}

\Sf{Calculer l'aire d'un carré, d'un rectangle, d'un triangle, d'un disque.}


\ExoCad{Calculer.}

 
 Un carré a un côté de $5,3\,cm$. Calcule son aire.\point{2}
 

\ExoCad{Calculer.}

 Un rectangle a une longueur de $11\,dm$ et une largeur de $4,5\,dm$. Calcule son aire.\point{2}
 

\ExoCad{Calculer.}
 
La base d'un triangle $ABC$ mesure $85\,mm $ et sa hauteur mesure $10\,mm$. Calcule l'aire du triangle $ABC$.\point{2}


\ExoCad{Calculer.}
 
 Calcule l'aire du disque de centre $O$ et de rayon $4\,cm$. \point{2}
 
\vspace{0.4cm}
 
\Sf{Savoir résoudre un problème d'aire.}



\ExoCad{Calculer.}
 

\begin{minipage}{.68\linewidth}
La figure ci-contre est composée d'un rectangle $ABCD$ et d'un triangle $ADE$. On donne les mesures suivantes :
$AB = 3\,dam$, $BC = 5\,dam$ et $EF = 2,6\,dam$.

Calcule l'aire de la surface $ABCDE$.

\point{5}
\end{minipage}
\hfill
\begin{minipage}{.28\linewidth}


\definecolor{xfqqff}{rgb}{0.4980392156862745,0.,1.}
\begin{tikzpicture}[line cap=round,line join=round,>=triangle 45,x=1.0cm,y=1.0cm]
\clip(-1.82,-0.4) rectangle (3.84,4.74);
\fill[line width=1.pt,color=xfqqff,fill=xfqqff,fill opacity=0.15000000596046448] (-1.,2.) -- (-1.,0.) -- (2.5,0.) -- (2.5,2.) -- cycle;
\fill[line width=1.pt,color=xfqqff,fill=xfqqff,fill opacity=0.15000000596046448] (-1.,2.) -- (1.62,4.02) -- (2.5,2.) -- cycle;
\draw [line width=1.pt,color=xfqqff] (-1.,2.)-- (-1.,0.);
\draw [line width=1.pt,color=xfqqff] (-1.,0.)-- (2.5,0.);
\draw [line width=1.pt,color=xfqqff] (2.5,0.)-- (2.5,2.);
\draw [line width=1.pt,dash pattern=on 1pt off 1pt,color=xfqqff] (2.5,2.)-- (-1.,2.);
\draw [line width=1.pt,color=xfqqff] (-1.,2.)-- (1.62,4.02);
\draw [line width=1.pt,color=xfqqff] (1.62,4.02)-- (2.5,2.);
\draw [line width=1.pt,color=xfqqff] (1.62,2.)-- (1.62,4.02);
\begin{scriptsize}
\draw [color=xfqqff] (-1.,2.)-- ++(-2.5pt,0 pt) -- ++(5.0pt,0 pt) ++(-2.5pt,-2.5pt) -- ++(0 pt,5.0pt);
\draw[color=xfqqff] (-1.42,2.31) node {$A$};
\draw [color=xfqqff] (-1.,0.)-- ++(-2.5pt,0 pt) -- ++(5.0pt,0 pt) ++(-2.5pt,-2.5pt) -- ++(0 pt,5.0pt);
\draw[color=xfqqff] (-1.36,-0.09) node {$B$};
\draw [color=xfqqff] (2.5,0.)-- ++(-2.5pt,0 pt) -- ++(5.0pt,0 pt) ++(-2.5pt,-2.5pt) -- ++(0 pt,5.0pt);
\draw[color=xfqqff] (2.78,-0.11) node {$C$};
\draw [color=xfqqff] (2.5,2.)-- ++(-2.5pt,0 pt) -- ++(5.0pt,0 pt) ++(-2.5pt,-2.5pt) -- ++(0 pt,5.0pt);
\draw[color=xfqqff] (2.82,2.17) node {$D$};
\draw [color=xfqqff] (1.62,4.02)-- ++(-2.5pt,0 pt) -- ++(5.0pt,0 pt) ++(-2.5pt,-2.5pt) -- ++(0 pt,5.0pt);
\draw[color=xfqqff] (1.76,4.39) node {$E$};
\draw [color=xfqqff] (1.62,2.)-- ++(-2.0pt,0 pt) -- ++(4.0pt,0 pt) ++(-2.0pt,-2.0pt) -- ++(0 pt,4.0pt);
\draw[color=xfqqff] (1.76,2.33) node {$F$};
\end{scriptsize}
\end{tikzpicture}





\end{minipage}

\ExoCad{Calculer.}
 

\begin{minipage}{.68\linewidth}
La surface ci-contre est un carré et un demi-disque.
 Calcule son aire en $cm^2$. \point{5}
\end{minipage}
\hfill
\begin{minipage}{.28\linewidth}


\definecolor{ffqqqq}{rgb}{1.,0.,0.}
\begin{tikzpicture}[line cap=round,line join=round,>=triangle 45,x=1.0cm,y=1.0cm]
\clip(-1.32,-0.44) rectangle (2.38,3.5);
\fill[line width=1.pt,color=ffqqqq,fill=ffqqqq,fill opacity=0.5] (-1.,2.) -- (-1.,0.) -- (1.,0.) -- (1.,2.) -- cycle;
\draw [line width=1.pt,color=ffqqqq] (-1.,2.)-- (-1.,0.);
\draw [line width=1.pt,color=ffqqqq] (-0.88,1.05) -- (-1.12,1.05);
\draw [line width=1.pt,color=ffqqqq] (-0.88,0.95) -- (-1.12,0.95);
\draw [line width=1.pt,color=ffqqqq] (-1.,0.)-- (1.,0.);
\draw [line width=1.pt,color=ffqqqq] (-0.05,0.12) -- (-0.05,-0.12);
\draw [line width=1.pt,color=ffqqqq] (0.05,0.12) -- (0.05,-0.12);
\draw [line width=1.pt,color=ffqqqq] (1.,0.)-- (1.,2.);
\draw [line width=1.pt,color=ffqqqq] (0.88,0.95) -- (1.12,0.95);
\draw [line width=1.pt,color=ffqqqq] (0.88,1.05) -- (1.12,1.05);
\draw [line width=1.pt,dash pattern=on 2pt off 2pt,color=ffqqqq] (1.,2.)-- (-1.,2.);
\draw [shift={(0.,2.)},line width=1.pt,color=ffqqqq,fill=ffqqqq,fill opacity=0.5]  plot[domain=0.:3.141592653589793,variable=\t]({1.*1.*cos(\t r)+0.*1.*sin(\t r)},{0.*1.*cos(\t r)+1.*1.*sin(\t r)});
\begin{scriptsize}
\draw[color=ffqqqq] (0.04,-0.25) node {$4cm$};
\end{scriptsize}
\end{tikzpicture}

\end{minipage}
\end{pageAD} 
%%%%%%%%%%%%%%%%%%%%%%%%%%%%%%%%%%%%%%%%%%%%%%%%%%%%%%%%%%%%%%%%%%%%%%%%%%%%%%%%%%%%%%%%%%%%%%%%%%%%%
%%%%%%%%%% Parcours 1
%%%%%%%%%%%%%%%%%%%%%%%%%%%%%%%%%%%%%%%%%%%%%%%%%%%%%%%%%%%%%%%%%%%%%%%%%%%%%%%%%%%%%%%%%%%%%%%%%%%%%
\begin{pageParcoursu} 



\ExoCu{Chercher.Représenter.}

\begin{minipage}{0.68\linewidth}
Calcule l'aire du triangle $PAS$ tel que $PA = 30\,m$ ; $AR = 10\,m$ ; 

$AS = 18\,m$. \point{4}
\end{minipage}
\hfill
\begin{minipage}{0.68\linewidth}

\definecolor{uuuuuu}{rgb}{0.26666666666666666,0.26666666666666666,0.26666666666666666}
\begin{tikzpicture}[line cap=round,line join=round,>=triangle 45,x=0.7cm,y=0.7cm]
\clip(-4.,-2.42) rectangle (1.54,3.46);
\draw [line width=1.pt] (-3.5,3.)-- (-1.5,-2.);
\draw [line width=1.pt] (-1.5,-2.)-- (1.,-1.);
\draw [line width=1.pt] (1.,-1.)-- (-3.5,3.);
\draw [line width=1.pt] (-1.5,-2.)-- (1.,-1.);
\draw [line width=1.pt] (1.,-1.)-- (-0.5,0.33333333333333326);
\draw [line width=1.pt] (-0.5,0.33333333333333326)-- (-2.1666666666666665,-0.3333333333333334);
\draw [line width=1.pt] (-2.1666666666666665,-0.3333333333333334)-- (-1.5,-2.);
\begin{scriptsize}
\draw [color=black] (-3.5,3.)-- ++(-2.5pt,0 pt) -- ++(5.0pt,0 pt) ++(-2.5pt,-2.5pt) -- ++(0 pt,5.0pt);
\draw[color=black] (-3.8,3.19) node {$P$};
\draw [color=black] (-1.5,-2.)-- ++(-2.5pt,0 pt) -- ++(5.0pt,0 pt) ++(-2.5pt,-2.5pt) -- ++(0 pt,5.0pt);
\draw[color=black] (-1.3,-2.07) node {$R$};
\draw [color=black] (1.,-1.)-- ++(-2.5pt,0 pt) -- ++(5.0pt,0 pt) ++(-2.5pt,-2.5pt) -- ++(0 pt,5.0pt);
\draw[color=black] (1.14,-0.63) node {$C$};
\draw [color=black] (-0.5,0.33333333333333326)-- ++(-2.5pt,0 pt) -- ++(5.0pt,0 pt) ++(-2.5pt,-2.5pt) -- ++(0 pt,5.0pt);
\draw[color=black] (-0.36,0.71) node {$S$};
\draw [color=uuuuuu] (-2.1666666666666665,-0.3333333333333334)-- ++(-2.0pt,0 pt) -- ++(4.0pt,0 pt) ++(-2.0pt,-2.0pt) -- ++(0 pt,4.0pt);
\draw[color=uuuuuu] (-2.52,-0.13) node {$A$};
\end{scriptsize}
\end{tikzpicture}
\end{minipage}
 
\ExoCu{Représenter.}

\begin{minipage}{0.58\linewidth}
Calcule l'aire du triangle $ABC$ tel que $AB = 6\,cm$ et 

$CH=5,4\,cm$. \point{7}
\end{minipage}
\hfill
\begin{minipage}{0.48\linewidth}


 
\definecolor{xfqqff}{rgb}{0.4980392156862745,0.,1.}
\begin{tikzpicture}[line cap=round,line join=round,>=triangle 45,x=1.0cm,y=1.0cm]
\clip(-5.02,-1.44) rectangle (2.16,3.86);
\fill[line width=1.pt,color=xfqqff,fill=xfqqff,fill opacity=0.10000000149011612] (1.32,-0.64) -- (-3.84,2.7) -- (-4.36,-0.16) -- cycle;
\draw[line width=1.pt,fill=black,fill opacity=0.10000000149011612] (-3.2579954901272354,2.323276150586234) -- (-3.488534723940048,1.9671137414622484) -- (-3.1323723148160623,1.7365745076494357) -- (-2.9018330810032498,2.0927369167734216) -- cycle; 
\draw [line width=1.pt,color=xfqqff] (1.32,-0.64)-- (-3.84,2.7);
\draw [line width=1.pt,color=xfqqff] (-3.84,2.7)-- (-4.36,-0.16);
\draw [line width=1.pt,color=xfqqff] (-4.36,-0.16)-- (1.32,-0.64);
\draw [line width=1.pt,domain=-5.02:2.16] plot(\x,{(-1.1064--3.34*\x)/-5.16});
\draw [line width=1.pt,domain=-5.02:2.16] plot(\x,{(-21.9632-5.16*\x)/-3.34});
\begin{scriptsize}
\draw [color=xfqqff] (1.32,-0.64)-- ++(-2.0pt,0 pt) -- ++(4.0pt,0 pt) ++(-2.0pt,-2.0pt) -- ++(0 pt,4.0pt);
\draw[color=xfqqff] (1.12,-0.99) node {$A$};
\draw [color=xfqqff] (-3.84,2.7)-- ++(-2.0pt,0 pt) -- ++(4.0pt,0 pt) ++(-2.0pt,-2.0pt) -- ++(0 pt,4.0pt);
\draw[color=xfqqff] (-3.74,2.99) node {$B$};
\draw [color=xfqqff] (-4.36,-0.16)-- ++(-2.0pt,0 pt) -- ++(4.0pt,0 pt) ++(-2.0pt,-2.0pt) -- ++(0 pt,4.0pt);
\draw[color=xfqqff] (-4.36,-0.43) node {$C$};
\draw[color=xfqqff] (-2.,-0.65) node {$4\,cm$};
\draw [color=xfqqff] (-10.380907275320972,6.933843081312412)-- ++(-2.0pt,0 pt) -- ++(4.0pt,0 pt) ++(-2.0pt,-2.0pt) -- ++(0 pt,4.0pt);
\draw[color=xfqqff] (4.5,4.75) node {$D$};
\draw [color=xfqqff] (-2.9018330810032498,2.0927369167734216)-- ++(-2.0pt,0 pt) -- ++(4.0pt,0 pt) ++(-2.0pt,-2.0pt) -- ++(0 pt,4.0pt);
\draw[color=xfqqff] (-2.42,2.17) node {$H$};
\end{scriptsize}
\end{tikzpicture}

\end{minipage}
 
 \ExoCu{Calculer.}
 
\begin{minipage}{0.64\linewidth}
Calcule l'aire du disque de diamètre $6\,cm$ ci-contre. \point{5}
\end{minipage}
\hfill
\begin{minipage}{0.32\linewidth}


 
\definecolor{xfqqff}{rgb}{0.4980392156862745,0.,1.}
\begin{tikzpicture}[line cap=round,line join=round,>=triangle 45,x=1cm,y=1cm]
\clip(-4.94,-1.34) rectangle (0.06,3.66);
\draw [line width=1.pt,color=xfqqff,fill=xfqqff,fill opacity=0.07000000029802322] (-2.5,1.) circle (1.6cm);
\begin{scriptsize}
\draw [color=xfqqff] (-2.5,1.)-- ++(-2.5pt,0 pt) -- ++(5.0pt,0 pt) ++(-2.5pt,-2.5pt) -- ++(0 pt,5.0pt);
\draw[color=xfqqff] (-2.36,1.37) node {$A$};
\end{scriptsize}
\end{tikzpicture}
\end{minipage}


 
 
 
\ExoCu{Représenter. Calculer.}

\begin{minipage}{0.48\linewidth}
Calcule l'aire du carré $COTE$ de coté $c$ de longueur $6\,cm$. \point{4}
\end{minipage}
\hfill
\begin{minipage}{0.48\linewidth}
Calcule l'aire du rectangle $ABCD$ de longueur $12\,cm$ et de largeur $7,5\,cm$. \point{4}
\end{minipage}

  




\end{pageParcoursu}

%%%%%%%%%%%%%%%%%%%%%%%%%%%%%%%%%%%%%%%%%%%%%%%%%%%%%%%%%%%%%%%%%%%%%%%%%%%%%%%%%%%%%%%%%%%%%%%%%%%%%
%%%%%%%%%% Parcours 2
%%%%%%%%%%%%%%%%%%%%%%%%%%%%%%%%%%%%%%%%%%%%%%%%%%%%%%%%%%%%%%%%%%%%%%%%%%%%%%%%%%%%%%%%%%%%%%%%%%%%%

\begin{pageParcoursd} 



\ExoCd{Représenter.}
 

 
\begin{minipage}{0.58\linewidth}
Calcule l'aire du triangle $ABC$ ci-contre. \point{6}
\end{minipage}
\hfill
\begin{minipage}{0.48\linewidth}


\definecolor{xfqqff}{rgb}{0.4980392156862745,0.,1.}
\begin{tikzpicture}[line cap=round,line join=round,>=triangle 45,x=0.8cm,y=0.8cm]
\clip(-5.38,-1.8) rectangle (2.9,2.44);
\fill[line width=1.pt,color=xfqqff,fill=xfqqff,fill opacity=0.10000000149011612] (-4.92,0.46) -- (-1.5,-0.5) -- (1.5,1.) -- cycle;
\draw[line width=1.pt,fill=black,fill opacity=0.10000000149011612] (1.2990613983232864,-1.2857014451433786) -- (1.4137214739944826,-0.8772249255647429) -- (1.0052449544158468,-0.7625648498935468) -- (0.8905848787446506,-1.1710413694721826) -- cycle; 
\draw [line width=1.pt,color=xfqqff] (-4.92,0.46)-- (-1.5,-0.5);
\draw [line width=1.pt,color=xfqqff] (-1.5,-0.5)-- (1.5,1.);
\draw [line width=1.pt,color=xfqqff] (1.5,1.)-- (-4.92,0.46);
\draw [line width=1.pt,domain=-5.38:2.6] plot(\x,{(-3.15-0.96*\x)/3.42});
\draw [line width=1.pt,domain=-5.38:2.6] plot(\x,{(-4.17--3.42*\x)/0.96});
\begin{scriptsize}
\draw [color=xfqqff] (-4.92,0.46)-- ++(-2.0pt,0 pt) -- ++(4.0pt,0 pt) ++(-2.0pt,-2.0pt) -- ++(0 pt,4.0pt);
\draw[color=xfqqff] (-5.12,0.31) node {$A$};
\draw [color=xfqqff] (-1.5,-0.5)-- ++(-2.0pt,0 pt) -- ++(4.0pt,0 pt) ++(-2.0pt,-2.0pt) -- ++(0 pt,4.0pt);
\draw[color=xfqqff] (-1.4,-0.77) node {$B$};
\draw [color=xfqqff] (1.5,1.)-- ++(-2.0pt,0 pt) -- ++(4.0pt,0 pt) ++(-2.0pt,-2.0pt) -- ++(0 pt,4.0pt);
\draw[color=xfqqff] (1.84,1.33) node {$C$};
\draw[color=xfqqff] (-3.72,-0.37) node {$6\,cm$};
\draw[color=xfqqff] (-2.18,1.11) node {$10\,cm$};
\draw [color=xfqqff] (0.8905848787446506,-1.1710413694721826)-- ++(-2.0pt,0 pt) -- ++(4.0pt,0 pt) ++(-2.0pt,-2.0pt) -- ++(0 pt,4.0pt);
\draw[color=xfqqff] (1.06,-1.47) node {$H$};
\draw[color=black] (2.02,-0.19) node {$HC = 4cm$};
\end{scriptsize}
\end{tikzpicture}

\end{minipage}
 


\ExoCd{Représenter.}

\begin{minipage}{0.68\linewidth}
Calcule l'aire de la surface composée d'un rectangle $ABCD$ et d'un triangle $BEF$ ci-contre. \point{5}
\end{minipage}
\hfill
\begin{minipage}{0.68\linewidth}

\definecolor{xfqqff}{rgb}{0.4980392156862745,0.,1.}
\begin{tikzpicture}[line cap=round,line join=round,>=triangle 45,x=0.7cm,y=0.7cm]
\clip(-5.,-1.8) rectangle (1.44,4.5);
\fill[line width=1.pt,color=xfqqff,fill=xfqqff,fill opacity=0.10000000149011612] (-3.,-1.5) -- (-0.5,-1.) -- (-1.5,4.) -- (-4.,3.5) -- cycle;
\fill[line width=1.pt,color=xfqqff,fill=xfqqff,fill opacity=0.10000000149011612] (-0.5,-1.) -- (1.1115384615384616,-0.6776923076923077) -- (-1.,1.5) -- cycle;
\draw[line width=1.pt,fill=black,fill opacity=0.10000000149011612] (-3.9167949705662157,3.083974852831078) -- (-3.5007698233972935,3.1671798822648625) -- (-3.583974852831078,3.5832050294337843) -- (-4.,3.5) -- cycle; 
\draw[line width=1.pt,fill=black,fill opacity=0.10000000149011612] (-2.583974852831078,-1.4167949705662155) -- (-2.6671798822648625,-1.0007698233972937) -- (-3.0832050294337843,-1.0839748528310782) -- (-3.,-1.5) -- cycle; 
\draw[line width=1.pt,fill=black,fill opacity=0.10000000149011612] (-1.916025147168922,3.9167949705662157) -- (-1.8328201177351375,3.5007698233972935) -- (-1.4167949705662157,3.583974852831078) -- (-1.5,4.) -- cycle; 
\draw [line width=1.pt,color=xfqqff] (-3.,-1.5)-- (-0.5,-1.);
\draw [line width=1.pt,color=xfqqff] (-1.5,4.)-- (-4.,3.5);
\draw [line width=1.pt,color=xfqqff] (-4.,3.5)-- (-3.,-1.5);
\draw [line width=1.pt,color=xfqqff] (-0.5,-1.)-- (1.1115384615384616,-0.6776923076923077);
\draw [line width=1.pt,color=xfqqff] (1.1115384615384616,-0.6776923076923077)-- (-1.,1.5);
\draw [line width=1.pt,color=xfqqff] (-1.5,4.)-- (-1.,1.5);
\begin{scriptsize}
\draw [color=xfqqff] (-3.,-1.5)-- ++(-2.5pt,0 pt) -- ++(5.0pt,0 pt) ++(-2.5pt,-2.5pt) -- ++(0 pt,5.0pt);
\draw[color=xfqqff] (-2.86,-1.73) node {$A$};
\draw [color=xfqqff] (-0.5,-1.)-- ++(-2.5pt,0 pt) -- ++(5.0pt,0 pt) ++(-2.5pt,-2.5pt) -- ++(0 pt,5.0pt);
\draw[color=xfqqff] (-0.36,-0.63) node {$B$};
\draw [color=xfqqff] (-1.5,4.)-- ++(-2.5pt,0 pt) -- ++(5.0pt,0 pt) ++(-2.5pt,-2.5pt) -- ++(0 pt,5.0pt);
\draw[color=xfqqff] (-1.36,4.37) node {$C$};
\draw [color=xfqqff] (-4.,3.5)-- ++(-2.5pt,0 pt) -- ++(5.0pt,0 pt) ++(-2.5pt,-2.5pt) -- ++(0 pt,5.0pt);
\draw[color=xfqqff] (-3.86,3.87) node {$D$};
\draw[color=xfqqff] (-1.64,-1.59) node {$5\,cm$};
\draw[color=xfqqff] (-2.76,4.25) node {$5\,cm$};
\draw[color=xfqqff] (-4.14,1.03) node {$9\,cm$};
\draw [color=xfqqff] (1.1115384615384616,-0.6776923076923077)-- ++(-2.5pt,0 pt) -- ++(5.0pt,0 pt) ++(-2.5pt,-2.5pt) -- ++(0 pt,5.0pt);
\draw[color=xfqqff] (1.26,-0.31) node {$E$};
\draw [color=xfqqff] (-1.,1.5)-- ++(-2.5pt,0 pt) -- ++(5.0pt,0 pt) ++(-2.5pt,-2.5pt) -- ++(0 pt,5.0pt);
\draw[color=xfqqff] (-0.86,1.87) node {$F$};
\draw[color=xfqqff] (0.47,-1.2) node {$2,5\,cm$};
\draw[color=xfqqff] (-0.3,2.87) node {$5,3\,cm$};
\end{scriptsize}
\end{tikzpicture} 

\end{minipage}


\ExoCd{Représenter.}

Convertis les aires suivantes ou complète la bonne unité:

\begin{minipage}{0.28\linewidth}
$1,5\;  km^2 = \ldots\ldots\ldots\;  m^2$ \vspace{0.4cm}
 
$10\;  m^2  = \ldots\ldots\ldots\;  dam^2$\vspace{0.4cm}
 
$45\;  cm^2  =  \ldots\ldots\ldots\;   m^2$
 
\end{minipage}
\hfill
\begin{minipage}{0.28\linewidth}
 
$25\;  mm^2  =  \ldots\ldots\ldots \;  cm^2$\vspace{0.4cm}
 
$3,12\;  dm^2  = \ldots\ldots\ldots  \;  cm^2$\vspace{0.4cm}

$14,3\;  m^2  =   \ldots\ldots\ldots \; dm^2$
 \end{minipage}
\hfill
\begin{minipage}{0.28\linewidth}
 
$124 \; m^2  = 1,24 \;  \ldots \ldots$\vspace{0.4cm}
 
$67 \; hm^2  = \np{0,67}\;  \ldots \ldots $\vspace{0.4cm}

$6,23\;  m^2  = \np{62300}\;  \ldots \ldots  $
 \end{minipage}


\ExoCd{Calculer.}

\begin{minipage}{0.68\linewidth}
Détermine  l'aire de la surface bleue ci-contre. \point{4}
\end{minipage}
\hfill
\begin{minipage}{0.28\linewidth}

\definecolor{wqwqwq}{rgb}{0.3764705882352941,0.3764705882352941,0.3764705882352941}
\definecolor{ududff}{rgb}{0.30196078431372547,0.30196078431372547,1.}
\definecolor{ffffff}{rgb}{1.,1.,1.}
\definecolor{wwzzff}{rgb}{0.4,0.6,1.}
\definecolor{yqyqyq}{rgb}{0.5019607843137255,0.5019607843137255,0.5019607843137255}
\begin{tikzpicture}[line cap=round,line join=round,>=triangle 45,x=0.7cm,y=0.7cm]
\clip(-2.51,-0.86) rectangle (1.74,3.52);
\draw [line width=1.pt,color=wwzzff,fill=wwzzff,fill opacity=0.5] (-0.5,1.5) circle (1.4cm);
\draw [line width=1.pt,color=ffffff,fill=ffffff,fill opacity=1.0] (0.5,1.5) circle (0.7cm);
\draw [line width=1.pt,color=ffffff,fill=ffffff,fill opacity=1.0] (-1.5,1.5) circle (0.7cm);
\draw [line width=1.pt,color=wqwqwq] (-0.5,1.5)-- (0.5,1.5);
\draw [line width=1.pt,color=wqwqwq] (0.,1.62) -- (0.,1.38);
\draw [line width=1.pt,color=wqwqwq] (-0.5,1.5)-- (-1.5,1.5);
\draw [line width=1.pt,color=wqwqwq] (-1.,1.38) -- (-1.,1.62);
\begin{scriptsize}
\draw [color=yqyqyq] (-0.5,1.5)-- ++(-1.5pt,0 pt) -- ++(3.0pt,0 pt) ++(-1.5pt,-1.5pt) -- ++(0 pt,3.0pt);
\draw[color=yqyqyq] (-0.36,1.79) node {$A$};
\draw [color=yqyqyq] (0.5,1.5)-- ++(-1.5pt,0 pt) -- ++(3.0pt,0 pt) ++(-1.5pt,-1.5pt) -- ++(0 pt,3.0pt);
\draw[color=yqyqyq] (0.62,1.87) node {$C$};
\draw [color=ududff] (-1.5,1.5)-- ++(-1.5pt,0 pt) -- ++(3.0pt,0 pt) ++(-1.5pt,-1.5pt) -- ++(0 pt,3.0pt);
\draw[color=ududff] (-1.74,1.83) node {$D$};
\draw[color=wqwqwq] (0.1,1.21) node {$4\,cm$};
\end{scriptsize}
\end{tikzpicture}

\end{minipage}


\ExoCd{Représenter. Calculer.}

Le triangle $ABC$ a une hauteur de $10\,mm$ et une aire de $85\,mm^2$. Calcule la longueur de sa base.\point{4}
 

\end{pageParcoursd}

%%%%%%%%%%%%%%%%%%%%%%%%%%%%%%%%%%%%%%%%%%%%%%%%%%%%%%%%%%%%%%%%%%%%%%%%%%%%%%%%%%%%%%%%%%%%%%%%%%%%%
%%%%%%%%%% Parcours 3
%%%%%%%%%%%%%%%%%%%%%%%%%%%%%%%%%%%%%%%%%%%%%%%%%%%%%%%%%%%%%%%%%%%%%%%%%%%%%%%%%%%%%%%%%%%%%%%%%%%%%
\begin{pageParcourst}


\ExoCt{Représenter. Calculer.}

\begin{enumerate}
\item Un rectangle a une largeur de $6\,km$ et une aire de $9\,km^2$. Calcule sa longueur.\point{4}
\item Un rectangle a une longueur de $17\,dam$ et une largeur de $9\,dam$. Calcule son aire.\point{4}
\end{enumerate}
 

 

\ExoCt{Représenter. Calculer.}

Un disque $\mathcal{D}$ a une aire de $28,26\,dm^2$. Calcule son rayon $r$.\point{5}
 


\ExoCt{Représenter. Calculer.}

\begin{minipage}{0.58\linewidth}

Calcule l'aire de la partie rouge. \point{7}
\end{minipage}
\hfill
\begin{minipage}{0.38\linewidth}

 
\definecolor{aqaqaq}{rgb}{0.6274509803921569,0.6274509803921569,0.6274509803921569}
\definecolor{ffffff}{rgb}{1.,1.,1.}
\definecolor{ffqqtt}{rgb}{1.,0.,0.2}
\definecolor{ffqqqq}{rgb}{1.,0.,0.}
\begin{tikzpicture}[line cap=round,line join=round,>=triangle 45,x=1.0cm,y=1.0cm]
\clip(-4.18,1.56) rectangle (-0.68,4.3);
\draw [shift={(-2.5,2.5)},line width=1.pt,color=ffqqtt,fill=ffqqtt,fill opacity=0.25]  plot[domain=0.:3.141592653589793,variable=\t]({1.*1.5*cos(\t r)+0.*1.5*sin(\t r)},{0.*1.5*cos(\t r)+1.*1.5*sin(\t r)});
\draw [shift={(-3.25,2.5)},line width=1.pt,color=ffffff,fill=ffffff,fill opacity=1.0]  plot[domain=0.:3.141592653589793,variable=\t]({1.*0.75*cos(\t r)+0.*0.75*sin(\t r)},{0.*0.75*cos(\t r)+1.*0.75*sin(\t r)});
\draw [shift={(-1.75,2.5)},line width=1.pt,color=ffqqtt,fill=ffqqtt,fill opacity=0.25]  plot[domain=3.141592653589793:6.283185307179586,variable=\t]({1.*0.75*cos(\t r)+0.*0.75*sin(\t r)},{0.*0.75*cos(\t r)+1.*0.75*sin(\t r)});
\draw [line width=1.pt,color=aqaqaq] (-4.,2.5)-- (-2.5,2.5);
\draw [line width=1.pt,color=aqaqaq] (-3.25,2.62) -- (-3.25,2.38);
\draw [line width=1.pt,color=aqaqaq] (-2.5,2.5)-- (-1.,2.5);
\draw [line width=1.pt,color=aqaqaq] (-1.75,2.62) -- (-1.75,2.38);
\draw [shift={(-3.25,2.5)},line width=1.pt,color=ffqqqq]  plot[domain=0.:3.141592653589793,variable=\t]({1.*0.75*cos(\t r)+0.*0.75*sin(\t r)},{0.*0.75*cos(\t r)+1.*0.75*sin(\t r)});
\begin{scriptsize}
\draw [color=ffqqqq] (-4.,2.5)-- ++(-2.5pt,0 pt) -- ++(5.0pt,0 pt) ++(-2.5pt,-2.5pt) -- ++(0 pt,5.0pt);
\draw[color=ffqqqq] (-3.86,2.07) node {$A$};
\draw [color=ffqqqq] (-1.,2.5)-- ++(-2.5pt,0 pt) -- ++(5.0pt,0 pt) ++(-2.5pt,-2.5pt) -- ++(0 pt,5.0pt);
\draw[color=ffqqqq] (-0.86,2.87) node {$B$};
\draw[color=ffqqtt] (-2.34,4.31) node {$c$};
\draw [color=ffqqqq] (-2.5,2.5)-- ++(-2.5pt,0 pt) -- ++(5.0pt,0 pt) ++(-2.5pt,-2.5pt) -- ++(0 pt,5.0pt);
\draw[color=ffqqqq] (-2.36,2.87) node {$C$};
\draw[color=aqaqaq] (-3.34,2.21) node {$4\,m$};
\end{scriptsize}
\end{tikzpicture}
\end{minipage}

\ExoCt{Représenter. Calculer.}

\begin{minipage}{0.58\linewidth}
On donne les dimensions de la figure ci-contre : $AE=4\,cm$,  $BC=5\,cm$ et  $AC=9\,cm$   

Calcule la longueur $DB$. \point{7}
\end{minipage}
\hfill
\begin{minipage}{0.38\linewidth}

 
 \definecolor{xdxdff}{rgb}{0.49019607843137253,0.49019607843137253,1.}
\definecolor{uuuuuu}{rgb}{0.26666666666666666,0.26666666666666666,0.26666666666666666}
\definecolor{xfqqff}{rgb}{0.4980392156862745,0.,1.}
\definecolor{ududff}{rgb}{0.30196078431372547,0.30196078431372547,1.}
\begin{tikzpicture}[line cap=round,line join=round,>=triangle 45,x=1.0cm,y=1.0cm]
\clip(-5.04,0.82) rectangle (0.22,3.94);
\fill[line width=1.pt,color=xfqqff,fill=xfqqff,fill opacity=0.10000000149011612] (-1.,3.5) -- (-2.,1.5) -- (-4.5,1.5) -- cycle;
\draw[line width=1.pt,fill=black,fill opacity=0.10000000149011612] (-0.5757359312880714,1.5) -- (-0.5757359312880713,1.9242640687119286) -- (-1.,1.9242640687119286) -- (-1.,1.5) -- cycle; 
\draw[line width=1.pt,fill=black,fill opacity=0.10000000149011612] (-2.4048906907509284,2.2085587088141247) -- (-2.036526322641976,2.4190526334478117) -- (-2.247020247275663,2.787417001556764) -- (-2.6153846153846154,2.576923076923077) -- cycle; 
\draw [line width=1.pt,color=xfqqff] (-1.,3.5)-- (-2.,1.5);
\draw [line width=1.pt,color=xfqqff] (-1.,3.5)-- (-2.,1.5);
\draw [line width=1.pt,color=xfqqff] (-2.,1.5)-- (-4.5,1.5);
\draw [line width=1.pt,color=xfqqff] (-4.5,1.5)-- (-1.,3.5);
\draw [line width=1.pt] (-1.,0.82) -- (-1.,3.94);
\draw [line width=1.pt,domain=-5.04:0.22] plot(\x,{(--4.--3.5*\x)/-2.});
\draw [line width=1.pt,domain=-5.04:0.22] plot(\x,{(--3.75-0.*\x)/2.5});
\begin{scriptsize}
\draw[color=ududff] (-0.86,3.7) node {$A$};
\draw[color=ududff] (-2.1,1.31) node {$B$};
\draw[color=ududff] (-4.56,1.25) node {$C$};
\draw[color=uuuuuu] (-2.58,2.91) node {$D$};
\draw[color=xdxdff] (-0.92,1.27) node {$E$};

\end{scriptsize}
\end{tikzpicture}
 
 
\end{minipage}



\end{pageParcourst}



\begin{pageAuto} 

\ExoAuto

\begin{minipage}{0.58\linewidth}
$BCDE$ est un carré de coté de longueur $5\,cm$. 
Calcule l'aire de la figure proposée. \point{7}
\end{minipage}
\hfill
\begin{minipage}{0.38\linewidth}

\definecolor{xdxdff}{rgb}{0.49019607843137253,0.49019607843137253,1.}
\definecolor{uuuuuu}{rgb}{0.26666666666666666,0.26666666666666666,0.26666666666666666}
\definecolor{xfqqff}{rgb}{0.4980392156862745,0.,1.}
\definecolor{ududff}{rgb}{0.30196078431372547,0.30196078431372547,1.}
\definecolor{cqcqcq}{rgb}{0.7529411764705882,0.7529411764705882,0.7529411764705882}
\begin{tikzpicture}[line cap=round,line join=round,>=triangle 45,x=1.0cm,y=1.0cm]
\clip(-5.24,-1.34) rectangle (-0.88,3.2);
\fill[line width=1.pt,color=xfqqff,fill=xfqqff,fill opacity=0.10000000149011612] (-2.1,1.98) -- (-4.5,1.5) -- (-4.02,-0.9) -- (-1.62,-0.42) -- cycle;
\draw[line width=1.pt,fill=black,fill opacity=0.10000000149011612] (-4.288277308082134,2.1992725488213662) -- (-3.8822175507921433,2.0763275468505906) -- (-3.759272548821367,2.4823873041405813) -- (-4.165332306111358,2.605332306111357) -- cycle; 
\fill[line width=1.pt,color=xfqqff,fill=xfqqff,fill opacity=0.10000000149011612] (-4.165332306111358,2.605332306111357) -- (-2.1,1.98) -- (-4.5,1.5) -- cycle;
\draw [line width=1.pt,color=xfqqff] (-2.1,1.98)-- (-4.5,1.5);
\draw [line width=1.pt,color=xfqqff] (-4.5,1.5)-- (-4.02,-0.9);
\draw [line width=1.pt,color=xfqqff] (-4.02,-0.9)-- (-1.62,-0.42);
\draw [line width=1.pt,color=xfqqff] (-1.62,-0.42)-- (-2.1,1.98);
\draw [line width=1.pt,color=xfqqff] (-4.165332306111358,2.605332306111357)-- (-4.5,1.5);
\draw [line width=1.pt,color=xfqqff] (-4.165332306111358,2.605332306111357)-- (-2.1,1.98);
\draw [line width=1.pt,color=xfqqff] (-2.1,1.98)-- (-4.5,1.5);
\draw [line width=1.pt,color=xfqqff] (-4.5,1.5)-- (-4.165332306111358,2.605332306111357);
\begin{scriptsize}
\draw[color=ududff] (-2.04,2.31) node {$B$};
\draw[color=ududff] (-4.78,1.79) node {$C$};
\draw[color=uuuuuu] (-4.4,-1.07) node {$D$};
\draw[color=uuuuuu] (-1.34,-0.51) node {$E$};
\draw[color=xdxdff] (-4.02,2.97) node {$A$};
\draw[color=black] (-2.78,2.65) node {$4\,cm$};
\draw[color=xfqqff] (-4.82,2.27) node {$3\,cm$};
\end{scriptsize}
\end{tikzpicture}


\end{minipage}

 
 \ExoAuto
 

Convertis les aires suivantes ou complète la bonne unité:\vspace{0.4cm}

\begin{minipage}{0.25\linewidth}
$1,2\; dm^2 =  \ldots\ldots\ldots \;m^2$ \vspace{0.4cm}
 
$360\; m^2  =\ldots\ldots\ldots \;dam^2$\vspace{0.4cm}
 
$45\; dam^2  = \ldots\ldots\ldots  \;dm^2$
 
\end{minipage}
\hfill
\begin{minipage}{0.33\linewidth}
 
$2,56 \;dm^2  = \ldots \ldots\ldots\ldots\;  mm^2$\vspace{0.4cm}
 
$4,302\; hm^2  = \ldots \ldots\ldots\ldots  \; dam^2$\vspace{0.4cm}

$83,5\; cm^2  =  \ldots \ldots\ldots\ldots\; m^2$
 \end{minipage}
\hfill
\begin{minipage}{0.28\linewidth}
 
$14\; m^2  = 140 \; \ldots \ldots$\vspace{0.4cm}
 
$1,9 \;dam^2  = \np{0,0019}\; \ldots \ldots $\vspace{0.4cm}

$53400 \;m^2  = \np{5,34}\; \ldots \ldots  $
 \end{minipage}


   
\ExoAuto
 

\begin{minipage}{0.58\linewidth}

Les triangles $PAS$ et $PRC$ sont deux triangles rectangles respectivement en $A$ et en $R$. On donne les dimensions suivantes : 
$PA =30 \; m$ ; $AS = 18 \; m$ , $AR = 5\; m$ et $RC = 6\; m$ 

Calcule l'aire de la partie rouge. \point{7}
\end{minipage}
\hfill
\begin{minipage}{0.38\linewidth}

\definecolor{dcrutc}{rgb}{0.8627450980392157,0.0784313725490196,0.23529411764705882}
\definecolor{uuuuuu}{rgb}{0.26666666666666666,0.26666666666666666,0.26666666666666666}
\begin{tikzpicture}[line cap=round,line join=round,>=triangle 45,x=0.7cm,y=0.7cm]
\clip(-4.2,-2.36) rectangle (1.76,3.62);
\fill[line width=1.pt,color=dcrutc,fill=dcrutc,fill opacity=0.25] (-1.5,-2.) -- (1.,-1.) -- (-0.5,0.33333333333333326) -- (-2.1666666666666665,-0.3333333333333334) -- cycle;
\draw [line width=1.pt] (-3.5,3.)-- (-1.5,-2.);
\draw [line width=1.pt] (-1.5,-2.)-- (1.,-1.);
\draw [line width=1.pt] (1.,-1.)-- (-3.5,3.);
\draw [line width=1.pt,color=dcrutc] (-1.5,-2.)-- (1.,-1.);
\draw [line width=1.pt,color=dcrutc] (1.,-1.)-- (-0.5,0.33333333333333326);
\draw [line width=1.pt,color=dcrutc] (-0.5,0.33333333333333326)-- (-2.1666666666666665,-0.3333333333333334);
\draw [line width=1.pt,color=dcrutc] (-2.1666666666666665,-0.3333333333333334)-- (-1.5,-2.);
\begin{scriptsize}
\draw [color=black] (-3.5,3.)-- ++(-2.5pt,0 pt) -- ++(5.0pt,0 pt) ++(-2.5pt,-2.5pt) -- ++(0 pt,5.0pt);
\draw[color=black] (-3.8,3.19) node {$P$};
\draw [color=black] (-1.5,-2.)-- ++(-2.5pt,0 pt) -- ++(5.0pt,0 pt) ++(-2.5pt,-2.5pt) -- ++(0 pt,5.0pt);
\draw[color=black] (-1.3,-2.17) node {$R$};
\draw [color=black] (1.,-1.)-- ++(-2.5pt,0 pt) -- ++(5.0pt,0 pt) ++(-2.5pt,-2.5pt) -- ++(0 pt,5.0pt);
\draw[color=black] (1.14,-0.63) node {$C$};
\draw [color=black] (-0.5,0.33333333333333326)-- ++(-2.5pt,0 pt) -- ++(5.0pt,0 pt) ++(-2.5pt,-2.5pt) -- ++(0 pt,5.0pt);
\draw[color=black] (-0.36,0.71) node {$S$};
\draw [color=uuuuuu] (-2.1666666666666665,-0.3333333333333334)-- ++(-2.0pt,0 pt) -- ++(4.0pt,0 pt) ++(-2.0pt,-2.0pt) -- ++(0 pt,4.0pt);
\draw[color=uuuuuu] (-2.52,-0.13) node {$A$};
\end{scriptsize}
\end{tikzpicture}
\end{minipage}



 

\end{pageAuto}


\begin{pageBrouillon} 
 
\ligne{30}

\end{pageBrouillon}
