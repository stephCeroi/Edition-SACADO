%----------------------------------------------------------------------------------------
%	PACKAGES AND OTHER DOCUMENT CONFIGURATIONS
%----------------------------------------------------------------------------------------

%----------------------------------------------------------------------------------------
%		Géometrie de la page
%----------------------------------------------------------------------------------------
\documentclass[dvipsnames,french,10pt]{book}

\usepackage[
paperheight=29.7cm, %hauteur du papier
paperwidth=21cm, %largeur du papier
left=1cm, %marge de gauche
right=1cm, %marge de droite
top=1.5cm, %marge du haut
bottom=1cm, %marge du bas
%marginparsep=0pt, %distance entre le texte et les notes de marges 
reversemp, %inverse l'emplacement de la marge
headheight=20.60pt %hauteur du header
%showframe, %permet d'afficher le cadre défini ci-dessus
%bindingoffset=1cm %permet d'ajouter le décalage dû au reliage
]{geometry} %Redéfinition de la taille des pages
\raggedbottom


%----------------------------------------------------------------------------------------
%		Generals
%----------------------------------------------------------------------------------------
%\usepackage{fourier} %!! A changer plus tard !!
\usepackage[scaled]{uarial}
\renewcommand*\familydefault{\sfdefault} %% Only if the base font of the document is to be sans serif
\usepackage{frcursive}
\usepackage[T1]{fontenc} %Accents handling
\usepackage[utf8]{inputenc} % Use UTF-8 encoding
%\usepackage{microtype} % Slightly tweak font spacing for aesthetics
\usepackage[english, francais]{babel} % Language hyphenation and typographical rules
\usepackage{marginnote}


%----------------------------------------------------------------------------------------
%		Graphics
%----------------------------------------------------------------------------------------
\usepackage{xcolor}
\usepackage{graphicx, multicol} % Enhanced support for graphics
\graphicspath{FIG/}
\usepackage{wrapfig}
\usepackage{colortbl}
\usepackage[framemethod=tikz]{mdframed}
%\usepackage{xsavebox}
% Il faudrait utiliser xsavebox à l'avenir pour réduire la taille du pdf

%----------------------------------------------------------------------------------------
%		Other packages
%----------------------------------------------------------------------------------------
\usepackage{hyperref}
\hypersetup{
	colorlinks=true, %colorise les liens
	breaklinks=true, %permet le retour à la ligne dans les liens trop longs
	urlcolor= sacado_violet,  %couleur des hyperliens et des QR codes
	linkcolor= sacado_violet, %couleur des liens internes
	plainpages=false  %pour palier à "Bookmark problems can occur when you have duplicate page numbers, for example, if you have a page i and a page 1."
}
\usepackage{tabularx}
\newcolumntype{M}[1]{>{\arraybackslash}m{#1}} %Defines a scalable column type in tabular
\usepackage{booktabs} % Enhances quality of tables
\usepackage{diagbox} % barre en diagonale dans un tableau
\usepackage{multicol}
\usepackage[explicit]{titlesec}
\usepackage{xr}
\usepackage{xspace}
\usepackage{array}
\usepackage{listings}
\usepackage{fancyvrb} %verbatim
\usepackage{stmaryrd}
\usepackage{float}



% Python style for highlighting
\lstdefinestyle{mystyle}{
    backgroundcolor=\color{white},   
    commentstyle=\color{sacado_green},
    keywordstyle=\color{sacado_red},
    numberstyle=\tiny\color{sacado_orange},
    stringstyle=\color{sacado_blue},
    basicstyle=\ttfamily\footnotesize,
    breakatwhitespace=false,         
    breaklines=true,                 
    captionpos=b,                    
    keepspaces=false,                 
    numbers=left,                    
    numbersep=5pt,                  
    showspaces=false,                
    showstringspaces=false,
    showtabs=false,                  
    tabsize=4
}

\lstset{style=mystyle}

%----------------------------------------------------------------------------------------
%		Headers and footers
%----------------------------------------------------------------------------------------

\pagestyle{empty}
\usepackage{fancyhdr}
\pagestyle{fancy}
\renewcommand{\headrulewidth}{0pt} % pas de filet sous le header

%----------------------------------------------------------------------------------------
%		Mathematics packages
%----------------------------------------------------------------------------------------
\usepackage{amsthm, amsmath, amssymb, mathrsfs} % Mathematical typesetting
\usepackage{marvosym, wasysym} % More symbols
\usepackage[makeroom]{cancel}
\usepackage{xlop}
\usepackage{pgf,tikz,pgfplots}
\pgfplotsset{compat=1.16}
\usepackage{pgf-pie}
\usetikzlibrary{positioning}
\usetikzlibrary{arrows}
\usepackage{pst-plot,pst-tree,pst-func, pstricks-add,pst-node,pst-text}
%\usepackage{units}
\usepackage{nicefrac}
\usepackage[np]{numprint} %Séparation milliers dans un nombre \np{12345} donne 12 345
\usepackage{multido}
\newcommand{\RNum}[1]{\uppercase\expandafter{\romannumeral #1\relax}}

%----------------------------------------------------------------------------------------
%		New text commands
%----------------------------------------------------------------------------------------
\usepackage{calc}
\usepackage{boites}
 \renewcommand{\arraystretch}{1.6}

%%%%% Pour les imports.
\usepackage{import}

%%%%% Pour faire des boites
\usepackage[tikz]{bclogo}
\usepackage{bclogo}
\usepackage{framed}
\usepackage[skins]{tcolorbox}
\tcbuselibrary{breakable}
\tcbuselibrary{skins}
\usetikzlibrary{quotes,babel,arrows.meta,shadows,decorations.pathmorphing,decorations.markings,patterns}
\usepackage{tikzpagenodes}
\usetikzlibrary{plotmarks}



%%%%% Pour une double minipage
\newcommand{\mini}[4]{
\begin{minipage}[c]{#1}
#2
\end{minipage}
\hfill
\begin{minipage}[c]{#3}
#4
\end{minipage}
}


\usepackage{enumitem}
\newlist{todolist}{itemize}{2} %Pour faire des QCM
\setlist[todolist]{label=$\square$} %Pour faire des QCM \begin{todolist} instead of itemize
\renewcommand{\FrenchLabelItem}{\textbullet} %bullet dans les items


%----------------------------------------------------------------------------------------
%		Définition de couleurs pour ...
%----------------------------------------------------------------------------------------

%GEOGEBRA

\definecolor{zzttqq}{rgb}{0.6,0.2,0.} %rouge des polygones
\definecolor{qqqqff}{rgb}{0.,0.,1.}
\definecolor{xdxdff}{rgb}{0.49019607843137253,0.49019607843137253,1.}%bleu
\definecolor{qqwuqq}{rgb}{0.,0.39215686274509803,0.} %vert des angles
\definecolor{ffqqqq}{rgb}{1.,0.,0.} %rouge vif
\definecolor{uuuuuu}{rgb}{0.26666666666666666,0.26666666666666666,0.26666666666666666}
\definecolor{qqzzqq}{rgb}{0.,0.6,0.}
\definecolor{cqcqcq}{rgb}{0.7529411764705882,0.7529411764705882,0.7529411764705882} %gris
\definecolor{qqffqq}{rgb}{0.,1.,0.}
\definecolor{ffdxqq}{rgb}{1.,0.8431372549019608,0.}
\definecolor{ffffff}{rgb}{1.,1.,1.}
\definecolor{ududff}{rgb}{0.30196078431372547,0.30196078431372547,1.}
\definecolor{ffqqff}{rgb}{1.,0.,1.}
\definecolor{ffxfqq}{rgb}{1,0.4980392156862745,0}
\definecolor{ffffqq}{rgb}{1,1,0}
\definecolor{qqttzz}{rgb}{0,0.2,0.6}
\definecolor{qqccqq}{rgb}{0,0.8,0}
\definecolor{qqzzff}{rgb}{0,0.6,1}
\definecolor{qqwwzz}{rgb}{0,0.4,0.6}
\definecolor{eqeqeq}{rgb}{0.8784313725490196,0.8784313725490196,0.8784313725490196}

%SACADO

\definecolor{fond}{HTML}{5D4391}  %couleur des entetes etc.  violet sacado
\definecolor{sacado_purple}{HTML}{5D4391} %% Violet foncé Sacado
\definecolor{sacado_violet}{HTML}{9274C7} %% Violet clair Sacado
\definecolor{texte}{HTML}{FFFFFF} % couleur du texte des entetes etc.
\definecolor{sacado_blue_light}{HTML}{0093CA} %% Bleu Sacado
\definecolor{sacado_blue}{HTML}{0960B5} %% Bleu Sacado
\definecolor{sacado_green}{HTML}{00B999} %% Vert Sacado
\definecolor{sacado_green_dark}{HTML}{4D8075} %% Vert Sacado foncé
\definecolor{sacado_yellow}{HTML}{F9F871} %% Jaune Sacado
\definecolor{sacado_orange}{HTML}{FF8B69} %% Orange Sacado
\definecolor{sacado_red}{HTML}{9F1E17} %% Rouge Sacado
\definecolor{sacado_gray}{HTML}{7B7485} %% Gris Sacado
%BOITES 

\definecolor{bleu1}{rgb}{0.54,0.79,0.95} %% Bleu
\definecolor{sapgreen}{rgb}{0.4, 0.49, 0}
\definecolor{dvzfxr}{rgb}{0.7,0.4,0.}
\definecolor{beamer}{rgb}{0.5176470588235295,0.49019607843137253,0.32941176470588235} % couleur beamer
\definecolor{preuveRbeamer}{rgb}{0.8,0.4,0}
\definecolor{sectioncolor}{rgb}{0.24,0.21,0.44}
\definecolor{subsectioncolor}{rgb}{0.1,0.21,0.61}
\definecolor{subsubsectioncolor}{rgb}{0.1,0.21,0.61}
\definecolor{info}{rgb}{0.82,0.62,0}
\definecolor{bleu2}{rgb}{0.38,0.56,0.68}
\definecolor{bleu3}{rgb}{0.24,0.34,0.40}
\definecolor{bleu4}{rgb}{0.12,0.20,0.25}
\definecolor{vert}{rgb}{0.21,0.33,0}
\definecolor{vertS}{rgb}{0.05,0.6,0.42}
\definecolor{red}{rgb}{0.78,0,0}
\definecolor{color5}{rgb}{0,0.4,0.58}
\definecolor{eduscol4B}{rgb}{0.19,0.53,0.64}
\definecolor{eduscol4P}{rgb}{0.62,0.12,0.39}
\definecolor{ill_frame}{HTML}{F0F0F0} %Boite illustration contour
\definecolor{ill_back}{HTML}{F7F7F7}  %Boite illustration background
\definecolor{ill_title}{HTML}{AAAAAA} %Boite illustration titre

%----------------------------------------------------------------------------------------
%		QR codes
%----------------------------------------------------------------------------------------

\usepackage[
final %Pour la compilation finale
%draft %Pour le travail sur les documents
]{qrcode}
\usepackage{fontawesome}
\usepackage{fancyqr}
\FancyQrLoad{flat}
\fancyqrset{
%image=\scalebox{.8}{\includegraphics[scale=1]{sacadoA1.png}},image padding=.5,
l color=sacado_green,r color=sacado_blue}
\newcommand{\qr}[2]{\centering \fancyqr{https://sacado.xyz/qcm/show_course_from_qrcode/#1}

\vspace{.2cm}

#2} %\qr{id} Pour obtenir un qrcode en indiquant seulement l'id de l'exercice





\newcommand{\miniqr}[3]{
\begin{minipage}[c]{.8\linewidth}
#1
\end{minipage}
\hfill
\fbox{
\begin{minipage}[c]{.18\linewidth}
\begin{center}
\fancyqr{https://sacado.xyz/qcm/show\_course\_from\_qrcode/#2}

\vspace{.2cm}

#3
\end{center}
\end{minipage}
}
}

%practice/frombook/<int:ide>/ pour accéder à un exercice depuis le livre.

\usepackage{pythontex}
\begin{pycode}
import qrcode
def qr(data):
     fic=r"QRcodes/qr"+data+'.png'
     urlcourte=r"sacado.xyz/"+data
     urllongue=r"https://"+urlcourte
     qr = qrcode.QRCode(version=1,
error_correction=qrcode.constants.ERROR_CORRECT_L,box_size=2, border=0)
     qr.add_data(urlcourte)
     qr.make(fit=True)
     qr_image = qr.make_image(fill_color="black", back_color="white")
     qr_image.save(fic)
     return r"""\parbox{3.5cm}{\begin{center}
\includegraphics{"""+fic+r"}\\{\scriptsize\tt "+urlcourte+r"}\end{center}}"
\end{pycode}

%\renewcommand{\qr}[1]{\py{qr("#1")}} % compilation complete
%utiliser :
%  pdflatex --shell-escape MANUEL_6e_V1.tex ; pythontex MANUEL_6e_V1.tex --interpreter python:python3 ; pdflatex --shell-escape % %MANUEL_6e_V1.tex

%  draft
\renewcommand{\qr}[1]{
\parbox{5cm}{\begin{center}
       \includegraphics{QRcodes/qrDummy.png}
       %\\{\tt dummy}
\end{center}}
}

\renewcommand{\miniqr}[1]{
       \includegraphics[height=1cm]{QRcodes/qrDummy.png}
       %\\{\tt dummy}}
}


\usepackage[absolute]{textpos}
\newcommand{\qrHautDroite}[1]{
\setlength{\TPHorizModule}{1cm}
\setlength{\TPVertModule}{1cm}
\begin{textblock}{3.51}(12.5,1.5){\qr{#1}}
\end{textblock}
}

 






\usepackage{makeidx}
\makeindex

%----------------------------------------
%
%   Définitions des environnements "pageCours" et "pageExos"
%
%----------------------------------------

\newcounter{cpt}
\newcounter{exo}
\newcounter{cptr}

\newcommand{\titreChap}{Titre de chapitre à définir}

\renewcommand{\chapter}[3]{
  \stepcounter{chapter}
  \setcounter{exo}{0}
  \setcounter{cpt}{0}
  
%\cleardoublepage  % pour commencer à droite
{\Huge \hfill Chapitre \Roman{chapter}.\\
  \bigskip
  #1\\
  \bigskip {\begin{center}
  \fancyqr[image={\includegraphics[scale=.6]{sacadoA1.png}},image padding=.5,height=5cm]{#2}
  \end{center}}  {\normalsize #3}}
\renewcommand{\titreChap}{#1}

%\ifthenelse{\equal{#2}{}}{}{\par
%  \bigskip\bigskip
%  #2}
\newpage
}

\renewenvironment{leftbar}[1][\hsize]
{%
    \def\FrameCommand
    {%
        {\color{black}\vrule width 0.5pt}%
        \hspace{4pt}%must no space.
        \fboxsep=\FrameSep%\colorbox{yellow}%
    }%
    \MakeFramed{\hsize#1\advance\hsize-\width\FrameRestore}%
}
{\endMakeFramed}


\newcommand{\headerGeneral}[3]{ % intitulé, couleur, qrcode
\begin{tikzpicture}[remember picture,overlay]
\coordinate(NO) at (-2,0);
\coordinate(SW) at (22,1);
\coordinate(titre) at (0,0.2);
\coordinate(qr) at (16.85,0.);
\shade[left color=#2 , right color=#2 ] (NO) rectangle (SW);
\draw (titre) node[color=texte, anchor=west]{ {\large \bf  #1} \quad\quad \bf {\small \titreChap} };
\draw (qr) node {\qr{#3}};
\end{tikzpicture}
}

\newenvironment{pageCours}{\lhead{%
\pagecolor{white!100}
\headerGeneral{COURS}{fond!70}{p/1234}
}\begin{leftbar}}{\end{leftbar}\newpage}

\newenvironment{pageAD}{\lhead{%
\pagecolor{sacado_violet!6}
\headerGeneral{APPLICATIONS DIRECTES}{sacado_violet!70}{p/1234}
} }{ \newpage}

\newenvironment{pageParcoursu}{\lhead{%
\pagecolor{sacado_green!6} 
\headerGeneral{PARCOURS 1}{sacado_green}{p/1234}
} }{ \newpage}


\newenvironment{pageParcoursd}{\lhead{%
\pagecolor{sacado_blue!6}
\headerGeneral{PARCOURS 2}{sacado_blue_light}{p/1234}
} }{ \newpage}

\newenvironment{pageParcourst}{\lhead{%
\pagecolor{sacado_red!6}
\headerGeneral{PARCOURS 3}{sacado_red}{p/1234}
} }{ \newpage}

\newenvironment{pageBrouillon}{\lhead{%
\pagecolor{sacado_gray!6}
\headerGeneral{BROUILLON}{sacado_gray}{p/1234}
} }{ \newpage}

\newenvironment{pageRituels}{\lhead{%
\pagecolor{fond!6}
\headerGeneral{RITUELS}{fond!70}{p/1234}
} }{ \newpage}

\newenvironment{pageAuto}{\lhead{%
\pagecolor{sacado_orange!6}
\headerGeneral{AUTOÉVALUATION}{sacado_orange}{p/1234}
} }{ \newpage}

\newenvironment{pageHistoire}{\lhead{%
\pagecolor{olive!6}
\headerGeneral{HISTOIRE}{olive}{p/1234}
} }{ \newpage}



\newenvironment{pageExercices}{\lhead{%
\pagecolor{white!100}
\headerGeneral{ACTIVITÉS}{fond}{p/1234}
}\begin{leftbar}}{\end{leftbar}\newpage}



\fancyfoot[L]{\colorbox{fond!70}{\color{texte}\thepage}}
\fancyfoot[C]{}


\newcommand{\titresec}[2]{\phantom{.}\begin{textblock}{1}[0,1](-1.24,0.25)\colorbox{fond!70}{%
\makebox[0.8cm]{\raisebox{0.05cm}[0.6cm][0.15cm]{\color{texte}\LARGE\bf #1}}}\end{textblock}{\LARGE\bf #2}\\\bigskip}

\renewcommand{\thesection}{\arabic{section}}
\titleformat{\section}{}{%
\hspace{-1.15cm}\colorbox{fond!70}{%
\makebox[0.8cm]{\raisebox{0.05cm}[0.6cm][0.15cm]{\color{texte}\LARGE\bf \thesection}}}}{1em}{\bf \LARGE #1}
  
\renewcommand{\thesubsection}{\arabic{subsection}}
            
\titleformat{\subsection}
{%\begin{textblock}{1}[0,1](-1,0.42) toto
  %\end{textblock}
%\reversemarginpar\marginnote[\rule{0.8cm}{0.8cm}]{}[0pt]  \color{red}\normalfont\Large\bfseries}
}{\hspace{-0.83em}
\colorbox{fond!70}{\makebox[0.6cm]{\raisebox{0cm}[1em][0.2em]\normalfont\large\bfseries\color{texte}\thesubsection}}}{1em}{\bf \large #1}




\makeatletter
\newenvironment{TraitV}[1]{%
% #1 couleur du trait (par défaut CouleurA)
% #2 largeur du trait
% #3 distance entre le trait et le texte
\def\FrameCommand{{\color{#1}\vrule width 2pt}
\hspace{1em}}\MakeFramed {\advance\hsize-\width}}%
{\endMakeFramed}
\makeatother

%----------------------------------------
%
%   Définitions des environnements de Définitions, propriétés...
%
%----------------------------------------

%%%%%%%%%%%%% Définitions
\newenvironment{Def}{%
\medskip \begin{tcolorbox}[widget,colback=sacado_violet!15,colframe=sacado_violet!75!black,
title= \stepcounter{cpt} Définition \thecpt. ]}{%
\end{tcolorbox}\par}


\newenvironment{DefT}[1]{%
\medskip \begin{tcolorbox}[widget,colback=sacado_violet!15,colframe=sacado_violet!75!black,
title= \stepcounter{cpt} Définition \thecpt : #1.]}
{%
\end{tcolorbox}\par}


%%%%%%%%%%%%% Proposition
\newenvironment{Prop}{%
\medskip \begin{tcolorbox}[widget,colback=sacado_blue!15,colframe=sacado_blue!75!black,
title= \stepcounter{cpt} Proposition \thecpt.]}
{%
\end{tcolorbox}\par}


%%%%%%%%%%%%% Propriétés
\newenvironment{Pp}{%
\medskip \begin{tcolorbox}[widget,colback=white!100,colframe=sacado_violet!75!black,
title= \stepcounter{cpt} Propriété \thecpt.]}
{%
\end{tcolorbox}\par}

\newenvironment{PpT}[1]{%
\medskip \begin{tcolorbox}[widget,colback=white!100,colframe=sacado_violet!75!black,
title= \stepcounter{cpt} Propriété \thecpt : #1. ]}
{%
\end{tcolorbox}\par}

\newenvironment{Pps}{%
\medskip \begin{tcolorbox}[widget,colback=white!100,colframe=sacado_violet!75!black,
title= \stepcounter{cpt} Propriétés \thecpt.]}
{%
\end{tcolorbox}\par}


%%%%%%%%%%%%% Conséquence
\newenvironment{Cq}{%
\medskip \begin{tcolorbox}[widget,colback=white,colframe=sacado_blue,
title= \stepcounter{cpt} Conséquence \thecpt.]}
{%
\end{tcolorbox}\par}



%%%%%%%%%%%%% Théorèmes
\newenvironment{ThT}[1]{% théorème avec titre
\medskip \begin{tcolorbox}[widget,colback=white!100,colframe=sacado_violet!75!black,
title= \stepcounter{cpt} Théorème \thecpt : #1.]}
{%
\end{tcolorbox}\par}

\newenvironment{Th}{%
\medskip \begin{tcolorbox}[widget,colback=white!100,colframe=sacado_violet!75!black,
title= \stepcounter{cpt} Théorème \thecpt.]}
{%
\end{tcolorbox}\par}


%%%%%%%%%%%%% Règles
\newenvironment{Reg}{%
\medskip \begin{tcolorbox}[widget,colback=sacado_blue!15,colframe=sacado_blue,
title= \stepcounter{cpt} Règle \thecpt.]}
{%
\end{tcolorbox}\par}

%%%%%%%%%%%%% Représentations
\newenvironment{Rep}{%
\medskip \begin{tcolorbox}[widget,colback=white,colframe=sacado_violet!75!white,
title= \stepcounter{cpt} Représentation \thecpt.]}
{%
\end{tcolorbox}\par}

 
%%%%%%%%%%%%% REMARQUES
\newenvironment{Rq}{%
\medskip \begin{tcolorbox}[widget,colback=sacado_orange!15,colframe=sacado_orange,
title= \stepcounter{cpt} Remarque \thecpt.]}
{%
\end{tcolorbox}\par}

\newenvironment{Rqs}{%
\medskip \begin{tcolorbox}[widget,colback=sacado_orange!15,colframe=sacado_orange,
title= \stepcounter{cpt} Remarques \thecpt.]}
{%
\end{tcolorbox}\par}


%%%%%%%%%%%%% EXEMPLES
\newenvironment{Ex}{%
\medskip \begin{tcolorbox}[widget,colback=white,colframe=sacado_blue_light,
title= \stepcounter{cpt} Exemple \thecpt.]}
{%
\end{tcolorbox}\par}

\newenvironment{Exs}{%
\medskip \begin{tcolorbox}[widget,colback=white!15,colframe=sacado_blue_light,
title= \stepcounter{cpt} Exemples \thecpt.]}
{%
\end{tcolorbox}\par}

 
\newenvironment{ExT}[1]{%
\medskip \begin{tcolorbox}[widget,colback=white,colframe=sacado_blue_light,
title= \stepcounter{cpt} Exemple \thecpt   : #1.]}
{%
\end{tcolorbox}\par}

 
\newenvironment{ExCor}{%
\medskip \begin{tcolorbox}[widget,colback=white,colframe=sacado_blue ,
title= \stepcounter{cpt} Exercice commenté \thecpt.]}
{%
\end{tcolorbox}\par}

\newenvironment{ExQr}[1]{%
\medskip \begin{tcolorbox}[widget,colback=white,colframe=sacado_blue_light ,
title= \stepcounter{cpt} Exemple  \thecpt. \hfill {\color{sacado_blue}https://sacado.xyz/a/#1} ]
\begin{minipage}{1.5cm}
\miniqr{#1}
\end{minipage}
\begin{minipage}{0.8\linewidth}
}
{%
\end{minipage}
\end{tcolorbox}\par}


\newenvironment{OuQr}[1]{%
\medskip \begin{tcolorbox}[widget,colback=white,colframe=sacado_orange ,
title= \stepcounter{cpt} Outil \thecpt. \hfill {\color{sacado_orange}https://sacado.xyz/a/#1} ]
\begin{minipage}{1.5cm}
\miniqr{#1}
\end{minipage}
\begin{minipage}{0.8\linewidth}
 \color{sacado_orange!90!black}
}
{%
 
\end{minipage}
\end{tcolorbox}\par}



\newenvironment{MeQr}[1]{%
\medskip \begin{tcolorbox}[widget,colback=white,colframe=sacado_blue,
title= \stepcounter{cpt} Méthode \thecpt. \hfill {\color{sacado_blue}https://sacado.xyz/a/#1} ]
\begin{minipage}{1.5cm}
\miniqr{#1}
\end{minipage}
\begin{minipage}{0.8\linewidth}
}
{%
\end{minipage}
\end{tcolorbox}\par}




%%%%%%%%%%%%% Logique
\newenvironment{Log}{%
\medskip \begin{tcolorbox}[widget,colback=sacado_blue!10,colframe=sacado_blue,
title= \stepcounter{cpt} Logique mathématique \thecpt.]}
{%
\end{tcolorbox}\par}
%%%%%%%%%%%%% Logique avec paramètre
\newenvironment{LogT}[1]{%
\medskip \begin{tcolorbox}[widget,colback=sacado_blue!10,colframe=sacado_blue,
title= \stepcounter{cpt} Logique mathématique \thecpt. #1]}
{%
\end{tcolorbox}\par}

%%%%%%%%%%%%% Preuve
\newenvironment{Pv}[1][]{%
\begin{tcolorbox}[breakable, enhanced,widget, colback=sacado_blue!10!white,boxrule=0pt,frame hidden,
borderline west={1mm}{0mm}{sacado_blue!75}]
\textbf{Preuve#1 : }}
{%
\end{tcolorbox}
\par}


%%%%%%%%%%%%% PreuveROC
\newenvironment{PvR}[1][]{%
\begin{tcolorbox}[breakable, enhanced,widget, colback=sacado_blue!10!white,boxrule=0pt,frame hidden,
borderline west={1mm}{0mm}{sacado_blue!75}]
\textbf{Preuve (ROC)#1 : }}
{%
\end{tcolorbox}
\par}


%%%%%%%%%%%%% DemoExigible
\newenvironment{DemoE}{%
\medskip \begin{tcolorbox}[widget,colback=sacado_blue!10,colframe=sacado_blue,
title= \stepcounter{cpt} Démonstration exigible \thecpt. ]}
{%
\end{tcolorbox}\par}





%%%%%%%%%%%%% Compétences
\newenvironment{Cps}[1][]{%
\vspace{0.4cm}
\begin{tcolorbox}[enhanced, lifted shadow={0mm}{0mm}{0mm}{0mm}%
{black!60!white}, attach boxed title to top left={xshift=5mm, yshift*=-3mm}, coltitle=white, colback=white, boxed title style={colback=sacado_green!100}, colframe=sacado_green!75!black,title=\textbf{Compétences associées#1}]}
{%
\end{tcolorbox}
\par}

%%%%%%%%%%%%% Chapitres connexes
\newenvironment{CCon}[1][]{%
\vspace{0.4cm}
\begin{tcolorbox}[breakable, enhanced,widget, colback=white ,boxrule=0pt,frame hidden,
borderline west={2mm}{0mm}{sacado_violet}]
\textbf{#1}}
{%
\end{tcolorbox}
\par}
%%%%%%%%%%%%% Compétences Collège
\newenvironment{CpsCol}[1][]{%
\vspace{0.4cm}
\begin{tcolorbox}[breakable, enhanced,widget, colback=white ,boxrule=0pt,frame hidden,
borderline west={2mm}{0mm}{sacado_violet}]
\textbf{#1}}
{%
\end{tcolorbox}
\par}


 

%%%%%%%%%%%%% Rituel
\newenvironment{Rit}{%
\medskip \begin{tcolorbox}[widget,colback=white!15,colframe=sacado_violet!75!black,
title= \stepcounter{cpt} Rituel \thecpt. ]}{%
\end{tcolorbox}\par}


%%%%%%%%%%%%% Méthode
\newenvironment{Mt}{%
\medskip \begin{tcolorbox}[widget,colback=white!15,colframe=sacado_violet!75!black,
title= \stepcounter{cpt} Méthode \thecpt. ]}{%
\end{tcolorbox}\par}

%%%%%%%%%%%%% Méthode
\newenvironment{MtT}[1]{%
\medskip \begin{tcolorbox}[widget,colback=white!15,colframe=sacado_violet!75!black,
title= \stepcounter{cpt} Méthode \thecpt. #1 ]}{%
\end{tcolorbox}\par}


%%%%%%%%%%%%% VocU
\newenvironment{VocU}[1]{%
\medskip \begin{tcolorbox}[widget,colback=white!15,colframe=sacado_violet!75,
title= \stepcounter{cpt} Vocabulaire \thecpt. #1 ]}{%
\end{tcolorbox}\par}


%%%%%%%%%%%%% Notation
\newenvironment{Nt}[1]{%
\medskip \begin{tcolorbox}[widget,colback=white!5,colframe=sacado_red!75,
title= \stepcounter{cptr} Notation \thecptr. #1 ]}{%
\end{tcolorbox}\par}

%%%%%%%%%%%%% Ety
\newenvironment{Ety}[1]{%
\medskip \begin{tcolorbox}[widget,colback=white!15,colframe=sacado_violet!75,
title= \stepcounter{cpt} Étymologie \thecpt. #1 ]}{%
\end{tcolorbox}\par}


%%%%%%%%%%%%% His
\newenvironment{His}[1]{%
\begin{tcolorbox}[right=5mm, enhanced, lifted shadow={0mm}{0mm}{0mm}{0mm}%
{sacado_green_dark!90!white}, attach boxed title to top left={xshift=0.3cm, yshift*=-2mm}, coltitle=sacado_green_dark, colback=sacado_green!10!white, boxed title style={colback=white}, colframe=sacado_green_dark,title= Les mathématiciennes et mathématiciens ]
}{%
\end{tcolorbox}\par}


%%%%%%%%%%%%% Attention
\newenvironment{Att}[1]{%
\medskip \begin{tcolorbox}[widget,colback=sacado_red!5,colframe=sacado_red!95!white,
title= \stepcounter{cpt} Notation \thecpt. #1 ]}{%
\end{tcolorbox}\par}



%%%%%%%%%%%%%%%%%%%%%%%%%%%%%%%%%%%%%%%%%%%%%%%%%%%%%%%%%%%%%%%%%%%%%%%%%%%%%%%%%%%%%%%%%%%%%%%%%%%%%%%%%%%%%%%%%%%%%%
%%%%%%%%%%%%%%%%%%%%%%%%%%%%%%%%%%%%%%%%%%%%%%%%%%%%%%%%%%%%%%%%%%%%%%%%%%%%%%%%%%%%%%%%%%%%%%%%%%%%%%%%%%%%%%%%%%%%%%
%%%%%%%%%%%%%%%%  Exercices                                            %%%%%%%%%%%%%%%%%%%%%%%%%%%%%%%%%%%%%%%%%%%%%%%
%%%%%%%%%%%%%%%%%%%%%%%%%%%%%%%%%%%%%%%%%%%%%%%%%%%%%%%%%%%%%%%%%%%%%%%%%%%%%%%%%%%%%%%%%%%%%%%%%%%%%%%%%%%%%%%%%%%%%%
%%%%%%%%%%%%%%%%%%%%%%%%%%%%%%%%%%%%%%%%%%%%%%%%%%%%%%%%%%%%%%%%%%%%%%%%%%%%%%%%%%%%%%%%%%%%%%%%%%%%%%%%%%%%%%%%%%%%%%
 
 
 
%%%%%%%%%%%%% ExoCad 7 paramètres : Compétences, qrcode , calculatrice, python, scratch, tableur, annales
\newenvironment{ExoCad}[7]{% code avant
\tcbset{top=-0.2cm }
\stepcounter{exo}

\begin{tcolorbox}[right=-5mm, enhanced, lifted shadow={0mm}{0mm}{0mm}{0mm}%
{black!60!white}, attach boxed title to top right={xshift=-0.3cm, yshift*=-2mm}, coltitle=sacado_violet!85!black, colback=white!100!white, boxed title style={colback=white}, colframe=sacado_violet!100!black,title= {\footnotesize  #1}  ]
 
\hspace{-1.3cm} 
\begin{minipage}[t]{0.7cm}

 \begin{tikzpicture}
 	\node[fill=sacado_violet,minimum width=0.7cm]{\textcolor{white}{\bf {\Large \theexo}}};
 \end{tikzpicture}


%%%%%%%%%%%%%%%%%%%%%%%% Condition pour la calculatrice
 \ifthenelse{\equal{#3}{1}}{
 \begin{tikzpicture}
 	\node[minimum width=0.7cm]{\includegraphics[scale=0.5]{MISC/calculator.png} };
 \end{tikzpicture} 
 }{
 \ifthenelse{\equal{#3}{2}}{
 \begin{tikzpicture}
 	\node[minimum width=0.7cm]{\includegraphics[scale=0.5]{MISC/no_calculator.png} };
 \end{tikzpicture} 
 }{}
 }

\end{minipage}
\hfill
\begin{minipage}[t]{17.3cm}
} 
{ 
\end{minipage}%code  après
\hfill
\begin{minipage}[t]{1cm}

\begin{center}
\colorbox{sacado_violet}{\includegraphics[height=1cm]{qrcodes/qrDummy.png}}
\colorbox{white}{ {\footnotesize /b/ABCD} }
\end{center}

\end{minipage}

\end{tcolorbox}
\par 
}

 
%%%%%%%%%%%%% ExoCu 7 paramètres : Compétences, qrcode , calculatrice, python, scratch, tableur, annales
\newenvironment{ExoCu}[7]{% code avant
\tcbset{top=-0.2cm }
\stepcounter{exo}

\begin{tcolorbox}[right=-5mm, enhanced, lifted shadow={0mm}{0mm}{0mm}{0mm}%
{black!60!white}, attach boxed title to top right={xshift=-0.3cm, yshift*=-2mm}, coltitle=sacado_green!85!black, colback=white!100!white, boxed title style={colback=white}, colframe=sacado_green!100!black,title= {\footnotesize  #1}  ]
 
\hspace{-1.3cm} 
\begin{minipage}[t]{0.7cm}

 \begin{tikzpicture}
 	\node[fill=sacado_green,minimum width=0.7cm]{\textcolor{white}{\bf {\Large \theexo}}};
 \end{tikzpicture}


%%%%%%%%%%%%%%%%%%%%%%%% Condition pour la calculatrice
 \ifthenelse{\equal{#3}{1}}{
 \begin{tikzpicture}
 	\node[minimum width=0.7cm]{\includegraphics[scale=0.5]{MISC/calculator.png} };
 \end{tikzpicture} 
 }{
 \ifthenelse{\equal{#3}{2}}{
 \begin{tikzpicture}
 	\node[minimum width=0.7cm]{\includegraphics[scale=0.5]{MISC/no_calculator.png} };
 \end{tikzpicture} 
 }{}
 }

\end{minipage}
\hfill
\begin{minipage}[t]{17.3cm}
} 
{ 
\end{minipage}%code  après
\hfill
\begin{minipage}[t]{1cm}

 
\begin{center}
\colorbox{sacado_green}{\includegraphics[height=1cm]{qrcodes/qrDummy.png}}
\colorbox{white}{ {\footnotesize /b/ABCD}  }
\end{center}
 

\end{minipage}

\end{tcolorbox}
\par
}  


 %%%%%%%%%%%%% ExoCd 7 paramètres : Compétences, qrcode , calculatrice, python, scratch, tableur, annales
\newenvironment{ExoCd}[7]{% code avant
\tcbset{top=-0.2cm }
\stepcounter{exo}

\begin{tcolorbox}[right=-5mm, enhanced, lifted shadow={0mm}{0mm}{0mm}{0mm}%
{black!60!white}, attach boxed title to top right={xshift=-0.3cm, yshift*=-2mm}, coltitle=sacado_blue!85!black, colback=white!100!white, boxed title style={colback=white}, colframe=sacado_blue!100!black,title= {\footnotesize  #1}  ]
 
\hspace{-1.3cm} 
\begin{minipage}[t]{0.7cm}

 \begin{tikzpicture}
 	\node[fill=sacado_blue,minimum width=0.7cm]{\textcolor{white}{\bf {\Large \theexo}}};
 \end{tikzpicture}


%%%%%%%%%%%%%%%%%%%%%%%% Condition pour la calculatrice
 \ifthenelse{\equal{#3}{1}}{
 \begin{tikzpicture}
 	\node[minimum width=0.7cm]{\includegraphics[scale=0.5]{MISC/calculator.png} };
 \end{tikzpicture} 
 }{
 \ifthenelse{\equal{#3}{2}}{
 \begin{tikzpicture}
 	\node[minimum width=0.7cm]{\includegraphics[scale=0.5]{MISC/no_calculator.png} };
 \end{tikzpicture} 
 }{}
 }

\end{minipage}
\hfill
\begin{minipage}[t]{17.3cm}
} 
{ 
\end{minipage}%code  après
\hfill
\begin{minipage}[t]{1cm}

\begin{center}
\colorbox{sacado_blue}{\includegraphics[height=1cm]{qrcodes/qrDummy.png}}
\colorbox{white}{ {\footnotesize /b/ABCD} }
\end{center}

\end{minipage}

\end{tcolorbox}
\par
}  


%%%%%%%%%%%%% ExoCt 7 paramètres : Compétences, qrcode , calculatrice, python, scratch, tableur, annales
\newenvironment{ExoCt}[7]{% code avant
\tcbset{top=-0.2cm }
\stepcounter{exo}

\begin{tcolorbox}[right=-5mm, enhanced, lifted shadow={0mm}{0mm}{0mm}{0mm}%
{black!60!white}, attach boxed title to top right={xshift=-0.3cm, yshift*=-2mm}, coltitle=sacado_red!85!black, colback=white!100!white, boxed title style={colback=white}, colframe=sacado_red!100!black,title= {\footnotesize  #1}  ]
 
\hspace{-1.3cm} 
\begin{minipage}[t]{0.7cm}

 \begin{tikzpicture}
 	\node[fill=sacado_red,minimum width=0.7cm]{\textcolor{white}{\bf {\Large \theexo}}};
 \end{tikzpicture}


%%%%%%%%%%%%%%%%%%%%%%%% Condition pour la calculatrice
 \ifthenelse{\equal{#3}{1}}{
 \begin{tikzpicture}
 	\node[minimum width=0.7cm]{\includegraphics[scale=0.5]{MISC/calculator.png} };
 \end{tikzpicture} 
 }{
 \ifthenelse{\equal{#3}{2}}{
 \begin{tikzpicture}
 	\node[minimum width=0.7cm]{\includegraphics[scale=0.5]{MISC/no_calculator.png} };
 \end{tikzpicture} 
 }{}
 }

\end{minipage}
\hfill
\begin{minipage}[t]{17.3cm}
} 
{ 
\end{minipage}%code  après
\hfill
\begin{minipage}[t]{1cm}

\begin{center}
\colorbox{sacado_red}{\includegraphics[height=1cm]{qrcodes/qrDummy.png}}
\colorbox{white}{ {\footnotesize /b/ABCD} }
\end{center}

\end{minipage}

\end{tcolorbox}
\par
}  

 
 
 

%%%%%%%%%%%%% ExoCu 7 paramètres : compétences , qrcode , calculatrice, python, scratch, tableur, annales
\newenvironment{ExoAuto}[7]{% code avant
\tcbset{top=-0.2cm }
\stepcounter{exo}

\begin{tcolorbox}[right=-5mm, enhanced, lifted shadow={0mm}{0mm}{0mm}{0mm}%
{black!60!white}, attach boxed title to top right={xshift=-0.3cm, yshift*=-2mm}, coltitle=sacado_orange!85!black, colback=white!100!white, boxed title style={colback=white}, colframe=sacado_orange!100!black,title= {\footnotesize  #1}  ]
 
\hspace{-1.3cm} 
\begin{minipage}[t]{0.7cm}

 \begin{tikzpicture}
 	\node[fill=sacado_orange,minimum width=0.7cm]{\textcolor{white}{\bf {\Large \theexo}}};
 \end{tikzpicture}


%%%%%%%%%%%%%%%%%%%%%%%% Condition pour la calculatrice
 \ifthenelse{\equal{#3}{1}}{
 \begin{tikzpicture}
 	\node[minimum width=0.7cm]{\includegraphics[scale=0.5]{MISC/calculator.png} };
 \end{tikzpicture} 
 }{
 \ifthenelse{\equal{#3}{2}}{
 \begin{tikzpicture}
 	\node[minimum width=0.7cm]{\includegraphics[scale=0.5]{MISC/no_calculator.png} };
 \end{tikzpicture} 
 }{}
 }

\end{minipage}
\hfill
\begin{minipage}[t]{17.3cm}
} 
{ 
\end{minipage}%code  après
\hfill
\begin{minipage}[t]{1cm}

\begin{center}
\colorbox{sacado_orange}{\includegraphics[height=1cm]{qrcodes/qrDummy.png}}
\colorbox{white}{  {\footnotesize /b/ABCD}  }
\end{center}

\end{minipage}

\end{tcolorbox}
\par
}  

 

%%%%%%%%%%%%% ExoDec 7 paramètres : compétences , qrcode , calculatrice, python, scratch, tableur, annales
 
\newenvironment{ExoDec}[6]{% code avant
\tcbset{top=-0.2cm }
\stepcounter{exo}

\begin{tcolorbox}[right=-5mm, enhanced, lifted shadow={0mm}{0mm}{0mm}{0mm}%
{black!60!white}, attach boxed title to top right={xshift=-0.3cm, yshift*=-2mm}, coltitle=sacado_violet!85!black, colback=white!100!white, boxed title style={colback=white}, colframe=sacado_violet!100!black,title= {\footnotesize  #1}  ]
 
\hspace{-1.3cm} 
\begin{minipage}[t]{1cm}

 \begin{tikzpicture}
 	\node[fill=sacado_violet,minimum width=0.7cm]{\textcolor{white}{\bf {\Large \theexo}}};
 \end{tikzpicture}


%%%%%%%%%%%%%%%%%%%%%%%% Condition pour la calculatrice
 \ifthenelse{\equal{#3}{1}}{
 \begin{tikzpicture}
 	\node[minimum width=0.7cm]{\includegraphics[scale=0.5]{MISC/calculator.png} };
 \end{tikzpicture} 
 }{
 \ifthenelse{\equal{#3}{2}}{
 \begin{tikzpicture}
 	\node[minimum width=0.7cm]{\includegraphics[scale=0.5]{MISC/no_calculator.png} };
 \end{tikzpicture} 
 }{}
 }

\end{minipage}
\begin{minipage}[t]{17.3cm}
} 
{ 
\end{minipage}
\end{tcolorbox}
\par
}  
%%%%%%%%%%%%%%%%%%%%%%%%%%%%%%%%%%%%%%%%%%%%%%%%%%%%%%%%%%%%%%%%%%%%%%%%%%%%%%%%%%%%%%%%%%%%%%%%%%%%%%%%%%%%%%%%%%%%%%
%%%%%%%%%%%%%%%%%%%%%%%%%%%%%%%%%%%%%%%%%%%%%%%%%%%%%%%%%%%%%%%%%%%%%%%%%%%%%%%%%%%%%%%%%%%%%%%%%%%%%%%%%%%%%%%%%%%%%%
%%%%%%%%%%%%%%%%  Exercices sans qrcode                                %%%%%%%%%%%%%%%%%%%%%%%%%%%%%%%%%%%%%%%%%%%%%%%
%%%%%%%%%%%%%%%%%%%%%%%%%%%%%%%%%%%%%%%%%%%%%%%%%%%%%%%%%%%%%%%%%%%%%%%%%%%%%%%%%%%%%%%%%%%%%%%%%%%%%%%%%%%%%%%%%%%%%%
%%%%%%%%%%%%%%%%%%%%%%%%%%%%%%%%%%%%%%%%%%%%%%%%%%%%%%%%%%%%%%%%%%%%%%%%%%%%%%%%%%%%%%%%%%%%%%%%%%%%%%%%%%%%%%%%%%%%%%
 
 
 
%%%%%%%%%%%%% ExoCad 7 paramètres : Compétences , calculatrice, python, scratch, tableur, annales
\newenvironment{ExoCadN}[6]{% code avant
\tcbset{top=-0.2cm }
\stepcounter{exo}

\begin{tcolorbox}[right=-5mm, enhanced, lifted shadow={0mm}{0mm}{0mm}{0mm}%
{black!60!white}, attach boxed title to top right={xshift=-0.3cm, yshift*=-2mm}, coltitle=sacado_violet!85!black, colback=white!100!white, boxed title style={colback=white}, colframe=sacado_violet!100!black,title= {\footnotesize  #1}  ]
 
\hspace{-1.3cm} 
\begin{minipage}[t]{1cm}

 \begin{tikzpicture}
 	\node[fill=sacado_violet,minimum width=0.7cm]{\textcolor{white}{\bf {\Large \theexo}}};
 \end{tikzpicture}


%%%%%%%%%%%%%%%%%%%%%%%% Condition pour la calculatrice
 \ifthenelse{\equal{#3}{1}}{
 \begin{tikzpicture}
 	\node[minimum width=0.7cm]{\includegraphics[scale=0.5]{MISC/calculator.png} };
 \end{tikzpicture} 
 }{
 \ifthenelse{\equal{#3}{2}}{
 \begin{tikzpicture}
 	\node[minimum width=0.7cm]{\includegraphics[scale=0.5]{MISC/no_calculator.png} };
 \end{tikzpicture} 
 }{}
 }

\end{minipage}
\begin{minipage}[t]{17.3cm}
} 
{ 
\end{minipage}%code  après
\end{tcolorbox}
\par 
}

 
%%%%%%%%%%%%% ExoCu 7 paramètres : Compétences , calculatrice, python, scratch, tableur, annales
\newenvironment{ExoCuN}[6]{% code avant
\tcbset{top=-0.2cm }
\stepcounter{exo}

\begin{tcolorbox}[right=-5mm, enhanced, lifted shadow={0mm}{0mm}{0mm}{0mm}%
{black!60!white}, attach boxed title to top right={xshift=-0.3cm, yshift*=-2mm}, coltitle=sacado_green!85!black, colback=white!100!white, boxed title style={colback=white}, colframe=sacado_green!100!black,title= {\footnotesize  #1}  ]
 
\hspace{-1.3cm} 
\begin{minipage}[t]{1cm}

 \begin{tikzpicture}
 	\node[fill=sacado_green,minimum width=0.7cm]{\textcolor{white}{\bf {\Large \theexo}}};
 \end{tikzpicture}


%%%%%%%%%%%%%%%%%%%%%%%% Condition pour la calculatrice
 \ifthenelse{\equal{#2}{1}}{
 \begin{tikzpicture}
 	\node[minimum width=0.7cm]{\includegraphics[scale=0.5]{MISC/calculator.png} };
 \end{tikzpicture} 
 }{
 \ifthenelse{\equal{#2}{2}}{
 \begin{tikzpicture}
 	\node[minimum width=0.7cm]{\includegraphics[scale=0.5]{MISC/no_calculator.png} };
 \end{tikzpicture} 
 }{}
 }

\end{minipage}
\begin{minipage}[t]{17.3cm}
} 
{ 
\end{minipage}
\end{tcolorbox}
\par
}  


 %%%%%%%%%%%%% ExoCd 6 paramètres : Compétences, calculatrice, python, scratch, tableur, annales
\newenvironment{ExoCdN}[6]{% code avant
\tcbset{top=-0.2cm }
\stepcounter{exo}

\begin{tcolorbox}[right=-5mm, enhanced, lifted shadow={0mm}{0mm}{0mm}{0mm}%
{black!60!white}, attach boxed title to top right={xshift=-0.3cm, yshift*=-2mm}, coltitle=sacado_blue!85!black, colback=white!100!white, boxed title style={colback=white}, colframe=sacado_blue!100!black,title= {\footnotesize  #1}  ]
 
\hspace{-1.3cm} 
\begin{minipage}[t]{1cm}

 \begin{tikzpicture}
 	\node[fill=sacado_blue,minimum width=0.7cm]{\textcolor{white}{\bf {\Large \theexo}}};
 \end{tikzpicture}


%%%%%%%%%%%%%%%%%%%%%%%% Condition pour la calculatrice
 \ifthenelse{\equal{#2}{1}}{
 \begin{tikzpicture}
 	\node[minimum width=0.7cm]{\includegraphics[scale=0.5]{MISC/calculator.png} };
 \end{tikzpicture} 
 }{
 \ifthenelse{\equal{#2}{2}}{
 \begin{tikzpicture}
 	\node[minimum width=0.7cm]{\includegraphics[scale=0.5]{MISC/no_calculator.png} };
 \end{tikzpicture} 
 }{}
 }

\end{minipage}
\begin{minipage}[t]{17.3cm}
} 
{ 
\end{minipage}%code  après
\end{tcolorbox}
\par
}  


%%%%%%%%%%%%% ExoCt 6 paramètres : Compétences,   calculatrice, python, scratch, tableur, annales
\newenvironment{ExoCtN}[6]{% code avant
\tcbset{top=-0.2cm }
\stepcounter{exo}

\begin{tcolorbox}[right=-5mm, enhanced, lifted shadow={0mm}{0mm}{0mm}{0mm}%
{black!60!white}, attach boxed title to top right={xshift=-0.3cm, yshift*=-2mm}, coltitle=sacado_red!85!black, colback=white!100!white, boxed title style={colback=white}, colframe=sacado_red!100!black,title= {\footnotesize  #1}  ]
 
\hspace{-1.3cm} 
\begin{minipage}[t]{1cm}

 \begin{tikzpicture}
 	\node[fill=sacado_red,minimum width=0.7cm]{\textcolor{white}{\bf {\Large \theexo}}};
 \end{tikzpicture}


%%%%%%%%%%%%%%%%%%%%%%%% Condition pour la calculatrice
 \ifthenelse{\equal{#2}{1}}{
 \begin{tikzpicture}
 	\node[minimum width=0.7cm]{\includegraphics[scale=0.5]{MISC/calculator.png} };
 \end{tikzpicture} 
 }{
 \ifthenelse{\equal{#2}{2}}{
 \begin{tikzpicture}
 	\node[minimum width=0.7cm]{\includegraphics[scale=0.5]{MISC/no_calculator.png} };
 \end{tikzpicture} 
 }{}
 }

\end{minipage}
\begin{minipage}[t]{17.3cm}
} 
{ 
\end{minipage}%code  après
\end{tcolorbox}
\par
}  

 
 
 

%%%%%%%%%%%%% ExoCu 7 paramètres : compétences , qrcode , calculatrice, python, scratch, tableur, annales
\newenvironment{ExoAutoN}[6]{% code avant
\tcbset{top=-0.2cm }
\stepcounter{exo}

\begin{tcolorbox}[right=5mm, enhanced, lifted shadow={0mm}{0mm}{0mm}{0mm}%
{black!60!white}, attach boxed title to top right={xshift=-0.3cm, yshift*=-2mm}, coltitle=sacado_orange!85!black, colback=white!100!white, boxed title style={colback=white}, colframe=sacado_orange!100!black,title= {\footnotesize  #1}  ]
 
\hspace{-1.4cm} 
\begin{minipage}[t]{0.7cm}

 \begin{tikzpicture}
 	\node[fill=sacado_orange,minimum width=0.7cm]{\textcolor{white}{\bf {\Large \theexo}}};
 \end{tikzpicture}


%%%%%%%%%%%%%%%%%%%%%%%% Condition pour la calculatrice
 \ifthenelse{\equal{#2}{1}}{
 \begin{tikzpicture}
 	\node[minimum width=0.7cm]{\includegraphics[scale=0.5]{MISC/calculator.png} };
 \end{tikzpicture} 
 }{
 \ifthenelse{\equal{#2}{2}}{
 \begin{tikzpicture}
 	\node[minimum width=0.7cm]{\includegraphics[scale=0.5]{MISC/no_calculator.png} };
 \end{tikzpicture} 
 }{}
 }

\end{minipage}
\hfill
\begin{minipage}[t]{17.3cm}
} 
{ 
\end{minipage}%code  après
\end{tcolorbox}
\par
}  
%%%%%%%%%%%%%%%%%%%%%%%%%%%%%%%%%%%%%%%%%%%%%%%%%%%%%%%%%%%%%%%%%%%%%%%%%%%%%%%%%%%%%%%%%%%%%%%%%%%%%%%%%%%%%%%%%%%%%%
%%%%%%%%%%%%%%%%%%%%%%%%%%%%%%%%%%%%%%%%%%%%%%%%%%%%%%%%%%%%%%%%%%%%%%%%%%%%%%%%%%%%%%%%%%%%%%%%%%%%%%%%%%%%%%%%%%%%%%
%%%%%%%%%%%%%%%%  Exercices   sans contours                            %%%%%%%%%%%%%%%%%%%%%%%%%%%%%%%%%%%%%%%%%%%%%%%
%%%%%%%%%%%%%%%%%%%%%%%%%%%%%%%%%%%%%%%%%%%%%%%%%%%%%%%%%%%%%%%%%%%%%%%%%%%%%%%%%%%%%%%%%%%%%%%%%%%%%%%%%%%%%%%%%%%%%%
%%%%%%%%%%%%%%%%%%%%%%%%%%%%%%%%%%%%%%%%%%%%%%%%%%%%%%%%%%%%%%%%%%%%%%%%%%%%%%%%%%%%%%%%%%%%%%%%%%%%%%%%%%%%%%%%%%%%%%


\newcommand{\Sf}[1]{ \vspace{0.1cm}
{\color{fond}{\Large \textbf{#1}}  } 
} 

\newcommand{\Sfe}[1]{ \vspace{0.1cm}
{\color{sacado_blue}{\Large \textbf{#1}}  } 
} 
% fin de la procédure



%%%%%%%%%%%%% Pointillés ou ligne
\newcommand{\point}[1]{\vspace{0.1cm}\multido{}{#1}{ \dotfill \medskip \endgraf}}
\newcommand{\ligne}[1]{\vspace{0.1cm}\multido{}{#1}{ {\color{cqcqcq}\hrulefill} \medskip \endgraf}}
%----------------------------------------
%
%   Macros et opérateurs
%
%----------------------------------------

\newcommand{\second}{2\up{d}\xspace}
\newcommand{\seconde}{2\up{de}\xspace}
\newcommand{\R}{\mathbb R}
\newcommand{\Rp}{\R_+}
\newcommand{\Rpe}{\R_+^*}
\newcommand{\Rm}{\R_-}
\newcommand{\Rme}{\R_-^*}
\newcommand{\N}{\mathbb N}
\newcommand{\D}{\mathbb D}
\newcommand{\Q}{\mathbb Q}
\newcommand{\Z}{\mathbb Z}
\newcommand{\C}{\mathbb C}
\newcommand{\grs}{\mathfrak S}
\newcommand{\IN}[1]{\llbracket 1,#1\rrbracket}
\newcommand{\card}{\text{Card}\,}
\usepackage{mathrsfs}
\newcommand{\parties}{\mathscr P}
\renewcommand{\epsilon}{\varepsilon}
\newcommand{\rmd}{\text{d}}
\newcommand{\diff}{\mathrm D}
\newcommand{\Id}{{\rm Id}}
\newcommand{\e}{{\rm e}}
\newcommand{\I}{{\rm i}}
\newcommand{\J}{{\rm j}}
\newcommand{\ro}{\circ}
\newcommand{\exu}{\exists\,!\,}
\newcommand{\telq}{\,\, \mid \,\,}
\newcommand{\para}{\raisebox{0.1em}{\text{\footnotesize /\hspace{-0.1em}/}}}   
\newcommand{\vect}[1]{\overrightarrow{#1}}
\newcommand{\scal}[2]{\left(\, #1 \mid #2 \, \right)}
\newcommand{\ortho}[1]{{#1}^\perp}
\newcommand{\veci}{\vec{\text{\it \i}}}
\newcommand{\vecj}{\vec{\text{\it \j}}}
\newcommand{\rep}{$(O;\veci,\vecj,\vec{k})$\xspace}
\newcommand{\Oijk}{$(O, \veci,\vecj,\vec{k})$\xspace}
\newcommand{\rond}{repère orthonormal direct}
\newcommand{\bond}{base orthonormale directe}
\newcommand{\eq}{\Longleftrightarrow}
\newcommand{\implique}{\Longrightarrow}
\newcommand{\noneq}{\ \ \ /\hspace{-1.45em}\eq}
\newcommand{\tend}{\longrightarrow}
\newcommand{\egx}[2]{\underset{#1 \tend #2}=}
\newcommand{\asso}{\longmapsto}
\newcommand{\vers}{\longrightarrow}
\newcommand{\eqn}{~\underset{n \rightarrow \infty}{\sim}~}
\newcommand{\eqx}[2]{~\underset{#1 \rightarrow #2}{\sim}~}

\newcommand{\egn}{~\underset{n \rightarrow \infty}{=}~}
\renewcommand{\descriptionlabel}{\hspace{\labelsep}$\bullet$}
\renewcommand{\bar}{\overline}
\DeclareMathOperator{\ash}{{Argsh}}
\DeclareMathOperator{\cotan}{{cotan}}
\DeclareMathOperator{\ach}{{Argch}}
\DeclareMathOperator{\ath}{{Argth}}
\DeclareMathOperator{\sh}{{sh}}
\DeclareMathOperator{\ch}{{ch}}
\DeclareMathOperator{\Mat}{{Mat}}
\DeclareMathOperator{\Vect}{{Vect}}
\DeclareMathOperator{\trace}{{tr}}
\newcommand{\tr}{{}^{\mathrm t}}
\newcommand{\divi}{~\big|~}
\newcommand{\ndivi}{~\not{\big|}~}
\newcommand{\et}{\wedge}
\newcommand{\ou}{\vee}
\renewcommand{\det}{\operatorname{\text{dét}}}
\DeclareMathOperator{\grad}{{grad}}
\renewcommand{\arcsin}{{\mathop{\mathrm{Arcsin}}}}
\renewcommand{\arccos}{{\mathop{\mathrm{Arccos}}}}
\renewcommand{\arctan}{{\mathop{\mathrm{Arctan}}}}
\renewcommand{\tanh}{{\mathop{\mathrm{th}}}}
\newcommand{\pgcd}{\mathop{\mathrm{pgcd}}}
\newcommand{\ppcm}{\mathop{\mathrm{ppcm}}}
\newcommand{\fonc}[4]{\left\{\begin{tabular}{ccc}$#1$ & $\vers$ & $#2$ \\
$#3$ & $\asso$ & $#4$ \end{tabular}\right.}
\renewcommand{\geq}{\geqslant}
\renewcommand{\leq}{\leqslant}
\renewcommand{\Re}{\text{\rm Re}}
\renewcommand{\Im}{\text{\rm Im}}
\renewcommand{\ker}{\mathop{\mathrm{Ker}}}
\newcommand{\Lin}{\mathcal L}
\newcommand{\GO}{\mathcal O}
\newcommand{\GSO}{\mathcal{SO}}
\newcommand{\GL}{\mathcal{GL}}
\renewcommand{\emptyset}{\varnothing}
%\newcommand{\arc}[1]{\overset{\frown}{#1}}
\newcommand{\rg}{\mathop{\mathrm{rg}}}
\newcommand{\ds}{\displaystyle}
\newcommand{\co}[3]{\begin{pmatrix}#1 \\ #2 \\ #3\end{pmatrix}}
\newcommand{\demi}{\frac 1 2}
\newcommand{\limi}[2]{\underset{#1 \rightarrow #2}\lim}
%\usetikzlibrary{quotes,arrows.meta}
% Pour les pavés droits sans annotations de hLl 
\tikzset{
  nonannotated cuboid/.pic={
    \tikzset{%
      every edge quotes/.append style={midway, auto},
      /cuboid/.cd,
      #1
    }
    \draw [every edge/.append style={pic actions, densely dashed, opacity=.5}, pic actions]
    (0,0,0) coordinate (o) -- ++(-\cubescale*\cubex,0,0) coordinate (a) -- ++(0,-\cubescale*\cubey,0) coordinate (b) edge coordinate [pos=1] (g) ++(0,0,-\cubescale*\cubez)  -- ++(\cubescale*\cubex,0,0) coordinate (c) -- cycle
    (o) -- ++(0,0,-\cubescale*\cubez) coordinate (d) -- ++(0,-\cubescale*\cubey,0) coordinate (e) edge (g) -- (c) -- cycle
    (o) -- (a) -- ++(0,0,-\cubescale*\cubez) coordinate (f) edge (g) -- (d) -- cycle;
%    \path [every edge/.append style={pic actions, |-|}]
%    (b) +(0,-5pt) coordinate (b1) edge ["\cubex \cubeunits"'] (b1 -| c)
%    (b) +(-5pt,0) coordinate (b2) edge ["\cubey \cubeunits"] (b2 |- a)
%    (c) +(3.5pt,-3.5pt) coordinate (c2) edge ["\cubez \cubeunits"'] ([xshift=3.5pt,yshift=-3.5pt]e)
    ;
  },
  /cuboid/.search also={/tikz},
  /cuboid/.cd,
  width/.store in=\cubex,
  height/.store in=\cubey,
  depth/.store in=\cubez,
  units/.store in=\cubeunits,
  scale/.store in=\cubescale,
  width=10,
  height=10,
  depth=10,
  units=cm,
  scale=.1,
}

% Pour les pavés droits avec annotations de hLl
\tikzset{
  annotated cuboid/.pic={
    \tikzset{%
      every edge quotes/.append style={midway, auto},
      /cuboid/.cd,
      #1
    }
    \draw [every edge/.append style={pic actions, densely dashed, opacity=.5}, pic actions]
    (0,0,0) coordinate (o) -- ++(-\cubescale*\cubex,0,0) coordinate (a) -- ++(0,-\cubescale*\cubey,0) coordinate (b) edge coordinate [pos=1] (g) ++(0,0,-\cubescale*\cubez)  -- ++(\cubescale*\cubex,0,0) coordinate (c) -- cycle
    (o) -- ++(0,0,-\cubescale*\cubez) coordinate (d) -- ++(0,-\cubescale*\cubey,0) coordinate (e) edge (g) -- (c) -- cycle
    (o) -- (a) -- ++(0,0,-\cubescale*\cubez) coordinate (f) edge (g) -- (d) -- cycle;
    \path [every edge/.append style={pic actions, |-|}]
    (b) +(0,-5pt) coordinate (b1) edge ["\cubex \cubeunits"'] (b1 -| c)
    (b) +(-5pt,0) coordinate (b2) edge ["\cubey \cubeunits"] (b2 |- a)
    (c) +(3.5pt,-3.5pt) coordinate (c2) edge ["\cubez \cubeunits"'] ([xshift=3.5pt,yshift=-3.5pt]e)
    ;
  },
  /cuboid/.search also={/tikz},
  /cuboid/.cd,
  width/.store in=\cubex,
  height/.store in=\cubey,
  depth/.store in=\cubez,
  units/.store in=\cubeunits,
  scale/.store in=\cubescale,
  width=10,
  height=10,
  depth=10,
  units=cm,
  scale=.1,
}

% Pour les pavés droits avec annotations "hauteur", "Longueur", "largeur".
\tikzset{
  defannotated cuboid/.pic={
    \tikzset{%
      every edge quotes/.append style={midway, auto},
      /cuboid/.cd,
      #1
    }
    \draw [every edge/.append style={pic actions, densely dashed, opacity=.5}, pic actions]
    (0,0,0) coordinate (o) -- ++(-\cubescale*\cubex,0,0) coordinate (a) -- ++(0,-\cubescale*\cubey,0) coordinate (b) edge coordinate [pos=1] (g) ++(0,0,-\cubescale*\cubez)  -- ++(\cubescale*\cubex,0,0) coordinate (c) -- cycle
    (o) -- ++(0,0,-\cubescale*\cubez) coordinate (d) -- ++(0,-\cubescale*\cubey,0) coordinate (e) edge (g) -- (c) -- cycle
    (o) -- (a) -- ++(0,0,-\cubescale*\cubez) coordinate (f) edge (g) -- (d) -- cycle;
    \path [every edge/.append style={pic actions, |-|}]
    (b) +(0,-5pt) coordinate (b1) edge ["Longueur"'] (b1 -| c)
    (b) +(-5pt,0) coordinate (b2) edge ["Hauteur"] (b2 |- a)
    (c) +(3.5pt,-3.5pt) coordinate (c2) edge ["largeur"'] ([xshift=3.5pt,yshift=-3.5pt]e)
    ;
  },
  /cuboid/.search also={/tikz},
  /cuboid/.cd,
  width/.store in=\cubex,
  height/.store in=\cubey,
  depth/.store in=\cubez,
  units/.store in=\cubeunits,
  scale/.store in=\cubescale,
  width=10,
  height=10,
  depth=10,
  units=cm,
  scale=.1,
}

\tikzset{
  defiannotated cuboid/.pic={
    \tikzset{%
      every edge quotes/.append style={midway, auto},
      /cuboid/.cd,
      #1
    }
    \draw [every edge/.append style={pic actions, densely dashed, opacity=.5}, pic actions]
    (0,0,0) coordinate (o) -- ++(-\cubescale*\cubex,0,0) coordinate (a) -- ++(0,-\cubescale*\cubey,0) coordinate (b) edge coordinate [pos=1] (g) ++(0,0,-\cubescale*\cubez)  -- ++(\cubescale*\cubex,0,0) coordinate (c) -- cycle
    (o) -- ++(0,0,-\cubescale*\cubez) coordinate (d) -- ++(0,-\cubescale*\cubey,0) coordinate (e) edge (g) -- (c) -- cycle
    (o) -- (a) -- ++(0,0,-\cubescale*\cubez) coordinate (f) edge (g) -- (d) -- cycle;
    \path [every edge/.append style={pic actions, |-|}]
    (b) +(0,-5pt) coordinate (b1) edge ["c"'] (b1 -| c)
    (b) +(-5pt,0) coordinate (b2) edge ["c"] (b2 |- a)
    (c) +(3.5pt,-3.5pt) coordinate (c2) edge ["c"'] ([xshift=3.5pt,yshift=-3.5pt]e)
    ;
  },
  /cuboid/.search also={/tikz},
  /cuboid/.cd,
  width/.store in=\cubex,
  height/.store in=\cubey,
  depth/.store in=\cubez,
  units/.store in=\cubeunits,
  scale/.store in=\cubescale,
  width=10,
  height=10,
  depth=10,
  units=cm,
  scale=.1,
}

\tikzset{
  defiiannotated cuboid/.pic={
    \tikzset{%
      every edge quotes/.append style={midway, auto},
      /cuboid/.cd,
      #1
    }
    \draw [every edge/.append style={pic actions, densely dashed, opacity=.5}, pic actions]
    (0,0,0) coordinate (o) -- ++(-\cubescale*\cubex,0,0) coordinate (a) -- ++(0,-\cubescale*\cubey,0) coordinate (b) edge coordinate [pos=1] (g) ++(0,0,-\cubescale*\cubez)  -- ++(\cubescale*\cubex,0,0) coordinate (c) -- cycle
    (o) -- ++(0,0,-\cubescale*\cubez) coordinate (d) -- ++(0,-\cubescale*\cubey,0) coordinate (e) edge (g) -- (c) -- cycle
    (o) -- (a) -- ++(0,0,-\cubescale*\cubez) coordinate (f) edge (g) -- (d) -- cycle;
    \path [every edge/.append style={pic actions, |-|}]
    (b) +(0,-5pt) coordinate (b1) edge ["L"'] (b1 -| c)
    (b) +(-5pt,0) coordinate (b2) edge ["h"] (b2 |- a)
    (c) +(3.5pt,-3.5pt) coordinate (c2) edge ["l"'] ([xshift=3.5pt,yshift=-3.5pt]e)
    ;
  },
  /cuboid/.search also={/tikz},
  /cuboid/.cd,
  width/.store in=\cubex,
  height/.store in=\cubey,
  depth/.store in=\cubez,
  units/.store in=\cubeunits,
  scale/.store in=\cubescale,
  width=10,
  height=10,
  depth=10,
  units=cm,
  scale=.1,
}

\tikzset{
  xannotated cuboid/.pic={
    \tikzset{%
      every edge quotes/.append style={midway, auto},
      /cuboid/.cd,
      #1
    }
    \draw [every edge/.append style={pic actions, densely dashed, opacity=.5}, pic actions]
    (0,0,0) coordinate (o) -- ++(-\cubescale*\cubex,0,0) coordinate (a) -- ++(0,-\cubescale*\cubey,0) coordinate (b) edge coordinate [pos=1] (g) ++(0,0,-\cubescale*\cubez)  -- ++(\cubescale*\cubex,0,0) coordinate (c) -- cycle
    (o) -- ++(0,0,-\cubescale*\cubez) coordinate (d) -- ++(0,-\cubescale*\cubey,0) coordinate (e) edge (g) -- (c) -- cycle
    (o) -- (a) -- ++(0,0,-\cubescale*\cubez) coordinate (f) edge (g) -- (d) -- cycle;
    \path [every edge/.append style={pic actions, |-|}]
    (b) +(0,-5pt) coordinate (b1) edge ["1cm"'] (b1 -| c)
    (b) +(-5pt,0) coordinate (b2) edge ["1cm"] (b2 |- a)
    (c) +(3.5pt,-3.5pt) coordinate (c2) edge ["1cm"'] ([xshift=3.5pt,yshift=-3.5pt]e)
    ;
  },
  /cuboid/.search also={/tikz},
  /cuboid/.cd,
  width/.store in=\cubex,
  height/.store in=\cubey,
  depth/.store in=\cubez,
  units/.store in=\cubeunits,
  scale/.store in=\cubescale,
  width=10,
  height=10,
  depth=10,
  units=cm,
  scale=.1,
}

\tikzset{
  yannotated cuboid/.pic={
    \tikzset{%
      every edge quotes/.append style={midway, auto},
      /cuboid/.cd,
      #1
    }
    \draw [every edge/.append style={pic actions, densely dashed, opacity=.5}, pic actions]
    (0,0,0) coordinate (o) -- ++(-\cubescale*\cubex,0,0) coordinate (a) -- ++(0,-\cubescale*\cubey,0) coordinate (b) edge coordinate [pos=1] (g) ++(0,0,-\cubescale*\cubez)  -- ++(\cubescale*\cubex,0,0) coordinate (c) -- cycle
    (o) -- ++(0,0,-\cubescale*\cubez) coordinate (d) -- ++(0,-\cubescale*\cubey,0) coordinate (e) edge (g) -- (c) -- cycle
    (o) -- (a) -- ++(0,0,-\cubescale*\cubez) coordinate (f) edge (g) -- (d) -- cycle;
    \path [every edge/.append style={pic actions, |-|}]
    (b) +(0,-5pt) coordinate (b1) edge ["1mm"'] (b1 -| c)
    (b) +(-5pt,0) coordinate (b2) edge ["1mm"] (b2 |- a)
    (c) +(3.5pt,-3.5pt) coordinate (c2) edge ["1mm"'] ([xshift=3.5pt,yshift=-3.5pt]e)
    ;
  },
  /cuboid/.search also={/tikz},
  /cuboid/.cd,
  width/.store in=\cubex,
  height/.store in=\cubey,
  depth/.store in=\cubez,
  units/.store in=\cubeunits,
  scale/.store in=\cubescale,
  width=10,
  height=10,
  depth=10,
  units=cm,
  scale=.1,
}

%Cubes subdivisés
\newif\ifcuboidshade
\newif\ifcuboidemphedge

\tikzset{
  cuboid/.is family,
  cuboid,
  shiftx/.initial=0,
  shifty/.initial=0,
  dimx/.initial=3,
  dimy/.initial=3,
  dimz/.initial=3,
  scale/.initial=1,
  densityx/.initial=1,
  densityy/.initial=1,
  densityz/.initial=1,
  rotation/.initial=0,
  anglex/.initial=0,
  angley/.initial=90,
  anglez/.initial=225,
  scalex/.initial=1,
  scaley/.initial=1,
  scalez/.initial=0.5,
  front/.style={draw=black,fill=white},
  top/.style={draw=black,fill=white},
  right/.style={draw=black,fill=white},
  shade/.is if=cuboidshade,
  shadecolordark/.initial=black,
  shadecolorlight/.initial=white,
  shadeopacity/.initial=0.15,
  shadesamples/.initial=16,
  emphedge/.is if=cuboidemphedge,
  emphstyle/.style={thick},
}

\newcommand{\tikzcuboidkey}[1]{\pgfkeysvalueof{/tikz/cuboid/#1}}

% Commands
\newcommand{\tikzcuboid}[1]{
    \tikzset{cuboid,#1} % Process Keys passed to command
  \pgfmathsetlengthmacro{\vectorxx}{\tikzcuboidkey{scalex}*cos(\tikzcuboidkey{anglex})*28.452756}
  \pgfmathsetlengthmacro{\vectorxy}{\tikzcuboidkey{scalex}*sin(\tikzcuboidkey{anglex})*28.452756}
  \pgfmathsetlengthmacro{\vectoryx}{\tikzcuboidkey{scaley}*cos(\tikzcuboidkey{angley})*28.452756}
  \pgfmathsetlengthmacro{\vectoryy}{\tikzcuboidkey{scaley}*sin(\tikzcuboidkey{angley})*28.452756}
  \pgfmathsetlengthmacro{\vectorzx}{\tikzcuboidkey{scalez}*cos(\tikzcuboidkey{anglez})*28.452756}
  \pgfmathsetlengthmacro{\vectorzy}{\tikzcuboidkey{scalez}*sin(\tikzcuboidkey{anglez})*28.452756}
  \begin{scope}[xshift=\tikzcuboidkey{shiftx}, yshift=\tikzcuboidkey{shifty}, scale=\tikzcuboidkey{scale}, rotate=\tikzcuboidkey{rotation}, x={(\vectorxx,\vectorxy)}, y={(\vectoryx,\vectoryy)}, z={(\vectorzx,\vectorzy)}]
    \pgfmathsetmacro{\steppingx}{1/\tikzcuboidkey{densityx}}
  \pgfmathsetmacro{\steppingy}{1/\tikzcuboidkey{densityy}}
  \pgfmathsetmacro{\steppingz}{1/\tikzcuboidkey{densityz}}
  \newcommand{\dimx}{\tikzcuboidkey{dimx}}
  \newcommand{\dimy}{\tikzcuboidkey{dimy}}
  \newcommand{\dimz}{\tikzcuboidkey{dimz}}
  \pgfmathsetmacro{\secondx}{2*\steppingx}
  \pgfmathsetmacro{\secondy}{2*\steppingy}
  \pgfmathsetmacro{\secondz}{2*\steppingz}
  \foreach \x in {\steppingx,\secondx,...,\dimx}
  { \foreach \y in {\steppingy,\secondy,...,\dimy}
    {   \pgfmathsetmacro{\lowx}{(\x-\steppingx)}
      \pgfmathsetmacro{\lowy}{(\y-\steppingy)}
      \filldraw[cuboid/front] (\lowx,\lowy,\dimz) -- (\lowx,\y,\dimz) -- (\x,\y,\dimz) -- (\x,\lowy,\dimz) -- cycle;
    }
    }
  \foreach \x in {\steppingx,\secondx,...,\dimx}
  { \foreach \z in {\steppingz,\secondz,...,\dimz}
    {   \pgfmathsetmacro{\lowx}{(\x-\steppingx)}
      \pgfmathsetmacro{\lowz}{(\z-\steppingz)}
      \filldraw[cuboid/top] (\lowx,\dimy,\lowz) -- (\lowx,\dimy,\z) -- (\x,\dimy,\z) -- (\x,\dimy,\lowz) -- cycle;
        }
    }
    \foreach \y in {\steppingy,\secondy,...,\dimy}
  { \foreach \z in {\steppingz,\secondz,...,\dimz}
    {   \pgfmathsetmacro{\lowy}{(\y-\steppingy)}
      \pgfmathsetmacro{\lowz}{(\z-\steppingz)}
      \filldraw[cuboid/right] (\dimx,\lowy,\lowz) -- (\dimx,\lowy,\z) -- (\dimx,\y,\z) -- (\dimx,\y,\lowz) -- cycle;
    }
  }
  \ifcuboidemphedge
    \draw[cuboid/emphstyle] (0,\dimy,0) -- (\dimx,\dimy,0) -- (\dimx,\dimy,\dimz) -- (0,\dimy,\dimz) -- cycle;%
    \draw[cuboid/emphstyle] (0,\dimy,\dimz) -- (0,0,\dimz) -- (\dimx,0,\dimz) -- (\dimx,\dimy,\dimz);%
    \draw[cuboid/emphstyle] (\dimx,\dimy,0) -- (\dimx,0,0) -- (\dimx,0,\dimz);%
    \fi

    \ifcuboidshade
    \pgfmathsetmacro{\cstepx}{\dimx/\tikzcuboidkey{shadesamples}}
    \pgfmathsetmacro{\cstepy}{\dimy/\tikzcuboidkey{shadesamples}}
    \pgfmathsetmacro{\cstepz}{\dimz/\tikzcuboidkey{shadesamples}}
    \foreach \s in {1,...,\tikzcuboidkey{shadesamples}}
    {   \pgfmathsetmacro{\lows}{\s-1}
        \pgfmathsetmacro{\cpercent}{(\lows)/(\tikzcuboidkey{shadesamples}-1)*100}
        \fill[opacity=\tikzcuboidkey{shadeopacity},color=\tikzcuboidkey{shadecolorlight}!\cpercent!\tikzcuboidkey{shadecolordark}] (0,\s*\cstepy,\dimz) -- (\s*\cstepx,\s*\cstepy,\dimz) -- (\s*\cstepx,0,\dimz) -- (\lows*\cstepx,0,\dimz) -- (\lows*\cstepx,\lows*\cstepy,\dimz) -- (0,\lows*\cstepy,\dimz) -- cycle;
        \fill[opacity=\tikzcuboidkey{shadeopacity},color=\tikzcuboidkey{shadecolorlight}!\cpercent!\tikzcuboidkey{shadecolordark}] (0,\dimy,\s*\cstepz) -- (\s*\cstepx,\dimy,\s*\cstepz) -- (\s*\cstepx,\dimy,0) -- (\lows*\cstepx,\dimy,0) -- (\lows*\cstepx,\dimy,\lows*\cstepz) -- (0,\dimy,\lows*\cstepz) -- cycle;
        \fill[opacity=\tikzcuboidkey{shadeopacity},color=\tikzcuboidkey{shadecolorlight}!\cpercent!\tikzcuboidkey{shadecolordark}] (\dimx,0,\s*\cstepz) -- (\dimx,\s*\cstepy,\s*\cstepz) -- (\dimx,\s*\cstepy,0) -- (\dimx,\lows*\cstepy,0) -- (\dimx,\lows*\cstepy,\lows*\cstepz) -- (\dimx,0,\lows*\cstepz) -- cycle;
    }
    \fi 

  \end{scope}
}

%\makeatother




\begin{document}

\section{Utiliser et représenter les grands nombres entiers, des fractions simples, les nombres décimaux}

\subsection{Savoir différencier les chiffres d'un nombre}
\ExoCu{Représenter. Communiquer.}

\begin{enumerate}
\item Dans le nombre A = 428, le chiffre 2 est le chiffre des $\cdots\cdots\cdots\cdots\cdots\cdots\cdots\cdots$
\item Dans le nombre B = 8 948, le chiffre 9 est le chiffre des $\cdots\cdots\cdots\cdots\cdots\cdots\cdots\cdots$
\end{enumerate}


\ExoCu{Représenter. Communiquer.}

\begin{enumerate}
\item Dans le nombre A = 518 931 248, le chiffre 2 est le chiffre des $\cdots\cdots\cdots\cdots\cdots\cdots\cdots\cdots$
\item Dans le nombre B = 218 948, le chiffre 4 est le chiffre des $\cdots\cdots\cdots\cdots\cdots\cdots\cdots\cdots$
\end{enumerate}



\ExoCu{Représenter.}

Choisis la bonne réponse :
\begin{enumerate}
\item Dans la décomposition du nombre A = 456, le chiffre 4 représente :
\begin{tabular}{|c|c|c|c|}
\hline 
4000 & 400 & 40 & 4 \\ 
\hline 
\end{tabular}  

\item Dans la décomposition du nombre A = 6512, le chiffre 1 représente :
\begin{tabular}{|c|c|c|c|}
\hline 
1000 & 100 & 10 & 1 \\ 
\hline 
\end{tabular} 
\end{enumerate}


\ExoCd{Représenter.}

Choisis la bonne réponse :
\begin{enumerate}
\item Dans la décomposition du nombre A = 12378456, le chiffre 8 représente :
\begin{tabular}{|c|c|c|c|}
\hline 
8000 & 800 & 80 & 8 \\ 
\hline 
\end{tabular}  

\item Dans la décomposition du nombre A = 65934172, le chiffre 1 représente :
\begin{tabular}{|c|c|c|c|}
\hline 
1000 & 100 & 10 & 1 \\ 
\hline 
\end{tabular} 
\end{enumerate}


\ExoCu{Représenter. Communiquer.}

\begin{enumerate}
\item Combien de centaines contient le nombre 456 ?
\item Quel est le nombre de dizaines 237 ?
\end{enumerate}

\ExoCu{Représenter. Communiquer.}

\begin{enumerate}
\item Combien de centaines contient le nombre 12378456 ?
\item Quel est le nombre de milliers dans 84516237 ?
\end{enumerate}


\ExoCu{Représenter. Communiquer.}

On donne le  nombre 1250,43. 
\begin{description}
\item[$\triangleright$] Le chiffre des milliers est : $ \cdots\cdots\cdots\cdots $
\item[$\triangleright$] Le chiffre des centaines est : $ \cdots\cdots\cdots\cdots $
\item[$\triangleright$] Le chiffre des dizaines est : $ \cdots\cdots\cdots\cdots $
\item[$\triangleright$] Le chiffre des unités est : $ \cdots\cdots\cdots\cdots $
\item[$\triangleright$] Le chiffre des dixièmes est : $ \cdots\cdots\cdots\cdots $
\item[$\triangleright$] Le chiffre des centièmes est : $ \cdots\cdots\cdots\cdots $
\end{description}


\ExoCd{Représenter. Communiquer.}

On donne le  nombre 1857250,43. 
\begin{description}
\item[•] 1 est le chiffres des $ \cdots\cdots\cdots\cdots $.
\item[•] 3 est le chiffres des $ \cdots\cdots\cdots\cdots $.
\item[•] 2 est le chiffres des $ \cdots\cdots\cdots\cdots $.

\item[$\triangleright$] Le chiffre des centaines de milliers est : $ \cdots\cdots\cdots\cdots $
\item[$\triangleright$] Le chiffre des dizaines est : $ \cdots\cdots\cdots\cdots $
\item[$\triangleright$] Le chiffre des dixièmes est : $ \cdots\cdots\cdots\cdots $
\item[$\triangleright$] Le chiffre des milliers est :  $ \cdots\cdots\cdots\cdots $
\item[$\triangleright$] Le chiffre des unités est : $ \cdots\cdots\cdots\cdots $
\item[$\triangleright$] Le chiffre des dizaines de milliers est : $ \cdots\cdots\cdots\cdots $
\item[$\triangleright$] Le chiffre des centièmes est : $ \cdots\cdots\cdots\cdots $

\end{description}


\subsection{Savoir différencier les chiffres d'un nombre décimal}

\ExoCu{Représenter. Communiquer.}

On donne le  nombre 1250,43. 
\begin{description}
\item[$\triangleright$] Le chiffre des milliers est : $ \cdots\cdots\cdots\cdots $
\item[$\triangleright$] Le chiffre des centaines est : $ \cdots\cdots\cdots\cdots $
\item[$\triangleright$] Le chiffre des dizaines est : $ \cdots\cdots\cdots\cdots $
\item[$\triangleright$] Le chiffre des unités est : $ \cdots\cdots\cdots\cdots $
\item[$\triangleright$] Le chiffre des dixièmes est : $ \cdots\cdots\cdots\cdots $
\item[$\triangleright$] Le chiffre des centièmes est : $ \cdots\cdots\cdots\cdots $
\end{description}


\ExoCd{Représenter. Communiquer.}

On donne le  nombre 1857250,43. 
\begin{description}
\item[•] 1 est le chiffres des $ \cdots\cdots\cdots\cdots $.
\item[•] 3 est le chiffres des $ \cdots\cdots\cdots\cdots $.
\item[•] 2 est le chiffres des $ \cdots\cdots\cdots\cdots $.

\item[$\triangleright$] Le chiffre des centaines de milliers est : $ \cdots\cdots\cdots\cdots $
\item[$\triangleright$] Le chiffre des dizaines est : $ \cdots\cdots\cdots\cdots $
\item[$\triangleright$] Le chiffre des dixièmes est : $ \cdots\cdots\cdots\cdots $
\item[$\triangleright$] Le chiffre des milliers est :  $ \cdots\cdots\cdots\cdots $
\item[$\triangleright$] Le chiffre des unités est : $ \cdots\cdots\cdots\cdots $
\item[$\triangleright$] Le chiffre des dizaines de milliers est : $ \cdots\cdots\cdots\cdots $
\item[$\triangleright$] Le chiffre des centièmes est : $ \cdots\cdots\cdots\cdots $

\end{description}







\subsection{Utiliser des nombres décimaux ayant au plus quatre décimales.}



\ExoCu{Représenter.}

Écrire les nombres suivants sous forme décimale :
\begin{enumerate}
\item quatre-cent-douze unités et six-dixièmes $= \cdots\cdots\cdots\cdots\cdots\cdots\cdots\cdots $
\item $\frac{6}{10}= \cdots\cdots\cdots\cdots\cdots\cdots\cdots\cdots$
\item $\frac{162}{100}= \cdots\cdots\cdots\cdots\cdots\cdots\cdots\cdots$
\item $\frac{5129}{100}= \cdots\cdots\cdots\cdots\cdots\cdots\cdots\cdots$
\end{enumerate}
 

\ExoCu{Représenter.}
Complète la décomposition décimale du nombre $324,67$.

$324,67 = \cdots\cdots\cdots  \times 100 + \cdots\cdots\cdots \times 10 + \cdots\cdots\cdots  + \cdots\cdots\cdots \times \dfrac{1}{10} + \cdots\cdots\cdots \times \dfrac{1}{100} $



\ExoCu{Représenter.}

\begin{enumerate}


\item Mathilde a décomposé le nombre décimal  $A = 5\times 100 + 2\times 10 + 4 + 3\times \dfrac{1}{10} +  6\times \dfrac{1}{100} $. Peux-tu le retrouver ?
$A = \cdots\cdots\cdots\cdots\cdots\cdots\cdots\cdots $

\item Nour a décomposé le nombre décimal  $A = 6\times 1000 + 2\times 100 + 5\times 10 + 8 + 7\times \dfrac{1}{10} +   \dfrac{1}{100} $. Peux-tu le retrouver ?
$A = \cdots\cdots\cdots\cdots\cdots\cdots\cdots\cdots $
\end{enumerate}


\ExoCu{Représenter.}

Léon a décomposé un nombre décimal  $A = 5\times 100 + 2\times 10 + 4 + 3\times 0,1 +  6\times 0,01 $. Peux-tu le retrouver ?
$N = \cdots\cdots\cdots\cdots\cdots\cdots\cdots\cdots $



\ExoCd{Représenter.}

Écrire les nombres suivants sous forme décimale :
\begin{enumerate}
\item quatre-mille-deux-cent-sept unités et six-dixièmes $= \cdots\cdots\cdots\cdots\cdots\cdots\cdots\cdots $
\item $\frac{8619}{1000}= \cdots\cdots\cdots\cdots\cdots\cdots\cdots\cdots$
\item $\frac{62}{100}= \cdots\cdots\cdots\cdots\cdots\cdots\cdots\cdots$
\item $\frac{652}{10}= \cdots\cdots\cdots\cdots\cdots\cdots\cdots\cdots$
\end{enumerate}
 

\ExoCd{Représenter.}

\begin{enumerate}
\item Mathilde a décomposé le nombre décimal  $A = 5\times 100 + 2\times 10 + 4 + 3\times \dfrac{1}{10} +  6\times \dfrac{1}{100} $. Peux-tu le retrouver ?
$A = \cdots\cdots\cdots\cdots\cdots\cdots\cdots\cdots $

\item Nour a décomposé le nombre décimal  $A = 6\times 1000 + 2\times \dfrac{1}{10} +  7\times 100 + 5\times 10 + 6\times 1000 + 8 +   \dfrac{1}{100} $. Peux-tu le retrouver ?
$A = \cdots\cdots\cdots\cdots\cdots\cdots\cdots\cdots $
\end{enumerate}


\ExoCd{Représenter.}

\begin{enumerate}
\item Léon a décomposé un nombre décimal  $A = 5\times 100 + 2\times 10 + 4 + 3\times 0,1 +  6\times 0,01 $. Peux-tu le retrouver ?
$N = \cdots\cdots\cdots\cdots\cdots\cdots\cdots\cdots $

\item Aziz a décomposé un nombre décimal  $A = 5\times 100 + 2\times 10 + 4 + 3\times 0,1 +  6\times 0,01 $. Peux-tu le retrouver ?
$N = \cdots\cdots\cdots\cdots\cdots\cdots\cdots\cdots $

\end{enumerate}

\ExoCd{Représenter.}

\begin{enumerate}
\item Léon a décomposé un nombre décimal  $A = 5\times 100 + 2\times 10 + 4 + 3\times 0,1 +  6\times 0,01 $. Peux-tu le retrouver ?
$N = \cdots\cdots\cdots\cdots\cdots\cdots\cdots\cdots $

\item Aziz a décomposé un nombre décimal  $A = 5\times 100 + 2\times 10 + 4 + 3\times 0,1 +  6\times 0,01 $. Peux-tu le retrouver ?
$N = \cdots\cdots\cdots\cdots\cdots\cdots\cdots\cdots $

\end{enumerate}




\subsection{Représenter un nombre}


\ExoCu{Représenter.}

Quel est le nombre représenté sur le boulier chinois ?


\definecolor{ccqqqq}{rgb}{0.8,0.,0.}
\definecolor{ffdxqq}{rgb}{1.,0.8431372549019608,0.}
\definecolor{qqqqcc}{rgb}{0.,0.,0.8}
\definecolor{wwqqcc}{rgb}{0.4,0.,0.8}
\definecolor{ffffff}{rgb}{1.,1.,1.}
\definecolor{qqttcc}{rgb}{0.,0.2,0.8}
\definecolor{ffffqq}{rgb}{1.,1.,0.}
\definecolor{qqwwtt}{rgb}{0.,0.4,0.2}
\definecolor{qqqqff}{rgb}{0.,0.,1.}
\definecolor{ffqqqq}{rgb}{1.,0.,0.}
\definecolor{ffttww}{rgb}{1.,0.2,0.4}
\definecolor{xfqqff}{rgb}{0.4980392156862745,0.,1.}
\definecolor{wqwqwq}{rgb}{0.3764705882352941,0.3764705882352941,0.3764705882352941}
\definecolor{yqyqyq}{rgb}{0.5019607843137255,0.5019607843137255,0.5019607843137255}
\begin{tikzpicture}[line cap=round,line join=round,>=triangle 45,x=0.4791579887901929cm,y=0.4791579887901929cm]
\clip(-1.0142613504052427,-2.28441497381301) rectangle (11.507704557226107,5.3309215237823535);
\fill[line width=2.pt,color=yqyqyq,fill=yqyqyq,fill opacity=0.15000000596046448] (0.,-1.) -- (0.,5.) -- (0.4,5.) -- (0.4,-1.) -- cycle;
\fill[line width=1.pt,color=yqyqyq,fill=yqyqyq,fill opacity=0.15000000596046448] (2.,-1.) -- (2.,5.) -- (2.4,5.) -- (2.4,-1.) -- cycle;
\fill[line width=1.pt,color=yqyqyq,fill=yqyqyq,fill opacity=0.15000000596046448] (4.,-1.) -- (4.,5.) -- (4.4,5.) -- (4.4,-1.) -- cycle;
\fill[line width=1.pt,color=yqyqyq,fill=yqyqyq,fill opacity=0.15000000596046448] (6.,-1.) -- (6.,5.) -- (6.4,5.) -- (6.4,-1.) -- cycle;
\fill[line width=1.pt,color=yqyqyq,fill=yqyqyq,fill opacity=0.15000000596046448] (8.,-1.) -- (8.,5.) -- (8.4,5.) -- (8.4,-1.) -- cycle;
\fill[line width=1.pt,color=yqyqyq,fill=yqyqyq,fill opacity=0.15000000596046448] (10.,-1.) -- (10.,5.) -- (10.4,5.) -- (10.4,-1.) -- cycle;
\fill[line width=1.pt,color=wqwqwq,fill=wqwqwq,fill opacity=0.550000011920929] (-1.,-1.02) -- (11.4,-1.) -- (11.4,-1.62) -- (-1.,-1.58) -- cycle;
\fill[line width=1.pt,color=xfqqff,fill=xfqqff,fill opacity=0.800000011920929] (-0.1599699272374422,-1.018645112785867) -- (-0.159815194430859,-1.582710273566352) -- (0.6001123817650171,-1.5851616528444032) -- (0.6000280956605215,-1.0174193095231283) -- cycle;
\fill[line width=1.pt,color=ffttww,fill=ffttww,fill opacity=1.0] (1.7799927679688659,-1.0155161406968245) -- (2.6199905827508254,-1.0141613055116923) -- (2.619897815839586,-1.5916770897285148) -- (1.7800355875589222,-1.588967856734061) -- cycle;
\fill[line width=1.pt,color=ffqqqq,fill=ffqqqq,fill opacity=1.0] (3.7999875130397687,-1.0122580846563876) -- (4.580017741889329,-1.0109999713840494) -- (4.58000645154577,-1.598000020811438) -- (3.8196917826037193,-1.595547392847109) -- cycle;
\fill[line width=1.pt,color=qqqqff,fill=qqqqff,fill opacity=1.0] (5.793500443483623,-1.0090427412201877) -- (6.579948023028037,-1.007774277382213) -- (6.580114671023194,-1.6044519828097523) -- (5.799224061931935,-1.6019329808449418) -- cycle;
\fill[line width=1.pt,color=qqwwtt,fill=qqwwtt,fill opacity=1.0] (7.797945721326517,-1.0058097649656024) -- (8.620039490011731,-1.0044838072741746) -- (8.619899896983382,-1.6110319351515594) -- (7.797911394786256,-1.6083803593380204) -- cycle;
\fill[line width=1.pt,color=ffffqq,fill=ffffqq,fill opacity=1.0] (9.819348133489356,-1.614901123011256) -- (10.600072840032881,-1.6174195898065578) -- (10.602994997347249,-1.0012854919397625) -- (9.820004110291077,-1.0025483804672723) -- cycle;
\fill[line width=1.pt,color=ffttww,fill=ffttww,fill opacity=0.75] (-1.,-1.58) -- (-0.9946285256656453,-2.171460492800347) -- (5.4269881747162945,-2.1983855104120114) -- (5.413412630970773,-1.6006884278418414) -- cycle;
\fill[line width=1.pt,color=qqttcc,fill=qqttcc,fill opacity=0.800000011920929] (5.4269881747162945,-2.1983855104120114) -- (11.390879575699941,-2.184923001606179) -- (11.4,-1.62) -- (5.413412630970773,-1.6006884278418414) -- cycle;
\draw [line width=1.pt,color=yqyqyq] (0.,-1.)-- (0.,5.);
\draw [line width=1.pt,color=yqyqyq] (0.,5.)-- (0.4,5.);
\draw [line width=1.pt,color=yqyqyq] (0.4,5.)-- (0.4,-1.);
\draw [line width=1.pt,color=yqyqyq] (0.4,-1.)-- (0.,-1.);
\draw [line width=1.pt,color=yqyqyq] (2.,-1.)-- (2.,5.);
\draw [line width=1.pt,color=yqyqyq] (2.,5.)-- (2.4,5.);
\draw [line width=1.pt,color=yqyqyq] (2.4,5.)-- (2.4,-1.);
\draw [line width=1.pt,color=yqyqyq] (2.4,-1.)-- (2.,-1.);
\draw [line width=1.pt,color=yqyqyq] (4.,-1.)-- (4.,5.);
\draw [line width=1.pt,color=yqyqyq] (4.,5.)-- (4.4,5.);
\draw [line width=1.pt,color=yqyqyq] (4.4,5.)-- (4.4,-1.);
\draw [line width=1.pt,color=yqyqyq] (4.4,-1.)-- (4.,-1.);
\draw [line width=1.pt,color=yqyqyq] (6.,-1.)-- (6.,5.);
\draw [line width=1.pt,color=yqyqyq] (6.,5.)-- (6.4,5.);
\draw [line width=1.pt,color=yqyqyq] (6.4,5.)-- (6.4,-1.);
\draw [line width=1.pt,color=yqyqyq] (6.4,-1.)-- (6.,-1.);
\draw [line width=1.pt,color=yqyqyq] (8.,-1.)-- (8.,5.);
\draw [line width=1.pt,color=yqyqyq] (8.,5.)-- (8.4,5.);
\draw [line width=1.pt,color=yqyqyq] (8.4,5.)-- (8.4,-1.);
\draw [line width=1.pt,color=yqyqyq] (8.4,-1.)-- (8.,-1.);
\draw [line width=1.pt,color=yqyqyq] (10.,-1.)-- (10.,5.);
\draw [line width=1.pt,color=yqyqyq] (10.,5.)-- (10.4,5.);
\draw [line width=1.pt,color=yqyqyq] (10.4,5.)-- (10.4,-1.);
\draw [line width=1.pt,color=yqyqyq] (10.4,-1.)-- (10.,-1.);
\draw [line width=1.pt,color=wqwqwq] (-1.,-1.02)-- (11.4,-1.);
\draw [line width=1.pt,color=wqwqwq] (11.4,-1.)-- (11.4,-1.62);
\draw [line width=1.pt,color=wqwqwq] (11.4,-1.62)-- (-1.,-1.58);
\draw [line width=1.pt,color=wqwqwq] (-1.,-1.58)-- (-1.,-1.02);
\draw [line width=1.pt,color=xfqqff] (-0.1599699272374422,-1.018645112785867)-- (-0.159815194430859,-1.582710273566352);
\draw [line width=1.pt,color=xfqqff] (-0.159815194430859,-1.582710273566352)-- (0.6001123817650171,-1.5851616528444032);
\draw [line width=1.pt,color=xfqqff] (0.6001123817650171,-1.5851616528444032)-- (0.6000280956605215,-1.0174193095231283);
\draw [line width=1.pt,color=xfqqff] (0.6000280956605215,-1.0174193095231283)-- (-0.1599699272374422,-1.018645112785867);
\draw [line width=1.pt,color=ffttww] (1.7799927679688659,-1.0155161406968245)-- (2.6199905827508254,-1.0141613055116923);
\draw [line width=1.pt,color=ffttww] (2.6199905827508254,-1.0141613055116923)-- (2.619897815839586,-1.5916770897285148);
\draw [line width=1.pt,color=ffttww] (2.619897815839586,-1.5916770897285148)-- (1.7800355875589222,-1.588967856734061);
\draw [line width=1.pt,color=ffttww] (1.7800355875589222,-1.588967856734061)-- (1.7799927679688659,-1.0155161406968245);
\draw [line width=1.pt,color=ffqqqq] (3.7999875130397687,-1.0122580846563876)-- (4.580017741889329,-1.0109999713840494);
\draw [line width=1.pt,color=ffqqqq] (4.580017741889329,-1.0109999713840494)-- (4.58000645154577,-1.598000020811438);
\draw [line width=1.pt,color=ffqqqq] (4.58000645154577,-1.598000020811438)-- (3.8196917826037193,-1.595547392847109);
\draw [line width=1.pt,color=ffqqqq] (3.8196917826037193,-1.595547392847109)-- (3.7999875130397687,-1.0122580846563876);
\draw [line width=1.pt,color=qqqqff] (5.793500443483623,-1.0090427412201877)-- (6.579948023028037,-1.007774277382213);
\draw [line width=1.pt,color=qqqqff] (6.579948023028037,-1.007774277382213)-- (6.580114671023194,-1.6044519828097523);
\draw [line width=1.pt,color=qqqqff] (6.580114671023194,-1.6044519828097523)-- (5.799224061931935,-1.6019329808449418);
\draw [line width=1.pt,color=qqqqff] (5.799224061931935,-1.6019329808449418)-- (5.793500443483623,-1.0090427412201877);
\draw [line width=1.pt,color=qqwwtt] (7.797945721326517,-1.0058097649656024)-- (8.620039490011731,-1.0044838072741746);
\draw [line width=1.pt,color=qqwwtt] (8.620039490011731,-1.0044838072741746)-- (8.619899896983382,-1.6110319351515594);
\draw [line width=1.pt,color=qqwwtt] (8.619899896983382,-1.6110319351515594)-- (7.797911394786256,-1.6083803593380204);
\draw [line width=1.pt,color=qqwwtt] (7.797911394786256,-1.6083803593380204)-- (7.797945721326517,-1.0058097649656024);
\draw [line width=1.pt,color=ffffqq] (9.819348133489356,-1.614901123011256)-- (10.600072840032881,-1.6174195898065578);
\draw [line width=1.pt,color=ffffqq] (10.600072840032881,-1.6174195898065578)-- (10.602994997347249,-1.0012854919397625);
\draw [line width=1.pt,color=ffffqq] (10.602994997347249,-1.0012854919397625)-- (9.820004110291077,-1.0025483804672723);
\draw [line width=1.pt,color=ffffqq] (9.820004110291077,-1.0025483804672723)-- (9.819348133489356,-1.614901123011256);
\draw [line width=1.pt,color=ffttww] (-1.,-1.58)-- (-0.9946285256656453,-2.171460492800347);
\draw [line width=1.pt,color=ffttww] (-0.9946285256656453,-2.171460492800347)-- (5.4269881747162945,-2.1983855104120114);
\draw [line width=1.pt,color=ffttww] (5.4269881747162945,-2.1983855104120114)-- (5.413412630970773,-1.6006884278418414);
\draw [line width=1.pt,color=ffttww] (5.413412630970773,-1.6006884278418414)-- (-1.,-1.58);
\draw [line width=1.pt,color=qqttcc] (5.4269881747162945,-2.1983855104120114)-- (11.390879575699941,-2.184923001606179);
\draw [line width=1.pt,color=qqttcc] (11.390879575699941,-2.184923001606179)-- (11.4,-1.62);
\draw [line width=1.pt,color=qqttcc] (11.4,-1.62)-- (5.413412630970773,-1.6006884278418414);
\draw [line width=1.pt,color=qqttcc] (5.413412630970773,-1.6006884278418414)-- (5.4269881747162945,-2.1983855104120114);
\draw [color=ffffff](1.555140801222451,-1.8045868611596438) node[anchor=north west] {\textbf{MILLIERS}};
\draw [color=ffffff](7.777427939501565,-1.8045868611596438) node[anchor=north west] {\textbf{UNITÉS}};
\draw [color=ffffff](0.053743158403858855,-1.2473671174331538) node[anchor=north west] {\textbf{C}};
\draw [color=ffffff](2.05044724009044,-1.2473671174331538) node[anchor=north west] {\textbf{D}};
\draw [color=ffffff](4.047151321777021,-1.2473671174331538) node[anchor=north west] {\textbf{U}};
\draw [color=ffffff](8.056037811364808,-1.2473671174331538) node[anchor=north west] {\textbf{D}};
\draw (10.05274189305139,-1.2473671174331538) node[anchor=north west] {\textbf{U}};
\draw [color=ffffff](6.043855403463603,-1.2473671174331538) node[anchor=north west] {\textbf{C}};
\draw [rotate around={-0.7638984609300066:(0.16131752559549964,-0.6940169552603201)},line width=1.pt,color=wwqqcc,fill=wwqqcc,fill opacity=0.75] (0.16131752559549964,-0.6940169552603201) ellipse (0.3417694071893479cm and 0.15229751977382844cm);
\draw [rotate around={-0.7638984609300061:(0.1783436844315867,0.004055557019254948)},line width=1.pt,color=wwqqcc,fill=wwqqcc,fill opacity=0.75] (0.1783436844315867,0.004055557019254948) ellipse (0.3417694071893479cm and 0.15229751977382855cm);
\draw [rotate around={-0.7638984609300069:(0.1953698432676738,0.7021280692988301)},line width=1.pt,color=wwqqcc,fill=wwqqcc,fill opacity=0.75] (0.1953698432676738,0.7021280692988301) ellipse (0.34176940718934784cm and 0.15229751977382852cm);
\draw [rotate around={-0.7638984609300069:(0.19536984326767365,1.4002005815784049)},line width=1.pt,color=wwqqcc,fill=wwqqcc,fill opacity=0.75] (0.19536984326767365,1.4002005815784049) ellipse (0.3417694071893482cm and 0.1522975197738287cm);
\draw [rotate around={-0.763898460930007:(6.120473118225988,-0.6630603028310708)},line width=1.pt,color=qqqqcc,fill=qqqqcc,fill opacity=0.75] (6.120473118225988,-0.6630603028310708) ellipse (0.3417694071893227cm and 0.15229751977381722cm);
\draw [rotate around={-0.7638984609300075:(10.13090744043527,-0.6150774915657343)},line width=1.pt,color=ffdxqq,fill=ffdxqq,fill opacity=0.75] (10.13090744043527,-0.6150774915657343) ellipse (0.3417694071894155cm and 0.15229751977385875cm);
\draw [rotate around={-0.7638984609300066:(2.219934912140578,-0.6429384787520583)},line width=1.pt,color=ffttww,fill=ffttww,fill opacity=0.75] (2.219934912140578,-0.6429384787520583) ellipse (0.34176940718934695cm and 0.15229751977382802cm);
\draw [rotate around={-0.7638984609300068:(4.179491010912057,-0.6429384787520582)},line width=1.pt,color=ccqqqq,fill=ccqqqq,fill opacity=0.75] (4.179491010912057,-0.6429384787520582) ellipse (0.3417694071893353cm and 0.15229751977382303cm);
\draw [rotate around={-0.7638984609300074:(4.170204015183288,0.7299890564851544)},line width=1.pt,color=ccqqqq,fill=ccqqqq,fill opacity=0.75] (4.170204015183288,0.7299890564851544) ellipse (0.34176940718934157cm and 0.1522975197738258cm);
\draw [rotate around={-0.7638984609300068:(4.202708500233998,0.05049053566312913)},line width=1.pt,color=ccqqqq,fill=ccqqqq,fill opacity=0.75] (4.202708500233998,0.05049053566312913) ellipse (0.3417694071893534cm and 0.15229751977383105cm);
\draw [rotate around={-0.7638984609300066:(2.1765955987396324,0.03656004206996701)},line width=1.pt,color=ffttww,fill=ffttww,fill opacity=0.75] (2.1765955987396324,0.03656004206996701) ellipse (0.34176940718934956cm and 0.1522975197738292cm);
\draw [rotate around={-0.7638984609300065:(10.116976946842083,0.08454285333530363)},line width=1.pt,color=ffdxqq,fill=ffdxqq,fill opacity=0.75] (10.116976946842083,0.08454285333530363) ellipse (0.34176940718934506cm and 0.15229751977382705cm);
\draw [rotate around={-0.7638984609300076:(10.132455273056683,0.7810675329934165)},line width=1.pt,color=ffdxqq,fill=ffdxqq,fill opacity=0.75] (10.132455273056683,0.7810675329934165) ellipse (0.34176940718933235cm and 0.1522975197738219cm);
\draw [rotate around={-0.7638984609300076:(10.163411925485965,1.462113886436904)},line width=1.pt,color=ffdxqq,fill=ffdxqq,fill opacity=0.75] (10.163411925485965,1.462113886436904) ellipse (0.341769407189356cm and 0.1522975197738324cm);
\draw [rotate around={-0.7638984609300273:(10.163411925485962,2.1586385660950165)},line width=1.pt,color=ffdxqq,fill=ffdxqq,fill opacity=0.75] (10.163411925485962,2.1586385660950165) ellipse (0.3417694071893621cm and 0.15229751977383504cm);
\end{tikzpicture}


\ExoCd{Représenter.}


Écris le nombre $\np{793034056}$ sur le boulier chinois en coloriant des boules.


\begin{tikzpicture}[line cap=round,line join=round,>=triangle 45,x=0.4791579887901929cm,y=0.4791579887901929cm]
\clip(-1.0142613504052427,-2.28441497381301) rectangle (11.507704557226107,5.3309215237823535);
\fill[line width=2.pt,color=yqyqyq,fill=yqyqyq,fill opacity=0.15000000596046448] (0.,-1.) -- (0.,5.) -- (0.4,5.) -- (0.4,-1.) -- cycle;
\fill[line width=1.pt,color=yqyqyq,fill=yqyqyq,fill opacity=0.15000000596046448] (2.,-1.) -- (2.,5.) -- (2.4,5.) -- (2.4,-1.) -- cycle;
\fill[line width=1.pt,color=yqyqyq,fill=yqyqyq,fill opacity=0.15000000596046448] (4.,-1.) -- (4.,5.) -- (4.4,5.) -- (4.4,-1.) -- cycle;
\fill[line width=1.pt,color=yqyqyq,fill=yqyqyq,fill opacity=0.15000000596046448] (6.,-1.) -- (6.,5.) -- (6.4,5.) -- (6.4,-1.) -- cycle;
\fill[line width=1.pt,color=yqyqyq,fill=yqyqyq,fill opacity=0.15000000596046448] (8.,-1.) -- (8.,5.) -- (8.4,5.) -- (8.4,-1.) -- cycle;
\fill[line width=1.pt,color=yqyqyq,fill=yqyqyq,fill opacity=0.15000000596046448] (10.,-1.) -- (10.,5.) -- (10.4,5.) -- (10.4,-1.) -- cycle;
\fill[line width=1.pt,color=wqwqwq,fill=wqwqwq,fill opacity=0.550000011920929] (-1.,-1.02) -- (11.4,-1.) -- (11.4,-1.62) -- (-1.,-1.58) -- cycle;
\fill[line width=1.pt,color=xfqqff,fill=xfqqff,fill opacity=0.800000011920929] (-0.1599699272374422,-1.018645112785867) -- (-0.159815194430859,-1.582710273566352) -- (0.6001123817650171,-1.5851616528444032) -- (0.6000280956605215,-1.0174193095231283) -- cycle;
\fill[line width=1.pt,color=ffttww,fill=ffttww,fill opacity=1.0] (1.7799927679688659,-1.0155161406968245) -- (2.6199905827508254,-1.0141613055116923) -- (2.619897815839586,-1.5916770897285148) -- (1.7800355875589222,-1.588967856734061) -- cycle;
\fill[line width=1.pt,color=ffqqqq,fill=ffqqqq,fill opacity=1.0] (3.7999875130397687,-1.0122580846563876) -- (4.580017741889329,-1.0109999713840494) -- (4.58000645154577,-1.598000020811438) -- (3.8196917826037193,-1.595547392847109) -- cycle;
\fill[line width=1.pt,color=qqqqff,fill=qqqqff,fill opacity=1.0] (5.793500443483623,-1.0090427412201877) -- (6.579948023028037,-1.007774277382213) -- (6.580114671023194,-1.6044519828097523) -- (5.799224061931935,-1.6019329808449418) -- cycle;
\fill[line width=1.pt,color=qqwwtt,fill=qqwwtt,fill opacity=1.0] (7.797945721326517,-1.0058097649656024) -- (8.620039490011731,-1.0044838072741746) -- (8.619899896983382,-1.6110319351515594) -- (7.797911394786256,-1.6083803593380204) -- cycle;
\fill[line width=1.pt,color=ffffqq,fill=ffffqq,fill opacity=1.0] (9.819348133489356,-1.614901123011256) -- (10.600072840032881,-1.6174195898065578) -- (10.602994997347249,-1.0012854919397625) -- (9.820004110291077,-1.0025483804672723) -- cycle;
\fill[line width=1.pt,color=ffttww,fill=ffttww,fill opacity=0.75] (-1.,-1.58) -- (-0.9946285256656453,-2.171460492800347) -- (5.4269881747162945,-2.1983855104120114) -- (5.413412630970773,-1.6006884278418414) -- cycle;
\fill[line width=1.pt,color=qqttcc,fill=qqttcc,fill opacity=0.800000011920929] (5.4269881747162945,-2.1983855104120114) -- (11.390879575699941,-2.184923001606179) -- (11.4,-1.62) -- (5.413412630970773,-1.6006884278418414) -- cycle;
\draw [line width=1.pt,color=yqyqyq] (0.,-1.)-- (0.,5.);
\draw [line width=1.pt,color=yqyqyq] (0.,5.)-- (0.4,5.);
\draw [line width=1.pt,color=yqyqyq] (0.4,5.)-- (0.4,-1.);
\draw [line width=1.pt,color=yqyqyq] (0.4,-1.)-- (0.,-1.);
\draw [line width=1.pt,color=yqyqyq] (2.,-1.)-- (2.,5.);
\draw [line width=1.pt,color=yqyqyq] (2.,5.)-- (2.4,5.);
\draw [line width=1.pt,color=yqyqyq] (2.4,5.)-- (2.4,-1.);
\draw [line width=1.pt,color=yqyqyq] (2.4,-1.)-- (2.,-1.);
\draw [line width=1.pt,color=yqyqyq] (4.,-1.)-- (4.,5.);
\draw [line width=1.pt,color=yqyqyq] (4.,5.)-- (4.4,5.);
\draw [line width=1.pt,color=yqyqyq] (4.4,5.)-- (4.4,-1.);
\draw [line width=1.pt,color=yqyqyq] (4.4,-1.)-- (4.,-1.);
\draw [line width=1.pt,color=yqyqyq] (6.,-1.)-- (6.,5.);
\draw [line width=1.pt,color=yqyqyq] (6.,5.)-- (6.4,5.);
\draw [line width=1.pt,color=yqyqyq] (6.4,5.)-- (6.4,-1.);
\draw [line width=1.pt,color=yqyqyq] (6.4,-1.)-- (6.,-1.);
\draw [line width=1.pt,color=yqyqyq] (8.,-1.)-- (8.,5.);
\draw [line width=1.pt,color=yqyqyq] (8.,5.)-- (8.4,5.);
\draw [line width=1.pt,color=yqyqyq] (8.4,5.)-- (8.4,-1.);
\draw [line width=1.pt,color=yqyqyq] (8.4,-1.)-- (8.,-1.);
\draw [line width=1.pt,color=yqyqyq] (10.,-1.)-- (10.,5.);
\draw [line width=1.pt,color=yqyqyq] (10.,5.)-- (10.4,5.);
\draw [line width=1.pt,color=yqyqyq] (10.4,5.)-- (10.4,-1.);
\draw [line width=1.pt,color=yqyqyq] (10.4,-1.)-- (10.,-1.);
\draw [line width=1.pt,color=wqwqwq] (-1.,-1.02)-- (11.4,-1.);
\draw [line width=1.pt,color=wqwqwq] (11.4,-1.)-- (11.4,-1.62);
\draw [line width=1.pt,color=wqwqwq] (11.4,-1.62)-- (-1.,-1.58);
\draw [line width=1.pt,color=wqwqwq] (-1.,-1.58)-- (-1.,-1.02);
\draw [line width=1.pt,color=xfqqff] (-0.1599699272374422,-1.018645112785867)-- (-0.159815194430859,-1.582710273566352);
\draw [line width=1.pt,color=xfqqff] (-0.159815194430859,-1.582710273566352)-- (0.6001123817650171,-1.5851616528444032);
\draw [line width=1.pt,color=xfqqff] (0.6001123817650171,-1.5851616528444032)-- (0.6000280956605215,-1.0174193095231283);
\draw [line width=1.pt,color=xfqqff] (0.6000280956605215,-1.0174193095231283)-- (-0.1599699272374422,-1.018645112785867);
\draw [line width=1.pt,color=ffttww] (1.7799927679688659,-1.0155161406968245)-- (2.6199905827508254,-1.0141613055116923);
\draw [line width=1.pt,color=ffttww] (2.6199905827508254,-1.0141613055116923)-- (2.619897815839586,-1.5916770897285148);
\draw [line width=1.pt,color=ffttww] (2.619897815839586,-1.5916770897285148)-- (1.7800355875589222,-1.588967856734061);
\draw [line width=1.pt,color=ffttww] (1.7800355875589222,-1.588967856734061)-- (1.7799927679688659,-1.0155161406968245);
\draw [line width=1.pt,color=ffqqqq] (3.7999875130397687,-1.0122580846563876)-- (4.580017741889329,-1.0109999713840494);
\draw [line width=1.pt,color=ffqqqq] (4.580017741889329,-1.0109999713840494)-- (4.58000645154577,-1.598000020811438);
\draw [line width=1.pt,color=ffqqqq] (4.58000645154577,-1.598000020811438)-- (3.8196917826037193,-1.595547392847109);
\draw [line width=1.pt,color=ffqqqq] (3.8196917826037193,-1.595547392847109)-- (3.7999875130397687,-1.0122580846563876);
\draw [line width=1.pt,color=qqqqff] (5.793500443483623,-1.0090427412201877)-- (6.579948023028037,-1.007774277382213);
\draw [line width=1.pt,color=qqqqff] (6.579948023028037,-1.007774277382213)-- (6.580114671023194,-1.6044519828097523);
\draw [line width=1.pt,color=qqqqff] (6.580114671023194,-1.6044519828097523)-- (5.799224061931935,-1.6019329808449418);
\draw [line width=1.pt,color=qqqqff] (5.799224061931935,-1.6019329808449418)-- (5.793500443483623,-1.0090427412201877);
\draw [line width=1.pt,color=qqwwtt] (7.797945721326517,-1.0058097649656024)-- (8.620039490011731,-1.0044838072741746);
\draw [line width=1.pt,color=qqwwtt] (8.620039490011731,-1.0044838072741746)-- (8.619899896983382,-1.6110319351515594);
\draw [line width=1.pt,color=qqwwtt] (8.619899896983382,-1.6110319351515594)-- (7.797911394786256,-1.6083803593380204);
\draw [line width=1.pt,color=qqwwtt] (7.797911394786256,-1.6083803593380204)-- (7.797945721326517,-1.0058097649656024);
\draw [line width=1.pt,color=ffffqq] (9.819348133489356,-1.614901123011256)-- (10.600072840032881,-1.6174195898065578);
\draw [line width=1.pt,color=ffffqq] (10.600072840032881,-1.6174195898065578)-- (10.602994997347249,-1.0012854919397625);
\draw [line width=1.pt,color=ffffqq] (10.602994997347249,-1.0012854919397625)-- (9.820004110291077,-1.0025483804672723);
\draw [line width=1.pt,color=ffffqq] (9.820004110291077,-1.0025483804672723)-- (9.819348133489356,-1.614901123011256);
\draw [line width=1.pt,color=ffttww] (-1.,-1.58)-- (-0.9946285256656453,-2.171460492800347);
\draw [line width=1.pt,color=ffttww] (-0.9946285256656453,-2.171460492800347)-- (5.4269881747162945,-2.1983855104120114);
\draw [line width=1.pt,color=ffttww] (5.4269881747162945,-2.1983855104120114)-- (5.413412630970773,-1.6006884278418414);
\draw [line width=1.pt,color=ffttww] (5.413412630970773,-1.6006884278418414)-- (-1.,-1.58);
\draw [line width=1.pt,color=qqttcc] (5.4269881747162945,-2.1983855104120114)-- (11.390879575699941,-2.184923001606179);
\draw [line width=1.pt,color=qqttcc] (11.390879575699941,-2.184923001606179)-- (11.4,-1.62);
\draw [line width=1.pt,color=qqttcc] (11.4,-1.62)-- (5.413412630970773,-1.6006884278418414);
\draw [line width=1.pt,color=qqttcc] (5.413412630970773,-1.6006884278418414)-- (5.4269881747162945,-2.1983855104120114);
\draw [color=ffffff](1.555140801222451,-1.8045868611596438) node[anchor=north west] {\textbf{MILLIERS}};
\draw [color=ffffff](7.777427939501565,-1.8045868611596438) node[anchor=north west] {\textbf{UNITÉS}};
\draw [color=ffffff](0.053743158403858855,-1.2473671174331538) node[anchor=north west] {\textbf{C}};
\draw [color=ffffff](2.05044724009044,-1.2473671174331538) node[anchor=north west] {\textbf{D}};
\draw [color=ffffff](4.047151321777021,-1.2473671174331538) node[anchor=north west] {\textbf{U}};
\draw [color=ffffff](8.056037811364808,-1.2473671174331538) node[anchor=north west] {\textbf{D}};
\draw (10.05274189305139,-1.2473671174331538) node[anchor=north west] {\textbf{U}};
\draw [color=ffffff](6.043855403463603,-1.2473671174331538) node[anchor=north west] {\textbf{C}};
\end{tikzpicture}




\ExoCu{Représenter.}

Asma a écrit le nombre 15,3062 de plusieurs façons.

15 unités et 3 062 dix-millièmes ; 153 062 dix-millièmes ; $(1\times 10) + (5\times 1) + \dfrac{3}{10}+ \dfrac{6}{1000}+ \dfrac{2}{10000} $; $15+ \dfrac{3062}{10000}$.


Peux tu l'aider à écrire le nombre 45,3701 ?

$\cdots\cdots\cdots\cdots\cdots\cdots\cdots\cdots\cdots\cdots\cdots \quad ;\quad \cdots\cdots\cdots\cdots\cdots\cdots\cdots\cdots \quad ; \quad \cdots\cdots\cdots\cdots\cdots\cdots\cdots\cdots \quad ;\quad \cdots\cdots\cdots\cdots\cdots\cdots\cdots\cdots $
 

\ExoCu{Représenter.}

À partir des renseignements qui suivent, trouve le nombre caché :

\begin{enumerate}
\item C’est un nombre décimal de 5 chiffres.
\item  Son chiffre des dixièmes est le même que celui de 17,54.
\item  Son chiffre des centièmes est le chiffre des unités de millions de 738 214 006.
\item  Son chiffre des unités est le chiffre des dizaines de mille de 120 008.
\item  Son chiffre des millièmes est la moitié de celui des centièmes.
\item  Son chiffre des dix-millièmes est égal au chiffre des unités.
\end{enumerate}




\ExoCd{Représenter.}

\begin{enumerate}
\item Quel est le nombre décimal écrit ? $A = 9\times 1000 + 6\times 100 + 5\times 10 + 2 + 8\times \dfrac{1}{10} +  9\times \dfrac{1}{100} $.

$A = \cdots\cdots\cdots\cdots\cdots\cdots\cdots\cdots $

\item Quel est le nombre décimal écrit ? $B = 3\times 1000 + 2\times 100 + 1\times 10 + 6 + 8\times \dfrac{1}{10} +  5\times \dfrac{1}{1000} $.

$B = \cdots\cdots\cdots\cdots\cdots\cdots\cdots\cdots $
\end{enumerate}






\subsection{Savoir faire le lien entre "multiplier par 0,5" et la "moitié de".}


\ExoCu{Représenter. Communiquer}

Traduis par une phrase l'opération mathématique chacune des phrase suivantes :
\begin{enumerate}
\item Pierre a bu la moitié de la bouteille de 25 cl. Quelle quantité d'eau a-t-il bu ?

$\cdots\cdots\cdots\cdots\cdots\cdots\cdots\cdots\cdots\cdots\cdots\cdots\cdots\cdots\cdots\cdots\cdots\cdots\cdots\cdots\cdots\cdots\cdots\cdots $

\item Sasha pèse 46kg. Sa petite sœur Anabel pèse la moitié de son poids*. Quel est le poids d'Anabel ?

$\cdots\cdots\cdots\cdots\cdots\cdots\cdots\cdots\cdots\cdots\cdots\cdots\cdots\cdots\cdots\cdots\cdots\cdots\cdots\cdots\cdots\cdots\cdots\cdots $

\textit{*On devrait dire la masse. Le poids est une donnée de Sciences Physiques... A ne pas confondre avec le pois}
\end{enumerate}



\subsection{Ajouter des fractions décimales de même dénominateur.}


\ExoCu{Calculer.}

Calculer

\begin{enumerate}
\item $A = \dfrac{12}{10} +\dfrac{5}{10} = \dfrac{\cdots\cdots\cdots }{10}  $
\item $B = \dfrac{37}{100} +\dfrac{15}{100} = \dfrac{\cdots\cdots\cdots }{100}  $
\item $C = \dfrac{25}{10} +\dfrac{6}{10} = \dfrac{\cdots\cdots\cdots }{\cdots\cdots\cdots } $
\end{enumerate}



\ExoCu{Calculer.}

Calculer

\begin{enumerate}
\item $A = \dfrac{7}{4} +\dfrac{3}{4} = \dfrac{\cdots\cdots\cdots }{4}  $
\item $B = \dfrac{6}{5} +\dfrac{13}{5} = \dfrac{\cdots\cdots\cdots }{5}  $
\item $C = \dfrac{4}{2} +\dfrac{5}{2} = \dfrac{\cdots\cdots\cdots }{\cdots\cdots\cdots } $
\end{enumerate}





\ExoCd{Calculer.}

Calculer

\begin{enumerate}
\item $A = \dfrac{12}{10} +\dfrac{5}{10} =  \cdots\cdots\cdots  $
\item $B = \dfrac{37}{100} +\dfrac{15}{100} =  \cdots\cdots\cdots  $
\item $C = \dfrac{25}{10} +\dfrac{6}{10} =  \cdots\cdots\cdots $
\end{enumerate}



\ExoCd{Calculer.}

Calculer

\begin{enumerate}
\item $A = \dfrac{7}{2} +\dfrac{3}{2} = \dfrac{\cdots\cdots\cdots }{2} =  \cdots\cdots\cdots  $
\item $B = \dfrac{7}{5} +\dfrac{6}{5} = \dfrac{\cdots\cdots\cdots }{5} $
\item $C = \dfrac{11}{4} +\dfrac{5}{4} = \dfrac{\cdots\cdots\cdots }{\cdots\cdots\cdots } $
\end{enumerate}





\subsection{Ajouter des fractions de même dénominateur.}


\ExoCu{Calculer.}

Calculer

\begin{enumerate}
\item $A = \dfrac{12}{3} +\dfrac{5}{3} = \dfrac{\cdots\cdots\cdots }{3}  $
\item $B = \dfrac{37}{7} +\dfrac{15}{7} = \dfrac{\cdots\cdots\cdots }{7}  $
\item $C = \dfrac{25}{11} +\dfrac{6}{11} = \dfrac{\cdots\cdots\cdots }{\cdots\cdots\cdots } $
\end{enumerate}





\subsection{Savoir utiliser des fractions pour exprimer un quotient.}


\ExoCu{Représenter. Communiquer.}


Pour chaque cas, écris la fraction dont :
\begin{enumerate}
\item le numérateur est 13 et le dénominateur est 4 : $\dfrac{\cdots\cdots}{\cdots\cdots} $
\item le dénominateur est 3 et le numérateur est 7  : $\dfrac{\cdots\cdots}{\cdots\cdots} $
\item le numérateur est 5 et le dénominateur est 2 : $\dfrac{\cdots\cdots}{\cdots\cdots} $
\end{enumerate}


\ExoCu{Représenter. Communiquer.}

\begin{enumerate}
\item Voici le nombre $\dfrac{3}{7}$. Le numérateur est $\cdots\cdots$
\item Voici le nombre $\dfrac{5}{2}$. Le numérateur est $\cdots\cdots$
\item Voici le nombre $\dfrac{4}{3}$. Le dénominateur est $cdots\cdots$
\end{enumerate}


\ExoCu{Calculer.}

Calcule chaque produit
\begin{enumerate}
\item $\dfrac{5}{3} \times 3 = \cdots\cdots $
\item $\dfrac{3}{4} \times 4 = \cdots\cdots $
\item $\dfrac{23}{7} \times 7 = \cdots\cdots $
\end{enumerate}



\subsection{Savoir placer une fraction sur la droite graduée.}

\ExoCd{Représenter.}

\begin{enumerate}
\item  Place les nombres $0,5$ , $0,8$ et $0,25$ dans les étiquettes.

\definecolor{qqwwzz}{rgb}{0.,0.4,0.6}
\begin{tikzpicture}[line cap=round,line join=round,>=triangle 45,x=0.5cm,y=0.5cm]
\clip(0.76,0.78) rectangle (11.56,5.28);
\draw [line width=1.pt] (1.,3.)-- (11.,3.);
\draw [line width=1.pt] (6.,3.)-- (6.,4.);
\draw [line width=1.pt] (3.5,3.)-- (3.5,2.);
\draw [line width=1.pt] (9.,3.)-- (9.,2.);
\draw [line width=1.pt,color=qqwwzz] (2.6,2.)-- (4.5,2.);
\draw [line width=1.pt,color=qqwwzz] (4.5,2.)-- (4.5,1.);
\draw [line width=1.pt,color=qqwwzz] (4.5,1.)-- (2.6,1.);
\draw [line width=1.pt,color=qqwwzz] (2.6,1.)-- (2.6,2.);
\draw [line width=1.pt,color=qqwwzz] (8.,2.)-- (9.9,2.);
\draw [line width=1.pt,color=qqwwzz] (9.9,2.)-- (9.9,1.);
\draw [line width=1.pt,color=qqwwzz] (9.9,1.)-- (8.,1.);
\draw [line width=1.pt,color=qqwwzz] (8.,1.)-- (8.,2.);
\draw [line width=1.pt,color=qqwwzz] (5.,5.)-- (6.9,5.);
\draw [line width=1.pt,color=qqwwzz] (6.9,5.)-- (6.9,4.);
\draw [line width=1.pt,color=qqwwzz] (6.9,4.)-- (5.,4.);
\draw [line width=1.pt,color=qqwwzz] (5.,4.)-- (5.,5.);
\draw (0.82,3.14) node[anchor=north west] {$0$};
\draw (10.86,3.12) node[anchor=north west] {$1$};
\begin{scriptsize}
\draw [color=black] (1.,3.)-- ++(-2.5pt,0 pt) -- ++(5.0pt,0 pt) ++(-2.5pt,-2.5pt) -- ++(0 pt,5.0pt);
\draw [color=black] (11.,3.)-- ++(-2.5pt,0 pt) -- ++(5.0pt,0 pt) ++(-2.5pt,-2.5pt) -- ++(0 pt,5.0pt);
\end{scriptsize}
\end{tikzpicture}



\item Place les nombres $3,2$, $3,75$ et  $3,125$  dans les étiquettes.

\definecolor{qqwwzz}{rgb}{0.,0.4,0.6}
\definecolor{ududff}{rgb}{0.30196078431372547,0.30196078431372547,1.}
\definecolor{xdxdff}{rgb}{0.49019607843137253,0.49019607843137253,1.}
\begin{tikzpicture}[line cap=round,line join=round,>=triangle 45,x=0.5cm,y=0.5cm]
\clip(-0.52,0.66) rectangle (10.4,5.38);
\draw [line width=1.pt] (0.,3.)-- (10.,3.);
\draw [line width=1.pt] (1.25,3.)-- (1.25,4);
\draw [line width=1.pt] (2.,3.)-- (2.,2.);
\draw [line width=1.pt] (7.5,3.)-- (7.5,2);
\draw [line width=1.pt,color=qqwwzz] (1.,2.)-- (3,2.);
\draw [line width=1.pt,color=qqwwzz] (3,2.)-- (3.,1.);
\draw [line width=1.pt,color=qqwwzz] (3.,1.)-- (1.1,1.);
\draw [line width=1.pt,color=qqwwzz] (1,1.)-- (1.,2.);
\draw [line width=1.pt,color=qqwwzz] (6.6,2)-- (8.5,2);
\draw [line width=1.pt,color=qqwwzz] (8.5,2)-- (8.5,1);
\draw [line width=1.pt,color=qqwwzz] (8.5,1)-- (6.6,1);
\draw [line width=1.pt,color=qqwwzz] (6.6,1)-- (6.6,2);
\draw [line width=1.pt,color=qqwwzz] (0.3,5.)-- (2.3,5.);
\draw [line width=1.pt,color=qqwwzz] (2.3,5.)-- (2.3,4.);
\draw [line width=1.pt,color=qqwwzz] (2.3,4.)-- (0.3,4.);
\draw [line width=1.pt,color=qqwwzz] (0.3,4.)-- (0.3,5.);
\draw (-0.2,3.04) node[anchor=north west] {$3$};
\draw (9.84,3.08) node[anchor=north west] {$4$};
\begin{scriptsize}
\draw [color=black] (0.,3.)-- ++(-2.5pt,0 pt) -- ++(5.0pt,0 pt) ++(-2.5pt,-2.5pt) -- ++(0 pt,5.0pt);
\draw [color=black] (10.,3.)-- ++(-2.5pt,0 pt) -- ++(5.0pt,0 pt) ++(-2.5pt,-2.5pt) -- ++(0 pt,5.0pt);
\end{scriptsize}
\end{tikzpicture}



\end{enumerate}



\ExoCu{Représenter.}


  On a placé les points des points A, V et E. 

\definecolor{uuuuuu}{rgb}{0.26666666666666666,0.26666666666666666,0.26666666666666666}
\begin{tikzpicture}[line cap=round,line join=round,>=triangle 45,x=1.0cm,y=1.0cm]
\clip(-4.78,0.94) rectangle (11.84,3.2);
\draw [line width=1.pt,domain=-4.78:11.84] plot(\x,{(--32.-0.*\x)/16.});
\draw (-4.18,2.) node[anchor=north west] {$0$};
\draw (-1.18,2.04) node[anchor=north west] {$1$};
\begin{scriptsize}
\draw [color=black] (-4.,2.)-- ++(-3.5pt,0 pt) -- ++(7.0pt,0 pt) ++(-3.5pt,-3.5pt) -- ++(0 pt,7.0pt);
\draw[color=black] (-6.18,1.85) node {$f$};
\draw [color=black] (-1.,2.)-- ++(-3.5pt,0 pt) -- ++(7.0pt,0 pt) ++(-3.5pt,-3.5pt) -- ++(0 pt,7.0pt);
\draw [color=black] (2.,2.)-- ++(-3.5pt,0 pt) -- ++(7.0pt,0 pt) ++(-3.5pt,-3.5pt) -- ++(0 pt,7.0pt);
\draw [color=black] (5.,2.)-- ++(-3.5pt,0 pt) -- ++(7.0pt,0 pt) ++(-3.5pt,-3.5pt) -- ++(0 pt,7.0pt);
\draw [color=black] (8.,2.)-- ++(-3.5pt,0 pt) -- ++(7.0pt,0 pt) ++(-3.5pt,-3.5pt) -- ++(0 pt,7.0pt);
\draw[color=black] (7.92,2.49) node {$E$};
\draw [color=black] (11.,2.)-- ++(-3.5pt,0 pt) -- ++(7.0pt,0 pt) ++(-3.5pt,-3.5pt) -- ++(0 pt,7.0pt);
\draw [color=black] (-3.,2.)-- ++(-2.5pt,0 pt) -- ++(5.0pt,0 pt) ++(-2.5pt,-2.5pt) -- ++(0 pt,5.0pt);
\draw [color=black] (-2.,2.)-- ++(-2.5pt,0 pt) -- ++(5.0pt,0 pt) ++(-2.5pt,-2.5pt) -- ++(0 pt,5.0pt);
\draw [color=uuuuuu] (0.,2.)-- ++(-2.5pt,0 pt) -- ++(5.0pt,0 pt) ++(-2.5pt,-2.5pt) -- ++(0 pt,5.0pt);
\draw[color=uuuuuu] (-0.04,2.47) node {$A$};
\draw [color=black] (1.,2.)-- ++(-2.5pt,0 pt) -- ++(5.0pt,0 pt) ++(-2.5pt,-2.5pt) -- ++(0 pt,5.0pt);
\draw [color=black] (3.,2.)-- ++(-2.5pt,0 pt) -- ++(5.0pt,0 pt) ++(-2.5pt,-2.5pt) -- ++(0 pt,5.0pt);
\draw [color=black] (4.,2.)-- ++(-2.5pt,0 pt) -- ++(5.0pt,0 pt) ++(-2.5pt,-2.5pt) -- ++(0 pt,5.0pt);
\draw[color=black] (4.,2.45) node {$V$};
\draw [color=black] (6.,2.)-- ++(-2.5pt,0 pt) -- ++(5.0pt,0 pt) ++(-2.5pt,-2.5pt) -- ++(0 pt,5.0pt);
\draw [color=black] (7.,2.)-- ++(-2.5pt,0 pt) -- ++(5.0pt,0 pt) ++(-2.5pt,-2.5pt) -- ++(0 pt,5.0pt);
\draw [color=black] (9.,2.)-- ++(-2.5pt,0 pt) -- ++(5.0pt,0 pt) ++(-2.5pt,-2.5pt) -- ++(0 pt,5.0pt);
\draw [color=black] (10.,2.)-- ++(-2.5pt,0 pt) -- ++(5.0pt,0 pt) ++(-2.5pt,-2.5pt) -- ++(0 pt,5.0pt);
\end{scriptsize}
\end{tikzpicture}

 
Écris l'abscisse des points $A \left( \dfrac{ \ldots\ldots }{ \ldots\ldots } \right)$ , $V \left( \dfrac{ \ldots\ldots }{ \ldots\ldots } \right)$ et $E \left(   \ldots\ldots\ldots\ldots   \right)$



\ExoCd{Représenter.}

Place les points $T \left( \dfrac{ 1 }{ 4 } \right)$ , $O \left( 1,75 \right)$ et $P \left(  \dfrac{ 5 }{ 2 } \right)$


\definecolor{uuuuuu}{rgb}{0.26666666666666666,0.26666666666666666,0.26666666666666666}
\begin{tikzpicture}[line cap=round,line join=round,>=triangle 45,x=1.0cm,y=1.0cm]
\clip(-4.88,1.02) rectangle (11.44,2.74);
\draw [line width=1.pt,domain=-4.88:11.44] plot(\x,{(--32.-0.*\x)/16.});
\draw (-4.18,2.) node[anchor=north west] {$0$};
\draw (-0.16,1.98) node[anchor=north west] {$1$};
\begin{scriptsize}
\draw [color=black] (-4.,2.)-- ++(-3.5pt,0 pt) -- ++(7.0pt,0 pt) ++(-3.5pt,-3.5pt) -- ++(0 pt,7.0pt);
\draw[color=black] (-6.18,1.85) node {$f$};
\draw [color=black] (-1.,2.)-- ++(-2.5pt,0 pt) -- ++(5.0pt,0 pt) ++(-2.5pt,-2.5pt) -- ++(0 pt,5.0pt);
\draw [color=black] (2.,2.)-- ++(-2.5pt,0 pt) -- ++(5.0pt,0 pt) ++(-2.5pt,-2.5pt) -- ++(0 pt,5.0pt);
\draw [color=black] (5.,2.)-- ++(-2.5pt,0 pt) -- ++(5.0pt,0 pt) ++(-2.5pt,-2.5pt) -- ++(0 pt,5.0pt);
\draw [color=black] (8.,2.)-- ++(-3.5pt,0 pt) -- ++(7.0pt,0 pt) ++(-3.5pt,-3.5pt) -- ++(0 pt,7.0pt);
\draw [color=black] (11.,2.)-- ++(-2.5pt,0 pt) -- ++(5.0pt,0 pt) ++(-2.5pt,-2.5pt) -- ++(0 pt,5.0pt);
\draw [color=black] (-3.,2.)-- ++(-2.5pt,0 pt) -- ++(5.0pt,0 pt) ++(-2.5pt,-2.5pt) -- ++(0 pt,5.0pt);
\draw [color=black] (-2.,2.)-- ++(-2.5pt,0 pt) -- ++(5.0pt,0 pt) ++(-2.5pt,-2.5pt) -- ++(0 pt,5.0pt);
\draw [color=uuuuuu] (0.,2.)-- ++(-3.5pt,0 pt) -- ++(7.0pt,0 pt) ++(-3.5pt,-3.5pt) -- ++(0 pt,7.0pt);
\draw [color=black] (1.,2.)-- ++(-2.5pt,0 pt) -- ++(5.0pt,0 pt) ++(-2.5pt,-2.5pt) -- ++(0 pt,5.0pt);
\draw [color=black] (3.,2.)-- ++(-2.5pt,0 pt) -- ++(5.0pt,0 pt) ++(-2.5pt,-2.5pt) -- ++(0 pt,5.0pt);
\draw [color=black] (4.,2.)-- ++(-3.5pt,0 pt) -- ++(7.0pt,0 pt) ++(-3.5pt,-3.5pt) -- ++(0 pt,7.0pt);
\draw [color=black] (6.,2.)-- ++(-2.5pt,0 pt) -- ++(5.0pt,0 pt) ++(-2.5pt,-2.5pt) -- ++(0 pt,5.0pt);
\draw [color=black] (7.,2.)-- ++(-2.5pt,0 pt) -- ++(5.0pt,0 pt) ++(-2.5pt,-2.5pt) -- ++(0 pt,5.0pt);
\draw [color=black] (9.,2.)-- ++(-2.5pt,0 pt) -- ++(5.0pt,0 pt) ++(-2.5pt,-2.5pt) -- ++(0 pt,5.0pt);
\draw [color=black] (10.,2.)-- ++(-2.5pt,0 pt) -- ++(5.0pt,0 pt) ++(-2.5pt,-2.5pt) -- ++(0 pt,5.0pt);
\end{scriptsize}
\end{tikzpicture}


\subsection{Savoir encadrer un nombre décimal avec une précision donnée.}

\ExoCu{Représenter.}

Complète pour obtenir un encadrement à l'unité près 

\begin{enumerate}
 \item $\cdots\cdots < 4,2 < \cdots\cdots$ 
 \item $\cdots\cdots < 5,84 <\cdots\cdots$ 
 \item $\cdots\cdots < 19,99 <\cdots\cdots$ 
\end{enumerate}

\ExoCd{Représenter.}

Complète pour obtenir un encadrement  

\begin{enumerate}
 \item au dixième près $\cdots\cdots < 40,23 <\cdots\cdots$ 
 \item à l'unité près $\cdots\cdots < 909,98 <\cdots\cdots$ 
 \item au centième près $\cdots\cdots < 19,75 <\cdots\cdots$ 
\end{enumerate}


\ExoCd{Représenter. Communiquer.}

Usain Bolt a couru le 100 mètres en 9 secondes 58 établi le 16 aout 2009. Donne un encadrement de ce temps au dixième de seconde près.

$\cdots\cdots < \cdots\cdots\cdots\cdots < \cdots\cdots$ 


\subsection{Savoir comparer des fractions.}

\ExoCu{Représenter.}

Combien de nombres différents sont-ils écrits dans la liste ci-dessous : 

$\dfrac{1284}{10000}$ ; $\dfrac{1}{4}$ ; 0,25 ; 1,4 ; $\dfrac{25}{100}$



\ExoCd{Représenter.}

Range dans l'ordre croissant les six nombres suivants écrits de différentes façons :

$\dfrac{6}{10} + \dfrac{1}{100}+ \dfrac{1}{10000} $ ; six cent onze millièmes ; 6,1111 ; $6 +\dfrac{101}{1000}$
; 6 111 dix-millièmes ; $\dfrac{6 101}{10000}$.


\subsection{Composer, décomposer des fractions.}





\section{Calculer avec des nombres entiers et des nombres décimaux}



\subsection{Savoir multiplier un nombre décimal par 0,5 ou 0,1.}

\ExoCu{Calculer. Représenter}

Calcule les produits ci-dessous
\begin{enumerate}
 \item $6 \times 0,5 = \cdots\cdots\cdots\cdots$  
 \item $154 \times 0,5 = \cdots\cdots\cdots\cdots$ 
 \item $75 \times 0,1 = \cdots\cdots\cdots\cdots$
 \item $719 \times 0,1 = \cdots\cdots\cdots\cdots$
\end{enumerate}



\ExoCu{Calculer. Représenter}

Calcule les produits ci-dessous
\begin{enumerate}
 \item $4,2 \times 0,5 = \cdots\cdots\cdots\cdots$ 
 \item $18,6 \times 0,1 = \cdots\cdots\cdots\cdots$ 
 \item $12,4 \times 0,5 = \cdots\cdots\cdots\cdots$  
\end{enumerate}



\ExoCd{Calculer. Représenter}

Calcule les produits ci-dessous
\begin{enumerate}
 \item $6,2 \times 0,5 = \cdots\cdots\cdots\cdots$ 
 \item $7,1 \times 0,1 = \cdots\cdots\cdots\cdots$ 
 \item $14,6 \times 0,5 = \cdots\cdots\cdots\cdots$  
\end{enumerate}



\ExoCd{Calculer. Représenter}

Calcule les produits ci-dessous
\begin{enumerate}
 \item $6,2 \times 0,5 = \cdots\cdots\cdots\cdots$ 
 \item $47,1 \times 0,1 = \cdots\cdots\cdots\cdots$ 
 \item $25,8 \times 0,5 = \cdots\cdots\cdots\cdots$  
\end{enumerate}



\subsection{Savoir utiliser la distributivité simple dans les deux sens}


\ExoCu{Calculer}

Complète les nombres manquants

\begin{description}
\item[] $A=2\times(3+7)$
\item[] $A=2\times\cdots\cdots + 2\times\cdots\cdots $
\item[] $A=\cdots\cdots + \cdots\cdots$
\item[] $A=\cdots\cdots\cdots$
\end{description}




\ExoCu{Calculer}

Calcule le périmètre du rectangle

\begin{tikzpicture}[line cap=round,line join=round,>=triangle 45,x=0.8cm,y=0.8cm]
\clip(0.7,0.82) rectangle (8.42,4.72); 
\draw [line width=1.pt] (1.,4.)-- (1.,1.);
\draw [line width=1.pt] (1.,1.)-- (7.,1.);
\draw [line width=1.pt] (7.,1.)-- (7.,4.);
\draw [line width=1.pt] (7.,4.)-- (1.,4.);
\draw (3.52,4.74) node[anchor=north west] {$4 ~\text{cm}$};
\draw (7.02,2.88) node[anchor=north west] {$1,5 ~\text{cm}$};
\end{tikzpicture}


\ExoCd{Calculer. Représenter.}

$ABCD$ est un rectangle tel que $AB=5,5$ cm et $BC =4,5$ cm.  Calcule le périmètre du rectangle ABCD.




\ExoCu{Calculer.}

Calcule
\begin{enumerate}
\item $8 \times 27 + 2 \times 27 = (\cdots \cdots+ \cdots\cdots)\times \cdots\cdots = \cdots \cdots$
\item $13 \times 7 + 13 \times 3 = \cdots \cdots \times (\cdots \cdots + \cdots\cdots) = \cdots \cdots$ 
\end{enumerate}


\ExoCu{Calculer.}

Calcule
\begin{enumerate}
\item $32 \times 11 = \cdots \cdots \times 10 + \cdots\cdots \times 1 = \cdots \cdots \times (\cdots \cdots + \cdots\cdots) = \cdots \cdots$
\item $13 \times 7 + 13 \times 3 = \cdots \cdots\cdots \cdots\cdots \cdots\cdots \cdots = \cdots \cdots$ 
\end{enumerate}


\ExoCd{Calculer.}

Calcule
\begin{enumerate}
\item $8 \times 23 + 2 \times 23 =  \cdots \cdots  \cdots\cdots \cdots\cdots \cdots\cdots = \cdots \cdots$
\item $13 \times 12 - 13 \times 2 = \cdots \cdots\cdots \cdots\cdots \cdots\cdots  = \cdots \cdots$ 
\end{enumerate}


\ExoCd{Calculer.}

Calcule
\begin{enumerate}
\item $32 \times 11 = \cdots \cdots   \cdots\cdots  \cdots\cdots = \cdots\cdots \cdots\cdots \cdots\cdots= \cdots \cdots$
\item $13 \times 99 = \cdots \cdots\cdots \cdots\cdots \cdots\cdots  = \cdots\cdots \cdots\cdots \cdots\cdots= \cdots \cdots$
\end{enumerate}






\subsection{Apprendre à organiser un calcul en une seule ligne, utilisant si nécessaire des parenthèses}

\ExoCu{Calculer.}

Calcule astucieusement en regroupant les facteurs.

\begin{enumerate}
\item $A = 2 \times 39 \times 5 = \cdots \cdots\cdots \cdots\cdots \cdots\cdots  = \cdots \cdots$
\item $B = 25 \times 14 \times 4 = \cdots \cdots\cdots \cdots\cdots \cdots\cdots  = \cdots \cdots$
\end{enumerate}


\ExoCd{Calculer.}

Calcule astucieusement en regroupant les facteurs.

\begin{enumerate}
\item $A = 25 \times 2,45 \times 4 = \cdots \cdots\cdots \cdots\cdots \cdots\cdots  = \cdots \cdots$
\item $B = 50 \times 14,36 \times 2 = \cdots \cdots\cdots \cdots\cdots \cdots\cdots  = \cdots \cdots$
\end{enumerate}








\subsection{Savoir utiliser une calculatrice pour introduire la priorité de la multiplication sur l'addition et la soustraction}


\ExoCu{Calculer. Modéliser}

Paolo achète dans un magasin un DVD à 7 euros et trois CD à 4 euros l'unité. 
\begin{enumerate}
\item Écris le calcul que tu dois effectuer 

\ligne{1}
\item Combien va-t-il payer ?

\ligne{1}
\end{enumerate}



\ExoCd{Calculer. Modéliser}

Paolo achète dans un magasin un DVD à 7,50 euros et trois CD à 4,90 euros l'unité. 
\begin{enumerate}
\item Écris le calcul que tu dois effectuer 

\ligne{1}
\item Combien va-t-il payer ?

\ligne{1}
\end{enumerate}


\ExoCu{Calculer.}

Arthur calcule mentalement $3 + 4 \times 8$ et trouve 35. Alice utilise une calculatrice et trouve 56. 
Sais-tu expliquer d'où vient cette différence ?




\ExoCu{Communiquer.}

Parmi les calculs proposés, lequel exprime le produit de 12 par la somme de 10 et de 4 ?
 
\fbox{A. $\dfrac{12}{4+10}$} ; \fbox{B. $12\times(10+4)$} ; \fbox{C. $12\times 10+4 $} ; \fbox{D. $12\times(10-4)$}  






\ExoCd{Communiquer.}

Parmi les calculs proposés, lequel exprime le quotient de 23 par la somme de 4 et de 10 ?
 
\fbox{A. $23\times(10+4)$} ; \fbox{B. $23\times 10+4 $} ; \fbox{C. $\dfrac{23}{4+10}$} ; \fbox{D. $23\times(10-4)$}  


\ExoCd{Calculer. Communiquer.}

Simon a effectué un calcul et a trouvé 70. Coche l'expression du calcul effectué.

\begin{description}
\item[A] \fbox{Le quotient de 300 par 5}
\item[B] \fbox{Le produit de 14 et de 5}
\item[C] \fbox{La différence de 389 et 312} 
\item[D] \fbox{La somme de 77 et 7}  

\end{description}



\subsection{Savoir multiplier deux nombres décimaux}



\ExoCu{Calculer.}

Calcule les opérations posées


\begin{minipage}{0.48\linewidth}
\newcommand\hole[1]{\texttt{\_}}
\opmul[decimalsepsymbol={,},voperator=bottom,intermediarystyle=\hole,
resultstyle=\hole,
resultstyle.d=\white]{5.4}{2.3}
\end{minipage}
\begin{minipage}{0.48\linewidth}

\newcommand\hole[1]{\texttt{\_}}
\opmul[decimalsepsymbol={,},voperator=bottom,intermediarystyle=\hole,
resultstyle=\hole,
resultstyle.d=\white]{35.7}{2.46}
\end{minipage}



\ExoCu{Calculer.}

Calcule mentalement les opérations suivantes

\begin{description}
\item $A = 6 \times 5$
\item $B = 9 \times 7$
\item $C = 8 \times 3$
\end{description}

\ExoCu{Calculer.}

Calcule les opérations suivantes

\begin{description}
\item $\cdots \cdots = 1265,4 \times 10 $
\item $\cdots \cdots = 6,44 \times 100 $
\item $\cdots \cdots =  52,38 \times 1000 $
\end{description}



\ExoCu{Calculer.}

Calcule les opérations suivantes

\begin{description}
\item $ 1265,4 \times 0,1  =\cdots \cdots \cdots \cdots\cdots \cdots=\cdots \cdots$
\item $ 6,44 \times 0,01   =\cdots \cdots \cdots \cdots\cdots \cdots=\cdots \cdots$
\item $  72,18 \times 0,01 =\cdots \cdots\cdots \cdots\cdots \cdots =\cdots \cdots  $
\end{description}



\ExoCu{Calculer.}

Calcule les opérations suivantes

\begin{description}
\item $2654 = 26,54 \times \cdots \cdots $
\item $64 = 6,4 \times \cdots \cdots $
\item $152738 = 152,738 \times \cdots \cdots $
\end{description}


\ExoCd{Calculer.}

Calcule les opérations suivantes

\begin{description}
\item $2365 = 236,5 \times \cdots \cdots $
\item $136,4 = 1364 \times \cdots \cdots $
\item $1281 = 12,81 \times \cdots \cdots $
\item $152738 = 152,738 \times \cdots \cdots $
\end{description}



\subsection{Effectuer une division euclidienne}

\ExoCu{Calculer.}

Effectue les divisions suivantes :

\begin{minipage}{0.48\linewidth}
\begin{equation*}
\renewcommand{\arraystretch}{1.2}
\renewcommand{\arraycolsep}{1pt}
  \begin{array}{ rrr|rrr}
  5 & 9 & 8 &   & 1 & 3 \\
\cline{4-6}
   &   &   &   &   &   \\
   &   &   &   &   &   \\
   &   &   &   &   &   \\
   &   &   &   &   &   \\
  \end{array}
\end{equation*}
\end{minipage}
\begin{minipage}{0.48\linewidth}
\begin{equation*}
\renewcommand{\arraystretch}{1.2}
\renewcommand{\arraycolsep}{1pt}
  \begin{array}{rrrr|rrr}
3 & 4 & 3 & 2 &   & 2 & 6 \\
\cline{5-7}
  &   &   &   &   &   &   \\
  &   &   &   &   &   &   \\
  &   &   &   &   &   &   \\
  &   &   &   &   &   &   \\
  \end{array}
\end{equation*}
\end{minipage}


\ExoCu{Modéliser. Calculer. }

Sarah, Lisa et Chloë se partagent équitablement des 48 bonbons du paquet. Combien de bonbons chacune va-t-elle obtenir ?




\ExoCu{Représenter. Modéliser. Calculer. }

Un maçon construit un mur avec des parpaings de 0,50 m de long sur 0,2 m de hauteur. Combien doit il acheter de parpaings pour construire un mur de 6 m de longueur et 1 m de haut ?


\ExoCd{Représenter. Modéliser. Calculer. }

Un maçon construit un mur avec des parpaings de 50 cm de long sur 20 cm de hauteur. Combien doit il acheter de parpaings pour construire un mur de 6 m de longueur et 2,6 m de haut ?



\ExoCd{Représenter. Modéliser. Calculer. }

Un maçon construit un mur avec des parpaings de 50 cm de long sur 20 cm de hauteur. Combien doit il acheter de parpaings pour construire un mur de 6 m de longueur et 2,6 m de haut ?



\subsection{Savoir ajouter ou soustraire des nombres décimaux}


\ExoCu{Calculer. Raisonner.}

En utilisant ses connaissances sur le produit de deux décimaux et un ordre de grandeur, il sait 
trouver la réponse exacte du calcul $9,52 \times 51,3$ parmi les réponses proposées :

\fbox{$488,76$} ; \fbox{$48,376$} ; \fbox{$488,375$} ; \fbox{$488,376$} ; \fbox{$488 376$}.



\subsection{Effectuer une division décimale}



 


\ExoCu{Calculer.}

Effectue les divisions suivantes :
 






\section{Résoudre des problèmes en utilisant des fractions simples, les nombres décimaux et le calcul}

\subsection{Résoudre des problèmes en utilisant des fractions simples, les nombres décimaux et le calcul.}

\ExoCu{Modéliser. Calculer.}

Un vase pouvant contenir 3L contient déjà 2L d'eau. Si on ajoute à nouveau 50cL d'eau, l'eau débordera-t-elle ?



\ExoCu{Modéliser. Calculer.}

Eva et sa famille sont parties à 20h45 de leur maison. Elles sont arrivées à 22h55 sur leur lieu de vacances. Quelle est la durée de leur voyage ?


\ExoCu{Modéliser. Calculer.}

Le CDI du collège achète 57 revues à 1,90 euro l'unité. Quel est le prix total de la facture ?

\textit{Le calcul malin propose $57 \times 19 = 1083$}  


\ExoCd{Modéliser. Calculer.}

Le CDI du collège achète 95 revues à 8,60 euro l'unité. Quel est le prix total de la facture ?


\ExoCd{Modéliser. Calculer.}

Le prix d'un kilogramme de fraise coute 6,10 euros. Elisa achète 1,5kg de fraises. Quel est le prix que va payer Elisa au maraicher ?


\ExoCd{Modéliser. Calculer.}

La maman d'Assia lui donne 10 euros pour acheter 1,8kg de myrtilles. Le prix d'un kilogramme de myrtilles est de 5,2 euros. La somme dont dispose Assia est-elle suffisante  ?



\ExoCd{Chercher. Calculer.}

Un professeur de tennis achète sur Internet 15 raquettes de tennis à 6,80 euros et 24 cerceaux. Il paie au total 176,40 euros. Quel est le prix d'un cerceau  ?


\ExoCd{Chercher. Calculer.}

Pendant 6 jours, le pirate Long John Silver a déposé tous les jours six pièces d'or dans son coffre. Il a maintenant 108 pièces d'or.
Combien de pièces avait-il avant les 6 jours  ?

\ExoCd{Chercher. Calculer.}
Je suis un multiple de 7 compris entre 40 et 100 dont la somme des chiffres est un multiple 
de 4. Qui suis-je ?


\subsection{Collecter les informations utiles à la résolution d'un problème à partir de supports variés, les exploite et les organise en produisant des tableaux à double entrée, des diagrammes circulaires, semi-circulaires, en bâtons ou des graphiques}



\ExoCu{Représenter.}

Lors de l'élection des délégués de la classe, 4 élèves se sont présentés. Chaque élève a voté pour un seul candidat. Voici les résultats.

\begin{tabular}{|c|c|c|c|c|}
\hline 
  & Marc & Donia & Yassine & Ezer \\ 
\hline 
Nombre de voix obtenues & 2 & 8 & 10 & 4 \\ 
\hline 
\end{tabular} 


Représente ces données par un diagramme en bâtons.


\ExoCu{Représenter.}

Dans la classe de 6ème 4, il y 14 filles et 12 garçons. 6 garçons et 7 filles ont les yeux bleus.


Complète ce tableau dit à double entrée.

\begin{tabular}{|c|c|c|c|}
\hline 
  & Yeux bleus & autres couleurs & Total \\ 
\hline 
Filles &  &  &   \\ 
\hline 
Garçons &   &   &   \\ 
\hline 
Total &   &   &   \\ 
\hline 
\end{tabular} 



\ExoCd{Représenter.}

Dans la classe de 6ème 4, il y 14 filles et 12 garçons. 6 garçons et 7 filles ont les yeux bleus.

\begin{enumerate}
\item  Complète ce tableau dit à double entrée.

\begin{tabular}{|c|c|c|c|}
\hline 
  & Yeux bleus & autres couleurs & Total \\ 
\hline 
Filles &  &  &   \\ 
\hline 
Garçons &   &   &   \\ 
\hline 
Total &   &   &   \\ 
\hline 
\end{tabular} 
\item Combien d'élèves n'ont pas les yeux bleus ?
\end{enumerate}







\ExoCd{Modéliser. Calculer.}

Lors de l'élection des délégués de la classe, 4 élèves se sont présentés. Chaque élève a voté pour un seul candidat. Voici les résultats.

\begin{tabular}{|c|c|c|c|c|}
\hline 
  & Marc & Donia & Yassine & Ezer \\ 
\hline 
Nombre de voix obtenues & 2 & 8 & 10 & 4 \\ 
\hline 
\end{tabular} 


Représente ces données par un diagramme circulaire.



\ExoCd{Modéliser. Calculer.}

Dans un collège, les enfants ont le choix d'étudier 3 langues pour la langue vivante 2 : italien,  allemand ou espagnol.
\begin{description}
\item En 5e A, il y a 25 élèves. 12 ont choisi espagnol, 6 allemand et les autres italien.
\item En 5e B, 13 élèves ont choisi espagnol et 5 élèves allemand.
Dans ces deux classes, 12 élèves ont choisi italien.
\end{description} 
Présenter ces données dans un tableau à double entrée.




\subsection{Remobiliser les procédures déjà étudiées pour résoudre des problèmes relevant de la proportionnalité et les enrichir par l'utilisation du coefficient de proportionnalité}

\ExoCu{Calculer.}

Voici les tarifs des pains dans une boulangerie :

\begin{tabular}{|c|c|c|c|}
\hline 
Nombre de pains achetés & 1 & 4 & 10 \\ 
\hline 
Prix en euros & 1,8 & 7 & 16,20 \\ 
\hline 
\end{tabular} 

Le prix à payer est-il proportionnel au nombre de pains achetés ?



\ExoCu{Calculer.}

Voici la recette de la pâte à crêpes. Ingrédients pour 4 personnes :
\begin{description}
\item 200 g de farine ;
\item 4 œufs ;
\item trois quarts de litre de lait ;
\item 40 g de beurre ;
\item 2 cuillerées à soupe de sucre.
\end{description}

\begin{enumerate}
\item Quelle quantité de farine est nécessaire pour 12 personnes ?
\item Pour 6 personnes, combien faut-il de cuillerées de sucre ?
\item Quelle quantité de beurre faut-il prévoir pour 7 personnes ?
\item Quelle quantité de lait faut-il prévoir pour 12 personnes ? 
\end{enumerate}


\ExoCu{Chercher. Raisonner.}

La taille et l'âge d'une personne sont-ils proportionnels ?

Donne trois exemples de relations entre deux objets qui ne sont pas proportionnelles ?


\ExoCu{Chercher. Raisonner.}

8 oranges coûtent 4 euros, 3 citrons coûtent 2 euros et 7 poires coûtent 4 euros.

\begin{description}
\item Quel est le fruit le plus cher ? 
\item Quel est le fruit le moins cher ?
\end{description}

\ExoCd{Chercher. Calculer.}

30 maçons construisent 30 maisons en 30 jours. Combien de maison construisent 15 maçons en 15 jours ?




\subsection{Remobiliser les procédures déjà étudiées pour résoudre des problèmes relevant de la proportionnalité}


\ExoCu{Chercher. Raisonner.}

Un rectangle a pour longueur 15 cm et pour largeur 12 cm. Comme il est trop grand, on a décidé de tracer sa longueur de 7 cm. Quelle sera alors sa largeur ?



\ExoCu{Chercher. Raisonner.}

Kristina Vogel, double championne olympique roule à une vitesse moyenne de 40 km/h. Quelle distance parcourt-elle en 30 minutes ?





\ExoCd{Chercher. Raisonner.}

Kristina Vogel, double championne olympique roule à une vitesse moyenne de 42km/h. Quelle distance parcourt-elle en 45 minutes ?




\subsection{Savoir appliquer un pourcentage}

\ExoCu{Chercher. Calculer.}

Donne un ordre de grandeur de 48 \% de 60,45.



\ExoCu{Calculer.}

Lors des soldes, un magasin propose un rabais de 20\% sur tous ces produits.

Quel est le prix d'un pantalon affiché à 225 euros ?



\ExoCu{Chercher. Calculer.}

Calculer mentalement 

\begin{description}
\item 50 \% de 120 élèves soit $\cdots\cdots\cdots\cdots$ élèves 
\item 25 \% de 40 euros  soit $\cdots\cdots\cdots\cdots$ euros 
\item 10 \% de 50 voitures soit $\cdots\cdots\cdots\cdots$ voitures 
\end{description}


 
\ExoCu{Chercher. Calculer.}

Un collège comporte 775 élèves. 24 \% des élèves sont externes.
Calcule le nombre d'élèves externes.




\end{document}
