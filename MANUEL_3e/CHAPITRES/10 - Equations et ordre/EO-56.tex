
\begin{enumerate}

\item  Détermine par la méthode de ton choix la solution de chaque problème.

\begin{tabular}{|>{\centering\arraybackslash}p{0.5cm}|>{\centering\arraybackslash}p{10.5cm}|>{\centering\arraybackslash}p{6cm}|>{\centering\arraybackslash}p{2cm}|>{\centering\arraybackslash}p{3cm}|}
\hline 
&Enoncé & Calcul & Réponses &  \\ 
\hline 
1&Jean a trois ans de plus que Pierre. Jean a 21 ans. Quel est l’âge de Pierre ? &  &  &  \\ 
\hline 
2&Marion a cinq ans de plus que Léa. Léa a 19 ans.
Quel âge a Marion ? &  &  &  \\ 
\hline 
3&Les longueurs sont exprimées en centimètres. $[AB]$ mesure 17 cm.
Trouve la mesure de [AM] 

\definecolor{uuuuuu}{rgb}{0.26666666666666666,0.26666666666666666,0.26666666666666666}
\begin{tikzpicture}[line cap=round,line join=round,>=triangle 45,x=1.0cm,y=1.0cm]
\clip(0.44,0.26) rectangle (9.62,1.9);
\draw (1.,1.)-- (9.,1.);
\draw (4.965,1.09) -- (4.965,0.91);
\draw (5.035,1.09) -- (5.035,0.91);
\draw (1.,1.)-- (2.09,1.);
\draw (1.51,1.09) -- (1.51,0.91);
\draw (1.58,1.09) -- (1.58,0.91);
\draw (2.09,1.)-- (3.18,1.);
\draw (2.6,1.09) -- (2.6,0.91);
\draw (2.67,1.09) -- (2.67,0.91);
\draw (3.18,1.)-- (4.27,1.);
\draw (3.69,1.09) -- (3.69,0.91);
\draw (3.76,1.09) -- (3.76,0.91);
\draw (5.46,0.74)-- (8.96,0.74);
\draw (6.92,0.9) node[anchor=north west] {$7 $};
\begin{scriptsize}
\draw [color=black] (1.,1.)-- ++(-2.5pt,0 pt) -- ++(5.0pt,0 pt) ++(-2.5pt,-2.5pt) -- ++(0 pt,5.0pt);
\draw[color=black] (0.92,1.37) node {$A$};
\draw [color=black] (9.,1.)-- ++(-2.5pt,0 pt) -- ++(5.0pt,0 pt) ++(-2.5pt,-2.5pt) -- ++(0 pt,5.0pt);
\draw[color=black] (9.14,1.37) node {$B$};
\draw [color=black] (5.36,1.)-- ++(-2.5pt,0 pt) -- ++(5.0pt,0 pt) ++(-2.5pt,-2.5pt) -- ++(0 pt,5.0pt);
\draw[color=black] (5.36,1.37) node {$C$};
\draw [color=uuuuuu] (3.18,1.)-- ++(-2.5pt,0 pt) -- ++(5.0pt,0 pt) ++(-2.5pt,-2.5pt) -- ++(0 pt,5.0pt);
\draw [color=uuuuuu] (4.27,1.)-- ++(-2.5pt,0 pt) -- ++(5.0pt,0 pt) ++(-2.5pt,-2.5pt) -- ++(0 pt,5.0pt);
\draw [color=uuuuuu] (2.09,1.)-- ++(-2.5pt,0 pt) -- ++(5.0pt,0 pt) ++(-2.5pt,-2.5pt) -- ++(0 pt,5.0pt);
\draw[color=uuuuuu] (2.24,1.37) node {$M$};
\draw [fill=black,shift={(5.46,0.74)},rotate=90] (0,0) ++(0 pt,3.0pt) -- ++(2.598076211353316pt,-4.5pt)--++(-5.196152422706632pt,0 pt) -- ++(2.598076211353316pt,4.5pt);
\draw [fill=black,shift={(8.96,0.74)},rotate=270] (0,0) ++(0 pt,3.0pt) -- ++(2.598076211353316pt,-4.5pt)--++(-5.196152422706632pt,0 pt) -- ++(2.598076211353316pt,4.5pt);
\end{scriptsize}
\end{tikzpicture}

&  &  &   \\ 
\hline 
4&On considère le programme de calcul suivant :
\begin{verbatim}
Choisir un nombre.
Multiplier ce nombre par 3.
Soustraire 2 au résultat.
\end{verbatim}
Quel nombre de départ faut-il choisir pour que le résultat du
programme soit le nombre 22 ? &  &  &   \\ 
\hline 
5&Un téléphone avec sa coque de protection coûtent 60€.
Le téléphone coûte 50€ de plus que la coque.
Quel est le prix de la coque ? &  &  &   \\ 
\hline 
\end{tabular} 



\item Complète les 2 premières colonnes du tableau ci-dessous

\begin{tabular}{|>{\centering\arraybackslash}p{5.5cm}|>{\centering\arraybackslash}p{6cm}|>{\centering\arraybackslash}p{8cm}|>{\centering\arraybackslash}p{2cm}|}
\hline 
 & Modélise avec $x$ & Trouve la valeur de $x$ pour que cette égalité soit vraie. &  \\ 
\hline 
$x$ plus à 3 est égal à 21 &  &  &  \\ 
\hline 
Le double de $x$ plus 50 est égal à 60 &  &  &  \\ 
\hline 
$x$ moins 5 est égal à 19 &  &  &   \\ 
\hline 
Le triple de $x$ moins 2 est égal à 22 &  &  &   \\ 
\hline 
Le quadruple de $x$ plus 7 est égal à 17&  &  &   \\ 
\hline 
\end{tabular} 

\item Chaque égalité correspond à une modélisation des problèmes de la partie A. Que représente chaque valeur de $x$ pour chaque problème ? Complète la dernière colonne du tableau en associant le problème de la partie A.
\end{enumerate}

