\documentclass[10pt]{article}

\input{../preambule}
\input{../styles}
\input{../bas_de_page_quatrieme} 

%%%%%%%%%%%   Marges de pages  %%%%%%%%%%%%%%%% 
 \usepackage{geometry}
 \geometry{top=2cm, bottom=1cm, left=2cm , right=2cm}
%%%%%%%%%%%%%%%%%%%%%%%%%%%%%%%%%%%%%%%%%%%%%%%

%%%%%%%%%%%%%%%  Indentation  %%%%%%%%%%%%%%%%%%
\parindent=0pt
%%%%%%%%%%%%%%%%%%%%%%%%%%%%%%%%%%%%%%%%%%%%%%%%
 \usepackage{eurosym}

\begin{document}

%%%%%%%%%%%%%%%%%%%%%%%%%%%%%%%%%%%%%%%%%%%%%%%
%%%%		 Titre encadré
%%%%%%%%%%%%%%%%%%%%%%%%%%%%%%%%%%%%%%%%%%%%%%%
\begin{encadrementombre}{Thème 8 Calcul littéral}
{\LARGE Equations et ordre }\\

%% Laisse la ligne vide ci-dessus
{\Large Algèbre}
\end{encadrementombre}


%%%%%%%%%%%%%%%%%%%%%%%%%%%%%%%%%%%%%%%%%%%%%%%
%%%%		 Corps du document
%%%%%%%%%%%%%%%%%%%%%%%%%%%%%%%%%%%%%%%%%%%%%%%

%%%%%%%%%%%%%%%%%%%%%%%%%%%%%%%%%%%%%%%%%%%%%%%
%%%% \renewcommand{\arraystretch}{1.8}
\definecolor{shadecolor}{gray}{0.9}
%%%%%%%%%%%%%%%%%%%%%%%%%%%%%%%%%%%%%%%%%%%%%%%
%%%%%%%%%%%   Hauteur de ligne  %%%%%%%%%%%%%%%%
{\setlength{\baselineskip}{1.5\baselineskip}
%%%%%%%%%%%%%%%%%%%%%%%%%%%%%%%%%%%%%%%%%%%%%%%%

\section{Introduction}
\subsection{Trois programmes de calcul(voir Sesamath)}
\begin{shaded}
Cécile et Mathieu saisissent le même nombre de départ sur leurs calculatrices puis effectuent les programmes de calcul suivants:
\begin{itemize}
\item Cécile multiplie le nombre de départ par $8$ puis ajoute au résultat obtenu
\item Mathieu multiplie le nombre de départ par $6$ puis ajoute $13$ au résultat obtenu 
\end{itemize}
Ils s'aperçoivent alors que leurs calculatrices affichent le même résultat.\\
\end{shaded}

\begin{enumerate}
\item \textbf{En calculant}
\begin{enumerate}
\item Le nombre $1$ est-il le nombre de départ? Et $2$?
\item En poursuivant les essais, détermines le nombre qu'ont saisi Cécile et Mathieu sur leur calculatrice.
\item Vincent effectue, avec le même nombre de départ, le programme de calcul suivant: il multiplie le nombre par $3$ puis ajoute $30$. Vincent trouve-il le même résultat que Cécile et Mathieu?
\end{enumerate}
\item \textbf{Avec un tableur}
\begin{enumerate}
\item A l'aide d'un tableur, construis le tableau ci-dessous. Programme la cellule B2 en fonction de la cellule B1 pour pour obtenir le résultat de la suite de calculs de Cécile. Copie alors cette formule dans les cellules C2 à L2.\\
Procède de la même façon pour les programmes de calculs de Mathieu et Vincent.

\newcolumntype{R}[1]{>{\raggedleft\arraybackslash}b{#1}}
	\newcolumntype{L}[1]{>{\raggedright\arraybackslash}b{#1}}
	\newcolumntype{C}[1]{>{\centering\arraybackslash}b{#1}}
\begin{tabular}{|L{0.3cm}||C{3cm}|C{0.3cm}|C{0.3cm}|C{0.3cm}|C{0.3cm}|C{0.3cm}|C{0.3cm}|C{0.3cm}|C{0.3cm}|C{0.3cm}|C{0.3cm}|C{0.3cm}|}
	\hline &A&B&C&D&E&F&G&H&I&J&K&L\\
	\hline 1&Nombre de départ&0&1&2&3&4&5&6&7&8&9&10\\
	\hline 2&Cécile&&&&&&&&&&&\\
	\hline 3&Mathieu&&&&&&&&&&&\\
	\hline 4&Vincent&&&&&&&&&&&\\
	\hline
\end{tabular}

\item retrouve la valeur solution de la question b) de la partie 1
\item A l'aide du tableur, aide Cécile et Vincent à déterminer le nombre qu'il faut choisir au départ pour obtenir le même résultat.
\end{enumerate}
\end{enumerate}

\subsection{A la ferme}
\begin{shaded}
Dans son champ un fermier compte $83$ têtes et $236$ pattes. Sachant qu'il n'a que des vaches et des poules, peux-tu retrouver le nombre de poules et le nombre de vaches dans le champ du fermier?
\end{shaded}

\newpage


%%%%%%%%%%%%%%%%%%%%%%%%%%%%%%%%%%%%%%%%%%%%%%%%%%%%%%%
%     Notions
%%%%%%%%%%%%%%%%%%%%%%%%%%%%%%%%%%%%%%%%%%%%%%%%%%%%%%%
\section{Cours}
\subsection{Equations}
\subsubsection{Effet de l'addition et de la soustraction sur une égalité}
\begin{shaded}
\begin{Pp}
Si on additionne ou si on soustrait un même nombre aux deux membres d'une égalité
alors on obtient une nouvelle égalité.
\\Autrement dit : ($a$, $b$ et $c$ désignent des nombres relatifs)\hspace{1cm} Si     $a = b$ 	alors 	$a + c = b + c$		et 	 $a – c = b – c$
\end{Pp}
\end{shaded}

\begin{Ex}
\begin{multicols}{2}
$x – 3=–2,3$\\
Donc $x – 3 + 3=–2,3 + 3$\\
Donc $x=0,7$\\


$6,1 + t=–1,4$\\
Donc $6,1 + t – 6,1=–1,4 – 6,1$\\
Donc $t=–7,5$
\end{multicols}
\end{Ex}
\subsubsection{Effet de la multiplication et de la division sur une égalité}
\begin{shaded}
\begin{Pp}
Si on multiplie ou si on divise par un même nombre \textbf{non nul} les deux membres d'une égalité
alors on obtient une nouvelle égalité.
\\Autrement dit : ($a$, $b$ et $c$ désignent des nombres relatifs, $c \neq 0$.)Si     $a = b$ 	alors 	$a \times c = b \times c$		et 	$a \div c = b \div c$ ou encore $\frac{a}{c}=\frac{b}{c}$
\end{Pp}
\end{shaded}

\begin{Ex}
\begin{multicols}{2}
$–4n=3$\\
Donc $–4n \div \left(–4\right)=3 \div \left(–4\right)$\\
Donc $n=–0,75$\\

$\frac{n}{5}=–2,1$\\
Donc $\frac{n}{5} \times 5=–2,1 \times 5$\\
Donc $n=–10,5$

\end{multicols}
\end{Ex}

\subsection{Résolution d'équations}

\begin{Df}
Une \textbf{équation} est une égalité dans laquelle intervient un (ou plusieurs)
nombre(s) inconnu(s). Ce nombre est désigné le plus souvent par une lettre.
\end{Df}


\begin{Ex}
L'égalité $5x + 2y – 3 = x – 3y + 7$ est une \textbf{équation}.\\
Les \textbf{inconnues} sont désignées par $x$ et $y$ .\\$5x + 2y – 3$ est le \textbf{membre de gauche} de l'équation.
$x – 3y + 7$ est le \textbf{membre de droite} de l'équation.
\end{Ex}
 		 
\begin{Df} \textbf{Résoudre une équation}, c'est trouver TOUTES les valeurs possibles de l'inconnue 
telles que le membre de gauche soit égal au membre de droite.
\end{Df}


\begin{Rq}
Chacune de ces valeurs est appelée \textbf{\og  solution de l'équation \fg{}}.
\end{Rq}

\begin{Df} \textbf{\og Tester une égalité \fg{}},
c'est :
\begin{enumerate}
\item 	calculer son membre de gauche ;
\item   calculer son membre de droite ;
\item   comparer les résultats.
\end{enumerate}
Ce test permet de savoir si un nombre donné EST ou N'EST PAS une solution d'une équation.
\end{Df}


\begin{Ex}
$–1$ est-il une solution de l'équation $2x^2 – 6 = x + 10$ ?
\begin{enumerate}
\item Pour $x = –1$, on calcule le membre de gauche : 	$2x^2 – 6 = 2 \times \left(–1\right)^2 – 6 = –4$
\item Pour $x = –1$, on calcule le membre de droite : 	$x + 10 = –1 + 10 = 9$
\item On compare :  $–4 \neq 9$	 donc $–1$ N'EST PAS une solution de l'équation.
\end{enumerate}
\end{Ex}		
		
\begin{Ex} \textbf{(RESOLUTION D UNE EQUATION)}
Résolvons l'équation $5x – 3 = x + 7$.\\
Donc $5x – 3 – x=x + 7 – x$ (on soustrait $x$ de part et d'autre)\\
Soit $4x – 3=7$ (on réduit les deux membres)\\
Donc $4x – 3 + 3=7 + 3$ (on ajoute $3$ de part et d'autre),\\
Soit $4x=10$ (on réduit les deux membres)
Donc $\frac{4x}{4}=\frac{10}{4}$ (on divise par $4$ de part et d'autre) et donc 
$x=2,5$ (on réduit les deux membres une dernière fois \ldots{})\\

\textbf{Vérification:}
\begin{enumerate}
\item 	Pour $x = 2,5$ , on calcule le membre de gauche : $5x – 3 = 5 \times 2,5 – 3 = 9,5$
\item Pour $x = 2,5$ , on calcule le membre de droite :    $x + 7 = 2,5 + 7 = 9,5$
\item On compare :  $9,5 = 9,5$
\end{enumerate}
Conclusion :	$2,5$ est LA solution de l'équation $5x – 3 = x + 7$. Elle est unique.
\end{Ex}

\subsection{Mathématisation et résolution d'un problème}
\begin{shaded}
Alfred passe un examen comportant trois épreuves notées chacune sur $20$ : les mathématiques coefficient $4$, le français coefficient $3$ et l'anglais coefficient $2$.\\
Il a obtenu $12$ en mathématiques et $9$ en français.\\
Avec quelle note en anglais la moyenne d'Alfred sera-t-elle égale à $10$ ?
\end{shaded}

On peut résoudre un tel problème en suivant les étapes ci-dessous.
\begin{enumerate}
\item \textbf{Choix de l'inconnue}
L'inconnue est la note d'anglais. 
Désignons cette inconnue par $x$ par exemple.
\begin{Rq}
C'est la question du problème
qui permet de choisir l'inconnue.
\end{Rq}
\item \textbf{Mise en équation du problème}
On doit avoir :  $\frac{12 \times 4+9 \times 3 +x \times 2}{4+3+2}= 10$
\begin{Rq}
Ce sont les informations relevées
dans le texte et dans la question qui permettent de mettre le problème en équation.
\end{Rq} 
\item \textbf{Résolution de cette équation}
$\frac{12 \times 4+9 \times 3 +x \times 2}{4+3+2} \times 9  =10 \times 9$\\
Soit $\frac{72+2x}{9} \times 9  =90$ ou encore $72+2x=90$ (on réduit le membre de gauche puis on \og  neutralise \fg{} le dénominateur)\\
Donc $75 + 2x – 75=90 – 75$ (Puis on \og  neutralise \fg{} la constante du membre de gauche)\\
Donc $2x=90$ et $\frac{2x}{2}=\frac{90}{2}$ (on réduit les deux membres, puis on \og  neutralise \fg{} le coefficient de $x$)\\
Finalement $x=7,5$ (on réduit les deux membres
une dernière fois)
\item \textbf{Conclusion}
Avec 7,5 en anglais, 
la moyenne d'Alfred sera égale à $10$ (on rédige la réponse à la question posée).
\end{enumerate}


\subsection{Ordre}

\begin{Df}
$a \leqslant b$ signifie $a<b$ ou $a=b$\\
De même, $a \geqslant b$ signifie $a>b$ ou $a=b$
\end{Df}
 
\begin{Ex} 	
$7,8 \geqslant 7,09$ et $–5,4 \leqslant 3,2$\\
On peut aussi écrire :	$–6,1 \geqslant –6,1$		et	$–6,1 \leqslant –6,1$
\end{Ex}

\begin{shaded}
\textbf{Lecture}
\begin{itemize}
\item $a<b$ se lit $a$ (est \textbf{strictement}) inférieur à $b$.
\item $a>b$ se lit$a$ (est \textbf{strictement}) supérieur à $b$.
\item $a \leqslant b$ se lit $a$ (est) inférieur \textbf{ou égal} à $b$.
\item $a \geqslant b$ se lit $a$ (est) supérieur
\textbf{ou égal} à $b$.
\end{itemize}
\end{shaded}

\begin{Rq} \textbf{Cas particuliers}
\begin{itemize}
\item $x<0$ se lit $x$ (est strictement)
négatif.
\item $x>0$ se lit $x$ (est strictement)
positif.
\item $x \leqslant 0$ se lit $x$ (est) négatif
ou nul.
\item $x \geqslant 0$ se lit $x$ (est) positif
ou nul.
\end{itemize}
\end{Rq}


\subsection{Signe de la différence}
\begin{shaded}
\begin{Pp}
On peut comparer deux nombres à l'aide du \textbf{signe de leur différence}.
Plus précisément: 
\begin{itemize}
\item Si $a-b<0$, alors $a<b$
\item Si $a-b>0$, alors $a>b$
\item Si $a-b\leqslant 0$, alors $a\leqslant b$
\item Si $a-b\geqslant 0$, alors $a\geqslant b$
\end{itemize}
\end{Pp}
\end{shaded}


\begin{Ex}
Comparons $S = 3\pi + 0,4$ et $T = \pi^2$ à l'aide de la calculatrice.\\
$S – T = 3\pi + 0,4 – \pi^2 \approx –0,04$ et comme $–0,04 < 0$\hspace{0.5cm} DONC \hspace{0.5cm}$S < T$
\end{Ex}
		
\begin{Rq}
On aurait pu étudier le signe de $T – S$.
\end{Rq}
				
\subsection{Inégalités et Opérations}
\subsubsection{Effet de l'addition et de la soustraction}
\begin{shaded}
\begin{Pp}
Si on additionne ou si on soustrait un même nombre aux deux membres d'une inégalité, alors on obtient une inégalité de même sens.
Autrement dit : ($a$, $b$ et $c$ désignent des nombres relatifs.)
\begin{itemize}
\item Si $a<b$, alors $a+c<b+c$ et $a-c<b-c$    	
\item Si $a>b$, alors $a+c>b+c$ et $a-c>b-c$
\item Si $a\leqslant b$, alors $a+c\leqslant b+c$ et $a-c\leqslant b-c$
\item Si $a\geqslant b$, alors $a+c\geqslant b+c$ et $a-c\geqslant b-c$
\end{itemize}
\end{Pp}
\end{shaded}


\begin{Ex}
\begin{multicols}{2}	
$x\leqslant –3,4$\\
Donc,$x + 3,4\leqslant –3,4 + 3,4$, soit $x + 3,4\leqslant 0$\\ 

$t +\frac{7}{9} \geqslant \frac{-5}{6}$\\
Donc $t +\frac{7}{9} –\frac{7}{9}\geqslant \frac{-5}{6}–\frac{7}{9}$, soit $t\geqslant \frac{-15}{18}–\frac{14}{18}= \frac{-29}{18}$
\end{multicols}
\end{Ex}


\subsubsection{Effet de la multiplication et de la division par un nombre strictement POSITIF} 
\begin{shaded}
\begin{Pp}
Si 	on multiplie ou si on divise par un même nombre \textbf{strictement positif} les deux membres d'une inégalité, alors 	on obtient une inégalité de même sens. Autrement dit : ($a$, $b$ et $c$ désignent des nombres relatifs; $c > 0$).
\begin{itemize}
\item Si $a<b$, alors $ac<bc$ et $\frac{a}{c} < \frac{b}{c}$    	
\item Si $a>b$, alors $ac>bc$ et $\frac{a}{c} > \frac{b}{c}$ 
\item Si $a\leqslant b$, alors $ac\leqslant bc$ et $\frac{a}{c}\leqslant \frac{b}{c}$
\item Si $a\geqslant b$, alors $ac\geqslant bc$ et $\frac{a}{c}\geqslant \frac{b}{c}$     	
\end{itemize}
\end{Pp}
\end{shaded}


\begin{Ex}	
\begin{multicols}{2}

$4x\geqslant –3$\\
Donc $\frac{4x}{4}\geqslant \frac{-3}{4}$\\
Donc $x\geqslant \frac{-3}{4}$\\

$\pi<3,15$\\
Donc $3 \pi<3 \times 3,15$\\
Et donc $3 \pi< 9,45$
\end{multicols}
\end{Ex}


\subsubsection{Effet de la multiplication et de la division par un nombre strictement NEGATIF}
\begin{shaded}
\begin{Pp}
Si 	on multiplie ou si on divise par un même nombre \textbf{strictement négatif} les deux membres d'une inégalité, alors on obtient une inégalité de sens contraire. Autrement dit : ($a$, $b$ et $c$ désignent des nombres relatifs ; $c < 0$).
\begin{itemize}
\item Si $a<b$, alors $ac>bc$ et $\frac{a}{c} > \frac{b}{c}$    	
\item Si $a>b$, alors $ac<bc$ et $\frac{a}{c} < \frac{b}{c}$ 
\item Si $a\leqslant b$, alors $ac\geqslant bc$ et $\frac{a}{c}\geqslant \frac{b}{c}$
\item Si $a\geqslant b$, alors $ac\leqslant bc$ et $\frac{a}{c}\leqslant \frac{b}{c}$
\end{itemize}	
\end{Pp}
\end{shaded}

\begin{Ex}
\begin{multicols}{2}
$–4x>–5$\\
Donc $\frac{-4x}{-4}<\frac{-5}{-4}$\\
Soit $x<\frac{5}{4}$\\

$p<7$\\
Donc $-3 \times p>-3 \times 7$\\
Soit $-3p>-21$
\end{multicols}
\end{Ex}


\subsubsection{Troncature, arrondi et encadrement}
\begin{shaded}
\begin{Ex}
$\frac{355}{17}= 355 \div 17 = 20,8823529411 \ldots{}$
\end{Ex}
\end{shaded}
\newcolumntype{R}[1]{>{\raggedleft\arraybackslash}b{#1}}
	\newcolumntype{L}[1]{>{\raggedright\arraybackslash}b{#1}}
	\newcolumntype{C}[1]{>{\centering\arraybackslash}b{#1}}
\begin{tabular}{|L{3cm}||C{3cm}|C{3cm}|C{3cm}|}
	\hline &Troncature&Arrondi&Encadrement\\
	\hline à l'unité&$20$&$21$&$20<\frac{355}{17}<21$\\
	\hline au dixième&$20,8$&$20,9$&$20,8<\frac{355}{17}<20,9$\\
	\hline au centième&$20,88$&$20,88$&$20,88<\frac{355}{17}<20,89$\\
	\hline
\end{tabular}







\newpage

\section{Quelques exercices}
\begin{Exo}
Résoudre les équations suivantes :
\begin{multicols}{3}
\begin{enumerate}
\item $7x=5$
\item $2x+7=3$
\item $3x-2=6x+8$
\end{enumerate}
\end{multicols}
\end{Exo}

\begin{Exo}
Résous les équations suivantes :
$
3x+5=6x-3\kern2cm-4x+5=2x-7\kern2cm2(x-5)=3x+6
$
\end{Exo}

\begin{Exo}
Résous les équations suivantes :
$
4x+3=2(x+3)-1\kern2cm (2x+1)(x-2)=2x^2+3\kern2cm \frac{x+1}3=\frac{-3x+2}2
$
\end{Exo}

\begin{Exo}
En additionnant la moitié et le triple d'un nombre, on trouve 98.
\par Quel est ce nombre ?
\end{Exo}

\begin{Exo}
Dans une assemblée, les $\dfrac34$ des participants sont des
femmes, les $\dfrac23$ des hommes portent des lunettes et l'on
compte 10 hommes qui ne portent pas de lunettes.
\par Combien y-a-t-il de participants au total ?
\end{Exo}

\begin{Exo}
Jean a obtenu 18 ; 09 ; 12 en Devoir Maison (coefficient 1) et 08 en
Devoir Surveillé (coefficient 3). Combien doit-il obtenir au prochain
Devoir Surveillé pour avoir 10 de moyenne ?  De quel pourcentage
a-t-il réduit ou augmenté sa note de Devoir Surveillé ?
\end{Exo}


\begin{Exo}
Un chevalier voulait se rendre au château d'une princesse. Il
devait arriver à 17h00 exactement. Habile en mathématiques, il
calcula s'il voyageait 15 kilomètres par heure, il arriverait une
heure trop tôt.

S' il voyageait 10 kilomètres par heure, il arriverait une heure
trop tard.\\
\begin{enumerate}
    \item Quelle est l' heure de son départ?
    \item Quelle distance voyagea-t-il?
    \item \`A quelle vitesse voyageait-il?
\end{enumerate}
\end{Exo}

\begin{Exo}
\begin{leftbar}
Le nouveau gérant d'un restaurant s'aperçoit qu'il possède des
carafes, des petites et des grandes, dont les contenances ne sont pas
indiquées.
\\\og Bof, ce n'est pas grave, je vais attendre ce soir lors du 1\ier\
service.\fg
\\En effet, lors de ce service, il s'aperçoit que les 1\,000~cL de vin
qu'il avait préparé ont été vendus à l'aide de 5 petites carafes et 10
grandes.
\\Fier, le restaurateur en conclue qu'une petite carafe contient 50~cL
et une grande 75~cL.
\\Mais, son fils, bon en calcul mental, lui fait remarquer que si la
petite contient 30~cL et la grande 85~cL alors l'expérience serait
vraie aussi.
\\Le gérant, perplexe, fait alors une nouvelle expérience et il
s'aperçoit que deux petites carafes remplissent une grande.
\\\og Alors, là, j'ai trouvé !\fg.
\end{leftbar}
Peux-tu expliquer pourquoi ?
\end{Exo}

\begin{Exo}
\begin{center}
  \psset{unit=5mm}
  \begin{pspicture}(0,-1)(9,6)
    \psline(0,0)(9,0)(9,6)(4,6)(4,0)
    \psline(0,0)(4,4)
    \pcline[offset=-5pt]{<->}(0,0)(4,0)
    \Bput{4 cm}
    \pcline[offset=-5pt]{<->}(4,0)(9,0)
    \Bput{5 cm}
    \pcline[offset=-5pt]{<->}(4,0)(4,4)
    \Bput{$x$}
    \pcline[offset=5pt]{<->}(4,4)(4,6)
    \Aput{2 cm}
    \psline(8.6,0)(8.6,0.4)(9,0.4)
    \psline(8.6,6)(8.6,5.6)(9,5.6)
    \psline(4,5.6)(4.4,5.6)(4.4,6)
    \psline(3.6,0)(3.6,0.4)(4,.4)
  \end{pspicture}
\end{center}
La figure ci-dessus est composée d'un rectangle et d'un triangle
rectangle. $x$ désigne une mesure exprimée en centimètres.

%\begin{enumerate}
%  \item Exprimer les dimensions (largeur et longueur) du
%  rectangle, en fonction de $x$.
%  \item Exprimer l'aire du rectangle en fonction de $x$.
%  \item Exprimer l'aire du triangle en fonction de $x$.
%  \item Exprimer l'aire de la figure en fonction de $x$.
%  \item \'Ecrire une équation qui traduit la phrase : <<~l'aire de la
%  figure vaut $26,1\,\mathrm{cm}^2$.~>>
%  \item
 Quelle valeur doit-on donner à $x$ pour que l'aire de la
  figure soit égale à $26,1\,\mathrm{cm}^2$ ?
%\end{enumerate}
\end{Exo}


\begin{Exo}
%Exercice qui peut être repris en posant x=nb pièces de 20 c. On arrive
%alors au calcul : (-11.1/-0.3)
Renaud a mis dans une boîte des pièces de 20 centimes et des pièces de
50 centimes. Il y a 100 pièces en tout.
%\begin{enumerate}
%  \item S'il y avait 30 pièces de 50 centimes, combien y aurait-il de
%  pièces de 20 centimes ? Combien d'argent cela ferait-il en tout dans
%  la boîte ?
%  \item Il n'y a pas 30 pièces de 50 centimes dans la boîte. Nous
%  désignerons le nombre de pièces de 50 centimes par la lettre $x$.\\
%  Exprimer, en fonction de $x$, le nombre de pièces de 20 centimes.
%  \item Exprimer, en fonction de $x$ la somme d'argent dans la boîte.
%  \item 

Il y a 38,90\euro{} dans la boîte. Trouver le nombre de pièces
  de 50 centimes.
%\end{enumerate}
\end{Exo}

\begin{Exo}
\begin{center}
  \psset{unit=5mm}
  \begin{pspicture}(4,4)
    \psline(0,0)(4,0)(4,4)(0,4)(0,0)
    \psline(0,0)(2.7,4)
    \pcline[offset=5pt]{<->}(0,4)(2.7,4)\Aput{$x$}
    \pcline[offset=-5pt]{<->}(4,0)(4,4)\Bput{10 cm}
    \uput[ul](0,4)A\uput[dl](0,0)B\uput[dr](4,0)C\uput[ur](4,4)D
    \uput[ur](2.7,4)E
  \end{pspicture}
\end{center}
ABCD est un carré et E est un point sur le segment [AD].
%On note $x$, la longueur AE, exprimée en centimètres.
%\begin{enumerate}
%  \item Exprimer, en fonction de $x$, l'aire du triangle ABE.
%  \item 

Quelle valeur faut-il donner à $x$ pour que l'aire du triangle
  soit égale au quart de l'aire du carré ?
%\end{enumerate}
\end{Exo}

\begin{Exo}
Nadège dit : <<~Dans 11 ans, mon âge sera égal au double de l'âge que
j'avais il y a 13 ans.~>>\\
%On désignera par $x$, l'âge de Nadège aujourd'hui.
%\begin{enumerate}
%  \item Exprimer en fonction de $x$, l'âge qu'aura Nadège dans 11 ans.
%  \item Exprimer en fonction de $x$, l'âge qu'avais Nadège il y a 13
%  ans.
%  \item \'Ecrire une équation qui correspond à l'affirmation de
%  Nadège.
%  \item

 Quel âge a Nadège ?
%\end{enumerate}
\end{Exo}

\begin{Exo}
\begin{center}
  \psset{unit=5mm}
  \begin{pspicture}(12,3.5)
    \psline(0,0)(10,0)(10,1)
    \psline(0,1)(11,1)(11,4)(6,4)(6,1)(0,1)(0,0)
    \pcline[offset=-5pt]{<->}(10,1)(11,1)\bput[2pt]{:U}(.8){1 cm}
    \pcline[offset=-5pt]{<->}(11,1)(11,4)\Bput{3 cm}
    \pcline[offset=5pt]{<->}(0,1)(6,1)\Aput{6 cm}
    \pcline[offset=5pt]{<->}(0,0)(0,1)\Aput{1 cm}
    \pcline[offset=5pt]{<->}(6,1)(10,1)\Aput{$x$}
  \end{pspicture}
\end{center}
La figure ci-dessus est composée de deux rectangles. Les dimensions
sont indiquées sur la figure. L'une d'elle est notée $x$ et est exprimée en centimètres.\\
Quelle valeur peut-on donner à $x$ pour les deux rectangles aient la
même aire ?
\end{Exo}

\begin{Exo}
Dans une boulangerie, un pain au chocolat coûte 0,30\euro\ de plus
qu'un croissant.\\
On paye le même prix pour 10 croissants et 4 pains au chocolat que
pour 4 croissants et 8 pains au chocolat.\\
Quel est le prix d'un croissant ?
\end{Exo}

\begin{Exo}
\begin{center}
  \begin{pspicture}(-.25,-.25)(5,2.25)
\psline(0,0)(3,0)%@3= segment [AB]
\psline(3,0)(3,2)%@5= segment [BC]
\psline(2.925,0.9)(3.075,0.9)%Marqueur : 2 traits
\psline(2.925,1)(3.075,1)%Marqueur : 2 traits
\psline(3,2)(0,2)%@7= segment [CD]
\psline(0,2)(0,0)%@8= segment [DA]
\psline(3,2)(4.732051,1)%@13= segment [CE]
\psline(3.816923,1.614952)(3.741923,1.485048)%Marqueur : 2 traits
\psline(3.903525,1.564952)(3.828525,1.435048)%Marqueur : 2 traits
\psline(4.732051,1)(3,0)%@14= segment [EB]
\psline(3.990128,0.485048)(3.915128,0.614952)%Marqueur : 2 traits
\psline(3.903525,0.435048)(3.828525,0.564952)%Marqueur : 2 traits
%\psdots(0,0)%Point A
\uput{-3pt}[u](-0.175,-0.175){A}
%\psdots(3,0)%Point B
\uput{-3pt}[u](3,-0.25){B}
%\psdots(3,2)%Point C
\uput{-3pt}[u](3.1,2.25){C}
%\psdots(0,2)%Point D
\uput{-3pt}[u](-0.175,2.175){D}
%\psdots(4.732051,1)%Point E
\uput{-3pt}[u](4.975,1){E}
\pcline[offset=5pt]{<->}(0,2)(3,2)\Aput{8 cm}
\end{pspicture}
\end{center}
ABCD est un rectangle.\\
Quelle doit-être la longueur du côté du triangle équilatéral pour que
le rectangle et le triangle aient le même périmètre ?
\end{Exo}

\begin{Exo}
Au magasin, on achète des packs de bouteilles de jus d'orange.

On achète 10 packs à moitié prix et encore 7 packs avec 7 bons de
réduction de 2\euro. On paye en tout 55,60\euro.

Quel est le prix d'un pack (sans réduction) ?
\end{Exo}

\begin{Exo}
{\em Toutes les valeurs sont en centimètres.}\\
Soit $ABC$ un triangle rectangle en $A$. On suppose que $3<AB<5$ et 
$4<AC<6$.
\begin{enumerate}
\item Donne un encadrement de $BC$.
\item Donne un encadrement du périmètre du triangle.
\item Donne un encadrement de l'aire du triangle.
\end{enumerate}
\end{Exo}

\begin{Exo}
La mesure du rayon d'un disque est de 3,5~cm. Sachant que
$3,14<\pi<3,15$, trouve un encadrement au centième du périmètre et de
l'aire de ce cercle.
\end{Exo}

\end{document}