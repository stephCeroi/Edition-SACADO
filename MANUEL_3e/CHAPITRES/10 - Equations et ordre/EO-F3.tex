\begin{titre}[Équations et ordre]

\Titre{Inéquation du premier degré}{2,5}
\end{titre}


\begin{CpsCol}
\textbf{Utiliser le calcul littéral}
\begin{description}
\item[$\square$] Résoudre une inéquation du premier degré
\item[$\square$] Faire le lien entre forme algébrique et représentation graphique
\end{description}
\end{CpsCol}


\Rec{1}{EO-0}


\begin{DefT}{Inéquation}
On appelle \textbf{inéquation}\index{Inéquation} une proposition mathématique qui compare deux expressions dont au moins une contient une inconnue. 

\textbf{Résoudre une inéquation} \index{Inéquation!Résoudre}, c'est déterminer toutes les valeurs de l'inconnue qui vérifient la comparaison entre les deux membres.
\end{DefT}

\begin{Reg}
\begin{enumerate}
\item Lorsqu'on ajoute (ou soustrait ) un \textbf{même} nombre à \textit{chaque membre} de l'inéquation, l'ordre ne change pas.
\item Lorsqu'on multiplie (ou divise ) un \textbf{même} nombre \textbf{positif non nul} à \textit{chaque membre} de l'inéquation, l'ordre ne change pas.
\item \rotatebox{90}{{\color{red}$\blacktriangleright$}} Lorsqu'on multiplie (ou divise ) un \textbf{même} nombre \textbf{négatif non nul} à \textit{chaque membre} de l'inéquation, l'ordre change.
\end{enumerate}
\end{Reg}


\begin{Mt}   
On souhaite résoudre l'inéquation $x + 7 < 5$
\begin{description}
\item[1a. Isoler $x$.] En soustrayant 7 à chaque membre, on obtient : $x + 7 - 7 < 5 - 7$. L'ordre ne change pas car on opère une soustraction.
\item[1b. Simplifier.]  $x +0 < - 2$ 
\item[1c. Réduire.]  $x < - 2$  
\item[2. Conclure.] La solution est composée de tous les nombres strictement inférieurs  à $-2$.
\item[3. Représentation] On représente l'ensemble solution sur la droite graduée, ici en rouge :
 
\begin{tikzpicture}[line cap=round,line join=round,>=triangle 45,x=0.5cm,y=1.0cm]
\draw[->,color=black] (-10.193119401716665,0.) -- (10.5522947678536005,0.);
\foreach \x in {-10.,-8.,-6.,-4.,-2.,2.,4.,6.,8.,10.}
\draw[shift={(\x,0)},color=black] (0pt,2pt) -- (0pt,-2pt) node[below] {\footnotesize $\x$};
\draw[color=black] (0pt,-10pt) node[right] {\footnotesize $0$};
\clip(-10.193119401716665,-0.5) rectangle (10.5522947678536005,0.5);
\draw [line width=3.4pt,color=red] (-2.,0.)-- (-12.,0.);
\end{tikzpicture}
\end{description}
\end{Mt} 

\begin{Mt}   
On souhaite résoudre l'inéquation $-2x - 5 \geq 11 $
\begin{description}
\item[1a. Isoler $x$.] En ajoutant 5 à chaque membre, on obtient :  $-2x - 5 +5 \geq 11 +5 $. L'ordre ne change pas car on opère une addition.
\item[1b. Simplifier.]  $-2x  \geq 16 $.
\item[2a. Diviser.]  $\frac{-2}{-2}x  \leq \frac{16}{-2}$. On divise chaque membre par $-2 <0$. Donc {\color{red}l'ordre change}.  
\item[2b. Simplifier.]  $x  \leq -8$  
\item[2. Conclure.] La solution est composée de tous les nombres inférieurs ou égaux à $-8$.
\item[3. Représentation]
On représente l'ensemble solution sur la droite graduée, ici en rouge : 

\begin{tikzpicture}[line cap=round,line join=round,>=triangle 45,x=0.5cm,y=1.0cm]
\draw[->,color=black] (-20.193119401716665,0.) -- (2.5522947678536005,0.);
\foreach \x in {-20.,-18.,-16.,-14.,-12.,-10.,-8.,-6.,-4.,-2.,2.}
\draw[shift={(\x,0)},color=black] (0pt,2pt) -- (0pt,-2pt) node[below] {\footnotesize $\x$};
\draw[color=black] (0pt,-10pt) node[right] {\footnotesize $0$};
\clip(-20.193119401716665,-1.262783989122121) rectangle (2.5522947678536005,1.1022719129769702);
\draw [line width=3.4pt,color=red] (-7.929866576017677,0.)-- (-22.96695039800572,0.);
\end{tikzpicture}
\end{description}
\end{Mt} 


\Exo{1}{EO-25}

\Exo{1}{EO-35}

\mini{
\Exo{1}{EO-27}


}{
\Exo{1}{EO-28}

}


