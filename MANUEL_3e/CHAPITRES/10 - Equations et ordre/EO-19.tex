
Le spectre des couleurs visibles est donné par le tableau ci dessous.

\begin{center}
\begin{tabular}{|c|c|c|c|}
\hline 
Couleur &  & Longueur d'ondes (nm) & Fréquence (Thz)\\ 
\hline 
violet & \cellcolor{purple} & 380 à 450   & 725 \\ 
\hline 
bleu & \cellcolor{blue} & 450 à 490   & 640 \\ 
\hline 
vert & \cellcolor{vert} & 490 à 570   & 565 \\ 
\hline 
jaune & \cellcolor{yellow} & 570 à 585   & 520 \\ 
\hline 
orange & \cellcolor{orange} & 585 à 620   & 500 \\ 
\hline 
rouge & \cellcolor{red} & 620 à 670  & 465 \\ 
\hline 
\end{tabular}
\end{center}
 
\begin{enumerate}
\item En nommant clairement l'inconnue et en posant une équation, modélise chaque donnée.
\begin{enumerate}
\item Je vois une couleur, si j'enlève 160 Tera Hertz à sa  longueur d'onde, j'obtiens la couleur verte. Quelle est la couleur que je vois ?
\item Je vois la couleur orange. J'augmente la fréquence de 225 THz. Quelle est la couleur que je vois ?
\item Je vois la couleur bleue et en modulant la fréquence je vois la couleur rouge. Quelle est la variation de fréquence ?
\end{enumerate}
\item Résous alors chaque équation et réponds aux questions.
\end{enumerate}