
Il y a 3 ans, Cécile avait le tiers de l'âge de son père. Son père a aujourd'hui 39 ans. 

\begin{enumerate}
\item En notant $x$ l'âge de Cécile, laquelle de ces équations modélise le problème ?
\begin{tabular}{cccc}
\textbf{a.} $(x-3)\div 3=39-3$ & \textbf{b.} $3x-3=39\div 3$ & \textbf{c.} $(x-3)\div 3=39$ & \textbf{d.} $3(x-3)=39-3$ \\ 
\end{tabular}
\item Quel est l'âge de Cécile ?
\item Quel est l'âge  de son père ?
 \end{enumerate}