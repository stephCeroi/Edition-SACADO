\documentclass[openany]{book}

\input{../../../latex_preambule_style/preambule}
\input{../../../latex_preambule_style/styleCoursCycle4}
\input{../../../latex_preambule_style/styleExercices}
\input{../../../latex_preambule_style/styleExercicesAideCompetences}
%\input{../../latex_preambule_style/styleCahier}
\input{../../../latex_preambule_style/bas_de_page_cycle4}
\input{../../../latex_preambule_style/algobox}



%%%%%%%%%%%%%%%  Affichage ou impression  %%%%%%%%%%%%%%%%%%
 \usepackage{geometry}
 \geometry{top=1.5cm, bottom=0cm, left=2cm , right=2cm}
%%%%%%%%%%%%%%%%%%%%%%%%%%%%%%%%%%%%%%%%%%%%%%%%

\begin{document}


\begin{seance}[Le calcul littéral]

\Titre{Les identités remarquables}{3}
\end{seance}


\begin{CpsCol}
\textbf{Utiliser le calcul littéral}
\begin{description}
\item[$\square$] Utiliser le calcul littéral
\item[$\square$] Utiliser les égalités remarquables
\end{description}
\end{CpsCol}

\Rec{1}{CL-61}

\Rec{1}{CL-62}


\Rec{1}{CL-48}


\begin{ThT}{Les identités remarquables}
Pour tous nombres $a$ et $b$, 
\begin{description}
\item $(a+b)^2=a^2+2ab+b^2$
\item $(a-b)^2=a^2-2ab+b^2$
\item $(a-b)(a+b)=a^2-b^2$
\end{description}
\end{ThT}

\begin{Rq}
Dans la propriété ci dessus, le membre de gauche est la forme factorisée et dans celui de droite est la forme développée.
\end{Rq}


\AD{1}{CL-54}


\AD{1}{CL-00}

\begin{seance}[Le calcul littéral]

\Titre{Les identités remarquables}{3}
\end{seance}

\begin{Rq}
Les identités remarquables sont primordiales dans la \textbf{factorisation} d'expressions.
\end{Rq}

\begin{ThT}{Les identités remarquables}
Pour tous nombres $a$ et $b$, 
\begin{description}
\item [Forme factorisée] $(a-b)(a+b)=a^2-b^2$ [Forme développée] 
\end{description}
\end{ThT}

\DNB{1}{Métropole Juin}{CL-03}

\DNB{1}{Métropole Septembre}{CL-04}



\begin{seance}[Le calcul littéral]

\Titre{Vus au DNB}{3}
\end{seance}

\DNB{1}{Pondichery}{CL-02}


\DNB{1}{Amérique du Nord}{CL-01}



\begin{seance}[Le calcul littéral]

\Titre{Quelques défis}{3}
\end{seance}

\begin{description}
\item[Développement] https://www.geogebra.org/m/fgHDPjEF
\item[Factorisation] https://www.geogebra.org/m/WzrRd7Kz
\item[Identités remarquables] https://www.geogebra.org/m/FAGxTYCk
\end{description}


\begin{minipage}{0.49\linewidth}
\App{1}{CL-46}

\PO{1}{CL-65}
\end{minipage}
\hfill
\begin{minipage}{0.49\linewidth}
\PO{1}{CL-64}
\end{minipage}




\end{document}
