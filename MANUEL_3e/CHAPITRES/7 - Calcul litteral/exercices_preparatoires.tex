 \documentclass[10pt]{article}


\input{../../../latex_preambule_style/preambule}
\input{../../../latex_preambule_style/styleCoursCycle4}
\input{../../../latex_preambule_style/styleExercices}

\input{../../../latex_preambule_style/bas_de_page_troisieme}
\usepackage{geometry}
\geometry{top=3cm, bottom=0cm, left=2cm , right=2cm}
\begin{document}

%%%%%%%%%%%%%%%%%%%%%%%%%%%%%%%%%%%%%%%%%%%%%%%
%%%%		 Titre encadré
%%%%%%%%%%%%%%%%%%%%%%%%%%%%%%%%%%%%%%%%%%%%%%%
\begin{titre}{Travail préparatoire}
{\LARGE Techniques de calcul }   \\

 

 
\end{titre}


 
%%%%%%%%%%%%%%%%%%%%%%%%%%%%%%%%%%%%%%%%%%%%%%%
%%%%		 Corps du document
%%%% \ExoSN   exercice non noté
%%%% \ExoN{2} exercice noté sur 2 points
%%%%%%%%%%%%%%%%%%%%%%%%%%%%%%%%%%%%%%%%%%%%%%%


\bigskip

\subsection*{exercice}

Simplifier et réduire les expressions suivantes :
 
 \begin{enumerate}
  \item $5x + 2 -  2x - 1  = $ \ligne{1} \vplus  
  \item $4y - 5  -y -1  = $ \ligne{1} \vplus
  \item $13x - 2(3x - 3) = $ \ligne{1} \vplus
 \end{enumerate}

\subsection*{exercice}

Développer les expressions suivantes en explicitant les calculs :
 
 \begin{enumerate}
  \item $(x + 2)(2x - 1) = $ \ligne{1} \vplus  
  \item $(3x - 2)(3x + 2) = $ \ligne{1} \vplus
 \end{enumerate}

\subsection*{exercice}

Mettre en évidence et identifier le facteur commun de chaque expression :

 \begin{enumerate}
 \item  $2x - 2a   $ \ligne{1} \vplus
  \item  $3x - 9   $ \ligne{1} \vplus
  \item  $4(2x - 7) + (2x - 7)(4x + 1)   $ \ligne{1} \vplus
  \item  $(x - 2)(2x + 1) - (x - 2)(x - 2)   $ \ligne{1} \vplus
 \end{enumerate}
 
 
\end{document}