\documentclass[openany]{book}

\input{../../../latex_preambule_style/preambule}
\input{../../../latex_preambule_style/styleCoursCycle4}
\input{../../../latex_preambule_style/styleExercices}
\input{../../../latex_preambule_style/styleExercicesAideCompetences}
%\input{../../latex_preambule_style/styleCahier}
\input{../../../latex_preambule_style/bas_de_page_cycle4}
\input{../../../latex_preambule_style/algobox}


%%%%%%%%%%%%%%%  Affichage ou impression  %%%%%%%%%%%%%%%%%%
\newcommand{\impress}[2]{
\ifthenelse{\equal{#1}{1}}  %   1 imprime / affiche sur livre  -----    0 affiche sur cahier 
{%condition vraieé
#2
}% fin condition vraie
{%condition fausse
}% fin condition fausse
} % fin de la procédure
%%%%%%%%%%%%%%%  Affichage ou impression  %%%%%%%%%%%%%%%%%%

%%%%%%%%%%%%%%%%%%%%%%%%%%%%%%%%%%%%%%%%%%%%%%%%

\begin{document}


\begin{seance}[Probabilités]

\begin{description}[leftmargin=*]
\item[$\square$] Aborder les questions relatives au hasard à partir de problèmes simples de la vie courante.
\item[$\square$] Connaitre les propriétés de base : la probabilité d’un événement est comprise entre 0 et 1, probabilité d'évènements certains, impossibles, contraires. 
\item[$\square$] Comprendre et utiliser des notions élémentaires de probabilités.
\end{description}
\end{seance}

\AD{1}{Prob-40}

\EPC{1}{Prob-38}{Chercher. Représenter.}

\TP{1}{Prob-36}



\vspace{1cm}

\includegraphics[scale=1]{seance1.png} 

\begin{seance}[Probabilités]

\begin{description}
\item[$\square$] Comprendre et utiliser des notions élémentaires de probabilités.
\item[$\square$] Calculer des probabilités dans des cas d'équiprobabilité.
\end{description}
\end{seance}

\Fl{1}{Prob-30}

\EPC{1}{Prob-31}{Chercher. Représenter.}

\begin{minipage}{.68\linewidth}
\EPC{1}{Prob-14}{Chercher. Représenter.}
\end{minipage}
\hfill
\begin{minipage}{.28\linewidth}
\includegraphics[scale=1]{seance2.png}
\end{minipage}


\begin{seance}[Probabilités]

\begin{description}
\item[$\square$] Savoir modéliser par un tableur des expériences aléatoires
à deux épreuves.
\end{description}
\end{seance}

\DNB{1}{en Polynésie - 2017}{Prob-33}



\EPC{1}{Prob-0}{Représenter. Chercher. Calculer.}

\includegraphics[scale=1]{seance5.png} 


\begin{seance}[Probabilités]

\begin{description}
\item[$\square$] Faire le lien entre fréquence et probabilité, en constatant matériellement le phénomène de stabilisation des fréquences.
\end{description}
\end{seance}

\Rec{1}{Prob-37}

\begin{seance}[Probabilités]

\begin{description}
\item[$\square$] Calculer des probabilités dans des cas d'équiprobabilité.
\end{description}
\end{seance}

\EPC{1}{Prob-32}{Chercher. Représenter.}


\DNB{1}{en Métropole - 2017 - Sujet de remplacement}{Prob-35}


\EPC{1}{Prob-34}{Communiquer. Chercher.}

\includegraphics[scale=1]{seance4.png} 



\end{document}