
On considère l'algorithme suivant, où $A$ et $B$ sont des entiers naturels tels que $A < B$ : 

\begin{center}
\begin{tabular}{|l l|}\hline
\textbf{Entrées :}& $A$ et $B$ entiers naturels tels que $A < B$\\
&\\ 
\textbf{Variables :}& $D$ est un entier\\ 
&Les variables d'entrées $A$ et $B$ \\
&\\
\textbf{Traitement :}&\\ 
&Affecter à  $D$ la valeur de $B - A$\\ 
&\\
&Tant que $D > 0$\\ 
&$B$ prend la valeur de $A$\\ 
&$A$ prend la valeur de $D$\\ 
&\hspace{0,5cm}Si $B > A$ Alors\\ 
&\hspace{1cm}$D$ prend la valeur de $B - A$\\ 
&\hspace{0,5cm} Sinon\\ 
&\hspace{1cm}$D$ prend la valeur de $A - B$\\ 
&\hspace{0,5cm}Fin Si\\ 
&Fin Tant que\\ 
&\\
\textbf{Sortie :} &Afficher $A$\\ \hline
\end{tabular} 
\end{center}

On entre $A = 12$ et $B = 14$. 

En remplissant le tableau donné en \textbf{annexe}, déterminer la valeur affichée par l'algorithme.

\bigskip

\textbf{Annexe}\\
\begin{tabularx}{0.6\linewidth}{|*{3}{>{\centering \arraybackslash}X|}}\hline
$A$ &  $B$&   $D$\\ \hline   
12&   14& \\ \hline    
&	&\\ \hline
&	&\\ \hline
&	&\\ \hline
&	&\\ \hline
&	&\\ \hline
&	&\\ \hline
&	&\\ \hline
&	&\\ \hline
&	&\\ \hline
&	&\\ \hline
\end{tabularx}
