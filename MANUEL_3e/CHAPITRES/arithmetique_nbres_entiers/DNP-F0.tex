\begin{titre}[Arithmétique]

\Titre{Multiples et diviseurs}{1,5}
\end{titre}


\begin{CpsCol}
\textbf{Comprendre et utiliser les notions de divisibilité et de nombres premiers}
\begin{description}
\item[$\square$] Utiliser la définition de multiple ou un diviseur
\item[$\square$] Déterminer des multiples ou des diviseurs d'un nombre donné
\item[$\square$] Déterminer si un entier est un multiple ou un diviseur d'un autre entier 
\item[$\square$] Utiliser les critères de divisibilité par 2, 3, 5, 9, 10 
\item[$\square$] Modéliser et résoudre des problèmes mettant en jeu la divisibilité (engrenages, conjonction de phénomènes, etc.).
\end{description}
\end{CpsCol}

\ExeComp{Chercher.}

Déterminer les diviseurs des entiers 12, 25, 306 et 124. 




\Rec{1}{DNP-53}



\begin{DefT}{ Multiples et diviseurs}
Dire que l´entier naturel $a$ est \textbf{multiple} de l'entier naturel $b$ signifie qu'il existe un entier naturel $k$ tel que $a=b \times k$.\\
Autrement dit: Le reste de la division euclidienne de $a$ par $b$ vaut $0$.\\
Autrement dit: Le nombre $b$ et dans la table de multiplication du nombre $a$.\\
On dit aussi dans ce cas que $b$ est un \textbf{diviseur} de $a$ ou que $a$ est \textbf{divisible} par $b$
\end{DefT}



\begin{Ex}
$252=36 \times 7$\\
On peut donc dire que $252$ est un multiple de $7$ , et aussi de $36$.\\
On peut aussi dire que $7$ est un diviseur de $252$.\\
Ou: $252$ a pour diviseur $7$ ou  $252$ est divisible par $7$ , et aussi par $26$.
\end{Ex}

 \ExeComp{Chercher. Calculer.}
 
\begin{enumerate}
\item Déterminer la liste des diviseurs de 156 et 130. 
\item Déterminer le plus grand diviseur commun de 156 et 130.
\end{enumerate}


\ExeComp{Chercher. Représenter.}

Deux ampoules clignotent. L'une s'allume toutes les 153 secondes et l'autre toutes les 187 secondes. À minuit, elles s'allument ensemble.
Détermine l'heure à laquelle elles s'allumeront de nouveau ensemble.

\ExeComp{Chercher. Représenter.}


Un phare émet trois signaux différents, le premier toutes les 16 secondes, le second toutes les 45 secondes, le troisième toutes les 2 minutes 30 secondes. Ces trois signaux sont émis simultanément à minuit.
\begin{enumerate}
  \item \`A quels intervalles de temps sont émis simultanément deux de ces signaux (premier et deuxième, ou premier et troisième, ou deuxième et troisième) ?
  \item \`A quels intervalles de temps les trois signaux sont-ils émis simultanément ?
\end{enumerate}






\ExeComp{Chercher. Représenter.}


Dire qu'un entier naturel est {\em parfait} signifie qu'il est égal à la somme de ses diviseurs propres, c'est-à-dire ses diviseurs différents de lui-même.\\Ainsi, le chiffre 6 est parfait car ses diviseurs sont 1, 2, 3 et 6 et la somme de ses diviseurs propres est donc :
\[1+2+3=6\]
\begin{enumerate}
\item Vérifie que les nombres 28 et 496 sont parfaits.
\item Recherche d'autres nombres parfaits \textit{(on pourra effectuer une recherche sur internet)}
\end{enumerate}



%\PO{1}{DNP-0}

\ExeComp{Communiquer. Représenter.}

Le système des temps est basée sur 60 et non sur 100 ! 
\begin{enumerate}
\item Déterminer le nombre de diviseurs de 100, puis de 60. 
\item Est-ce que cela facilite les calculs ? Préparer un argumentaire.
\end{enumerate}


\begin{scriptsize}
Ressources : 
https://www.futura-sciences.com/sciences/questions-reponses/physique-heure-dure-60-minutes-minute-60-secondes-7338/
\end{scriptsize}





\begin{Reg}[Critères de divisibilités]

Un nombre est divisible :
\begin{itemize}
\item[$\bullet$] Par 2 lorsque son chiffre des unités est $0$ , $2$, $4$, $6$ ou $8$;
\item[$\bullet$] Par 3 lorsque la somme de ses chiffres est un multiple de $3$;
\item[$\bullet$] Par 4 lorsque son chiffre des dizaines et celui des unités forment un nombre multiple de 4.
\item[$\bullet$]Par 5 lorsque don chiffre des unités est $0$ ou $5$.
\item[$\bullet$]Par 9 Lorsque la somme de ses chiffres est un multiple de $9$.
\item[$\bullet$] Par 10 lorsque son chiffre des unités est $0$.
\end{itemize} 
\end{Reg}

 
  \ExeComp{Raisonner.}

Damien affirme que : "Si un nombre est divisible par 3 alors il est divisible par 9". Qu'en pensez-vous ?


  \ExeComp{Chercher.}

Je suis un nombre entier pair, compris entre 100 et 400, divisible par 3, par 5 et par 11. Qui suis-je ?


\ExeComp{ Raisonner.}

\begin{enumerate}
\item Donner trois multiples de 6.
\item Donner une écriture littérale de tous les multiples de 6.
\item En déduire que tous les multiples de 6 sont des multiples de 2 et de 3.
\end{enumerate}



\ExeComp{ Chercher.}

Le professeur d'EPS organise un tournoi de Softball avec toutes les classes de Troisième du collège. Il souhaite qu'il y ait dans chaque équipe le même nombre de garçons, le même nombre de filles, qu'il n'y ait aucun remplaçant et qu'une équipe soit composée entre 8 et 15 joueurs.

Sachant qu'il y a 72 filles et 108 garçons, donner toutes les compositions possibles d'équipes.


\ExeComp{ Chercher.}

Louison veut réaliser un collier de perle. Elle empile les perles de la façon suivante : une perle rouge puis 4 perles bleues puis trois perles blanches et ainsi de suite. Quelle est la couleur de la 109$^\text{ème}$ perle ?

%\PO{1}{DNP-43} 

\sacado{50b20772}

\sacado{d4b51e3a}
 

\sacado{28eb3b51}
 

 
 