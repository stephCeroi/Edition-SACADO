\begin{titre}[Arithmetique]

\Titre{Applications à la simplification de fractions}{2}
\end{titre}

\vspace{2cm}

{\Large {\color{violet}Aucun exercice sur ce chapitre dans le cahier}}


\begin{autoeval}
\begin{tabular}{p{12cm}p{0.5cm}p{0.5cm}p{0.5cm}p{1cm}}
\textbf{Compétences visées} &  M I & MF & MS  & TBM \vcomp \\ 
Savoir déterminer une fraction irréductible & $\square$ & $\square$  & $\square$ & $\square$ \vcomp \\
Utiliser le PGCD de deux entiers pour simplifier une fraction et la rendre irréductible & $\square$ & $\square$  & $\square$ & $\square$ \vcomp \\  
\end{tabular}
{\footnotesize MI : maitrise insuffisante ; MF = Maitrise fragile ; MS = Maitrise satisfaisante ; TBM = Très bonne maitrise}
\end{autoeval}

\vspace{2cm}

\begin{titre}[Arithmetique]

\Titre{Démonstrations en arithmétique}{3}
\end{titre}


\vspace{2cm}

{\Large {\color{violet}Aucun exercice sur ce chapitre dans le cahier}}


\begin{autoeval}
\begin{tabular}{p{12cm}p{0.5cm}p{0.5cm}p{0.5cm}p{1cm}}
\textbf{Compétences visées} &  M I & MF & MS  & TBM \vcomp \\ 
Utiliser la définition de diviseurs et multiples pour démontrer les critères de divisibilité & $\square$ & $\square$  & $\square$ & $\square$ \vcomp \\
Utiliser la notation an base 10 & $\square$ & $\square$  & $\square$ & $\square$ \vcomp \\  
\end{tabular}
{\footnotesize MI : maitrise insuffisante ; MF = Maitrise fragile ; MS = Maitrise satisfaisante ; TBM = Très bonne maitrise}
\end{autoeval}
