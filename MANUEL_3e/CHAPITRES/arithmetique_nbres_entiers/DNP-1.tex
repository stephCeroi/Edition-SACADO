
Ecrire la \textbf{division euclidienne} d´un nombre entier naturel $a$ par un entier naturel $b$, tous deux non nuls, c'est déterminer les nombres entiers $q$ et $r$ tels que $a=b \times q + r$ avec $0 \leq r < b$\\
$q$ est appelé le \textbf{quotien}t de la division euclidienne de $a$ par $b$.\\
$r$ est appelé le \textbf{reste} de la division euclidienne de $a$ par $b$. 

