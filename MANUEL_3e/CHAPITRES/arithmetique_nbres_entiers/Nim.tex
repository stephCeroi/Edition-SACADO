\begin{titre}[Arithmétique]

\Titre{Le jeu de Nim}{1}
\end{titre}
 

\begin{CpsCol}
\textbf{ Utiliser les nombres}
\end{CpsCol}


 
 \vspace{0.4cm}

\textbf{ Partie A}
\vspace{0.4cm}

On dispose au départ de 13 allumettes; chaque joueur, à tour de rôle,
en enlève 1, 2 ou 3. Celui qui prend la dernière allumette
a gagné.

\medskip

Exemple de partie : le joueur A commence.

\centerline{\begin{tabular}{|c|c|c|}
\hline
A prend ... allumettes & 
B prend ... allumettes &
allumettes restantes\\
\hline
1   & & 12  \\\hline 
   & 1 & 11\\\hline 
 2 &  & 9\\\hline 
   & 1  & 8\\\hline
 3 &    & 5\\\hline  
   & 2  & 3\\\hline
 2 &    & 1\\\hline
%   & 1  & 0\\\hline
\end{tabular}
}
Dans cette partie, B a perdu.

 
\begin{enumerate}
\item Jouez !
\item Expliquer pourquoi, s'il reste 1, 2 ou 3 allumettes, le joueur
dont c'est le tour peut gagner. Comment doit-il procéder ?

{\em On dira que 1, 2 et 3 sont des « positions gagnantes »}

\item Montrer que s'il reste 4 allumettes, le joueur dont c'est le tour
est sûr de perdre, si l'autre joueur joue correctement.

{\em On dira que 4 est une « position perdante »}

\item Déterminer toutes les positions gagnantes et perdantes.
\end{enumerate}

\vspace{0.4cm}
\textbf{ Partie B. Vers le codage}
 \vspace{0.4cm}

L'objectif est d'écrire un programme pour jouer contre l'ordinateur
à ce jeu. 
On suppose que le joueur humain commence.
%L'ordinateur sera le joueur « 1 ».

Convenons des variables suivantes :
\begin{itemize}
\item \verb!joueur! : un entier qui vaut 0 si c'est à l'humain de jouer,
1 si c'est à l'ordinateur.
\item \verb!position! : le nombre d'allumettes restantes.
\end{itemize}

\begin{enumerate}
\item Écrire un algorithme qui demande au joueur le nombre d'allumettes
qu'il veut prendre. On fera attention au fait que, par exemple, s'il reste
deux allumettes, le joueur ne peut pas en prendre trois !


 
\item Écrire un algorithme qui, étant donné le nombre d'allumettes
restantes, donne le nombre d'allumettes qu'il faut prendre pour mettre
l'adversaire sur une position perdante (si c'est possible).


\item Traduire avec Scratch les algorithmes précédents. Coder le jeu.
\end{enumerate}
 \vspace{0.4cm}
 {\bf Partie C. Compléments }
 \vspace{0.4cm}
 
Modifier le programme précédent pour que :
\begin{enumerate}
\item l'ordinateur commence à jouer;
\item l'utilisateur décide qui commence à jouer;
\item le nombre d'allumettes au départ soit choisi au hasard,
entre 13 et 50;
\item le gagnant soit le joueur qui prend la dernière allumette. 
\end{enumerate}
 

