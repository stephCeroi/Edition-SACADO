
Si $n$ est premier, il admet bien un diviseur premier: lui-même.

Si $n$ n'est pas premier alors il admet un plus petit diviseur positif $p\neq 1$.
$p$ est premier sinon $p$ aurait lui-même un diviseur positif différent de 1 qui serait un diviseur de $n$, mais plus petit que $p$.

De plus, $n$ peut s'écrire $n=p \times r$ avec $p\leq r$ donc $p^{2}\leq p \times r$ soit $p^{2} \leq n$ et $p\leq\sqrt{n}$

