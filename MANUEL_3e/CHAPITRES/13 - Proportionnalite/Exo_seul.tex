\documentclass[openany]{book}

\input{../../../latex_preambule_style/preambule}
\input{../../../latex_preambule_style/styleCoursCycle4}
\input{../../../latex_preambule_style/styleExercices}
\input{../../../latex_preambule_style/styleExercicesAideCompetences}
%\input{../../latex_preambule_style/styleCahier}
\input{../../../latex_preambule_style/bas_de_page_cycle4}
\input{../../../latex_preambule_style/algobox}



%%%%%%%%%%%%%%%  Affichage ou impression  %%%%%%%%%%%%%%%%%%
 \usepackage{geometry}
 \geometry{top=3cm, bottom=0cm, left=2cm , right=2cm}
%%%%%%%%%%%%%%%%%%%%%%%%%%%%%%%%%%%%%%%%%%%%%%%%

\begin{document}

\begin{atelier}[Proportionnalité]

\begin{description}
\item[$\square$] Résoudre des problèmes de pourcentage.
\item[$\triangleright$] Calculer des taux d'évolution.
\end{description}
\end{atelier}


\Exe


La note de restaurant suivante est partiellement effacée.  

Retrouvez les éléments manquants.

\begin{center}
\begin{tabularx}{0.7\linewidth}{|X>{\raggedleft \arraybackslash}X|}\hline
\multicolumn{2}{|c|}{RESTAURANT \og la Gavotte \fg }\\ \hline
4 menus à 16,50~\euro{} l'unité	& \ldots\ldots \\ 
1 bouteille d'eau minérale 		&\ldots\ldots\\ 
3 cafés à 1,20~\euro{} l'unité	&\ldots\ldots\\  
\textbf{Sous total} 			&\ldots\ldots\\ 
Service 5,5\:\% du sous total 	&\ldots\ldots\\  
\textbf{Total} 					&\ldots\ldots\\ \hline
\end{tabularx}
\end{center}




\Exe

\begin{enumerate}
\item Augmenter de 40\% c'est multiplier par \ldots
\item Diminuer de 25\% c'est multiplier par \ldots
\end{enumerate}


\Exe


Après une augmentation de 20\%, un objet coute 40 \euro{}.
Quel était son prix initial ?






\end{document}