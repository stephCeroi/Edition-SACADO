\begin{titre}[La proportionnalité]

\Titre{Taux d'évolution}{3}
\end{titre}

\begin{CpsCol}
\textbf{Calculer avec des grandeurs mesurables ; exprimer les résultats dans les unités adaptées}
\begin{description}
\item[$\square$] Résoudre des problèmes de pourcentage
\item[$\square$] Établir le lien entre coefficient multiplicateur et pourcentage
\end{description}
\end{CpsCol}


\begin{DefT}{Coefficient multiplicateur}\index{Coefficient multiplicateur}
Lorsque une valeur évolue selon un pourcentage de $t \%$, le coefficient multiplicateur est égal à $(1+t\%)$.
\end{DefT}


\begin{Ex}
\textit{La Taxe sur la Valeur Ajoutée en France est de 20,6\%. Un produit alimentaire coute 15,20 € Hors Taxe. Quel est sa valeur TTC ?}\\
Soit P le prix du produit alimentaire.\\
$P = 15,20 \times (1+20,6\%)= 15,20 \times (1+0,206) = 15,20 \times 1,206 \approx 18,33$.\\
Le prix TTC est donc 18,33 €.
\end{Ex}

\begin{Ex}
\textit{Le jour des soldes, une paire de ski à 428 € est soldée à 25\%. Quel est le prix après la solde ? }\\
Soit P le prix de la paire de ski.\\
$P = 428 \times (1-25\%)= 428 \times (1-0,25) = 428 \times 0,75 =321 $.\\
Le prix soldé est donc 321 €.
\end{Ex}

