
On donne la droite ci dessous.

\begin{center}
\definecolor{xdxdff}{rgb}{0.49019607843137253,0.49019607843137253,1.}
\definecolor{qqqqff}{rgb}{0.,0.,1.}
\definecolor{cqcqcq}{rgb}{0.7529411764705882,0.7529411764705882,0.7529411764705882}
\begin{tikzpicture}[line cap=round,line join=round,>=triangle 45,x=1.0cm,y=1.0cm]
\draw [color=cqcqcq,, xstep=1.0cm,ystep=1.0cm] (-0.44,-0.48) grid (10.6,7.72);
\draw[->,color=black] (0.,0.) -- (10.6,0.);
\foreach \x in {,1.,2.,3.,4.,5.,6.,7.,8.,9.,10.}
\draw[shift={(\x,0)},color=black] (0pt,2pt) -- (0pt,-2pt) node[below] {\footnotesize $\x$};
\draw[->,color=black] (0.,0.) -- (0.,7.72);
\foreach \y in {,1.,2.,3.,4.,5.,6.,7.}
\draw[shift={(0,\y)},color=black] (2pt,0pt) -- (-2pt,0pt) node[left] {\footnotesize $\y$};
\draw[color=black] (0pt,-10pt) node[right] {\footnotesize $0$};
\clip(-0.44,-0.48) rectangle (10.6,7.72);
\draw [color=qqqqff] (0.,2.)-- (10.,7.);
\begin{scriptsize}
\draw [color=xdxdff] (4.,4.)-- ++(-2.5pt,0 pt) -- ++(5.0pt,0 pt) ++(-2.5pt,-2.5pt) -- ++(0 pt,5.0pt);
\draw[color=xdxdff] (4.14,4.36) node {$A$};
\draw [color=xdxdff] (8.,6.)-- ++(-2.5pt,0 pt) -- ++(5.0pt,0 pt) ++(-2.5pt,-2.5pt) -- ++(0 pt,5.0pt);
\draw[color=xdxdff] (8.14,6.36) node {$B$};
\end{scriptsize}
\end{tikzpicture}
\end{center}

\begin{enumerate}
\item Cette droite représente-t-elle une situation de proportionnalité ?

\Point{2}
\item 
\begin{enumerate}
\item Place dans le tableau des coordonnées des points A et B de la droite.

\begin{tabular}{|c|>{\centering\arraybackslash}p{2cm}|>{\centering\arraybackslash}p{2cm}|}
\hline 
 & A & B  \vplus  \\ 
\hline 
abscisse &  &  \vplus \\ 
\hline 
ordonnée &  &   \vplus  \\ 
\hline 
\end{tabular} 

\item Confirme la réponse dans la question 1.

\Point{2}
\end{enumerate}

\end{enumerate}
