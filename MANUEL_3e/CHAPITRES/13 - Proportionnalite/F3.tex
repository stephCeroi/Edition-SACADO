\begin{titre}[Les nombres rationnels]

{\LARGE {\color{bleu3}L'addition, la soustraction}}
\end{titre}



\begin{CpsCol}
\textbf{Utiliser des nombres pour calculer et résoudre des problèmes}
\begin{description}
\item[$\square$] Comparer des fractions
\item[$\square$] Repérer et placer des nombres rationnels sur une droite graduée
\end{description}
\end{CpsCol}



\section{Cours (suite): Règles de calcul avec les fractions}

\begin{Reg}[Addition/soustraction]
Pour additionner ou soustraire deux nombres écrits en écriture fractionnaire, on doit les écrire avec le même dénominateur, puis on additionne ou on soustrait leur numérateur.
\end{Reg}

   
\begin{Ex}   
$\frac{3}{4}+\frac{-1}{2} = \frac{3}{4}+\frac{-2}{4} = \frac{1}{4}$ 
\end{Ex}

\begin{Rq}   
\begin{itemize}
\item Une fois avoir additionné ou soustrait les numérateurs, il faut essayer de simplifier l'écriture obtenue, lorsque c'est possible.
\item La recherche d'un dénominateur commun n'est parfois pas chose simple (voir méthodes)
\item Il faut souvent essayer de rechercher le dénominateur commun le plus petit, ou le plus simple, pour faciliter les calculs.
\end{itemize}
\end{Rq}







\begin{autoeval}
\begin{tabular}{p{12cm}p{0.5cm}p{0.5cm}p{0.5cm}p{1cm}}
\textbf{Compétences visées} &  M I & MF & MF  & TBM \vcomp \\ 
Pratiquer le calcul exact, approché, mental, à la main ou instrumenté & $\square$ & $\square$  & $\square$ & $\square$ \vcomp \\ 
Repérer et placer des nombres sur une droite graduée & $\square$ & $\square$ & $\square$ & $\square$ \vcomp \\ 
\end{tabular}
{\footnotesize MI : maitrise insuffisante ; MF = Maitrise fragile ; MS = Maitrise satisfaisante ; TBM = Très bonne maitrise}
 
\end{autoeval}