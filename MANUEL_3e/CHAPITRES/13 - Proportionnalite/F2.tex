\begin{titre}[Les nombres rationnels]

{\LARGE {\color{bleu3}Comparaison de deux écritures fractions}}
\end{titre}



\begin{CpsCol}
\textbf{Utiliser des nombres pour calculer et résoudre des problèmes}
\begin{description}
\item[$\square$] Comparer des fractions
\item[$\square$] Repérer et placer des nombres rationnels sur une droite graduée
\end{description}
\end{CpsCol}



\subsection{Comparer deux nombres écrits en écriture fractionnaire}

Pour comparer deux nombres écrits en écriture fractionnaire, on peut:
\begin{itemize} 
\item Les écrire avec le même dénominateur et comparer leur numérateur: les nombres sont rangés dans le même ordre que leur numérateur.
\item Les écrire avec le même numérateur et comparer leur dénominateur: les nombres sont rangés dans le sens contraire que leur dénominateur
\item Les comparer à $1$ ou à l'entier le plus proche
\item Utiliser la calculatrice et l'écriture décimale ou l'écriture décimale \textbf{approchée}
\end{itemize}


\begin{Ex}   
\begin{itemize}
\item $\frac{33}{12}<\frac{34}{12}$
\item $\frac{33}{12}<\frac{33}{11}$
\item $\frac{33}{34}<1$
\item $\pi<3,15$, car $\pi\approx3,1459$
\end{itemize}
\end{Ex}

\subsection{Rechercher un dénominateur commun}


\begin{enumerate}
\item Si l'un des deux dénominateurs est un multiple de l'autre, on choisit le plus grand des deux dénominateurs.
\item Dans les autres cas, on doit recherche un multiple commun aux deux dénominateurs : on peut utiliser plusieurs méthodes
\begin{itemize} 
\item Le produit des dénominateurs est toujours un \og dénominateur commun\fg{}, mais ce n'est pas toujours le plus petit! 
\item On peut faire la liste des multiples de chaque dénominateur, en commençant par les plus petits, et dès qu'on trouve un multiple commun, on choisit ce multiple comme \og dénominateur commun\fg{}.
\item nous verrons en classe de 3e et au lycée d'autres méthodes pour rechercher plus efficacement ce \og dénominateur commun\fg{} pour de grands nombres (hors-programme)
\end{itemize}
\end{enumerate}


\begin{Ex}   
\begin{itemize}
\item Un dénominateur commun de $\frac{33}{12}$ et $\frac{7}{10}$ est $120$, mais en faisant la liste, on peut s'apercevoir que $60$ aussi et il est plus petit.
\item Un dénominateur commun de $\frac{33}{12}$ et de $\frac{5}{4}$ est $12$
\end{itemize}
\end{Ex}




\begin{autoeval}
\begin{tabular}{p{12cm}p{0.5cm}p{0.5cm}p{0.5cm}p{1cm}}
\textbf{Compétences visées} &  M I & MF & MF  & TBM \vcomp \\ 
Comparer des écritures fractionnaires & $\square$ & $\square$  & $\square$ & $\square$ \vcomp \\ 
Repérer et placer des nombres sur une droite graduée & $\square$ & $\square$ & $\square$ & $\square$ \vcomp \\ 
\end{tabular}
{\footnotesize MI : maitrise insuffisante ; MF = Maitrise fragile ; MS = Maitrise satisfaisante ; TBM = Très bonne maitrise}
 
\end{autoeval}