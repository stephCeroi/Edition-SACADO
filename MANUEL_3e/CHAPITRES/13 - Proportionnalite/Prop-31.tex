

\textbf{Disponibilité alimentaire des produits animaux et produits végétaux dans l'apport de calories. En kcal par habitant et par jour.}

\begin{enumerate}
\item Le tableau ci dessous est incomplet. Complète le.

\begin{tabular}{|c|c|c|c|c|c|c|c|c|}
\hline 
 & France & Allemagne & UK & Espagne & USA & Brésil & Inde & Chine \vplus  \\ 
\hline 
Produits végétaux & 2344 & 2446 & 2425 & 2351 & 2644 & 2484 & 2231 & 2383 \vplus  \\ 
\hline 
Produits animaux & 1180 & 1093 & 989 & 831 & 995 & 803 & 228 & 691 \vplus  \\ 
\hline 
Total & 3524 & 3539 & 3414 & 3183 & 3639 & 3287 & 2459 & 3074  \vplus  \\ 
\hline 
Parts produits animaux & 33\% & 31\% & 29\% &  26\% &  &  &  &  \vplus \\ 
\hline 
\end{tabular} 
\begin{flushright}
{\scriptsize Source : http://www.viande.info/comparaison-internationale}
\end{flushright}

\item Quel est sa signification ?

\item Dans quel pays la disponibilité alimentaire des produits animaux est la plus importante ? Justifie ta réponse.
\item En comparant la colonne UK et USA, quelle constatation peux tu faire ?
\end{enumerate}

