\begin{titre}[La proportionnalité]

\Titre{Les proportions}{1}
\end{titre}


\begin{CpsCol}
\textbf{Utiliser des nombres pour calculer et résoudre des problèmes}
\begin{description}
\item[$\square$] Calculer une proportion
\item[$\square$] Interpréter une proportion
\item[$\square$] Reconnaitre un tableau de proportionnalité
\item[$\square$] Calculer un coefficient de proportionnalité
\item[$\square$] Rechercher une valeur proportionnelle
\end{description}
\end{CpsCol}



\begin{DefT}{Proportionnalité \index{Proportionnalité}}
Deux \textbf{grandeurs} A et B sont \index{Proportionnalité!Grandeurs proportionnelles} \textbf{proportionnelles} lorsque les valeurs de A sont obtenues en multipliant par le même nombre non nul les valeurs de B.

Le nombre non nul est appelé \textbf{coefficient de proportionnalité}\index{Proportionnalité!Coefficient}.

Un \textbf{tableau de proportionnalité} représente des grandeurs proportionnelles.\index{Proportionnalité!Tableau}
\end{DefT}


\begin{DefT}{Représentation graphique}
La représentation graphique d'une situation de proportionnalité est une droite qui passe par l'origine du repère.
\end{DefT}