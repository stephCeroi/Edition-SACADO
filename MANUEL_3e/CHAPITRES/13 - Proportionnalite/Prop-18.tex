
\begin{minipage}{0.48\linewidth}
Voici la fiche technique d'un AirBus A380. 

\begin{center}
\begin{tabular}{|l|}
\hline 
\textbf{Fiche technique de l'A380} \\
Envergure : 79,75 mètres \\
Longueur : 72,72 mètres \\
Hauteur : 24,1 mètres \\
Capacité en carburant : \np{325 000} litres \\
Vitesse de croisière : Mach 0,85 \\
Altitude de croisière : \np{10700} mètres \\
\hline 
\end{tabular} 
\end{center}
\end{minipage}
\hfill
\begin{minipage}{0.48\linewidth}
\begin{enumerate}
\item Quelle est selon toi une bonne échelle pour représenter un AirBus A380 sur ton cahier ? Justifie.
\item Représente la carlingue, la queue et les ailes vue de dessus.

\item \textbf{vitesse.}
Le nombre de Mach est un nombre sans dimension, noté Ma, qui exprime le rapport de la vitesse locale d'un fluide à la vitesse du son dans ce même fluide. Aux températures habituelles et dans l'air, la vitesse du son vaut environ 1 224 $km.h^{-1}$.
A quelle vitesse, exprimée en $km.h^{-1}$, vole l'A380 ?
\item \textbf{Altitude.}
1 mètre est environ égal à 3 pieds. Quelle est l'altitude de croisière en pied de l'A380 ?
\end{enumerate}
\end{minipage}