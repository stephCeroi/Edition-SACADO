\begin{titre}[La proportionnalité]

\Titre{Échelle et vitesse}{1,5}
\end{titre}

\begin{CpsCol}
\textbf{Calculer avec des grandeurs mesurables ; exprimer les résultats dans les unités adaptées}
\begin{description}
\item[$\square$] Mener des calculs impliquant des grandeurs mesurables
\item[$\square$] Commenter des résultats authentiques
\item[$\square$] Calculer une échelle
\item[$\square$] Utiliser une échelle
\end{description}
\end{CpsCol}




\begin{DefT}{Échelle}
Il n'est pas possible de représenter le monde réel sur une feuille, sur un écran, un GPS $\cdots$. Lorsqu'on souhaite le dessiner, on utilise une \textbf{échelle}\index{Échelle} pour respecter les distances réelles.
\end{DefT}

\begin{ThT}{Échelle}
\begin{description}
\item L'échelle d'une représentation est le coefficient égal à $\frac{\text{Distance sur la carte}}{\text{Distance réelle}}$. 
\item Google Maps utilise une longueur en guise d'échelle.
\end{description}
\end{ThT}



\begin{DefT}{Vitesse moyenne}
La vitesse moyenne \index{Vitesse moyenne} entre deux point A et B est le quotient de la distance entre A et B et le temps mis pour la parcourir. $$ V_{moyenne}= \frac{d}{t} = \frac{\text{Distance parcourue}AB}{\text{temps mis pour parcourir} AB} $$
\end{DefT}

\begin{Mt}
Pour Calculer une vitesse :
\begin{enumerate}
\item On lit attentivement les unités : $km.h^{-1}$ ou $m.s^{-1}$ ou $km.s^{-1}$ ....
\item On convertit les unités (éventuellement).
\item On utilise la formule $V_{moyenne}= \frac{d}{t}$ en remplaçant $d$ par la distance convertie et $t$ par le temps converti (étape 3)
\item On simplifie la fraction si possible.
\end{enumerate}
\end{Mt}


\begin{Ex}
Pour aller de Toulon à Marseille, deux villes distantes de 70 km, j'ai roulé pendant 45 minutes. Je souhaite connaitre ma vitesse moyenne en $km.h^{-1}$.

45 minutes  = $\frac{3}{4}$ heure. 

Donc la distance est 70 km et le temps pour parcourir cette distance est 0,75 heure.

$ V_{moyenne}= \frac{70}{0,75} \approx 93,3 km.h^{-1}$
\end{Ex}

