\documentclass[openany]{book}

\input{../../../latex_preambule_style/preambule}
\input{../../../latex_preambule_style/styleCoursCycle4}
\input{../../../latex_preambule_style/styleExercices}
\input{../../../latex_preambule_style/styleExercicesAideCompetences}
%\input{../../latex_preambule_style/styleCahier}
\input{../../../latex_preambule_style/bas_de_page_cycle4}
\input{../../../latex_preambule_style/algobox}


%%%%%%%%%%%%%%%  Affichage ou impression  %%%%%%%%%%%%%%%%%%
\newcommand{\impress}[2]{
\ifthenelse{\equal{#1}{1}}  %   1 imprime / affiche sur livre  -----    0 affiche sur cahier 
{%condition vraieé
#2
}% fin condition vraie
{%condition fausse
}% fin condition fausse
} % fin de la procédure
%%%%%%%%%%%%%%%  Affichage ou impression  %%%%%%%%%%%%%%%%%%
 \usepackage{geometry}
 \geometry{top=1.5cm, bottom=0cm, left=2cm , right=2cm}
%%%%%%%%%%%%%%%%%%%%%%%%%%%%%%%%%%%%%%%%%%%%%%%%

\begin{document}


\begin{seance}[Statistiques]

\begin{description}
\item[$\square$] Calculer et interpréter des caractéristiques de position ou de dispersion d’une série statistique - Indicateurs : moyenne.
\item[$\square$] Calculer des effectifs, des fréquences.
\item[$\square$] Lire des données sous forme de données brutes, de tableau, de graphique.
\end{description}
\end{seance}


\Exe


Pierre, Jean et Alain ont passé un examen comportant quatre
disciplines. Pour être reçu, il faut atteindre 10 de moyenne.

\begin{center}
  \begin{tabular}{|c|c|c|c|c|}
\cline{2-5}
\multicolumn{1}{c|}{}&Français&Mathématiques&Anglais&Technologie\\
\hline
Pierre&15&9&11&7\\
\hline
Jean&10&11&12&9\\
\hline
Alain&7&14&13&8\\
\hline
  \end{tabular}
\end{center}

\begin{enumerate}
\item En utilisant un tableur, calculer la moyenne, sans coefficient, des trois candidats et la moyenne par matière. On doit construire le tableau suivant :

\begin{center}
  \begin{tabular}{|c|c|c|c|c|c|c|}
\hline  
{\cellcolor{gray}}&{\cellcolor{gray}}&{\cellcolor{gray}A}&{\cellcolor{gray}B}&{\cellcolor{gray}C}&{\cellcolor{gray}D}&{\cellcolor{gray}E} \\
\hline
{\cellcolor{gray}1}&Prénom&Français&Mathématiques&Anglais&Technologie&Moyenne\\
\hline
{\cellcolor{gray}2}&Pierre&15&9&11&7&\\
\hline
{\cellcolor{gray}3}&Jean&10&11&12&9&\\
\hline
{\cellcolor{gray}4}&Alain&7&14&13&8&\\
\hline
{\cellcolor{gray}5}&Moyenne&&&&&\\
\hline
  \end{tabular}
\end{center}

\begin{Rappel}
Pour utiliser une fonction sur un tableur, on utiliser le signe =. \par
La fonction =SOMME(A1:A5) permet le calcul des valeurs dans les cellules de A1 à A5.
\end{Rappel}


\item Pour cet examen, le français, les mathématiques, l'anglais et la
technologie ont respectivement pour coefficient 6; 4; 2 et 5.\par
A l'aide de Scratch, créer un programme qui permet de calculer la moyenne pondérée de ces 4 matières et qui dit si un candidat est reçu ou non.

\vspace{0.4cm}
On  créera donc 2 listes une appelé notes et une appelée coefficient.

\begin{Rappel}
Pour créer une liste, on utilise la séquence \fcolorbox{orange}{orange}{Données} puis \fcolorbox{gray}{gray}{Créer une liste}
\end{Rappel}



\end{enumerate}





\end{document}