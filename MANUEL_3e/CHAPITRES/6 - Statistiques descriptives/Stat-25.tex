
\begin{minipage}[top]{10cm}
L'entreprise est à la recherche de qualifications de plus en plus élevées pour faire face au développement de technologies en constante évolution et pour une bonne compréhension des consignes de travail. Lors de sa scolarité, un jeune doit développer de l'intérêt et de la curiosité, si utiles pour réussir ensuite sa vie professionnelle. Face au nombre, en baisse mais encore inquiétant, de sorties du système scolaire sans qualification, il paraît intéressant d'étudier ce phénomène du point de vue européen à la lumière des mathématiques. En France, 13\% des jeunes de 18 à 24 ans qui ne poursuivent pas d'études ni de formation n'ont ni CAP, ni BEP, ni bac et sont sortants précoces.
\end{minipage}
\begin{minipage}{6cm}
\includegraphics[scale=0.5]{stat12.jpg} 
\end{minipage}

\begin{enumerate}
\item Détermine la médiane des sorties précoces en Europe. 

\item 
\begin{enumerate}
\item Compléter le tableau suivant.

\begin{tabular}{|c|c|c|c|c|c|c|}
\hline 
Sorties précoces en 2007 & [0;5[ & [5;10[ & [10;15[ & [15;20[ & [20;25[ & [25;30] \\ 
\hline 
Effectif de pays européens &  &  &  &  &  &  \\ 
\hline 
E.C.C. des pays européens &  &  &  &  &  &  \\ 
\hline 
\end{tabular} 

\item Construis le polygone des effectifs cumulés croissants.

\item En déduire la médiane des sorties précoces en Europe. 
 
\item Compare avec la question 1.
\end{enumerate}
\end{enumerate}



  