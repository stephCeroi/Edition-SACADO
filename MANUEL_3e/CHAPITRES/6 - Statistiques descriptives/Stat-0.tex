
Le travail de statistiques est souvent fastidieux, on a toujours recours à un tableur lorsque les données sont trop importantes. Une feuille de calcul d'un tableur se présente comme cela, en cellules. La cellule rouge est la cellule C3.

L'intérêt d'un tableur est le calcul automatique des caractéristiques demandées : moyenne, fréquence, effectif cumulés,... Pour cela, il faut utiliser des formules mathématiques.

\vspace{0.4cm}

\begin{tabular}{|c|m{2cm}|m{2cm}|m{2cm}|m{2cm}|m{2cm}|}
\hline 
\rowcolor{gray} & A & B & C & D &...\\ 
\hline 
\cellcolor{gray}1 & &4 &  & &  \\ 
\hline 
\cellcolor{gray}2 & & & 5 & &   \\ 
\hline 
\cellcolor{gray}3 & & & \cellcolor{red} & &  \\ 
\hline 
\cellcolor{gray}... & & &  & & \\ 
\hline 
\end{tabular} 

\subsubsection*{Quelques formules : Utilisation du signe = avant toute formule}

\begin{description}
\item[=a+b] calcule la somme de  $a$ et de $b$. 
\item[=SOMME(plage)] calcule la somme des cellules dans une plage rectangulaire.
\item[=NB.SI(plage;critère] renvoie parmi les cellules de la plage celle qui vérifie le critères.
\item[=a/b] calcule le quotient de  $a$ par $b$. $b$ doit être non nul.
\end{description}

\subsubsection*{Utilisations}
\begin{description}[leftmargin=*]
\item Pour calculer 4+5 dans la cellule D3, on tapera dans la cellule D3, \textbf{=B1+C2}. Lorsque une des valeurs des cellules B1 ou C2 change, la somme change alors. 
\item Pour calculer 4/5 dans la cellule B3, on tapera dans la cellule DB3, \textbf{=B1/C2}. 
\end{description}

\subsection*{Application concrète}

Voici les notes obtenues à un contrôle sur 10 par une classe de quatrième :\\
0 -- 1 -- 2 -- 2 -- 3 -- 3 -- 3 -- 3 -- 4 -- 4 -- 5 -- 5 -- 5 -- 6 -- 6 -- 6 -- 7 -- 7 -- 8 -- 8 -- 8 -- 9 -- 9 -- 10 -- 10.

\begin{enumerate}
\item Créer un tableau sur une feuille de calcul et complète le tableau ci-dessous avec les formules adéquates.
\item Combien d'élèves ont obtenu moins de 5 ?
\item Quel est le pourcentage d'élèves qui ont obtenu 8 ?
\item Quel est le pourcentage d'élèves qui ont obtenu au moins 8 ?
\item Change un 3 et par un 8 dans la série statistique. Remarque le changement de résultats.
\end{enumerate}

\begin{tabular}{|p{4.2cm}*{11}{|p{5mm}}|p{8mm}|}
\hline
Note&0 & 1 & 2& 3 & 4 & 5 & 6&7 &8 &9 &10&Total \\
\hline
Effectifs&1 & 1 & 2& &&&& & &&& \\
\hline
Effectifs cumulés&1 & 2 & 4& &  & & & && & & \\
\hline
Fréquences &0,04 & 0,04 & 0,08 & &  & & & & & & & \\
\hline
Fréquences cumulées &0,04  & 0,08  & 0,16  &  &  & & & & & & & \\
\hline
\end{tabular}




\subsubsection*{Effectifs -- Effectifs cumulés}
	Pour chaque note, {\em l'effectif} est le nombre d'élèves ayant eu cette note.	Par exemple (dans le tableau), 1 élève a eu 0 ;	2 élèves ont eu 1\dots\\ 
	Pour chaque note, {\em l'effectif cumulé} est le nombre d'élèves ayant eu cette note ou une note inférieure. Pour le calculer, il suffit à chaque fois de cumuler les effectifs.\\
	Par exemple (dans le tableau), 1 élève a eu 0 ;	$1+2=3$ élèves ont eu 1 au plus ; $3+4=7$ élèves ont eu 2 au plus\dots
	
\subsubsection*{Fréquences -- Fréquences cumulées}
	Pour chaque note, {\em la fréquence} exprime la proportion d'élèves. Par exemple, sur les 20 élèves, 4 élèves ont eu une note de 2.
	La proportion est de 4 sur 20 ou $4\over20$ que l'on exprime en \%\ par le calcul $100\times{4\over20}=20$.\\
	Comme pour les effectifs cumulés, les {\em fréquences cumulées}	sont obtenues en cumulant les fréquences.