\begin{enumerate}
\item On donne les nombres suivants : $$4~~ - ~~ 16 ~~  - ~~ 8 ~~ - ~~ 13 ~~ - ~~ 15 $$
La moyenne est $$a. 11~~ ~~ ~~ b. 11,5 ~~  ~~ ~~ c. 11,2 ~~ ~~ ~~ d. 12$$

\item Pour calculer la moyenne des cellules A5 à B5, avec un tableur, on tape la formule
$$a. MOYENNE(A5:B5)~~ ~~ ~~ b. =MOYENNE(A5:B5) ~~  ~~ ~~ c. MOYENNE(A5;B5) ~~ ~~ ~~ d. =MOYENNE(A5;B5)$$

\item Voici le diagramme à bâtons des notes d'un devoir. 

\definecolor{qqqqff}{rgb}{0.,0.,1.}
\definecolor{cqcqcq}{rgb}{0.7529411764705882,0.7529411764705882,0.7529411764705882}
\begin{tikzpicture}[line cap=round,line join=round,>=triangle 45,x=1.0cm,y=1.0cm]
\draw [color=cqcqcq,, xstep=1.0cm,ystep=1.0cm] (-0.72,-1.1) grid (10.62,7.46);
\draw[->,color=black] (-0.72,0.) -- (10.62,0.);
\foreach \x in {,1.,2.,3.,4.,5.,6.,7.,8.,9.,10.}
\draw[shift={(\x,0)},color=black] (0pt,2pt) -- (0pt,-2pt) node[below] {\footnotesize $\x$};
\draw[->,color=black] (0.,-1.1) -- (0.,7.46);
\foreach \y in {-1.,1.,2.,3.,4.,5.,6.,7.}
\draw[shift={(0,\y)},color=black] (2pt,0pt) -- (-2pt,0pt) node[left] {\footnotesize $\y$};
\draw[color=black] (0pt,-10pt) node[right] {\footnotesize $0$};
\clip(-0.72,-1.1) rectangle (10.62,7.46);
\draw [color=qqqqff] (1.,0.)-- (1.,1.);
\draw [color=qqqqff] (3.,0.)-- (3.,2.);
\draw [color=qqqqff] (4.,0.)-- (4.,3.);
\draw [color=qqqqff] (5.,4.)-- (5.02,-0.14);
\draw [color=qqqqff] (6.,0.)-- (6.,6.);
\draw [color=qqqqff] (7.,6.)-- (7.,0.);
\draw [color=qqqqff] (8.,4.)-- (8.,0.);
\draw [color=qqqqff] (10.,1.)-- (10.,0.);
\draw (9.44,-0.5) node[anchor=north west] {notes};
\draw (0.16,6.42) node[anchor=north west] {Effectifs};
\end{tikzpicture}

La moyenne est $$a. \approx 5,9 ~~ ~~ ~~ b. \approx 6,5 ~~  ~~ ~~ c. \approx 1,7 ~~ ~~ ~~ d. \approx 5,5$$

\item Pierre n'a retrouvé que 7 contrôles où il a obtenu chaque fois 8 sur 10. Il a perdu une copie mais il sait que sa moyenne est de 7,75. Quelle est la note de la copie perdue ?

\item Grégory espère obtenir au Bac 17/20 en Maths, 14/20 en Lettres et 17/20 en Langues. Le coefficient de l'épreuve des maths est 7, celui de l'épreuve des lettres est 3 et celui de l'épreuve des langues est 2. Peut il espérer obtenir la mention "Très bien" ?

\item Le tableau donne le nombre de voitures qui passent dans un virage dangereux où la vitesse est limitée à 60 km/h. 

\begin{tabular}{|c|c|c|c|c|}
\hline 
Vitesse & [30;40[ & [40;50[ & [50;60[ & [60;70[ \\ 
\hline 
Effectif & 15 & 60 & 80 & 25 \\ 
\hline 
\end{tabular} 

Peut on dire les conducteurs respectent la limitation de vitesse ?

\end{enumerate}