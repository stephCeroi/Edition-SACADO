\documentclass[10pt]{article}

 

\input{../../../latex_preambule_style/preambule}
\input{../../../latex_preambule_style/styleCoursCycle4}
\input{../../../latex_preambule_style/styleExercices}
 
 
\input{../../../latex_preambule_style/algobox}

 
%%%%%%%%%%%   Marges de pages  %%%%%%%%%%%%%%%% 
 \usepackage{geometry}
 \geometry{top=2cm, bottom=0cm, left=2cm , right=2cm}
%%%%%%%%%%%%%%%%%%%%%%%%%%%%%%%%%%%%%%%%%%%%%%%

%%%%%%%%%%%%%%%  Indentation  %%%%%%%%%%%%%%%%%%
\parindent=0pt
%%%%%%%%%%%%%%%%%%%%%%%%%%%%%%%%%%%%%%%%%%%%%%%%


\begin{document}

Sujet A\\ 
Nom : ................................................\\


\vspace{0.5cm}

Dans la figure ci-dessous, les droites $(BE)$ et $(CD)$ sont parallèles.

\begin{tikzpicture}[line cap=round,line join=round,>=triangle 45,x=1.0cm,y=1.0cm]
\clip(-0.52,0.48) rectangle (7.64,4.56);
\draw [line width=2.pt] (0.34,1.12)-- (7.08,1.06);
\draw (2.5,4.12) node[anchor=north west] {$4cm$};
\draw (2.44,1.) node[anchor=north west] {$8cm$};
\draw (5.62,2.16) node[anchor=north west] {5cm};
\draw [line width=2.pt] (1.,4.)-- (0.34,1.12);
\draw [line width=2.pt] (1.,4.)-- (7.08,1.06);
\draw [line width=2.pt] (0.69376,2.66368)-- (3.9771034482758623,2.5604137931034483);
\begin{scriptsize}
\draw [color=black] (1.,4.)-- ++(-1.5pt,0 pt) -- ++(3.0pt,0 pt) ++(-1.5pt,-1.5pt) -- ++(0 pt,3.0pt);
\draw[color=black] (1.14,4.29) node {$A$};
\draw [color=black] (0.34,1.12)-- ++(-1.5pt,0 pt) -- ++(3.0pt,0 pt) ++(-1.5pt,-1.5pt) -- ++(0 pt,3.0pt);
\draw[color=black] (-0.22,0.87) node {$C$};
\draw [color=black] (7.08,1.06)-- ++(-1.5pt,0 pt) -- ++(3.0pt,0 pt) ++(-1.5pt,-1.5pt) -- ++(0 pt,3.0pt);
\draw[color=black] (7.22,1.35) node {$D$};
\draw [color=black] (0.69376,2.66368)-- ++(-1.5pt,0 pt) -- ++(3.0pt,0 pt) ++(-1.5pt,-1.5pt) -- ++(0 pt,3.0pt);
\draw[color=black] (0.28,2.61) node {$B$};
\draw [color=black] (3.9771034482758623,2.5604137931034483)-- ++(-1.5pt,0 pt) -- ++(3.0pt,0 pt) ++(-1.5pt,-1.5pt) -- ++(0 pt,3.0pt);
\draw[color=black] (4.12,2.85) node {$E$};
\end{scriptsize}
\end{tikzpicture}

Calcule la valeur exacte de la longueur $BE$ en justifiant la réponse.

\ligne{8}

\vspace{0.5cm}
Sujet B\\ 
Nom : ................................................\\


\vspace{0.5cm}

Dans la figure ci-dessous, les droites $(UV)$ et $(ST)$ sont parallèles.

\begin{tikzpicture}[line cap=round,line join=round,>=triangle 45,x=1.0cm,y=1.0cm]
\clip(-0.52,0.48) rectangle (7.64,4.56);
\draw [line width=2.pt] (0.34,1.12)-- (7.08,1.06);
\draw (-0.3,3.52) node[anchor=north west] {$5cm$};
\draw (2.44,1.) node[anchor=north west] {$22cm$};
\draw (-0.4,2.16) node[anchor=north west] {$6cm$};
\draw [line width=2.pt] (1.,4.)-- (0.34,1.12);
\draw [line width=2.pt] (1.,4.)-- (7.08,1.06);
\draw [line width=2.pt] (0.69376,2.66368)-- (3.9771034482758623,2.5604137931034483);
\begin{scriptsize}
\draw [color=black] (1.,4.)-- ++(-1.5pt,0 pt) -- ++(3.0pt,0 pt) ++(-1.5pt,-1.5pt) -- ++(0 pt,3.0pt);
\draw[color=black] (1.14,4.29) node {$R$};
\draw [color=black] (0.34,1.12)-- ++(-1.5pt,0 pt) -- ++(3.0pt,0 pt) ++(-1.5pt,-1.5pt) -- ++(0 pt,3.0pt);
\draw[color=black] (0.,0.97) node {$S$};
\draw [color=black] (7.08,1.06)-- ++(-1.5pt,0 pt) -- ++(3.0pt,0 pt) ++(-1.5pt,-1.5pt) -- ++(0 pt,3.0pt);
\draw[color=black] (7.22,1.35) node {$T$};
\draw [color=black] (0.69376,2.66368)-- ++(-1.5pt,0 pt) -- ++(3.0pt,0 pt) ++(-1.5pt,-1.5pt) -- ++(0 pt,3.0pt);
\draw[color=black] (0.28,2.61) node {$U$};
\draw [color=black] (3.9771034482758623,2.5604137931034483)-- ++(-1.5pt,0 pt) -- ++(3.0pt,0 pt) ++(-1.5pt,-1.5pt) -- ++(0 pt,3.0pt);
\draw[color=black] (4.12,2.85) node {$V$};
\end{scriptsize}
\end{tikzpicture}

Calcule la valeur exacte de la longueur $ST$ en justifiant la réponse.

\ligne{8}



 
\end{document}