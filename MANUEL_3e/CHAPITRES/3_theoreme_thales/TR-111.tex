
Soient $ A'$ et $ B'$ les images de A et de B
dans la symétrie de centre O.
D'après la définition de la symétrie, $ OA= OA'$
 et $ OB= OB'$. La symétrie centrale conserve les
longueurs donc $ AB= A'B'$.
Dans une symétrie centrale, une droite et son image sont parallèles donc 
$( AB)//( A'B')$.
Or $( AB)//( CD)$ donc $( A'B')//(CD)$.\\
Nous nous trouvons donc dans une configuration de Thalès classique et
${OA'\over OD}={OB'\over OC}={A'B'\over CD}$ d'où
${OA\over OD}={OB\over OC}={AB\over CD}$.