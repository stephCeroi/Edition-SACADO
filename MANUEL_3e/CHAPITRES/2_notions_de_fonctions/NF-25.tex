
Le poids d'un corps sur un astre dépend de la masse et de l'accélération de la pesanteur.
 
On peut montrer que la relation est $P = mg$,
 
$P$ est le poids (en Newton) d'un corps sur un astre (c'est-à-dire la force que l'astre exerce sur le corps),
 
$m$ la masse (en kg) de ce corps,
 
$g$ l'accélération de la pesanteur de cet astre.

\medskip
 
\begin{enumerate}
\item Sur la terre, l'accélération de la pesanteur de la Terre $g_{T}$ est environ de $9,8$. Calculer le poids (en Newton) sur Terre d'un homme ayant une masse de $70$~kg. 
\item Sur la lune, la relation $P = mg$ est toujours valable.
 
On donne le tableau ci-dessous de correspondance poids-masse sur la Lune : 

\medskip

\begin{tabularx}{\linewidth}{|l|*{5}{>{\centering \arraybackslash}X|}}\hline
Masse (kg)	&3	&10	&25		&40	&55 \\ \hline
Poids (N)	&5,1&17 &42,5	&68	&93,5\\ \hline
\end{tabularx}

\medskip
 
	\begin{enumerate}
		\item Est-ce que le tableau ci-dessus est un tableau de proportionnalité ? 
		\item Calculer l'accélération de la pesanteur sur la lune noté $g_{L}$ 
		\item Est-il vrai que l'on pèse environ 6 fois moins lourd sur la lune que sur la Terre ?
	\end{enumerate} 

\end{enumerate} 