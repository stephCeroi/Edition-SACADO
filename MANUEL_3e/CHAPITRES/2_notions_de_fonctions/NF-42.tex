
\begin{enumerate}
\item Denis souhaite connaître sa fréquence cardiaque maximale conseillée (FCMC) afin de ne pas la dépasser et ainsi de ménager son cœur. La FCMC d'un individu dépend de son âge $a$, exprimé en années, elle peut s'obtenir grâce à la formule suivante établie par Astrand et Ryhming :
	
\begin{center}
\begin{tabularx}{0.8\linewidth}{|X|}\hline	
Fréquence cardiaque maximale conseillée = $220 - $âge.\\ \hline
\end{tabularx}
\end{center}

On note $f(a)$ la FCMC en fonction de l'âge $a$, on a donc $f(a) = 220 - a$.
	\begin{enumerate}
		\item Vérifier que la FCMC de Denis est égale à $188$ pulsations/minute.
		\item Comparer la FCMC de Denis avec la FCMC d'une personne de $15$ ans.
 	\end{enumerate}
\item  Après quelques recherches, Denis trouve une autre formule permettant d'obtenir sa FCMC de façon plus précise. Si $a$ désigne l'âge d'un individu, sa FCMC peut être calculée à l'aide de la formule de Gellish :
	
\begin{center}
\begin{tabularx}{0.9\linewidth}{|X|}\hline	
Fréquence cardiaque maximale conseillée = $191,5 - 0,007 \times \text{âge}^2$\rule[-3mm]{0mm}{8mm}\\ \hline
\end{tabularx}
\end{center}

On note $g(a)$ la FCMC en fonction de l'âge $a$, on a donc 

$g(a) = 191,5 - 0,007 \times a^2$.

Denis utilise un tableur pour comparer les résultats obtenus à l'aide des deux formules :

\begin{center}
\begin{tabularx}{\linewidth}{|c|c|*{2}{>{\centering \arraybackslash}X|}}\hline
\multicolumn{1}{|l}{B2}&\multicolumn{1}{l|}{~}&\multicolumn{1}{l|}{=220-A2}&\multicolumn{1}{l|}{~}\\ \hline
	&A 			&B 									&C\\ \hline
1	&Âge $a$	&FCMC $f(a)$ (Astrand et Ryhming)	&FCMC $g(a)$ (Gellish)\\ \hline
2	&30 		&190 								&185,2\\ \hline
3	&31 		&189 								&184,773\\ \hline
4	&32 		&188 								&184,332\\ \hline
5	&33 		&187 								&183,877\\ \hline
\end{tabularx}
\end{center}

Quelle formule faut-il insérer dans la cellule C2 puis recopier vers le bas, pour pouvoir compléter la colonne \og FCMC $g(a)$ (Gellish) \fg ?

\end{enumerate}