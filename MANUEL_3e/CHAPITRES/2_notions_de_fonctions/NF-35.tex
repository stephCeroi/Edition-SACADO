
Lors d'une étape cycliste, les distances parcourues par un cycliste ont été relevées
chaque heure après le départ.

Ces données sont précisées dans le graphique ci-dessous :

\begin{center}
\definecolor{qqqqff}{rgb}{0.,0.,1.}
\definecolor{uuuuuu}{rgb}{0.26666666666666666,0.26666666666666666,0.26666666666666666}
\definecolor{cqcqcq}{rgb}{0.7529411764705882,0.7529411764705882,0.7529411764705882}
\begin{tikzpicture}[line cap=round,line join=round,>=triangle 45,x=1.9737248840803714cm,y=0.04822088724583995cm]
\draw [color=cqcqcq,, xstep=0.9868624420401857cm,ystep=0.4822088724583995cm] (-0.30158730158730146,-14.472361809045118) grid (4.833333333333331,203.01507537688428);
\draw[->,color=black] (-0.30158730158730146,0.) -- (4.833333333333331,0.);
\foreach \x in {,0.5,1.,1.5,2.,2.5,3.,3.5,4.,4.5}
\draw[shift={(\x,0)},color=black] (0pt,2pt) -- (0pt,-2pt) node[below] {\footnotesize $\x$};
\draw[->,color=black] (0.,-14.472361809045118) -- (0.,203.01507537688428);
\foreach \y in {,20.,40.,60.,80.,100.,120.,140.,160.,180.,200.}
\draw[shift={(0,\y)},color=black] (2pt,0pt) -- (-2pt,0pt) node[left] {\footnotesize $\y$};
\draw[color=black] (0pt,-10pt) node[right] {\footnotesize $0$};
\clip(-0.30158730158730146,-14.472361809045118) rectangle (4.833333333333331,203.01507537688428);
\draw [color=qqqqff] (0.,0.)-- (1.,40.);
\draw [color=qqqqff] (1.,40.)-- (2.0113636363636362,70.47945205479446);
\draw [color=qqqqff] (2.0113636363636362,70.47945205479446)-- (3.,130.);
\draw [color=qqqqff] (3.,130.)-- (4.,170.);
\draw [color=qqqqff] (4.,170.)-- (4.507936507936506,191.35678391959786);
\draw (1.928571428571427,9.246231155778975) node[anchor=north west] {Durée de parcours (en heure)};
\draw (0.023809523809523798,197.38693467336668) node[anchor=north west] {Distance parcourue (en km)};
\begin{scriptsize}
\draw [color=uuuuuu] (0.,0.)-- ++(-1.5pt,-1.5pt) -- ++(3.0pt,3.0pt) ++(-3.0pt,0) -- ++(3.0pt,-3.0pt);
\draw [color=qqqqff] (1.,40.)-- ++(-1.5pt,-1.5pt) -- ++(3.0pt,3.0pt) ++(-3.0pt,0) -- ++(3.0pt,-3.0pt);
\draw [color=qqqqff] (2.0113636363636362,70.47945205479446)-- ++(-1.5pt,-1.5pt) -- ++(3.0pt,3.0pt) ++(-3.0pt,0) -- ++(3.0pt,-3.0pt);
\draw [color=qqqqff] (3.,130.)-- ++(-1.5pt,-1.5pt) -- ++(3.0pt,3.0pt) ++(-3.0pt,0) -- ++(3.0pt,-3.0pt);
\draw [color=qqqqff] (4.,170.)-- ++(-1.5pt,-1.5pt) -- ++(3.0pt,3.0pt) ++(-3.0pt,0) -- ++(3.0pt,-3.0pt);
\draw [color=qqqqff] (4.507936507936506,191.35678391959786)-- ++(-1.5pt,-1.5pt) -- ++(3.0pt,3.0pt) ++(-3.0pt,0) -- ++(3.0pt,-3.0pt);
\end{scriptsize}
\end{tikzpicture}

\end{center}

Par lecture graphique, répondre aux questions suivantes.

\emph{Aucune justification n'est demandée.}

\medskip

\begin{enumerate}
\item 
	\begin{enumerate}
		\item Quelle est la distance totale de cette étape ?
		\item En combien de temps le cycliste a-t-il parcouru les cent premiers kilomètres ?
		\item Quelle est la distance parcourue lors de la dernière demi-heure de course ?
	\end{enumerate}
\item  Y-a-t-il proportionnalité entre la distance parcourue et la durée de parcours de cette
étape ?
	
Justifier votre réponse et proposer une explication.
\end{enumerate}
