\documentclass[openany]{book}



\input{../../../latex_preambule_style/preambule}
\input{../../../latex_preambule_style/styleCoursCycle4}
\input{../../../latex_preambule_style/styleExercices}
\input{../../../latex_preambule_style/styleExercicesAideCompetences}
%\input{../../latex_preambule_style/styleCahier}
\input{../../../latex_preambule_style/bas_de_page_cycle4}
\input{../../../latex_preambule_style/algobox}


%%%%%%%%%%%%%%%  Affichage ou impression  %%%%%%%%%%%%%%%%%%
\newcommand{\impress}[2]{
\ifthenelse{\equal{#1}{1}}  %   1 imprime / affiche sur livre  -----    0 affiche sur cahier 
{%condition vraieé
#2
}% fin condition vraie
{%condition fausse
}% fin condition fausse
} % fin de la procédure
%%%%%%%%%%%%%%%  Affichage ou impression  %%%%%%%%%%%%%%%%%%
%%%%%%%%%%%   Marges de pages  %%%%%%%%%%%%%%%% 
 \usepackage{geometry}
 \geometry{top=1cm, bottom=0cm, left=2cm , right=2cm}
%%%%%%%%%%%%%%%%%%%%%%%%%%%%%%%%%%%%%%%%%%%%%%%
%%%%%%%%%%%%%%%%%%%%%%%%%%%%%%%%%%%%%%%%%%%%%%%%

\begin{document}


\begin{seance}[Notions de fonction.]

\section{Préparer le DNB}


\end{seance}

\Exe

Un condensateur est un composant électronique qui permet de stocker de l'énergie électrique pour la restituer plus tard.

Le graphique suivant montre l'évolution de la tension mesurée aux bornes d'un condensateur en fonction du
temps lorsqu'il est en charge.

\begin{center}
\psset{xunit=14cm,yunit=0.8cm,comma=true}
\begin{pspicture}(-0.1,-1)(0.6,6)
\multido{\n=0.00+0.01}{61}{\psline[linewidth=0.1pt](\n,0)(\n,6)}
\multido{\n=0.0+0.1}{7}{\psline[linewidth=1pt](\n,0)(\n,6)}
\multido{\n=0.0+0.2}{31}{\psline[linewidth=0.1pt](0,\n)(0.6,\n)}
\multido{\n=0+1}{7}{\psline[linewidth=1pt](0,\n)(0.6,\n)}
\psaxes[linewidth=1.25pt,Dx=0.1]{->}(0,0)(0,0)(0.6,6)
\psaxes[linewidth=1.25pt,Dx=0.1](0,0)(0,0)(0.61,6)
\uput[d](0.55,-0.6){Temps (s)}
\rput{90}(-0.075,3){Tension (V)}
\psplot[plotpoints=3000,linewidth=1.25pt,linecolor=blue]{0}{0.6}{5 5 2.71828 x 10 mul  exp div sub}
\end{pspicture}
\end{center}


\begin{enumerate}
\item S'agit-il d'une situation de proportionnalité ? Justifier.
\item Quelle est la tension mesurée au bout de $0,2$~s ?
\item Au bout de combien de temps la tension aux bornes du condensateur aura-t-elle atteint 60\,\% de la tension maximale qui est estimée à 5 V ?
\end{enumerate}

\Exe


Le jardinier d'un club de football décide de semer à nouveau du gazon sur l'aire de
jeu. Pour que celui-ci pousse correctement, il installe un système d'arrosage
automatique qui se déclenche le matin et le soir, à chaque fois, pendant 15 minutes.

\setlength\parindent{9mm}
\begin{itemize}
\item[$\bullet~~$] Le système d'arrosage est constitué de 12 circuits indépendants.
\item[$\bullet~~$] Chaque circuit est composé de 4 arroseurs.
\item[$\bullet~~$] Chaque arroseur a un débit de 0,4 m$^3$ d'eau par heure.
\end{itemize}
\setlength\parindent{0mm}

Combien de litres d'eau auront été consommés si on arrose le gazon pendant tout le
mois de juillet ?

On rappelle que 1 m$^3 = \np{1000}$~litres et que le mois de juillet compte $31$~jours.

\Exe

Le 17 juillet 2016, une spectatrice regarde l'étape \og Bourg-en-Bresse
/ Culoz \fg{} du Tour de France.

Elle note, toutes les demi-heures, la distance parcourue par le
cycliste français Thomas Vœckler qui a mis 4~h 30~min pour
parcourir cette étape de 160~km ; elle oublie seulement de noter la
distance parcourue par celui-ci au bout de 1~h de course.

Elle obtient le tableau suivant : 

\begin{center}
\begin{tabularx}{\linewidth}{|c|*{10}{>{\centering \arraybackslash}X|}}\hline
Temps en heure &0 &0,5&1 		&1,5 	&2 	&2,5 	&3 	&3,5 &4 	&4,5\\ \hline
Distance en km &0 &15 &\ldots 	&55 	&70 &80 	&100&110 &135 	&160\\ \hline
\end{tabularx}
\end{center}

\medskip

\begin{enumerate}
\item Quelle distance a-t-il parcourue au bout de 2~h 30~min de course?
\item Montrer qu'il a parcouru 30 km lors de la troisième heure de course.
\item A-t-il été plus rapide lors de la troisième ou bien lors de la quatrième heure de
course ?
\item Répondre aux questions qui suivent sur la feuille ANNEXE , qui est
à rendre avec la copie.
	\begin{enumerate}
		\item Placer les 9 points du tableau dans le repère. On ne peut pas placer le
point d'abscisse 1 puisque l'on ne connaît pas son ordonnée.
		\item En utilisant votre règle, relier les points consécutifs entre eux.
 	\end{enumerate}
\item En considérant que la vitesse du cycliste est constante entre deux relevés,
déterminer, par lecture graphique, le temps qu'il a mis pour parcourir 75~km.
\item On considère que la vitesse du cycliste est constante entre le premier relevé
effectué au bout de 0,5~h de course et le relevé effectué au bout de 1,5~h de
course ; déterminer par lecture graphique la distance parcourue au bout de 1~h
de course.
\item Soit $f$ la fonction, qui au temps de parcours du cycliste Thomas Vœckler,
associe la distance parcourue. La fonction $f$ est-elle linéaire ?
\end{enumerate}





\psset{xunit=1.7cm,yunit=0.1cm}
\begin{pspicture}(-0.5,-10)(5.5,200)
\uput[u](5,0){Temps en $h$}
\uput[r](0,192.50){Distance en km}
\multido{\n=0.0+0.5}{12}{\psline[linewidth=0.2pt](\n,0)(\n,190)}
\multido{\n=0+5}{39}{\psline[linewidth=0.2pt](0,\n)(5.5,\n)}
\psaxes[linewidth=1.25pt,Dy=20]{->}(0,0)(0,0)(5.5,190)
\psaxes[linewidth=1.25pt,Dy=20](0,0)(0,0)(5.5,190)
\end{pspicture}


\end{document}