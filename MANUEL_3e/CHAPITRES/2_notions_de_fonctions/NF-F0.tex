\begin{titre}[Notions de fonction]

\Titre{Dépendance entre 2 grandeurs}{1}
\end{titre}



\begin{CpsCol}
\textbf{Comprendre et utiliser la notion de fonction}
\begin{description}
\item[$\square$] Traduire la dépendance entre 2 grandeurs proportionnelles
\item[$\square$] Utiliser différents mode de représentations
\end{description}
\end{CpsCol}


\Rec{1}{NF-1}

\begin{DefT}{En fonction de\index{En fonction de}}
Lorsqu'une relation associe deux quantités, on dit que l'on peut exprimer une quantité \textbf{en fonction de} l'autre. \\ En général, on peut alors établir une formule qui lie ces deux quantités.
\end{DefT}

\begin{Ex}
1 pain au chocolat coûtent 1,45 \euro{} donc $n$ pains au chocolats coûtent $1,45n$. On peut dire que le prix $p$ de $n$ pains au chocolat s'obtient par la formule $p=1,45n$. Le prix est \textbf{en fonction du} nombre de pains au chocolats.
\end{Ex}


\begin{Nt}
Pour exprimer que la quantité $\mathscr{Q}$ est en fonction de la quantité $x$, on note $\mathscr{Q}(x)$.
\end{Nt}

\AD{1}{NF-3}

 
\Exo{1}{NF-6}

\Exo{1}{NF-2}

\AD{1}{NF-5}

\AD{1}{NF-11}

\Exo{1}{NF-24}
 
\AD{1}{NF-18}

\Exo{1}{NF-19}

\Exo{1}{NF-4}



\begin{autotest}
\textbf{Exercice 1.} 
La distance verticale dans une chute libre est donnée par $h(t)=\frac{1}{2}g\times t^2$, $g$ est une constante égale à $9,81$ et le temps $t$ est exprimé en seconde. La distance est exprimée en mètres.

\begin{enumerate}
\item Quelle est la distance parcourue en $5$ secondes ? \point{2}
\item Complète le tableau c-dessous.

 \begin{tabular}{|c|c|c|c|c|}
 \hline 
 $t$ & $0$ & $\dfrac{3}{5}$ &  $5$ & $10$    \\ 
 \hline 
 $h(t)$ &  &  & $122,625$ &  \\ 
 \hline 
 \end{tabular} 
\item La distance est elle proportionnelle au temps ?\point{3}
\end{enumerate}


\textbf{Exercice 2.} 
Le son parcourt 330 m en 1 seconde. Exprime la distance en fonction du temps.
\end{autotest}
