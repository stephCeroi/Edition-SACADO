\documentclass[openany]{book}



\input{../../../latex_preambule_style/preambule}
\input{../../../latex_preambule_style/styleCoursCycle4}
\input{../../../latex_preambule_style/styleExercices}
\input{../../../latex_preambule_style/styleExercicesAideCompetences}
%\input{../../latex_preambule_style/styleCahier}
\input{../../../latex_preambule_style/bas_de_page_cycle4}
\input{../../../latex_preambule_style/algobox}


%%%%%%%%%%%%%%%  Affichage ou impression  %%%%%%%%%%%%%%%%%%
\newcommand{\impress}[2]{
\ifthenelse{\equal{#1}{1}}  %   1 imprime / affiche sur livre  -----    0 affiche sur cahier 
{%condition vraieé
#2
}% fin condition vraie
{%condition fausse
}% fin condition fausse
} % fin de la procédure
%%%%%%%%%%%%%%%  Affichage ou impression  %%%%%%%%%%%%%%%%%%

%%%%%%%%%%%%%%%%%%%%%%%%%%%%%%%%%%%%%%%%%%%%%%%%

\begin{document}

\begin{seance}[Notions de fonction.]

\section{Méthodes et notions importantes}


\end{seance}


\EPCN{Communiquer}

Écrire à l'aide d'une égalité les deux phrases suivante;
\begin{enumerate}
\item L'image de $-1$ par la fonction $f$ est 2.
\item L'antécédent de 4 par la fonction $g$ est 0.
\end{enumerate}


\EPCN{Représenter. Raisonner.}

\begin{minipage}{0.5\linewidth}
On donne l'algorithme suivant qui représente l'expression $h(x)$ de la fonction $h$.

\begin{description}
\item On choisit un nombre $x$
\item On le multiplie par 2
\item On ajoute 5 au résultat
\item On élève le résultat obtenu au carré 
\end{description}
\end{minipage}
\begin{minipage}{0.5\linewidth}
\begin{enumerate}
\item Exprimer $h(x)$ en fonction de $x$.
\item Calculer l'image de $\frac{1}{3}$ par $h$.
\item $-1$ a-t-il un antécédent par $h$ ? Préciser.
\item Chercher un antécédent de 25 par $h$.
\end{enumerate}
\end{minipage}

\EPCN{Représenter. Communiquer.}

\begin{minipage}{0.5\linewidth}
La courbe de la fonction $f$ est représentée ci-contre.

Déterminer graphiquement avec la précision permise :
\begin{enumerate}
\item $f(-1)$.
\item l'image de 1 par $f$.
\item le nombre de solutions de l'équation $f(x)=0$.
\item une solution de l'équation $f(x)=0$.
\item un nombre qui a pour antécédent 2 par $f$.
\end{enumerate}

\end{minipage}
\begin{minipage}{0.5\linewidth}

\begin{center}
\definecolor{zzttqq}{rgb}{0.6,0.2,0.}
\definecolor{cqcqcq}{rgb}{0.7529411764705882,0.7529411764705882,0.7529411764705882}
\begin{tikzpicture}[line cap=round,line join=round,>=triangle 45,x=1.0cm,y=1.0cm]
\draw [color=cqcqcq,, xstep=1.0cm,ystep=1.0cm] (-3.4290909090909114,-2.2490909090909104) grid (2.261818181818179,3.4236363636363665);
\draw[->,color=black] (-3.4290909090909114,0.) -- (2.261818181818179,0.);
\foreach \x in {-3.,-2.,-1.,1.,2.}
\draw[shift={(\x,0)},color=black] (0pt,2pt) -- (0pt,-2pt) node[below] {\footnotesize $\x$};
\draw[->,color=black] (0.,-2.2490909090909104) -- (0.,3.4236363636363665);
\foreach \y in {-2.,-1.,1.,2.,3.}
\draw[shift={(0,\y)},color=black] (2pt,0pt) -- (-2pt,0pt) node[left] {\footnotesize $\y$};
\draw[color=black] (0pt,-10pt) node[right] {\footnotesize $0$};
\clip(-3.4290909090909114,-2.2490909090909104) rectangle (2.261818181818179,3.4236363636363665);
\draw[line width=1.2pt,color=zzttqq,smooth,samples=100,domain=-3.4290909090909114:2.261818181818179] plot(\x,{cos((4.0*((\x)+0.12)+4.0)*180/pi)+0.2*((\x)+0.12)^(3.0)+0.24181818181818215});
\begin{scriptsize}
\draw[color=zzttqq] (-3.2472727272727293,-8.194545454545459) node {$f$};
\end{scriptsize}
\end{tikzpicture}
\end{center}
\end{minipage}



\DNBS{Asie 2019}

Nina et Claire ont chacune un programme de calcul.

\begin{center}
\begin{tabularx}{\linewidth}{|X|X|}\hline
\textbf{Programme de Nina}&\textbf{Programme de Claire}\\
Choisir un nombre de départ&Choisir un nombre de départ\\
Soustraire 1.&Multiplier ce nombre par $- \dfrac{1}{2}$\\
Multiplier le résultat par $-2$&Ajouter 1 au résultat\\
Ajouter 2.&\\ \hline
\end{tabularx}
\end{center}
\smallskip

\begin{enumerate}
\item Montrer que si les deux filles choisissent 1 comme nombre de départ, Nina
obtiendra un résultat final 4 fois plus grand que celui de Claire.
\item  Quel nombre de départ Nina doit-elle choisir pour obtenir $0$ à la fin ?
\item  Nina dit à Claire: \og Si on choisit le même nombre de départ, mon résultat sera
toujours quatre fois plus grand que le tien \fg.

A-t-elle raison ?
\end{enumerate}

\end{document}
