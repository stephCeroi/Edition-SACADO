
\parbox{0.22\linewidth}{On considère le programme de calcul ci-contre dans lequel x, Étape 1,
Étape 2 et 

Résultat sont quatre variables.}
\parbox{0.28\linewidth}{
\includegraphics[scale=1]{NF-42.png} 
}
\parbox{0.48\linewidth}{
\includegraphics[scale=1]{NF-43.png}
}
\medskip

\begin{enumerate}
\item 
	\begin{enumerate}
		\item Julie a fait fonctionner ce programme en choisissant le nombre 5. Vérifier que
ce qui est dit à la fin est: \og J'obtiens finalement 20 \fg.
		\item Que dit le programme si Julie le fait fonctionner
en choisissant au départ le nombre 7 ?
	\end{enumerate}
\item Julie fait fonctionner le programme, et ce qui est dit à
la fin est: \og J'obtiens finalement 8 \fg.
Quel nombre Julie a-t-elle choisi au départ ?
\item Si l'on appelle x le nombre choisi au départ, écrire en
fonction de x l' expression obtenue à la fin du programme, puis réduire cette expression autant que
possible.
\item Maxime utilise le programme de calcul ci-dessous :
\begin{center}	
\begin{tabularx}{0.5\linewidth}{|l X|}\hline
$\bullet~~$&Choisir un nombre.\\
$\bullet~~$&Lui ajouter 2\\
$\bullet~~$&Multiplier le résultat par 5\\\hline
\end{tabularx}
\end{center}

Peut-on choisir un nombre pour lequel le résultat obtenu par Maxime est le même que celui obtenu par
Julie ?	
\end{enumerate}