
Le club de golf de Trifouilli-les-oies propose 3 formules 
\begin{description}
\item[Formule A :] abonnement annuel : 2300 \euro{} et la gratuite de tous les parcours du 1\ier janvier au 31 Décembre.
\item[Formule B :] abonnement annuel : 1600 \euro{} et 25 \euro{} le parcours.
\item[Formule C :] 65  \euro{} le parcours.
\end{description}

\begin{enumerate}
\begin{multicols}{2}
\item Complète le tableau.

\begin{center}
\begin{tabular}{|>{\centering\arraybackslash}p{1cm}|>{\centering\arraybackslash}p{0.8cm}|>{\centering\arraybackslash}p{0.8cm}|>{\centering\arraybackslash}p{0.8cm}|>{\centering\arraybackslash}p{0.8cm}|>{\centering\arraybackslash}p{0.8cm}|}
\hline 
$x$ & 5 & 10 & 25 & 30 & 40 \vplus \\ 
\hline 
Formule A &  &  &  &  &  \\ 
\hline 
Formule B &  &  &  &  &  \\ 
\hline 
Formule C &  &  &  &  &  \\ 
\hline 
\end{tabular} 
\end{center}

\item Quelle est la meilleure formule pour Patrick qui joue 27 fois par an ?
\Point{1}
\item Détermine trois fonctions qui formalisent les 3 formules.
\Point{3}
\end{multicols}

\item Représente dans le repère ci dessous les trois fonctions.

\definecolor{cqcqcq}{rgb}{0.7529411764705882,0.7529411764705882,0.7529411764705882}
\begin{tikzpicture}[line cap=round,line join=round,>=triangle 45,x=0.07511687434985508cm,y=0.0035360821101843637cm]
\draw [color=cqcqcq,, xstep=0.7511687434985508cm,ystep=0.7072164220368727cm] (-8.432814152603989,-147.75375795330476) grid (124.69306583842486,2680.2348713827437);
\draw[->,color=black] (-8.432814152603989,0.) -- (124.69306583842486,0.);
\foreach \x in {,10.,20.,30.,40.,50.,60.,70.,80.,90.,100.,110.,120.}
\draw[shift={(\x,0)},color=black] (0pt,2pt) -- (0pt,-2pt) node[below] {\footnotesize $\x$};
\draw[->,color=black] (0.,-147.75375795330476) -- (0.,2680.2348713827437);
\foreach \y in {,200.,400.,600.,800.,1000.,1200.,1400.,1600.,1800.,2000.,2200.,2400.,2600.}
\draw[shift={(0,\y)},color=black] (2pt,0pt) -- (-2pt,0pt) node[left] {\footnotesize $\y$};
\draw[color=black] (0pt,-10pt) node[right] {\footnotesize $0$};
\clip(-8.432814152603989,-147.75375795330476) rectangle (124.69306583842486,2680.2348713827437);
\end{tikzpicture}


\item Quelle est la formule la plus adaptée selon le nombre de parcours joués. Une conclusion rédigée est attendue.
\Point{5}
\end{enumerate}
