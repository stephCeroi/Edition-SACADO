\documentclass[openany]{book}



\input{../../../latex_preambule_style/preambule}
\input{../../../latex_preambule_style/styleCoursCycle4}
\input{../../../latex_preambule_style/styleExercices}
\input{../../../latex_preambule_style/styleExercicesAideCompetences}
%\input{../../latex_preambule_style/styleCahier}
\input{../../../latex_preambule_style/bas_de_page_cycle4}
\input{../../../latex_preambule_style/algobox}


%%%%%%%%%%%%%%%  Affichage ou impression  %%%%%%%%%%%%%%%%%%
\newcommand{\impress}[2]{
\ifthenelse{\equal{#1}{1}}  %   1 imprime / affiche sur livre  -----    0 affiche sur cahier 
{%condition vraieé
#2
}% fin condition vraie
{%condition fausse
}% fin condition fausse
} % fin de la procédure
%%%%%%%%%%%%%%%  Affichage ou impression  %%%%%%%%%%%%%%%%%%
%%%%%%%%%%%   Marges de pages  %%%%%%%%%%%%%%%% 
 \usepackage{geometry}
 \geometry{top=1cm, bottom=0cm, left=2cm , right=2cm}
%%%%%%%%%%%%%%%%%%%%%%%%%%%%%%%%%%%%%%%%%%%%%%%
%%%%%%%%%%%%%%%%%%%%%%%%%%%%%%%%%%%%%%%%%%%%%%%%

\begin{document}


\begin{seance}[Notions de fonction.]

\section{Préparer le DNB}


\end{seance}

\Exe

\begin{enumerate}
\item Ce n'est pas une situation de proportionnalité car le graphique montrant
l'évolution de la tension en fonction du temps n'est pas une droite.
\item La tension mesurée au bout de 0,2 s, la tension mesurée est de 4,4 V.
\item Je calcule 60\,\% de la tension maximale : $\dfrac{60}{100} \times 5 = 0,6 \times 5 = 3$.

60\,\% de la tension maximale correspond à 3~V.

Par lecture graphique, on détermine que cette tension est atteinte au bout
d'environ $0,09$~s.
\end{enumerate}

\Exe

\begin{center}
\begin{tabularx}{\linewidth}{|c|*{10}{>{\centering \arraybackslash}X|}}\hline
Temps en heure &0 &0,5&1 		&1,5 	&2 	&2,5 	&3 	&3,5 &4 	&4,5\\ \hline
Distance en km &0 &15 &\ldots 	&55 	&70 &80 	&100&110 &135 	&160\\ \hline
\end{tabularx}
\end{center}

\medskip

\begin{enumerate}
\item %Quelle distance a-t-il parcourue au bout de 2~h 30~min de course?
2~h 30~min ou 2,5~h : la distance parcourue est égale à 80 km.
\item %Montrer qu'il a parcouru 30 km lors de la troisième heure de course.
De la 2\up{e} à la 3\up{e} heure il a parcouru $100 - 70 = 30$~km.
\item %A-t-il été plus rapide lors de la troisième ou bien lors de la quatrième heure de
%course ?
De la 3\up{e} à la 4\up{e} heure il a parcouru $135 - 100 = 35$~km, soit plus que pendant la 3\up{e} heure.
\item %Répondre aux questions qui suivent sur la feuille ANNEXE , qui est à rendre avec la copie.
	\begin{enumerate}
		\item Placer les 9 points du tableau dans le repère. On ne peut pas placer le
point d'abscisse 1 puisque l'on ne connaît pas son ordonnée.
		\item En utilisant votre règle, relier les points consécutifs entre eux.
 	\end{enumerate}
\item %En considérant que la vitesse du cycliste est constante entre deux relevés,
%déterminer, par lecture graphique, le temps qu'il a mis pour parcourir 75~km.
On lit environ 2,25~h soit 2~h 15~min.
\item %On considère que la vitesse du cycliste est constante entre le premier relevé
%effectué au bout de 0,5~h de course et le relevé effectué au bout de 1,5~h de
%course ; déterminer par lecture graphique la distance parcourue au bout de 1~h
%de course.
Si la vitesse est constante pendant cette heure, la représentation sur cet intervalle est affine ; on trace donc la verticale ($x = 1$) qui coupe la représentation en un point dont l'ordonnée est environ 35~(km).
\item %Soit $f$ la fonction, qui au temps de parcours du cycliste Thomas Vœckler,
%associe la distance parcourue. La fonction $f$ est-elle linéaire ?
La fonction n'est pas linéaire puisque les points ne sont pas alignés.

Plus mathématique on a vu qu'il faisait 30 km en une heure et plus tard 35~km en une heure. La fonction n'est pas linéaire.
\end{enumerate}

\vspace{0,5cm}

\Exe

\medskip

%Le jardinier d'un club de football décide de semer à nouveau du gazon sur l'aire de
%jeu. Pour que celui-ci pousse correctement, il installe un système d'arrosage
%automatique qui se déclenche le matin et le soir, à chaque fois, pendant 15 minutes.
%
%\setlength\parindent{9mm}
%\begin{itemize}
%\item[$\bullet~~$] Le système d'arrosage est constitué de 12 circuits indépendants.
%\item[$\bullet~~$] Chaque circuit est composé de 4 arroseurs.
%\item[$\bullet~~$] Chaque arroseur a un débit de 0,4 m$^3$ d'eau par heure.
%\end{itemize}
%\setlength\parindent{0mm}
%
%Combien de litres d'eau auront été consommés si on arrose le gazon pendant tout le
%mois de juillet ?

%On rappelle que 1 m$^3 = \np{1000}$~litres et que le mois de juillet compte $31$~jours.
Chaque jour l'arrosage fonctionne pendant $2 \times 15 = 30$~min soit 0,5~h. Un arroseur débite donc pendant cette demi-heure 0,2~m$^3$.

Pendant le mois de juillet on aura donc déversé :

$31 \times 12 \times 4 \times 0,2 = 297,6$~m$^3$, soit \np{297600}~litres d'eau.
\vspace{0,5cm}
\begin{center}
\begin{flushleft}
\textbf{Exercice 2 question 4}
\end{flushleft}
\bigskip

\psset{xunit=1.7cm,yunit=0.1cm,arrowsize=2pt 5}
\begin{pspicture}(-0.5,-10)(5.5,200)
\uput[u](5,0){Temps en $h$}
\uput[r](0,192.5){Distance en km}
\multido{\n=0.0+0.5}{12}{\psline[linewidth=0.2pt](\n,0)(\n,190)}
\multido{\n=0+5}{39}{\psline[linewidth=0.2pt](0,\n)(5.5,\n)}
\psaxes[linewidth=1.25pt,Dy=20]{->}(0,0)(0,0)(5.5,190)
\psaxes[linewidth=1.25pt,Dy=20](0,0)(0,0)(5.5,190)
\psdots(0,0)(0.5,15)(1.5,55)(2,70)(2.5,80)(3,100)(3.5,110)(4,135)(4.5,160)
\psline[linewidth=1pt](0,0)(0.5,15)(1.5,55)(2,70)(2.5,80)(3,100)(3.5,110)(4,135)(4.5,160)
\psline[linewidth=1pt,linecolor=blue,ArrowInside=->](0,75)(2.25,75)(2.25,0)
\psline[linewidth=1pt,linecolor=red,ArrowInside=->](1,0)(1,35)(0,35)
\end{pspicture}
\end{center}


\end{document}