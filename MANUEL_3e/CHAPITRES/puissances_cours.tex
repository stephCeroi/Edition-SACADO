\chapter{Puissance d'un nombre}
{https://sacado.xyz/qcm/parcours_show_course/0/117120}
{ 

 \begin{CpsCol}
 \textbf{Les savoir-faire du parcours} 
 \begin{itemize}
\item Écrire un nombre avec une puissance de base 10
\item Calculer avec des nombres écrits en puissance de 10
\item Écrire un nombre en écriture scientifique
\item Calculer avec des puissances de base quelconque et exposant entier
 \end{itemize}
 \end{CpsCol}
}
%
%
%\begin{pageHistoire} 
% 
%En 1585, dans son ouvrage \textbf{La Disme}, Simon Stevin (1548 - 1620) ingénieur et mathématicien flamand, propose une écriture des nombres qui permet de simplifier les calculs (quelquefois très lourds en écriture fractionnaire).\\
%
%Il est considéré comme un précurseur de l'écriture décimale.
% 
%
%\end{pageHistoire} 



%%%%%%%%%%%%%%%%%%%%%%%%%%%%%%%%%%%%%%%%%%%%%%%%%%%%%%%%%%%%%%%%%%%%%%%%%%%%%%%%%%%%
%%%%%%%%%%        Cours             %%%%%%%%%%%%%%%%%%%%%%%%%%%%%%%%%%%%%%%%%%%%%%%%
%%%%%%%%%%%%%%%%%%%%%%%%%%%%%%%%%%%%%%%%%%%%%%%%%%%%%%%%%%%%%%%%%%%%%%%%%%%%%%%%%%%%
\begin{pageCours} 

\section{Puissance de base 10}


\begin{DefT}{Puissance de base 10}\index{Puissance!de base 10}
Soit $n$ un nombre entier. Le produit $\underbrace{10 \times 10 \times 10 \times \cdots \times 10 \times 10}_n$ se note $10^n$ et se lit "10 exposant $n$". 
\end{DefT}

\begin{Ex}
$10^1 = 10$.  $100 = 10^2$ donc 100 est une puissance de 10 et $\np{10000} = 10^4$ donc 10000 est aussi une puissance de 10. 
\end{Ex}

\begin{Rq}
Le nombre de zéros est égal à l'exposant. $10^n= 1\underbrace{0000 \cdots 000}_{n \text{zéros}}$
\end{Rq}

\begin{Def}
Par convention, $10^0=1$.
\end{Def}




\begin{DefT}{Puissance de base 10 d'exposant négatif}\index{Puissance!de base d'exposant négatif}
L'écriture $10^{-n}$ désigne l'inverse de $10^n$, c'est à dire : $10^{-n}= \frac{1}{10^n}$. 
\end{DefT}


\begin{Rq}
L'exposant correspond au nombre de chiffres après la virgule. $10^{-n}= 0,\underbrace{0000 \cdots 001}_{n \text{chiffres}}$
\end{Rq}


\section{Écriture scientifique}

\begin{DefT}{Écriture scientifique}\index{Écriture scientifique}
L'écriture scientifique d'un nombre est le produit d'un nombre décimal dont la partie entière comporte un seul chiffre différent de zéro par une puissance de 10.

L'écriture scientifique est de la forme $a\times 10^n$, où $1 \leq a < 10$ et $n$ est un entier relatif.

L'écriture scientifique d'un nombre est unique.
\end{DefT}



\end{pageCours} 
%%%%%%%%%%%%%%%%%%%%%%%%%%%%%%%%%%%%%%%%%%%%%%%%%%%%%%%%%%%%%%%%%%%%%%%%%%%%%%%%%%%%
%%%%%%%%%%   Application directe    %%%%%%%%%%%%%%%%%%%%%%%%%%%%%%%%%%%%%%%%%%%%%%%%
%%%%%%%%%%%%%%%%%%%%%%%%%%%%%%%%%%%%%%%%%%%%%%%%%%%%%%%%%%%%%%%%%%%%%%%%%%%%%%%%%%%%
\begin{pageAD} 

\Sf{Représenter avec les puissances de 10.}


\begin{ExoCad}{Représenter.}{1234}{0}{0}{0}{0}

Écrire les nombres suivants avec une puissance de $10$.

\begin{enumerate}
\begin{minipage}{0.3\linewidth}
\item 
\item
\end{minipage}
\begin{minipage}{0.3\linewidth}
\item 
\item
\end{minipage}
\begin{minipage}{0.3\linewidth}
\item 
\item
\end{minipage}
\end{enumerate}
\end{ExoCad}

\begin{ExoCad}{Représenter.}{1234}{0}{0}{0}{0}

Le nombre $10^{-6}$ est égal à l'un des nombre suivant. Lequel ?
$$ -60 \quad ;  \quad -10^6 \quad ; \quad \np{0,000 0001}  \quad ; \quad \text{un millionième}$$

\end{ExoCad}


\begin{ExoCad}{Représenter.}{1234}{0}{0}{0}{0}
\begin{enumerate}
\item Écris 1 millième sous forme décimale
\item Écris sous forme de fraction 1 millième
\item Déduis en l'écriture de 1 millième sous forme d'une puissance de 10.
\end{enumerate}
\end{ExoCad}


\Sf{Calculer avec les puissances de 10.}

\begin{ExoCad}{Représenter.}{1234}{0}{0}{0}{0}

Durant les inondations dans la région parisienne de Juin 2016, la région Ile de France a fait un stock de bouteilles d'eau pour la population. Chaque habitant bénéficie de 2 litres d'eau par jour. La région Ile de France compte \np{10 000000} habitants.

Quel est le nombre de litres d'eau stockés pour les 5 jours d'inondations ?
\end{ExoCad}

\begin{ExoCad}{Représenter.}{1234}{0}{0}{0}{0}
Le tardigrade mesure est 1 mm. La longueur d'un stage de rugby est 100 m environ. Combien de tardigrades peut-on mettre bout à bout sur la longueur d'un stade de rugby ?
\end{ExoCad}


\Sf{Écrire en notation scientifique}


\begin{enumerate}
\item Écrire en notation scientifique le nombre intervenant dans la phrase suivante :
« La masse du Soleil est environ égale à 1 989 000 000 000 000 000 000 000 000 000 kg ».

\item Le proton et le neutron sont deux particules composant le noyau des atomes. Leur taille est environ égale à $10^{-15}$ m. Exprimer cette taille en millimètre (mm), puis en micromètre ($\mu$m).
\end{enumerate}




\end{pageAD}  

%%%%%%%%%%%%%%%%%%%%%%%%%%%%%%%%%%%%%%%%%%%%%%%%%%%%%%%%%%%%%%%%%%%%%%%%%%%%%%%%%%%%
%%%%%%%%%%        Cours             %%%%%%%%%%%%%%%%%%%%%%%%%%%%%%%%%%%%%%%%%%%%%%%%
%%%%%%%%%%%%%%%%%%%%%%%%%%%%%%%%%%%%%%%%%%%%%%%%%%%%%%%%%%%%%%%%%%%%%%%%%%%%%%%%%%%%
\begin{pageCours} 



\section{Puissance de base $a$}


\Dec{1}{Puis-2}

\begin{DefT}{Puissance de base a}\index{Puissance!de base a}
Le produit $\underbrace{a \times a \times a \times \cdots \times a \times a}_n$ se note $a^n$ et se lit "a exposant $n$".  
La puissance du nombre $a$, $a^n$, est un produit de $n$ fois le même nombre $a$.
\end{DefT}
 
\begin{minipage}[t]{0.49\linewidth}
\begin{Prop}[Produit de puissances]\index{Puissances!Produit}
Soit $n$ et $m$ deux nombres entiers et $a$ un nombre.\\
$a^n \times a^m = a^{n+m}$.
\end{Prop}
 \begin{Ex}
 \begin{description}
 \item[•] $10^3 \times 10^4 = 10^{3+4}=10^7$
 \item[•] $2^2 \times 2^3 = 2^{2+3}=2^5$ 
  \end{description}
 \end{Ex}
\end{minipage}
 \hfill
\begin{minipage}[t]{0.49\linewidth}
\begin{Prop}[Quotient de puissances]\index{Puissances!Quotient}
Soit $n$ et $m$ deux nombres entiers et $a$ un nombre.\\
$\frac{a^n}{a^m} = a^{n-m}$.
\end{Prop}
 \begin{Ex}
$\frac{10^8}{10^2} = 10^6$ et $\frac{5^9}{5^3} = 5^6$
 \end{Ex}
\end{minipage}
 

\begin{minipage}{0.48\linewidth}
\Exo{1}{Puis-13}

\end{minipage}
\hfill
\begin{minipage}{0.48\linewidth}

\Exo{1}{Puis-20}

\end{minipage}

 Ariane affirme que $2^{40}$ est le double de $2^{39}$. A-t-elle raison ? \point{3}
 
 
\Exo{1}{Puis-21}

 

\Exo{1}{Puis-22}


 
\begin{minipage}{0.48\linewidth}
\Exo{1}{DNP-56}

\end{minipage}
\hfill
\begin{minipage}{0.48\linewidth}

\Fl{1}{Puis-12}

\end{minipage}

%\PO{1}{Puis-23}

%\App{1}{Puis-24}

\App{1}{Puis-14}

\App{1}{Puis-15}

%\PO{1}{Puis-18}



%\PO{1}{Puis-19}
%\begin{autoeval}
%\begin{tabular}{p{12cm}p{0.5cm}p{0.5cm}p{0.5cm}p{1cm}}
%\textbf{Compétences visées} &  M I & MF & MS  & TBM \vcomp \\ 
%Calculer avec des puissances de base quelconque et exposant entier & $\square$ & $\square$  & $\square$ & $\square$ \vcomp \\  
%\end{tabular}
%{\footnotesize MI : maitrise insuffisante ; MF = Maitrise fragile ; MS = Maitrise satisfaisante ; TBM = Très bonne maitrise}
% 
%\end{autoeval}


 
\end{pageCours}
%%%%%%%%%%%%%%%%%%%%%%%%%%%%%%%%%%%%%%%%%%%%%%%%%%%%%%%%%%%%%%%%%%%%%%%%%%%%%%%%%%%%
%%%%%%%%%%   Application directe    %%%%%%%%%%%%%%%%%%%%%%%%%%%%%%%%%%%%%%%%%%%%%%%%
%%%%%%%%%%%%%%%%%%%%%%%%%%%%%%%%%%%%%%%%%%%%%%%%%%%%%%%%%%%%%%%%%%%%%%%%%%%%%%%%%%%%
\begin{pageAD} 

\Sf{Repérer un nombre décimal sur la droite graduée.}

\ExoCad{Représenter.}

\ExoCad{Représenter.}

\ExoCad{Représenter.}

\end{pageAD}
%%%%%%%%%%%%%%%%%%%%%%%%%%%%%%%%%%%%%%%%%%%%%%%%%%%%%%%%%%%%%%%%%%%%%%%%%%%%%%%%%%%%
%%%%%%%%%%        Cours             %%%%%%%%%%%%%%%%%%%%%%%%%%%%%%%%%%%%%%%%%%%%%%%%
%%%%%%%%%%%%%%%%%%%%%%%%%%%%%%%%%%%%%%%%%%%%%%%%%%%%%%%%%%%%%%%%%%%%%%%%%%%%%%%%%%%%
\begin{pageCours}

\section{Encadrer un nombre décimal}

 



\section{Ranger, classer des nombres décimaux}


\end{pageCours}
%%%%%%%%%%%%%%%%%%%%%%%%%%%%%%%%%%%%%%%%%%%%%%%%%%%%%%%%%%%%%%%%%%%%%%%%%%%%%%%%%%%%
%%%%%%%%%%   Application directe    %%%%%%%%%%%%%%%%%%%%%%%%%%%%%%%%%%%%%%%%%%%%%%%%
%%%%%%%%%%%%%%%%%%%%%%%%%%%%%%%%%%%%%%%%%%%%%%%%%%%%%%%%%%%%%%%%%%%%%%%%%%%%%%%%%%%%
\begin{pageAD} 

\Sf{ }

\ExoCad{Représenter. Communiquer.}

 
\ExoCad{Représenter. Communiquer.}

 
\Sf{ }


\ExoCad{Représenter. Communiquer.}

 
\end{pageAD} 
%%%%%%%%%%%%%%%%%%%%%%%%%%%%%%%%%%%%%%%%%%%%%%%%%%%%%%%%%%%%%%%%%%%%%%%%%%%%%%%%%%%%
%%%%%%%%%%        Parcours 1    %%%%%%%%%%%%%%%%%%%%%%%%%%%%%%%%%%%%%%%%%%%%%%%%%%%%
%%%%%%%%%%%%%%%%%%%%%%%%%%%%%%%%%%%%%%%%%%%%%%%%%%%%%%%%%%%%%%%%%%%%%%%%%%%%%%%%%%%%
\begin{pageParcoursu} 



\begin{ExoCu}{Modéliser. Calculer.}{1234}{1}{0}{0}{0}

\begin{minipage}{0.48\linewidth}

\begin{enumerate}
\item En informatique, l'information est codée à partir de bits, qui ne prennent que deux valeurs : 0 et 1. Un octet est un regroupement de 8 bits. Combien d'informations différentes peuvent être codées sur un octet ?
\item Les capacités de stockage des mémoires informatiques (disques durs, clé USB, ...) utilisent un grand nombre d'octets. Cela conduit à utiliser des multiples de l'octet, dont voici les principaux ci-contre.
\end{enumerate}

À l'aide des unités précédentes, donner un ordre de grandeur de la taille d'un fichier relatif aux données suivantes :
\begin{description}
\item[•] une photographie numérique ;
\item[•] l’ensemble des données circulant sur le web en 2015 ;
\item[•] un texte de dix lignes sur un traitement de textes ;
\item[•] l’ensemble des données générées chaque année à travers le monde ;
\item[•] la capacité d’un disque dur vendu en 2015 ;
\item[•] un DVD
\end{description}

\end{minipage}
\hfill
\begin{minipage}{0.48\linewidth}

\begin{center}
\begin{tabular}{|c|c|c|}
\hline 
NOM & SYMBOLE & NOMBRE D'OCTETS \\ 
\hline 
Kilooctet & Ko & $10^{3}$ \\ 
\hline 
Megaoctet & Mo & $10^{6}$  \\ 
\hline 
Gigaoctet& Go & $10^{9}$  \\ 
\hline 
Teraoctet & To & $10^{12}$  \\ 
\hline 
Petaoctet & Po & $10^{15}$  \\ 
\hline 
Exaoctet & Eo & $10^{18}$  \\ 
\hline 
\end{tabular} 
\end{center}

\end{minipage}
\end{ExoCu}

\begin{ExoCu}{Modéliser. Calculer.}{1234}{1}{0}{0}{0}


\end{ExoCu}
 
 
 

\end{pageParcoursu}
%%%%%%%%%%%%%%%%%%%%%%%%%%%%%%%%%%%%%%%%%%%%%%%%%%%%%%%%%%%%%%%%%%%%%%%%%%%%%%%%%%%%
%%%%%%%%%%        Parcours 2    %%%%%%%%%%%%%%%%%%%%%%%%%%%%%%%%%%%%%%%%%%%%%%%%%%%%
%%%%%%%%%%%%%%%%%%%%%%%%%%%%%%%%%%%%%%%%%%%%%%%%%%%%%%%%%%%%%%%%%%%%%%%%%%%%%%%%%%%%
\begin{pageParcoursd} 

\begin{ExoCd}{Modéliser. Calculer.}{1234}{1}{0}{0}{0}


\end{ExoCd}

 

\begin{ExoCd}{Modéliser. Calculer.}{1234}{1}{0}{0}{0}


\end{ExoCd}


\end{pageParcoursd}
%%%%%%%%%%%%%%%%%%%%%%%%%%%%%%%%%%%%%%%%%%%%%%%%%%%%%%%%%%%%%%%%%%%%%%%%%%%%%%%%%%%%
%%%%%%%%%%        Parcours 3    %%%%%%%%%%%%%%%%%%%%%%%%%%%%%%%%%%%%%%%%%%%%%%%%%%%%
%%%%%%%%%%%%%%%%%%%%%%%%%%%%%%%%%%%%%%%%%%%%%%%%%%%%%%%%%%%%%%%%%%%%%%%%%%%%%%%%%%%%
\begin{pageParcourst}

\begin{ExoCt}{Modéliser. Calculer.}{1234}{1}{0}{0}{0}


\end{ExoCt}


 


\end{pageParcourst}
%%%%%%%%%%%%%%%%%%%%%%%%%%%%%%%%%%%%%%%%%%%%%%%%%%%%%%%%%%%%%%%%%%%%%%%%%%%%%%%%%%%%
%%%%%%%%%%   Auto-evaluation    %%%%%%%%%%%%%%%%%%%%%%%%%%%%%%%%%%%%%%%%%%%%%%%%%%%%
%%%%%%%%%%%%%%%%%%%%%%%%%%%%%%%%%%%%%%%%%%%%%%%%%%%%%%%%%%%%%%%%%%%%%%%%%%%%%%%%%%%%
\begin{pageAuto} 

\begin{ExoAuto}{Modéliser. Calculer.}{1234}{1}{0}{0}{0}


\end{ExoAuto}

  
\end{pageAuto}
 
