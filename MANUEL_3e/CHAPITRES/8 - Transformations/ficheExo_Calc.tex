\documentclass[openany]{book}

\input{../../../latex_preambule_style/preambule}
\input{../../../latex_preambule_style/styleCoursCycle4}
\input{../../../latex_preambule_style/styleExercices}
\input{../../../latex_preambule_style/styleExercicesAideCompetences}
%\input{../../latex_preambule_style/styleCahier}
\input{../../../latex_preambule_style/bas_de_page_cycle4}
\input{../../../latex_preambule_style/algobox}



%%%%%%%%%%%%%%%  Affichage ou impression  %%%%%%%%%%%%%%%%%%
 \usepackage{geometry}
 \geometry{top=2.5cm, bottom=0cm, left=2cm , right=2cm}
%%%%%%%%%%%%%%%%%%%%%%%%%%%%%%%%%%%%%%%%%%%%%%%%

\begin{document}


\begin{seance}[Les homothéties]

\Titre{Découverte}{3}
\end{seance}


\begin{CpsCol}
\textbf{Utiliser les transformations du plans pour démontrer}
\begin{description}
\item[$\square$] Connaitre les homothéties.
\end{description}
\end{CpsCol}

\EPC{1}{T-00}{Représenter, raisonner}

\Dec{1}{T-01}

\begin{DefT}{Homothétie}
Soit $O$ un point fixe du plan et $k$ un nombre non nul.
Transformer une figure par une homothétie de centre O et de rapport $k$ c'est l'agrandir ou la réduire en faisant apparaitre une situation de Thalès de rapport $k$ ou $-k$.
\begin{description}
\item[•] Lorsque $k$ est positif, la figure est en voile.
\item[•] Lorsque $k$ est négatif, la figure est en papillon. 
\end{description}

{\footnotesize Voir les figures ci-dessus.}
\end{DefT}


\Rec{1}{T-02}



\begin{ThT}{Image des figures usuelles}
\begin{description}
\item L'image d'une droite $d$ par une homothétie est une droite parallèle à $d$.
\item L'image d'un segment $[AB]$ par une homothétie de rapport $k$ est un segment $[A'B']$ tel que $A'B'=kAB$. 
\end{description}
\end{ThT}


\begin{seance}[Les homothéties]

\Titre{Utiliser les homothéties}{3}
\end{seance}

\begin{CpsCol}
\textbf{Utiliser les transformations du plans pour démontrer}
\begin{description}
\item[$\square$] Construire des figures avec les homothéties.
\item[$\square$] Connaitre les propriétés des homothéties.
\end{description}
\end{CpsCol}

\Rec{1}{T-03}

\Rec{1}{T-04}

\Rec{1}{T-05}


\begin{ThT}{Image des figures usuelles}
L'image d'une figure usuelle $\mathscr{F}$ par une homothétie est une figure usuelle $\mathscr{F'}$ dont les dimensions sont multipliées par :
\begin{description}
\item  $k$, si $k$ est positif 
\item  $-k$ si $k$ est négatif.
\end{description}
\end{ThT}

\begin{Rq}
\begin{description}[leftmargin=*]
\item  Les aires d'une figure et de son image sont donc multipliées par $k^2$ 
\item  Les volumes d'un solide et de son image sont donc multipliés par $k^3$ ou $-k^3$.
\item  Les homothéties permettent des agrandissements et réductions de figures et de formes. De nombreux logiciels utilisent cette transformation.
\end{description}
\end{Rq}

\AD{1}{T-11}

\paragraphe{Exercices du livre}

\begin{itemize}
\item 16 p 196.
\item 28 p 197.
\item 34 p 200.
\item 42 p 201. 
\item 48 p 203.
\end{itemize}




\input{T-10}

%
%\begin{seance}[Les constructions]
%
%\Titre{Pour aller plus loin}{3}
%\end{seance}
%
%\begin{CpsCol}
%\textbf{Utiliser les transformations du plan pour démontrer}
%\begin{description}
%\item[$\square$] Conjecturer.
%\item[$\square$] Démontrer.
%\end{description}
%\end{CpsCol}
%
%\AD{1}{T-06}
%
%\Rec{1}{T-07}
%
%\begin{seance}[Les constructions]
%
%\Titre{Utiliser les TICE}{3}
%\end{seance}
%
%\begin{CpsCol}
%\textbf{Utiliser les transformations du plan pour démontrer}
%\begin{description}
%\item[$\square$] Conjecturer.
%\item[$\square$] Démontrer.
%\end{description}
%\end{CpsCol}
%
%\Rec{1}{T-09}
%
%\Rec{1}{T-08}

\end{document}
