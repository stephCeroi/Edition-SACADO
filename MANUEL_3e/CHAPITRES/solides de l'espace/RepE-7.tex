
\begin{minipage}{0.48\linewidth}

Étant donné un repère orthonormal (O,I,J), tout point $M$ du plan peut être repéré par un couple de nombres relatifs $\left(x;y\right)$ appelé \textbf{coordonnées cartésiennes} du point $M$ (voir figure ci-dessous).

\begin{description}
\item $A(-1;2)$
\item $B(4;3)$
\item $C(2;-1)$
\end{description}

\end{minipage}
\hfill
\begin{minipage}{0.48\linewidth}

\definecolor{qqqqff}{rgb}{0.,0.,1.}
\definecolor{cqcqcq}{rgb}{0.7529411764705882,0.7529411764705882,0.7529411764705882}
\begin{tikzpicture}[line cap=round,line join=round,>=triangle 45,x=1.0cm,y=1.0cm]
\draw [color=cqcqcq,, xstep=1.0cm,ystep=1.0cm] (-2.073026147439821,-1.4530480083130866) grid (4.6098020671918265,3.6082115954152947);
\draw[->,color=black] (-2.073026147439821,0.) -- (4.6098020671918265,0.);
\foreach \x in {-2.,-1.,1.,2.,3.,4.}
\draw[shift={(\x,0)},color=black] (0pt,2pt) -- (0pt,-2pt) node[below] {\footnotesize $\x$};
\draw[->,color=black] (0.,-1.4530480083130866) -- (0.,3.6082115954152947);
\foreach \y in {-1.,1.,2.,3.}
\draw[shift={(0,\y)},color=black] (2pt,0pt) -- (-2pt,0pt) node[left] {\footnotesize $\y$};
\draw[color=black] (0pt,-10pt) node[right] {\footnotesize $0$};
\clip(-2.073026147439821,-1.4530480083130866) rectangle (4.6098020671918265,3.6082115954152947);
\begin{scriptsize}
\draw [color=qqqqff] (-1.,2.)-- ++(-2.5pt,0 pt) -- ++(5.0pt,0 pt) ++(-2.5pt,-2.5pt) -- ++(0 pt,5.0pt);
\draw[color=qqqqff] (-0.8773240404101388,2.2978531219581084) node {$A$};
\draw [color=qqqqff] (4.,3.)-- ++(-2.5pt,0 pt) -- ++(5.0pt,0 pt) ++(-2.5pt,-2.5pt) -- ++(0 pt,5.0pt);
\draw[color=qqqqff] (4.118417639645382,3.297001457969213) node {$B$};
\draw [color=qqqqff] (2.,-1.)-- ++(-2.5pt,0 pt) -- ++(5.0pt,0 pt) ++(-2.5pt,-2.5pt) -- ++(0 pt,5.0pt);
\draw[color=qqqqff] (2.1201209676231736,-0.6995918860752046) node {$C$};
\end{scriptsize}
\end{tikzpicture}
\end{minipage}