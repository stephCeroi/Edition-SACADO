
On considère le parcours suivant, partant du point A de l’équateur ayant pour longitude 0 :
\begin{itemize}
\item on suit l’équateur vers l’Est sur 5000 km, jusqu’à un point B ;
\item on suit vers le Nord le méridien passant par B sur 5000 km, jusqu’à un point C ;
\item on suit vers l’Ouest le parallèle passant par C sur 5000 km, jusqu’à un point D ;
\item on suit vers le Sud le méridien passant par D sur 5000 km, jusqu’à un point E.
\end{itemize}

Déterminer les coordonnées géographiques du point E, puis calculer la distance séparant les points A et E.

%COMMENTAIRES 
%La mise à disposition d’une mappemonde permet aux élèves de se repérer plus facilement et
%leur fournit un outil expérimental efficace.
%Pour aider les élèves à calculer le rayon du parallèle passant par le point C, l’enseignant
%pourra les amener à réfléchir aux représentations spatiales les plus adaptées.