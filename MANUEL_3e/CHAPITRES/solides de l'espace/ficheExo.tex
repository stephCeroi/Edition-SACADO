\documentclass[openany]{book}

\input{../../../latex_preambule_style/preambule}
\input{../../../latex_preambule_style/styleCoursCycle4}
\input{../../../latex_preambule_style/styleExercices}
\input{../../../latex_preambule_style/styleExercicesAideCompetences}
%\input{../../latex_preambule_style/styleCahier}
\input{../../../latex_preambule_style/bas_de_page_cycle4}
\input{../../../latex_preambule_style/algobox}


%%%%%%%%%%%%%%%  Affichage ou impression  %%%%%%%%%%%%%%%%%%
\newcommand{\impress}[2]{
\ifthenelse{\equal{#1}{1}}  %   1 imprime / affiche sur livre  -----    0 affiche sur cahier 
{%condition vraieé
#2
}% fin condition vraie
{%condition fausse
}% fin condition fausse
} % fin de la procédure
%%%%%%%%%%%%%%%  Affichage ou impression  %%%%%%%%%%%%%%%%%%
 \usepackage{geometry}
 \geometry{top=2.5cm, bottom=0cm, left=2cm , right=2cm}
%%%%%%%%%%%%%%%%%%%%%%%%%%%%%%%%%%%%%%%%%%%%%%%%

\begin{document}

 
 

\begin{seance}[Solides de l'espace]

\begin{description}
\item[$\square$]  Utiliser, produire et mettre en relation des représentations de solides et de situations
spatiales.
\item[$\square$] Développer sa vision de l’espace.
\end{description}
\end{seance}

\Exe

\parbox{0.45\linewidth}{La figure ci-contre représente un solide constitué de l'assemblage de quatre cubes :
\setlength\parindent{8mm}
\begin{itemize}
\item trois cubes d'arête 2~cm ;
\item un cube d'arête 4~cm.
\end{itemize}
\setlength\parindent{0mm}}\hfill
\parbox{0.5\linewidth}{\psset{unit=1cm}
\def\unite{\pspolygon[fillstyle=solid,fillcolor=gray](0,0.3)(0.8,0)(0.8,1)(0,1.3)
\pspolygon[fillstyle=solid,fillcolor=lightgray](0.8,0)(1.65,0.3)(1.65,1.3)(0.8,1)
\pspolygon[fillstyle=solid,fillcolor=lightgray](0.8,1)(1.65,1.3)(0.85,1.6)(0,1.3)}
\def\gros{\pspolygon[fillstyle=solid,fillcolor=gray](0,0.6)(1.6,0)(1.6,2)(0,2.6)
\pspolygon[fillstyle=solid,fillcolor=lightgray](1.6,0)(3.3,0.6)(3.3,2.6)(1.6,2)
\pspolygon[fillstyle=solid,fillcolor=lightgray](1.6,2)(3.3,2.6)(1.7,3.2)(0,2.6)}
\begin{pspicture}(2.3,0)(8,6)
\rput(2.9,0.1){\gros}
\rput(2.1,2.7){\unite}\rput(5.35,0.1){\unite}
\rput(2.85,0.8){\unite}
\rput(1.5,2){Vue de }\rput(1.5,1.6){face}
\rput(7.6,1){Vue de }\rput(7.6,0.6){droite}
\rput(4.6,5){Vue de}
\rput(4.6,4.6){dessus}
\psline[linewidth=2pt]{->}(4.6,4.2)(4.6,3.3)
\psline[linewidth=2pt]{->}(2.1,2.1)(2.7,2.4)
\psline[linewidth=2pt]{->}(7.6,1.4)(6.4,2.)
\end{pspicture}
}

\medskip

\begin{enumerate}
\item Quel est le volume de ce solide ? 
\item On a dessiné deux vues de ce solide (elles ne sont pas en vraie grandeur).

Dessiner la \textbf{vue de droite} de ce solide \textbf{en vraie grandeur}.


\begin{center}
\begin{tabularx}{\linewidth}{*{2}{>{\centering \arraybackslash}X}}
\textbf{Vue de face}&\textbf{Vue de dessus}\\
\psset{unit=0.6cm}
\begin{pspicture}(8,6)
\psframe(0,4)(2,6)\psframe(2,0)(6,4)\psframe(4,2)(6,4)\psframe(6,0)(8,2)
\end{pspicture}&\psset{unit=0.6cm}
\begin{pspicture}(8,6)
\psframe(0,2)(2,4)\psframe(2,2)(6,6)\psframe(6,4)(8,6)\psframe(4,0)(6,2)
\end{pspicture}
\end{tabularx}
\end{center}

 \end{enumerate}

\includegraphics[scale=0.5]{solide1.eps} 


\Exe



\parbox{0.48\linewidth}{Pour présenter ses macarons, une boutique souhaite utiliser des présentoirs dont la forme est une pyramide régulière à base carrée de côté 30 cm et dont les
arêtes latérales mesurent 55~cm.

On a schématisé le présentoir par la figure suivante :}\hfill \parbox{0.48\linewidth}{
\psset{unit=1cm}
\begin{pspicture}(5,4.5)
\pspolygon(0.5,0.5)(3.2,0.5)(4.5,2)(2.5,4)%ABCS
\psline(3.2,0.5)(2.5,4)
\psline[linestyle=dotted](0.5,0.5)(4.5,2)(1.8,2)(3.2,0.5)%ACDB
\psline[linestyle=dotted](0.5,0.5)(1.8,2)(2.5,4)
\psline[linestyle=dashed](2.5,4)(2.5,1.3)
\psline[linewidth=0.3pt](0.5,0.5)(1,0.5)(1.2,0.8)(0.7,0.8)
\psline[linewidth=0.3pt](3.2,0.5)(2.7,0.5)(2.9,0.8)(3.4,0.8)
\psline[linewidth=0.3pt](4.5,2)(4,2)(3.75,1.7)(4.25,1.7)
\psline[linewidth=0.3pt](1.8,2)(2.3,2)(2.1,1.7)(1.6,1.7)
\uput[dl](0.5,0.5){A} \uput[dr](3.2,0.5){B} \uput[ur](4.5,2){C} \uput[ul](1.8,2){D} \uput[d](2.5,1.3){O}\uput[u](2.5,4){S} 
\end{pspicture}
}

Peut-on placer ce présentoir dans une vitrine réfrigérée parallélépipédique dont la hauteur est de
50 cm ?

\includegraphics[scale=0.5]{solide2.eps}


%
%\begin{seance}[Les solides]
%
%\begin{minipage}{0.9\linewidth}
%\begin{description}[leftmargin=*]
%\item[$\square$] Connaitre et savoir utiliser les formules donnant le volume d’une pyramide,
%d’un cylindre, d’un cône ou d’une boule.
%\end{description}
%\end{minipage}
%\hfill
%\begin{minipage}{0.09\linewidth}
%\includegraphics[scale=0.4]{solides.eps} 
%\end{minipage}
%\end{seance}




%\parbox{0.35\linewidth}{Un TeraWattheure est noté: 1~TWh.
%
%La géothermie permet la production d'énergie électrique grâce à la chaleur des nappes d'eau souterraines.
%
%Le graphique ci-contre représente les productions d'électricité par différentes
%sources d'énergie en France en 2014.}\hfill
%\parbox{0.64\linewidth}{\psset{unit=0.7cm}
%\begin{pspicture}(-7.5,-3.8)(3.1,4)
%%\psgrid
%\psarc(0,0){2.9}{-134.6}{137.6}
%\psarc(0,0){2.25}{-134.6}{137.6}
%\psline(2.9;137.6)(2.25;137.6)\psline(2.9;-134.6)(2.25;-134.6)
%\psarc(0,0){2.9}{140.6}{155.8}
%\psarc(0,0){2.25}{140.6}{155.8}
%\psline(2.9;140.6)(2.25;140.6)\psline(2.9;155.8)(2.25;155.8)
%\psarc(0,0){2.9}{157.8}{202.7}
%\psarc(0,0){2.25}{157.8}{202.7}
%\psline(2.9;157.8)(2.25;157.8)\psline(2.9;202.7)(2.25;202.7)
%\psarc(0,0){2.9}{204.7}{222.7}
%\psarc(0,0){2.25}{204.7}{222.7}
%\psline(2.9;204.7)(2.25;204.7)\psline(2.9;222.7)(2.25;222.7)
%\rput(-2,-3.5){\scriptsize Statistiques de l'électricité en France 2014 RTE - chiffres de production 2014 - EDF}
%\rput(1,3.6){\small Nucléaire : 415,9~TWh}\psline{->}(1,3.4)(1.2,2.6)
%\rput(-4.8,3.2){\small Thermique à flamme :}
%\rput(-4.8,2.8){\small 25,8 TWh}\psline{->}(-4.8,2.6)(-2.4,1.6)
%\rput(-5,0.4){\small Hydraulique : }\psline{->}(-3.8,0.2)(-2.95,-0.2)
%\rput(-5,0){\small 67,5 TWh}
%\rput(-5,-2.2){\small Autres énergies}\psline{->}(-3.2,-2.2)(-2.4,-1.6)
%\rput(-4.7,-2.8){\small (dont la géothermie) : 31 TWh}
%\end{pspicture}}
%
%\medskip
%
%\begin{enumerate}
%\item 
%	\begin{enumerate}
%		\item Calculer la production totale d'électricité en France en 2014.
%		\item Montrer que la proportion d'électricité produite par les \og Autres énergies (dont la géothermie) \fg{} est environ égale à 5,7\,\%.
%	\end{enumerate}
%\item Le tableau suivant présente les productions d'électricité par les différentes sources d'énergie, en France, en 2013 et en 2014.
%	
%\begin{center}
%\begin{tabularx}{\linewidth}{|m{3cm}|*{4}{>{\centering \arraybackslash}X|}}\cline{2-5}
%\multicolumn{1}{c|}{~}&	Thermique à flamme& Hydrauli\-que &\footnotesize Autres énergies (dont la géothermie) & Nucléaire\\ \hline
%Production en 2013 (en TWh) &43,5 &75,1 &28,1 &403,8\\ \hline
%Production en 2014 (en TWh) &25,8 &67,5 &31 &415,9\\ \hline
%Variation de production entre 2013 et 2014&$- 40,7\,\%$& $-10,1\,\%$& $+ 10,3\,\%$&$+ 3\,\% $\\ \hline
%\end{tabularx}	
%\end{center}
%
%
%
%Alice et Tom ont discuté pour savoir quelle est la source d'énergie qui a le plus augmenté sa production d'électricité. 
%
%Tom pense qu'il s'agit des \og Autres énergies (dont la géothermie) \fg{} et Alice pense qu'il s'agit du \og Nucléaire \fg. 
%
%Quel est le raisonnement tenu par chacun d'entre eux ?
%\item La centrale géothermique de Rittershoffen (Bas Rhin) a été inaugurée le 7 juin 2016. On y a creusé un puits pour capter de l'eau chaude sous pression, à \np{2500}~m de profondeur, à une température de $170$~degrés Celsius.
%	
%\medskip
%	
%\parbox{0.5\linewidth}{Ce puits a la forme du tronc de cône représenté ci-contre.
%
%Les proportions ne sont pas respectées.
%
%On calcule le volume d'un tronc de cône grâce à la formule suivante:
%
%\[V = \dfrac{\pi}{3} \times h \times \left(R^2 + R \times r  + r^2\right)\]
%
%où $h$ désigne la hauteur du tronc de cône, $R$ le rayon de la grande base et $r$ le
%rayon de la petite base.
%
%\textbf{a.} Vérifier que le volume du puits est environ égal à $225$~m$^3$.
%
%\textbf{b.} La terre est tassée quand elle est dans le sol. Quand on l'extrait, elle n'est
%plus tassée et son volume augmente de 30\,\%.
%		
%Calculer le volume final de terre à stocker après le forage du puits.}
%\hfill
%\parbox{0.48\linewidth}{\psset{unit=0.8cm,arrowsize=2pt 4}
%\begin{pspicture}(-3,0)(3,9.5)
%%\psgrid
%\psellipse(0,7)(2.3,0.7)
%\scalebox{1}[0.3]{\psarc[linewidth=1.25pt](0,2.7){1.1}{180}{0}}%
%\scalebox{1}[0.3]{\psarc[linestyle=dashed,linewidth=1.5pt](0,2.7){1.1}{0}{180}}%
%\psline{<->}(-2.3,8)(2.3,8)
%\psline{<->}(-1.1,0.4)(1.1,0.4)
%\psline{<->}(2.4,7)(2.4,0.7)
%\rput(0,0){\small Petite base de 20 cm de diamètre}
%\rput(0,8.4){\small Grande base de 46 cm de diamètre}
%\rput{-90}(2.8,3.85){\small Hauteur : \np{2500}~m}
%\psline[linestyle=dashed](0,0.7)(0,7)
%\psline(1.1,0.8)(1.4,2.4)\psline(-1.1,0.8)(-1.4,2.4)\psline[linestyle=dashed](1.4,2.4)(1.8,4.4)
%\psline(1.8,4.4)(2.3,7)\psline(-1.8,4.4)(-2.3,7)\psline[linestyle=dashed](-1.4,2.4)(-1.8,4.4) 
%\end{pspicture}}
%
%\end{enumerate}





\begin{seance}[Les solides]

\begin{description}
\item[$\square$] Comprendre l’effet d’un déplacement, d’un agrandissement ou d’une réduction sur les longueurs, les aires, les volumes.
\item[$\square$] Utiliser un rapport de réduction ou d’agrandissement (architecture, maquettes), l’échelle d’une carte.
\end{description}
\end{seance}



\Exe

\emph{Cocher la bonne réponse}.

On triple la longueur de l’arête d’un cube. Son volume est ...

\begin{tabular}{cccc}

a. inchangé & b. multiplié par 3 & c. multiplié par 9 & d. multiplié par 27 \\ 

\end{tabular} 


\Exe

\emph{Cocher la bonne réponse}.

Les cônes $\mathscr{C}$ et $\mathscr{C'}$ ont la même base mais la hauteur de cône $\mathscr{C'}$ est la moitié de celle du cône $\mathscr{C}$.

Le volume de $\mathscr{C}$ est

\begin{tabular}{cccc}

a. inchangé & b. le double & c. le quadruple & d. l'octuple 

\end{tabular} 

du volume de $\mathscr{C'}$.
 
\Dnb

\parbox{0.65\linewidth}{La dernière bouteille de parfum de chez Chenal a la forme
d'une pyramide SABC à base triangulaire de hauteur [AS] telle que :

$\bullet~~$ABC est un triangle rectangle et isocèle en A ;

$\bullet~~$AB = 7,5~cm et AS = 15~cm.

\medskip

\begin{enumerate}
\item Calculer le volume de la pyramide SABC. (On arrondira au cm$^3$ près.)
\item Pour fabriquer son bouchon SS$'$MN, les concepteurs ont
coupé cette pyramide par un plan P parallèle à sa base et passant par le point S$'$ tel que SS$'$ = 6~cm.
	\begin{enumerate}
		\item Quelle est la nature de la section plane S$'$MN obtenue ?
		\item Calculer la longueur S$'$N.
	\end{enumerate}
\item Calculer le volume maximal de parfum que peut contenir cette bouteille en cm$^3$.
\end{enumerate}} \hfill
\parbox{0.32\linewidth}{\psset{unit=0.75cm}
\begin{pspicture}(5.5,12)
%\psgrid
\pspolygon(0.4,1.4)(0.4,10.9)(3.2,0.5)%ASB
\psline(0.4,10.9)(5.2,1.4)(3.2,0.5)%SCB
\psline[linestyle=dashed](0.4,1.4)(5.2,1.4)%AC
\psline(0.4,7.1)(1.5,6.7)(2.3,7.1)%S'MN
\psline[linestyle=dashed](0.4,7.1)(2.3,7.1)%S'N
\psframe(0.4,1.4)(0.9,1.9)
\psline(0.9,1.4)(1.4,1.2)(1,1.2)
\uput[u](0.4,10.9){S}\uput[l](0.4,7.1){S$'$}
\uput[dr](1.5,6.7){M}\uput[r](2.3,7.1){N}
\uput[l](0.4,1.4){A}\uput[d](3.2,0.5){B}\uput[r](5.2,1.4){C}
\rput(2.6,1.4){$\circ$}\rput(2,0.9){$\circ$}
\end{pspicture}}

\includegraphics[scale=1]{rapports.eps} 




\begin{seance}[Les solides]

\begin{description}
\item[$\square$]  Utiliser, produire et mettre en relation des représentations de solides et de situations
spatiales.
\item[$\square$] Développer sa vision de l’espace.
\item[$\square$] Mener des calculs impliquant des grandeurs mesurables, avec une formule donnant le volume d’une pyramide, d’un cylindre, d’un cône ou d’une boule.
\end{description}
\end{seance}


\Dnb


Voici les dimensions de quatre solides: 

\begin{itemize}
\item[$\bullet$] Une pyramide de 6 cm de hauteur dont la base est un rectangle de 6 cm de longueur et de 3 cm de largeur. 

\item[$\bullet$] Un cylindre de 2 cm de rayon et de 3 cm de hauteur. 


\item[$\bullet$] Un cône de 3 cm de rayon et de 3 cm de hauteur. 

\item[$\bullet$] Une boule de 2 cm de rayon. 
\end{itemize}

\begin{enumerate}
\item

\begin{enumerate}

\item Représenter approximativement les quatre solides. 

\item Placer les dimensions données sur les représentations. 
\end{enumerate}
\item  Classer ces quatre solides dans l'ordre croissant de leur volume. 
\end{enumerate}




\textit{Quelques formules }: 
$$\dfrac{4}{3}\times \pi\times rayon^3\qquad\hfill\qquad \pi\times rayon^2\times hauteur$$

$$\dfrac{1}{3}\times \pi\times  rayon^2\times hauteur \qquad\hfill\qquad \dfrac{1}{3}\times aire\:de\:la\:base\times hauteur$$


\includegraphics[scale=1]{solides.eps}

\Dnb

\parbox{0.5\linewidth}{Un aquarium a la forme d'une sphère de 10~cm de
rayon, coupée en sa partie haute: c'est une \og calotte
sphérique \fg.

La hauteur totale de l'aquarium est 18 cm.}\hfill
\parbox{0.47\linewidth}{\psset{unit=0.9cm}
\begin{pspicture*}(0,-0.1)(5.6,4.1)
%\psgrid
\psarc(2.5,2.5){2.5}{144}{36}
\psline(0.45,4)(4.55,4)
\psline{<->}(2.5,2.5)(5,2.5)\uput[u](3.75,2.5){$r$}
\psline{<->}(5.3,0)(5.3,4)\rput{90}(5.45,2){$h$}
\end{pspicture*}}

\medskip

\begin{enumerate}
\item Le volume d'une calotte sphérique est donné par la formule :

\[V \dfrac{\pi}{3} \times h^2 \times (3r - h)\]

où $r$ est le rayon de la sphère et $h$ est la hauteur de la calotte sphérique.
	\begin{enumerate}
		\item Prouver que la valeur exacte du volume en cm$^3$ de l'aquarium est $\np{1296}\pi$.
		\item Donner la valeur approchée du volume de l'aquarium au litre près.
	\end{enumerate}
\item On remplit cet aquarium à ras bord, puis on verse la totalité de son contenu dans
un autre aquarium parallélépipédique. La base du nouvel aquarium est un rectangle
de $15$~cm par $20$~cm.

Déterminer la hauteur atteinte par l'eau (on arrondira au cm).

* Rappel: 1 $\ell$ = 1 dm$^3 = \np{1000}$ cm$^3$
\end{enumerate}



\begin{seance}[Probabilités]

\begin{description}
\item[$\square$]  Utiliser, produire et mettre en relation des représentations de solides et de situations
spatiales.
\item[$\square$] Mener des calculs impliquant des grandeurs mesurables, avec une formule donnant le volume d’une pyramide, d’un cylindre, d’un cône ou d’une boule.          
\end{description}
\end{seance}



\Dnb

Romane souhaite préparer un cocktail pour son anniversaire.

\begin{center}
\begin{tabularx}{\linewidth}{|*{2}{>{\centering \arraybackslash}X|}}\hline
Document 1 : Recette du cocktail

Ingrédients pour 6 personnes :&Document 2 : Récipient de Romane\\ 

\begin{pspicture}(5,4)
\uput[r](0,3.75){$\bullet~~$60 cl de jus de mangue}
\uput[r](0,3.25){$\bullet~~$30 cl de jus de poire}
\uput[r](0,2.75){$\bullet~~$12 cl de jus de citron vert}
\uput[r](0,2.25){$\bullet~~$12 cl de sirop de cassis}
\end{pspicture}&\psset{unit=1cm}
\begin{pspicture}(5,4)
\psellipse(2.5,3)(2,0.4)
\psarc(2.5,3){2cm}{-180}{0}
\end{pspicture}\\
Préparation : &\\
Verser les différents ingrédients dans un récipient et remuer.

Garder au frais pendant au moins 4~h.&On considère qu'il a la forme d'une
demi-sphère de diamètre 26 cm.\\ \hline
\end{tabularx}
\end{center}
\emph{Rappels :}

$\bullet~~$Volume d'une sphère : $V = \dfrac{4}{3}\pi r^3$

$\bullet~~$1~L = 1~dm$^3$ = \np{1000}~cm$^3$

\medskip

Le récipient choisi par Romane est-il assez grand pour préparer le cocktail pour 20
personnes ?

\medskip

\textbf{Il est rappelé que, pour l'ensemble du sujet, les réponses doivent être justifiées.\\
Il est rappelé que toute trace de recherche sera prise en compte dans la
correction.}





\begin{seance}[Les solides]

\begin{description}
\item[$\square$]  Utiliser, produire et mettre en relation des représentations de solides et de situations
spatiales.
\item[$\square$] Développer sa vision de l’espace.
\item[$\square$] Mener des calculs impliquant des grandeurs mesurables, avec une formule donnant le volume d’une pyramide, d’un cylindre, d’un cône ou d’une boule.
\end{description}
\end{seance}



\Dnb



Léo a ramassé des fraises pour faire de la confiture.

\medskip

\begin{enumerate}
\item Il utilise les proportions de sa grand-mère : 700 g de sucre pour 1 kg de fraises.

Il a ramassé 1,8 kg de fraises. De quelle quantité de sucre a-t-il besoin ?
\item  Après cuisson, Léo a obtenu 2,7 litres de confiture.

Il verse la confiture dans des pots cylindriques de 6 cm de diamètre et de
12 cm de haut, qu'il remplit jusqu'à 1 cm du bord supérieur.

Combien pourra-t-il remplir de pots ?

\emph{Rappels} : 1 litre = 1000 cm$^3$ \quad 
\emph{Volume d'un cylindre}  $= \pi \times R^2 \times h$.
\item  Il colle ensuite sur ses pots une étiquette rectangulaire de fond blanc qui recouvre toute la surface latérale du pot.
	\begin{enumerate}
		\item Montrer que la longueur de l'étiquette est d'environ 18,8 cm.
		\item Dessiner l'étiquette à l' échelle $\dfrac{1}{3}$.
	\end{enumerate} 
 \end{enumerate}




\Dnb

\emph{Magic The Gathering} est un jeu de cartes. Aurel voudrait participer à un tournoi le week-end prochain. Il décide de s'acheter de nouvelles cartes sur Internet.

L'annexe 2  est une capture d'écran d'un tableau obtenu à l'aide d'un tableur. Il permet de calculer le coût des achats d'Aurel.

\medskip

\begin{enumerate}
\item Quelle formule peut-on saisir dans la cellule D2 avant de l'étirer sur la colonne D ?
\item Sur l'annexe 2, compléter chaque cellule de la colonne D par les prix obtenus.
\item Aurel range ses cartes dans une boîte à chaussures. Il les place à plat au fond de la boîte comme indiqué sur la figure de façon à former des piles.

On dispose des informations suivantes :

\begin{center}
\begin{tabularx}{\linewidth}{*{2}{>{\centering \arraybackslash}X}}
Dimensions de la boîte &Dimensions de la carte\\
\psset{unit=1cm}
\begin{pspicture}(5,5)
\psframe(0,0.3)(2.3,1.6)
\psline(2.3,0.3)(4.6,2.6)(4.6,3.9)(2.3,1.6)
\psline(4.6,3.9)(2.3,3.9)(0,1.6)
\psline[linestyle=dotted](0,0.3)(2.3,2.6)(2.3,3.9)
\psline[linestyle=dotted](2.3,2.6)(4.6,2.6)
\pspolygon[fillstyle=solid,fillcolor=lightgray](2.3,0.3)(2.9,0.9)(2.4,0.9)(1.8,0.3)
\pspolygon[fillstyle=solid,fillcolor=lightgray](2.9,0.9)(3.5,1.5)(3,1.5)(2.4,0.9)
\pspolygon[fillstyle=solid,fillcolor=lightgray](1.8,0.3)(2.4,0.9)(1.9,0.9)(1.3,0.3)
\psframe(0,0.3)(2.3,1.6)
\psline{<->}(0,0.2)(2.3,0.2)\uput[d](1.15,0.2){24,5 cm}
\psline{<->}(2.5,0.3)(4.8,2.7)\rput{46}(3.9,1.5){37,5 cm}
\psline{<->}(4.8,2.6)(4.8,3.9)\rput{90}(5,3.25){17 cm}
\end{pspicture}&
\psset{unit=1cm}
\begin{pspicture}(-2,0)(5,5)
\psframe(0.3,0.3)(2.6,4.1)
\rput(1.45,3.5){\textbf{\scriptsize Magic The Gathering}}
\psline{<->}(0.3,0.1)(2.6,0.1)\uput[d](1.45,0.1){6,2 cm}
\psline{<->}(2.8,0.3)(2.8,4.1)\rput{90}(3,2.2){8,7 cm}
\end{pspicture}
\\
\end{tabularx}
\end{center}

\medskip

Quel est alors le nombre maximum de piles que peut contenir cette boîte ? Justifier.
\end{enumerate}

\vspace{0,5cm}

\begin{tabularx}{\linewidth}{|c|*{4}{>{\centering \arraybackslash}X|}}\hline
	&A 							&B 							&C		&D\\ \hline
1	&\textbf{Nouvelles cartes}	&\textbf{Quantité}&\textbf{Prix unitaire (en F)}& \textbf{Prix (en F)}\\ \hline
2	&\psset{unit=1cm}
\begin{pspicture}(2.2,3.2)
\psframe(2.2,3)\rput(1.1,2.6){\tiny{\textbf{Magic The Gathering 1}}}
\end{pspicture}					&2							&322	&\ldots\\ \hline
3	&\psset{unit=1cm}
\begin{pspicture}(2.2,3.2)
\psframe(2.2,3)\rput(1.1,2.6){\tiny{\textbf{Magic The Gathering 2}}}
\end{pspicture}					&3							&112				&\ldots\\ \hline
4	&\psset{unit=1cm}
\begin{pspicture}(2.2,3.2)
\psframe(2.2,3)\rput(1.1,2.6){\tiny{\textbf{Magic The Gathering 3}}}
\end{pspicture}
								&4							&480				&\ldots\\ \hline
5	&\multicolumn{3}{r|}{Montant de la commande :}						&\np{2900}\\ \hline
6	&\multicolumn{3}{r|}{Frais de transport : + 10\,\% de la commande}	&\ldots\\ \hline
7	&\multicolumn{3}{r|}{Montant total :} 								&\ldots\\ \hline
\end{tabularx}

\end{document}