\begin{titre}[Représentation et utilisation des solides]

\Titre{Dessiner différentes vues d'un parallélépipède rectangle. (face, perspective cavalière, coupe)}{1,5}
\end{titre}

 

\begin{CpsCol}
\textbf{Représenter l'espace}
\begin{description}
\item[$\square$] Dessiner différentes vues d'un parallélépipède rectangle. (face, perspective cavalière, coupe)
\end{description}
\end{CpsCol}

\mini{
\Rec{1}{RepS-31}

\Fl{1}{RepS-32}
}{
\Exo{0}{RepS-40}

\Exo{1}{RepS-33}

\Exo{0}{RepS-37}

\Exo{1}{RepS-39}

\Exo{0}{RepS-34}
}




\Exo{1}{RepS-41}







\Exo{1}{RepS-36}




%\begin{autoeval}
%\begin{tabular}{p{12cm}p{0.5cm}p{0.5cm}p{0.5cm}p{1cm}}
%\textbf{Compétences visées} &  M I & MF & MS  & TBM \vcomp \\ 
%Dessiner différentes vues d'un parallélépipède rectangle. (face, perspective cavalière, coupe) & $\square$ & $\square$  & $\square$ & $\square$ \vcomp \\
%  
%\end{tabular}
%{\footnotesize MI : maitrise insuffisante ; MF = Maitrise fragile ; MS = Maitrise satisfaisante ; TBM = Très bonne maitrise}
% 
%\end{autoeval}