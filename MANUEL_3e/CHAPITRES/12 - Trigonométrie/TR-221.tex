
\subsection{Connexion au support d'activité}


\begin{enumerate}
\item Se connecter à sacado.fr avec ses identifiants
\item Ouvrir le thème : Trigonométrie
\item Sélectionner l'activité : Découvrir le cosinus, le sinus, la tangente
\end{enumerate}


\subsection{Activité}

Le triangle $ADE$ est rectangle en $D$. On souhaite établir une relation entre l'angle $\widehat{DAE}$ et les cotés du triangle. 


\subsubsection{ Une figure dynamique}

\begin{enumerate}
\item Faire varier le point $E$. Vérifier que les longueurs des 3 cotés varient.
\item Faire varier, avec le curseur, l'angle $\alpha$. Vérifier que la longueur $AE$ ne varie pas.
\item Quel ensemble décrit le point $E$ ? \point{1}
\item Qu'est ce qu'une figure dynamique ? \point{1}
\end{enumerate}

\subsubsection{Une approche du cosinus}


\begin{enumerate}
\item Cocher la case du cosinus.
\item Faire varier le point $E$. Comment réagit la valeur du cosinus de $\widehat{DAE}$ ? \point{1}
\item Faire varier l'angle $\alpha$. Comment réagit la valeur du cosinus de $\widehat{DAE}$ ? \point{1}

\begin{Conjecture}
Le cosinus d'un angle est le quotient de la longueur du coté \ldots\ldots\ldots\ldots\ldots\ldots par la longueur de  \ldots\ldots\ldots\ldots\ldots\ldots\ldots\ldots
\end{Conjecture}


\item Quelle est la plus grande et la plus petite valeur du cosinus d'un angle aigu ? 
\end{enumerate}

\subsubsection{Une approche du sinus}


\begin{enumerate}
\item Décocher la case du cosinus et cocher la case du sinus.
\item Faire varier le point $E$. Comment réagit la valeur du sinus de $\widehat{DAE}$ ? \point{1}
\item Faire varier, avec le curseur, l'angle $\alpha$. Comment réagit la valeur du sinus de $\widehat{DAE}$ ? \point{1}
\begin{Conjecture}

Le sinus d'un angle est le quotient de la longueur du coté \ldots\ldots\ldots\ldots\ldots\ldots
par la longueur de  \ldots\ldots\ldots\ldots\ldots\ldots\ldots\ldots
\end{Conjecture}

\item Quelle est la plus grande et la plus petite valeur du sinus d'un angle aigu ? 
\end{enumerate}


\subsubsection{Une approche de la tangente}


\begin{enumerate}

\item Décocher la case du sinus et cocher la case de la tangente.
\item Faire varier le point $E$. Comment réagit la valeur de la tangente de $\widehat{DAE}$ ? \point{1}
\item Faire varier, avec le curseur, l'angle $\alpha$. Comment réagit la valeur de la tangente de $\widehat{DAE}$ ? \point{1}


\begin{Conjecture}
La tangente  d'un angle est le quotient de la longueur du coté \ldots\ldots\ldots\ldots\ldots\ldots par  la longueur du coté \ldots\ldots\ldots\ldots\ldots\ldots 
\end{Conjecture}


\item Quelle est la plus grande et la plus petite valeur de la tangente d'un angle aigu ? 
\end{enumerate}

\begin{Def}
Le  cosinus, le  sinus et la tangente  d'un angle sont des rapports de deux longueurs. 

\textbf{Réciproquement}, si on connait, un sinus ou un sinus ou une tangente, on peut alors connaitre un angle.
\end{Def}

\begin{Mt}
\begin{description}
\item[•] On utilise la touche \touche{Acos} ou \touche{Arccos} pour obtenir un angle à partir d'un cosinus.
\item[•] On utilise la touche \touche{Asin} ou \touche{Arcsin} pour obtenir un angle à partir d'un sinus.
\item[•] On utilise la touche \touche{Atan} ou \touche{Arctan} pour obtenir un angle à partir d'une tangente.
\end{description}
\end{Mt}

\begin{minipage}{0.48\linewidth}

\subsection{Démonstration}


\definecolor{yqyqyq}{rgb}{0.5019607843137255,0.5019607843137255,0.5019607843137255}
\definecolor{aqaqaq}{rgb}{0.6274509803921569,0.6274509803921569,0.6274509803921569}
\definecolor{qqwuqq}{rgb}{0.,0.39215686274509803,0.}
\definecolor{uuuuuu}{rgb}{0.26666666666666666,0.26666666666666666,0.26666666666666666}
\definecolor{ffqqqq}{rgb}{1.,0.,0.}
\begin{tikzpicture}[line cap=round,line join=round,>=triangle 45,x=1.0cm,y=1.0cm]
\clip(-2.280510827889474,-0.723406479627207) rectangle (6.199728096948752,6.946060248262661);
\draw [shift={(-1.8203428242160824,0.)},line width=2.pt,color=ffqqqq,fill=ffqqqq,fill opacity=0.8862745098039215] (0,0) -- (0.:0.6573828623905602) arc (0.:40.:0.6573828623905602) -- cycle;
\draw[line width=2.pt,color=qqwuqq,fill=qqwuqq,fill opacity=0.28999999165534973] (2.398914908076793,0.4648398798321881) -- (1.9340750282446049,0.46483987983218816) -- (1.9340750282446049,0.) -- (2.398914908076793,0.) -- cycle; 
\draw[line width=2.pt,color=qqwuqq,fill=qqwuqq,fill opacity=0.1899999976158142] (5.608083520797247,0.4648398798321881) -- (5.143243640965059,0.46483987983218816) -- (5.143243640965059,0.) -- (5.608083520797247,0.) -- cycle; 
\draw[line width=4.pt,color=ffqqqq,fill=ffqqqq,fill opacity=0.8899999856948853] (7.120064104295536,6.310590147951786) -- (11.064361278638897,6.310590147951786);
\draw (-3.1789340731565727,13.43223782384952) node[anchor=north west] {$AD=4.22\quad  \quad AE =5.51 \quad  \quad ED=3.54$};
\draw [line width=2.pt,color=aqaqaq] (-1.8203428242160824,0.)-- (2.398914908076793,0.);
\draw [line width=2.pt,color=aqaqaq] (-1.8203428242160824,0.)-- (2.398914908076793,3.540377607008838);
\draw [line width=2.pt,color=aqaqaq] (2.398914908076793,3.540377607008838)-- (2.398914908076793,0.);
\draw [line width=2.pt,color=aqaqaq] (-1.8203428242160824,0.)-- (5.608083520797247,0.);
\draw [line width=2.pt,color=yqyqyq] (-1.8203428242160824,0.)-- (5.608083520797246,6.233189806328275);
\draw [line width=2.pt,color=yqyqyq] (5.608083520797246,6.233189806328275)-- (5.608083520797247,0.);
\begin{scriptsize}
\draw [color=black] (-1.8203428242160824,0.)-- ++(-2.5pt,0 pt) -- ++(5.0pt,0 pt) ++(-2.5pt,-2.5pt) -- ++(0 pt,5.0pt);
\draw[color=black] (-1.973732158773879,0.4927518157953291) node {$A$};
\draw [color=black] (5.608083520797247,0.)-- ++(-2.5pt,0 pt) -- ++(5.0pt,0 pt) ++(-2.5pt,-2.5pt) -- ++(0 pt,5.0pt);
\draw[color=black] (5.695734569115989,-0.361845905312399) node {$B$};
\draw [fill=ffqqqq] (8.873085070670363,6.310590147951786) circle (2.5pt);
\draw [color=black] (2.398914908076793,3.540377607008838)-- ++(-2.5pt,0 pt) -- ++(5.0pt,0 pt) ++(-2.5pt,-2.5pt) -- ++(0 pt,5.0pt);
\draw[color=black] (1.9705650155694816,3.933055462305927) node {$E$};
\draw [color=uuuuuu] (2.398914908076793,0.)-- ++(-2.0pt,0 pt) -- ++(4.0pt,0 pt) ++(-2.0pt,-2.0pt) -- ++(0 pt,4.0pt);
\draw[color=uuuuuu] (2.364994733003818,-0.29610761907334304) node {$D$};
\draw [color=uuuuuu] (5.608083520797246,6.233189806328275)-- ++(-2.0pt,0 pt) -- ++(4.0pt,0 pt) ++(-2.0pt,-2.0pt) -- ++(0 pt,4.0pt);
\draw[color=uuuuuu] (5.761472855355045,6.584499673947853) node {$C$};
\end{scriptsize}
\end{tikzpicture}

\subsubsection{Le cosinus}

Démontrer, à l'aide du théorème de Thalès appliqué au triangle $ADE$ et $ABC$, que $$\frac{AD}{AE}=\frac{AB}{AC}$$

\begin{Th}
Dans tout triangle rectangle $ABC$, $$\cos \widehat{BAC} = \frac{ \text{ longueur du coté }\ldots\ldots\ldots\ldots\ldots\ldots}{ \text{ longueur de } \ldots\ldots\ldots\ldots\ldots\ldots\ldots}$$
\end{Th}

\end{minipage}
\hfill
\begin{minipage}{0.48\linewidth}
\subsubsection{Le sinus}

Démontrer, à l'aide du théorème de Thalès appliqué au triangle $ADE$ et $ABC$, que $$\frac{DE}{AE}=\frac{BC}{AC}$$

\begin{Th}
Dans tout triangle rectangle $ABC$, $$\sin \widehat{BAC} = \frac{ \text{ longueur du coté } \ldots\ldots\ldots\ldots\ldots\ldots}{ \text{ longueur de  } \ldots\ldots\ldots\ldots\ldots\ldots\ldots}$$
\end{Th}


\subsubsection{La tangente}

Démontrer, à l'aide du théorème de Thalès appliqué au triangle $ADE$ et $ABC$, que $$\frac{ED}{AD}=\frac{BC}{AB}$$

\begin{Th}
Dans tout triangle rectangle $ABC$, $$\tan \widehat{BAC} = \frac{ \text{ longueur du coté } \ldots\ldots\ldots\ldots\ldots\ldots}{ \text{ longueur du coté } \ldots\ldots\ldots\ldots\ldots\ldots\ldots}$$
\end{Th}
\end{minipage}

\begin{Rq}
Le  cosinus, le  sinus et la tangente permettent de calculer des longueurs et des angles dans un triangle rectangle.
\end{Rq}