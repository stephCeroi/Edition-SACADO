  
\begin{itemize}
\item Le théorème de Pythagore peut être énoncé de la façon suivante: \og \textbf{Si un triangle est rectangle, alors le carré de l'hypoténuse est égal à la somme des carrés des deux autres côtés du triangle} \fg{} 
\item Ce théorème sert à calculer les longueurs d'un côté connaissant les deux autres dans un triangle rectangle.
\item Pour l'utiliser, il faut savoir que le triangle est rectangle et repérer l'hypoténuse ou l'angle droit.
\end{itemize}
