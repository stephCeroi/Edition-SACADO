\chapter{Arithmétique}
{https://sacado.xyz/qcm/parcours_show_course/0/117129}
{
 \begin{CpsCol}
\textbf{Les savoir-faire du parcours}
 \begin{itemize}
 \item Utiliser les notations et le vocabulaire fonctionnels.
 \item Passer d'un mode de représentation d'une fonction à un autre.
 \item Déterminer, à partir de tous les modes de représentation, l'image d'un nombre.
 \item Déterminer un antécédent à partir d'une représentation graphique ou d'un tableau de valeurs d'une fonction.
 \item Déterminer de manière algébrique l'antécédent par une fonction, dans des cas se ramenant à la résolution d'une équation du premier degré.
 \end{itemize}
 
 \end{CpsCol}
 
 \begin{His}
 
\begin{wrapfigure}[15]{r}{3.6cm}
\vspace{-7mm}
\includegraphics[scale=0.5]{FIG//liebniz.jpg}
\begin{center}
 G. W. \textsc{Leibniz}
 \end{center} 
\end{wrapfigure}

\textbf{\textsc{Gottfried Wilhelm Leibniz}},  dans  deux  textes  en  latin  des  « \textit{Acta  cruditorum}  »  de  $1673$  et  $1692$,  use  du vocable  «  \textit{functio, functiones} », mot  forgé  à  partir  du  participe  passé  du  verbe \textit{fungor}   (accomplir,   remplir   une  charge).  Puis  il  francise  le  mot.  Dans  le  « \textit{Journal des sçavans} » de $1694$, il écrit (en   français)   :   «   entre   deux  fonctions   quelconques de la ligne AC... ». Et plus  loin  suit  une  définition  en  lien  avec  la  question    étudiée    alors    :  " J'appelle fonction  toutes  les  portions  de  lignes qu'on  fait  en  menant... ".  Ailleurs  il  désigne,  toujours  en  français,  l'abscisse,  l'ordonnée,  la  corde  comme  fonctions  d'une courbe.

\vspace{0.4cm}
Le terme fut alors repris par d'autres. En juillet $1698$,  Leibniz écrit à  Jean Bernoulli :   «   J'ai   plaisir   à   vous   voir   
employer  le  terme  fonction  dans  mon  sens  ».  Et  Bernoulli  de  lui  répondre  de  Groningen  au  mois  d'aout  «  Pour  noter  
une   fonction   d'une   certaine   quantité indéterminée $x$, j'aime  utiliser  la majuscule  correspondante  $X$  ou  la lettre  grecque $\zeta$. On peut voir immédiatement de quelle indéterminée dépend la fonction. » 

\vspace{0.4cm}

Le  concept  perd  alors,  petit  à  petit,  son  caractère géométrique immédiat. Dans  les  « \textit{ Mémoires  de  l'Académie  des 
Sciences} », en $1718$, Bernoulli écrit : 

« DÉFINITION  :    On  appelle  ici FONCTION  d'une  grandeur  variable, une   quantité   composée   de   quelque   manière  que  ce  soit  de  cette  grandeur  variable et de constantes. » 

\vspace{0.4cm}

Euler prend la suite et, dans une note de l’Académie de Saint Petersbourg ($1734$), il  introduit  la  notation  $f\left(x \right)$ pour : « un fonction arbitraire de $x$».
 \end{His}

}




\begin{pageCours}




\section{Notion de fonction}


\begin{DefT}{En fonction de\index{En fonction de}}
Lorsqu'une relation associe deux quantités, on dit que l'on peut exprimer une quantité \textbf{en fonction de} l'autre. \\ En général, on peut alors établir une formule qui lie ces deux quantités.
\end{DefT}

\begin{Ex}
1 pain au chocolat coûtent 1,45 \euro{} donc $n$ pains au chocolats coûtent $1,45n$. On peut dire que le prix $p$ de $n$ pains au chocolat s'obtient par la formule $p=1,45n$. Le prix est \textbf{en fonction du} nombre de pains au chocolats.
\end{Ex}


\begin{DefT}{Fonction}
Une \textbf{fonction} \index{Fonction} est un procédé qui à un nombre donné associe un unique nombre. La fonction est explicitée par une expression littérale en fonction de la variable.
 
On écrit $f : x \longmapsto f(x)$ et on lit $f$ est la fonction qui à $x$ associe le nombre $f(x)$. $x$ est appelée la \textit{variable} \index{variable} de la fonction.
\end{DefT}


\begin{Ex}
Un carré a pour coté $x$. Son périmètre est $\mathscr P = 4x$. On préfère écrire $\mathscr{P}(x) = 4x$. $x$ est la variable car pour chaque valeur de $x$ positive, on obtient une valeur du périmètre. Le périmètre est en fonction de $x$.
\end{Ex}


\begin{Rq} 
Une fonction peut être définie de 4 façons : Une expression, un graphique, un algorithme ou un tableau.
\end{Rq}


 
\section{Image, antécédent}


\begin{DefT}{Image et antécédent}
L'\textbf{image} \index{Image} d'un nombre par $f$ est l'unique nombre obtenu après le procédé calculatoire de la fonction.\\ Lorsque $f : a \longmapsto b$, on dit que $b$ est l'image de $a$ par $f$. On note alors $f(a)=b$. \\
$a$ est appelé un \index{Antécédent} \textbf{antécédent} de $b$ par $f$.
\end{DefT}

\begin{Ex}
Un triangle équilatéral de coté 4 cm a un périmètre égal à 12 cm.

On peut alors dire que 12 est l'\textbf{image} de 4 par la fonction $f$, ou $f : 4 \mapsto 12$ ou encore l'\textbf{image} de 4 par la fonction $f$ est $12$. 

Mais aussi : $4$ est un antécédent de 12 par $f$.

Plus généralement, $f : x \mapsto 3x$. L'expression de la fonction est ou $f(x)=3x$. 

$f(x)$ est l'image de $x$ par la fonction $f$.\\
La fonction $f$ est la fonction qui a une longueur du coté d'un triangle équilatéral associe son périmètre.
\end{Ex}

\begin{Ety}
\textbf{Antécédent} est composé de anté - cédent : qui vient avant le procédé. \\Un antécédent vient donc avant la flèche qui symbolise  le procédé. Antécédent $\mapsto$  image.
\end{Ety}
 





\end{pageCours} 
\begin{pageAD} 
 

\Sf{Utiliser les notations et le vocabulaire fonctionnels.}
 
\begin{ExoCad}{Calculer.}{1234}{0}{0}{0}{0}{0}

\begin{enumerate}[leftmargin=*]
\item Exprimer le périmètre $\mathscr P$ d'un carré \textbf{en fonction de} la longueur d'un coté $c$. \point{1}
\item La largeur d'un rectangle est 3 cm. Exprimer le périmètre $\mathscr P$ de ce rectangle \textbf{en fonction de} sa longueur $L$.\point{2}
\end{enumerate}
\end{ExoCad}

 

\begin{ExoCad}{Calculer.}{1234}{0}{0}{0}{0}{0}
Un rectangle a pour dimension $x$ cm de longueur sur $x-5$ cm de largeur. 
\begin{enumerate}[leftmargin=*]
\item Quelle est la valeur la plus petite pour $x$ ? \point{2}
\item Déterminer l'aire $\mathcal{A}$ de ce rectangle en fonction de $x$.  \point{1}
\item Déterminer \textbf{l'image} de 7 par $\mathcal{A}$.  \point{1}
\end{enumerate}
\end{ExoCad}

\begin{ExoCad}{Calculer.}{1234}{0}{0}{0}{0}{0}
 \input{CHAPITRES/2_notions_de_fonctions/NF-18}
\end{ExoCad}


\begin{ExoCad}{Calculer.}{1234}{0}{0}{0}{0}{0}

Traduire les phrases suivantes par une expression mathématique et les expression mathématiques par une phrase.

\begin{enumerate}[leftmargin=*]
\item $5$ est l'image de $2$ par la fonction $f$. \point{1}
\item L'image de $-3$ par la fonction $g$ est égale à $6$.\point{1}
\item $6$ est un antécédent de $0$ par $h$. \point{1}
\item $f(3)=5$. \point{1}
\item $f(-6)=1$.\point{1}
\end{enumerate}

\end{ExoCad}



\end{pageAD}


%%%%%%%%%%%%%%%%%%%%%%%%%%%%%%%%%%%%%%%%%%%%%%%%%%%%%%%%%%%%%%%%%%%
%%%%  Niveau 1
%%%%%%%%%%%%%%%%%%%%%%%%%%%%%%%%%%%%%%%%%%%%%%%%%%%%%%%%%%%%%%%%%%%
\begin{pageParcoursu} 

 %%%%%%%%%%%%%%%%%%%%%%%%%%%
 

\begin{ExoCu}{Communiquer.}{1234}{0}{0}{0}{0}{0}

Traduire chaque expression mathématique par une phrase contenant le mot \textbf{image}.

\begin{enumerate}[leftmargin=*]
\item $f(3)=5$. \point{1}
\item $f(-6)=1$.\point{1}
\item $f(a)=b$. \point{1}
\end{enumerate}
\end{ExoCu}


\begin{ExoCu}{Communiquer.}{1234}{0}{0}{0}{0}{0}

Traduire chaque expression mathématique par une phrase contenant le mot \textbf{antécédent}.

\begin{enumerate}[leftmargin=*]
\item $f(3)=5$. \point{1}
\item $f(-6)=1$.\point{1}
\item $f(a)=b$. \point{1}
\end{enumerate}
\end{ExoCu}


 \begin{ExoCu}{Modéliser. Calculer.}{1234}{0}{0}{0}{0}{0}
 \input{CHAPITRES/2_notions_de_fonctions/NF-24}
\end{ExoCu}



 
\end{pageParcoursu}

  
%%%%%%%%%%%%%%%%%%%%%%%%%%%%%%%%%%%%%%%%%%%%%%%%%%%%%%%%%%%%%%%%%%%
%%%%  Niveau 2
%%%%%%%%%%%%%%%%%%%%%%%%%%%%%%%%%%%%%%%%%%%%%%%%%%%%%%%%%%%%%%%%%%%



\begin{pageParcoursd} 
 
%%%%%%%%%%%%%%%%%%%%%%%%%%%%%%%%%%%%%%%%%%%%%%%%%%%%%%%%%%%%%%%%%%%

\begin{ExoCd}{Calculer.}{1234}{0}{0}{0}{0}{0}
 Soit $f$ la fonction définie par $f(x)=\dfrac{x+5}{x^2}$
\begin{enumerate}
\item Calculer l'image de $4$ par $f$ \point{1}
\item Calculer $f \left(   \dfrac{1}{2}  \right) $ \point{1}
\item Calculer $f \left( \dfrac{2}{3}  \right)$ \point{1}
\end{enumerate} 
\end{ExoCd}
 

\begin{ExoCd}{Modéliser. Calculer.}{1234}{0}{0}{0}{0}{0}

Voici un algorithme. 

\begin{tabular}{|l|}
\hline 
Choisir un nombre \\ 
Ajouter $5$ \\ 
Calculer le carré de la somme obtenue \\ 
\hline  
\end{tabular} 

  

\begin{enumerate}
\item Complète le tableau

\begin{tabular}{|c|>{\centering\arraybackslash}p{2cm}|>{\centering\arraybackslash}p{2cm}|>{\centering\arraybackslash}p{2cm}|>{\centering\arraybackslash}p{2cm}|>{\centering\arraybackslash}p{2cm}|}
\hline 
$x$ & $-1$ & $0$ & $\dfrac{1}{3}$ & $3$ & $a$ \\ 
\hline 
$f(x)$ &  &  &   & &   \\ 
\hline 
\end{tabular} 

\vspace{0.2cm}
\item Donner l'image de $0$ par $f$ \point{1}
\item Donner un antécédent de $64$ par $f$ \point{1}
\item Exprimer $f(x)$ en fonction de $x$. \point{1}
\end{enumerate} 
\end{ExoCd}
 
 
\begin{ExoCd}{Calculer.}{1234}{0}{0}{0}{0}{0}


Monsieur Philibert voyage un mois à travers Costa Rica. A l'arrivée à l'aéroport, il reçoit un sms sur son smartphone. 

\begin{center}
\begin{description}
\item  0,85 \euro{} par sms
\item  1,95 \euro{} par appel
\end{description}
\end{center}

Pour bénéficier de ces tarifs, il paie un abonnement mensuel de 9,90 \euro{} à son opérateur français.

\begin{enumerate}[leftmargin=*]
\item Quel est le prix que M. Philibert va payer pour 10 appels passés durant son voyage ? \point{3}
\item 
\begin{enumerate}
\item Exprimer $f(n)$ qui détermine le coût de $n$ appels dans le mois durant lequel il visite au Costa Rica. \point{2}
\item Calculer alors l'image de 10 par $f$. \point{1}
\end{enumerate}
\end{enumerate}

\end{ExoCd}

 \begin{ExoCd}{Calculer.}{1234}{0}{0}{0}{0}{0}
Le volume d'une boule s'exprime en fonction de son rayon par $\mathcal{V}(r) = \dfrac{4}{3} \pi\times r^3$. 

Calculer le volume d'une boule de rayon $5$ cm. 

\point{3}
\end{ExoCd}
 

\begin{ExoCd}{Calculer.}{1234}{0}{0}{0}{0}{0}
 \input{CHAPITRES/2_notions_de_fonctions/NF-19}
\end{ExoCd}
 
 
 
\begin{ExoCd}{DNB 2022 - Représenter.}{1234}{0}{0}{0}{0}{0}

\begin{minipage}{0.3\linewidth}

Voici un extrait de feuille de calcul sur un tableur.

\begin{tabular}{|c|c|c|c|}
\hline 
  & \texttt{A} & \texttt{B} & \texttt{C} \\ 
\hline 
1 & $x$ & $-2$ & $-1$ \\ 
\hline 
2 & $f(x)$ &   &   \\ 
\hline 
\end{tabular} 
\end{minipage}
\hfill
\begin{minipage}{0.65\linewidth}
Dans cette feuille de calcul extraite d’un tableur, entourer la formule à saisir dans la cellule \texttt{B2} avant de l'étirer vers la droite.

\begin{tabular}{ c c c }
\texttt{=2*A1+3} & \texttt{=2*B1+3} &  \texttt{=2*(-2)+3}  \\ 
\end{tabular}
\end{minipage}
\end{ExoCd}
 

 
 
 
 
 \begin{ExoCd}{Calculer.}{1234}{0}{0}{0}{0}{0}

\begin{minipage}{0.5\linewidth}
On considère le programme de calcul ci-contre :
\end{minipage}
\begin{minipage}{0.5\linewidth}

\begin{tabular}{c}
 
Nombre de départ \\ 
 $\Downarrow$ \\ 
Soustraire 7 \\ 
Multiplier par 5. \\ 
Soustraire le double du nombre de départ\\  
$\Downarrow$ \\ 
Résultat\\ 
 
\end{tabular} 


\end{minipage}

\begin{enumerate}
\item Montrer que si le nombre de départ est $10$, le résultat obtenu est $-5$. \point{3}
\item On note $x$ le nombre de départ auquel on applique ce programme de calcul. 
Parmi les expressions suivantes, quelle est celle qui correspond au résultat du programme de calcul ? 


\begin{tabular}{cc}

Expression A : $x-7 \times 5-2x$ &Expression C : $5(x-7)-2x$ \\
Expression B : $5(x -7)- x^2$& Expression D : $5x-7-2x$\\
\end{tabular} 

\end{enumerate}

\end{ExoCd} 
 
 
 
\end{pageParcoursd}

%%%%%%%%%%%%%%%%%%%%%%%%%%%%%%%%%%%%%%%%%%%%%%%%%%%%%%%%%%%%%%%%%%%
%%%%  Niveau 3
%%%%%%%%%%%%%%%%%%%%%%%%%%%%%%%%%%%%%%%%%%%%%%%%%%%%%%%%%%%%%%%%%%%
\begin{pageParcourst}

%%%%%%%%%%%%%%%%%%%%%%%%%%%%%%%%%%%%%%%%%%%%%%%%%%%%%%%%%%%%%%%%%%%
\begin{ExoCt}{Représenter.}{1234}{2}{0}{0}{0}{0}

 
 \input{CHAPITRES/2_notions_de_fonctions/NF-4}
 
\end{ExoCt}

%%%%%%%%%%%%%%%%%%%%%%%%%%%%%%%%%%%%%%%%%%%%%%%%%%%%%%%%%%%%%%%%%%%
\begin{ExoCt}{Modéliser. Calculer.}{1234}{2}{0}{0}{0}{0}
 
 
 \input{CHAPITRES/2_notions_de_fonctions/NF-10}
 

\end{ExoCt}
 

 
\end{pageParcourst}



%%%%%%%%%%%%%%%%%%%%%%%%%%%%%%%%%%%%%%%%%%%%%%%%%%%%%%%%%%%%%%%%%%%
%%%%  Auto
%%%%%%%%%%%%%%%%%%%%%%%%%%%%%%%%%%%%%%%%%%%%%%%%%%%%%%%%%%%%%%%%%%%


%%%%%%%%%%%%%%%%%%%%%%%%%%%%%%%%%%%%%%%%%%%%%%%%%%%%%%%%%%%%%%%%%%%
\begin{pageAuto} 


\begin{ExoAuto}{Raisonner.}{1234}{2}{0}{0}{0}{0}

  \input{CHAPITRES/2_notions_de_fonctions/NF-7}

%%%%%%%%%%%%%%%%%%%%%%%%%%%%%%%%%%%%%%%%%%%%%%%%%%%%%%%%%%%%%%%%%%%
\end{ExoAuto}



 
%%%%%%%%%%%%%%%%%%%%%%%%%%%%%%%%%%%%%%%%%%%%%%%%%%%%%%%%%%%%%%%%%%%
\begin{ExoAuto}{Raisonner.}{1234}{2}{0}{0}{0}{0}

 Une boutique en ligne vend des photos et affiche les tarifs suivants :

 \begin{tabular}{|c|c|}
 \hline 
 Nombre de photos commandées & Prix à payer \\ 
 \hline 
De 1 à 100 photos & 0,17 \euro{} par photo \\ 
 \hline 
Plus de 100 photos & Plus de 100 photos 17 \euro{} pour l’ensemble des 100 premières photos
\\  
  & et 0,13 \euro{} par photo supplémentaire  \\ 
 \hline 
 \end{tabular} 

\begin{enumerate}
\item 

\begin{enumerate}
\item Quel est le prix à payer pour $35$ photos ?\point{2}
\item  Vérifier que le prix à payer pour $150$ photos est $23,50$ \euro{}. \point{2}
\item  On dispose d'un budget de $10$ \euro{}. Combien de photos peut-on commander au maximum ? \point{3}
\end{enumerate}

\item  On a commencé à construire un programme qui doit permettre de calculer le prix à payer en
fonction du nombre de photos commandées :

\includegraphics[scale=1]{FIG/scratch_fonction_cours.jpg} 

Par quelles valeurs peut-on compléter les instructions des lignes 4, 5 et 8 pour que
le programme permette de calculer le prix à payer en fonction du nombre de photos
commandées ?

\end{enumerate}


\end{ExoAuto}

  


%%%%%%%%%%%%%%%%%%%%%%%%%%%%%%%%%%%%%%%%%%%%%%%%%%%%%%%%%%%%%%%%%%%

 \begin{ExoAuto}{Chercher.}{1234}{2}{0}{0}{0}{0}
 
  \input{CHAPITRES/2_notions_de_fonctions/NF-51bis}
 
 \end{ExoAuto}
 
 
\end{pageAuto}
%%%%%%%%%%%%%%%%%%%%%%%%%%%%%%%%%%%%%%%%%%%%%%%%%%%%%%%%%%%%%%%%%%%
%%%%  Brouillon
%%%%%%%%%%%%%%%%%%%%%%%%%%%%%%%%%%%%%%%%%%%%%%%%%%%%%%%%%%%%%%%%%%%


\begin{pageBrouillon} 
 
\ligne{32}



\end{pageBrouillon}