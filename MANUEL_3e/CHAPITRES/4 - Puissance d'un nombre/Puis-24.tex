
En 1938, un mathématicien décide d'appeler le nombre $10^{100}$, le \textbf{googol}. Ce mot a inspiré le moteur de recherche Google. Écris avec une puissance de 10, les nombres :

\begin{description}
\begin{minipage}{0.49\linewidth}
\item A= googol$^3$
\item B= $\frac{\text{googol}^4}{\text{googol}^2}$
\end{minipage}
\hfill
\begin{minipage}{0.49\linewidth}
\item C= googol $\times $ googol
\item D= googol $^{\text{googol}}$  
\end{minipage}


\end{description}

