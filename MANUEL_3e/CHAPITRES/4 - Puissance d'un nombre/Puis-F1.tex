\begin{titre}[Les puissances]

\TitreSansTemps{Écriture scientifique} 
\end{titre}


\begin{CpsCol}
\textbf{}
\begin{description}
\item[$\square$] Exprimer un nombre en écriture scientifique
\end{description}
\end{CpsCol}


\Dec{1}{Puis-25}

\begin{DefT}{Écriture scientifique}\index{Puissance!Écriture scientifique}
L'écriture scientifique d'un nombre est le produit d'un nombre décimal dont la partie entière comporte un seul chiffre différent de zéro par une puissance de 10.

L'écriture scientifique d'un nombre est unique.
\end{DefT}
 
\begin{Ex}
$427,15$ a pour écriture scientifique $4,2715 \times 10^2$ et $0,256$ a pour écriture scientifique $2,56 \times 10^{-1}$.
\end{Ex}

%\begin{Mt}
%Soit $n$ un nombre entier.
%\begin{description}
%\item[•] Lorsqu'on multiplie un nombre par $10^n$, on décale sa virgule vers la droite de $n$ chiffres. 
%\item[•] Lorsqu'on multiplie un nombre par $10^{-n}$, on décale sa virgule vers la gauche de $n$ chiffres.
%\end{description}
%Si le nombre de chiffres n'est pas suffisant, on complète avec des zéros.
%\end{Mt}
%
%\begin{Ex}
%\begin{description}
%\item[•] $35 \times 10^3 = 35\overbrace{000}$, on remarque le décalage de 3 chiffres vers la droite.
%\item[•] $3425 \times 10^{-2} = 34,\underbrace{25}$, on remarque le décalage de 2 chiffres vers la gauche.
%\end{description}
%\end{Ex}
%
%\begin{minipage}{0.48\linewidth}
%\AD{1}{Puis-26}
%\end{minipage}
%\hfill
%\begin{minipage}{0.48\linewidth}
%\AD{1}{Puis-27}
%\end{minipage}

\begin{Rq}
L'écriture permet de donner un ordre de grandeur du nombre donné en lisant l'exposant de la puissance de 10. $\np{1529}=1,529\times 10^3$ donc ce nombre est de l'ordre des milliers ($10^3$).
\end{Rq}

%\begin{autoeval}
%\begin{tabular}{p{12cm}p{0.5cm}p{0.5cm}p{0.5cm}p{1cm}}
%\textbf{Compétences visées} &  M I & MF & MS  & TBM \vcomp \\ 
%Exprimer un nombre en écriture scientifique & $\square$ & $\square$  & $\square$ & $\square$ \vcomp \\  
%\end{tabular}
%{\footnotesize MI : maitrise insuffisante ; MF = Maitrise fragile ; MS = Maitrise satisfaisante ; TBM = Très bonne maitrise}
% 
%\end{autoeval}



