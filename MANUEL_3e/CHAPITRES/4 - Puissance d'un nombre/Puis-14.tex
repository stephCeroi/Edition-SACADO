
Marin Mersenne (1588-1648) était un religieux français érudit, philosophe, mathématicien et physicien. Il s’est intéressé à de nombreux domaines (acoustique, chute des corps, arithmétique, ...). Il a fourni une liste de nombres de la forme $2^2 - 1$, $2^3 - 1$, $2^5 - 1$, ..., jusqu’à $2^{257} - 1$, où chacun des exposants (2, 3, 5, ..., 257) est un nombre premier. Pour lui rendre hommage, on appelle aujourd'hui nombres de Mersenne les nombres de la forme $M_n = 2^n -1$, où $n$ est un entier naturel non nul.
\begin{enumerate}
%\item À l’aide d’un tableur, calculer les nombres $M_n$ pour tous les entiers $n$ allant de 2 à 20.
\item Pour chacun des nombres premiers 2, 3, 5, 7, vérifier que les nombres de Mersenne $M_2$, $M_3$, $M_5$, $M_7$ sont des nombres premiers.
\item La conjecture « Les nombres de Mersenne $M_n$ où $n$ est un nombre premier sont des nombres premiers » est-elle plausible ? Vérifier que $M_{11}$ admet un diviseur autre que 1 et lui-même. La conjecture précédente est-elle vraie ? Justifier.
\item Le nombre  $M_{13}$ est-il premier ?
%\item Un nombre parfait est un entier qui est égal à la somme de tous ses diviseurs positifs autres que lui-même. Ainsi 6 est un nombre parfait puisque les diviseurs de 6 autres que 6 sont 1, 2, et 3, et puisque $6 = 1 + 2 + 3$.

%Euclide a prouvé que l'on obtient des nombres parfaits à partir des nombres de Mersenne qui sont premiers, de la façon suivante : on multiplie le nombre de Mersenne par la puissance de 2 dont l'exposant est inférieur d'une unité à celui intervenant dans l'écriture du nombre. Ainsi, à partir du nombre de Mersenne premier $2^2 - 1$, on obtient le nombre parfait $6 = 2^{2-1} \times (2^2 - 1)$. Vérifier le résultat d’Euclide pour les nombres de Mersenne $M_3$ et $M_5$.

\end{enumerate}

