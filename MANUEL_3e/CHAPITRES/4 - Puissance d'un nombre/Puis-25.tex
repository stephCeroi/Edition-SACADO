
La mantisse est le nombre de chiffres utilisable par un calculateur (ordinateur, calculatrice) pour représenter un nombre.

Par exemple, si la mantisse est 10 (comme sur certaines calculatrice de collège) 1289,565874899214 ne peut pas s'écrire ce nombre comporte 16 chiffres. La calculatrice écrit donc seulement 9 chiffres et la virgule. Lesquels ? 

\begin{description}
\item[•] En partant de la gauche ? Mais si la partie entière a plus que 9 chiffres on ne peut pas l'écrire.....
\item[•] En partant de la droite ? Mais si la partie décimale a plus que 9 chiffres on ne peut pas l'écrire.....
\end{description}

Le modèle trouvé est d'écrire un seul chiffre différent de 0 dans la partie décimale puis de compléter la mantisse avec les chiffres qui se lisent de gauche à droite et écrire une puissance de 10 qui approxime le nombre donné. 

Ainsi, $1289,565874899214=1,28956587 \times 10^3$.

Complète les tableaux ci-dessous.
\ \\
\begin{enumerate}
\item 12,5926378125 = 
\begin{tikzpicture} 
\framebox(20,20)[lt]{bla}  \framebox(20,20)[lt]{}  \framebox(20,20)[lt]{bla}  \framebox(20,20)[lt]{bla}
\framebox(20,20)[lt]{bla}  \framebox(20,20)[lt]{bla}  \framebox(20,20)[lt]{bla}  \framebox(20,20)[lt]{bla} 
\framebox(20,20)[lt]{bla}  \framebox(20,20)[lt]{bla}    
\end{tikzpicture} 
\hspace{0.8cm} {\huge ,} 
\hspace{6.1cm} {\Large $\times 10^{\ldots}$} \\
\item 1658,941256371 =
\begin{tikzpicture} 
\framebox(20,20)[lt]{bla}  \framebox(20,20)[lt]{}  \framebox(20,20)[lt]{bla}  \framebox(20,20)[lt]{bla}
\framebox(20,20)[lt]{bla}  \framebox(20,20)[lt]{bla}  \framebox(20,20)[lt]{bla}  \framebox(20,20)[lt]{bla} 
\framebox(20,20)[lt]{bla}  \framebox(20,20)[lt]{bla}    
\end{tikzpicture} 
\hspace{0.8cm}  {\huge ,} 
\hspace{6.1cm} {\Large $\times 10^{\ldots}$} \\
\item 2015486,9701244 =
\begin{tikzpicture} 
\framebox(20,20)[lt]{bla}  \framebox(20,20)[lt]{}  \framebox(20,20)[lt]{bla}  \framebox(20,20)[lt]{bla}
\framebox(20,20)[lt]{bla}  \framebox(20,20)[lt]{bla}  \framebox(20,20)[lt]{bla}  \framebox(20,20)[lt]{bla} 
\framebox(20,20)[lt]{bla}  \framebox(20,20)[lt]{bla}    
\end{tikzpicture} 
\hspace{0.8cm}  {\huge ,} 
\hspace{6.1cm} {\Large $\times 10^{\ldots}$} \\
\item 2012145896545486,9701244 =
\begin{tikzpicture} 
\framebox(20,20)[lt]{bla}  \framebox(20,20)[lt]{}  \framebox(20,20)[lt]{bla}  \framebox(20,20)[lt]{bla}
\framebox(20,20)[lt]{bla}  \framebox(20,20)[lt]{bla}  \framebox(20,20)[lt]{bla}  \framebox(20,20)[lt]{bla} 
\framebox(20,20)[lt]{bla}  \framebox(20,20)[lt]{bla}    
\end{tikzpicture} 
\hspace{0.8cm}  {\huge ,} 
\hspace{6.1cm} {\Large $\times 10^{\ldots}$} \\
\item 0,0124674125896211 =
\begin{tikzpicture} 
\framebox(20,20)[lt]{bla}  \framebox(20,20)[lt]{}  \framebox(20,20)[lt]{bla}  \framebox(20,20)[lt]{bla}
\framebox(20,20)[lt]{bla}  \framebox(20,20)[lt]{bla}  \framebox(20,20)[lt]{bla}  \framebox(20,20)[lt]{bla} 
\framebox(20,20)[lt]{bla}  \framebox(20,20)[lt]{bla}    
\end{tikzpicture} 
\hspace{0.8cm} {\huge ,} 
\hspace{6.1cm} {\Large $\times 10^{\ldots}$} \\
\end{enumerate}


