%----------------------------------------------------------------------------------------
%	PACKAGES AND OTHER DOCUMENT CONFIGURATIONS
%----------------------------------------------------------------------------------------

%----------------------------------------------------------------------------------------
%		Géometrie de la page
%----------------------------------------------------------------------------------------
\documentclass[dvipsnames,french,10pt]{book}

\usepackage[
paperheight=29.7cm, %hauteur du papier
paperwidth=21cm, %largeur du papier
left=1cm, %marge de gauche
right=1cm, %marge de droite
top=1.5cm, %marge du haut
bottom=1cm, %marge du bas
%marginparsep=0pt, %distance entre le texte et les notes de marges 
reversemp, %inverse l'emplacement de la marge
headheight=20.60pt %hauteur du header
%showframe, %permet d'afficher le cadre défini ci-dessus
%bindingoffset=1cm %permet d'ajouter le décalage dû au reliage
]{geometry} %Redéfinition de la taille des pages
\raggedbottom


%----------------------------------------------------------------------------------------
%		Generals
%----------------------------------------------------------------------------------------
%\usepackage{fourier} %!! A changer plus tard !!
\usepackage[scaled]{uarial}
\renewcommand*\familydefault{\sfdefault} %% Only if the base font of the document is to be sans serif
\usepackage{frcursive}
\usepackage[T1]{fontenc} %Accents handling
\usepackage[utf8]{inputenc} % Use UTF-8 encoding
%\usepackage{microtype} % Slightly tweak font spacing for aesthetics
\usepackage[english, francais]{babel} % Language hyphenation and typographical rules
\usepackage{marginnote}


%----------------------------------------------------------------------------------------
%		Graphics
%----------------------------------------------------------------------------------------
\usepackage{xcolor}
\usepackage{graphicx, multicol} % Enhanced support for graphics
\graphicspath{FIG/}
\usepackage{wrapfig}
\usepackage{colortbl}
\usepackage[framemethod=tikz]{mdframed}
%\usepackage{xsavebox}
% Il faudrait utiliser xsavebox à l'avenir pour réduire la taille du pdf

%----------------------------------------------------------------------------------------
%		Other packages
%----------------------------------------------------------------------------------------
\usepackage{hyperref}
\hypersetup{
	colorlinks=true, %colorise les liens
	breaklinks=true, %permet le retour à la ligne dans les liens trop longs
	urlcolor= sacado_violet,  %couleur des hyperliens et des QR codes
	linkcolor= sacado_violet, %couleur des liens internes
	plainpages=false  %pour palier à "Bookmark problems can occur when you have duplicate page numbers, for example, if you have a page i and a page 1."
}
\usepackage{tabularx}
\newcolumntype{M}[1]{>{\arraybackslash}m{#1}} %Defines a scalable column type in tabular
\usepackage{booktabs} % Enhances quality of tables
\usepackage{diagbox} % barre en diagonale dans un tableau
\usepackage{multicol}
\usepackage[explicit]{titlesec}
\usepackage{xr}
\usepackage{xspace}
\usepackage{array}
\usepackage{listings}
\usepackage{fancyvrb} %verbatim
\usepackage{stmaryrd}
\usepackage{float}



% Python style for highlighting
\lstdefinestyle{mystyle}{
    backgroundcolor=\color{white},   
    commentstyle=\color{sacado_green},
    keywordstyle=\color{sacado_red},
    numberstyle=\tiny\color{sacado_orange},
    stringstyle=\color{sacado_blue},
    basicstyle=\ttfamily\footnotesize,
    breakatwhitespace=false,         
    breaklines=true,                 
    captionpos=b,                    
    keepspaces=false,                 
    numbers=left,                    
    numbersep=5pt,                  
    showspaces=false,                
    showstringspaces=false,
    showtabs=false,                  
    tabsize=4
}

\lstset{style=mystyle}

%----------------------------------------------------------------------------------------
%		Headers and footers
%----------------------------------------------------------------------------------------

\pagestyle{empty}
\usepackage{fancyhdr}
\pagestyle{fancy}
\renewcommand{\headrulewidth}{0pt} % pas de filet sous le header

%----------------------------------------------------------------------------------------
%		Mathematics packages
%----------------------------------------------------------------------------------------
\usepackage{amsthm, amsmath, amssymb, mathrsfs} % Mathematical typesetting
\usepackage{marvosym, wasysym} % More symbols
\usepackage[makeroom]{cancel}
\usepackage{xlop}
\usepackage{pgf,tikz,pgfplots}
\pgfplotsset{compat=1.16}
\usepackage{pgf-pie}
\usetikzlibrary{positioning}
\usetikzlibrary{arrows}
\usepackage{pst-plot,pst-tree,pst-func, pstricks-add,pst-node,pst-text}
%\usepackage{units}
\usepackage{nicefrac}
\usepackage[np]{numprint} %Séparation milliers dans un nombre \np{12345} donne 12 345
\usepackage{multido}
\newcommand{\RNum}[1]{\uppercase\expandafter{\romannumeral #1\relax}}

%----------------------------------------------------------------------------------------
%		New text commands
%----------------------------------------------------------------------------------------
\usepackage{calc}
\usepackage{boites}
 \renewcommand{\arraystretch}{1.6}

%%%%% Pour les imports.
\usepackage{import}

%%%%% Pour faire des boites
\usepackage[tikz]{bclogo}
\usepackage{bclogo}
\usepackage{framed}
\usepackage[skins]{tcolorbox}
\tcbuselibrary{breakable}
\tcbuselibrary{skins}
\usetikzlibrary{quotes,babel,arrows.meta,shadows,decorations.pathmorphing,decorations.markings,patterns}
\usepackage{tikzpagenodes}
\usetikzlibrary{plotmarks}



%%%%% Pour une double minipage
\newcommand{\mini}[4]{
\begin{minipage}[c]{#1}
#2
\end{minipage}
\hfill
\begin{minipage}[c]{#3}
#4
\end{minipage}
}


\usepackage{enumitem}
\newlist{todolist}{itemize}{2} %Pour faire des QCM
\setlist[todolist]{label=$\square$} %Pour faire des QCM \begin{todolist} instead of itemize
\renewcommand{\FrenchLabelItem}{\textbullet} %bullet dans les items


%----------------------------------------------------------------------------------------
%		Définition de couleurs pour ...
%----------------------------------------------------------------------------------------

%GEOGEBRA

\definecolor{zzttqq}{rgb}{0.6,0.2,0.} %rouge des polygones
\definecolor{qqqqff}{rgb}{0.,0.,1.}
\definecolor{xdxdff}{rgb}{0.49019607843137253,0.49019607843137253,1.}%bleu
\definecolor{qqwuqq}{rgb}{0.,0.39215686274509803,0.} %vert des angles
\definecolor{ffqqqq}{rgb}{1.,0.,0.} %rouge vif
\definecolor{uuuuuu}{rgb}{0.26666666666666666,0.26666666666666666,0.26666666666666666}
\definecolor{qqzzqq}{rgb}{0.,0.6,0.}
\definecolor{cqcqcq}{rgb}{0.7529411764705882,0.7529411764705882,0.7529411764705882} %gris
\definecolor{qqffqq}{rgb}{0.,1.,0.}
\definecolor{ffdxqq}{rgb}{1.,0.8431372549019608,0.}
\definecolor{ffffff}{rgb}{1.,1.,1.}
\definecolor{ududff}{rgb}{0.30196078431372547,0.30196078431372547,1.}
\definecolor{ffqqff}{rgb}{1.,0.,1.}
\definecolor{ffxfqq}{rgb}{1,0.4980392156862745,0}
\definecolor{ffffqq}{rgb}{1,1,0}
\definecolor{qqttzz}{rgb}{0,0.2,0.6}
\definecolor{qqccqq}{rgb}{0,0.8,0}
\definecolor{qqzzff}{rgb}{0,0.6,1}
\definecolor{qqwwzz}{rgb}{0,0.4,0.6}
\definecolor{eqeqeq}{rgb}{0.8784313725490196,0.8784313725490196,0.8784313725490196}

%SACADO

\definecolor{fond}{HTML}{5D4391}  %couleur des entetes etc.  violet sacado
\definecolor{sacado_purple}{HTML}{5D4391} %% Violet foncé Sacado
\definecolor{sacado_violet}{HTML}{9274C7} %% Violet clair Sacado
\definecolor{texte}{HTML}{FFFFFF} % couleur du texte des entetes etc.
\definecolor{sacado_blue_light}{HTML}{0093CA} %% Bleu Sacado
\definecolor{sacado_blue}{HTML}{0960B5} %% Bleu Sacado
\definecolor{sacado_green}{HTML}{00B999} %% Vert Sacado
\definecolor{sacado_green_dark}{HTML}{4D8075} %% Vert Sacado foncé
\definecolor{sacado_yellow}{HTML}{F9F871} %% Jaune Sacado
\definecolor{sacado_orange}{HTML}{FF8B69} %% Orange Sacado
\definecolor{sacado_red}{HTML}{9F1E17} %% Rouge Sacado
\definecolor{sacado_gray}{HTML}{7B7485} %% Gris Sacado
%BOITES 

\definecolor{bleu1}{rgb}{0.54,0.79,0.95} %% Bleu
\definecolor{sapgreen}{rgb}{0.4, 0.49, 0}
\definecolor{dvzfxr}{rgb}{0.7,0.4,0.}
\definecolor{beamer}{rgb}{0.5176470588235295,0.49019607843137253,0.32941176470588235} % couleur beamer
\definecolor{preuveRbeamer}{rgb}{0.8,0.4,0}
\definecolor{sectioncolor}{rgb}{0.24,0.21,0.44}
\definecolor{subsectioncolor}{rgb}{0.1,0.21,0.61}
\definecolor{subsubsectioncolor}{rgb}{0.1,0.21,0.61}
\definecolor{info}{rgb}{0.82,0.62,0}
\definecolor{bleu2}{rgb}{0.38,0.56,0.68}
\definecolor{bleu3}{rgb}{0.24,0.34,0.40}
\definecolor{bleu4}{rgb}{0.12,0.20,0.25}
\definecolor{vert}{rgb}{0.21,0.33,0}
\definecolor{vertS}{rgb}{0.05,0.6,0.42}
\definecolor{red}{rgb}{0.78,0,0}
\definecolor{color5}{rgb}{0,0.4,0.58}
\definecolor{eduscol4B}{rgb}{0.19,0.53,0.64}
\definecolor{eduscol4P}{rgb}{0.62,0.12,0.39}
\definecolor{ill_frame}{HTML}{F0F0F0} %Boite illustration contour
\definecolor{ill_back}{HTML}{F7F7F7}  %Boite illustration background
\definecolor{ill_title}{HTML}{AAAAAA} %Boite illustration titre

%----------------------------------------------------------------------------------------
%		QR codes
%----------------------------------------------------------------------------------------

\usepackage[
final %Pour la compilation finale
%draft %Pour le travail sur les documents
]{qrcode}
\usepackage{fontawesome}
\usepackage{fancyqr}
\FancyQrLoad{flat}
\fancyqrset{
%image=\scalebox{.8}{\includegraphics[scale=1]{sacadoA1.png}},image padding=.5,
l color=sacado_green,r color=sacado_blue}
\newcommand{\qr}[2]{\centering \fancyqr{https://sacado.xyz/qcm/show_course_from_qrcode/#1}

\vspace{.2cm}

#2} %\qr{id} Pour obtenir un qrcode en indiquant seulement l'id de l'exercice





\newcommand{\miniqr}[3]{
\begin{minipage}[c]{.8\linewidth}
#1
\end{minipage}
\hfill
\fbox{
\begin{minipage}[c]{.18\linewidth}
\begin{center}
\fancyqr{https://sacado.xyz/qcm/show\_course\_from\_qrcode/#2}

\vspace{.2cm}

#3
\end{center}
\end{minipage}
}
}

%practice/frombook/<int:ide>/ pour accéder à un exercice depuis le livre.

\usepackage{pythontex}
\begin{pycode}
import qrcode
def qr(data):
     fic=r"QRcodes/qr"+data+'.png'
     urlcourte=r"sacado.xyz/"+data
     urllongue=r"https://"+urlcourte
     qr = qrcode.QRCode(version=1,
error_correction=qrcode.constants.ERROR_CORRECT_L,box_size=2, border=0)
     qr.add_data(urlcourte)
     qr.make(fit=True)
     qr_image = qr.make_image(fill_color="black", back_color="white")
     qr_image.save(fic)
     return r"""\parbox{3.5cm}{\begin{center}
\includegraphics{"""+fic+r"}\\{\scriptsize\tt "+urlcourte+r"}\end{center}}"
\end{pycode}

%\renewcommand{\qr}[1]{\py{qr("#1")}} % compilation complete
%utiliser :
%  pdflatex --shell-escape MANUEL_6e_V1.tex ; pythontex MANUEL_6e_V1.tex --interpreter python:python3 ; pdflatex --shell-escape % %MANUEL_6e_V1.tex

%  draft
\renewcommand{\qr}[1]{
\parbox{5cm}{\begin{center}
       \includegraphics{QRcodes/qrDummy.png}
       %\\{\tt dummy}
\end{center}}
}

\renewcommand{\miniqr}[1]{
       \includegraphics[height=1cm]{QRcodes/qrDummy.png}
       %\\{\tt dummy}}
}


\usepackage[absolute]{textpos}
\newcommand{\qrHautDroite}[1]{
\setlength{\TPHorizModule}{1cm}
\setlength{\TPVertModule}{1cm}
\begin{textblock}{3.51}(12.5,1.5){\qr{#1}}
\end{textblock}
}

 






\usepackage{makeidx}
\makeindex

%----------------------------------------
%
%   Définitions des environnements "pageCours" et "pageExos"
%
%----------------------------------------

\newcounter{cpt}
\newcounter{exo}
\newcounter{cptr}

\newcommand{\titreChap}{Titre de chapitre à définir}

\renewcommand{\chapter}[3]{
  \stepcounter{chapter}
  \setcounter{exo}{0}
  \setcounter{cpt}{0}
  
%\cleardoublepage  % pour commencer à droite
{\Huge \hfill Chapitre \Roman{chapter}.\\
  \bigskip
  #1\\
  \bigskip {\begin{center}
  \fancyqr[image={\includegraphics[scale=.6]{sacadoA1.png}},image padding=.5,height=5cm]{#2}
  \end{center}}  {\normalsize #3}}
\renewcommand{\titreChap}{#1}

%\ifthenelse{\equal{#2}{}}{}{\par
%  \bigskip\bigskip
%  #2}
\newpage
}

\renewenvironment{leftbar}[1][\hsize]
{%
    \def\FrameCommand
    {%
        {\color{black}\vrule width 0.5pt}%
        \hspace{4pt}%must no space.
        \fboxsep=\FrameSep%\colorbox{yellow}%
    }%
    \MakeFramed{\hsize#1\advance\hsize-\width\FrameRestore}%
}
{\endMakeFramed}


\newcommand{\headerGeneral}[3]{ % intitulé, couleur, qrcode
\begin{tikzpicture}[remember picture,overlay]
\coordinate(NO) at (-2,0);
\coordinate(SW) at (22,1);
\coordinate(titre) at (0,0.2);
\coordinate(qr) at (16.85,0.);
\shade[left color=#2 , right color=#2 ] (NO) rectangle (SW);
\draw (titre) node[color=texte, anchor=west]{ {\large \bf  #1} \quad\quad \bf {\small \titreChap} };
\draw (qr) node {\qr{#3}};
\end{tikzpicture}
}

\newenvironment{pageCours}{\lhead{%
\pagecolor{white!100}
\headerGeneral{COURS}{fond!70}{p/1234}
}\begin{leftbar}}{\end{leftbar}\newpage}

\newenvironment{pageAD}{\lhead{%
\pagecolor{sacado_violet!6}
\headerGeneral{APPLICATIONS DIRECTES}{sacado_violet!70}{p/1234}
} }{ \newpage}

\newenvironment{pageParcoursu}{\lhead{%
\pagecolor{sacado_green!6} 
\headerGeneral{PARCOURS 1}{sacado_green}{p/1234}
} }{ \newpage}


\newenvironment{pageParcoursd}{\lhead{%
\pagecolor{sacado_blue!6}
\headerGeneral{PARCOURS 2}{sacado_blue_light}{p/1234}
} }{ \newpage}

\newenvironment{pageParcourst}{\lhead{%
\pagecolor{sacado_red!6}
\headerGeneral{PARCOURS 3}{sacado_red}{p/1234}
} }{ \newpage}

\newenvironment{pageBrouillon}{\lhead{%
\pagecolor{sacado_gray!6}
\headerGeneral{BROUILLON}{sacado_gray}{p/1234}
} }{ \newpage}

\newenvironment{pageRituels}{\lhead{%
\pagecolor{fond!6}
\headerGeneral{RITUELS}{fond!70}{p/1234}
} }{ \newpage}

\newenvironment{pageAuto}{\lhead{%
\pagecolor{sacado_orange!6}
\headerGeneral{AUTOÉVALUATION}{sacado_orange}{p/1234}
} }{ \newpage}

\newenvironment{pageHistoire}{\lhead{%
\pagecolor{olive!6}
\headerGeneral{HISTOIRE}{olive}{p/1234}
} }{ \newpage}



\newenvironment{pageExercices}{\lhead{%
\pagecolor{white!100}
\headerGeneral{ACTIVITÉS}{fond}{p/1234}
}\begin{leftbar}}{\end{leftbar}\newpage}



\fancyfoot[L]{\colorbox{fond!70}{\color{texte}\thepage}}
\fancyfoot[C]{}


\newcommand{\titresec}[2]{\phantom{.}\begin{textblock}{1}[0,1](-1.24,0.25)\colorbox{fond!70}{%
\makebox[0.8cm]{\raisebox{0.05cm}[0.6cm][0.15cm]{\color{texte}\LARGE\bf #1}}}\end{textblock}{\LARGE\bf #2}\\\bigskip}

\renewcommand{\thesection}{\arabic{section}}
\titleformat{\section}{}{%
\hspace{-1.15cm}\colorbox{fond!70}{%
\makebox[0.8cm]{\raisebox{0.05cm}[0.6cm][0.15cm]{\color{texte}\LARGE\bf \thesection}}}}{1em}{\bf \LARGE #1}
  
\renewcommand{\thesubsection}{\arabic{subsection}}
            
\titleformat{\subsection}
{%\begin{textblock}{1}[0,1](-1,0.42) toto
  %\end{textblock}
%\reversemarginpar\marginnote[\rule{0.8cm}{0.8cm}]{}[0pt]  \color{red}\normalfont\Large\bfseries}
}{\hspace{-0.83em}
\colorbox{fond!70}{\makebox[0.6cm]{\raisebox{0cm}[1em][0.2em]\normalfont\large\bfseries\color{texte}\thesubsection}}}{1em}{\bf \large #1}




\makeatletter
\newenvironment{TraitV}[1]{%
% #1 couleur du trait (par défaut CouleurA)
% #2 largeur du trait
% #3 distance entre le trait et le texte
\def\FrameCommand{{\color{#1}\vrule width 2pt}
\hspace{1em}}\MakeFramed {\advance\hsize-\width}}%
{\endMakeFramed}
\makeatother

%----------------------------------------
%
%   Définitions des environnements de Définitions, propriétés...
%
%----------------------------------------

%%%%%%%%%%%%% Définitions
\newenvironment{Def}{%
\medskip \begin{tcolorbox}[widget,colback=sacado_violet!15,colframe=sacado_violet!75!black,
title= \stepcounter{cpt} Définition \thecpt. ]}{%
\end{tcolorbox}\par}


\newenvironment{DefT}[1]{%
\medskip \begin{tcolorbox}[widget,colback=sacado_violet!15,colframe=sacado_violet!75!black,
title= \stepcounter{cpt} Définition \thecpt : #1.]}
{%
\end{tcolorbox}\par}


%%%%%%%%%%%%% Proposition
\newenvironment{Prop}{%
\medskip \begin{tcolorbox}[widget,colback=sacado_blue!15,colframe=sacado_blue!75!black,
title= \stepcounter{cpt} Proposition \thecpt.]}
{%
\end{tcolorbox}\par}


%%%%%%%%%%%%% Propriétés
\newenvironment{Pp}{%
\medskip \begin{tcolorbox}[widget,colback=white!100,colframe=sacado_violet!75!black,
title= \stepcounter{cpt} Propriété \thecpt.]}
{%
\end{tcolorbox}\par}

\newenvironment{PpT}[1]{%
\medskip \begin{tcolorbox}[widget,colback=white!100,colframe=sacado_violet!75!black,
title= \stepcounter{cpt} Propriété \thecpt : #1. ]}
{%
\end{tcolorbox}\par}

\newenvironment{Pps}{%
\medskip \begin{tcolorbox}[widget,colback=white!100,colframe=sacado_violet!75!black,
title= \stepcounter{cpt} Propriétés \thecpt.]}
{%
\end{tcolorbox}\par}


%%%%%%%%%%%%% Conséquence
\newenvironment{Cq}{%
\medskip \begin{tcolorbox}[widget,colback=white,colframe=sacado_blue,
title= \stepcounter{cpt} Conséquence \thecpt.]}
{%
\end{tcolorbox}\par}



%%%%%%%%%%%%% Théorèmes
\newenvironment{ThT}[1]{% théorème avec titre
\medskip \begin{tcolorbox}[widget,colback=white!100,colframe=sacado_violet!75!black,
title= \stepcounter{cpt} Théorème \thecpt : #1.]}
{%
\end{tcolorbox}\par}

\newenvironment{Th}{%
\medskip \begin{tcolorbox}[widget,colback=white!100,colframe=sacado_violet!75!black,
title= \stepcounter{cpt} Théorème \thecpt.]}
{%
\end{tcolorbox}\par}


%%%%%%%%%%%%% Règles
\newenvironment{Reg}{%
\medskip \begin{tcolorbox}[widget,colback=sacado_blue!15,colframe=sacado_blue,
title= \stepcounter{cpt} Règle \thecpt.]}
{%
\end{tcolorbox}\par}

%%%%%%%%%%%%% Représentations
\newenvironment{Rep}{%
\medskip \begin{tcolorbox}[widget,colback=white,colframe=sacado_violet!75!white,
title= \stepcounter{cpt} Représentation \thecpt.]}
{%
\end{tcolorbox}\par}

 
%%%%%%%%%%%%% REMARQUES
\newenvironment{Rq}{%
\medskip \begin{tcolorbox}[widget,colback=sacado_orange!15,colframe=sacado_orange,
title= \stepcounter{cpt} Remarque \thecpt.]}
{%
\end{tcolorbox}\par}

\newenvironment{Rqs}{%
\medskip \begin{tcolorbox}[widget,colback=sacado_orange!15,colframe=sacado_orange,
title= \stepcounter{cpt} Remarques \thecpt.]}
{%
\end{tcolorbox}\par}


%%%%%%%%%%%%% EXEMPLES
\newenvironment{Ex}{%
\medskip \begin{tcolorbox}[widget,colback=white,colframe=sacado_blue_light,
title= \stepcounter{cpt} Exemple \thecpt.]}
{%
\end{tcolorbox}\par}

\newenvironment{Exs}{%
\medskip \begin{tcolorbox}[widget,colback=white!15,colframe=sacado_blue_light,
title= \stepcounter{cpt} Exemples \thecpt.]}
{%
\end{tcolorbox}\par}

 
\newenvironment{ExT}[1]{%
\medskip \begin{tcolorbox}[widget,colback=white,colframe=sacado_blue_light,
title= \stepcounter{cpt} Exemple \thecpt   : #1.]}
{%
\end{tcolorbox}\par}

 
\newenvironment{ExCor}{%
\medskip \begin{tcolorbox}[widget,colback=white,colframe=sacado_blue ,
title= \stepcounter{cpt} Exercice commenté \thecpt.]}
{%
\end{tcolorbox}\par}

\newenvironment{ExQr}[1]{%
\medskip \begin{tcolorbox}[widget,colback=white,colframe=sacado_blue_light ,
title= \stepcounter{cpt} Exemple  \thecpt. \hfill {\color{sacado_blue}https://sacado.xyz/a/#1} ]
\begin{minipage}{1.5cm}
\miniqr{#1}
\end{minipage}
\begin{minipage}{0.8\linewidth}
}
{%
\end{minipage}
\end{tcolorbox}\par}


\newenvironment{OuQr}[1]{%
\medskip \begin{tcolorbox}[widget,colback=white,colframe=sacado_orange ,
title= \stepcounter{cpt} Outil \thecpt. \hfill {\color{sacado_orange}https://sacado.xyz/a/#1} ]
\begin{minipage}{1.5cm}
\miniqr{#1}
\end{minipage}
\begin{minipage}{0.8\linewidth}
 \color{sacado_orange!90!black}
}
{%
 
\end{minipage}
\end{tcolorbox}\par}



\newenvironment{MeQr}[1]{%
\medskip \begin{tcolorbox}[widget,colback=white,colframe=sacado_blue,
title= \stepcounter{cpt} Méthode \thecpt. \hfill {\color{sacado_blue}https://sacado.xyz/a/#1} ]
\begin{minipage}{1.5cm}
\miniqr{#1}
\end{minipage}
\begin{minipage}{0.8\linewidth}
}
{%
\end{minipage}
\end{tcolorbox}\par}




%%%%%%%%%%%%% Logique
\newenvironment{Log}{%
\medskip \begin{tcolorbox}[widget,colback=sacado_blue!10,colframe=sacado_blue,
title= \stepcounter{cpt} Logique mathématique \thecpt.]}
{%
\end{tcolorbox}\par}
%%%%%%%%%%%%% Logique avec paramètre
\newenvironment{LogT}[1]{%
\medskip \begin{tcolorbox}[widget,colback=sacado_blue!10,colframe=sacado_blue,
title= \stepcounter{cpt} Logique mathématique \thecpt. #1]}
{%
\end{tcolorbox}\par}

%%%%%%%%%%%%% Preuve
\newenvironment{Pv}[1][]{%
\begin{tcolorbox}[breakable, enhanced,widget, colback=sacado_blue!10!white,boxrule=0pt,frame hidden,
borderline west={1mm}{0mm}{sacado_blue!75}]
\textbf{Preuve#1 : }}
{%
\end{tcolorbox}
\par}


%%%%%%%%%%%%% PreuveROC
\newenvironment{PvR}[1][]{%
\begin{tcolorbox}[breakable, enhanced,widget, colback=sacado_blue!10!white,boxrule=0pt,frame hidden,
borderline west={1mm}{0mm}{sacado_blue!75}]
\textbf{Preuve (ROC)#1 : }}
{%
\end{tcolorbox}
\par}


%%%%%%%%%%%%% DemoExigible
\newenvironment{DemoE}{%
\medskip \begin{tcolorbox}[widget,colback=sacado_blue!10,colframe=sacado_blue,
title= \stepcounter{cpt} Démonstration exigible \thecpt. ]}
{%
\end{tcolorbox}\par}





%%%%%%%%%%%%% Compétences
\newenvironment{Cps}[1][]{%
\vspace{0.4cm}
\begin{tcolorbox}[enhanced, lifted shadow={0mm}{0mm}{0mm}{0mm}%
{black!60!white}, attach boxed title to top left={xshift=5mm, yshift*=-3mm}, coltitle=white, colback=white, boxed title style={colback=sacado_green!100}, colframe=sacado_green!75!black,title=\textbf{Compétences associées#1}]}
{%
\end{tcolorbox}
\par}

%%%%%%%%%%%%% Chapitres connexes
\newenvironment{CCon}[1][]{%
\vspace{0.4cm}
\begin{tcolorbox}[breakable, enhanced,widget, colback=white ,boxrule=0pt,frame hidden,
borderline west={2mm}{0mm}{sacado_violet}]
\textbf{#1}}
{%
\end{tcolorbox}
\par}
%%%%%%%%%%%%% Compétences Collège
\newenvironment{CpsCol}[1][]{%
\vspace{0.4cm}
\begin{tcolorbox}[breakable, enhanced,widget, colback=white ,boxrule=0pt,frame hidden,
borderline west={2mm}{0mm}{sacado_violet}]
\textbf{#1}}
{%
\end{tcolorbox}
\par}


 

%%%%%%%%%%%%% Rituel
\newenvironment{Rit}{%
\medskip \begin{tcolorbox}[widget,colback=white!15,colframe=sacado_violet!75!black,
title= \stepcounter{cpt} Rituel \thecpt. ]}{%
\end{tcolorbox}\par}


%%%%%%%%%%%%% Méthode
\newenvironment{Mt}{%
\medskip \begin{tcolorbox}[widget,colback=white!15,colframe=sacado_violet!75!black,
title= \stepcounter{cpt} Méthode \thecpt. ]}{%
\end{tcolorbox}\par}

%%%%%%%%%%%%% Méthode
\newenvironment{MtT}[1]{%
\medskip \begin{tcolorbox}[widget,colback=white!15,colframe=sacado_violet!75!black,
title= \stepcounter{cpt} Méthode \thecpt. #1 ]}{%
\end{tcolorbox}\par}


%%%%%%%%%%%%% VocU
\newenvironment{VocU}[1]{%
\medskip \begin{tcolorbox}[widget,colback=white!15,colframe=sacado_violet!75,
title= \stepcounter{cpt} Vocabulaire \thecpt. #1 ]}{%
\end{tcolorbox}\par}


%%%%%%%%%%%%% Notation
\newenvironment{Nt}[1]{%
\medskip \begin{tcolorbox}[widget,colback=white!5,colframe=sacado_red!75,
title= \stepcounter{cptr} Notation \thecptr. #1 ]}{%
\end{tcolorbox}\par}

%%%%%%%%%%%%% Ety
\newenvironment{Ety}[1]{%
\medskip \begin{tcolorbox}[widget,colback=white!15,colframe=sacado_violet!75,
title= \stepcounter{cpt} Étymologie \thecpt. #1 ]}{%
\end{tcolorbox}\par}


%%%%%%%%%%%%% His
\newenvironment{His}[1]{%
\begin{tcolorbox}[right=5mm, enhanced, lifted shadow={0mm}{0mm}{0mm}{0mm}%
{sacado_green_dark!90!white}, attach boxed title to top left={xshift=0.3cm, yshift*=-2mm}, coltitle=sacado_green_dark, colback=sacado_green!10!white, boxed title style={colback=white}, colframe=sacado_green_dark,title= Les mathématiciennes et mathématiciens ]
}{%
\end{tcolorbox}\par}


%%%%%%%%%%%%% Attention
\newenvironment{Att}[1]{%
\medskip \begin{tcolorbox}[widget,colback=sacado_red!5,colframe=sacado_red!95!white,
title= \stepcounter{cpt} Notation \thecpt. #1 ]}{%
\end{tcolorbox}\par}



%%%%%%%%%%%%%%%%%%%%%%%%%%%%%%%%%%%%%%%%%%%%%%%%%%%%%%%%%%%%%%%%%%%%%%%%%%%%%%%%%%%%%%%%%%%%%%%%%%%%%%%%%%%%%%%%%%%%%%
%%%%%%%%%%%%%%%%%%%%%%%%%%%%%%%%%%%%%%%%%%%%%%%%%%%%%%%%%%%%%%%%%%%%%%%%%%%%%%%%%%%%%%%%%%%%%%%%%%%%%%%%%%%%%%%%%%%%%%
%%%%%%%%%%%%%%%%  Exercices                                            %%%%%%%%%%%%%%%%%%%%%%%%%%%%%%%%%%%%%%%%%%%%%%%
%%%%%%%%%%%%%%%%%%%%%%%%%%%%%%%%%%%%%%%%%%%%%%%%%%%%%%%%%%%%%%%%%%%%%%%%%%%%%%%%%%%%%%%%%%%%%%%%%%%%%%%%%%%%%%%%%%%%%%
%%%%%%%%%%%%%%%%%%%%%%%%%%%%%%%%%%%%%%%%%%%%%%%%%%%%%%%%%%%%%%%%%%%%%%%%%%%%%%%%%%%%%%%%%%%%%%%%%%%%%%%%%%%%%%%%%%%%%%
 
 
 
%%%%%%%%%%%%% ExoCad 7 paramètres : Compétences, qrcode , calculatrice, python, scratch, tableur, annales
\newenvironment{ExoCad}[7]{% code avant
\tcbset{top=-0.2cm }
\stepcounter{exo}

\begin{tcolorbox}[right=-5mm, enhanced, lifted shadow={0mm}{0mm}{0mm}{0mm}%
{black!60!white}, attach boxed title to top right={xshift=-0.3cm, yshift*=-2mm}, coltitle=sacado_violet!85!black, colback=white!100!white, boxed title style={colback=white}, colframe=sacado_violet!100!black,title= {\footnotesize  #1}  ]
 
\hspace{-1.3cm} 
\begin{minipage}[t]{0.7cm}

 \begin{tikzpicture}
 	\node[fill=sacado_violet,minimum width=0.7cm]{\textcolor{white}{\bf {\Large \theexo}}};
 \end{tikzpicture}


%%%%%%%%%%%%%%%%%%%%%%%% Condition pour la calculatrice
 \ifthenelse{\equal{#3}{1}}{
 \begin{tikzpicture}
 	\node[minimum width=0.7cm]{\includegraphics[scale=0.5]{MISC/calculator.png} };
 \end{tikzpicture} 
 }{
 \ifthenelse{\equal{#3}{2}}{
 \begin{tikzpicture}
 	\node[minimum width=0.7cm]{\includegraphics[scale=0.5]{MISC/no_calculator.png} };
 \end{tikzpicture} 
 }{}
 }

\end{minipage}
\hfill
\begin{minipage}[t]{17.3cm}
} 
{ 
\end{minipage}%code  après
\hfill
\begin{minipage}[t]{1cm}

\begin{center}
\colorbox{sacado_violet}{\includegraphics[height=1cm]{qrcodes/qrDummy.png}}
\colorbox{white}{ {\footnotesize /b/ABCD} }
\end{center}

\end{minipage}

\end{tcolorbox}
\par 
}

 
%%%%%%%%%%%%% ExoCu 7 paramètres : Compétences, qrcode , calculatrice, python, scratch, tableur, annales
\newenvironment{ExoCu}[7]{% code avant
\tcbset{top=-0.2cm }
\stepcounter{exo}

\begin{tcolorbox}[right=-5mm, enhanced, lifted shadow={0mm}{0mm}{0mm}{0mm}%
{black!60!white}, attach boxed title to top right={xshift=-0.3cm, yshift*=-2mm}, coltitle=sacado_green!85!black, colback=white!100!white, boxed title style={colback=white}, colframe=sacado_green!100!black,title= {\footnotesize  #1}  ]
 
\hspace{-1.3cm} 
\begin{minipage}[t]{0.7cm}

 \begin{tikzpicture}
 	\node[fill=sacado_green,minimum width=0.7cm]{\textcolor{white}{\bf {\Large \theexo}}};
 \end{tikzpicture}


%%%%%%%%%%%%%%%%%%%%%%%% Condition pour la calculatrice
 \ifthenelse{\equal{#3}{1}}{
 \begin{tikzpicture}
 	\node[minimum width=0.7cm]{\includegraphics[scale=0.5]{MISC/calculator.png} };
 \end{tikzpicture} 
 }{
 \ifthenelse{\equal{#3}{2}}{
 \begin{tikzpicture}
 	\node[minimum width=0.7cm]{\includegraphics[scale=0.5]{MISC/no_calculator.png} };
 \end{tikzpicture} 
 }{}
 }

\end{minipage}
\hfill
\begin{minipage}[t]{17.3cm}
} 
{ 
\end{minipage}%code  après
\hfill
\begin{minipage}[t]{1cm}

 
\begin{center}
\colorbox{sacado_green}{\includegraphics[height=1cm]{qrcodes/qrDummy.png}}
\colorbox{white}{ {\footnotesize /b/ABCD}  }
\end{center}
 

\end{minipage}

\end{tcolorbox}
\par
}  


 %%%%%%%%%%%%% ExoCd 7 paramètres : Compétences, qrcode , calculatrice, python, scratch, tableur, annales
\newenvironment{ExoCd}[7]{% code avant
\tcbset{top=-0.2cm }
\stepcounter{exo}

\begin{tcolorbox}[right=-5mm, enhanced, lifted shadow={0mm}{0mm}{0mm}{0mm}%
{black!60!white}, attach boxed title to top right={xshift=-0.3cm, yshift*=-2mm}, coltitle=sacado_blue!85!black, colback=white!100!white, boxed title style={colback=white}, colframe=sacado_blue!100!black,title= {\footnotesize  #1}  ]
 
\hspace{-1.3cm} 
\begin{minipage}[t]{0.7cm}

 \begin{tikzpicture}
 	\node[fill=sacado_blue,minimum width=0.7cm]{\textcolor{white}{\bf {\Large \theexo}}};
 \end{tikzpicture}


%%%%%%%%%%%%%%%%%%%%%%%% Condition pour la calculatrice
 \ifthenelse{\equal{#3}{1}}{
 \begin{tikzpicture}
 	\node[minimum width=0.7cm]{\includegraphics[scale=0.5]{MISC/calculator.png} };
 \end{tikzpicture} 
 }{
 \ifthenelse{\equal{#3}{2}}{
 \begin{tikzpicture}
 	\node[minimum width=0.7cm]{\includegraphics[scale=0.5]{MISC/no_calculator.png} };
 \end{tikzpicture} 
 }{}
 }

\end{minipage}
\hfill
\begin{minipage}[t]{17.3cm}
} 
{ 
\end{minipage}%code  après
\hfill
\begin{minipage}[t]{1cm}

\begin{center}
\colorbox{sacado_blue}{\includegraphics[height=1cm]{qrcodes/qrDummy.png}}
\colorbox{white}{ {\footnotesize /b/ABCD} }
\end{center}

\end{minipage}

\end{tcolorbox}
\par
}  


%%%%%%%%%%%%% ExoCt 7 paramètres : Compétences, qrcode , calculatrice, python, scratch, tableur, annales
\newenvironment{ExoCt}[7]{% code avant
\tcbset{top=-0.2cm }
\stepcounter{exo}

\begin{tcolorbox}[right=-5mm, enhanced, lifted shadow={0mm}{0mm}{0mm}{0mm}%
{black!60!white}, attach boxed title to top right={xshift=-0.3cm, yshift*=-2mm}, coltitle=sacado_red!85!black, colback=white!100!white, boxed title style={colback=white}, colframe=sacado_red!100!black,title= {\footnotesize  #1}  ]
 
\hspace{-1.3cm} 
\begin{minipage}[t]{0.7cm}

 \begin{tikzpicture}
 	\node[fill=sacado_red,minimum width=0.7cm]{\textcolor{white}{\bf {\Large \theexo}}};
 \end{tikzpicture}


%%%%%%%%%%%%%%%%%%%%%%%% Condition pour la calculatrice
 \ifthenelse{\equal{#3}{1}}{
 \begin{tikzpicture}
 	\node[minimum width=0.7cm]{\includegraphics[scale=0.5]{MISC/calculator.png} };
 \end{tikzpicture} 
 }{
 \ifthenelse{\equal{#3}{2}}{
 \begin{tikzpicture}
 	\node[minimum width=0.7cm]{\includegraphics[scale=0.5]{MISC/no_calculator.png} };
 \end{tikzpicture} 
 }{}
 }

\end{minipage}
\hfill
\begin{minipage}[t]{17.3cm}
} 
{ 
\end{minipage}%code  après
\hfill
\begin{minipage}[t]{1cm}

\begin{center}
\colorbox{sacado_red}{\includegraphics[height=1cm]{qrcodes/qrDummy.png}}
\colorbox{white}{ {\footnotesize /b/ABCD} }
\end{center}

\end{minipage}

\end{tcolorbox}
\par
}  

 
 
 

%%%%%%%%%%%%% ExoCu 7 paramètres : compétences , qrcode , calculatrice, python, scratch, tableur, annales
\newenvironment{ExoAuto}[7]{% code avant
\tcbset{top=-0.2cm }
\stepcounter{exo}

\begin{tcolorbox}[right=-5mm, enhanced, lifted shadow={0mm}{0mm}{0mm}{0mm}%
{black!60!white}, attach boxed title to top right={xshift=-0.3cm, yshift*=-2mm}, coltitle=sacado_orange!85!black, colback=white!100!white, boxed title style={colback=white}, colframe=sacado_orange!100!black,title= {\footnotesize  #1}  ]
 
\hspace{-1.3cm} 
\begin{minipage}[t]{0.7cm}

 \begin{tikzpicture}
 	\node[fill=sacado_orange,minimum width=0.7cm]{\textcolor{white}{\bf {\Large \theexo}}};
 \end{tikzpicture}


%%%%%%%%%%%%%%%%%%%%%%%% Condition pour la calculatrice
 \ifthenelse{\equal{#3}{1}}{
 \begin{tikzpicture}
 	\node[minimum width=0.7cm]{\includegraphics[scale=0.5]{MISC/calculator.png} };
 \end{tikzpicture} 
 }{
 \ifthenelse{\equal{#3}{2}}{
 \begin{tikzpicture}
 	\node[minimum width=0.7cm]{\includegraphics[scale=0.5]{MISC/no_calculator.png} };
 \end{tikzpicture} 
 }{}
 }

\end{minipage}
\hfill
\begin{minipage}[t]{17.3cm}
} 
{ 
\end{minipage}%code  après
\hfill
\begin{minipage}[t]{1cm}

\begin{center}
\colorbox{sacado_orange}{\includegraphics[height=1cm]{qrcodes/qrDummy.png}}
\colorbox{white}{  {\footnotesize /b/ABCD}  }
\end{center}

\end{minipage}

\end{tcolorbox}
\par
}  

 

%%%%%%%%%%%%% ExoDec 7 paramètres : compétences , qrcode , calculatrice, python, scratch, tableur, annales
 
\newenvironment{ExoDec}[6]{% code avant
\tcbset{top=-0.2cm }
\stepcounter{exo}

\begin{tcolorbox}[right=-5mm, enhanced, lifted shadow={0mm}{0mm}{0mm}{0mm}%
{black!60!white}, attach boxed title to top right={xshift=-0.3cm, yshift*=-2mm}, coltitle=sacado_violet!85!black, colback=white!100!white, boxed title style={colback=white}, colframe=sacado_violet!100!black,title= {\footnotesize  #1}  ]
 
\hspace{-1.3cm} 
\begin{minipage}[t]{1cm}

 \begin{tikzpicture}
 	\node[fill=sacado_violet,minimum width=0.7cm]{\textcolor{white}{\bf {\Large \theexo}}};
 \end{tikzpicture}


%%%%%%%%%%%%%%%%%%%%%%%% Condition pour la calculatrice
 \ifthenelse{\equal{#3}{1}}{
 \begin{tikzpicture}
 	\node[minimum width=0.7cm]{\includegraphics[scale=0.5]{MISC/calculator.png} };
 \end{tikzpicture} 
 }{
 \ifthenelse{\equal{#3}{2}}{
 \begin{tikzpicture}
 	\node[minimum width=0.7cm]{\includegraphics[scale=0.5]{MISC/no_calculator.png} };
 \end{tikzpicture} 
 }{}
 }

\end{minipage}
\begin{minipage}[t]{17.3cm}
} 
{ 
\end{minipage}
\end{tcolorbox}
\par
}  
%%%%%%%%%%%%%%%%%%%%%%%%%%%%%%%%%%%%%%%%%%%%%%%%%%%%%%%%%%%%%%%%%%%%%%%%%%%%%%%%%%%%%%%%%%%%%%%%%%%%%%%%%%%%%%%%%%%%%%
%%%%%%%%%%%%%%%%%%%%%%%%%%%%%%%%%%%%%%%%%%%%%%%%%%%%%%%%%%%%%%%%%%%%%%%%%%%%%%%%%%%%%%%%%%%%%%%%%%%%%%%%%%%%%%%%%%%%%%
%%%%%%%%%%%%%%%%  Exercices sans qrcode                                %%%%%%%%%%%%%%%%%%%%%%%%%%%%%%%%%%%%%%%%%%%%%%%
%%%%%%%%%%%%%%%%%%%%%%%%%%%%%%%%%%%%%%%%%%%%%%%%%%%%%%%%%%%%%%%%%%%%%%%%%%%%%%%%%%%%%%%%%%%%%%%%%%%%%%%%%%%%%%%%%%%%%%
%%%%%%%%%%%%%%%%%%%%%%%%%%%%%%%%%%%%%%%%%%%%%%%%%%%%%%%%%%%%%%%%%%%%%%%%%%%%%%%%%%%%%%%%%%%%%%%%%%%%%%%%%%%%%%%%%%%%%%
 
 
 
%%%%%%%%%%%%% ExoCad 7 paramètres : Compétences , calculatrice, python, scratch, tableur, annales
\newenvironment{ExoCadN}[6]{% code avant
\tcbset{top=-0.2cm }
\stepcounter{exo}

\begin{tcolorbox}[right=-5mm, enhanced, lifted shadow={0mm}{0mm}{0mm}{0mm}%
{black!60!white}, attach boxed title to top right={xshift=-0.3cm, yshift*=-2mm}, coltitle=sacado_violet!85!black, colback=white!100!white, boxed title style={colback=white}, colframe=sacado_violet!100!black,title= {\footnotesize  #1}  ]
 
\hspace{-1.3cm} 
\begin{minipage}[t]{1cm}

 \begin{tikzpicture}
 	\node[fill=sacado_violet,minimum width=0.7cm]{\textcolor{white}{\bf {\Large \theexo}}};
 \end{tikzpicture}


%%%%%%%%%%%%%%%%%%%%%%%% Condition pour la calculatrice
 \ifthenelse{\equal{#3}{1}}{
 \begin{tikzpicture}
 	\node[minimum width=0.7cm]{\includegraphics[scale=0.5]{MISC/calculator.png} };
 \end{tikzpicture} 
 }{
 \ifthenelse{\equal{#3}{2}}{
 \begin{tikzpicture}
 	\node[minimum width=0.7cm]{\includegraphics[scale=0.5]{MISC/no_calculator.png} };
 \end{tikzpicture} 
 }{}
 }

\end{minipage}
\begin{minipage}[t]{17.3cm}
} 
{ 
\end{minipage}%code  après
\end{tcolorbox}
\par 
}

 
%%%%%%%%%%%%% ExoCu 7 paramètres : Compétences , calculatrice, python, scratch, tableur, annales
\newenvironment{ExoCuN}[6]{% code avant
\tcbset{top=-0.2cm }
\stepcounter{exo}

\begin{tcolorbox}[right=-5mm, enhanced, lifted shadow={0mm}{0mm}{0mm}{0mm}%
{black!60!white}, attach boxed title to top right={xshift=-0.3cm, yshift*=-2mm}, coltitle=sacado_green!85!black, colback=white!100!white, boxed title style={colback=white}, colframe=sacado_green!100!black,title= {\footnotesize  #1}  ]
 
\hspace{-1.3cm} 
\begin{minipage}[t]{1cm}

 \begin{tikzpicture}
 	\node[fill=sacado_green,minimum width=0.7cm]{\textcolor{white}{\bf {\Large \theexo}}};
 \end{tikzpicture}


%%%%%%%%%%%%%%%%%%%%%%%% Condition pour la calculatrice
 \ifthenelse{\equal{#2}{1}}{
 \begin{tikzpicture}
 	\node[minimum width=0.7cm]{\includegraphics[scale=0.5]{MISC/calculator.png} };
 \end{tikzpicture} 
 }{
 \ifthenelse{\equal{#2}{2}}{
 \begin{tikzpicture}
 	\node[minimum width=0.7cm]{\includegraphics[scale=0.5]{MISC/no_calculator.png} };
 \end{tikzpicture} 
 }{}
 }

\end{minipage}
\begin{minipage}[t]{17.3cm}
} 
{ 
\end{minipage}
\end{tcolorbox}
\par
}  


 %%%%%%%%%%%%% ExoCd 6 paramètres : Compétences, calculatrice, python, scratch, tableur, annales
\newenvironment{ExoCdN}[6]{% code avant
\tcbset{top=-0.2cm }
\stepcounter{exo}

\begin{tcolorbox}[right=-5mm, enhanced, lifted shadow={0mm}{0mm}{0mm}{0mm}%
{black!60!white}, attach boxed title to top right={xshift=-0.3cm, yshift*=-2mm}, coltitle=sacado_blue!85!black, colback=white!100!white, boxed title style={colback=white}, colframe=sacado_blue!100!black,title= {\footnotesize  #1}  ]
 
\hspace{-1.3cm} 
\begin{minipage}[t]{1cm}

 \begin{tikzpicture}
 	\node[fill=sacado_blue,minimum width=0.7cm]{\textcolor{white}{\bf {\Large \theexo}}};
 \end{tikzpicture}


%%%%%%%%%%%%%%%%%%%%%%%% Condition pour la calculatrice
 \ifthenelse{\equal{#2}{1}}{
 \begin{tikzpicture}
 	\node[minimum width=0.7cm]{\includegraphics[scale=0.5]{MISC/calculator.png} };
 \end{tikzpicture} 
 }{
 \ifthenelse{\equal{#2}{2}}{
 \begin{tikzpicture}
 	\node[minimum width=0.7cm]{\includegraphics[scale=0.5]{MISC/no_calculator.png} };
 \end{tikzpicture} 
 }{}
 }

\end{minipage}
\begin{minipage}[t]{17.3cm}
} 
{ 
\end{minipage}%code  après
\end{tcolorbox}
\par
}  


%%%%%%%%%%%%% ExoCt 6 paramètres : Compétences,   calculatrice, python, scratch, tableur, annales
\newenvironment{ExoCtN}[6]{% code avant
\tcbset{top=-0.2cm }
\stepcounter{exo}

\begin{tcolorbox}[right=-5mm, enhanced, lifted shadow={0mm}{0mm}{0mm}{0mm}%
{black!60!white}, attach boxed title to top right={xshift=-0.3cm, yshift*=-2mm}, coltitle=sacado_red!85!black, colback=white!100!white, boxed title style={colback=white}, colframe=sacado_red!100!black,title= {\footnotesize  #1}  ]
 
\hspace{-1.3cm} 
\begin{minipage}[t]{1cm}

 \begin{tikzpicture}
 	\node[fill=sacado_red,minimum width=0.7cm]{\textcolor{white}{\bf {\Large \theexo}}};
 \end{tikzpicture}


%%%%%%%%%%%%%%%%%%%%%%%% Condition pour la calculatrice
 \ifthenelse{\equal{#2}{1}}{
 \begin{tikzpicture}
 	\node[minimum width=0.7cm]{\includegraphics[scale=0.5]{MISC/calculator.png} };
 \end{tikzpicture} 
 }{
 \ifthenelse{\equal{#2}{2}}{
 \begin{tikzpicture}
 	\node[minimum width=0.7cm]{\includegraphics[scale=0.5]{MISC/no_calculator.png} };
 \end{tikzpicture} 
 }{}
 }

\end{minipage}
\begin{minipage}[t]{17.3cm}
} 
{ 
\end{minipage}%code  après
\end{tcolorbox}
\par
}  

 
 
 

%%%%%%%%%%%%% ExoCu 7 paramètres : compétences , qrcode , calculatrice, python, scratch, tableur, annales
\newenvironment{ExoAutoN}[6]{% code avant
\tcbset{top=-0.2cm }
\stepcounter{exo}

\begin{tcolorbox}[right=5mm, enhanced, lifted shadow={0mm}{0mm}{0mm}{0mm}%
{black!60!white}, attach boxed title to top right={xshift=-0.3cm, yshift*=-2mm}, coltitle=sacado_orange!85!black, colback=white!100!white, boxed title style={colback=white}, colframe=sacado_orange!100!black,title= {\footnotesize  #1}  ]
 
\hspace{-1.4cm} 
\begin{minipage}[t]{0.7cm}

 \begin{tikzpicture}
 	\node[fill=sacado_orange,minimum width=0.7cm]{\textcolor{white}{\bf {\Large \theexo}}};
 \end{tikzpicture}


%%%%%%%%%%%%%%%%%%%%%%%% Condition pour la calculatrice
 \ifthenelse{\equal{#2}{1}}{
 \begin{tikzpicture}
 	\node[minimum width=0.7cm]{\includegraphics[scale=0.5]{MISC/calculator.png} };
 \end{tikzpicture} 
 }{
 \ifthenelse{\equal{#2}{2}}{
 \begin{tikzpicture}
 	\node[minimum width=0.7cm]{\includegraphics[scale=0.5]{MISC/no_calculator.png} };
 \end{tikzpicture} 
 }{}
 }

\end{minipage}
\hfill
\begin{minipage}[t]{17.3cm}
} 
{ 
\end{minipage}%code  après
\end{tcolorbox}
\par
}  
%%%%%%%%%%%%%%%%%%%%%%%%%%%%%%%%%%%%%%%%%%%%%%%%%%%%%%%%%%%%%%%%%%%%%%%%%%%%%%%%%%%%%%%%%%%%%%%%%%%%%%%%%%%%%%%%%%%%%%
%%%%%%%%%%%%%%%%%%%%%%%%%%%%%%%%%%%%%%%%%%%%%%%%%%%%%%%%%%%%%%%%%%%%%%%%%%%%%%%%%%%%%%%%%%%%%%%%%%%%%%%%%%%%%%%%%%%%%%
%%%%%%%%%%%%%%%%  Exercices   sans contours                            %%%%%%%%%%%%%%%%%%%%%%%%%%%%%%%%%%%%%%%%%%%%%%%
%%%%%%%%%%%%%%%%%%%%%%%%%%%%%%%%%%%%%%%%%%%%%%%%%%%%%%%%%%%%%%%%%%%%%%%%%%%%%%%%%%%%%%%%%%%%%%%%%%%%%%%%%%%%%%%%%%%%%%
%%%%%%%%%%%%%%%%%%%%%%%%%%%%%%%%%%%%%%%%%%%%%%%%%%%%%%%%%%%%%%%%%%%%%%%%%%%%%%%%%%%%%%%%%%%%%%%%%%%%%%%%%%%%%%%%%%%%%%


\newcommand{\Sf}[1]{ \vspace{0.1cm}
{\color{fond}{\Large \textbf{#1}}  } 
} 

\newcommand{\Sfe}[1]{ \vspace{0.1cm}
{\color{sacado_blue}{\Large \textbf{#1}}  } 
} 
% fin de la procédure



%%%%%%%%%%%%% Pointillés ou ligne
\newcommand{\point}[1]{\vspace{0.1cm}\multido{}{#1}{ \dotfill \medskip \endgraf}}
\newcommand{\ligne}[1]{\vspace{0.1cm}\multido{}{#1}{ {\color{cqcqcq}\hrulefill} \medskip \endgraf}}
%----------------------------------------
%
%   Macros et opérateurs
%
%----------------------------------------

\newcommand{\second}{2\up{d}\xspace}
\newcommand{\seconde}{2\up{de}\xspace}
\newcommand{\R}{\mathbb R}
\newcommand{\Rp}{\R_+}
\newcommand{\Rpe}{\R_+^*}
\newcommand{\Rm}{\R_-}
\newcommand{\Rme}{\R_-^*}
\newcommand{\N}{\mathbb N}
\newcommand{\D}{\mathbb D}
\newcommand{\Q}{\mathbb Q}
\newcommand{\Z}{\mathbb Z}
\newcommand{\C}{\mathbb C}
\newcommand{\grs}{\mathfrak S}
\newcommand{\IN}[1]{\llbracket 1,#1\rrbracket}
\newcommand{\card}{\text{Card}\,}
\usepackage{mathrsfs}
\newcommand{\parties}{\mathscr P}
\renewcommand{\epsilon}{\varepsilon}
\newcommand{\rmd}{\text{d}}
\newcommand{\diff}{\mathrm D}
\newcommand{\Id}{{\rm Id}}
\newcommand{\e}{{\rm e}}
\newcommand{\I}{{\rm i}}
\newcommand{\J}{{\rm j}}
\newcommand{\ro}{\circ}
\newcommand{\exu}{\exists\,!\,}
\newcommand{\telq}{\,\, \mid \,\,}
\newcommand{\para}{\raisebox{0.1em}{\text{\footnotesize /\hspace{-0.1em}/}}}   
\newcommand{\vect}[1]{\overrightarrow{#1}}
\newcommand{\scal}[2]{\left(\, #1 \mid #2 \, \right)}
\newcommand{\ortho}[1]{{#1}^\perp}
\newcommand{\veci}{\vec{\text{\it \i}}}
\newcommand{\vecj}{\vec{\text{\it \j}}}
\newcommand{\rep}{$(O;\veci,\vecj,\vec{k})$\xspace}
\newcommand{\Oijk}{$(O, \veci,\vecj,\vec{k})$\xspace}
\newcommand{\rond}{repère orthonormal direct}
\newcommand{\bond}{base orthonormale directe}
\newcommand{\eq}{\Longleftrightarrow}
\newcommand{\implique}{\Longrightarrow}
\newcommand{\noneq}{\ \ \ /\hspace{-1.45em}\eq}
\newcommand{\tend}{\longrightarrow}
\newcommand{\egx}[2]{\underset{#1 \tend #2}=}
\newcommand{\asso}{\longmapsto}
\newcommand{\vers}{\longrightarrow}
\newcommand{\eqn}{~\underset{n \rightarrow \infty}{\sim}~}
\newcommand{\eqx}[2]{~\underset{#1 \rightarrow #2}{\sim}~}

\newcommand{\egn}{~\underset{n \rightarrow \infty}{=}~}
\renewcommand{\descriptionlabel}{\hspace{\labelsep}$\bullet$}
\renewcommand{\bar}{\overline}
\DeclareMathOperator{\ash}{{Argsh}}
\DeclareMathOperator{\cotan}{{cotan}}
\DeclareMathOperator{\ach}{{Argch}}
\DeclareMathOperator{\ath}{{Argth}}
\DeclareMathOperator{\sh}{{sh}}
\DeclareMathOperator{\ch}{{ch}}
\DeclareMathOperator{\Mat}{{Mat}}
\DeclareMathOperator{\Vect}{{Vect}}
\DeclareMathOperator{\trace}{{tr}}
\newcommand{\tr}{{}^{\mathrm t}}
\newcommand{\divi}{~\big|~}
\newcommand{\ndivi}{~\not{\big|}~}
\newcommand{\et}{\wedge}
\newcommand{\ou}{\vee}
\renewcommand{\det}{\operatorname{\text{dét}}}
\DeclareMathOperator{\grad}{{grad}}
\renewcommand{\arcsin}{{\mathop{\mathrm{Arcsin}}}}
\renewcommand{\arccos}{{\mathop{\mathrm{Arccos}}}}
\renewcommand{\arctan}{{\mathop{\mathrm{Arctan}}}}
\renewcommand{\tanh}{{\mathop{\mathrm{th}}}}
\newcommand{\pgcd}{\mathop{\mathrm{pgcd}}}
\newcommand{\ppcm}{\mathop{\mathrm{ppcm}}}
\newcommand{\fonc}[4]{\left\{\begin{tabular}{ccc}$#1$ & $\vers$ & $#2$ \\
$#3$ & $\asso$ & $#4$ \end{tabular}\right.}
\renewcommand{\geq}{\geqslant}
\renewcommand{\leq}{\leqslant}
\renewcommand{\Re}{\text{\rm Re}}
\renewcommand{\Im}{\text{\rm Im}}
\renewcommand{\ker}{\mathop{\mathrm{Ker}}}
\newcommand{\Lin}{\mathcal L}
\newcommand{\GO}{\mathcal O}
\newcommand{\GSO}{\mathcal{SO}}
\newcommand{\GL}{\mathcal{GL}}
\renewcommand{\emptyset}{\varnothing}
%\newcommand{\arc}[1]{\overset{\frown}{#1}}
\newcommand{\rg}{\mathop{\mathrm{rg}}}
\newcommand{\ds}{\displaystyle}
\newcommand{\co}[3]{\begin{pmatrix}#1 \\ #2 \\ #3\end{pmatrix}}
\newcommand{\demi}{\frac 1 2}
\newcommand{\limi}[2]{\underset{#1 \rightarrow #2}\lim}
%\usetikzlibrary{quotes,arrows.meta}
% Pour les pavés droits sans annotations de hLl 
\tikzset{
  nonannotated cuboid/.pic={
    \tikzset{%
      every edge quotes/.append style={midway, auto},
      /cuboid/.cd,
      #1
    }
    \draw [every edge/.append style={pic actions, densely dashed, opacity=.5}, pic actions]
    (0,0,0) coordinate (o) -- ++(-\cubescale*\cubex,0,0) coordinate (a) -- ++(0,-\cubescale*\cubey,0) coordinate (b) edge coordinate [pos=1] (g) ++(0,0,-\cubescale*\cubez)  -- ++(\cubescale*\cubex,0,0) coordinate (c) -- cycle
    (o) -- ++(0,0,-\cubescale*\cubez) coordinate (d) -- ++(0,-\cubescale*\cubey,0) coordinate (e) edge (g) -- (c) -- cycle
    (o) -- (a) -- ++(0,0,-\cubescale*\cubez) coordinate (f) edge (g) -- (d) -- cycle;
%    \path [every edge/.append style={pic actions, |-|}]
%    (b) +(0,-5pt) coordinate (b1) edge ["\cubex \cubeunits"'] (b1 -| c)
%    (b) +(-5pt,0) coordinate (b2) edge ["\cubey \cubeunits"] (b2 |- a)
%    (c) +(3.5pt,-3.5pt) coordinate (c2) edge ["\cubez \cubeunits"'] ([xshift=3.5pt,yshift=-3.5pt]e)
    ;
  },
  /cuboid/.search also={/tikz},
  /cuboid/.cd,
  width/.store in=\cubex,
  height/.store in=\cubey,
  depth/.store in=\cubez,
  units/.store in=\cubeunits,
  scale/.store in=\cubescale,
  width=10,
  height=10,
  depth=10,
  units=cm,
  scale=.1,
}

% Pour les pavés droits avec annotations de hLl
\tikzset{
  annotated cuboid/.pic={
    \tikzset{%
      every edge quotes/.append style={midway, auto},
      /cuboid/.cd,
      #1
    }
    \draw [every edge/.append style={pic actions, densely dashed, opacity=.5}, pic actions]
    (0,0,0) coordinate (o) -- ++(-\cubescale*\cubex,0,0) coordinate (a) -- ++(0,-\cubescale*\cubey,0) coordinate (b) edge coordinate [pos=1] (g) ++(0,0,-\cubescale*\cubez)  -- ++(\cubescale*\cubex,0,0) coordinate (c) -- cycle
    (o) -- ++(0,0,-\cubescale*\cubez) coordinate (d) -- ++(0,-\cubescale*\cubey,0) coordinate (e) edge (g) -- (c) -- cycle
    (o) -- (a) -- ++(0,0,-\cubescale*\cubez) coordinate (f) edge (g) -- (d) -- cycle;
    \path [every edge/.append style={pic actions, |-|}]
    (b) +(0,-5pt) coordinate (b1) edge ["\cubex \cubeunits"'] (b1 -| c)
    (b) +(-5pt,0) coordinate (b2) edge ["\cubey \cubeunits"] (b2 |- a)
    (c) +(3.5pt,-3.5pt) coordinate (c2) edge ["\cubez \cubeunits"'] ([xshift=3.5pt,yshift=-3.5pt]e)
    ;
  },
  /cuboid/.search also={/tikz},
  /cuboid/.cd,
  width/.store in=\cubex,
  height/.store in=\cubey,
  depth/.store in=\cubez,
  units/.store in=\cubeunits,
  scale/.store in=\cubescale,
  width=10,
  height=10,
  depth=10,
  units=cm,
  scale=.1,
}

% Pour les pavés droits avec annotations "hauteur", "Longueur", "largeur".
\tikzset{
  defannotated cuboid/.pic={
    \tikzset{%
      every edge quotes/.append style={midway, auto},
      /cuboid/.cd,
      #1
    }
    \draw [every edge/.append style={pic actions, densely dashed, opacity=.5}, pic actions]
    (0,0,0) coordinate (o) -- ++(-\cubescale*\cubex,0,0) coordinate (a) -- ++(0,-\cubescale*\cubey,0) coordinate (b) edge coordinate [pos=1] (g) ++(0,0,-\cubescale*\cubez)  -- ++(\cubescale*\cubex,0,0) coordinate (c) -- cycle
    (o) -- ++(0,0,-\cubescale*\cubez) coordinate (d) -- ++(0,-\cubescale*\cubey,0) coordinate (e) edge (g) -- (c) -- cycle
    (o) -- (a) -- ++(0,0,-\cubescale*\cubez) coordinate (f) edge (g) -- (d) -- cycle;
    \path [every edge/.append style={pic actions, |-|}]
    (b) +(0,-5pt) coordinate (b1) edge ["Longueur"'] (b1 -| c)
    (b) +(-5pt,0) coordinate (b2) edge ["Hauteur"] (b2 |- a)
    (c) +(3.5pt,-3.5pt) coordinate (c2) edge ["largeur"'] ([xshift=3.5pt,yshift=-3.5pt]e)
    ;
  },
  /cuboid/.search also={/tikz},
  /cuboid/.cd,
  width/.store in=\cubex,
  height/.store in=\cubey,
  depth/.store in=\cubez,
  units/.store in=\cubeunits,
  scale/.store in=\cubescale,
  width=10,
  height=10,
  depth=10,
  units=cm,
  scale=.1,
}

\tikzset{
  defiannotated cuboid/.pic={
    \tikzset{%
      every edge quotes/.append style={midway, auto},
      /cuboid/.cd,
      #1
    }
    \draw [every edge/.append style={pic actions, densely dashed, opacity=.5}, pic actions]
    (0,0,0) coordinate (o) -- ++(-\cubescale*\cubex,0,0) coordinate (a) -- ++(0,-\cubescale*\cubey,0) coordinate (b) edge coordinate [pos=1] (g) ++(0,0,-\cubescale*\cubez)  -- ++(\cubescale*\cubex,0,0) coordinate (c) -- cycle
    (o) -- ++(0,0,-\cubescale*\cubez) coordinate (d) -- ++(0,-\cubescale*\cubey,0) coordinate (e) edge (g) -- (c) -- cycle
    (o) -- (a) -- ++(0,0,-\cubescale*\cubez) coordinate (f) edge (g) -- (d) -- cycle;
    \path [every edge/.append style={pic actions, |-|}]
    (b) +(0,-5pt) coordinate (b1) edge ["c"'] (b1 -| c)
    (b) +(-5pt,0) coordinate (b2) edge ["c"] (b2 |- a)
    (c) +(3.5pt,-3.5pt) coordinate (c2) edge ["c"'] ([xshift=3.5pt,yshift=-3.5pt]e)
    ;
  },
  /cuboid/.search also={/tikz},
  /cuboid/.cd,
  width/.store in=\cubex,
  height/.store in=\cubey,
  depth/.store in=\cubez,
  units/.store in=\cubeunits,
  scale/.store in=\cubescale,
  width=10,
  height=10,
  depth=10,
  units=cm,
  scale=.1,
}

\tikzset{
  defiiannotated cuboid/.pic={
    \tikzset{%
      every edge quotes/.append style={midway, auto},
      /cuboid/.cd,
      #1
    }
    \draw [every edge/.append style={pic actions, densely dashed, opacity=.5}, pic actions]
    (0,0,0) coordinate (o) -- ++(-\cubescale*\cubex,0,0) coordinate (a) -- ++(0,-\cubescale*\cubey,0) coordinate (b) edge coordinate [pos=1] (g) ++(0,0,-\cubescale*\cubez)  -- ++(\cubescale*\cubex,0,0) coordinate (c) -- cycle
    (o) -- ++(0,0,-\cubescale*\cubez) coordinate (d) -- ++(0,-\cubescale*\cubey,0) coordinate (e) edge (g) -- (c) -- cycle
    (o) -- (a) -- ++(0,0,-\cubescale*\cubez) coordinate (f) edge (g) -- (d) -- cycle;
    \path [every edge/.append style={pic actions, |-|}]
    (b) +(0,-5pt) coordinate (b1) edge ["L"'] (b1 -| c)
    (b) +(-5pt,0) coordinate (b2) edge ["h"] (b2 |- a)
    (c) +(3.5pt,-3.5pt) coordinate (c2) edge ["l"'] ([xshift=3.5pt,yshift=-3.5pt]e)
    ;
  },
  /cuboid/.search also={/tikz},
  /cuboid/.cd,
  width/.store in=\cubex,
  height/.store in=\cubey,
  depth/.store in=\cubez,
  units/.store in=\cubeunits,
  scale/.store in=\cubescale,
  width=10,
  height=10,
  depth=10,
  units=cm,
  scale=.1,
}

\tikzset{
  xannotated cuboid/.pic={
    \tikzset{%
      every edge quotes/.append style={midway, auto},
      /cuboid/.cd,
      #1
    }
    \draw [every edge/.append style={pic actions, densely dashed, opacity=.5}, pic actions]
    (0,0,0) coordinate (o) -- ++(-\cubescale*\cubex,0,0) coordinate (a) -- ++(0,-\cubescale*\cubey,0) coordinate (b) edge coordinate [pos=1] (g) ++(0,0,-\cubescale*\cubez)  -- ++(\cubescale*\cubex,0,0) coordinate (c) -- cycle
    (o) -- ++(0,0,-\cubescale*\cubez) coordinate (d) -- ++(0,-\cubescale*\cubey,0) coordinate (e) edge (g) -- (c) -- cycle
    (o) -- (a) -- ++(0,0,-\cubescale*\cubez) coordinate (f) edge (g) -- (d) -- cycle;
    \path [every edge/.append style={pic actions, |-|}]
    (b) +(0,-5pt) coordinate (b1) edge ["1cm"'] (b1 -| c)
    (b) +(-5pt,0) coordinate (b2) edge ["1cm"] (b2 |- a)
    (c) +(3.5pt,-3.5pt) coordinate (c2) edge ["1cm"'] ([xshift=3.5pt,yshift=-3.5pt]e)
    ;
  },
  /cuboid/.search also={/tikz},
  /cuboid/.cd,
  width/.store in=\cubex,
  height/.store in=\cubey,
  depth/.store in=\cubez,
  units/.store in=\cubeunits,
  scale/.store in=\cubescale,
  width=10,
  height=10,
  depth=10,
  units=cm,
  scale=.1,
}

\tikzset{
  yannotated cuboid/.pic={
    \tikzset{%
      every edge quotes/.append style={midway, auto},
      /cuboid/.cd,
      #1
    }
    \draw [every edge/.append style={pic actions, densely dashed, opacity=.5}, pic actions]
    (0,0,0) coordinate (o) -- ++(-\cubescale*\cubex,0,0) coordinate (a) -- ++(0,-\cubescale*\cubey,0) coordinate (b) edge coordinate [pos=1] (g) ++(0,0,-\cubescale*\cubez)  -- ++(\cubescale*\cubex,0,0) coordinate (c) -- cycle
    (o) -- ++(0,0,-\cubescale*\cubez) coordinate (d) -- ++(0,-\cubescale*\cubey,0) coordinate (e) edge (g) -- (c) -- cycle
    (o) -- (a) -- ++(0,0,-\cubescale*\cubez) coordinate (f) edge (g) -- (d) -- cycle;
    \path [every edge/.append style={pic actions, |-|}]
    (b) +(0,-5pt) coordinate (b1) edge ["1mm"'] (b1 -| c)
    (b) +(-5pt,0) coordinate (b2) edge ["1mm"] (b2 |- a)
    (c) +(3.5pt,-3.5pt) coordinate (c2) edge ["1mm"'] ([xshift=3.5pt,yshift=-3.5pt]e)
    ;
  },
  /cuboid/.search also={/tikz},
  /cuboid/.cd,
  width/.store in=\cubex,
  height/.store in=\cubey,
  depth/.store in=\cubez,
  units/.store in=\cubeunits,
  scale/.store in=\cubescale,
  width=10,
  height=10,
  depth=10,
  units=cm,
  scale=.1,
}

%Cubes subdivisés
\newif\ifcuboidshade
\newif\ifcuboidemphedge

\tikzset{
  cuboid/.is family,
  cuboid,
  shiftx/.initial=0,
  shifty/.initial=0,
  dimx/.initial=3,
  dimy/.initial=3,
  dimz/.initial=3,
  scale/.initial=1,
  densityx/.initial=1,
  densityy/.initial=1,
  densityz/.initial=1,
  rotation/.initial=0,
  anglex/.initial=0,
  angley/.initial=90,
  anglez/.initial=225,
  scalex/.initial=1,
  scaley/.initial=1,
  scalez/.initial=0.5,
  front/.style={draw=black,fill=white},
  top/.style={draw=black,fill=white},
  right/.style={draw=black,fill=white},
  shade/.is if=cuboidshade,
  shadecolordark/.initial=black,
  shadecolorlight/.initial=white,
  shadeopacity/.initial=0.15,
  shadesamples/.initial=16,
  emphedge/.is if=cuboidemphedge,
  emphstyle/.style={thick},
}

\newcommand{\tikzcuboidkey}[1]{\pgfkeysvalueof{/tikz/cuboid/#1}}

% Commands
\newcommand{\tikzcuboid}[1]{
    \tikzset{cuboid,#1} % Process Keys passed to command
  \pgfmathsetlengthmacro{\vectorxx}{\tikzcuboidkey{scalex}*cos(\tikzcuboidkey{anglex})*28.452756}
  \pgfmathsetlengthmacro{\vectorxy}{\tikzcuboidkey{scalex}*sin(\tikzcuboidkey{anglex})*28.452756}
  \pgfmathsetlengthmacro{\vectoryx}{\tikzcuboidkey{scaley}*cos(\tikzcuboidkey{angley})*28.452756}
  \pgfmathsetlengthmacro{\vectoryy}{\tikzcuboidkey{scaley}*sin(\tikzcuboidkey{angley})*28.452756}
  \pgfmathsetlengthmacro{\vectorzx}{\tikzcuboidkey{scalez}*cos(\tikzcuboidkey{anglez})*28.452756}
  \pgfmathsetlengthmacro{\vectorzy}{\tikzcuboidkey{scalez}*sin(\tikzcuboidkey{anglez})*28.452756}
  \begin{scope}[xshift=\tikzcuboidkey{shiftx}, yshift=\tikzcuboidkey{shifty}, scale=\tikzcuboidkey{scale}, rotate=\tikzcuboidkey{rotation}, x={(\vectorxx,\vectorxy)}, y={(\vectoryx,\vectoryy)}, z={(\vectorzx,\vectorzy)}]
    \pgfmathsetmacro{\steppingx}{1/\tikzcuboidkey{densityx}}
  \pgfmathsetmacro{\steppingy}{1/\tikzcuboidkey{densityy}}
  \pgfmathsetmacro{\steppingz}{1/\tikzcuboidkey{densityz}}
  \newcommand{\dimx}{\tikzcuboidkey{dimx}}
  \newcommand{\dimy}{\tikzcuboidkey{dimy}}
  \newcommand{\dimz}{\tikzcuboidkey{dimz}}
  \pgfmathsetmacro{\secondx}{2*\steppingx}
  \pgfmathsetmacro{\secondy}{2*\steppingy}
  \pgfmathsetmacro{\secondz}{2*\steppingz}
  \foreach \x in {\steppingx,\secondx,...,\dimx}
  { \foreach \y in {\steppingy,\secondy,...,\dimy}
    {   \pgfmathsetmacro{\lowx}{(\x-\steppingx)}
      \pgfmathsetmacro{\lowy}{(\y-\steppingy)}
      \filldraw[cuboid/front] (\lowx,\lowy,\dimz) -- (\lowx,\y,\dimz) -- (\x,\y,\dimz) -- (\x,\lowy,\dimz) -- cycle;
    }
    }
  \foreach \x in {\steppingx,\secondx,...,\dimx}
  { \foreach \z in {\steppingz,\secondz,...,\dimz}
    {   \pgfmathsetmacro{\lowx}{(\x-\steppingx)}
      \pgfmathsetmacro{\lowz}{(\z-\steppingz)}
      \filldraw[cuboid/top] (\lowx,\dimy,\lowz) -- (\lowx,\dimy,\z) -- (\x,\dimy,\z) -- (\x,\dimy,\lowz) -- cycle;
        }
    }
    \foreach \y in {\steppingy,\secondy,...,\dimy}
  { \foreach \z in {\steppingz,\secondz,...,\dimz}
    {   \pgfmathsetmacro{\lowy}{(\y-\steppingy)}
      \pgfmathsetmacro{\lowz}{(\z-\steppingz)}
      \filldraw[cuboid/right] (\dimx,\lowy,\lowz) -- (\dimx,\lowy,\z) -- (\dimx,\y,\z) -- (\dimx,\y,\lowz) -- cycle;
    }
  }
  \ifcuboidemphedge
    \draw[cuboid/emphstyle] (0,\dimy,0) -- (\dimx,\dimy,0) -- (\dimx,\dimy,\dimz) -- (0,\dimy,\dimz) -- cycle;%
    \draw[cuboid/emphstyle] (0,\dimy,\dimz) -- (0,0,\dimz) -- (\dimx,0,\dimz) -- (\dimx,\dimy,\dimz);%
    \draw[cuboid/emphstyle] (\dimx,\dimy,0) -- (\dimx,0,0) -- (\dimx,0,\dimz);%
    \fi

    \ifcuboidshade
    \pgfmathsetmacro{\cstepx}{\dimx/\tikzcuboidkey{shadesamples}}
    \pgfmathsetmacro{\cstepy}{\dimy/\tikzcuboidkey{shadesamples}}
    \pgfmathsetmacro{\cstepz}{\dimz/\tikzcuboidkey{shadesamples}}
    \foreach \s in {1,...,\tikzcuboidkey{shadesamples}}
    {   \pgfmathsetmacro{\lows}{\s-1}
        \pgfmathsetmacro{\cpercent}{(\lows)/(\tikzcuboidkey{shadesamples}-1)*100}
        \fill[opacity=\tikzcuboidkey{shadeopacity},color=\tikzcuboidkey{shadecolorlight}!\cpercent!\tikzcuboidkey{shadecolordark}] (0,\s*\cstepy,\dimz) -- (\s*\cstepx,\s*\cstepy,\dimz) -- (\s*\cstepx,0,\dimz) -- (\lows*\cstepx,0,\dimz) -- (\lows*\cstepx,\lows*\cstepy,\dimz) -- (0,\lows*\cstepy,\dimz) -- cycle;
        \fill[opacity=\tikzcuboidkey{shadeopacity},color=\tikzcuboidkey{shadecolorlight}!\cpercent!\tikzcuboidkey{shadecolordark}] (0,\dimy,\s*\cstepz) -- (\s*\cstepx,\dimy,\s*\cstepz) -- (\s*\cstepx,\dimy,0) -- (\lows*\cstepx,\dimy,0) -- (\lows*\cstepx,\dimy,\lows*\cstepz) -- (0,\dimy,\lows*\cstepz) -- cycle;
        \fill[opacity=\tikzcuboidkey{shadeopacity},color=\tikzcuboidkey{shadecolorlight}!\cpercent!\tikzcuboidkey{shadecolordark}] (\dimx,0,\s*\cstepz) -- (\dimx,\s*\cstepy,\s*\cstepz) -- (\dimx,\s*\cstepy,0) -- (\dimx,\lows*\cstepy,0) -- (\dimx,\lows*\cstepy,\lows*\cstepz) -- (\dimx,0,\lows*\cstepz) -- cycle;
    }
    \fi 

  \end{scope}
}

%\makeatother


\title{Mathématiques 2  : le livre sacado}
\author{L'équipe SACADO}

\begin{document}

\maketitle

%\ \\
\vspace{0.5cm}
\ \\

\begin{flushright}
{\huge {\color{bleu3} {\sffamily Codification des exercices}}}

{\color{bleu3}\rule{0.6\linewidth}{0.4pt}}
\end{flushright}

\vspace{2cm}

\begin{description}
\item \tikz\node[rounded corners=0pt,draw,fill=bleu2]{\color{white}\textbf{ $n$}}; \quad  {\color{bleu2}\textbf{ Application directe}}

Les applications directes portent uniquement sur les capacités et compétences annoncés en entête de la notion. Ils permettent de comprendre et de mettre en œuvre les acquisitions de base.


\item \tikz\node[rounded corners=0pt,draw,fill=bleu2]{\color{white}\textbf{ $n$}}; \quad  {\color{bleu2}\textbf{ Exercice d'application}}

Les exercices d'applications portent principalement sur les capacités et compétences annoncées en entête de la notion mais mettent en jeu des savoir faire et des notions antérieures. Ils renforcent leur compréhension.

\item \tikz\node[rounded corners=0pt,draw,fill=bleu2]{\color{white}\textbf{$n$}}; \quad  {\color{bleu2}\textbf{ Exercice de découverte}}

Les activités de découverte permettent la découverte d'une notion encore inconnue et sont axées sur des situations problèmes, situations où a priori les outils de résolution ne sont pas encore déterminés, ni connus.

\item \tikz\node[rounded corners=0pt,draw,fill=bleu3]{\color{white}\textbf{ $n$}}; \quad  {\color{bleu3}\textbf{ Situation de recherche}}

Les activités de recherche amènent une notion encore inconnue et sont axées sur des situation problèmes, situation où à priori les outils de résolution ne sont pas encore déterminés, ni connus. Ces activités ont pour objectif la construction de savoir, savoir faire.

\item \tikz\node[rounded corners=0pt,draw,fill=bleu3]{\color{white}\textbf{ $n$}}; \quad  {\color{bleu3}\textbf{ Défi}}

Les défis ou problèmes ouverts sont des activités où le mode de résolution et opératoire n'est pas défini. C'est à l'élève de prendre toutes les initiatives pour prendre le chemin de la solution. La résolution n'est pas nécessairement attendue. La recherche est primordiale. La prise d'initiative est valorisée.

\item \tikz\node[rounded corners=0pt,draw,fill=black]{\color{white}\textbf{ $n$}}; \quad  {\color{black}\textbf{ Approfondissement}}

Ces activités sont plus difficiles et demandent la synthèse de nombreuses compétences.

\item \tikz\node[rounded corners=0pt,draw,fill=orange]{\color{white}\textbf{ $n$}}; \quad  {\color{orange}\textbf{  Activité Scratch}}

Activités dont le support est le logiciel Scratch

\item \tikz\node[rounded corners=0pt,draw,fill=yellow]{\color{blue}\textbf{ $n$}}; \quad  {\color{blue}\textbf{  Activité Python}}

Activités dont le support est le logiciel Python


\item \tikz\node[rounded corners=0pt,draw,fill=red]{\color{white}\textbf{ $n$}}; \quad  {\color{red}\textbf{ Compte rendu}}

Outre les compétences purement disciplinaires, ces activités mettent en pratique les compétences suivantes :
\begin{description}[leftmargin=*]
\item[•] Faire le lien entre le langage naturel et le langage algébrique. Distinguer des spécificités du langage mathématique par rapport à la langue française.
\item[•] Expliquer à l'oral ou à l'écrit (sa démarche, son raisonnement, un calcul, un protocole de construction géométrique, un algorithme), comprendre les explications d'un autre et argumenter dans l'échange.
\end{description}

\item \tikz\node[rounded corners=0pt,draw,fill=vert]{\color{white}\textbf{ $n$}}; \quad  {\color{vert}\textbf{ Parcours}}

Ces activités sont en lien avec un des 3 parcours éducatifs du cycle :
\begin{description}
\item[•] Enseignement moral et civique.
\item[•] Avenir
\item[•] Éducation artistique et culturelle
\end{description}

\item \tikz\node[rounded corners=0pt,draw,fill=violet]{\color{white}\textbf{ $n$}}; \quad {\color{violet}\textbf{ En ligne}}

En ligne propose un lien extérieur vers un support numérique :Géogébra, Youtube, Python, Scratch, liens externes.


\item \tikz\node[rounded corners=2pt,draw,fill=red]{\color{white}\textbf{ $n$}}; \quad {\color{red}\textbf{ Question Flash}}

Ces questions sont à faire rapidement, certaines à l'oral ou en activité mentale. Elles peuvent amener des automatismes de résolution.

\end{description}




%\include{00-progression_detaillee}

%\chapter{Les ensembles de nombres}
%%%\begin{titre}[Probabilités]

\Titre{Tests et boucles en Python}{1}
\end{titre}


\begin{CpsCol}
\begin{description}
\item[$\square$] Connaitre la syntaxe de test
\item[$\square$] Connaitre la syntaxe de boucle
\end{description}
\end{CpsCol}

\subsection*{Instruction conditionnelle}

\color{orange} if \color{black}  Condition :

\hspace{0.4cm}    instruction 1
    
\color{orange} else \color{black} :

\hspace{0.4cm}     instruction 2

\subsection*{La boucle for}

La boucle ci dessous génère 5 nombres aléatoires naturels de 1 à 6

\color{orange} import \color{black} random

\color{orange} for \color{black} i \color{orange} in \color{purple}range\color{black}(5):

\hspace{0.4cm}     x = random.randint(1,6)
 
\hspace{0.4cm}   \color{purple} print \color{black}(x)

\subsection*{La boucle While}
\color{orange} import \color{black} random

i=1

\color{orange}  while \color{black} i <= 5:

\hspace{0.4cm}    x = random.randint(1,6)
    
\hspace{0.4cm}   \color{purple} print \color{black}   (x)
   
\hspace{0.4cm} \color{black}     i=i+1
%%
%%   NOMBRES ET CALCULS
%%-----------------------------
%
%\chapter{Ensembles de nombres}
{https://sacado.xyz/qcm/parcours_show_course/0/117129}
{


 \begin{CpsCol}
\textbf{Les savoir-faire du parcours}
 \begin{itemize}
 \item \textbf{Connaitre le vocabulaire ensembliste et logique}
 \item Connaitre les ensembles de nombres
 \item Représenter les ensembles de nombres sur la droite graduée
 \item Identifier les intervalles de $\R$
 \item Lire et écrire des propositions contenant les connecteurs « et », « ou » ;

 \end{itemize}
 \end{CpsCol}


\begin{His}

\textbf{Georg Cantor} est un mathématicien allemand, né le 3 mars 1845 à Saint-Pétersbourg et mort le $6$ janvier $1918$ à Halle.\\ 
Il est connu pour être le créateur de la théorie des ensembles.\\ 
Il prouva que les nombres réels sont "plus nombreux" que les entiers naturels. En fait, le théorème de Cantor implique\\ 
l'existence d'une "infinité d'infinis". Il définit les nombres cardinaux, les nombres ordinaux et leur arithmétique. \\ 
Une partie du travail de Cantor consista à dire que l'infini a plusieurs taille. Et qu'il y a des infini plus grand que d'autres. \\ 
Cantor rejoint ainsi le monde de la philosophie, comme \textbf{Pascal} 3 siècle plus tôt. 

\end{His}

 

\begin{ExoDec}{Chercher.}{1234}{2}{0}{0}{0}

L'\textbf{E}urope est composée de 27 pays dont la \textbf{F}rance, l'\textbf{I}talie et la \textbf{G}rèce. Les capitales respectives sont \textbf{P}aris, \textbf{R}ome et \textbf{A}thènes. 
\begin{enumerate}
\item Un habitant de Rome est-il en Italie ? 
\item Un habitant de Grèce est-il une habitant de Rome ? 
\item Un habitant d'Italie est-il en Europe ?
\item Si je ne suis pas en Europe, puis-je être en France ?
\item Construire un diagramme de Venn qui illustre cette situation.
\end{enumerate}

{\small \textit{Pour faciliter l'écriture, on pourra utilisera l'initiale des noms respectifs : $E$, $F$, $I$, $G$, $P$, $R$ et $A$.}}
\end{ExoDec}


\begin{ExoDec}{Chercher.}{1234}{2}{0}{0}{0}
 
L'INSEE estime qu'un couple avec deux enfants appartient à la classe moyenne quand les revenus du foyer sont situés dans l'intervalle $[3253;5609]$. Monsieur Twicks gagne $2731$ euros et madame Twicks gagne $2952$ euros et ont deux enfants Mary et Paul.

 La famille appartient-elle à la classe moyenne ?

%Le couple gagne donc $s_{global}=2731+2952 = 5683$ euros. On compare cette valeur à l'intervalle de la classe moyenne $[3253;5609]$.
%
%$5683>5609 $ donc $5683 \not \in [3253;5609]$. Le couple n'appartient donc pas à la classe moyenne.
 
 \end{ExoDec}



}

\begin{pageCours}

\section{Ensembles de nombres}

\begin{DefT}{Ensemble des réels}
L'ensemble de tous les nombres connus en Seconde est appelé ensemble des nombres réels, noté $\R$.
\end{DefT}

\begin{DefT}{Ensemble de nombres}\index{Ensemble de nombres!Réels $\R$}
Les autres ensembles de nombres, inclus dans $\R$.
\begin{enumerate}
\item On appelle \textbf{entiers naturels} les nombres : 0 ; 1 ; 2 ; 3 . . . Leur ensemble est noté $\N$.\index{Ensemble de nombres! Entiers naturels $\N$}
On a donc : $\N =  \lbrace 0 ; 1 ; 2 ; 3 \cdots \rbrace $
\item  On appelle \textbf{entiers relatifs} les nombres entiers naturels et leurs symétriques par rapport à 0. Leur ensemble est noté $\Z$.\index{Ensemble de nombres! Entiers $\Z$}. L'utilisation de la lettre $\Z$ vient de l'allemand Zahlen (Chiffre).
On a donc : $\Z = \lbrace \cdots -3 ; -2 ; -1 ; 0 ; 1 ; 2 ; 3  \cdots  \rbrace$
\item  On appelle \textbf{nombres rationnels} les nombres de la forme $\frac{a}{b}$, $a$ et $b$ entiers et $b$ non nuls.  Leur ensemble est noté $\Q$. \index{Ensemble de nombres! Rationnels $\Q$}
On a donc : $\Q = \lbrace \cdots \frac{5}{3} ; -\frac{5}{7} ; -\frac{13}{22} \cdots   \rbrace$. Les réels non rationnels sont dit \textbf{irrationnels} : $\sqrt{2}$
\item Par construction, $\N$ est inclus dans $\Z$  est inclus dans $\D$  est inclus dans $\Q$  est inclus dans $\R$. Ces ensembles sont dits "emboités".
\item  On peut représenter l'ensemble des réels sur une droite graduée.
\begin{center}
\begin{tikzpicture}[line cap=round,line join=round,>=triangle 45,x=1.0cm,y=1.0cm]
\draw[->,color=black] (-4.36,0.) -- (10.66,0.);
\foreach \x in {-4.,-3.,-2.,-1.,1.,2.,3.,4.,5.,6.,7.,8.,9.,10.}
\draw[shift={(\x,0)},color=black] (0pt,2pt) -- (0pt,-2pt) node[below] {\footnotesize $\x$};
\draw[color=black] (0pt,-10pt) node[right] {\footnotesize $0$};
\clip(-4.36,-0.5) rectangle (10.66,0.5);
\end{tikzpicture}
 \end{center} 
\end{enumerate}
\end{DefT}


%\begin{Nt}
%
%Soit $M$ un point d'abscisse $a$, sur la droite graduée.\\
%On note $\vert a\vert$ la distance de $M$ à $O$.
%
%Ainsi, on peut établir que $\vert -a \vert = \vert a \vert$. En effet, deux nombres opposés sont à la même distance de l'origine.
%\end{Nt}

 

\begin{DefT}{Appartenance}\index{Ensemble!Appartenance}
Pour symboliser l'appartenance d'un nombre à un ensemble, on utilise le symbole $\in$ : 
$-2 \in \Z \quad \quad \frac{5}{3} \in \Q $ 
 
 $ \not\in $ signifie n'appartient pas : $-5 \not \in \N \quad \quad \pi \not\in \Q$
\end{DefT}


\begin{Nt}

\begin{description}[leftmargin=*]
\item  On appelle $\R^+$ l'ensemble des réels positifs, $\R^-$ l'ensemble des réels négatifs.
\item  On utilise une étoile pour enlever $0$ d'un ensemble. $\R^*$ l'ensemble des réels non nuls.
\end{description}

\end{Nt}

\begin{DefT}{Nombres décimaux}
Les \textbf{nombres décimaux} sont des nombres rationnels dont le dénominateur est une puissance de 2, de 5 ou de 10 ou un produit de puissances de ces nombres. Les nombres décimaux peuvent donc s'écrire sous la forme $\dfrac{a}{10^n}$, $a \in \Z$ et $n \in \N$.
\end{DefT}

\begin{Ex}
\begin{description}[leftmargin=*]
\item  $\dfrac{3}{25}$ est un nombre décimal car $25 = 5^2$, 25 est donc une puissance de $5$.
\item  $\dfrac{13}{20}$ est un nombre décimal car $20 = 2^2 \times 5$, $20$ est donc un produit d'une puissance de $2$ et de $5$.
\end{description}
\end{Ex}

\begin{Att}

$\dfrac{7}{15}$ n'est pas un nombre décimal. $15=3 \times 5$ n'est pas une puissance de $5$ mais un multiple de $5$ !

\end{Att}






%%\PESP{https://fr.wikipedia.org/wiki/D\%C3\%A9veloppement\_d\%C3\%A9cimal\_p\%C3\%A9riodique}
%%
%%\AV{https://www.youtube.com/watch?v=N_cDA6tF-40}{Conjecture de Cantor}

\end{pageCours} 

\begin{pageAD} 
 

\Sf{Connaitre les ensembles de nombres}
 
  
\begin{ExoCad}{Représenter.}{1234}{2}{0}{0}{0}{0}
 
 \begin{minipage}{0.48\linewidth}
Placer, dans le plus petit ensemble, les nombres suivants :
$$ -4 ~~ ;~~  \sqrt{36}~~  ; ~~ \frac{5}{7} ~~  ;~~  -\sqrt{28}~~ ;~~  \frac{72}{6} ~~  ; ~~ 2,3 ~~  ; ~~ 0 ;~~ -1,785 $$
\end{minipage}
\hfill
\begin{minipage}{0.48\linewidth}

\definecolor{qqqqff}{rgb}{0.,0.,1.}
\definecolor{qqwuqq}{rgb}{0.,0.39215686274509803,0.}
\definecolor{xfqqff}{rgb}{0.4980392156862745,0.,1.}
\definecolor{ffqqqq}{rgb}{1.,0.,0.}
\begin{tikzpicture}[line cap=round,line join=round,>=triangle 45,x=0.6782617187500004cm,y=0.6782617187500004cm]
\clip(-6.0955652959673365,-0.7955347642346219) rectangle (5.699291731705649,4.717946580732306);
\draw [rotate around={0.:(-2.,2.)},color=ffqqqq] (-2.,2.) ellipse (0.8410791202168588cm and 0.4973682009769633cm);
\draw [rotate around={-0.20389871046679844:(-1.6955307407221418,2.00727956851319)},color=xfqqff] (-1.6955307407221418,2.00727956851319) ellipse (1.661680538260386cm and 0.7868931785368128cm);
\draw [rotate around={-0.7690246825780399:(-0.8815627427186662,1.9688848516262334)},color=qqwuqq] (-0.8815627427186662,1.9688848516262334) ellipse (2.664297761397899cm and 1.2951216094501943cm);
\draw [rotate around={-0.002358769768743201:(-0.14948545367760763,1.9998003125678991)},color=qqqqff] (-0.14948545367760763,1.9998003125678991) ellipse (3.765828366589848cm and 1.832457538495793cm);
\draw [color=ffqqqq](-1.242473081456056,2.3296214542697094) node[anchor=north west] {$\mathbb{N}$};
\draw [color=xfqqff](0.26259982051263214,2.3142635675149267) node[anchor=north west] {$\mathbb{Z}$};
\draw [color=qqwuqq](2.6123564939943598,2.1989056807601443) node[anchor=north west] {$\mathbb{Q}$};
\draw [color=qqqqff](4.946755280721304,2.3142635675149267) node[anchor=north west] {$\mathbb{R}$};
\end{tikzpicture}
\end{minipage}

\end{ExoCad}


  
\begin{ExoCad}{Raisonner. Communiquer.}{1234}{0}{0}{0}{0}{0}

Complète par  $\in$ ou $\notin$.

\begin{enumerate}
\begin{minipage}{0.32\linewidth}
\item $\pi \ldots \ldots \R$
\item $-5 \ldots \ldots \Z$
\item $\dfrac{1}{2} \ldots \ldots \D$
\end{minipage}
\hfill
\begin{minipage}{0.32\linewidth}
\item $-9 \ldots \ldots \N$
\item $\dfrac{1}{2}  \ldots \ldots \R$
\item $\dfrac{7}{3}  \ldots \ldots \Q$
\end{minipage}
\hfill
\begin{minipage}{0.32\linewidth}
\item $\sqrt{4} \ldots \ldots \N$
\item $-1,5 \ldots \ldots \Z$
\item $\dfrac{12}{4}  \ldots \ldots \N$
\end{minipage}
\end{enumerate}
 
\end{ExoCad}
 
 
 
\begin{ExoCad}{Représenter.}{1234}{0}{0}{0}{0}{0}

On pose $a=4\times \left(\frac{3}{5}\right)^3 \times \frac{20}{3}$. Montrer que $\sqrt{a}$ est un nombre décimal. \point{5}
 
 \end{ExoCad}



 
\begin{DemoE}

Démontrer que $\dfrac{1}{3}$ n'est pas un nombre décimal.

\point{5}

 
\end{DemoE}

 
 

 


\begin{DemoE}

Démontrer que $\sqrt{2}$ est un irrationnel. Une autre formulation est : démontrer que $\sqrt{2} \in \R\setminus\Q$.

\point{8}

\end{DemoE}
   


 
\end{pageAD}


\begin{pageCours}

\section{Intervalles de $\R$ }

\begin{DefT}{Intervalle fermé. Intervalle ouvert}\index{Intervalle}
Un \textbf{intervalle fermé} de $\R$ est un sous-ensemble borné de $\R$, c'est à dire un ensemble de nombres compris entre deux valeurs réelles.
 
Un \textbf{intervalle ouvert} de $\R$ est un sous-ensemble de $\R$ dont les bornes ne sont pas incluses dans l'ensemble, c'est à dire un ensemble de nombres compris entre deux valeurs réelles non comprises.
\end{DefT}

\begin{Rep}
\begin{enumerate}
\item On a représenté sur la droite des nombres réels tous les nombres réels $x$ tels que $-1 \leq x \leq 3$.

\begin{center}
\definecolor{ffxfqq}{rgb}{1.,0.4980392156862745,0.}
\begin{tikzpicture}[line cap=round,line join=round,>=triangle 45,x=1.0cm,y=1.0cm]
\draw[->,color=black] (-4.390839866186475,0.) -- (7.64974334956303,0.);
\foreach \x in {-4.,-3.,-2.,-1.,1.,2.,3.,4.,5.,6.,7.}
\draw[shift={(\x,0)},color=black] (0pt,2pt) -- (0pt,-2pt) node[below] {\footnotesize $\x$};
\draw[color=black] (0pt,-10pt) node[right] {\footnotesize $0$};
\clip(-4.390839866186475,-0.5880295569511441) rectangle (7.64974334956303,0.5863275715079787);
\draw [line width=2.4pt,color=ffxfqq] (-1.,0.)-- (3.,0.);
\draw [color=ffxfqq](-1.16,0.35) node[anchor=north west] {\Large{[}};
\draw [color=ffxfqq](2.85,0.35) node[anchor=north west] {\Large{]}};
\end{tikzpicture}
 \end{center} 
 
Cet intervalle est noté $[-1;3]$ et on écrit alors $x \in [-1;3]$. Cet intervalle est dit \textbf{fermé}.
 \item On a représenté sur la droite des nombres réels tous les nombres réels $x$ tels que $-1 < x < 3$.

\begin{center}
\definecolor{ffxfqq}{rgb}{1.,0.4980392156862745,0.}
\begin{tikzpicture}[line cap=round,line join=round,>=triangle 45,x=1.0cm,y=1.0cm]
\draw[->,color=black] (-4.390839866186475,0.) -- (7.64974334956303,0.);
\foreach \x in {-4.,-3.,-2.,-1.,1.,2.,3.,4.,5.,6.,7.}
\draw[shift={(\x,0)},color=black] (0pt,2pt) -- (0pt,-2pt) node[below] {\footnotesize $\x$};
\draw[color=black] (0pt,-10pt) node[right] {\footnotesize $0$};
\clip(-4.390839866186475,-0.5295569511441) rectangle (7.64974334956303,0.563275715079787);
\draw [line width=2.4pt,color=ffxfqq] (-1.,0.)-- (3.,0.);
\draw [color=ffxfqq](-1.2,0.35) node[anchor=north west] {\Large{]}};
\draw [color=ffxfqq](2.85,0.35) node[anchor=north west] {\Large{[}};
\end{tikzpicture}
 \end{center}
Cet intervalle ouvert est noté $]-1;3[$ et on écrit alors $x \in ]-1;3[$. Cet intervalle est dit \textbf{ouvert}.

 \item On a représenté sur la droite des nombres réels tous les nombres réels $x$ tels que $x \geq -1$.


\begin{center}
\definecolor{ffxfqq}{rgb}{1.,0.4980392156862745,0.}
\begin{tikzpicture}[line cap=round,line join=round,>=triangle 45,x=1.0cm,y=1.0cm]
\draw[->,color=black] (-4.390839866186475,0.) -- (7.64974334956303,0.);
\foreach \x in {-4.,-3.,-2.,-1.,0,1.,2.,3.,4.,5.,6.,7.}
\draw[shift={(\x,0)},color=black] (0pt,2pt) -- (0pt,-2pt) node[below] {\footnotesize $\x$};
\clip(-4.390839866186475,-0.5880295569511441) rectangle (7.64974334956303,0.53275715079787);
\draw [line width=2.4pt,color=ffxfqq] (-1.,0.)-- (8.,0.);
\draw [color=ffxfqq](-1.2,0.35) node[anchor=north west] {\Large{[}};
\end{tikzpicture}
 \end{center} 
Cet ensemble est noté $[-1 ; +\infty[$ et on écrit alors $x \in [-1 ; +\infty[$. Cet intervalle est semi-ouvert à droite.


 \item On a représenté sur la droite des nombres réels tous les nombres réels $x$ tels que $x \geq -1$.


\begin{center}
\definecolor{ffxfqq}{rgb}{1.,0.4980392156862745,0.}
\begin{tikzpicture}[line cap=round,line join=round,>=triangle 45,x=1.0cm,y=1.0cm]
\draw[->,color=black] (-4.390839866186475,0.) -- (7.64974334956303,0.);
\foreach \x in {-4.,-3.,-2.,-1.,0,1.,2.,3.,4.,5.,6.,7.}
\draw[shift={(\x,0)},color=black] (0pt,2pt) -- (0pt,-2pt) node[below] {\footnotesize $\x$};
\clip(-4.390839866186475,-0.5880295569511441) rectangle (7.64974334956303,0.53275715079787);
\draw [line width=2.4pt,color=ffxfqq] (-5.,0.)-- (2.,0.);
\draw [color=ffxfqq](1.8,0.35) node[anchor=north west] {\Large{]}};
\end{tikzpicture}
 \end{center} 
Cet ensemble est noté $]-\infty;2]$ et on écrit alors $x \in ]-\infty;2]$. Cet intervalle est semi-ouvert à gauche.



\end{enumerate}
\end{Rep}

\begin{Rqs}
\begin{enumerate}
\item  $+ \infty$ se lit "plus l’infini". L'ensemble des nombres réels $\R$ est l'intervalle $]-\infty ; +\infty[ = \R$.
\item Un intervalle est une partie de $\R$ "sans trou", en "un seul morceau".
\item $+\infty$ et $-\infty$ ne sont pas des nombres. Ce ne sont que des notations (ce qui explique qu'ils soient toujours exclus).
\item Les intervalles correspondants aux quatre premières lignes du tableau sont dits bornés.
\item  Plus généralement, les différents types d'intervalles sont donnés dans le tableau ci-dessous (où $a$ et $b$ représentent deux réels, avec $a < b$).
\end{enumerate}
\end{Rqs}
 

\end{pageCours} 

\begin{pageAD} 
 

\Sf{Connaitre et représenter les intervalles}
 

  
\begin{ExoCad}{Représenter.}{1234}{0}{0}{0}{0}{0}

Soit $x$ un nombre. Écrire sous forme d'intervalle les inégalités suivantes :


\begin{enumerate}[leftmargin=*]
\begin{minipage}{0.49\linewidth}
	\item $-2 \leq x \leq 3$  \point{1}
	\item $-\pi \leq x <  \pi$  \point{1}
	\item Soient $a$ et $b$ deux réels, $a \leq x \leq  b$  \point{1}
\end{minipage}
\hfill
\begin{minipage}{0.49\linewidth}
	\item $ x \leq -2$  \point{1}
	\item $0 \geq x $  \point{1}
	\item Soient $a$ un réel, $x <  a$  \point{1}
\end{minipage}
\end{enumerate} 
\end{ExoCad}

  
\begin{ExoCad}{Représenter.}{1234}{0}{0}{0}{0}{0}

Soit $x$ un nombre. Écrire sous forme d'inégalités  les  intervalles suivants :

\begin{enumerate}[leftmargin=*]
\begin{minipage}{0.3\linewidth}
	\item $x \in [-3;2]$  \point{1}
	\item $x \in ]-\infty;3]$  \point{1}
\end{minipage}
\hfill
\begin{minipage}{0.3\linewidth}
	\item $x \in ]0;1]$  \point{1}
	\item $x \in ]1;+\infty[$  \point{1}
\end{minipage}
\hfill
\begin{minipage}{0.3\linewidth}
	\item $x \in ]-5;8[$  \point{1}
	\item $x \in [a;b]$  \point{1}
\end{minipage}
\end{enumerate} 
\end{ExoCad}



\begin{ExoCad}{Représenter.}{1234}{0}{0}{0}{0}{0}
\begin{enumerate}[leftmargin=*]
	\item Écrire l'ensemble des réels strictement supérieurs à $-2$ et inférieurs à $4$  \point{1}
	\item Écrire l'ensemble des réels supérieurs à $2$ et inférieurs à $6$  \point{1}
	\item Écrire l'ensemble des réels inférieurs à $-1$  \point{1}
\end{enumerate}
\end{ExoCad}



\begin{ExoCad}{Représenter.}{1234}{0}{0}{0}{0}{0}

Recopier et compléter le tableau.

\begin{tabular}{|c|c|c|}
\hline 
Intervalle & Inégalité & Représentation  sur la droite graduée  \\ 
\hline 
$x\in \left[ -6 ; \dfrac{2}{7}\right]$ & $-6  \leq x \leq  \dfrac{2}{7} $  &   \\ 
\hline 
 & $-3 \leq x <5$ &    \\ 
\hline 
$x\in \left[ 5 ; 8 \right[ $  &  &     \\ 
\hline 
 &  & 
 \definecolor{ffdxqq}{rgb}{1.,0.8431372549019608,0.}
\definecolor{ffxfqq}{rgb}{1.,0.4980392156862745,0.}
\begin{tikzpicture}[line cap=round,line join=round,>=triangle 45,x=1.0cm,y=1.0cm]
\draw[->,color=black] (-5.174092090680384,0.) -- (2.566282833730012,0.);
\foreach \x in {-5.,-4.,-3.,-2.,-1.,1.,2.}
\draw[shift={(\x,0)},color=black] (0pt,2pt) -- (0pt,-2pt) node[below] {\footnotesize $\x$};
\draw[color=black] (0pt,-10pt) node[right] {\footnotesize $0$};
\clip(-5.174092090680384,-0.4115875953650586) rectangle (2.566282833730012,0.4791698364123281);
\draw [line width=2.4pt,color=ffxfqq] (-4.,0.)-- (1.,0.);
\draw [color=ffdxqq](-4.2,0.35) node[anchor=north west] {\Large{]}};
\draw [color=ffdxqq](0.88 ,0.35) node[anchor=north west] {\Large{]}};
\end{tikzpicture} 
 \\ 
\hline 
\end{tabular} 
 
\end{ExoCad}



\begin{ExoCad}{Représenter.}{1234}{0}{0}{0}{0}{0}

Représenter sur une droite graduée les intervalles donnés. Soit $x$ un réel,
\begin{enumerate}[leftmargin=*]
	\item $-10< x < 3$  \point{1}
	\item $ x \in [0;\pi]$  \point{1}
	\item $ x \geq \dfrac{1}{3}$  \point{1}
	\item l'ensemble des réels strictement supérieurs à $-1$ et inférieurs à $3$  \point{1}
\end{enumerate}
\end{ExoCad}


 
 
 
\begin{ExoCad}{Raisonner. Communiquer.}{1234}{0}{0}{0}{0}{0}

Complète par  $\in$ ou $\notin$.

\begin{enumerate}
\begin{minipage}{0.32\linewidth}
\item $4 \ldots \ldots [4;5]$ \vspace{0.1cm}
\item $1,5 \ldots \ldots [1;3]$\vspace{0.1cm}
\item $-5,9 \ldots \ldots ]-\infty;-6]$
\end{minipage}
\hfill
\begin{minipage}{0.32\linewidth}
\item $3 \ldots \ldots [0;3[$\vspace{0.1cm}
\item $\dfrac{2}{3} \ldots \ldots [2;3]$\vspace{0.1cm}
\item $\dfrac{1}{2} \ldots \ldots ]-1;1[$
\end{minipage}
\hfill
\begin{minipage}{0.32\linewidth}
\item $\dfrac{5}{3} \ldots \ldots \left[ \dfrac{5}{6}; \dfrac{11}{6}  \right]$ 
\item $\sqrt{5} \ldots \ldots \left[ 1;3 \right]$\vspace{0.1cm}
\item $\sqrt{24} \ldots \ldots \left[0; 4\right]$ 
\end{minipage}

\end{enumerate}
 
\end{ExoCad}
 
 
 
\end{pageAD}
 
 \begin{pageCours}
 
 
\section{Opérations d'ensembles de nombres et logique mathématique}
 
\begin{minipage}[t]{0.69\linewidth}

\begin{DefT}{Inclusion}\index{Ensemble!Inclusion}

\begin{minipage}{0.58\linewidth}
Un ensemble $A$ \textbf{est inclus dans} un ensemble $B$ lorsque \textbf{tous} les éléments de $A$ sont contenus dans $B$. On note $A \subset B$. \point{3}
\end{minipage}
\hfill
\begin{minipage}{0.38\linewidth}

\definecolor{ffttww}{rgb}{1.,0.2,0.4}
\definecolor{xdxdff}{rgb}{0.49019607843137253,0.49019607843137253,1.}
\begin{tikzpicture}[line cap=round,line join=round,>=triangle 45,x=1.0cm,y=1.0cm]
\clip(1.32,0.48) rectangle (6.42,3.82);
\draw [rotate around={-1.123302714075422:(3.77,2.23)},line width=2.pt,color=xdxdff,fill=xdxdff,fill opacity=0.30000001192092896] (3.77,2.23) ellipse (2.1702715668569548cm and 1.5389212695611632cm);
\draw [rotate around={-61.97549946792974:(3.77,2.32)},line width=2.pt,color=ffttww,fill=ffttww,fill opacity=0.3499999940395355] (3.77,2.32) ellipse (1.1237882510756807cm and 0.8772685069325904cm);
\draw (4.76,3.48) node[anchor=north west] {$B$};
\draw (3.6,3.24) node[anchor=north west] {$A$};
\end{tikzpicture}
\end{minipage}
\end{DefT}
\end{minipage}
\begin{minipage}[t]{0.28\linewidth}
\begin{Rq}
\begin{description}[leftmargin=*]
\item Un ensemble \textbf{est inclus dans} un ensemble. $A$ et $B$ deux ensembles : $A \subset B$.
\item Un élément \textbf{appartient à} un ensemble. $x$ un élément et $A$ un ensemble : $x \in B$.
\end{description}
\end{Rq}
\end{minipage}


\begin{minipage}[t]{0.55\linewidth}
\begin{LogT}{Le contre exemple}

$\Z$ est-il inclus dans $\N$ ? Autrement dit, tous les éléments de $\Z$ sont -ils contenus dans $\N$ ?\\ On cherche  un seul élément de $\Z$ qui n'appartient pas à $\N$ : $-2 \in \Z$ mais $-2 \not\in \N$ donc $\Z \not\subset \N$. 

$-2$ est appelé un \textbf{contre exemple}\index{Contre exemple!Logique}. 
 

\end{LogT}
\end{minipage}
\begin{minipage}[t]{0.43\linewidth}
 
\begin{LogT}{La contra-posée} 
Soit $A$ et $B$ deux ensembles tels que $A \subset B$. 

Ces deux propositions disent la même idée :

Si $x \in A$ alors $x \in B \Longleftrightarrow$ Si $x \not\in B$ alors $x \not\in A$.

$\sqrt{3} \not\in \Q \Rightarrow \sqrt{3} \not\in \Z$
\end{LogT}

\end{minipage}

\begin{DefT}{Complémentaire}\index{Ensemble!Complémentaire}

\begin{minipage}{0.68\linewidth}
Soit $\Omega$ un ensemble contenant un ensemble $A$. On appelle \textbf{complémentaire} de $A$ dans $\Omega$, \textbf{tous} les éléments de $\Omega$ qui n'appartiennent pas à $A$. \point{2}
Si $\Omega=\lbrace{0;1;2;3;4;5;6\rbrace}$ et $A=\lbrace{3;4;6\rbrace}$ alors $\Omega\setminus A=\lbrace{0;1;2;5\rbrace}$
\end{minipage}
\begin{minipage}{0.28\linewidth}

\definecolor{ffffff}{rgb}{1.,1.,1.}
\definecolor{xfqqff}{rgb}{0.4980392156862745,0.,1.}
\definecolor{ffttww}{rgb}{1.,0.2,0.4}
\definecolor{xdxdff}{rgb}{0.49019607843137253,0.49019607843137253,1.}
\definecolor{ududff}{rgb}{0.30196078431372547,0.30196078431372547,1.}
\begin{tikzpicture}[line cap=round,line join=round,>=triangle 45,x=1.0cm,y=1.0cm]
\clip(1.32,0.5) rectangle (5.9,3.96);
\draw [rotate around={0.916654256385289:(3.49,2.28)},line width=2.pt,color=xdxdff,fill=xdxdff,fill opacity=0.5] (3.49,2.28) ellipse (2.0420114277198222cm and 1.6145930357022946cm);
\draw [rotate around={-61.97549946792974:(2.99,2.34)},line width=2.pt,color=ffttww,fill=ffttww,fill opacity=1.0] (2.99,2.34) ellipse (1.1237882510756807cm and 0.8772685069325904cm);
\draw [color=xfqqff](5.36,3.78) node[anchor=north west] {$B$};
\draw [color=ffffff](2.32,3.3) node[anchor=north west] {$A$};
\draw (3.8,3.48) node[anchor=north west] {$B\setminus A$};
\end{tikzpicture}

\end{minipage}
\end{DefT}
 


 
\begin{minipage}[t]{0.68\linewidth}
\begin{DefT}{Intersection}\index{Ensemble!Intersection}

\begin{minipage}{0.58\linewidth}
L'\textbf{intersection} de deux ensembles $A$ et $B$ est l'ensemble $A \cap B$ qui contient \textbf{tous} les éléments communs aux deux ensembles. 

On lit $A$ "inter" $B$. 

L'ensemble $A \cap B$ est l'ensemble qui réunit les éléments de $A$ \textbf{et} de $B$ pris une seule fois. \point{2}
Si $A=\lbrace{0;1;2;3;4;5;6\rbrace}$ et $B=\lbrace{3;4;7;8\rbrace}$ alors $A \cap B=\lbrace{3;4 \rbrace}$
\end{minipage}
\hfill
\begin{minipage}{0.38\linewidth}

\definecolor{ffttww}{rgb}{1.,0.2,0.4}
\definecolor{xdxdff}{rgb}{0.49019607843137253,0.49019607843137253,1.}
\begin{tikzpicture}[line cap=round,line join=round,>=triangle 45,x=1.0cm,y=1.0cm]
\clip(-0.06,0.86) rectangle (4.72,4.1);
\draw [rotate around={-2.7263109939062526:(3.08,2.22)},line width=2.pt,color=xdxdff,fill=xdxdff,fill opacity=0.30000001192092896] (3.08,2.22) ellipse (1.44092369450198cm and 1.1700688412983384cm);
\draw [rotate around={-61.97549946792966:(1.29,2.28)},line width=2.pt,color=ffttww,fill=ffttww,fill opacity=0.3499999940395355] (1.29,2.28) ellipse (1.123788251075678cm and 0.8772685069325883cm);
\draw (4.16,3.52) node[anchor=north west] {$B$};
\draw (0.28,3.98) node[anchor=north west] {$A$};
\draw [->,line width=1.pt] (1.84,3.38) -- (1.88,2.16);
\draw (1.24,3.98) node[anchor=north west] {$A\cap B$};
\end{tikzpicture}

\end{minipage}

\end{DefT}

\end{minipage}
\begin{minipage}[t]{0.28\linewidth}


\begin{Rq}\index{Ensembles disjoints}
Deux ensembles sont \textbf{disjoints} lorsque $A \cap B = \oslash$. $\oslash$ est l'ensemble vide.\index{Ensemble!vide}
\vspace{0.2cm}

Si $C=\lbrace{0;1;2;3\rbrace}$ 

et $D=\lbrace{4;5;6\rbrace}$ 

alors $A \cap B= \oslash $

\definecolor{xdxdff}{rgb}{0.49019607843137253,0.49019607843137253,1.}
\definecolor{ffttww}{rgb}{1.,0.2,0.4}
\begin{tikzpicture}[line cap=round,line join=round,>=triangle 45,x=1.0cm,y=1.0cm]
\clip(-5.398001213875861,3.010943854292776) rectangle (-0.9359834713983067,4.7475129216353915);
\draw [rotate around={9.950626687951598:(-4.276467024550424,3.8551093731398804)},line width=2.pt,color=ffttww,fill=ffttww,fill opacity=0.15000000596046448] (-4.276467024550424,3.8551093731398804) ellipse (0.8703197956159994cm and 0.5200054996195858cm);
\draw [rotate around={-0.9240453527727059:(-2.0695771681358472,3.7706928212551696)},line width=2.pt,color=xdxdff,fill=xdxdff,fill opacity=0.25] (-2.0695771681358472,3.7706928212551696) ellipse (0.9314625917474598cm and 0.5553715874828972cm);
\draw (-4.867382887743395,4.10984671409655) node[anchor=north west] {$1$};
\draw (-4.360883576435132,3.892775580678723) node[anchor=north west] {$2$};
\draw (-4.553835695028756,4.35103686233858) node[anchor=north west] {$3$};
\draw (-4.095574413368899,4.182203758569159) node[anchor=north west] {$0$};
\draw (-2.624314509092516,4.158084743744956) node[anchor=north west] {$5$};
\draw (-1.7801489902454115,4.158084743744956) node[anchor=north west] {$6$};
\draw (-2.3590053460262834,3.823942476909302) node[anchor=north west] {$4$};
\end{tikzpicture}

\end{Rq}
\end{minipage}


\begin{DefT}{Réunion}\index{Ensemble!Réunion}

\begin{minipage}{0.68\linewidth}
La \textbf{réunion} de deux ensembles est l'ensemble $A \cup B$ qui contient \textbf{tous} les éléments de $A$ \textbf{ou} de $B$ pris une seule fois. 

On lit $A$ "union" $B$. \point{2}

Si $A=\lbrace{0;1;2;3;4;5;6\rbrace}$ et $B=\lbrace{3;4;7;8\rbrace}$ alors $A \cup B=\lbrace{0;1;2;3;4;5;6;7;8 \rbrace}$
\end{minipage}
\hfill
\begin{minipage}{0.28\linewidth}

\definecolor{ffqqqq}{rgb}{1.,0.,0.}
\definecolor{xfqqff}{rgb}{0.4980392156862745,0.,1.}
\definecolor{ffttww}{rgb}{1.,0.2,0.4}
\definecolor{xdxdff}{rgb}{0.49019607843137253,0.49019607843137253,1.}
\begin{tikzpicture}[line cap=round,line join=round,>=triangle 45,x=1.0cm,y=1.0cm]
\clip(0.06,0.8) rectangle (4.82,4.12);
\draw [rotate around={-2.7263109939062526:(3.08,2.22)},line width=2.pt,color=xdxdff,fill=xdxdff,fill opacity=0.5] (3.08,2.22) ellipse (1.44092369450198cm and 1.1700688412983384cm);
\draw [rotate around={-61.97549946792966:(1.29,2.28)},line width=2.pt,color=ffttww,fill=ffttww,fill opacity=0.550000011920929] (1.29,2.28) ellipse (1.123788251075678cm and 0.8772685069325883cm);
\draw [color=xfqqff](4.16,3.52) node[anchor=north west] {$B$};
\draw [color=ffqqqq](0.28,3.98) node[anchor=north west] {$A$};
\draw (1.36,3.96) node[anchor=north west] {$A\cup B$};
\draw [rotate around={-2.7263109939062526:(3.08,2.22)},line width=1.pt,fill=black,pattern=north east lines,pattern color=black] (3.08,2.22) ellipse (1.44092369450198cm and 1.1700688412983384cm);
\draw [rotate around={-61.97549946792966:(1.29,2.28)},line width=1.pt,fill=black,pattern=north east lines,pattern color=black] (1.29,2.28) ellipse (1.123788251075678cm and 0.8772685069325883cm);
\end{tikzpicture}


\end{minipage}

\end{DefT}






\end{pageCours} 
\begin{pageAD} 
 





\Sf{Opérer avec les ensembles}
 
  
\begin{ExoCad}{Communiquer.}{1234}{2}{0}{0}{0}{0}

Compléter par $\subset$  ou $\not\subset$. Justifier l'utilisation de  $\not\subset$.

\begin{enumerate}
\begin{minipage}{0.49\linewidth}
\item $\Z \ldots \ldots \R$ \point{1}
\item $\Q \ldots \ldots \Z$ \point{1}
\item $\Z \ldots \ldots \N$ \point{1}
\end{minipage}
\hfill
\begin{minipage}{0.49\linewidth}
\item $\Z \ldots \ldots \D$ \point{1}
\item $\D \ldots \ldots \Q$ \point{1}
\item $\Q \ldots \ldots \N$ \point{1}
\end{minipage}
\end{enumerate}
 
\end{ExoCad}

\begin{ExoCad}{Représenter.}{1234}{0}{0}{0}{0}{0}


On propose dans chaque cas deux ensembles $A$ et $B$. Lequel est-il inclus dans l'autre ? \vspace{0.2cm} 

\begin{enumerate}

\begin{minipage}{0.48\linewidth}
\item $A=[-1,1;3]$ et $B=]-2,9;6]$ \point{1}
\item $A=[0,7;0,8]$ et $B=[0,5;+\infty[$\point{1}
\end{minipage}
\hfill
\begin{minipage}{0.48\linewidth}
\item $A=]1;2[$ et $B=[1;2]$\point{1}
\item $A=\Q$ et $B=\Z$\point{1}
\end{minipage}

\end{enumerate} 
 
\end{ExoCad}




\begin{ExoCad}{Raisonner.}{1234}{0}{0}{0}{0}{0}

Déterminer les complémentaires de l'ensemble $A$ par rapport $\Omega$. 
\begin{enumerate}
\item $\Omega = \lbrace 1;2;3;4;5;6\rbrace$ et $A = \lbrace 1;2 \rbrace$. $\Omega \setminus A = $\point{1}
\item $\Omega = \lbrace -4;-2;-1;0;1;2;\rbrace$ et $A = \lbrace -2;1;2 \rbrace$.  $\Omega \setminus A = $\point{1}
\item $\Omega = \R$ et $A = \Q$. $\Omega \setminus A = $\point{1}
\item $\Omega = \R$ et $A = \R^-$. $\Omega \setminus A = $\point{1}
\end{enumerate} 
 \end{ExoCad}


\begin{ExoCad}{Représenter.}{1234}{0}{0}{0}{0}{0}


Déterminer les intersections des ensembles $A$ et $B$ suivants.   $A \cap B = $ se lit $A$ inter $B$.
\begin{enumerate}
\item $A = \lbrace 1;2;3;4;5;6\rbrace$ et $B = \lbrace 1;2 \rbrace$. $A \cap B = $\point{1}
\item $A = \lbrace -4;-2;-1;0;1;2;\rbrace$ et $B = \lbrace -2;1;2 \rbrace$.  $A \cap B = $\point{1}
\item $A = \N$ et $B = \R$. $A \cap B = $ \point{1}
\item $A = [-4;3[$ et $B =[-2;7]$. $A \cap B = $ \point{1}
\item $A = [-2;1]$ et $B =[2;3]$. $A \cap B = $ \point{1}
\item $A = \N$ et $B =]-\infty;5]$. $A \cap B = $ \point{1}
\end{enumerate} 
 
\end{ExoCad}









\begin{ExoCad}{Représenter.}{e/1234}{0}{0}{0}{0}{0}


Déterminer les réunions des ensembles $A$ et $B$ suivants. 
\begin{enumerate}
\item $A = \lbrace 1;2;3;4;5;6\rbrace$, $B = \lbrace 1;2 \rbrace$. $A \cup B = $\point{1}
\item $A = \lbrace -4;-2;-1;0;1;2;\rbrace$,$B = \lbrace -2;1;2 \rbrace$.  $A \cup B = $\point{1}
\item $A=\Z$, $B =\R$. $A \cup B = $ \point{1}
\item $A=\left\lbrace 1;2;8;6  \right\rbrace $, $B =\left\lbrace 0;2;4;8  \right\rbrace $. $A \cup B = $ \point{1}
\item $A=[-2;1]$, $B =[2;3]$. $A \cup B = $ \point{1}
\item $A=[0;+\infty[$, $B =]-\infty;5]$. $A \cup B = $ \point{1}
\end{enumerate} 
 
\end{ExoCad}



 
\end{pageAD}
 
  
%%%%%%%%%%%%%%%%%%%%%%%%%%%%%%%%%%%%%%%%%%%%%%%%%%%%%%%%%%%%%%%%%%%
%%%%  Niveau 1
%%%%%%%%%%%%%%%%%%%%%%%%%%%%%%%%%%%%%%%%%%%%%%%%%%%%%%%%%%%%%%%%%%%
\begin{pageParcoursu}

\begin{ExoCu}{Communiquer.}{1234}{2}{0}{0}{0}{0}

Johan visite Vienne, la capitale de l'Autriche. L' Autriche est un pays Européen.
\begin{enumerate}[leftmargin=*]
\item Arrivé en Autriche, Johan est-il à Vienne ? \point{1}
\item A Vienne, Johan est-il en Autriche ?\point{1}
\item Peut-on dire que Johan est en Europe ?\point{1}
\item Lettres $A$, $E$ et $V$ sont les initiales respectives de Autriche, Europe et Vienne. Compléter avec les lettres $A$, $E$ et $V$ : $\ldots\ldots \subset \ldots\ldots \subset \ldots\ldots $
\end{enumerate}


\end{ExoCu}


\begin{ExoCu}{Communiquer.}{1234}{2}{0}{0}{0}{0}

Recopie et complète par $\subset$, $\in$, $\not\subset$ ou $\notin$.

\begin{enumerate}
\begin{minipage}{0.32\linewidth}
\item $\N \ldots \ldots \R$
\item $-5 \ldots \ldots \Z$
\item $\Z \ldots \ldots \N$
\end{minipage}
\hfill
\begin{minipage}{0.32\linewidth}
\item $\left\lbrace 0;1;2 \right\rbrace \ldots \ldots \N$
\item $]-\infty;1] \ldots \ldots \R$
\item $]0;+\infty[ \ldots \ldots \Z$
\end{minipage}
\hfill
\begin{minipage}{0.32\linewidth}
\item $\sqrt{3} \ldots \ldots \N$
\item $-1,5 \ldots \ldots \Z$
\item $\Q \ldots \ldots \Z$
\end{minipage}
\end{enumerate}
 
\end{ExoCu}



\begin{ExoCu}{Communiquer.}{1234}{2}{0}{0}{0}{0}

Déterminer les intersections des ensembles suivants.
\begin{enumerate}
\item $A = \Z$ et $B = \Q$. $A \cap B = $ \point{1}
\item $A = ]-2;3[$ et $B =[-1;5]$. $A \cap B = $ \point{1}
\item $A = [-4;3]$ et $B =[2;6]$. $A \cap B = $ \point{1}
\item $A = ]-6;1[$ et $B =[0;5]$. $A \cap B = $ \point{1}
\end{enumerate}
\end{ExoCu}


\begin{ExoCu}{Chercher.communiquer.}{2}{0}{0}{0}{0}
 
Dans chaque cas, trouver, lorsque cela est possible, un nombre $x$ qui remplit les critères suivants :
\begin{enumerate}
\item $x \in \Q$ et $x \not\in \Z$ \point{1}
\item $x \in \R$ et $x \not\in \N$ \point{1}
\end{enumerate}
\end{ExoCu}

 





\begin{ExoCu}{Représenter. Raisonner. Communiquer.}{1234}{2}{0}{0}{0}{0}


Déterminer, dans chaque cas, la réunion des ensembles suivants. On écrira : $A \cup B = $ où $A$ et $B$ sont les ensembles ci-dessous.
\begin{enumerate}
\item $\Q$ et $\R$
\item $\left\lbrace 1;3;5;7  \right\rbrace $ et $\left\lbrace 0;2;4;8  \right\rbrace $
\item $[-3;4]$ et $[2;6]$
\item $[0;+\infty[$ et $]-\infty;5]$
\end{enumerate}
On pourra représenter chaque intervalle sur une droite graduée tracée à main levée.
 \end{ExoCu}



\begin{ExoCu}{Modéliser.}{1234}{0}{0}{0}{0}{0}


Déterminer l'ensemble des valeurs de $x$ dans chaque cas.
\begin{enumerate}
\item $x < -4$ et $x \geq 10$\point{1}
\item $x \leq 6$ et $x \leq 3$\point{1}
\item $x \leq 6$ ou $x \geq 3$\point{1}
\end{enumerate} 
 
\end{ExoCu}
 


\end{pageParcoursu}
  
  
%%%%%%%%%%%%%%%%%%%%%%%%%%%%%%%%%%%%%%%%%%%%%%%%%%%%%%%%%%%%%%%%%%%
%%%%  Niveau 2
%%%%%%%%%%%%%%%%%%%%%%%%%%%%%%%%%%%%%%%%%%%%%%%%%%%%%%%%%%%%%%%%%%%
\begin{pageParcoursd} 

 

\begin{ExoCd}{Représenter. Raisonner.}{1234}{0}{0}{0}{0}{0}

Déterminer l'intervalle des valeurs de $x$ dans chaque cas.

\begin{enumerate}
\begin{minipage}{0.5\linewidth}
%
Déterminer l'ensemble des valeurs de $x$ dans chaque cas.
\begin{enumerate}
\item On jette un dé à 6 face et on regarde la face obtenue. Soit $x$ le numéro de la face. 
\item $[-1,1;3]$ et $[2,9;6]$
\item $x > -4$ et $x \leq 10$
\item $x \leq -3$ et $x \leq 5$
\item $x \leq 5$ ou $x \geq 2$
\end{enumerate} 

\item $[-1,1;3]$ et $[2,9;6]$
\item $x > -4$ et $x \leq 10$
\item $x \leq -3$ et $x \leq 5$


\end{minipage}
\begin{minipage}{0.5\linewidth}
%
Soit $x$ un réel.Écrire sous forme d'intervalle ou de réunion d'intervalles le plus grand ensemble auquel appartient $x$.


\begin{enumerate}
	\item $x \geq 1$ ou $x<3$.  
	\item $x \geq 1$ et $x<3$.  
\end{enumerate}
 
\item $x \geq 1$ ou $x<3$.  
\item $x \geq 1$ et $x<3$.  
\item $x \leq 5$ ou $x \geq 2$
\end{minipage}
\end{enumerate}

\end{ExoCd}


\begin{ExoCd}{Représenter.}{1234}{0}{0}{0}{0}{0}

Dans chaque cas, trouver, lorsque cela est possible, un nombre $x$ qui remplit les conditions suivantes :
 
\begin{enumerate}[leftmargin=*]
\item $x \not\in D$ et $x \in \R$  \point{1}
\item $x \in \Q$ et $x \not\in \Z$  \point{1}
\item $x \not\in \N$ et $x \in \Z$  \point{1}
\item $x \in \D$ ou $x \in \Q$  \point{1}
\item $x \not\in \N$ ou $x \in \Z$  \point{1}
\end{enumerate}
 
 \end{ExoCd}


 

\begin{ExoCd}{Représenter. Raisonner. Communiquer.}{1234}{0}{0}{0}{0}{0}


On propose dans chaque cas deux ensembles $A$ et $B$. Lequel est inclus dans l'autre ?  

\textit{{\small On pourra représenter chaque intervalle sur une droite graduée.}}

\begin{minipage}{0.48\linewidth}

 
\begin{enumerate}
\item $A = \left[ -\frac{11}{10};\frac{29}{10}\right]$ et $B=\left[-\frac{3}{2};3 \right]$
\item $A =\left[ \frac{1}{2}; +\infty \right[$ et $B=[0,7;0,8]$.
\item $A =[1;2]$ et $B=]1;2[$. 
\end{enumerate}

\end{minipage}
\hfill
\begin{minipage}{0.48\linewidth}
 
\begin{enumerate}

\item

\begin{tikzpicture}[line cap=round,line join=round,>=triangle 45,x=1.0cm,y=1.0cm]
\draw [->,line width=1.pt,domain=0.34:6.36] plot(\x,{(-14.-0.*\x)/7.});
\end{tikzpicture}
\item

\begin{tikzpicture}[line cap=round,line join=round,>=triangle 45,x=1.0cm,y=1.0cm]
\draw [->,line width=1.pt,domain=0.34:6.36] plot(\x,{(-14.-0.*\x)/7.});
\end{tikzpicture}
\item

\begin{tikzpicture}[line cap=round,line join=round,>=triangle 45,x=1.0cm,y=1.0cm]
\draw [->,line width=1.pt,domain=0.34:6.36] plot(\x,{(-14.-0.*\x)/7.});
\end{tikzpicture}
\end{enumerate}

\end{minipage}
  
 
\end{ExoCd}


\begin{ExoCd}{Représenter. Raisonner. Communiquer.}{1234}{0}{0}{0}{0}{0}


Déterminer les intersections des ensembles suivants. On écrira : $A \cap B = $ où $A$ et $B$ sont les ensembles ci-dessous.
 

\textit{{\small On pourra représenter chaque intervalle sur une droite graduée.}}



\begin{minipage}{0.48\linewidth}

\begin{enumerate}
\item $\Z$ et $\Q$
\item $[-5;2[$ et $[0;7]$
\item $[-1;4]$ et $[-3;-1]$
\item $\N$ et $]-\infty;5]$
\item $[-5;0[$ et $[0;3]$
\end{enumerate}

\end{minipage}
\hfill
\begin{minipage}{0.48\linewidth}
 
\begin{enumerate}
\item

\begin{tikzpicture}[line cap=round,line join=round,>=triangle 45,x=1.0cm,y=1.0cm]
\draw [->,line width=1.pt,domain=0.34:6.36] plot(\x,{(-14.-0.*\x)/7.});
\end{tikzpicture}
\item

\begin{tikzpicture}[line cap=round,line join=round,>=triangle 45,x=1.0cm,y=1.0cm]
\draw [->,line width=1.pt,domain=0.34:6.36] plot(\x,{(-14.-0.*\x)/7.});
\end{tikzpicture}
\item

\begin{tikzpicture}[line cap=round,line join=round,>=triangle 45,x=1.0cm,y=1.0cm]
\draw [->,line width=1.pt,domain=0.34:6.36] plot(\x,{(-14.-0.*\x)/7.});
\end{tikzpicture}
\item

\begin{tikzpicture}[line cap=round,line join=round,>=triangle 45,x=1.0cm,y=1.0cm]
\draw [->,line width=1.pt,domain=0.34:6.36] plot(\x,{(-14.-0.*\x)/7.});
\end{tikzpicture}
\item

\begin{tikzpicture}[line cap=round,line join=round,>=triangle 45,x=1.0cm,y=1.0cm]
\draw [->,line width=1.pt,domain=0.34:6.36] plot(\x,{(-14.-0.*\x)/7.});
\end{tikzpicture}
\end{enumerate}

\end{minipage} 
 
\end{ExoCd}







\begin{ExoCd}{Chercher.}{1234}{0}{0}{0}{0}{0}


On donne le programme en Python ci dessous. 
 
\begin{lstlisting}
def is_in(x,a,b):
    if x > a and x < b :
    	test = "{} is in  ]{};{}[".format(x,a,b) 
    else :
        test = "{} is not in ]{};{}[".format(x,a,b) 
    return test    
x=int(input("Entrer un nombre  :")) 
a=int(input("Entrer la borne inf :"))
b=int(input("Entrer la borne sup :"))    
print(is_in(x,a,b))
\end{lstlisting}
 


\begin{enumerate}
\item Ouvrir le logiciel PyScripter et taper ce code. Que fait ce programme ? Vous pouvez aussi ouvrir en ligne l'éditeur Python : \url{https://www.tutorialspoint.com/execute_python_online.php}
\item Modifier ce programme pour qu'il teste si un nombre $x$ appartient à l'intervalle $[a;b]$.
\end{enumerate} 
 
\end{ExoCd}




\end{pageParcoursd}
 
%%%%%%%%%%%%%%%%%%%%%%%%%%%%%%%%%%%%%%%%%%%%%%%%%%%%%%%%%%%%%%%%%%%%
%%%%%  Niveau 3
%%%%%%%%%%%%%%%%%%%%%%%%%%%%%%%%%%%%%%%%%%%%%%%%%%%%%%%%%%%%%%%%%%%%
\begin{pageParcourst}


\begin{ExoCtN}{Représenter.}{1}{1}{0}{0}{0}

Démontrer que si $p^2$ est impair alors $p$ est impair.

 
\end{ExoCtN}

\begin{DefT}{Nombre décimal périodique}
Le nombre $a_0,\underline{a_1a_2a_3}$ est un nombre décimal périodique de période $a_1a_2a_3$. Les chiffres $a_1$, $a_2$, $a_3$ se répètent indéfiniment.
\end{DefT}

\begin{ExoCtN}{Représenter. Calculer.}{1234}{0}{0}{0}{0} 
 
Démontrer que $0,\underline{9}=1$.   \point{6}
 
\end{ExoCtN}

 
\begin{ExoCtN}{Représenter. Calculer.}{1234}{1}{0}{0}{0}

\begin{enumerate}
\item On considère le nombre $\frac{19}{11}$.

\begin{enumerate}
\item Donner le développement décimal de $\frac{19}{11}$ avec 8 chiffres significatifs. $\frac{19}{11}$ semble-t-il décimal ?
\item On dit que $\frac{19}{11}$ a une écriture périodique.
Préciser sa période (série de chiffres qui se répète à l'infini dans le développement décimal).
\end{enumerate}
\item On considère le nombre $x=0,13131313....$ dont le développement décimal a pour période 13.
\begin{enumerate}
\item Démontrer que $100x = 13 + x$. 
\item  En déduire une écriture fractionnaire de $x$. Quelle est la nature du nombre $x$ ?
\end{enumerate}
\end{enumerate} 
 
\end{ExoCtN}

\begin{ExoCtN}{Représenter. Calculer.}{2}{1}{0}{0} 

\begin{enumerate}
\item Démontrer que $0,\underline{12}$ est un nombre rationnel à préciser.
\item Démontrer que $x=3,\underline{412}$ est un nombre rationnel. 
\end{enumerate}
\end{ExoCtN} 
 

%%%%%%%%%%%%%%%%%%%%%%%%%%%%%%%%%%%%%%%%%%%%%%%%%%%%%%%%%%%%%%%%%%%
\begin{ExoCt}{Raisonner.}{1234}{2}{0}{0}{0}{0}
 
Représenter graphiquement dans le plan muni d'un repère orthonormal 
 
\begin{enumerate}
\item l'ensemble des points $M(x;y)$ tes que  $$1 < x < 3 \text{ et} 0 \leq y <4$$
\item l'ensemble des points $M(x;y)$ tes que  $$1 \leq  x \leq  5 \text{ et } -2 \leq y   \leq  1$$
\end{enumerate}
 
\end{ExoCt}


%%%%%%%%%%%%%%%%%%%%%%%%%%%%%%%%%%%%%%%%%%%%%%%%%%%%%%%%%%%%%%%%%%%
\begin{ExoCt}{Représenter.}{1234}{2}{0}{0}{0}{0}
Représenter graphiquement, dans le plan muni d'un repère orthonormal, l'ensemble des points $M(x;y)$ tes que  $$1 \leq  2x+1 \leq  5 \text{ et }  -2 \leq 3y + 4  \leq  13$$
\end{ExoCt} 
 

\end{pageParcourst}
 
\begin{pageAuto}

\begin{ExoAutoN}{Raisonner.}{2}{0}{0}{0}{0}

Compléter avec $\in$, $\not\in$, $\subset$ ou $\not\subset$.

 \begin{tabular}{ccc}

$\frac{2}{10}..........\Z$ & $-\sqrt{25}..........\Z$ & $\frac{\sqrt{3}}{4}..........\Q$ \\ 

$\pi..........\R$  & $-\frac{5}{3}..........\Q$  &  $\sqrt{11}..........\R$ \\ 

\end{tabular} 

\end{ExoAutoN}
%%%%%%%%%%%%%%%%%%%%%%%%%%%%%%%%%%%%%%%%%%%%%%%%%%%%%%%%%%%%%%%%%%%
\begin{ExoAutoN}{Raisonner.}{2}{0}{0}{0}{0}
 

Déterminer, dans chaque cas, l'intersection puis la réunion des ensembles suivants. 

\begin{enumerate}
\item $A=\left\lbrace 1;3;5;7  \right\rbrace $ et $B=\left\lbrace 0;2;4;5;7;8  \right\rbrace $\point{2}
\item $A=[-3;4]$ et $B=[2;6]$\point{2}
\item $A=[0;+\infty[$ et $B=]-\infty;5]$\point{2}
\end{enumerate}
On pourra représenter les intervalles sur une droite graduée tracée à main levée.
 
 

\end{ExoAutoN}
%%%%%%%%%%%%%%%%%%%%%%%%%%%%%%%%%%%%%%%%%%%%%%%%%%%%%%%%%%%%%%%%%%%
\begin{ExoAutoN}{Représenter. Raisonner.}{2}{0}{0}{0}{0}

 Recopier et compléter le tableau.

\begin{tabular}{|c|c|c|}
\hline 
Intervalle & Inégalité & Représentation   \\ 
\hline 
$x\in \left[ -2 ; 6\right]$ & $-2  \leq x \leq  6 $  &  \\ 
\hline 
 & $1 \leq x <3$ &     \\ 
\hline 
$x\in \left[ -6 ; 6 \right[ $  &  &  \\ 
\hline 
 &  & \definecolor{ffdxqq}{rgb}{1.,0.8431372549019608,0.}
\definecolor{ffxfqq}{rgb}{1.,0.4980392156862745,0.}
\begin{tikzpicture}[line cap=round,line join=round,>=triangle 45,x=1.0cm,y=1.0cm]
\draw[->,color=black] (-5.174092090680384,0.) -- (2.566282833730012,0.);
\foreach \x in {-5.,-4.,-3.,-2.,0,-1.,1.,2.}
\draw[shift={(\x,0)},color=black] (0pt,2pt) -- (0pt,-2pt) node[below] {\footnotesize $\x$};
\clip(-5.174092090680384,-0.4115875953650586) rectangle (2.566282833730012,0.4791698364123281);
\draw [line width=2.4pt,color=ffxfqq] (-4.,0.)-- (1.,0.);
\draw [color=ffxfqq](0.8,0.35) node[anchor=north west] {\Large{]}};
\draw [color=ffxfqq](-4.2,0.35) node[anchor=north west] {\Large{]}};
\end{tikzpicture}    \\ 
\hline 
\end{tabular} 
 

\end{ExoAutoN}


\begin{ExoAutoN}{Raisonner.}{2}{0}{0}{0}{0}

Démontrer que $0,\underline{485}$ est un nombre rationnel à préciser.

\end{ExoAutoN}


\end{pageAuto}
% Ensemble de nombres 
\chapter{Intervalles}
{https://sacado.xyz/qcm/parcours_show_course/0/117129}
{


 \begin{CpsCol}
\textbf{Les savoir-faire du parcours}
 \begin{itemize}
 \item \textbf{Utiliser des nombres pour calculer et résoudre des problèmes}
	\item[$\square$] Représenter un intervalle de la droite numérique. 
	\item[$\square$]  Déterminer si un nombre réel appartient à un intervalle donné.
 \end{itemize}
 \end{CpsCol}

}




%%%%%%%%%%%%%%%%%%%%%%%%%%%%%%%%%%%%%%%%%%%%%%%%%%%%%%%%%%%%%%%%%%%
%%%%  Niveau 1
%%%%%%%%%%%%%%%%%%%%%%%%%%%%%%%%%%%%%%%%%%%%%%%%%%%%%%%%%%%%%%%%%%%
\begin{pageExercices} 


 %%%%%%%%%%%%%%%%%%%%%%%%%%%
\begin{ExoCuN}{Représenter.}{2}{0}{0}{0}{0}

Démontrer que $0,\underline{1}$ est un nombre rationnel à préciser.
 
\end{ExoCuN}
 
%%%%%%%%%%%%%%%%%%%%%%%%%%%
\begin{ExoCuN}{Représenter.}{2}{0}{0}{0}{0}
Je suis un nombre à trois chiffres non nuls. Je suis divisible par 94. Changez l'ordre de mes chiffres et je deviens divisible par 49.
Qui suis-je ? 
\end{ExoCuN}


\begin{ExoCdN}{Chercher.communiquer.}{2}{0}{0}{0}{0}
 
Dans chaque cas, trouver, lorsque cela est possible, le nombre $x$ qui remplit les critères suivants :
\begin{enumerate}
\item $x \in \Q$ et $x \not\in \Z$
\item $x \in \R$ et $x \not\in \N$. 
\end{enumerate}
\end{ExoCdN}

\begin{ExoCdN}{Représenter. Raisonner.}{2}{0}{0}{0}{0}

Démontrer que $\sqrt{2}$ est un irrationnel ou encore que $\sqrt{2} \in \R\setminus\Q$

\end{ExoCdN}

%%%%%%%%%%%%%%%%%%%%%%%%%%%%%%%%%%%%%%%%%%%%%%%%%%%%%%%%%%%%%%%%%%%
\begin{ExoCtN}{Représenter.}{2}{1}{0}{0}{0}

\begin{enumerate}
\item Démontrer que $0,\underline{12}$ est un nombre rationnel à préciser.
\item Démontrer que $0,\underline{485}$ est un nombre rationnel à préciser.
\end{enumerate}
\end{ExoCtN}

%%%%%%%%%%%%%%%%%%%%%%%%%%%%%%%%%%%%%%%%%%%%%%%%%%%%%%%%%%%%%%%%%%%
\begin{ExoCtN}{Raisonner.}{1}{0}{0}{0}{0}

\begin{enumerate}
\item Démontrer que tout entier $n$ multiple de $9$ est aussi un multiple de $3$.
\item Démontrer que si $m$ est un multiple de $6$ alors $m$ est aussi un multiple de $3$.
\end{enumerate}
\end{ExoCtN}


\begin{ExoCtN}{Représenter.}{1}{0}{0}{0}{0}

Dans un pays où le système monétaire n’est constitué que de pièces de 3 et de 5, il s’agit d’aider les habitants en créant un programme  qui donne le nombre de pièces nécessaires à tout achat d’un montant entier supérieur ou égal à 8.

\hfill{{\footnotesize Source : d’après PISA, items libérés}}

\end{ExoCtN}


\end{pageExercices}

  
%%%%%%%%%%%%%%%%%%%%%%%%%%%%%%%%%%%%%%%%%%%%%%%%%%%%%%%%%%%%%%%%%%%
%%%%  Niveau 2
%%%%%%%%%%%%%%%%%%%%%%%%%%%%%%%%%%%%%%%%%%%%%%%%%%%%%%%%%%%%%%%%%%%



%\begin{pageParcoursd} 
% 
%%%%%%%%%%%%%%%%%%%%%%%%%%%%%%%%%%%%%%%%%%%%%%%%%%%%%%%%%%%%%%%%%%%%
%
%
% 
%%%%%%%%%%%%%%%%%%%%%%%%%%%%%%%%%%%%%%%%%%%%%%%%%%%%%%%%%%%%%%%%%%%%
%
%
%
%%%%%%%%%%%%%%%%%%%%%%%%%%%%%%%%%%%%%%%%%%%%%%%%%%%%%%%%%%%%%%%%%%%%
%
%
% %%%%%%%%%%%%%%%%%%%%%%%%%%%%%%%%%%%%%%%%%%%%%%%%%%%%%%%%%%%%%%%%%%%
%\begin{ExoCd}{Représenter. Raisonner.}{1234}{2}{0}{0}{0}{0}
%
%
%\end{ExoCd}
% 
%%%%%%%%%%%%%%%%%%%%%%%%%%%%%%%%%%%%%%%%%%%%%%%%%%%%%%%%%%%%%%%%%%%%
%\begin{ExoCd}{Représenter. Raisonner.}{1234}{2}{0}{0}{0}{0}
%
%
%\end{ExoCd}
% 
%\end{pageParcoursd}
%
%%%%%%%%%%%%%%%%%%%%%%%%%%%%%%%%%%%%%%%%%%%%%%%%%%%%%%%%%%%%%%%%%%%%
%%%%%  Niveau 3
%%%%%%%%%%%%%%%%%%%%%%%%%%%%%%%%%%%%%%%%%%%%%%%%%%%%%%%%%%%%%%%%%%%%
%\begin{pageParcourst}
%
%
%
%
%%%%%%%%%%%%%%%%%%%%%%%%%%%%%%%%%%%%%%%%%%%%%%%%%%%%%%%%%%%%%%%%%%%%
%\begin{ExoCt}{Raisonner.}{1234}{2}{0}{0}{0}{0}
% 
%\end{ExoCt}
%
%%%%%%%%%%%%%%%%%%%%%%%%%%%%%%%%%%%%%%%%%%%%%%%%%%%%%%%%%%%%%%%%%%%%
%\begin{ExoCt}{Représenter.}{1234}{2}{0}{0}{0}{0}
%
% 
%
%\end{ExoCt}
%
%%%%%%%%%%%%%%%%%%%%%%%%%%%%%%%%%%%%%%%%%%%%%%%%%%%%%%%%%%%%%%%%%%%%
%\begin{ExoCt}{Représenter.}{1234}{2}{0}{0}{0}{0}
%
% 
%
%\end{ExoCt} 
% 
%\end{pageParcourst}
%
%%%%%%%%%%%%%%%%%%%%%%%%%%%%%%%%%%%%%%%%%%%%%%%%%%%%%%%%%%%%%%%%%%%%
%%%%%  Brouillon
%%%%%%%%%%%%%%%%%%%%%%%%%%%%%%%%%%%%%%%%%%%%%%%%%%%%%%%%%%%%%%%%%%%%


\begin{pageBrouillon} 
 
\ligne{32}



\end{pageBrouillon}

%%%%%%%%%%%%%%%%%%%%%%%%%%%%%%%%%%%%%%%%%%%%%%%%%%%%%%%%%%%%%%%%%%%
%%%%  Auto
%%%%%%%%%%%%%%%%%%%%%%%%%%%%%%%%%%%%%%%%%%%%%%%%%%%%%%%%%%%%%%%%%%%


%%%%%%%%%%%%%%%%%%%%%%%%%%%%%%%%%%%%%%%%%%%%%%%%%%%%%%%%%%%%%%%%%%%
\begin{pageAuto} 


\begin{ExoAuto}{Raisonner.}{1234}{2}{0}{0}{0}{0}

 
%%%%%%%%%%%%%%%%%%%%%%%%%%%%%%%%%%%%%%%%%%%%%%%%%%%%%%%%%%%%%%%%%%%
\end{ExoAuto}

\begin{ExoAuto}{Raisonner.}{1234}{2}{0}{0}{0}{0}
  

\end{ExoAuto}

%%%%%%%%%%%%%%%%%%%%%%%%%%%%%%%%%%%%%%%%%%%%%%%%%%%%%%%%%%%%%%%%%%%
\begin{ExoAuto}{Raisonner.}{1234}{2}{0}{0}{0}{0}

 
 

\end{ExoAuto}

 
%%%%%%%%%%%%%%%%%%%%%%%%%%%%%%%%%%%%%%%%%%%%%%%%%%%%%%%%%%%%%%%%%%%
\begin{ExoAuto}{Raisonner.}{1234}{2}{0}{0}{0}{0}

 
 

\end{ExoAuto}


\end{pageAuto}
% Intervalles de R 
%
%%-----------------------------
%
%\chapter{Distance et valeur absolue}
%\chapter{Distance entre deux nombres}
{https://sacado.xyz/qcm/parcours_show_course/0/117129}
{


 \begin{CpsCol}
\textbf{Les savoir-faire du parcours}
 \begin{itemize}
 \item \textbf{Utiliser des nombres pour calculer et résoudre des problèmes}
\item[$\square$] Notation $\vert a\vert$. Distance entre deux nombres réels
\item[$\square$] Représentation de l'intervalle $[a-r;a+r]$ puis caractérisation par la condition $\vert x-	a\vert \leq r$.
\item[$\square$] Donner un encadrement, d’amplitude donnée, d’un nombre réel par des décimaux.
 \end{itemize}
 \end{CpsCol}

\begin{His}
\textbf{Georg Ferdinand Ludwig Philipp Cantor} (3 mars 1845, Saint-Pétersbourg – 6 janvier 1918, Halle) est un mathématicien allemand, connu pour être le créateur de la théorie des ensembles. Il établit l'importance de la bijection entre les ensembles, définit les ensembles infinis et les ensembles bien ordonnés. Il prouva également que les nombres réels sont « plus nombreux » que les entiers naturels. En fait, le théorème de Cantor implique l'existence d'une « infinité d'infinis ». Il définit les nombres cardinaux, les nombres ordinaux et leur arithmétique. Le travail de Cantor est d'un grand intérêt philosophique...https://fr.wikipedia.org/wiki/Georg\_Cantor
\end{His}


\begin{ExoDec}{Chercher.}{1234}{1}{0}{0}{0}

\end{ExoDec}

}

\begin{pageCours}
 

\section{Distance à 0}

\begin{DefT}{Valeur absolue}\index{Valeur absolue}
Soit $M$ un point d'abscisse $x$ sur la droite graduée d'origine $O$ d'abscisse 0 et $x$ un réel.\\
On note $\vert x \vert$ la distance de $M$ à $O$.\\
L'écriture $\vert x\vert$ est appelée \textbf{valeur absolue} de $x$. On peut alors écrite : $OM = d(O,M)=\vert x\vert$
\end{DefT}

 

\begin{Rq}
 La valeur absolue d'un nombre est un nombre positif.
\end{Rq}


\begin{Ex}
Sur cet exemple, $OA = OB = 4$. On peut écrire que $\vert -4 \vert = \vert 4 \vert = 4$. 

\definecolor{xfqqff}{rgb}{0.4980392156862745,0.,1.}
\definecolor{ffxfqq}{rgb}{1.,0.4980392156862745,0.}
\begin{tikzpicture}[line cap=round,line join=round,>=triangle 45,x=1.0cm,y=1.0cm]
\clip(-3.54,1.14) rectangle (7.68,2.56);
\draw [ line width=1.pt,color=xfqqff,domain=-3.54:7.68] plot(\x,{(--16.-0.*\x)/8.});
\draw (1.84,1.86) node[anchor=north west] {0};
\draw (2.84,1.86) node[anchor=north west] {1};
\begin{scriptsize}
\draw [color=ffxfqq] (-2.,2.)-- ++(-2.5pt,0 pt) -- ++(5.0pt,0 pt) ++(-2.5pt,-2.5pt) -- ++(0 pt,5.0pt);
\draw[color=ffxfqq] (-1.86,2.20) node {${\large  A}$};
\draw [color=ffxfqq] (6.,2.)-- ++(-2.5pt,0 pt) -- ++(5.0pt,0 pt) ++(-2.5pt,-2.5pt) -- ++(0 pt,5.0pt);
\draw[color=ffxfqq] (6.14,2.20) node {${\large B}$};
\draw [color=xfqqff] (2.02,2.)-- ++(-2.5pt,0 pt) -- ++(5.0pt,0 pt) ++(-2.5pt,-2.5pt) -- ++(0 pt,5.0pt);
\draw[color=xfqqff] (2.04,2.20) node {$O$};
\draw [color=xfqqff] (3.,2.)-- ++(-2.5pt,0 pt) -- ++(5.0pt,0 pt) ++(-2.5pt,-2.5pt) -- ++(0 pt,5.0pt);
\draw [color=xfqqff] (4.,2.)-- ++(-2.5pt,0 pt) -- ++(5.0pt,0 pt) ++(-2.5pt,-2.5pt) -- ++(0 pt,5.0pt);
\draw [color=xfqqff] (5.,2.)-- ++(-2.5pt,0 pt) -- ++(5.0pt,0 pt) ++(-2.5pt,-2.5pt) -- ++(0 pt,5.0pt);
\draw [color=xfqqff] (1.,2.)-- ++(-2.5pt,0 pt) -- ++(5.0pt,0 pt) ++(-2.5pt,-2.5pt) -- ++(0 pt,5.0pt);
\draw [color=xfqqff] (0.,2.)-- ++(-2.5pt,0 pt) -- ++(5.0pt,0 pt) ++(-2.5pt,-2.5pt) -- ++(0 pt,5.0pt);
\draw [color=xfqqff] (-1.,2.)-- ++(-2.5pt,0 pt) -- ++(5.0pt,0 pt) ++(-2.5pt,-2.5pt) -- ++(0 pt,5.0pt);
\draw [color=xfqqff] (-3.,2.)-- ++(-2.5pt,0 pt) -- ++(5.0pt,0 pt) ++(-2.5pt,-2.5pt) -- ++(0 pt,5.0pt);
\draw [color=xfqqff] (7.,2.)-- ++(-2.5pt,0 pt) -- ++(5.0pt,0 pt) ++(-2.5pt,-2.5pt) -- ++(0 pt,5.0pt);
\end{scriptsize}
\end{tikzpicture}

\end{Ex}



\section{Distance entre deux nombres}

\begin{DefT}{Généralisation. Distance entre deux nombres}\index{Distance entre deux nombres}
Soit $a$ et $b$ deux réels.\\
La distance entre $a$ et $b$ est égale à $\vert b-a\vert=\vert a-b\vert$. On peut écrire $d(a,b)=\vert b-a\vert=\vert a-b\vert$.
\end{DefT}

\begin{DefT}{Généralisation. Distance entre deux points}\index{Distance entre deux points}
Soit $a$ et $x$ deux réels.

Soit $A$ le point d'abscisse $a$ et $M$ le point d'abscisse $x$. $AM = d(a,x) = \vert x-a\vert$
\end{DefT}

 

\section{Appartenance d'un point à un segment}

\begin{Th}
$A$ et $B$ deux points d'abscisse respective $a$ et $b$ sur la droite graduée. 

Alors le milieu $I$ du segment $[AB]$ a pour abscisse $m=\frac{a+b}{2}$
\end{Th}

\begin{ThT}{Appartenance d'un point à un segment}
Soit $[AB]$ un segment, $I$ le milieu de $[AB]$ d'abscisse $m$ et $r = IA = IB$.\\
Le segment $[AB]$ est l'ensemble des points $M$ d'abscisse $x$ de la droite graduée tels que $\vert x- m \vert \leq r$.
\end{ThT}
 
\end{pageCours}


\begin{pageAD}

\Sf{fdfbd}

\begin{ExoCad}{Représenter. Chercher.}{1234}{2}{0}{0}{0}{0}

\paragraph{Parie A}

\begin{enumerate}
\item Tracer une droite graduée d'unité 3 cm.
\item Placer les points $M$ tel que $\vert OM \vert = 1$. On notera $M_1$ et $M_2$.
\item Placer les points $A$ tel que $\vert OA \vert = \frac{2}{3}$. On notera $A_1$ et $A_2$.
\end{enumerate}

Remarque : Dans ce type de question, on ne spécifie pas les 2 points que l'élève doit trouver ! C'est à l'élève de savoir ce qu'il doit faire. Ce type d'exercice se pose alors comme :

\paragraph{Parie B}

\begin{enumerate}
\item Tracer une droite graduée d'unité 2 cm.
\item Placer le point $A$ d'abscisse $a$ tel que $\vert a \vert = 2$.
\item Placer le point $B$ d'abscisse $b$ tel que $\vert b \vert = \frac{5}{2}$.
\item Placer le point $M$ d'abscisse $x$ tel que $OM = \frac{3}{4}$.
\end{enumerate}
\end{ExoCad}

\begin{ExoCad}{Représenter. Chercher.}{1234}{2}{0}{0}{0}{0}
 
Soit $A$ et $B$ d'abscisse respective $4$ et $-2$.

\begin{enumerate}
\item Le point $I$ est le milieu de $[AB]$. Quelle est l'abscisse du point $I$ ? 
\item Soit $M$ le point d'abscisse $x$ de la droite $(AB)$. Calculer $IM$. 
\item Compléter le tableau suivant.

\begin{tabular}{|c|p{1.5cm}|c|p{1.5cm}|c|p{1.5cm}|}
\hline 
Abscisses de $M$ & $IM$ & Abscisses de $M$ & $IM$ & Abscisses de $M$ & $IM$ \\ 
\hline 
$1$ & & $-6$ &  & $-1$ & \\ 
\hline 
$-2$ & & $3$ &  & $1$ & \\ 
\hline 
$5$ & & $0$ &  & $4$ & \\ 
\hline 
\end{tabular}

\item  A quelle condition le point $M$ appartient-il au segment $[AB]$ ?

\end{enumerate}

On pourra se rendre à la page Se rendre à la page : \url{https://www.geogebra.org/m/jsvqnzbq} pour visualiser la situation.
\end{ExoCad}


\begin{ExoCad}{Représenter.}{1234}{2}{0}{0}{0}{0}

Soit $x$ un réel donné. Traduire chaque valeur absolue comme distance entre deux réels.
\begin{enumerate}
\item $\vert x - 5 \vert$
\item $\vert 3 - x \vert$
\item $\vert x + 1 \vert$
\item $\vert -2 - x \vert$
\item $\vert 4 - 2x \vert$
\end{enumerate}  
 
\end{ExoCad}


\begin{ExoCad}{Représenter. Raisonner.}{1234}{2}{0}{0}{0}{0}

Représenter et résoudre les équations suivantes :
\begin{enumerate}
\item $d((x;4)=3$
\item $\vert x + 3 \vert = 1$
\item $\vert \frac{1}{2} - x \vert = \frac{3}{2}$
\item $\vert x - \frac{1}{3} \vert = \frac{5}{6}$
\item $\vert 2x - 1 \vert = 4$
\end{enumerate}  
 
\end{ExoCad}

 


\begin{ExoCad}{Représenter. Chercher.}{1234}{2}{0}{0}{0}{0}
 
Soit $A$ et $B$ d'abscisse respective $4$ et $-2$.

\begin{enumerate}
\item Le point $I$ est le milieu de $[AB]$. Quelle est l'abscisse du point $I$ ? 
\item Soit $M$ le point d'abscisse $x$ de la droite $(AB)$. Calculer $IM$. 
\item Compléter le tableau suivant.

\begin{tabular}{|c|p{1.5cm}|c|p{1.5cm}|c|p{1.5cm}|}
\hline 
Abscisses de $M$ & $IM$ & Abscisses de $M$ & $IM$ & Abscisses de $M$ & $IM$ \\ 
\hline 
$1$ & & $-6$ &  & $-1$ & \\ 
\hline 
$-2$ & & $3$ &  & $1$ & \\ 
\hline 
$5$ & & $0$ &  & $4$ & \\ 
\hline 
\end{tabular}

\item  A quelle condition le point $M$ appartient-il au segment $[AB]$ ?

\end{enumerate}

On pourra se rendre à la page Se rendre à la page : \url{https://www.geogebra.org/m/jsvqnzbq} pour visualiser la situation. 
\end{ExoCad}
 

\begin{ExoCad}{Chercher. Communiquer.}{1234}{2}{0}{0}{0}{0}

 
\begin{enumerate}
\item Soit $A$ et $B$ d'abscisse respective $3$ et $-1$. Déterminer le rayon de l'intervalle $[AB]$.
\item Représenter le segment $[AB]$ par un intervalle puis par une inégalité.

\item Représenter sur la droite graduée le segment $[AB]$.
 
\end{enumerate}
  
\end{ExoCad}


 
 
\begin{ExoCad}{Représenter. Chercher.}{1234}{2}{0}{0}{0}{0}

On donne les segments $[AB]$ et $[EF]$ représentés ci-dessous.

\definecolor{ttqqqq}{rgb}{0.2,0.,0.}
\definecolor{qqzzff}{rgb}{0.,0.6,1.}
\definecolor{qqzzcc}{rgb}{0.,0.6,0.8}
\begin{tikzpicture}[line cap=round,line join=round,>=triangle 45,x=1.0cm,y=1.0cm]
\begin{axis}[
x=1.0cm,y=1.0cm,
axis lines=middle,
xmin=-2.200000000000002,
xmax=8.480000000000008,
ymin=-0.8000000000000048,
ymax=0.7799999999999957,
xtick={-2.0,-1.0,...,8.0},
ytick={-0.0,1.0,...,0.0},]
\clip(-2.2,-0.8) rectangle (8.48,0.78);
\draw [line width=2.pt,color=qqzzff] (-2.,0.)-- (4.,0.);
\draw [line width=2.pt] (5.,0.)-- (8.,0.);
\begin{scriptsize}
\draw [color=qqzzcc] (-2.,0.)-- ++(-2.5pt,0 pt) -- ++(5.0pt,0 pt) ++(-2.5pt,-2.5pt) -- ++(0 pt,5.0pt);
\draw[color=qqzzcc] (-1.86,0.37) node {$A$};
\draw [color=qqzzcc] (4.,0.)-- ++(-2.5pt,0 pt) -- ++(5.0pt,0 pt) ++(-2.5pt,-2.5pt) -- ++(0 pt,5.0pt);
\draw[color=qqzzcc] (3.96,0.37) node {$B$};
\draw [color=ttqqqq] (5.,0.)-- ++(-2.5pt,0 pt) -- ++(5.0pt,0 pt) ++(-2.5pt,-2.5pt) -- ++(0 pt,5.0pt);
\draw[color=ttqqqq] (5.14,0.37) node {$E$};
\draw [color=ttqqqq] (8.,0.)-- ++(-2.5pt,0 pt) -- ++(5.0pt,0 pt) ++(-2.5pt,-2.5pt) -- ++(0 pt,5.0pt);
\draw[color=ttqqqq] (8.14,0.37) node {$F$};
\end{scriptsize}
\end{axis}
\end{tikzpicture}


\begin{enumerate}
\item 
	\begin{enumerate}
		\item Déterminer le rayon de l'intervalle $[AB]$.
		\item Représenter $[AB]$ par une inégalité.
	\end{enumerate}
\item 
	\begin{enumerate}
		\item Déterminer le rayon de l'intervalle $[EF]$.
		\item Représenter $[EF]$ par une inégalité.
	\end{enumerate}
\end{enumerate}
 
\end{ExoCad}

\begin{ExoCad}{Représenter. Chercher.}{1234}{2}{0}{0}{0}{0}

Soit $x$ un réel.   


\begin{enumerate}
	\item Déterminer puis représenter l'ensemble des points $M$ d'abscisse $x$ tel que $\vert x- 3 \vert \leq 3$.
	\item Déterminer puis représenter l'ensemble des points $M$ d'abscisse $x$ tel que $\vert x+4 \vert \leq 1$.
	\item Déterminer puis représenter l'ensemble des points $M$ d'abscisse $x$ tel que $\vert x+\frac{2}{3} \vert \leq 4$.
\end{enumerate}
 
\end{ExoCad}

\begin{ExoCad}{Représenter. Chercher.}{1234}{2}{0}{0}{0}{0}

Soit $x$ un réel.   


\begin{enumerate}
	\item Déterminer puis représenter l'ensemble des points $M$ d'abscisse $x$ tel que $\vert x - \frac{4}{5} \vert \leq \frac{1}{2}$.
    \item Déterminer puis représenter l'ensemble des points $M$ d'abscisse $x$ tel que $\vert x- \pi \vert \leq 1$.
	\item Déterminer puis représenter l'ensemble des points $M$ d'abscisse $x$ tel que $\vert x - \sqrt{2}  \vert \leq  1$.
	\item Écrire à l'aide d'une double inégalité puis représenter  l'ensemble tel  que $\vert x + \frac{2}{3} \vert \leq 5$.
	\item Écrire à l'aide d'une double inégalité puis représenter  l'ensemble tel  que $\vert x+ \pi \vert \leq 10^{-1}$.	
\end{enumerate}
 
\end{ExoCad}

\begin{ExoCad}{Représenter. Chercher.}{1234}{2}{0}{0}{0}{0}

\begin{enumerate}
\item On s'intéresse à $\frac{3}{7}$.
\begin{enumerate}
	\item 1 est-elle une valeur approchée de $\frac{3}{7}$ à $ 10^{-1}$ près ?
	\item Déterminer une valeur approchée $a$ à $ 10^{-1}$ près de $\frac{3}{7}$.	
\end{enumerate}

\item
On s'intéresse à $\sqrt{10}$.
\begin{enumerate}
	\item Déterminer un encadrement de $\sqrt{10}$ à $ 10^{-2}$ près .
	\item Déterminer à la calculatrice  une valeur approchée de  $\sqrt{10}$.	
\end{enumerate}
\end{enumerate}
  
\end{ExoCad}


\end{pageAD}

%%%%%%%%%%%%%%%%%%%%%%%%%%%%%%%%%%%%%%%%%%%%%%%%%%%%%%%%%%%%%%%%%%%%%%%%%%%%%%%%%%%%%%%%%%%%%%%%%%%%%%%%%%%%%%%%%%%%%%%%%%%
%%%%%%%%%%%%%%%%%%%%%%%%%%%%%%%%%%%%%%%%%%%%%%%%%%%%%%%%%%%%%%%%%%%%%%%%%%%%%%%%%%%%%%%%%%%%%%%%%%%%%%%%%%%%%%%%%%%%%%%%%%%
%%%%%%%%%%%%%%%              pageParcoursu                         %%%%%%%%%%%%%%%%%%%%%%%%%%%%%%%%%%%%%%%%%%%%%%%%%%%%%%%%
%%%%%%%%%%%%%%%%%%%%%%%%%%%%%%%%%%%%%%%%%%%%%%%%%%%%%%%%%%%%%%%%%%%%%%%%%%%%%%%%%%%%%%%%%%%%%%%%%%%%%%%%%%%%%%%%%%%%%%%%%%%
%%%%%%%%%%%%%%%%%%%%%%%%%%%%%%%%%%%%%%%%%%%%%%%%%%%%%%%%%%%%%%%%%%%%%%%%%%%%%%%%%%%%%%%%%%%%%%%%%%%%%%%%%%%%%%%%%%%%%%%%%%%
\begin{pageParcoursu}

\begin{ExoCu}{Représenter. Chercher.}{1234}{2}{0}{0}{0}{0}
 
\end{ExoCu}

\begin{ExoCu}{Représenter. Chercher.}{1234}{2}{0}{0}{0}{0}
 
\end{ExoCu}

\begin{ExoCu}{Représenter. Chercher.}{1234}{2}{0}{0}{0}{0}
 
\end{ExoCu}


\end{pageParcoursu}
%%%%%%%%%%%%%%%%%%%%%%%%%%%%%%%%%%%%%%%%%%%%%%%%%%%%%%%%%%%%%%%%%%%%%%%%%%%%%%%%%%%%%%%%%%%%%%%%%%%%%%%%%%%%%%%%%%%%%%%%%%%
%%%%%%%%%%%%%%%%%%%%%%%%%%%%%%%%%%%%%%%%%%%%%%%%%%%%%%%%%%%%%%%%%%%%%%%%%%%%%%%%%%%%%%%%%%%%%%%%%%%%%%%%%%%%%%%%%%%%%%%%%%%
%%%%%%%%%%%%%%%              pageParcoursd                    %%%%%%%%%%%%%%%%%%%%%%%%%%%%%%%%%%%%%%%%%%%%%%%%%%%%%%%%%%%%%
%%%%%%%%%%%%%%%%%%%%%%%%%%%%%%%%%%%%%%%%%%%%%%%%%%%%%%%%%%%%%%%%%%%%%%%%%%%%%%%%%%%%%%%%%%%%%%%%%%%%%%%%%%%%%%%%%%%%%%%%%%%
%%%%%%%%%%%%%%%%%%%%%%%%%%%%%%%%%%%%%%%%%%%%%%%%%%%%%%%%%%%%%%%%%%%%%%%%%%%%%%%%%%%%%%%%%%%%%%%%%%%%%%%%%%%%%%%%%%%%%%%%%%%

\begin{pageParcoursd}
 

\begin{ExoCd}{Chercher. Communiquer.}{1234}{2}{0}{0}{0}{0}

 
\begin{enumerate}
\item Soit $A$ et $B$ d'abscisse respective $3$ et $-1$. Déterminer le rayon de l'intervalle $[AB]$.
\item Représenter le segment $[AB]$ par un intervalle puis par une inégalité.

\item Représenter sur la droite graduée le segment $[AB]$.
 
\end{enumerate}
  
\end{ExoCd}


 
 
\begin{ExoCd}{Représenter. Chercher.}{1234}{2}{0}{0}{0}{0}

On donne les segments $[AB]$ et $[EF]$ représentés ci-dessous.

\definecolor{ttqqqq}{rgb}{0.2,0.,0.}
\definecolor{qqzzff}{rgb}{0.,0.6,1.}
\definecolor{qqzzcc}{rgb}{0.,0.6,0.8}
\begin{tikzpicture}[line cap=round,line join=round,>=triangle 45,x=1.0cm,y=1.0cm]
\begin{axis}[
x=1.0cm,y=1.0cm,
axis lines=middle,
xmin=-2.200000000000002,
xmax=8.480000000000008,
ymin=-0.8000000000000048,
ymax=0.7799999999999957,
xtick={-2.0,-1.0,...,8.0},
ytick={-0.0,1.0,...,0.0},]
\clip(-2.2,-0.8) rectangle (8.48,0.78);
\draw [line width=2.pt,color=qqzzff] (-2.,0.)-- (4.,0.);
\draw [line width=2.pt] (5.,0.)-- (8.,0.);
\begin{scriptsize}
\draw [color=qqzzcc] (-2.,0.)-- ++(-2.5pt,0 pt) -- ++(5.0pt,0 pt) ++(-2.5pt,-2.5pt) -- ++(0 pt,5.0pt);
\draw[color=qqzzcc] (-1.86,0.37) node {$A$};
\draw [color=qqzzcc] (4.,0.)-- ++(-2.5pt,0 pt) -- ++(5.0pt,0 pt) ++(-2.5pt,-2.5pt) -- ++(0 pt,5.0pt);
\draw[color=qqzzcc] (3.96,0.37) node {$B$};
\draw [color=ttqqqq] (5.,0.)-- ++(-2.5pt,0 pt) -- ++(5.0pt,0 pt) ++(-2.5pt,-2.5pt) -- ++(0 pt,5.0pt);
\draw[color=ttqqqq] (5.14,0.37) node {$E$};
\draw [color=ttqqqq] (8.,0.)-- ++(-2.5pt,0 pt) -- ++(5.0pt,0 pt) ++(-2.5pt,-2.5pt) -- ++(0 pt,5.0pt);
\draw[color=ttqqqq] (8.14,0.37) node {$F$};
\end{scriptsize}
\end{axis}
\end{tikzpicture}


\begin{enumerate}
\item 
	\begin{enumerate}
		\item Déterminer le rayon de l'intervalle $[AB]$.
		\item Représenter $[AB]$ par une inégalité.
	\end{enumerate}
\item 
	\begin{enumerate}
		\item Déterminer le rayon de l'intervalle $[EF]$.
		\item Représenter $[EF]$ par une inégalité.
	\end{enumerate}
\end{enumerate}
 
\end{ExoCd}

\begin{ExoCd}{Représenter. Chercher.}{1234}{2}{0}{0}{0}{0}

Soit $x$ un réel.   


\begin{enumerate}
	\item Déterminer puis représenter l'ensemble des points $M$ d'abscisse $x$ tel que $\vert x- 3 \vert \leq 3$.
	\item Déterminer puis représenter l'ensemble des points $M$ d'abscisse $x$ tel que $\vert x+4 \vert \leq 1$.
	\item Déterminer puis représenter l'ensemble des points $M$ d'abscisse $x$ tel que $\vert x+\frac{2}{3} \vert \leq 4$.
\end{enumerate}
 
\end{ExoCd}

\begin{ExoCd}{Représenter. Chercher.}{1234}{2}{0}{0}{0}{0}

Soit $x$ un réel.   


\begin{enumerate}
	\item Déterminer puis représenter l'ensemble des points $M$ d'abscisse $x$ tel que $\vert x - \frac{4}{5} \vert \leq \frac{1}{2}$.
    \item Déterminer puis représenter l'ensemble des points $M$ d'abscisse $x$ tel que $\vert x- \pi \vert \leq 1$.
	\item Déterminer puis représenter l'ensemble des points $M$ d'abscisse $x$ tel que $\vert x - \sqrt{2}  \vert \leq  1$.
	\item Écrire à l'aide d'une double inégalité puis représenter  l'ensemble tel  que $\vert x + \frac{2}{3} \vert \leq 5$.
	\item Écrire à l'aide d'une double inégalité puis représenter  l'ensemble tel  que $\vert x+ \pi \vert \leq 10^{-1}$.	
\end{enumerate}
 
\end{ExoCd}

\begin{ExoCd}{Représenter. Chercher.}{1234}{2}{0}{0}{0}{0}

\begin{enumerate}
\item On s'intéresse à $\frac{3}{7}$.
\begin{enumerate}
	\item 1 est-elle une valeur approchée de $\frac{3}{7}$ à $ 10^{-1}$ près ?
	\item Déterminer une valeur approchée $a$ à $ 10^{-1}$ près de $\frac{3}{7}$.	
\end{enumerate}

\item
On s'intéresse à $\sqrt{10}$.
\begin{enumerate}
	\item Déterminer un encadrement de $\sqrt{10}$ à $ 10^{-2}$ près .
	\item Déterminer à la calculatrice  une valeur approchée de  $\sqrt{10}$.	
\end{enumerate}
\end{enumerate}
  
\end{ExoCd}


\end{pageParcoursd}

%%%%%%%%%%%%%%%%%%%%%%%%%%%%%%%%%%%%%%%%%%%%%%%%%%%%%%%%%%%%%%%%%%%%%%%%%%%%%%%%%%%%%%%%%%%%%%%%%%%%%%%%%%%%%%%%%%%%%%%%%%%
%%%%%%%%%%%%%%%%%%%%%%%%%%%%%%%%%%%%%%%%%%%%%%%%%%%%%%%%%%%%%%%%%%%%%%%%%%%%%%%%%%%%%%%%%%%%%%%%%%%%%%%%%%%%%%%%%%%%%%%%%%%
%%%%%%%%%%%%%%%            pageParcourst                      %%%%%%%%%%%%%%%%%%%%%%%%%%%%%%%%%%%%%%%%%%%%%%%%%%%%%%%%%%%%%
%%%%%%%%%%%%%%%%%%%%%%%%%%%%%%%%%%%%%%%%%%%%%%%%%%%%%%%%%%%%%%%%%%%%%%%%%%%%%%%%%%%%%%%%%%%%%%%%%%%%%%%%%%%%%%%%%%%%%%%%%%%
%%%%%%%%%%%%%%%%%%%%%%%%%%%%%%%%%%%%%%%%%%%%%%%%%%%%%%%%%%%%%%%%%%%%%%%%%%%%%%%%%%%%%%%%%%%%%%%%%%%%%%%%%%%%%%%%%%%%%%%%%%%

\begin{pageParcourst}

\begin{ExoCt}{Représenter. Chercher.}{1234}{2}{0}{0}{0}{0}
 
\end{ExoCt}

\begin{ExoCt}{Représenter. Chercher.}{1234}{2}{0}{0}{0}{0}
 
\end{ExoCt}

\begin{ExoCt}{Représenter. Chercher.}{1234}{2}{0}{0}{0}{0}
 
\end{ExoCt}


\end{pageParcourst}
%%%%%%%%%%%%%%%%%%%%%%%%%%%%%%%%%%%%%%%%%%%%%%%%%%%%%%%%%%%%%%%%%%%%%%%%%%%%%%%%%%%%%%%%%%%%%%%%%%%%%%%%%%%%%%%%%%%%%%%%%%%
%%%%%%%%%%%%%%%%%%%%%%%%%%%%%%%%%%%%%%%%%%%%%%%%%%%%%%%%%%%%%%%%%%%%%%%%%%%%%%%%%%%%%%%%%%%%%%%%%%%%%%%%%%%%%%%%%%%%%%%%%%%
%%%%%%%%%%%%%%%              pageAuto                         %%%%%%%%%%%%%%%%%%%%%%%%%%%%%%%%%%%%%%%%%%%%%%%%%%%%%%%%%%%%%
%%%%%%%%%%%%%%%%%%%%%%%%%%%%%%%%%%%%%%%%%%%%%%%%%%%%%%%%%%%%%%%%%%%%%%%%%%%%%%%%%%%%%%%%%%%%%%%%%%%%%%%%%%%%%%%%%%%%%%%%%%%
%%%%%%%%%%%%%%%%%%%%%%%%%%%%%%%%%%%%%%%%%%%%%%%%%%%%%%%%%%%%%%%%%%%%%%%%%%%%%%%%%%%%%%%%%%%%%%%%%%%%%%%%%%%%%%%%%%%%%%%%%%%
\begin{pageAuto}

\begin{ExoAuto}{Représenter. Chercher.}{1234}{2}{0}{0}{0}{0}
 
\end{ExoAuto}

\begin{ExoAuto}{Représenter. Chercher.}{1234}{2}{0}{0}{0}{0}
 
\end{ExoAuto}

\begin{ExoAuto}{Représenter. Chercher.}{1234}{2}{0}{0}{0}{0}
 
\end{ExoAuto}


\end{pageAuto}% Distance entre deux réels
%\chapter{Les ensembles de nombres et intervalles}
{https://sacado.xyz/qcm/parcours_show_course/0/117129}
{


 \begin{CpsCol}
\textbf{Les savoir-faire du parcours}
 \begin{itemize}
 \item \textbf{Utiliser des nombres pour calculer et résoudre des problèmes}
\item[$\square$] \textbf{Chercher :}  Tester, essayer plusieurs pistes de résolution.
\item[$\square$] \textbf{Représenter :} Produire et utiliser plusieurs représentations des nombres.
\item[$\square$] \textbf{Raisonner :} Mener collectivement une investigation en sachant prendre en compte le point de vue d’autrui.
\item[$\square$] \textbf{Communiquer :} Expliquer à l’oral ou à l’écrit (sa démarche, son raisonnement, un calcul, un protocole de construction géométrique, un algorithme), comprendre les explications d’un autre
et argumenter dans l’échange.
 \end{itemize}
 \end{CpsCol}

}

\begin{pageCours}


 \end{pageCours}

\begin{pageAD}


\begin{ExoCad}{Représenter. Raisonner.}{1234}{2}{0}{0}{0}{0}
 

\begin{tabular}{ccc}

$\frac{2}{10}..........\Z$ & $-\sqrt{25}..........\Z$ & $\frac{\sqrt{3}}{4}..........\Q$ \\ 

$\pi..........\R$  & $-\frac{5}{3}..........\Q$  &  $\sqrt{11}..........\R$ \\ 

\end{tabular} 

\end{ExoCad}


\begin{ExoCad}{Représenter. Raisonner.}{1234}{2}{0}{0}{0}{0}

Recopier et compléter le tableau.

%\begin{tabular}{|c|c|c|}
%\hline 
%Intervalle & Inégalité & Représentation  \vplus \\ 
%\hline 
%$x\in \left[ -5 ; \frac{2}{3}\right]$ & $-5  \leq x \leq  \frac{2}{3} $  &  \vplus \\ 
%\hline 
% & $-1 \leq x <4$ &  \vplus  \\ 
%\hline 
%$x\in \left[ 3 ; 6 \right[ $  &  &  \vplus  \\ 
%\hline 
% &  & \definecolor{ffdxqq}{rgb}{1.,0.8431372549019608,0.}
%\definecolor{ffxfqq}{rgb}{1.,0.4980392156862745,0.}
%\begin{tikzpicture}[line cap=round,line join=round,>=triangle 45,x=1.0cm,y=1.0cm]
%\draw[->,color=black] (-5.174092090680384,0.) -- (2.566282833730012,0.);
%\foreach \x in {-5.,-4.,-3.,-2.,-1.,1.,2.}
%\draw[shift={(\x,0)},color=black] (0pt,2pt) -- (0pt,-2pt) node[below] {\footnotesize $\x$};
%\draw[color=black] (0pt,-10pt) node[right] {\footnotesize $0$};
%\clip(-5.174092090680384,-0.4115875953650586) rectangle (2.566282833730012,0.4791698364123281);
%\draw [line width=2.4pt,color=ffxfqq] (-3.,0.)-- (2.,0.);
%\end{tikzpicture}  \vplus \\ 
%\hline 
%\end{tabular} 

\end{ExoCad}


\begin{ExoCad}{Représenter. Raisonner. Communiquer.}{1234}{2}{0}{0}{0}{0}
 

Déterminer l'ensemble des valeurs de $x$ dans chaque cas.
\begin{enumerate}
\item On jette un dé à 6 face et on regarde la face obtenue. Soit $x$ le numéro de la face. 
\item $[-1,1;3]$ et $[2,9;6]$
\item $x > -4$ et $x \leq 10$
\item $x \leq -3$ et $x \leq 5$
\item $x \leq 5$ ou $x \geq 2$
\end{enumerate}
\end{ExoCad}

\begin{ExoCad}{Représenter. Raisonner. Communiquer.}{1234}{2}{0}{0}{0}{0}
 


On propose dans chaque cas deux ensembles $A$ et $B$. Lequel est inclus dans l'autre ?  

\textit{{\small On pourra représenter chaque intervalle sur une droite graduée.}}

\begin{minipage}{0.48\linewidth}

 
\begin{enumerate}
\item $A = \left[ -\frac{11}{10};\frac{29}{10}\right]$ et $B=\left[-\frac{3}{2};3 \right]$
\item $A =\left[ \frac{1}{2}; +\infty \right[$ et $B=[0,7;0,8]$.
\item $A =[1;2]$ et $B=]1;2[$. 
\end{enumerate}

\end{minipage}
\hfill
\begin{minipage}{0.48\linewidth}
 
\begin{enumerate}

\item

\begin{tikzpicture}[line cap=round,line join=round,>=triangle 45,x=1.0cm,y=1.0cm]
\draw [->,line width=1.pt,domain=0.34:6.36] plot(\x,{(-14.-0.*\x)/7.});
\end{tikzpicture}
\item

\begin{tikzpicture}[line cap=round,line join=round,>=triangle 45,x=1.0cm,y=1.0cm]
\draw [->,line width=1.pt,domain=0.34:6.36] plot(\x,{(-14.-0.*\x)/7.});
\end{tikzpicture}
\item

\begin{tikzpicture}[line cap=round,line join=round,>=triangle 45,x=1.0cm,y=1.0cm]
\draw [->,line width=1.pt,domain=0.34:6.36] plot(\x,{(-14.-0.*\x)/7.});
\end{tikzpicture}
\end{enumerate}

\end{minipage}
 
 \end{ExoCad}
 
\begin{ExoCad}{Représenter. Raisonner. Communiquer.}{1234}{2}{0}{0}{0}{0}


Déterminer les intersections des ensembles suivants. On écrira : $A \cap B = $ où $A$ et $B$ sont les ensembles ci-dessous.
 

\textit{{\small On pourra représenter chaque intervalle sur une droite graduée.}}



\begin{minipage}{0.48\linewidth}

\begin{enumerate}
\item $\Z$ et $\Q$
\item $[-5;2[$ et $[0;7]$
\item $[-1;4]$ et $[-3;-1]$
\item $\N$ et $]-\infty;5]$
\item $[-5;0[$ et $[0;3]$
\end{enumerate}

\end{minipage}
\hfill
\begin{minipage}{0.48\linewidth}
 
\begin{enumerate}
\item

\begin{tikzpicture}[line cap=round,line join=round,>=triangle 45,x=1.0cm,y=1.0cm]
\draw [->,line width=1.pt,domain=0.34:6.36] plot(\x,{(-14.-0.*\x)/7.});
\end{tikzpicture}
\item

\begin{tikzpicture}[line cap=round,line join=round,>=triangle 45,x=1.0cm,y=1.0cm]
\draw [->,line width=1.pt,domain=0.34:6.36] plot(\x,{(-14.-0.*\x)/7.});
\end{tikzpicture}
\item

\begin{tikzpicture}[line cap=round,line join=round,>=triangle 45,x=1.0cm,y=1.0cm]
\draw [->,line width=1.pt,domain=0.34:6.36] plot(\x,{(-14.-0.*\x)/7.});
\end{tikzpicture}
\item

\begin{tikzpicture}[line cap=round,line join=round,>=triangle 45,x=1.0cm,y=1.0cm]
\draw [->,line width=1.pt,domain=0.34:6.36] plot(\x,{(-14.-0.*\x)/7.});
\end{tikzpicture}
\item

\begin{tikzpicture}[line cap=round,line join=round,>=triangle 45,x=1.0cm,y=1.0cm]
\draw [->,line width=1.pt,domain=0.34:6.36] plot(\x,{(-14.-0.*\x)/7.});
\end{tikzpicture}
\end{enumerate}

\end{minipage}
\end{ExoCad}

\begin{ExoCad}{Représenter. Raisonner. Communiquer.}{1234}{2}{0}{0}{0}{0}


Déterminer, dans chaque cas, la réunion des ensembles suivants. On écrira : $A \cup B = $ où $A$ et $B$ sont les ensembles ci-dessous.
\begin{enumerate}
\item $\Q$ et $\R$
\item $\left\lbrace 1;3;5;7  \right\rbrace $ et $\left\lbrace 0;2;4;8  \right\rbrace $
\item $[-3;4]$ et $[2;6]$
\item $[0;+\infty[$ et $]-\infty;5]$
\end{enumerate}
On pourra représenter chaque intervalle sur une droite graduée tracée à main levée.
 \end{ExoCad}

\begin{ExoCad}{Calculer. Représenter. Raisonner. Communiquer.}{1234}{2}{0}{0}{0}{0}

 

\begin{enumerate}
\item On considère le nombre $\frac{19}{11}$.

\begin{enumerate}
\item Donner le développement décimal de $\frac{19}{11}$ avec 8 chiffres significatifs. $\frac{19}{11}$ semble-t-il décimal ?
\item On dit que $\frac{19}{11}$ a une écriture périodique.
Préciser sa période (série de chiffres qui se répète à l'infini dans le développement décimal).
\end{enumerate}
\item On considère le nombre $x=0,13131313....$ dont le développement décimal a pour période 13.
\begin{enumerate}
\item Démontrer que $100x = 13 + x$. 
\item  En déduire une écriture fractionnaire de $x$. Quelle est la nature du nombre $x$ ?
\end{enumerate}
\item Démontrer que $x=3,412412412...$ est un nombre rationnel. 
\item Estimer le résultat avec la calculatrice.
\end{enumerate}
\end{ExoCad}





\begin{ExoCad}{Représenter. Chercher.}{1234}{2}{0}{0}{0}{0}
 
Soit $A$ et $B$ d'abscisse respective $4$ et $-2$.

\begin{enumerate}
\item Le point $I$ est le milieu de $[AB]$. Quelle est l'abscisse du point $I$ ? 
\item Soit $M$ le point d'abscisse $x$ de la droite $(AB)$. Calculer $IM$. 
\item Compléter le tableau suivant.

\begin{tabular}{|c|p{1.5cm}|c|p{1.5cm}|c|p{1.5cm}|}
\hline 
Abscisses de $M$ & $IM$ & Abscisses de $M$ & $IM$ & Abscisses de $M$ & $IM$ \\ 
\hline 
$1$ & & $-6$ &  & $-1$ & \\ 
\hline 
$-2$ & & $3$ &  & $1$ & \\ 
\hline 
$5$ & & $0$ &  & $4$ & \\ 
\hline 
\end{tabular}

\item  A quelle condition le point $M$ appartient-il au segment $[AB]$ ?

\end{enumerate}

On pourra se rendre à la page Se rendre à la page : \url{https://www.geogebra.org/m/jsvqnzbq} pour visualiser la situation. 
\end{ExoCad}
 

\begin{ExoCad}{Chercher. Communiquer.}{1234}{2}{0}{0}{0}{0}

 
\begin{enumerate}
\item Soit $A$ et $B$ d'abscisse respective $3$ et $-1$. Déterminer le rayon de l'intervalle $[AB]$.
\item Représenter le segment $[AB]$ par un intervalle puis par une inégalité.

\item Représenter sur la droite graduée le segment $[AB]$.
 
\end{enumerate}
  
\end{ExoCad}


 
 
\begin{ExoCad}{Représenter. Chercher.}{1234}{2}{0}{0}{0}{0}

On donne les segments $[AB]$ et $[EF]$ représentés ci-dessous.

\definecolor{ttqqqq}{rgb}{0.2,0.,0.}
\definecolor{qqzzff}{rgb}{0.,0.6,1.}
\definecolor{qqzzcc}{rgb}{0.,0.6,0.8}
\begin{tikzpicture}[line cap=round,line join=round,>=triangle 45,x=1.0cm,y=1.0cm]
\begin{axis}[
x=1.0cm,y=1.0cm,
axis lines=middle,
xmin=-2.200000000000002,
xmax=8.480000000000008,
ymin=-0.8000000000000048,
ymax=0.7799999999999957,
xtick={-2.0,-1.0,...,8.0},
ytick={-0.0,1.0,...,0.0},]
\clip(-2.2,-0.8) rectangle (8.48,0.78);
\draw [line width=2.pt,color=qqzzff] (-2.,0.)-- (4.,0.);
\draw [line width=2.pt] (5.,0.)-- (8.,0.);
\begin{scriptsize}
\draw [color=qqzzcc] (-2.,0.)-- ++(-2.5pt,0 pt) -- ++(5.0pt,0 pt) ++(-2.5pt,-2.5pt) -- ++(0 pt,5.0pt);
\draw[color=qqzzcc] (-1.86,0.37) node {$A$};
\draw [color=qqzzcc] (4.,0.)-- ++(-2.5pt,0 pt) -- ++(5.0pt,0 pt) ++(-2.5pt,-2.5pt) -- ++(0 pt,5.0pt);
\draw[color=qqzzcc] (3.96,0.37) node {$B$};
\draw [color=ttqqqq] (5.,0.)-- ++(-2.5pt,0 pt) -- ++(5.0pt,0 pt) ++(-2.5pt,-2.5pt) -- ++(0 pt,5.0pt);
\draw[color=ttqqqq] (5.14,0.37) node {$E$};
\draw [color=ttqqqq] (8.,0.)-- ++(-2.5pt,0 pt) -- ++(5.0pt,0 pt) ++(-2.5pt,-2.5pt) -- ++(0 pt,5.0pt);
\draw[color=ttqqqq] (8.14,0.37) node {$F$};
\end{scriptsize}
\end{axis}
\end{tikzpicture}


\begin{enumerate}
\item 
	\begin{enumerate}
		\item Déterminer le rayon de l'intervalle $[AB]$.
		\item Représenter $[AB]$ par une inégalité.
	\end{enumerate}
\item 
	\begin{enumerate}
		\item Déterminer le rayon de l'intervalle $[EF]$.
		\item Représenter $[EF]$ par une inégalité.
	\end{enumerate}
\end{enumerate}
 
\end{ExoCad}

\begin{ExoCad}{Représenter. Chercher.}{1234}{2}{0}{0}{0}{0}

Soit $x$ un réel.   


\begin{enumerate}
	\item Déterminer puis représenter l'ensemble des points $M$ d'abscisse $x$ tel que $\vert x- 3 \vert \leq 3$.
	\item Déterminer puis représenter l'ensemble des points $M$ d'abscisse $x$ tel que $\vert x+4 \vert \leq 1$.
	\item Déterminer puis représenter l'ensemble des points $M$ d'abscisse $x$ tel que $\vert x+\frac{2}{3} \vert \leq 4$.
\end{enumerate}
 
\end{ExoCad}

\begin{ExoCad}{Représenter. Chercher.}{1234}{2}{0}{0}{0}{0}

Soit $x$ un réel.   


\begin{enumerate}
	\item Déterminer puis représenter l'ensemble des points $M$ d'abscisse $x$ tel que $\vert x - \frac{4}{5} \vert \leq \frac{1}{2}$.
    \item Déterminer puis représenter l'ensemble des points $M$ d'abscisse $x$ tel que $\vert x- \pi \vert \leq 1$.
	\item Déterminer puis représenter l'ensemble des points $M$ d'abscisse $x$ tel que $\vert x - \sqrt{2}  \vert \leq  1$.
	\item Écrire à l'aide d'une double inégalité puis représenter  l'ensemble tel  que $\vert x + \frac{2}{3} \vert \leq 5$.
	\item Écrire à l'aide d'une double inégalité puis représenter  l'ensemble tel  que $\vert x+ \pi \vert \leq 10^{-1}$.	
\end{enumerate}
 
\end{ExoCad}

\begin{ExoCad}{Représenter. Chercher.}{1234}{2}{0}{0}{0}{0}

\begin{enumerate}
\item On s'intéresse à $\frac{3}{7}$.
\begin{enumerate}
	\item 1 est-elle une valeur approchée de $\frac{3}{7}$ à $ 10^{-1}$ près ?
	\item Déterminer une valeur approchée $a$ à $ 10^{-1}$ près de $\frac{3}{7}$.	
\end{enumerate}

\item
On s'intéresse à $\sqrt{10}$.
\begin{enumerate}
	\item Déterminer un encadrement de $\sqrt{10}$ à $ 10^{-2}$ près .
	\item Déterminer à la calculatrice  une valeur approchée de  $\sqrt{10}$.	
\end{enumerate}
\end{enumerate}
  
\end{ExoCad}


\end{pageAD}

%%%%%%%%%%%%%%%%%%%%%%%%%%%%%%%%%%%%%%%%%%%%%%%%%%%%%%%%%%%%%%%%%%%%%%%%%%%%%%%%%%%%%%%%%%%%%%%%%%%%%%%%%%%%%%%%%%%%%%%%%%%
%%%%%%%%%%%%%%%%%%%%%%%%%%%%%%%%%%%%%%%%%%%%%%%%%%%%%%%%%%%%%%%%%%%%%%%%%%%%%%%%%%%%%%%%%%%%%%%%%%%%%%%%%%%%%%%%%%%%%%%%%%%
%%%%%%%%%%%%%%%              pageParcoursu                         %%%%%%%%%%%%%%%%%%%%%%%%%%%%%%%%%%%%%%%%%%%%%%%%%%%%%%%%
%%%%%%%%%%%%%%%%%%%%%%%%%%%%%%%%%%%%%%%%%%%%%%%%%%%%%%%%%%%%%%%%%%%%%%%%%%%%%%%%%%%%%%%%%%%%%%%%%%%%%%%%%%%%%%%%%%%%%%%%%%%
%%%%%%%%%%%%%%%%%%%%%%%%%%%%%%%%%%%%%%%%%%%%%%%%%%%%%%%%%%%%%%%%%%%%%%%%%%%%%%%%%%%%%%%%%%%%%%%%%%%%%%%%%%%%%%%%%%%%%%%%%%%
\begin{pageParcoursu}

\begin{ExoCu}{Représenter. Chercher.}{1234}{2}{0}{0}{0}{0}
 
\end{ExoCu}

\begin{ExoCu}{Représenter. Chercher.}{1234}{2}{0}{0}{0}{0}
 
\end{ExoCu}

\begin{ExoCu}{Représenter. Chercher.}{1234}{2}{0}{0}{0}{0}
 
\end{ExoCu}


\end{pageParcoursu}
%%%%%%%%%%%%%%%%%%%%%%%%%%%%%%%%%%%%%%%%%%%%%%%%%%%%%%%%%%%%%%%%%%%%%%%%%%%%%%%%%%%%%%%%%%%%%%%%%%%%%%%%%%%%%%%%%%%%%%%%%%%
%%%%%%%%%%%%%%%%%%%%%%%%%%%%%%%%%%%%%%%%%%%%%%%%%%%%%%%%%%%%%%%%%%%%%%%%%%%%%%%%%%%%%%%%%%%%%%%%%%%%%%%%%%%%%%%%%%%%%%%%%%%
%%%%%%%%%%%%%%%              pageParcoursd                    %%%%%%%%%%%%%%%%%%%%%%%%%%%%%%%%%%%%%%%%%%%%%%%%%%%%%%%%%%%%%
%%%%%%%%%%%%%%%%%%%%%%%%%%%%%%%%%%%%%%%%%%%%%%%%%%%%%%%%%%%%%%%%%%%%%%%%%%%%%%%%%%%%%%%%%%%%%%%%%%%%%%%%%%%%%%%%%%%%%%%%%%%
%%%%%%%%%%%%%%%%%%%%%%%%%%%%%%%%%%%%%%%%%%%%%%%%%%%%%%%%%%%%%%%%%%%%%%%%%%%%%%%%%%%%%%%%%%%%%%%%%%%%%%%%%%%%%%%%%%%%%%%%%%%

\begin{pageParcoursd}

\begin{ExoCd}{Représenter. Chercher.}{1234}{2}{0}{0}{0}{0}
 
\end{ExoCd}

\begin{ExoCd}{Représenter. Chercher.}{1234}{2}{0}{0}{0}{0}
 
\end{ExoCd}

\begin{ExoCd}{Représenter. Chercher.}{1234}{2}{0}{0}{0}{0}
 
\end{ExoCd}


\end{pageParcoursd}

%%%%%%%%%%%%%%%%%%%%%%%%%%%%%%%%%%%%%%%%%%%%%%%%%%%%%%%%%%%%%%%%%%%%%%%%%%%%%%%%%%%%%%%%%%%%%%%%%%%%%%%%%%%%%%%%%%%%%%%%%%%
%%%%%%%%%%%%%%%%%%%%%%%%%%%%%%%%%%%%%%%%%%%%%%%%%%%%%%%%%%%%%%%%%%%%%%%%%%%%%%%%%%%%%%%%%%%%%%%%%%%%%%%%%%%%%%%%%%%%%%%%%%%
%%%%%%%%%%%%%%%            pageParcourst                      %%%%%%%%%%%%%%%%%%%%%%%%%%%%%%%%%%%%%%%%%%%%%%%%%%%%%%%%%%%%%
%%%%%%%%%%%%%%%%%%%%%%%%%%%%%%%%%%%%%%%%%%%%%%%%%%%%%%%%%%%%%%%%%%%%%%%%%%%%%%%%%%%%%%%%%%%%%%%%%%%%%%%%%%%%%%%%%%%%%%%%%%%
%%%%%%%%%%%%%%%%%%%%%%%%%%%%%%%%%%%%%%%%%%%%%%%%%%%%%%%%%%%%%%%%%%%%%%%%%%%%%%%%%%%%%%%%%%%%%%%%%%%%%%%%%%%%%%%%%%%%%%%%%%%

\begin{pageParcourst}

\begin{ExoCt}{Représenter. Chercher.}{1234}{2}{0}{0}{0}{0}
 
\end{ExoCt}

\begin{ExoCt}{Représenter. Chercher.}{1234}{2}{0}{0}{0}{0}
 
\end{ExoCt}

\begin{ExoCt}{Représenter. Chercher.}{1234}{2}{0}{0}{0}{0}
 
\end{ExoCt}


\end{pageParcourst}
%%%%%%%%%%%%%%%%%%%%%%%%%%%%%%%%%%%%%%%%%%%%%%%%%%%%%%%%%%%%%%%%%%%%%%%%%%%%%%%%%%%%%%%%%%%%%%%%%%%%%%%%%%%%%%%%%%%%%%%%%%%
%%%%%%%%%%%%%%%%%%%%%%%%%%%%%%%%%%%%%%%%%%%%%%%%%%%%%%%%%%%%%%%%%%%%%%%%%%%%%%%%%%%%%%%%%%%%%%%%%%%%%%%%%%%%%%%%%%%%%%%%%%%
%%%%%%%%%%%%%%%              pageAuto                         %%%%%%%%%%%%%%%%%%%%%%%%%%%%%%%%%%%%%%%%%%%%%%%%%%%%%%%%%%%%%
%%%%%%%%%%%%%%%%%%%%%%%%%%%%%%%%%%%%%%%%%%%%%%%%%%%%%%%%%%%%%%%%%%%%%%%%%%%%%%%%%%%%%%%%%%%%%%%%%%%%%%%%%%%%%%%%%%%%%%%%%%%
%%%%%%%%%%%%%%%%%%%%%%%%%%%%%%%%%%%%%%%%%%%%%%%%%%%%%%%%%%%%%%%%%%%%%%%%%%%%%%%%%%%%%%%%%%%%%%%%%%%%%%%%%%%%%%%%%%%%%%%%%%%
\begin{pageAuto}

\begin{ExoAuto}{Représenter. Chercher.}{1234}{2}{0}{0}{0}{0}
 
\end{ExoAuto}

\begin{ExoAuto}{Représenter. Chercher.}{1234}{2}{0}{0}{0}{0}
 
\end{ExoAuto}

\begin{ExoAuto}{Représenter. Chercher.}{1234}{2}{0}{0}{0}{0}
 
\end{ExoAuto}


\end{pageAuto} %Ensemble et intervalles
%
%\chapter{Fractions, puissances, racines carrés}
%\begin{titreTice}[Calculs numériques]

\Titre{Les fractions}{1}
\end{titreTice}


\begin{CpsCol}
\begin{description}
\item[$\square$] Manipuler les fractions
\end{description}
\end{CpsCol}


\begin{ThT}{Les racines carrées\index{Racines carrées}}
Pour tous nombres $a$ et $c$, et $b$ non nul, 
\begin{description}
\item $\frac{a}{b}=\frac{k \times a}{k \times b} , k \neq 0$
\item $\frac{a}{b} + \frac{c}{b}=\frac{a+c}{b}$
\end{description}
\end{ThT}


\begin{minipage}{0.5\linewidth}

\EPCN{Calculer }

Calculer et simplifier sans calculatrice 

\begin{enumerate}
\item $A = \frac{2}{3} + \frac{7}{15}$
\item $B = \frac{13}{30} - \frac{7}{15} + \frac{5}{3}$
\item $C = \frac{-2}{9} - \frac{-8}{21}$
\item $D = \frac{2}{11} + 2$
\end{enumerate}

\end{minipage}
\begin{minipage}{0.5\linewidth}

\EPCNM{Calculer }

Calculer et simplifier sans calculatrice

\begin{enumerate}
\item $A = \frac{7}{12} - \frac{5}{6} \times \frac{1}{2}$
\item $B = \frac{7}{4} \div 2 - \frac{1}{6} \times \frac{2}{-3}$
\item $C = \frac{7}{12} \div \left( \frac{5}{6} + \frac{1}{2} \right) $
\item $D = \left( \frac{3}{5}\right)^2 -\frac{-5}{6} $
\end{enumerate}

\end{minipage}

\EPCN{Calculer.}

Calculer et simplifier sans calculatrice

\begin{enumerate}
\item $A = \frac{2}{3} \left( \frac{5}{6} + \frac{2}{7} \right)$
\item $B = \frac{2}{3} \times - \frac{1}{6} \times \frac{2}{-3}$
\item $C = \frac{7}{12} \div \left( \frac{5}{6} + \frac{1}{2} \right) $
\item $D = \left( \frac{3}{5}\right)^2 -\frac{-5}{6} $
\end{enumerate}


\EPCN{Calculer.}

Montrer que pour tout entier naturel $n \neq 0$, $\frac{1}{n+1}-\frac{1}{n}=\frac{-1}{n(n+1)}$


\EPCN{Calculer.}

Soit $a, b, c, d$ quatre nombres non nuls.

Montrer que $\left( \frac{a}{b} + \frac{c}{d} \right) \times \frac{bd}{ad+bc} = 1 $

 % Fractions


%\part{Les fonctions}
%%\chapter{Variations de fonctions}
%%%%\intro{fusee}{Envoyer une fusée sur la lune n'est pas une mince affaire et le cout financier est excessivement important. Les calculs, la modélisation, le raisonnement sur des \textbf{expressions littérales} avant de lancer les satellites en orbite sont des phases de recherche obligatoires. C'est lorsque que tous les calculs littéraux sont validés que la fusée est acheminée sur son pas de tir. }
%%##########\begin{titre}[Fonctions et expressions algébriques]

\Titre{Expressions et problèmes}{4}
\end{titre}


\begin{CpsCol}
\textbf{Utiliser des nombres pour calculer et résoudre des problèmes}
\begin{description}
\item[$\square$] Associer à un problème une expression algébrique
\item[$\square$] Identifier une forme adéquate
\item[$\square$] Traduire le lien entre deux quantités
\end{description}
\end{CpsCol}



\Exo{1}{FEA-31}

\Exo{1}{FEA-33}

\mini{
\Exo{1}{FEA-34}
}{
\Exo{1}{FEA-35}
}
\mini{
\Exo{1}{FEA-36}

\Exo{1}{FEA-38}

\Exo{1}{FEA-41}

\Exo{1}{FEA-43}
}{
\Exo{1}{FEA-37}

\PO{1}{FEA-42}

\PO{1}{FEA-44}

\CR{1}{FEA-32}
}


%% \Exo{1}{FEA-69}

\EPCNA{Calculer. Raisonner.}

Démontrer chacune des égalités suivantes :
\begin{description}
\item[•] Pour $x \neq -1$ et $x \neq 0$, $\frac{1}{x}-\frac{1}{x+1}=\frac{1}{x(x+1)}$
\item[•] Pour $x \geq 0$ et $x \neq -1$, $\frac{1}{\sqrt{x}-1}-\frac{1}{\sqrt{x}+1}=\frac{3}{x-1}$
\end{description}


\EPCNA{Calculer. Raisonner.}

Soit $f(x) =\frac{1}{x-1} -\frac{1}{x+1}$, avec $x$ un nombre réel différent de $-1$ et $1$.

\begin{enumerate}
\item Déterminer le domaine de définition de la fonction $f$.
\item Démontrer que $f(x)=\frac{2}{x^2-1}$
\item Choisir la forme de $f(x)$ la plus adaptée pour calculer $f(3)$.
\item En déduire l'image de 3 par la fonction $f$.
\end{enumerate}
 % Exression et problèmes
%%##########\input{CHAPITRES/FEA-histoire2}
%%##########\chapter{Situations du premier degré}
%
%\chapter{Fonctions et expressions algébriques}
%\impress{\impressionEleve}{\begin{titre}[Fonctions et expressions algébriques]

\Titre{Notion de fonction}{4}
\end{titre}


\begin{CpsCol}
\begin{description}
\item[$\square$] Relier représentation graphique et tableau de variations.
\end{description}
\end{CpsCol}



\begin{DefT}{Fonction} \index{Fonction!Antécédent}\index{Fonction!}\index{Fonction!Ensemble de définition}

\begin{minipage}{0.48\linewidth}
Définir une fonction $f$ d'un ensemble D de réels dans $\R$, c'est associer à chaque réel $x$ de D un
unique réel noté $f(x)$.
\begin{description}
\item On dit que D est l'\textbf{ensemble de définition} de $f$. 
\item $f(x)$ est l'\textbf{image} de $x$ par $f$.
\item $x$ est un \textbf{antécédent} de $f(x)$ par $f$.
\end{description}
\end{minipage}
\begin{minipage}{0.48\linewidth}
\begin{center}
\definecolor{sqsqsq}{rgb}{0.12549019607843137,0.12549019607843137,0.12549019607843137}
\definecolor{ffqqqq}{rgb}{1.,0.,0.}
\begin{tikzpicture}[line cap=round,line join=round,>=triangle 45,x=1.0cm,y=1.0cm]
\clip(1.84,4.9) rectangle (9.52,7.);
\draw [color=ffqqqq] (4.,6.62)-- (7.,6.62);
\draw [color=ffqqqq] (7.,6.62)-- (7.,5.32);
\draw [color=ffqqqq] (7.,5.32)-- (4.,5.32);
\draw [color=ffqqqq] (4.,5.32)-- (4.,6.62);
\draw [->] (2.62,6.) -- (4.,6.);
\draw [->] (7.,5.98) -- (8.3,6.);
\draw [color=sqsqsq](2.44,6.25) node[anchor=north west] {|};
\draw [color=sqsqsq](4.58,6.2) node[anchor=north west] {fonction};
\draw [color=sqsqsq](6.34,6.2) node[anchor=north west] {$f$};
\draw [color=sqsqsq](2.06,6.2) node[anchor=north west] {$x$};
\draw [color=sqsqsq](8.36,6.2) node[anchor=north west] {$f(x)$};
\end{tikzpicture}
\end{center}
\end{minipage}
\end{DefT}
 
\begin{Nt}
$f : D \longrightarrow \R$

$x \mapsto f(x)$

Ce qui se lit : la fonction $f$ qui à $x$ associe $f(x)$.
\end{Nt}

\begin{Mt}[Déterminer algébriquement une image, un antécédent par $f$]
Soit $f$ la fonction suivante :

$f$ : $[-4 ; 5] \longrightarrow \R$

$x \mapsto 2x^2 - 6x + 3$.

\begin{description}
\item Pour déterminer l'image d'un nombre $x$ par $f$, il faut que ce nombre soit dans l'ensemble de définition de $f$. Dans ce cas, on remplace $x$ par ce nombre dans l'expression de $f(x)$.

Image de -2 : $f(-2) = 2 \times (-2)^2 -6 \times (-2) + 3 = 2 \times 4 + 12 + 3 = 8 + 12 + 3 = 23$

Image de 6 : Impossible car 6 n'appartient pas à $[-4 ; 5]$.

\item  Pour déterminer le (ou les) antécédent(s) d'un nombre $a$ par $f$, il faut et il suffit de résoudre l'équation $f(x) = 3$.

$2x^2 - 6x + 3 = 3$

$2x^2 - 6x = 0$

$2x (x - 3) = 0$

$2x = 0$ ou $x - 3 = 0$ , $x = 0$ ou  $x = 3$. $\mathscr{S}=\left\lbrace 0;3\right\rbrace $. Les antécédents de 3 par $f$ sont 0 et 3.
\end{description}
\end{Mt}




\EPC{1}{FEA-60}{Calculer. Communiquer}

\mini{
\EPC{0}{FEA-60bis}{Calculer. Communiquer}
}{
\EPC{1}{FEA-46}{Chercher. Calculer.}
}


\EPC{1}{FEA-107}{Calculer. Communiquer}

\mini{
\EPC{1}{FEA-48}{Raisonner. Communiquer}
}{

\EPC{1}{FEA-50}{Modéliser. Calculer}

\EPC{1}{FEA-61}{Raisonner. Communiquer}

\EPC{0}{FEA-45}{Raisonner. Calculer}
}


 
\EPC{0}{FEA-53}{Calculer}
 

\begin{Rqs}
\begin{description}
\item[•] Une fonction peut être donnée, sur un ensemble de définition $D$, par une \textbf{formule algébrique}, un \textbf{tableau de valeurs}, une \textbf{courbe}. 
\item[•] Le seul mode de définition qui permet le calcul d'images et d'antécédents est la formule algébrique. La courbe est imité par la précision et le tableau de valeur par le nombre de valeurs proposées.
\end{description}
\end{Rqs}


\EPCP{1}{FEA-109}{Calculer}}% Ensemble de définition et variables, Image et antécédent
%%##########\begin{titre}[Fonctions de référence]

\Titre{Fonctions de référence}{2}
\end{titre}


\subsection*{Matériel et mise en œuvre}

Pour chaque fonction de référence (carré, inverse, racine carrée, cube, affine : définitions et courbes représentatives) :

\begin{description}
 \item[•] Tableau de signes
  \item[•] Tableau de valeurs
 \item[•] Tableau de variations   
  \item[•] Courbe
  \item[•] Définition
 \end{description} 

Les élèves doivent reformer les fonctions de référence.  


\subsection*{Matériel}
 
 
\subsubsection*{Les définitions}


\begin{DefN}
On appelle \emph{fonction Inverse} la fonction définie sur $\mathbb{R}^*=]-\infty;0[\cup]0;+\infty[$ par $f : x\mapsto \dfrac{1}{x}$ ou $f(x)=\dfrac{1}{x}$.
\end{DefN}

\vspace{0.4cm}

\begin{DefN}
On appelle \emph{fonction Carré} la fonction définie sur $\mathbb{R}=]-\infty;+\infty[$ par $f : x\mapsto  x^2$ ou $f(x)=x^2$. 
\end{DefN}

\vspace{0.4cm} 

\begin{DefN}
On appelle \emph{fonction Cube} la fonction définie sur $\mathbb{R}=]-\infty;+\infty[$ par $f : x\mapsto  x^3$ ou $f(x)=x^3$. 
\end{DefN}

\vspace{0.4cm} 

\begin{DefN}
On appelle \emph{fonction Racine Carrée} la fonction définie sur $\mathbb{R^+}=[0;+\infty[$ par $f : x\mapsto  \sqrt{x}$ ou $f(x)=\sqrt{x}$. 
\end{DefN}

\vspace{0.4cm} 

\begin{DefN}
On appelle \emph{fonction affine} la fonction définie sur $\mathbb{R}=]-\infty;+\infty[$ par $f(x)=  ax+b$. 
\end{DefN} 

 \newpage
 
\subsubsection*{Les tableaux de signes} 

 \begin{tikzpicture}
   \tkzTabInit{$x$ / 1 , $f(x)$ / 1}{$-\infty$, $x_0$, $+\infty$}
   \tkzTabLine{,$-$, z, $+$, }
\end{tikzpicture}


\vspace{0.4cm}

\begin{tikzpicture}
   \tkzTabInit{$x$ / 1 , $f(x)$ / 1}{$-\infty$, $0$, $+\infty$}
   \tkzTabLine{,$+$, z, $+$, }
\end{tikzpicture}

\vspace{0.4cm}

\begin{tikzpicture}
   \tkzTabInit{$x$ / 1 , $f(x)$ / 1}{$-\infty$, $0$, $+\infty$}
   \tkzTabLine{,$-$, z, $+$, }
\end{tikzpicture}
\vspace{0.4cm}

\begin{tikzpicture}
   \tkzTabInit{$x$ / 1 , $f(x)$ / 1}{$-\infty$, $-2$, $+\infty$}
   \tkzTabLine{,$-$, z, $+$, }
\end{tikzpicture}

\vspace{0.4cm}

\begin{tikzpicture}
   \tkzTabInit{$x$ / 1 , $f(x)$ / 1}{$-\infty$, $0$, $+\infty$}
   \tkzTabLine{,$-$, d, $+$, }
\end{tikzpicture}

\vspace{0.4cm}

\begin{tikzpicture}
   \tkzTabInit{$x$ / 1 , $f(x)$ / 1}{$-\infty$, $0$, $+\infty$}
   \tkzTabLine{,$+$, d, $+$, }
\end{tikzpicture}

\vspace{0.4cm}

\begin{tikzpicture}
   \tkzTabInit{$x$ / 1 , $f(x)$ / 1}{$-\infty$, $0$, $+\infty$}
   \tkzTabLine{,$-$, z, $+$, }
\end{tikzpicture}

\vspace{0.4cm}

\begin{tikzpicture}
   \tkzTabInit{$x$ / 1 , $f(x)$ / 1}{  $0$, $+\infty$}
   \tkzTabLine{$0$,   $+$, }
\end{tikzpicture}

\vspace{0.4cm}

\begin{tikzpicture}
   \tkzTabInit{$x$ / 1 , $f(x)$ / 1}{  $0$, $+\infty$}
   \tkzTabLine{$0$,   $-$, }
\end{tikzpicture}
 \newpage
 
\subsubsection*{Les tableaux de valeurs} 

\begin{tabular}{|c|c|c|c|c|c|c|c|}
\hline 
$x$ & 0 & $\frac{1}{4}$ & 1 & 2,25 & 4 & 9 \\ 
\hline 
$f(x)$ & 0 & 0,5  & 1 & 1,5  & 2 & 3\\ 
\hline 
\end{tabular} 

\vspace{0.4cm}

\begin{tabular}{|c|c|c|c|c|c|c|c|c|c|c|c|c|c|c|c|}
\hline 
$x$ & $-3$ & $-\frac{5}{2}$ & $-2$ & $-\frac{3}{2}$ & $-1$ & $-0,5$ &  0 & $\frac{1}{2}$ & 1 & $\frac{3}{2}$ & 2 & 2,5 & 3 \\ 
\hline 
$f(x)$ & 9 & 6,25 & 4 & 2,25 & 1 & $\frac{1}{4}$ & 0 & 0,25 & 1 & 2,25 & 4 & 6,25 & 9 \\ 
\hline 
\end{tabular} 

\vspace{0.4cm}

\begin{tabular}{|c|c|c|c|c|c|c|c|c|c|c|c|c|c|c|c|}
\hline 
$x$ & $-3$ & $-\frac{5}{2}$ & $-2$ & $-\frac{3}{2}$ & $-1$ & $-0,5$ &  0 & $\frac{1}{2}$ & 1 & $\frac{3}{2}$ & 2 & 2,5 & 3 \\ 
\hline 
$f(x)$ & $-27$ & $-15,625$ & $-8$ & $-3,375$ & 1 & $\frac{1}{8}$ & 0 & 0,125 & 1 & 3,375 & 8 & 15,625  & 27 \\ 
\hline 
\end{tabular} 

\vspace{0.4cm}

\begin{tabular}{|c|c|c|c|c|c|c|c|c|c|c|c|c|c|c|c|}
\hline 
$x$ & $-5$ & $-4$ & $-2$ &   $-1$ & $-0,5$ &  0 & $\frac{1}{2}$ & 1 & 2 &   4 & 5 \\ 
\hline 
$f(x)$ & $-\frac{1}{5}$ & $-0,25$ &   $-0,5$ & $-1$ & $-0,2$ & 0 & 0,2  & 1 & 0,5 &  0,25  & 0,2 \\ 
\hline 
\end{tabular}

\vspace{0.4cm}

\begin{tabular}{|c|c|c|c|c|c|c|c|c|c|c|c|c|c|c|c|}
\hline 
$x$ & $-5$ & $-4$ & $-2$ &   $-1$ & $-0,5$ &  0 & $\frac{1}{2}$ & 1 & 2 &   4 & 5 \\ 
\hline 
$f(x)$ & $-\frac{1}{5}$ & $-0,25$ &   $-0,5$ & $-1$ & $-2$ & 0 & 2 & 1 & 0,5 &  0,25  & 0,2 \\ 
\hline 
\end{tabular}

\vspace{0.4cm} 

\begin{tabular}{|c|c|c|c|c|c|c|c|c|c|c|c|c|c|c|c|}
\hline 
$x$ & $-10$ &   $-5$ &  0 & 5 & 10 \\ 
\hline 
$f(x)$ & $-1$ &   $0$ &  1 & 2 & 3  \\ 
\hline 
\end{tabular}


\subsubsection*{Propriétés des variations} 

\begin{ThN}
La fonction \emph{Cube} est strictement croissante sur $\R$. 
\end{ThN}

\vspace{0.4cm} 

\begin{ThN}
La fonction \emph{Inverse} est décroissante sur $]-\infty;0[$ et  décroissante sur $]0;+\infty[$. 
\end{ThN}

\vspace{0.4cm}

\begin{ThN}
La fonction \emph{Carré} est strictement décroissante sur $\R^-$ et strictement croissante sur $\R^+$. 
\end{ThN}

\vspace{0.4cm} 

\begin{ThN}
La fonction \emph{Racine Carrée} est strictement croissante sur $\R+$. 
\end{ThN}

\vspace{0.4cm} 
 
 \begin{ThN}
La fonction \emph{affine}, $f:x \mapsto ax +b$ définie sur $\R$, est strictement croissante sur $\R$ lorsque $a>0$, strictement décroissante sur $\R$ lorsque $a<0$. 
\end{ThN}

 \newpage
 
\subsubsection*{Les tableaux de variations} 

\begin{tikzpicture}
   \tkzTabInit{$x$ / 1 , $f(x)$ / 2}{$-\infty$,   $+\infty$}
   \tkzTabVar{+/ $+\infty$, -/ $-\infty$}
\end{tikzpicture}



\vspace{0.4cm}

\begin{tikzpicture}
   \tkzTabInit{$x$ / 1 , $f(x)$ / 2}{$-\infty$,   $+\infty$}
   \tkzTabVar{-/ $+\infty$, +/ $-\infty$}
\end{tikzpicture}



\vspace{0.4cm}

\begin{tikzpicture}
   \tkzTabInit{$x$ / 1 , $f(x)$ / 2}{$-\infty$,   $+\infty$}
   \tkzTabVar{-/ $-\infty$, +/ $+\infty$}
\end{tikzpicture}

\vspace{0.4cm}

\begin{tikzpicture}
   \tkzTabInit{$x$ / 1 , $f(x)$ / 2}{$-\infty$, 0 ,  $+\infty$}
   \tkzTabVar{-/ $+\infty$,  +/0 ,  -/ $+\infty$}
\end{tikzpicture}

\vspace{0.4cm}

\begin{tikzpicture}
   \tkzTabInit{$x$ / 1 , $f(x)$ / 2}{$-\infty$, 0 ,  $+\infty$}
   \tkzTabVar{+/ $+\infty$,  -/0 ,  +/ $+\infty$}
\end{tikzpicture}

\vspace{0.4cm}

\begin{tikzpicture}
\tkzTabInit[lgt=1,espcl=2]{ $x$ / 1,$f $ / 2}
{ $-\infty$ , $0$ ,$+\infty$}
\tkzTabVar{+/$0$ , -D+ /$ $/$ $ , -/$0$}
\end{tikzpicture}
\vspace{0.4cm}

\begin{tikzpicture}
   \tkzTabInit{$x$ / 1 , $f(x)$ / 2}{$-\infty$,   $+\infty$}
   \tkzTabVar{-/ $-\infty$, +/ $+\infty$}
\end{tikzpicture}

\vspace{0.4cm}

\begin{tikzpicture}
   \tkzTabInit{$x$ / 1 , $f(x)$ / 2}{0,   $+\infty$}
   \tkzTabVar{-/ $-\infty$, +/ $+\infty$}
\end{tikzpicture}

  

 
 
 
 
 
 
\subsubsection*{Les courbes}

\begin{tikzpicture}[line cap=round,line join=round,>=triangle 45,x=1.0cm,y=1.0cm]
\begin{axis}[
x=1.0cm,y=1.0cm,
axis lines=middle,
ymajorgrids=true,
xmajorgrids=true,
xmin=-7.66,
xmax=4.12,
ymin=-2.38,
ymax=2.7,
xtick={-7.0,-6.0,...,4.0},
ytick={-2.0,-1.0,...,2.0},]
\clip(-7.66,-2.38) rectangle (4.12,2.7);
\draw [line width=2.pt,domain=-7.66:4.12] plot(\x,{(--1.--0.2*\x)/1.});
\end{axis}
\end{tikzpicture}



\begin{tikzpicture}[line cap=round,line join=round,>=triangle 45,x=1.0cm,y=1.0cm]
\begin{axis}[
x=1.0cm,y=1.0cm,
axis lines=middle,
ymajorgrids=true,
xmajorgrids=true,
xmin=-2.98,
xmax=3.54,
ymin=-0.8999999999999996,
ymax=6.659999999999997,
xtick={-2.0,-1.0,...,3.0},
ytick={-0.0,1.0,...,6.0},]
\clip(-2.98,-0.9) rectangle (3.54,6.66);
\draw [samples=50,rotate around={0.:(0.,0.)},xshift=0.cm,yshift=0.cm,line width=2.pt,domain=-4.0:4.0)] plot (\x,{(\x)^2/2/0.5});
\end{axis}
\end{tikzpicture}

 
\begin{tikzpicture}[line cap=round,line join=round,>=triangle 45,x=1.0cm,y=1.0cm]
\begin{axis}[
x=1.0cm,y=1.0cm,
axis lines=middle,
ymajorgrids=true,
xmajorgrids=true,
xmin=-0.640000000000001,
xmax=9.520000000000007,
ymin=-0.6600000000000019,
ymax=4.239999999999996,
xtick={-0.0,1.0,...,9.0},
ytick={-0.0,1.0,...,4.0},]
\clip(-0.64,-0.66) rectangle (9.52,4.24);
\draw[line width=2.pt,color=black,smooth,samples=100,domain=4.079999953157344E-8:9.520000000000007] plot(\x,{sqrt((\x))});
\end{axis}
\end{tikzpicture}


 
\begin{center}\begin{tikzpicture}[x=.8cm,y=.8cm,>=stealth]
\draw[line width=.1pt] (-5,-5) grid[step=1] (5,5);
\draw[line width=1pt] (0,0) node[below left, fill=white]{$O$} (1,-2pt) node[below, fill=white]{$1$} --(1,2pt) (-2pt,1) node[left, fill=white]{$1$} --(2pt,1);
\draw[color=red] (2.5,6.25) node[left,fill=white]{$\mathcal{C}_f$};
\draw[line width=1pt,->] (-5,0)--(5,0);
\draw[line width=1pt,->] (0,-5)--(0,5);
\draw[line width=1pt,color=red] plot[smooth, samples=100, domain=-5:-0.19] (\x,{1/\x});
\draw[line width=1pt,color=red] plot[smooth, samples=100, domain=0.19:5] (\x,{1/\x});
\end{tikzpicture}\end{center}


 
\begin{tikzpicture}[line cap=round,line join=round,>=triangle 45,x=1.0cm,y=0.38800705467372043cm]
\begin{axis}[
x=1.0cm,y=0.38800705467372043cm,
axis lines=middle,
ymajorgrids=true,
xmajorgrids=true,
xmin=-2.896551724137932,
xmax=2.8965517241379306,
ymin=-10.09090909090913,
ymax=15.681818181818203,
xtick={-2.0,-1.0,...,2.0},
ytick={-10.0,-5.0,...,15.0},]
\clip(-2.896551724137932,-10.09090909090913) rectangle (2.8965517241379306,15.681818181818203);
\draw[line width=2.pt,color=black,smooth,samples=100,domain=-2.896551724137932:2.8965517241379306] plot(\x,{(\x)^(3.0)});
 
\end{axis}
\end{tikzpicture}



% La fonction Carré + rep. graphique
%\impress{\impressionEleve}{\begin{titre}[Fonctions et expressions algébriques]

\Titre{Lecture graphique}{4}
\end{titre}


\begin{CpsCol}
\begin{description}
\item[$\square$] Exploiter l’équation $y = f(x)$ d’une courbe : appartenance, calcul de coordonnées.
\item[$\square$] Résoudre une équation ou une inéquation du type $f(x) = k$ ou $f(x) < k$.
\item[$\square$] Résoudre, graphiquement, une équation ou inéquation du type $f(x) = g(x)$ ou $f(x) < g(x)$.
\end{description}
\end{CpsCol}


\Rec{1}{FEA-52}

\begin{DefT}{Représentation graphique}\index{Représentation graphique}\index{Équation de courbe}
Soit $f$ une fonction définie sur l'ensemble $D$.\\
Le plan est muni d'un repère (O ; I ; J).\\
La \textbf{représentation graphique} ou courbe représentative $\mathscr{C}_f$ de la fonction $f$ dans le repère (O ; I ; J) est l'ensemble des points de coordonnées $(x ; f (x))$, où $x \in D$.
Une équation de $\mathscr{C}_f$ est $y=f(x)$.
\end{DefT}

\begin{Pp}[Appartenance d'un point à une courbe]
Un point $A(x_A;y_A)$ appartient à une courbe $\mathscr{C}$ d'équation $y=f(x)$ si et seulement si les coordonnées de $A$ vérifient l'équation de la courbe $\mathscr{C}$. On symbolise par : $ A \in \mathscr{C}  \Longleftrightarrow y_A=f(x_A)$.
\end{Pp}

\begin{Rqs}
\begin{itemize}
\item En général, le repère sera orthogonal ou orthonormal.
\item Le tracé d’une courbe représentative est toujours approximatif : on construit un tableau de valeurs, on place les points correspondants dans un repère et on les relie par une courbe régulière
(sans utiliser la règle, sauf dans certains cas particuliers).
\item On peut utiliser la calculatrice pour remplir un tableau de valeurs et tracer des courbes représentatives. 
\item Certaines fonctions ne sont connues que par leur courbe représentative
\end{itemize}
\end{Rqs}

\begin{Mt}[. Construction de courbe]\index{Construction de courbe}

Soit la fonction $f$ définie sur $\R$ par $f(x)=\frac{4x+3}{x^2+1}$.

$x^2+1 > 0$ donc pour toutes les valeurs de $x$, $f(x)$ existe. 

Donc le domaine de définition est $\R$.

On commence par construire, à l'aide de la calculatrice, un tableau de valeurs. Pour placer des points dans un
repère, des valeurs approchées suffisent.

On place les points correspondants dans le repère choisi, et on les joints par une courbe régulière (voir
figure ci dessous).

\begin{center}
\definecolor{ffqqqq}{rgb}{1.,0.,0.}
\definecolor{qqqqff}{rgb}{0.,0.,1.}
\definecolor{cqcqcq}{rgb}{0.7529411764705882,0.7529411764705882,0.7529411764705882}
\begin{tikzpicture}[line cap=round,line join=round,>=triangle 45,x=1.0cm,y=1.0cm]
\draw [color=cqcqcq,, xstep=0.5cm,ystep=0.5cm] (-4.486899146925993,-1.2029939912205303) grid (7.341764903922375,4.241161011261972);
\draw[->,color=black] (-4.486899146925993,0.) -- (7.341764903922375,0.);
\foreach \x in {-4.,-3.5,-3.,-2.5,-2.,-1.5,-1.,-0.5,0.5,1.,1.5,2.,2.5,3.,3.5,4.,4.5,5.,5.5,6.,6.5,7.}
\draw[shift={(\x,0)},color=black] (0pt,2pt) -- (0pt,-2pt) node[below] {\footnotesize $\x$};
\draw[->,color=black] (0.,-1.2029939912205303) -- (0.,4.241161011261972);
\foreach \y in {-1.,-0.5,0.5,1.,1.5,2.,2.5,3.,3.5,4.}
\draw[shift={(0,\y)},color=black] (2pt,0pt) -- (-2pt,0pt) node[left] {\footnotesize $\y$};
\draw[color=black] (0pt,-10pt) node[right] {\footnotesize $0$};
\clip(-4.486899146925993,-1.2029939912205303) rectangle (7.341764903922375,4.241161011261972);
\draw[line width=1.2pt,color=qqqqff,smooth,samples=100,domain=-4.486899146925993:7.341764903922375] plot(\x,{(4.0*(\x)+3.0)/((\x)^(2.0)+1.0)});
\begin{scriptsize}
\draw[color=qqqqff] (-6.298317993206538,-0.6090861727678938) node {$g$};
\draw [color=ffqqqq] (-1.,-0.5)-- ++(-2.5pt,0 pt) -- ++(5.0pt,0 pt) ++(-2.5pt,-2.5pt) -- ++(0 pt,5.0pt);
\draw [color=ffqqqq] (0.,3.)-- ++(-2.5pt,0 pt) -- ++(5.0pt,0 pt) ++(-2.5pt,-2.5pt) -- ++(0 pt,5.0pt);
\draw [color=ffqqqq] (0.98,3.5298918588043255)-- ++(-2.5pt,0 pt) -- ++(5.0pt,0 pt) ++(-2.5pt,-2.5pt) -- ++(0 pt,5.0pt);
\draw [color=ffqqqq] (1.98,2.2193317616453947)-- ++(-2.5pt,0 pt) -- ++(5.0pt,0 pt) ++(-2.5pt,-2.5pt) -- ++(0 pt,5.0pt);
\draw [color=ffqqqq] (3.,1.5)-- ++(-2.5pt,0 pt) -- ++(5.0pt,0 pt) ++(-2.5pt,-2.5pt) -- ++(0 pt,5.0pt);
\draw [color=ffqqqq] (4.,1.1176470588235294)-- ++(-2.5pt,0 pt) -- ++(5.0pt,0 pt) ++(-2.5pt,-2.5pt) -- ++(0 pt,5.0pt);
\draw [color=ffqqqq] (5.04,0.8772195624507605)-- ++(-2.5pt,0 pt) -- ++(5.0pt,0 pt) ++(-2.5pt,-2.5pt) -- ++(0 pt,5.0pt);
\draw [color=ffqqqq] (6.,0.7297297297297297)-- ++(-2.5pt,0 pt) -- ++(5.0pt,0 pt) ++(-2.5pt,-2.5pt) -- ++(0 pt,5.0pt);
\draw [color=ffqqqq] (7.,0.62)-- ++(-2.5pt,0 pt) -- ++(5.0pt,0 pt) ++(-2.5pt,-2.5pt) -- ++(0 pt,5.0pt);
\draw [color=ffqqqq] (-2.,-1.)-- ++(-2.5pt,0 pt) -- ++(5.0pt,0 pt) ++(-2.5pt,-2.5pt) -- ++(0 pt,5.0pt);
\draw [color=ffqqqq] (-2.98,-0.90279745759281)-- ++(-2.5pt,0 pt) -- ++(5.0pt,0 pt) ++(-2.5pt,-2.5pt) -- ++(0 pt,5.0pt);
\draw [color=ffqqqq] (-4.1,-0.7523862998315552)-- ++(-2.5pt,0 pt) -- ++(5.0pt,0 pt) ++(-2.5pt,-2.5pt) -- ++(0 pt,5.0pt);
\draw [color=ffqqqq] (0.5,4.)-- ++(-2.5pt,0 pt) -- ++(5.0pt,0 pt) ++(-2.5pt,-2.5pt) -- ++(0 pt,5.0pt);
\draw [color=ffqqqq] (-0.5176152269341983,0.7331180540879474)-- ++(-2.5pt,0 pt) -- ++(5.0pt,0 pt) ++(-2.5pt,-2.5pt) -- ++(0 pt,5.0pt);
\draw [color=ffqqqq] (1.50167135580477,2.7670165188890095)-- ++(-2.5pt,0 pt) -- ++(5.0pt,0 pt) ++(-2.5pt,-2.5pt) -- ++(0 pt,5.0pt);
\draw [color=ffqqqq] (-1.5,-0.9230769230769231)-- ++(-2.5pt,0 pt) -- ++(5.0pt,0 pt) ++(-2.5pt,-2.5pt) -- ++(0 pt,5.0pt);
\end{scriptsize}
\end{tikzpicture}
\end{center}
\end{Mt}

\mini{
\EPC{1}{FEA-66}{Représenter. Calculer.}

\EPC{1}{FEA-54}{Représenter. Calculer.}
}{

\EPC{1}{FEA-59}{Représenter. Chercher.}

\EPC{1}{FEA-55}{Représenter.}

\EPC{0}{FEA-67}{Calculer.}

\EPC{0}{FEA-62}{Représenter.}

}


\begin{Mt}[. Déterminer graphiquement une image, un antécédent par $f$]\index{Lecture graphique!Image}\index{Lecture graphique!Antécédent}
On se reportera à la figure ci dessous
\begin{itemize}
\item L’image de $a$ est l’ordonnée du point de la courbe d’abscisse $a$.
\item Les antécédents de $b$ sont les abscisses des points de la courbe dont l’ordonnée est $b$.
\end{itemize}

\begin{minipage}{0.48\linewidth}
\textbf{Lecture d'image}

\definecolor{ffqqqq}{rgb}{1.,0.,0.}
\definecolor{qqwuqq}{rgb}{0.,0.39215686274509803,0.}
\definecolor{uuuuuu}{rgb}{0.26666666666666666,0.26666666666666666,0.26666666666666666}
\definecolor{xdxdff}{rgb}{0.49019607843137253,0.49019607843137253,1.}
\definecolor{ccqqqq}{rgb}{0.8,0.,0.}
\begin{tikzpicture}[line cap=round,line join=round,>=triangle 45,x=1.0cm,y=1.0cm]
\draw[->,color=black] (-1.2,0.) -- (6.54,0.);
\foreach \x in {-1.,1.,2.,3.,4.,5.,6.}
\draw[shift={(\x,0)},color=black] (0pt,2pt) -- (0pt,-2pt) node[below] {\footnotesize $\x$};
\draw[->,color=black] (0.,-0.58) -- (0.,4.1);
\foreach \y in {,1.,2.,3.,4.}
\draw[shift={(0,\y)},color=black] (2pt,0pt) -- (-2pt,0pt) node[left] {\footnotesize $\y$};
\draw[color=black] (0pt,-10pt) node[right] {\footnotesize $0$};
\clip(-1.2,-0.58) rectangle (6.54,4.1);
\draw[line width=1.2pt,color=ccqqqq] (0.02000202000000007,1.0333351288922137) -- (0.02000202000000007,1.0333351288922137);
\draw[line width=1.2pt,color=ccqqqq] (0.02000202000000007,1.0333351288922137) -- (0.0350020088527628,1.0468526628865917);
\draw[line width=1.2pt,color=ccqqqq] (0.0350020088527628,1.0468526628865917) -- (0.050001997705525526,1.0607417842661409);
\draw[line width=1.2pt,color=ccqqqq] (0.050001997705525526,1.0607417842661409) -- (0.06500198655828826,1.075007280091787);
\draw[line width=1.2pt,color=ccqqqq] (0.06500198655828826,1.075007280091787) -- (0.08000197541105099,1.0896537782492035);
\draw[line width=1.2pt,color=ccqqqq] (0.08000197541105099,1.0896537782492035) -- (0.09500196426381372,1.1046857321084331);
\draw[line width=1.2pt,color=ccqqqq] (0.09500196426381372,1.1046857321084331) -- (0.11000195311657644,1.120107404667603);
\draw[line width=1.2pt,color=ccqqqq] (0.11000195311657644,1.120107404667603) -- (0.12500194196933917,1.1359228521975857);
\draw[line width=1.2pt,color=ccqqqq] (0.12500194196933917,1.1359228521975857) -- (0.1400019308221019,1.1521359074080868);
\draw[line width=1.2pt,color=ccqqqq] (0.1400019308221019,1.1521359074080868) -- (0.15500191967486465,1.1687501621594936);
\draw[line width=1.2pt,color=ccqqqq] (0.15500191967486465,1.1687501621594936) -- (0.1700019085276274,1.1857689497488706);
\draw[line width=1.2pt,color=ccqqqq] (0.1700019085276274,1.1857689497488706) -- (0.18500189738039013,1.2031953268027384);
\draw[line width=1.2pt,color=ccqqqq] (0.18500189738039013,1.2031953268027384) -- (0.20000188623315288,1.2210320548137101);
\draw[line width=1.2pt,color=ccqqqq] (0.20000188623315288,1.2210320548137101) -- (0.21500187508591562,1.2392815813626485);
\draw[line width=1.2pt,color=ccqqqq] (0.21500187508591562,1.2392815813626485) -- (0.23000186393867836,1.2579460210727733);
\draw[line width=1.2pt,color=ccqqqq] (0.23000186393867836,1.2579460210727733) -- (0.2450018527914411,1.2770271363470083);
\draw[line width=1.2pt,color=ccqqqq] (0.2450018527914411,1.2770271363470083) -- (0.2600018416442038,1.2965263179448447);
\draw[line width=1.2pt,color=ccqqqq] (0.2600018416442038,1.2965263179448447) -- (0.2750018304969665,1.3164445654600376);
\draw[line width=1.2pt,color=ccqqqq] (0.2750018304969665,1.3164445654600376) -- (0.29000181934972924,1.3367824677655373);
\draw[line width=1.2pt,color=ccqqqq] (0.29000181934972924,1.3367824677655373) -- (0.30500180820249195,1.357540183497144);
\draw[line width=1.2pt,color=ccqqqq] (0.30500180820249195,1.357540183497144) -- (0.32000179705525467,1.3787174216524065);
\draw[line width=1.2pt,color=ccqqqq] (0.32000179705525467,1.3787174216524065) -- (0.3350017859080174,1.4003134223862463);
\draw[line width=1.2pt,color=ccqqqq] (0.3350017859080174,1.4003134223862463) -- (0.3500017747607801,1.422326938089602);
\draw[line width=1.2pt,color=ccqqqq] (0.3500017747607801,1.422326938089602) -- (0.3650017636135428,1.4447562148420192);
\draw[line width=1.2pt,color=ccqqqq] (0.3650017636135428,1.4447562148420192) -- (0.3800017524663055,1.4675989743334903);
\draw[line width=1.2pt,color=ccqqqq] (0.3800017524663055,1.4675989743334903) -- (0.39500174131906823,1.4908523963549363);
\draw[line width=1.2pt,color=ccqqqq] (0.39500174131906823,1.4908523963549363) -- (0.41000173017183095,1.5145131019604348);
\draw[line width=1.2pt,color=ccqqqq] (0.41000173017183095,1.5145131019604348) -- (0.42500171902459366,1.5385771374075912);
\draw[line width=1.2pt,color=ccqqqq] (0.42500171902459366,1.5385771374075912) -- (0.4400017078773564,1.5630399589852422);
\draw[line width=1.2pt,color=ccqqqq] (0.4400017078773564,1.5630399589852422) -- (0.4550016967301191,1.587896418839931);
\draw[line width=1.2pt,color=ccqqqq] (0.4550016967301191,1.587896418839931) -- (0.4700016855828818,1.613140751914209);
\draw[line width=1.2pt,color=ccqqqq] (0.4700016855828818,1.613140751914209) -- (0.4850016744356445,1.6387665641107505);
\draw[line width=1.2pt,color=ccqqqq] (0.4850016744356445,1.6387665641107505) -- (0.5000016632884072,1.6647668217964553);
\draw[line width=1.2pt,color=ccqqqq] (0.5000016632884072,1.6647668217964553) -- (0.51500165214117,1.6911338427600933);
\draw[line width=1.2pt,color=ccqqqq] (0.51500165214117,1.6911338427600933) -- (0.5300016409939328,1.71785928873556);
\draw[line width=1.2pt,color=ccqqqq] (0.5300016409939328,1.71785928873556) -- (0.5450016298466955,1.7449341596004109);
\draw[line width=1.2pt,color=ccqqqq] (0.5450016298466955,1.7449341596004109) -- (0.5600016186994583,1.7723487893559928);
\draw[line width=1.2pt,color=ccqqqq] (0.5600016186994583,1.7723487893559928) -- (0.5750016075522211,1.8000928439911317);
\draw[line width=1.2pt,color=ccqqqq] (0.5750016075522211,1.8000928439911317) -- (0.5900015964049838,1.8281553213259696);
\draw[line width=1.2pt,color=ccqqqq] (0.5900015964049838,1.8281553213259696) -- (0.6050015852577466,1.8565245529261258);
\draw[line width=1.2pt,color=ccqqqq] (0.6050015852577466,1.8565245529261258) -- (0.6200015741105094,1.8851882081698808);
\draw[line width=1.2pt,color=ccqqqq] (0.6200015741105094,1.8851882081698808) -- (0.6350015629632721,1.9141333005425774);
\draw[line width=1.2pt,color=ccqqqq] (0.6350015629632721,1.9141333005425774) -- (0.6500015518160349,1.9433461962228766);
\draw[line width=1.2pt,color=ccqqqq] (0.6500015518160349,1.9433461962228766) -- (0.6650015406687977,1.9728126250149511);
\draw[line width=1.2pt,color=ccqqqq] (0.6650015406687977,1.9728126250149511) -- (0.6800015295215605,2.002517693669182);
\draw[line width=1.2pt,color=ccqqqq] (0.6800015295215605,2.002517693669182) -- (0.6950015183743232,2.0324459016215064);
\draw[line width=1.2pt,color=ccqqqq] (0.6950015183743232,2.0324459016215064) -- (0.710001507227086,2.0625811591682885);
\draw[line width=1.2pt,color=ccqqqq] (0.710001507227086,2.0625811591682885) -- (0.7250014960798488,2.0929068080795825);
\draw[line width=1.2pt,color=ccqqqq] (0.7250014960798488,2.0929068080795825) -- (0.7400014849326115,2.1234056446389857);
\draw[line width=1.2pt,color=ccqqqq] (0.7400014849326115,2.1234056446389857) -- (0.7550014737853743,2.154059945083057);
\draw[line width=1.2pt,color=ccqqqq] (0.7550014737853743,2.154059945083057) -- (0.7700014626381371,2.18485149339765);
\draw[line width=1.2pt,color=ccqqqq] (0.7700014626381371,2.18485149339765) -- (0.7850014514908998,2.215761611412598);
\draw[line width=1.2pt,color=ccqqqq] (0.7850014514908998,2.215761611412598) -- (0.8000014403436626,2.2467711911201036);
\draw[line width=1.2pt,color=ccqqqq] (0.8000014403436626,2.2467711911201036) -- (0.8150014291964254,2.2778607291261768);
\draw[line width=1.2pt,color=ccqqqq] (0.8150014291964254,2.2778607291261768) -- (0.8300014180491881,2.3090103631285572);
\draw[line width=1.2pt,color=ccqqqq] (0.8300014180491881,2.3090103631285572) -- (0.8450014069019509,2.340199910299052);
\draw[line width=1.2pt,color=ccqqqq] (0.8450014069019509,2.340199910299052) -- (0.8600013957547137,2.3714089074331723);
\draw[line width=1.2pt,color=ccqqqq] (0.8600013957547137,2.3714089074331723) -- (0.8750013846074765,2.4026166527156056);
\draw[line width=1.2pt,color=ccqqqq] (0.8750013846074765,2.4026166527156056) -- (0.8900013734602392,2.43380224893654);
\draw[line width=1.2pt,color=ccqqqq] (0.8900013734602392,2.43380224893654) -- (0.905001362313002,2.464944647981332);
\draw[line width=1.2pt,color=ccqqqq] (0.905001362313002,2.464944647981332) -- (0.9200013511657648,2.496022696404641);
\draw[line width=1.2pt,color=ccqqqq] (0.9200013511657648,2.496022696404641) -- (0.9350013400185275,2.527015181890066);
\draw[line width=1.2pt,color=ccqqqq] (0.9350013400185275,2.527015181890066) -- (0.9500013288712903,2.557900880387663);
\draw[line width=1.2pt,color=ccqqqq] (0.9500013288712903,2.557900880387663) -- (0.9650013177240531,2.5886586037145913);
\draw[line width=1.2pt,color=ccqqqq] (0.9650013177240531,2.5886586037145913) -- (0.9800013065768158,2.6192672473986587);
\draw[line width=1.2pt,color=ccqqqq] (0.9800013065768158,2.6192672473986587) -- (0.9950012954295786,2.649705838540742);
\draw[line width=1.2pt,color=ccqqqq] (0.9950012954295786,2.649705838540742) -- (1.0100012842823414,2.6799535834700716);
\draw[line width=1.2pt,color=ccqqqq] (1.0100012842823414,2.6799535834700716) -- (1.0250012731351041,2.709989914966185);
\draw[line width=1.2pt,color=ccqqqq] (1.0250012731351041,2.709989914966185) -- (1.040001261987867,2.7397945388229794);
\draw[line width=1.2pt,color=ccqqqq] (1.040001261987867,2.7397945388229794) -- (1.0550012508406297,2.769347479533794);
\draw[line width=1.2pt,color=ccqqqq] (1.0550012508406297,2.769347479533794) -- (1.0700012396933924,2.798629124881675);
\draw[line width=1.2pt,color=ccqqqq] (1.0700012396933924,2.798629124881675) -- (1.0850012285461552,2.8276202692260393);
\draw[line width=1.2pt,color=ccqqqq] (1.0850012285461552,2.8276202692260393) -- (1.100001217398918,2.8563021552855856);
\draw[line width=1.2pt,color=ccqqqq] (1.100001217398918,2.8563021552855856) -- (1.1150012062516808,2.8846565142276255);
\draw[line width=1.2pt,color=ccqqqq] (1.1150012062516808,2.8846565142276255) -- (1.1300011951044435,2.912665603885687);
\draw[line width=1.2pt,color=ccqqqq] (1.1300011951044435,2.912665603885687) -- (1.1450011839572063,2.940312244940369);
\draw[line width=1.2pt,color=ccqqqq] (1.1450011839572063,2.940312244940369) -- (1.160001172809969,2.9675798549126813);
\draw[line width=1.2pt,color=ccqqqq] (1.160001172809969,2.9675798549126813) -- (1.1750011616627318,2.994452479834413);
\draw[line width=1.2pt,color=ccqqqq] (1.1750011616627318,2.994452479834413) -- (1.1900011505154946,3.0209148234762777);
\draw[line width=1.2pt,color=ccqqqq] (1.1900011505154946,3.0209148234762777) -- (1.2050011393682574,3.046952274031464);
\draw[line width=1.2pt,color=ccqqqq] (1.2050011393682574,3.046952274031464) -- (1.2200011282210201,3.072550928169613);
\draw[line width=1.2pt,color=ccqqqq] (1.2200011282210201,3.072550928169613) -- (1.235001117073783,3.097697612393995);
\draw[line width=1.2pt,color=ccqqqq] (1.235001117073783,3.097697612393995) -- (1.2500011059265457,3.1223799016525424);
\draw[line width=1.2pt,color=ccqqqq] (1.2500011059265457,3.1223799016525424) -- (1.2650010947793084,3.1465861351712436);
\draw[line width=1.2pt,color=ccqqqq] (1.2650010947793084,3.1465861351712436) -- (1.2800010836320712,3.170305429496081);
\draw[line width=1.2pt,color=ccqqqq] (1.2800010836320712,3.170305429496081) -- (1.295001072484834,3.19352768874697);
\draw[line width=1.2pt,color=ccqqqq] (1.295001072484834,3.19352768874697) -- (1.3100010613375968,3.2162436121039275);
\draw[line width=1.2pt,color=ccqqqq] (1.3100010613375968,3.2162436121039275) -- (1.3250010501903595,3.2384446985617794);
\draw[line width=1.2pt,color=ccqqqq] (1.3250010501903595,3.2384446985617794) -- (1.3400010390431223,3.260123249005038);
\draw[line width=1.2pt,color=ccqqqq] (1.3400010390431223,3.260123249005038) -- (1.355001027895885,3.281272365668913);
\draw[line width=1.2pt,color=ccqqqq] (1.355001027895885,3.281272365668913) -- (1.3700010167486478,3.3018859490658063);
\draw[line width=1.2pt,color=ccqqqq] (1.3700010167486478,3.3018859490658063) -- (1.3850010056014106,3.3219586924688764);
\draw[line width=1.2pt,color=ccqqqq] (1.3850010056014106,3.3219586924688764) -- (1.4000009944541734,3.3414860740553234);
\draw[line width=1.2pt,color=ccqqqq] (1.4000009944541734,3.3414860740553234) -- (1.4150009833069361,3.360464346821912);
\draw[line width=1.2pt,color=ccqqqq] (1.4150009833069361,3.360464346821912) -- (1.430000972159699,3.378890526393823);
\draw[line width=1.2pt,color=ccqqqq] (1.430000972159699,3.378890526393823) -- (1.4450009610124617,3.396762376855241);
\draw[line width=1.2pt,color=ccqqqq] (1.4450009610124617,3.396762376855241) -- (1.4600009498652244,3.414078394736099);
\draw[line width=1.2pt,color=ccqqqq] (1.4600009498652244,3.414078394736099) -- (1.4750009387179872,3.4308377912941754);
\draw[line width=1.2pt,color=ccqqqq] (1.4750009387179872,3.4308377912941754) -- (1.49000092757075,3.447040473235208);
\draw[line width=1.2pt,color=ccqqqq] (1.49000092757075,3.447040473235208) -- (1.5050009164235127,3.462687022016026);
\draw[line width=1.2pt,color=ccqqqq] (1.5050009164235127,3.462687022016026) -- (1.5200009052762755,3.477778671876796);
\draw[line width=1.2pt,color=ccqqqq] (1.5200009052762755,3.477778671876796) -- (1.5350008941290383,3.492317286748513);
\draw[line width=1.2pt,color=ccqqqq] (1.5350008941290383,3.492317286748513) -- (1.550000882981801,3.5063053361808816);
\draw[line width=1.2pt,color=ccqqqq] (1.550000882981801,3.5063053361808816) -- (1.5650008718345638,3.5197458704336975);
\draw[line width=1.2pt,color=ccqqqq] (1.5650008718345638,3.5197458704336975) -- (1.5800008606873266,3.5326424948720256);
\draw[line width=1.2pt,color=ccqqqq] (1.5800008606873266,3.5326424948720256) -- (1.5950008495400894,3.5449993438017624);
\draw[line width=1.2pt,color=ccqqqq] (1.5950008495400894,3.5449993438017624) -- (1.6100008383928521,3.55682105387778);
\draw[line width=1.2pt,color=ccqqqq] (1.6100008383928521,3.55682105387778) -- (1.625000827245615,3.5681127372118286);
\draw[line width=1.2pt,color=ccqqqq] (1.625000827245615,3.5681127372118286) -- (1.6400008160983777,3.5788799543017804);
\draw[line width=1.2pt,color=ccqqqq] (1.6400008160983777,3.5788799543017804) -- (1.6550008049511404,3.5891286868977508);
\draw[line width=1.2pt,color=ccqqqq] (1.6550008049511404,3.5891286868977508) -- (1.6700007938039032,3.598865310914226);
\draw[line width=1.2pt,color=ccqqqq] (1.6700007938039032,3.598865310914226) -- (1.685000782656666,3.6080965694905713);
\draw[line width=1.2pt,color=ccqqqq] (1.685000782656666,3.6080965694905713) -- (1.7000007715094287,3.616829546295387);
\draw[line width=1.2pt,color=ccqqqq] (1.7000007715094287,3.616829546295387) -- (1.7150007603621915,3.625071639163038);
\draw[line width=1.2pt,color=ccqqqq] (1.7150007603621915,3.625071639163038) -- (1.7300007492149543,3.63283053414355);
\draw[line width=1.2pt,color=ccqqqq] (1.7300007492149543,3.63283053414355) -- (1.745000738067717,3.640114180039843);
\draw[line width=1.2pt,color=ccqqqq] (1.745000738067717,3.640114180039843) -- (1.7600007269204798,3.6469307634991366);
\draw[line width=1.2pt,color=ccqqqq] (1.7600007269204798,3.6469307634991366) -- (1.7750007157732426,3.6532886847183157);
\draw[line width=1.2pt,color=ccqqqq] (1.7750007157732426,3.6532886847183157) -- (1.7900007046260054,3.6591965338161034);
\draw[line width=1.2pt,color=ccqqqq] (1.7900007046260054,3.6591965338161034) -- (1.8050006934787681,3.6646630679182);
\draw[line width=1.2pt,color=ccqqqq] (1.8050006934787681,3.6646630679182) -- (1.820000682331531,3.6696971889950034);
\draw[line width=1.2pt,color=ccqqqq] (1.820000682331531,3.6696971889950034) -- (1.8350006711842937,3.6743079224853012);
\draw[line width=1.2pt,color=ccqqqq] (1.8350006711842937,3.6743079224853012) -- (1.8500006600370564,3.6785043967333264);
\draw[line width=1.2pt,color=ccqqqq] (1.8500006600370564,3.6785043967333264) -- (1.8650006488898192,3.6822958232609038);
\draw[line width=1.2pt,color=ccqqqq] (1.8650006488898192,3.6822958232609038) -- (1.880000637742582,3.6856914778910435);
\draw[line width=1.2pt,color=ccqqqq] (1.880000637742582,3.6856914778910435) -- (1.8950006265953447,3.6887006827343063);
\draw[line width=1.2pt,color=ccqqqq] (1.8950006265953447,3.6887006827343063) -- (1.9100006154481075,3.691332789044562);
\draw[line width=1.2pt,color=ccqqqq] (1.9100006154481075,3.691332789044562) -- (1.9250006043008703,3.6935971609464016);
\draw[line width=1.2pt,color=ccqqqq] (1.9250006043008703,3.6935971609464016) -- (1.940000593153633,3.6955031600324353);
\draw[line width=1.2pt,color=ccqqqq] (1.940000593153633,3.6955031600324353) -- (1.9550005820063958,3.6970601308250393);
\draw[line width=1.2pt,color=ccqqqq] (1.9550005820063958,3.6970601308250393) -- (1.9700005708591586,3.6982773870937407);
\draw[line width=1.2pt,color=ccqqqq] (1.9700005708591586,3.6982773870937407) -- (1.9850005597119214,3.6991641990164217);
\draw[line width=1.2pt,color=ccqqqq] (1.9850005597119214,3.6991641990164217) -- (2.000000548564684,3.699729781169811);
\draw[line width=1.2pt,color=ccqqqq] (2.000000548564684,3.699729781169811) -- (2.0150005374174467,3.6999832813323135);
\draw[line width=1.2pt,color=ccqqqq] (2.0150005374174467,3.6999832813323135) -- (2.0300005262702094,3.6999337700801442);
\draw[line width=1.2pt,color=ccqqqq] (2.0300005262702094,3.6999337700801442) -- (2.045000515122972,3.699590231155874);
\draw[line width=1.2pt,color=ccqqqq] (2.045000515122972,3.699590231155874) -- (2.060000503975735,3.6989615525869506);
\draw[line width=1.2pt,color=ccqqqq] (2.060000503975735,3.6989615525869506) -- (2.0750004928284977,3.6980565185304357);
\draw[line width=1.2pt,color=ccqqqq] (2.0750004928284977,3.6980565185304357) -- (2.0900004816812605,3.696883801819121);
\draw[line width=1.2pt,color=ccqqqq] (2.0900004816812605,3.696883801819121) -- (2.1050004705340233,3.695451957183331);
\draw[line width=1.2pt,color=ccqqqq] (2.1050004705340233,3.695451957183331) -- (2.120000459386786,3.6937694151220595);
\draw[line width=1.2pt,color=ccqqqq] (2.120000459386786,3.6937694151220595) -- (2.135000448239549,3.6918444763966294);
\draw[line width=1.2pt,color=ccqqqq] (2.135000448239549,3.6918444763966294) -- (2.1500004370923116,3.6896853071197677);
\draw[line width=1.2pt,color=ccqqqq] (2.1500004370923116,3.6896853071197677) -- (2.1650004259450744,3.687299934412854);
\draw[line width=1.2pt,color=ccqqqq] (2.1650004259450744,3.687299934412854) -- (2.180000414797837,3.6846962426041134);
\draw[line width=1.2pt,color=ccqqqq] (2.180000414797837,3.6846962426041134) -- (2.1950004036506,3.681881969940658);
\draw[line width=1.2pt,color=ccqqqq] (2.1950004036506,3.681881969940658) -- (2.2100003925033627,3.6788647057875483);
\draw[line width=1.2pt,color=ccqqqq] (2.2100003925033627,3.6788647057875483) -- (2.2250003813561254,3.6756518882873865);
\draw[line width=1.2pt,color=ccqqqq] (2.2250003813561254,3.6756518882873865) -- (2.240000370208888,3.6722508024544283);
\draw[line width=1.2pt,color=ccqqqq] (2.240000370208888,3.6722508024544283) -- (2.255000359061651,3.668668578677695);
\draw[line width=1.2pt,color=ccqqqq] (2.255000359061651,3.668668578677695) -- (2.2700003479144137,3.664912191608206);
\draw[line width=1.2pt,color=ccqqqq] (2.2700003479144137,3.664912191608206) -- (2.2850003367671765,3.6609884594060706);
\draw[line width=1.2pt,color=ccqqqq] (2.2850003367671765,3.6609884594060706) -- (2.3000003256199393,3.656904043323904);
\draw[line width=1.2pt,color=ccqqqq] (2.3000003256199393,3.656904043323904) -- (2.315000314472702,3.6526654476037743);
\draw[line width=1.2pt,color=ccqqqq] (2.315000314472702,3.6526654476037743) -- (2.330000303325465,3.6482790196656616);
\draw[line width=1.2pt,color=ccqqqq] (2.330000303325465,3.6482790196656616) -- (2.3450002921782276,3.6437509505662087);
\draw[line width=1.2pt,color=ccqqqq] (2.3450002921782276,3.6437509505662087) -- (2.3600002810309904,3.639087275707375);
\draw[line width=1.2pt,color=ccqqqq] (2.3600002810309904,3.639087275707375) -- (2.375000269883753,3.634293875775419);
\draw[line width=1.2pt,color=ccqqqq] (2.375000269883753,3.634293875775419) -- (2.390000258736516,3.629376477891493);
\draw[line width=1.2pt,color=ccqqqq] (2.390000258736516,3.629376477891493) -- (2.4050002475892787,3.6243406569559578);
\draw[line width=1.2pt,color=ccqqqq] (2.4050002475892787,3.6243406569559578) -- (2.4200002364420414,3.619191837169363);
\draw[line width=1.2pt,color=ccqqqq] (2.4200002364420414,3.619191837169363) -- (2.435000225294804,3.6139352937138796);
\draw[line width=1.2pt,color=ccqqqq] (2.435000225294804,3.6139352937138796) -- (2.450000214147567,3.608576154579775);
\draw[line width=1.2pt,color=ccqqqq] (2.450000214147567,3.608576154579775) -- (2.4650002030003297,3.6031194025223345);
\draw[line width=1.2pt,color=ccqqqq] (2.4650002030003297,3.6031194025223345) -- (2.4800001918530925,3.5975698771354256);
\draw[line width=1.2pt,color=ccqqqq] (2.4800001918530925,3.5975698771354256) -- (2.4950001807058553,3.5919322770286604);
\draw[line width=1.2pt,color=ccqqqq] (2.4950001807058553,3.5919322770286604) -- (2.510000169558618,3.586211162095877);
\draw[line width=1.2pt,color=ccqqqq] (2.510000169558618,3.586211162095877) -- (2.525000158411381,3.5804109558633774);
\draw[line width=1.2pt,color=ccqqqq] (2.525000158411381,3.5804109558633774) -- (2.5400001472641436,3.5745359479070764);
\draw[line width=1.2pt,color=ccqqqq] (2.5400001472641436,3.5745359479070764) -- (2.5550001361169064,3.568590296328387);
\draw[line width=1.2pt,color=ccqqqq] (2.5550001361169064,3.568590296328387) -- (2.570000124969669,3.5625780302793393);
\draw[line width=1.2pt,color=ccqqqq] (2.570000124969669,3.5625780302793393) -- (2.585000113822432,3.5565030525280488);
\draw[line width=1.2pt,color=ccqqqq] (2.585000113822432,3.5565030525280488) -- (2.6000001026751947,3.550369142056267);
\draw[line width=1.2pt,color=ccqqqq] (2.6000001026751947,3.550369142056267) -- (2.6150000915279574,3.5441799566813295);
\draw[line width=1.2pt,color=ccqqqq] (2.6150000915279574,3.5441799566813295) -- (2.63000008038072,3.537939035695348);
\draw[line width=1.2pt,color=ccqqqq] (2.63000008038072,3.537939035695348) -- (2.645000069233483,3.5316498025150844);
\draw[line width=1.2pt,color=ccqqqq] (2.645000069233483,3.5316498025150844) -- (2.6600000580862457,3.5253155673363583);
\draw[line width=1.2pt,color=ccqqqq] (2.6600000580862457,3.5253155673363583) -- (2.6750000469390085,3.518939529787418);
\draw[line width=1.2pt,color=ccqqqq] (2.6750000469390085,3.518939529787418) -- (2.6900000357917713,3.512524781576079);
\draw[line width=1.2pt,color=ccqqqq] (2.6900000357917713,3.512524781576079) -- (2.705000024644534,3.5060743091259257);
\draw[line width=1.2pt,color=ccqqqq] (2.705000024644534,3.5060743091259257) -- (2.720000013497297,3.4995909961972393);
\draw[line width=1.2pt,color=ccqqqq] (2.720000013497297,3.4995909961972393) -- (2.7350000023500596,3.493077626488729);
\draw[line width=1.2pt,color=ccqqqq] (2.7350000023500596,3.493077626488729) -- (2.7499999912028223,3.486536886216491);
\draw[line width=1.2pt,color=ccqqqq] (2.7499999912028223,3.486536886216491) -- (2.764999980055585,3.4799713666669723);
\draw[line width=1.2pt,color=ccqqqq] (2.764999980055585,3.4799713666669723) -- (2.779999968908348,3.4733835667210307);
\draw[line width=1.2pt,color=ccqqqq] (2.779999968908348,3.4733835667210307) -- (2.7949999577611107,3.4667758953464975);
\draw[line width=1.2pt,color=ccqqqq] (2.7949999577611107,3.4667758953464975) -- (2.8099999466138734,3.460150674056915);
\draw[line width=1.2pt,color=ccqqqq] (2.8099999466138734,3.460150674056915) -- (2.824999935466636,3.4535101393344085);
\draw[line width=1.2pt,color=ccqqqq] (2.824999935466636,3.4535101393344085) -- (2.839999924319399,3.4468564450148778);
\draw[line width=1.2pt,color=ccqqqq] (2.839999924319399,3.4468564450148778) -- (2.8549999131721617,3.440191664633944);
\draw[line width=1.2pt,color=ccqqqq] (2.8549999131721617,3.440191664633944) -- (2.8699999020249245,3.4335177937322947);
\draw[line width=1.2pt,color=ccqqqq] (2.8699999020249245,3.4335177937322947) -- (2.8849998908776873,3.426836752119268);
\draw[line width=1.2pt,color=ccqqqq] (2.8849998908776873,3.426836752119268) -- (2.89999987973045,3.4201503860937077);
\draw[line width=1.2pt,color=ccqqqq] (2.89999987973045,3.4201503860937077) -- (2.914999868583213,3.4134604706212963);
\draw[line width=1.2pt,color=ccqqqq] (2.914999868583213,3.4134604706212963) -- (2.9299998574359756,3.406768711467723);
\draw[line width=1.2pt,color=ccqqqq] (2.9299998574359756,3.406768711467723) -- (2.9449998462887383,3.4000767472872013);
\draw[line width=1.2pt,color=ccqqqq] (2.9449998462887383,3.4000767472872013) -- (2.959999835141501,3.3933861516659833);
\draw[line width=1.2pt,color=ccqqqq] (2.959999835141501,3.3933861516659833) -- (2.974999823994264,3.3866984351206373);
\draw[line width=1.2pt,color=ccqqqq] (2.974999823994264,3.3866984351206373) -- (2.9899998128470267,3.3800150470509793);
\draw[line width=1.2pt,color=ccqqqq] (2.9899998128470267,3.3800150470509793) -- (3.0049998016997894,3.3733373776476503);
\draw[line width=1.2pt,color=ccqqqq] (3.0049998016997894,3.3733373776476503) -- (3.019999790552552,3.366666759754422);
\draw[line width=1.2pt,color=ccqqqq] (3.019999790552552,3.366666759754422) -- (3.034999779405315,3.360004470685408);
\draw[line width=1.2pt,color=ccqqqq] (3.034999779405315,3.360004470685408) -- (3.0499997682580777,3.353351733997437);
\draw[line width=1.2pt,color=ccqqqq] (3.0499997682580777,3.353351733997437) -- (3.0649997571108405,3.346709721217908);
\draw[line width=1.2pt,color=ccqqqq] (3.0649997571108405,3.346709721217908) -- (3.0799997459636033,3.3400795535285246);
\draw[line width=1.2pt,color=ccqqqq] (3.0799997459636033,3.3400795535285246) -- (3.094999734816366,3.333462303405348);
\draw[line width=1.2pt,color=ccqqqq] (3.094999734816366,3.333462303405348) -- (3.109999723669129,3.326858996215683);
\draw[line width=1.2pt,color=ccqqqq] (3.109999723669129,3.326858996215683) -- (3.1249997125218916,3.3202706117723295);
\draw[line width=1.2pt,color=ccqqqq] (3.1249997125218916,3.3202706117723295) -- (3.1399997013746543,3.3136980858457967);
\draw[line width=1.2pt,color=ccqqqq] (3.1399997013746543,3.3136980858457967) -- (3.154999690227417,3.307142311635106);
\draw[line width=1.2pt,color=ccqqqq] (3.154999690227417,3.307142311635106) -- (3.16999967908018,3.3006041411978266);
\draw[line width=1.2pt,color=ccqqqq] (3.16999967908018,3.3006041411978266) -- (3.1849996679329426,3.2940843868400425);
\draw[line width=1.2pt,color=ccqqqq] (3.1849996679329426,3.2940843868400425) -- (3.1999996567857054,3.2875838224669462);
\draw[line width=1.2pt,color=ccqqqq] (3.1999996567857054,3.2875838224669462) -- (3.214999645638468,3.2811031848947896);
\draw[line width=1.2pt,color=ccqqqq] (3.214999645638468,3.2811031848947896) -- (3.229999634491231,3.2746431751249316);
\draw[line width=1.2pt,color=ccqqqq] (3.229999634491231,3.2746431751249316) -- (3.2449996233439937,3.268204459580743);
\draw[line width=1.2pt,color=ccqqqq] (3.2449996233439937,3.268204459580743) -- (3.2599996121967565,3.261787671308129);
\draw[line width=1.2pt,color=ccqqqq] (3.2599996121967565,3.261787671308129) -- (3.2749996010495193,3.2553934111404477);
\draw[line width=1.2pt,color=ccqqqq] (3.2749996010495193,3.2553934111404477) -- (3.289999589902282,3.2490222488285996);
\draw[line width=1.2pt,color=ccqqqq] (3.289999589902282,3.2490222488285996) -- (3.304999578755045,3.242674724137073);
\draw[line width=1.2pt,color=ccqqqq] (3.304999578755045,3.242674724137073) -- (3.3199995676078076,3.2363513479067283);
\draw[line width=1.2pt,color=ccqqqq] (3.3199995676078076,3.2363513479067283) -- (3.3349995564605703,3.2300526030851033);
\draw[line width=1.2pt,color=ccqqqq] (3.3349995564605703,3.2300526030851033) -- (3.349999545313333,3.2237789457250194);
\draw[line width=1.2pt,color=ccqqqq] (3.349999545313333,3.2237789457250194) -- (3.364999534166096,3.217530805952266);
\draw[line width=1.2pt,color=ccqqqq] (3.364999534166096,3.217530805952266) -- (3.3799995230188586,3.2113085889031354);
\draw[line width=1.2pt,color=ccqqqq] (3.3799995230188586,3.2113085889031354) -- (3.3949995118716214,3.2051126756325727);
\draw[line width=1.2pt,color=ccqqqq] (3.3949995118716214,3.2051126756325727) -- (3.409999500724384,3.1989434239936942);
\draw[line width=1.2pt,color=ccqqqq] (3.409999500724384,3.1989434239936942) -- (3.424999489577147,3.192801169489434);
\draw[line width=1.2pt,color=ccqqqq] (3.424999489577147,3.192801169489434) -- (3.4399994784299097,3.186686226097043);
\draw[line width=1.2pt,color=ccqqqq] (3.4399994784299097,3.186686226097043) -- (3.4549994672826725,3.1805988870661874);
\draw[line width=1.2pt,color=ccqqqq] (3.4549994672826725,3.1805988870661874) -- (3.4699994561354353,3.174539425691352);
\draw[line width=1.2pt,color=ccqqqq] (3.4699994561354353,3.174539425691352) -- (3.484999444988198,3.168508096059261);
\draw[line width=1.2pt,color=ccqqqq] (3.484999444988198,3.168508096059261) -- (3.499999433840961,3.1625051337720187);
\draw[line width=1.2pt,color=ccqqqq] (3.499999433840961,3.1625051337720187) -- (3.5149994226937236,3.156530756646637);
\draw[line width=1.2pt,color=ccqqqq] (3.5149994226937236,3.156530756646637) -- (3.5299994115464863,3.150585165391642);
\draw[line width=1.2pt,color=ccqqqq] (3.5299994115464863,3.150585165391642) -- (3.544999400399249,3.1446685442614033);
\draw[line width=1.2pt,color=ccqqqq] (3.544999400399249,3.1446685442614033) -- (3.559999389252012,3.13878106168883);
\draw[line width=1.2pt,color=ccqqqq] (3.559999389252012,3.13878106168883) -- (3.5749993781047746,3.1329228708970795);
\draw[line width=1.2pt,color=ccqqqq] (3.5749993781047746,3.1329228708970795) -- (3.5899993669575374,3.127094110490882);
\draw[line width=1.2pt,color=ccqqqq] (3.5899993669575374,3.127094110490882) -- (3.6049993558103,3.1212949050280896);
\draw[line width=1.2pt,color=ccqqqq] (3.6049993558103,3.1212949050280896) -- (3.619999344663063,3.115525365572048);
\draw[line width=1.2pt,color=ccqqqq] (3.619999344663063,3.115525365572048) -- (3.6349993335158257,3.1097855902253624);
\draw[line width=1.2pt,color=ccqqqq] (3.6349993335158257,3.1097855902253624) -- (3.6499993223685885,3.104075664645615);
\draw[line width=1.2pt,color=ccqqqq] (3.6499993223685885,3.104075664645615) -- (3.6649993112213513,3.0983956625435995);
\draw[line width=1.2pt,color=ccqqqq] (3.6649993112213513,3.0983956625435995) -- (3.679999300074114,3.0927456461645972);
\draw[line width=1.2pt,color=ccqqqq] (3.679999300074114,3.0927456461645972) -- (3.694999288926877,3.0871256667532174);
\draw[line width=1.2pt,color=ccqqqq] (3.694999288926877,3.0871256667532174) -- (3.7099992777796396,3.0815357650023243);
\draw[line width=1.2pt,color=ccqqqq] (3.7099992777796396,3.0815357650023243) -- (3.7249992666324023,3.0759759714865282);
\draw[line width=1.2pt,color=ccqqqq] (3.7249992666324023,3.0759759714865282) -- (3.739999255485165,3.0704463070807453);
\draw[line width=1.2pt,color=ccqqqq] (3.739999255485165,3.0704463070807453) -- (3.754999244337928,3.0649467833642774);
\draw[line width=1.2pt,color=ccqqqq] (3.754999244337928,3.0649467833642774) -- (3.7699992331906906,3.0594774030108827);
\draw[line width=1.2pt,color=ccqqqq] (3.7699992331906906,3.0594774030108827) -- (3.7849992220434534,3.0540381601652795);
\draw[line width=1.2pt,color=ccqqqq] (3.7849992220434534,3.0540381601652795) -- (3.799999210896216,3.04862904080651);
\draw[line width=1.2pt,color=ccqqqq] (3.799999210896216,3.04862904080651) -- (3.814999199748979,3.0432500230985973);
\draw[line width=1.2pt,color=ccqqqq] (3.814999199748979,3.0432500230985973) -- (3.8299991886017417,3.0379010777288915);
\draw[line width=1.2pt,color=ccqqqq] (3.8299991886017417,3.0379010777288915) -- (3.8449991774545045,3.032582168234511);
\draw[line width=1.2pt,color=ccqqqq] (3.8449991774545045,3.032582168234511) -- (3.8599991663072672,3.0272932513172646);
\draw[line width=1.2pt,color=ccqqqq] (3.8599991663072672,3.0272932513172646) -- (3.87499915516003,3.022034277147424);
\draw[line width=1.2pt,color=ccqqqq] (3.87499915516003,3.022034277147424) -- (3.889999144012793,3.0168051896567127);
\draw[line width=1.2pt,color=ccqqqq] (3.889999144012793,3.0168051896567127) -- (3.9049991328655556,3.0116059268208724);
\draw[line width=1.2pt,color=ccqqqq] (3.9049991328655556,3.0116059268208724) -- (3.9199991217183183,3.0064364209321317);
\draw[line width=1.2pt,color=ccqqqq] (3.9199991217183183,3.0064364209321317) -- (3.934999110571081,3.0012965988619293);
\draw[line width=1.2pt,color=ccqqqq] (3.934999110571081,3.0012965988619293) -- (3.949999099423844,2.996186382314201);
\draw[line width=1.2pt,color=ccqqqq] (3.949999099423844,2.996186382314201) -- (3.9649990882766066,2.9911056880695464);
\draw[line width=1.2pt,color=ccqqqq] (3.9649990882766066,2.9911056880695464) -- (3.9799990771293694,2.9860544282205805);
\draw[line width=1.2pt,color=ccqqqq] (3.9799990771293694,2.9860544282205805) -- (3.994999065982132,2.9810325103987623);
\draw[line width=1.2pt,color=ccqqqq] (3.994999065982132,2.9810325103987623) -- (4.0099990548348945,2.9760398379929835);
\draw[line width=1.2pt,color=ccqqqq] (4.0099990548348945,2.9760398379929835) -- (4.024999043687657,2.9710763103601954);
\draw[line width=1.2pt,color=ccqqqq] (4.024999043687657,2.9710763103601954) -- (4.03999903254042,2.96614182302834);
\draw[line width=1.2pt,color=ccqqqq] (4.03999903254042,2.96614182302834) -- (4.054999021393183,2.9612362678918487);
\draw[line width=1.2pt,color=ccqqqq] (4.054999021393183,2.9612362678918487) -- (4.069999010245946,2.956359533399947);
\draw[line width=1.2pt,color=ccqqqq] (4.069999010245946,2.956359533399947) -- (4.084998999098708,2.951511504738023);
\draw[line width=1.2pt,color=ccqqqq] (4.084998999098708,2.951511504738023) -- (4.099998987951471,2.946692064002288);
\draw[line width=1.2pt,color=ccqqqq] (4.099998987951471,2.946692064002288) -- (4.114998976804234,2.941901090367951);
\draw[line width=1.2pt,color=ccqqqq] (4.114998976804234,2.941901090367951) -- (4.129998965656997,2.9371384602511403);
\draw[line width=1.2pt,color=ccqqqq] (4.129998965656997,2.9371384602511403) -- (4.144998954509759,2.9324040474647726);
\draw[line width=1.2pt,color=ccqqqq] (4.144998954509759,2.9324040474647726) -- (4.159998943362522,2.9276977233685852);
\draw[line width=1.2pt,color=ccqqqq] (4.159998943362522,2.9276977233685852) -- (4.174998932215285,2.9230193570135268);
\draw[line width=1.2pt,color=ccqqqq] (4.174998932215285,2.9230193570135268) -- (4.189998921068048,2.918368815280701);
\draw[line width=1.2pt,color=ccqqqq] (4.189998921068048,2.918368815280701) -- (4.2049989099208105,2.9137459630150464);
\draw[line width=1.2pt,color=ccqqqq] (4.2049989099208105,2.9137459630150464) -- (4.219998898773573,2.9091506631539445);
\draw[line width=1.2pt,color=ccqqqq] (4.219998898773573,2.9091506631539445) -- (4.234998887626336,2.904582776850912);
\draw[line width=1.2pt,color=ccqqqq] (4.234998887626336,2.904582776850912) -- (4.249998876479099,2.90004216359457);
\draw[line width=1.2pt,color=ccqqqq] (4.249998876479099,2.90004216359457) -- (4.264998865331862,2.8955286813230297);
\draw[line width=1.2pt,color=ccqqqq] (4.264998865331862,2.8955286813230297) -- (4.279998854184624,2.8910421865338742);
\draw[line width=1.2pt,color=ccqqqq] (4.279998854184624,2.8910421865338742) -- (4.294998843037387,2.8865825343898734);
\draw[line width=1.2pt,color=ccqqqq] (4.294998843037387,2.8865825343898734) -- (4.30999883189015,2.8821495788205893);
\draw[line width=1.2pt,color=ccqqqq] (4.30999883189015,2.8821495788205893) -- (4.324998820742913,2.8777431726200113);
\draw[line width=1.2pt,color=ccqqqq] (4.324998820742913,2.8777431726200113) -- (4.339998809595675,2.8733631675403606);
\draw[line width=1.2pt,color=ccqqqq] (4.339998809595675,2.8733631675403606) -- (4.354998798448438,2.869009414382195);
\draw[line width=1.2pt,color=ccqqqq] (4.354998798448438,2.869009414382195) -- (4.369998787301201,2.8646817630809505);
\draw[line width=1.2pt,color=ccqqqq] (4.369998787301201,2.8646817630809505) -- (4.384998776153964,2.860380062790033);
\draw[line width=1.2pt,color=ccqqqq] (4.384998776153964,2.860380062790033) -- (4.3999987650067265,2.8561041619605927);
\draw[line width=1.2pt,color=ccqqqq] (4.3999987650067265,2.8561041619605927) -- (4.414998753859489,2.851853908418091);
\draw[line width=1.2pt,color=ccqqqq] (4.414998753859489,2.851853908418091) -- (4.429998742712252,2.8476291494357735);
\draw[line width=1.2pt,color=ccqqqq] (4.429998742712252,2.8476291494357735) -- (4.444998731565015,2.8434297318051582);
\draw[line width=1.2pt,color=ccqqqq] (4.444998731565015,2.8434297318051582) -- (4.459998720417778,2.8392555019036494);
\draw[line width=1.2pt,color=ccqqqq] (4.459998720417778,2.8392555019036494) -- (4.47499870927054,2.8351063057593677);
\draw[line width=1.2pt,color=ccqqqq] (4.47499870927054,2.8351063057593677) -- (4.489998698123303,2.8309819891133063);
\draw[line width=1.2pt,color=ccqqqq] (4.489998698123303,2.8309819891133063) -- (4.504998686976066,2.8268823974789035);
\draw[line width=1.2pt,color=ccqqqq] (4.504998686976066,2.8268823974789035) -- (4.519998675828829,2.82280737619912);
\draw[line width=1.2pt,color=ccqqqq] (4.519998675828829,2.82280737619912) -- (4.534998664681591,2.818756770501113);
\draw[line width=1.2pt,color=ccqqqq] (4.534998664681591,2.818756770501113) -- (4.549998653534354,2.814730425548598);
\draw[line width=1.2pt,color=ccqqqq] (4.549998653534354,2.814730425548598) -- (4.564998642387117,2.8107281864919704);
\draw[line width=1.2pt,color=ccqqqq] (4.564998642387117,2.8107281864919704) -- (4.57999863123988,2.8067498985162693);
\draw[line width=1.2pt,color=ccqqqq] (4.57999863123988,2.8067498985162693) -- (4.5949986200926425,2.8027954068870744);
\draw[line width=1.2pt,color=ccqqqq] (4.5949986200926425,2.8027954068870744) -- (4.609998608945405,2.7988645569943875);
\draw[line width=1.2pt,color=ccqqqq] (4.609998608945405,2.7988645569943875) -- (4.624998597798168,2.794957194394589);
\draw[line width=1.2pt,color=ccqqqq] (4.624998597798168,2.794957194394589) -- (4.639998586650931,2.7910731648505323);
\draw[line width=1.2pt,color=ccqqqq] (4.639998586650931,2.7910731648505323) -- (4.654998575503694,2.7872123143698415);
\draw[line width=1.2pt,color=ccqqqq] (4.654998575503694,2.7872123143698415) -- (4.669998564356456,2.7833744892414822);
\draw[line width=1.2pt,color=ccqqqq] (4.669998564356456,2.7833744892414822) -- (4.684998553209219,2.7795595360706606);
\draw[line width=1.2pt,color=ccqqqq] (4.684998553209219,2.7795595360706606) -- (4.699998542061982,2.775767301812121);
\draw[line width=1.2pt,color=ccqqqq] (4.699998542061982,2.775767301812121) -- (4.714998530914745,2.7719976338018943);
\draw[line width=1.2pt,color=ccqqqq] (4.714998530914745,2.7719976338018943) -- (4.729998519767507,2.7682503797875486);
\draw[line width=1.2pt,color=ccqqqq] (4.729998519767507,2.7682503797875486) -- (4.74499850862027,2.7645253879570113);
\draw[line width=1.2pt,color=ccqqqq] (4.74499850862027,2.7645253879570113) -- (4.759998497473033,2.760822506966001);
\draw[line width=1.2pt,color=ccqqqq] (4.759998497473033,2.760822506966001) -- (4.774998486325796,2.757141585964125);
\draw[line width=1.2pt,color=ccqqqq] (4.774998486325796,2.757141585964125) -- (4.7899984751785585,2.753482474619697);
\draw[line width=1.2pt,color=ccqqqq] (4.7899984751785585,2.753482474619697) -- (4.804998464031321,2.74984502314331);
\draw[line width=1.2pt,color=ccqqqq] (4.804998464031321,2.74984502314331) -- (4.819998452884084,2.746229082310224);
\draw[line width=1.2pt,color=ccqqqq] (4.819998452884084,2.746229082310224) -- (4.834998441736847,2.7426345034815998);
\draw[line width=1.2pt,color=ccqqqq] (4.834998441736847,2.7426345034815998) -- (4.8499984305896096,2.7390611386246313);
\draw[line width=1.2pt,color=ccqqqq] (4.8499984305896096,2.7390611386246313) -- (4.864998419442372,2.7355088403316152);
\draw[line width=1.2pt,color=ccqqqq] (4.864998419442372,2.7355088403316152) -- (4.879998408295135,2.7319774618379906);
\draw[line width=1.2pt,color=ccqqqq] (4.879998408295135,2.7319774618379906) -- (4.894998397147898,2.7284668570393973);
\draw[line width=1.2pt,color=ccqqqq] (4.894998397147898,2.7284668570393973) -- (4.909998386000661,2.7249768805077834);
\draw[line width=1.2pt,color=ccqqqq] (4.909998386000661,2.7249768805077834) -- (4.924998374853423,2.721507387506599);
\draw[line width=1.2pt,color=ccqqqq] (4.924998374853423,2.721507387506599) -- (4.939998363706186,2.7180582340051114);
\draw[line width=1.2pt,color=ccqqqq] (4.939998363706186,2.7180582340051114) -- (4.954998352558949,2.7146292766918734);
\draw[line width=1.2pt,color=ccqqqq] (4.954998352558949,2.7146292766918734) -- (4.969998341411712,2.7112203729873814);
\draw[line width=1.2pt,color=ccqqqq] (4.969998341411712,2.7112203729873814) -- (4.9849983302644745,2.7078313810559456);
\draw[line width=1.2pt,color=ccqqqq] (4.9849983302644745,2.7078313810559456) -- (4.999998319117237,2.704462159816811);
\draw[line width=1.2pt,color=ccqqqq] (4.999998319117237,2.704462159816811) -- (5.01499830797,2.701112568954554);
\draw[line width=1.2pt,color=ccqqqq] (5.01499830797,2.701112568954554) -- (5.029998296822763,2.697782468928782);
\draw[line width=1.2pt,color=ccqqqq] (5.029998296822763,2.697782468928782) -- (5.0449982856755256,2.6944717209831652);
\draw[line width=1.2pt,color=ccqqqq] (5.0449982856755256,2.6944717209831652) -- (5.059998274528288,2.691180187153822);
\draw[line width=1.2pt,color=ccqqqq] (5.059998274528288,2.691180187153822) -- (5.074998263381051,2.6879077302770895);
\draw[line width=1.2pt,color=ccqqqq] (5.074998263381051,2.6879077302770895) -- (5.089998252233814,2.6846542139966987);
\draw[line width=1.2pt,color=ccqqqq] (5.089998252233814,2.6846542139966987) -- (5.104998241086577,2.681419502770379);
\draw[line width=1.2pt,color=ccqqqq] (5.104998241086577,2.681419502770379) -- (5.119998229939339,2.6782034618759125);
\draw[line width=1.2pt,color=ccqqqq] (5.119998229939339,2.6782034618759125) -- (5.134998218792102,2.675005957416664);
\draw[line width=1.2pt,color=ccqqqq] (5.134998218792102,2.675005957416664) -- (5.149998207644865,2.6718268563266054);
\draw[line width=1.2pt,color=ccqqqq] (5.149998207644865,2.6718268563266054) -- (5.164998196497628,2.668666026374852);
\draw[line width=1.2pt,color=ccqqqq] (5.164998196497628,2.668666026374852) -- (5.1799981853503905,2.665523336169735);
\draw[line width=1.2pt,color=ccqqqq] (5.1799981853503905,2.665523336169735) -- (5.194998174203153,2.6623986551624252);
\draw[line width=1.2pt,color=ccqqqq] (5.194998174203153,2.6623986551624252) -- (5.209998163055916,2.6592918536501284);
\draw[line width=1.2pt,color=ccqqqq] (5.209998163055916,2.6592918536501284) -- (5.224998151908679,2.6562028027788704);
\draw[line width=1.2pt,color=ccqqqq] (5.224998151908679,2.6562028027788704) -- (5.2399981407614415,2.6531313745458833);
\draw[line width=1.2pt,color=ccqqqq] (5.2399981407614415,2.6531313745458833) -- (5.254998129614204,2.650077441801617);
\draw[line width=1.2pt,color=ccqqqq] (5.254998129614204,2.650077441801617) -- (5.269998118466967,2.647040878251386);
\draw[line width=1.2pt,color=ccqqqq] (5.269998118466967,2.647040878251386) -- (5.28499810731973,2.644021558456668);
\draw[line width=1.2pt,color=ccqqqq] (5.28499810731973,2.644021558456668) -- (5.299998096172493,2.641019357836072);
\draw[line width=1.2pt,color=ccqqqq] (5.299998096172493,2.641019357836072) -- (5.314998085025255,2.6380341526659836);
\draw[line width=1.2pt,color=ccqqqq] (5.314998085025255,2.6380341526659836) -- (5.329998073878018,2.635065820080907);
\draw[line width=1.2pt,color=ccqqqq] (5.329998073878018,2.635065820080907) -- (5.344998062730781,2.632114238073516);
\draw[line width=1.2pt,color=ccqqqq] (5.344998062730781,2.632114238073516) -- (5.359998051583544,2.629179285494419);
\draw[line width=1.2pt,color=ccqqqq] (5.359998051583544,2.629179285494419) -- (5.3749980404363065,2.6262608420516647);
\draw[line width=1.2pt,color=ccqqqq] (5.3749980404363065,2.6262608420516647) -- (5.389998029289069,2.6233587883099867);
\draw[line width=1.2pt,color=ccqqqq] (5.389998029289069,2.6233587883099867) -- (5.404998018141832,2.620473005689803);
\draw[line width=1.2pt,color=ccqqqq] (5.404998018141832,2.620473005689803) -- (5.419998006994595,2.617603376465985);
\draw[line width=1.2pt,color=ccqqqq] (5.419998006994595,2.617603376465985) -- (5.4349979958473575,2.6147497837664004);
\draw[line width=1.2pt,color=ccqqqq] (5.4349979958473575,2.6147497837664004) -- (5.44999798470012,2.6119121115702457);
\draw[line width=1.2pt,color=ccqqqq] (5.44999798470012,2.6119121115702457) -- (5.464997973552883,2.6090902447061706);
\draw[line width=1.2pt,color=ccqqqq] (5.464997973552883,2.6090902447061706) -- (5.479997962405646,2.6062840688502162);
\draw[line width=1.2pt,color=ccqqqq] (5.479997962405646,2.6062840688502162) -- (5.494997951258409,2.603493470523559);
\draw[line width=1.2pt,color=ccqqqq] (5.494997951258409,2.603493470523559) -- (5.509997940111171,2.6007183370900866);
\draw[line width=1.2pt,color=ccqqqq] (5.509997940111171,2.6007183370900866) -- (5.524997928963934,2.5979585567538024);
\draw[line width=1.2pt,color=ccqqqq] (5.524997928963934,2.5979585567538024) -- (5.539997917816697,2.595214018556068);
\draw[line width=1.2pt,color=ccqqqq] (5.539997917816697,2.595214018556068) -- (5.55499790666946,2.5924846123727012);
\draw[line width=1.2pt,color=ccqqqq] (5.55499790666946,2.5924846123727012) -- (5.5699978955222225,2.589770228910921);
\draw[line width=1.2pt,color=ccqqqq] (5.5699978955222225,2.589770228910921) -- (5.584997884374985,2.5870707597061613);
\draw[line width=1.2pt,color=ccqqqq] (5.584997884374985,2.5870707597061613) -- (5.599997873227748,2.584386097118754);
\draw[line width=1.2pt,color=ccqqqq] (5.599997873227748,2.584386097118754) -- (5.614997862080511,2.5817161343304864);
\draw[line width=1.2pt,color=ccqqqq] (5.614997862080511,2.5817161343304864) -- (5.6299978509332735,2.5790607653410427);
\draw[line width=1.2pt,color=ccqqqq] (5.6299978509332735,2.5790607653410427) -- (5.644997839786036,2.5764198849643365);
\draw[line width=1.2pt,color=ccqqqq] (5.644997839786036,2.5764198849643365) -- (5.659997828638799,2.573793388824737);
\draw[line width=1.2pt,color=ccqqqq] (5.659997828638799,2.573793388824737) -- (5.674997817491562,2.571181173353196);
\draw[line width=1.2pt,color=ccqqqq] (5.674997817491562,2.571181173353196) -- (5.689997806344325,2.568583135783284);
\draw[line width=1.2pt,color=ccqqqq] (5.689997806344325,2.568583135783284) -- (5.704997795197087,2.565999174147135);
\draw[line width=1.2pt,color=ccqqqq] (5.704997795197087,2.565999174147135) -- (5.71999778404985,2.563429187271312);
\draw[line width=1.2pt,color=ccqqqq] (5.71999778404985,2.563429187271312) -- (5.734997772902613,2.5608730747725907);
\draw[line width=1.2pt,color=ccqqqq] (5.734997772902613,2.5608730747725907) -- (5.749997761755376,2.5583307370536756);
\draw[line width=1.2pt,color=ccqqqq] (5.749997761755376,2.5583307370536756) -- (5.7649977506081385,2.555802075298842);
\draw[line width=1.2pt,color=ccqqqq] (5.7649977506081385,2.555802075298842) -- (5.779997739460901,2.5532869914695175);
\draw[line width=1.2pt,color=ccqqqq] (5.779997739460901,2.5532869914695175) -- (5.794997728313664,2.5507853882998024);
\draw[line width=1.2pt,color=ccqqqq] (5.794997728313664,2.5507853882998024) -- (5.809997717166427,2.548297169291934);
\draw[line width=1.2pt,color=ccqqqq] (5.809997717166427,2.548297169291934) -- (5.8249977060191895,2.545822238711702);
\draw[line width=1.2pt,color=ccqqqq] (5.8249977060191895,2.545822238711702) -- (5.839997694871952,2.5433605015838117);
\draw[line width=1.2pt,color=ccqqqq] (5.839997694871952,2.5433605015838117) -- (5.854997683724715,2.540911863687204);
\draw[line width=1.2pt,color=ccqqqq] (5.854997683724715,2.540911863687204) -- (5.869997672577478,2.5384762315503373);
\draw[line width=1.2pt,color=ccqqqq] (5.869997672577478,2.5384762315503373) -- (5.884997661430241,2.536053512446429);
\draw[line width=1.2pt,color=ccqqqq] (5.884997661430241,2.536053512446429) -- (5.899997650283003,2.533643614388664);
\draw[line width=1.2pt,color=ccqqqq] (5.899997650283003,2.533643614388664) -- (5.914997639135766,2.5312464461253694);
\draw[line width=1.2pt,color=ccqqqq] (5.914997639135766,2.5312464461253694) -- (5.929997627988529,2.528861917135165);
\draw[line width=1.2pt,color=ccqqqq] (5.929997627988529,2.528861917135165) -- (5.944997616841292,2.5264899376220864);
\draw[line width=1.2pt,color=ccqqqq] (5.944997616841292,2.5264899376220864) -- (5.9599976056940545,2.5241304185106825);
\draw[line width=1.2pt,color=ccqqqq] (5.9599976056940545,2.5241304185106825) -- (5.974997594546817,2.5217832714410995);
\draw[line width=1.2pt,color=ccqqqq] (5.974997594546817,2.5217832714410995) -- (5.98999758339958,2.5194484087641413);
\draw[line width=1.2pt,color=ccqqqq] (5.98999758339958,2.5194484087641413) -- (6.004997572252343,2.5171257435363166);
\draw [->,color=qqwuqq] (1.,0.) -- (1.,2.6598080383923213);
\draw [->,color=ffqqqq] (1.,2.6598080383923213) -- (0.,2.6598080383923213);
\draw [color=ffqqqq](5.18,3.46) node[anchor=north west] {$\mathscr{C}_f$};
\draw [color=qqwuqq](3.78,-0.02) node[anchor=north west] {$a$};
\draw [color=ffqqqq](-1.14,3.14) node[anchor=north west] {$f(a)$};
\begin{scriptsize}
\draw [color=xdxdff] (1.,0.)-- ++(-2.5pt,0 pt) -- ++(5.0pt,0 pt) ++(-2.5pt,-2.5pt) -- ++(0 pt,5.0pt);
\draw [color=uuuuuu] (1.,2.6598080383923213)-- ++(-2.5pt,0 pt) -- ++(5.0pt,0 pt) ++(-2.5pt,-2.5pt) -- ++(0 pt,5.0pt);
\draw [color=uuuuuu] (0.,2.6598080383923213)-- ++(-2.5pt,0 pt) -- ++(5.0pt,0 pt) ++(-2.5pt,-2.5pt) -- ++(0 pt,5.0pt);
\end{scriptsize}
\end{tikzpicture}

\MtV{https://www.geogebra.org/m/bxEUbV4M}{Lecture d'image}

\end{minipage}
\hfill
\begin{minipage}{0.48\linewidth}
\textbf{Lecture d'antécédent}

\definecolor{qqwuqq}{rgb}{0.,0.39215686274509803,0.}
\definecolor{xdxdff}{rgb}{0.49019607843137253,0.49019607843137253,1.}
\definecolor{ffqqqq}{rgb}{1.,0.,0.}
\definecolor{ccqqqq}{rgb}{0.8,0.,0.}
\begin{tikzpicture}[line cap=round,line join=round,>=triangle 45,x=1.0cm,y=1.0cm]
\draw[->,color=black] (-0.5,0.) -- (6.68,0.);
\foreach \x in {,1.,2.,3.,4.,5.,6.}
\draw[shift={(\x,0)},color=black] (0pt,2pt) -- (0pt,-2pt) node[below] {\footnotesize $\x$};
\draw[->,color=black] (0.,-0.46) -- (0.,3.9);
\foreach \y in {,1.,2.,3.}
\draw[shift={(0,\y)},color=black] (2pt,0pt) -- (-2pt,0pt) node[left] {\footnotesize $\y$};
\draw[color=black] (0pt,-10pt) node[right] {\footnotesize $0$};
\clip(-0.5,-0.46) rectangle (6.68,3.9);
\draw[line width=1.2pt,color=ccqqqq] (0.040000619999999994,1.0133338844447577) -- (0.040000619999999994,1.0133338844447577);
\draw[line width=1.2pt,color=ccqqqq] (0.040000619999999994,1.0133338844447577) -- (0.055000610999398326,1.0268513859465929);
\draw[line width=1.2pt,color=ccqqqq] (0.055000610999398326,1.0268513859465929) -- (0.07000060199879667,1.0407404744868902);
\draw[line width=1.2pt,color=ccqqqq] (0.07000060199879667,1.0407404744868902) -- (0.085000592998195,1.055005937142754);
\draw[line width=1.2pt,color=ccqqqq] (0.085000592998195,1.055005937142754) -- (0.10000058399759333,1.0696524018173492);
\draw[line width=1.2pt,color=ccqqqq] (0.10000058399759333,1.0696524018173492) -- (0.11500057499699166,1.0846843218995574);
\draw[line width=1.2pt,color=ccqqqq] (0.11500057499699166,1.0846843218995574) -- (0.13000056599639,1.1001059604077321);
\draw[line width=1.2pt,color=ccqqqq] (0.13000056599639,1.1001059604077321) -- (0.14500055699578834,1.1159213736343903);
\draw[line width=1.2pt,color=ccqqqq] (0.14500055699578834,1.1159213736343903) -- (0.16000054799518668,1.1321343943123316);
\draw[line width=1.2pt,color=ccqqqq] (0.16000054799518668,1.1321343943123316) -- (0.175000538994585,1.1487486143265166);
\draw[line width=1.2pt,color=ccqqqq] (0.175000538994585,1.1487486143265166) -- (0.19000052999398334,1.165767367000086);
\draw[line width=1.2pt,color=ccqqqq] (0.19000052999398334,1.165767367000086) -- (0.20500052099338167,1.1831937089871636);
\draw[line width=1.2pt,color=ccqqqq] (0.20500052099338167,1.1831937089871636) -- (0.22000051199278,1.2010304018095062);
\draw[line width=1.2pt,color=ccqqqq] (0.22000051199278,1.2010304018095062) -- (0.23500050299217834,1.2192798930786783);
\draw[line width=1.2pt,color=ccqqqq] (0.23500050299217834,1.2192798930786783) -- (0.2500004939915767,1.237944297450166);
\draw[line width=1.2pt,color=ccqqqq] (0.2500004939915767,1.237944297450166) -- (0.26500048499097506,1.2570253773607283);
\draw[line width=1.2pt,color=ccqqqq] (0.26500048499097506,1.2570253773607283) -- (0.2800004759903734,1.2765245236052576);
\draw[line width=1.2pt,color=ccqqqq] (0.2800004759903734,1.2765245236052576) -- (0.2950004669897718,1.2964427358144694);
\draw[line width=1.2pt,color=ccqqqq] (0.2950004669897718,1.2964427358144694) -- (0.31000045798917014,1.3167806028998197);
\draw[line width=1.2pt,color=ccqqqq] (0.31000045798917014,1.3167806028998197) -- (0.3250004489885685,1.337538283537137);
\draw[line width=1.2pt,color=ccqqqq] (0.3250004489885685,1.337538283537137) -- (0.34000043998796686,1.3587154867654947);
\draw[line width=1.2pt,color=ccqqqq] (0.34000043998796686,1.3587154867654947) -- (0.3550004309873652,1.380311452782799);
\draw[line width=1.2pt,color=ccqqqq] (0.3550004309873652,1.380311452782799) -- (0.3700004219867636,1.4023249340243915);
\draw[line width=1.2pt,color=ccqqqq] (0.3700004219867636,1.4023249340243915) -- (0.38500041298616194,1.4247541766155845);
\draw[line width=1.2pt,color=ccqqqq] (0.38500041298616194,1.4247541766155845) -- (0.4000004039855603,1.4475969022934447);
\draw[line width=1.2pt,color=ccqqqq] (0.4000004039855603,1.4475969022934447) -- (0.41500039498495866,1.470850290897205);
\draw[line width=1.2pt,color=ccqqqq] (0.41500039498495866,1.470850290897205) -- (0.430000385984357,1.4945109635304155);
\draw[line width=1.2pt,color=ccqqqq] (0.430000385984357,1.4945109635304155) -- (0.4450003769837554,1.5185749665012274);
\draw[line width=1.2pt,color=ccqqqq] (0.4450003769837554,1.5185749665012274) -- (0.46000036798315375,1.543037756150003);
\draw[line width=1.2pt,color=ccqqqq] (0.46000036798315375,1.543037756150003) -- (0.4750003589825521,1.5678941846756855);
\draw[line width=1.2pt,color=ccqqqq] (0.4750003589825521,1.5678941846756855) -- (0.49000034998195047,1.5931384870739869);
\draw[line width=1.2pt,color=ccqqqq] (0.49000034998195047,1.5931384870739869) -- (0.5050003409813488,1.6187642693013782);
\draw[line width=1.2pt,color=ccqqqq] (0.5050003409813488,1.6187642693013782) -- (0.5200003319807471,1.6447644977790625);
\draw[line width=1.2pt,color=ccqqqq] (0.5200003319807471,1.6447644977790625) -- (0.5350003229801454,1.671131490350478);
\draw[line width=1.2pt,color=ccqqqq] (0.5350003229801454,1.671131490350478) -- (0.5500003139795437,1.6978569088044029);
\draw[line width=1.2pt,color=ccqqqq] (0.5500003139795437,1.6978569088044029) -- (0.565000304978942,1.7249317530733346);
\draw[line width=1.2pt,color=ccqqqq] (0.565000304978942,1.7249317530733346) -- (0.5800002959783404,1.7523463572134532);
\draw[line width=1.2pt,color=ccqqqq] (0.5800002959783404,1.7523463572134532) -- (0.5950002869777387,1.7800903872681388);
\draw[line width=1.2pt,color=ccqqqq] (0.5950002869777387,1.7800903872681388) -- (0.610000277977137,1.8081528411116294);
\draw[line width=1.2pt,color=ccqqqq] (0.610000277977137,1.8081528411116294) -- (0.6250002689765353,1.8365220503629962);
\draw[line width=1.2pt,color=ccqqqq] (0.6250002689765353,1.8365220503629962) -- (0.6400002599759336,1.8651856844531383);
\draw[line width=1.2pt,color=ccqqqq] (0.6400002599759336,1.8651856844531383) -- (0.6550002509753319,1.8941307569189902);
\draw[line width=1.2pt,color=ccqqqq] (0.6550002509753319,1.8941307569189902) -- (0.6700002419747302,1.92334363398958);
\draw[line width=1.2pt,color=ccqqqq] (0.6700002419747302,1.92334363398958) -- (0.6850002329741285,1.9528100455180224);
\draw[line width=1.2pt,color=ccqqqq] (0.6850002329741285,1.9528100455180224) -- (0.7000002239735268,1.9825150983020163);
\draw[line width=1.2pt,color=ccqqqq] (0.7000002239735268,1.9825150983020163) -- (0.7150002149729251,2.0124432918229904);
\draw[line width=1.2pt,color=ccqqqq] (0.7150002149729251,2.0124432918229904) -- (0.7300002059723234,2.042578536420775);
\draw[line width=1.2pt,color=ccqqqq] (0.7300002059723234,2.042578536420775) -- (0.7450001969717217,2.0729041739066703);
\draw[line width=1.2pt,color=ccqqqq] (0.7450001969717217,2.0729041739066703) -- (0.76000018797112,2.103403000603101);
\draw[line width=1.2pt,color=ccqqqq] (0.76000018797112,2.103403000603101) -- (0.7750001789705183,2.1340572927828516);
\draw[line width=1.2pt,color=ccqqqq] (0.7750001789705183,2.1340572927828516) -- (0.7900001699699166,2.164848834465217);
\draw[line width=1.2pt,color=ccqqqq] (0.7900001699699166,2.164848834465217) -- (0.8050001609693149,2.1957589475105097);
\draw[line width=1.2pt,color=ccqqqq] (0.8050001609693149,2.1957589475105097) -- (0.8200001519687132,2.2267685239382895);
\draw[line width=1.2pt,color=ccqqqq] (0.8200001519687132,2.2267685239382895) -- (0.8350001429681115,2.2578580603786422);
\draw[line width=1.2pt,color=ccqqqq] (0.8350001429681115,2.2578580603786422) -- (0.8500001339675098,2.2890076945499636);
\draw[line width=1.2pt,color=ccqqqq] (0.8500001339675098,2.2890076945499636) -- (0.8650001249669081,2.320197243641163);
\draw[line width=1.2pt,color=ccqqqq] (0.8650001249669081,2.320197243641163) -- (0.8800001159663065,2.3514062444611854);
\draw[line width=1.2pt,color=ccqqqq] (0.8800001159663065,2.3514062444611854) -- (0.8950001069657048,2.3826139952043817);
\draw[line width=1.2pt,color=ccqqqq] (0.8950001069657048,2.3826139952043817) -- (0.9100000979651031,2.4137995986667464);
\draw[line width=1.2pt,color=ccqqqq] (0.9100000979651031,2.4137995986667464) -- (0.9250000889645014,2.44494200673552);
\draw[line width=1.2pt,color=ccqqqq] (0.9250000889645014,2.44494200673552) -- (0.9400000799638997,2.476020065963267);
\draw[line width=1.2pt,color=ccqqqq] (0.9400000799638997,2.476020065963267) -- (0.955000070963298,2.507012564027485);
\draw[line width=1.2pt,color=ccqqqq] (0.955000070963298,2.507012564027485) -- (0.9700000619626963,2.5378982768681024);
\draw[line width=1.2pt,color=ccqqqq] (0.9700000619626963,2.5378982768681024) -- (0.9850000529620946,2.5686560162881316);
\draw[line width=1.2pt,color=ccqqqq] (0.9850000529620946,2.5686560162881316) -- (1.000000043961493,2.599264677797237);
\draw[line width=1.2pt,color=ccqqqq] (1.000000043961493,2.599264677797237) -- (1.0150000349608912,2.6297032884741953);
\draw[line width=1.2pt,color=ccqqqq] (1.0150000349608912,2.6297032884741953) -- (1.0300000259602895,2.6599510546222453);
\draw[line width=1.2pt,color=ccqqqq] (1.0300000259602895,2.6599510546222453) -- (1.0450000169596878,2.6899874089911178);
\draw[line width=1.2pt,color=ccqqqq] (1.0450000169596878,2.6899874089911178) -- (1.060000007959086,2.719792057341188);
\draw[line width=1.2pt,color=ccqqqq] (1.060000007959086,2.719792057341188) -- (1.0749999989584844,2.7493450241286705);
\draw[line width=1.2pt,color=ccqqqq] (1.0749999989584844,2.7493450241286705) -- (1.0899999899578827,2.778626697096022);
\draw[line width=1.2pt,color=ccqqqq] (1.0899999899578827,2.778626697096022) -- (1.104999980957281,2.8076178705587456);
\draw[line width=1.2pt,color=ccqqqq] (1.104999980957281,2.8076178705587456) -- (1.1199999719566793,2.836299787188474);
\draw[line width=1.2pt,color=ccqqqq] (1.1199999719566793,2.836299787188474) -- (1.1349999629560776,2.8646541781024695);
\draw[line width=1.2pt,color=ccqqqq] (1.1349999629560776,2.8646541781024695) -- (1.149999953955476,2.8926633010814236);
\draw[line width=1.2pt,color=ccqqqq] (1.149999953955476,2.8926633010814236) -- (1.1649999449548742,2.9203099767505103);
\draw[line width=1.2pt,color=ccqqqq] (1.1649999449548742,2.9203099767505103) -- (1.1799999359542725,2.947577622572938);
\draw[line width=1.2pt,color=ccqqqq] (1.1799999359542725,2.947577622572938) -- (1.1949999269536709,2.9744502845205356);
\draw[line width=1.2pt,color=ccqqqq] (1.1949999269536709,2.9744502845205356) -- (1.2099999179530692,3.0009126663021295);
\draw[line width=1.2pt,color=ccqqqq] (1.2099999179530692,3.0009126663021295) -- (1.2249999089524675,3.0269501560473198);
\draw[line width=1.2pt,color=ccqqqq] (1.2249999089524675,3.0269501560473198) -- (1.2399998999518658,3.052548850360701);
\draw[line width=1.2pt,color=ccqqqq] (1.2399998999518658,3.052548850360701) -- (1.254999890951264,3.077695575679274);
\draw[line width=1.2pt,color=ccqqqq] (1.254999890951264,3.077695575679274) -- (1.2699998819506624,3.102377906883717);
\draw[line width=1.2pt,color=ccqqqq] (1.2699998819506624,3.102377906883717) -- (1.2849998729500607,3.1265841831320254);
\draw[line width=1.2pt,color=ccqqqq] (1.2849998729500607,3.1265841831320254) -- (1.299999863949459,3.150303520901681);
\draw[line width=1.2pt,color=ccqqqq] (1.299999863949459,3.150303520901681) -- (1.3149998549488573,3.1735258242438267);
\draw[line width=1.2pt,color=ccqqqq] (1.3149998549488573,3.1735258242438267) -- (1.3299998459482556,3.1962417922696646);
\draw[line width=1.2pt,color=ccqqqq] (1.3299998459482556,3.1962417922696646) -- (1.344999836947654,3.21844292390539);
\draw[line width=1.2pt,color=ccqqqq] (1.344999836947654,3.21844292390539) -- (1.3599998279470522,3.2401215199672784);
\draw[line width=1.2pt,color=ccqqqq] (1.3599998279470522,3.2401215199672784) -- (1.3749998189464505,3.261270682622911);
\draw[line width=1.2pt,color=ccqqqq] (1.3749998189464505,3.261270682622911) -- (1.3899998099458488,3.2818843123178665);
\draw[line width=1.2pt,color=ccqqqq] (1.3899998099458488,3.2818843123178665) -- (1.4049998009452471,3.301957102259472);
\draw[line width=1.2pt,color=ccqqqq] (1.4049998009452471,3.301957102259472) -- (1.4199997919446454,3.3214845305602694);
\draw[line width=1.2pt,color=ccqqqq] (1.4199997919446454,3.3214845305602694) -- (1.4349997829440437,3.340462850153706);
\draw[line width=1.2pt,color=ccqqqq] (1.4349997829440437,3.340462850153706) -- (1.449999773943442,3.3588890766031403);
\draw[line width=1.2pt,color=ccqqqq] (1.449999773943442,3.3588890766031403) -- (1.4649997649428403,3.37676097393257);
\draw[line width=1.2pt,color=ccqqqq] (1.4649997649428403,3.37676097393257) -- (1.4799997559422386,3.3940770386135126);
\draw[line width=1.2pt,color=ccqqqq] (1.4799997559422386,3.3940770386135126) -- (1.494999746941637,3.4108364818472126);
\draw[line width=1.2pt,color=ccqqqq] (1.494999746941637,3.4108364818472126) -- (1.5099997379410353,3.427039210284871);
\draw[line width=1.2pt,color=ccqqqq] (1.5099997379410353,3.427039210284871) -- (1.5249997289404336,3.442685805330859);
\draw[line width=1.2pt,color=ccqqqq] (1.5249997289404336,3.442685805330859) -- (1.5399997199398319,3.4577775011750465);
\draw[line width=1.2pt,color=ccqqqq] (1.5399997199398319,3.4577775011750465) -- (1.5549997109392302,3.4723161617003644);
\draw[line width=1.2pt,color=ccqqqq] (1.5549997109392302,3.4723161617003644) -- (1.5699997019386285,3.4863042564107363);
\draw[line width=1.2pt,color=ccqqqq] (1.5699997019386285,3.4863042564107363) -- (1.5849996929380268,3.499744835522502);
\draw[line width=1.2pt,color=ccqqqq] (1.5849996929380268,3.499744835522502) -- (1.599999683937425,3.5126415043596286);
\draw[line width=1.2pt,color=ccqqqq] (1.599999683937425,3.5126415043596286) -- (1.6149996749368234,3.524998397189292);
\draw[line width=1.2pt,color=ccqqqq] (1.6149996749368234,3.524998397189292) -- (1.6299996659362217,3.536820150630032);
\draw[line width=1.2pt,color=ccqqqq] (1.6299996659362217,3.536820150630032) -- (1.64499965693562,3.548111876759653);
\draw[line width=1.2pt,color=ccqqqq] (1.64499965693562,3.548111876759653) -- (1.6599996479350183,3.558879136044456);
\draw[line width=1.2pt,color=ccqqqq] (1.6599996479350183,3.558879136044456) -- (1.6749996389344166,3.569127910205345);
\draw[line width=1.2pt,color=ccqqqq] (1.6749996389344166,3.569127910205345) -- (1.689999629933815,3.5788645751299235);
\draw[line width=1.2pt,color=ccqqqq] (1.689999629933815,3.5788645751299235) -- (1.7049996209332132,3.588095873932973);
\draw[line width=1.2pt,color=ccqqqq] (1.7049996209332132,3.588095873932973) -- (1.7199996119326115,3.5968288902607624);
\draw[line width=1.2pt,color=ccqqqq] (1.7199996119326115,3.5968288902607624) -- (1.7349996029320098,3.6050710219275315);
\draw[line width=1.2pt,color=ccqqqq] (1.7349996029320098,3.6050710219275315) -- (1.7499995939314081,3.6128299549653358);
\draw[line width=1.2pt,color=ccqqqq] (1.7499995939314081,3.6128299549653358) -- (1.7649995849308064,3.620113638161218);
\draw[line width=1.2pt,color=ccqqqq] (1.7649995849308064,3.620113638161218) -- (1.7799995759302047,3.626930258148554);
\draw[line width=1.2pt,color=ccqqqq] (1.7799995759302047,3.626930258148554) -- (1.794999566929603,3.6332882151123487);
\draw[line width=1.2pt,color=ccqqqq] (1.794999566929603,3.6332882151123487) -- (1.8099995579290014,3.6391960991613397);
\draw[line width=1.2pt,color=ccqqqq] (1.8099995579290014,3.6391960991613397) -- (1.8249995489283997,3.6446626674130647);
\draw[line width=1.2pt,color=ccqqqq] (1.8249995489283997,3.6446626674130647) -- (1.839999539927798,3.6496968218315082);
\draw[line width=1.2pt,color=ccqqqq] (1.839999539927798,3.6496968218315082) -- (1.8549995309271963,3.6543075878507123);
\draw[line width=1.2pt,color=ccqqqq] (1.8549995309271963,3.6543075878507123) -- (1.8699995219265946,3.65850409381176);
\draw[line width=1.2pt,color=ccqqqq] (1.8699995219265946,3.65850409381176) -- (1.8849995129259929,3.6622955512348403);
\draw[line width=1.2pt,color=ccqqqq] (1.8849995129259929,3.6622955512348403) -- (1.8999995039253912,3.6656912359427656);
\draw[line width=1.2pt,color=ccqqqq] (1.8999995039253912,3.6656912359427656) -- (1.9149994949247895,3.668700470047255);
\draw[line width=1.2pt,color=ccqqqq] (1.9149994949247895,3.668700470047255) -- (1.9299994859241878,3.6713326048046175);
\draw[line width=1.2pt,color=ccqqqq] (1.9299994859241878,3.6713326048046175) -- (1.944999476923586,3.6735970043430846);
\draw[line width=1.2pt,color=ccqqqq] (1.944999476923586,3.6735970043430846) -- (1.9599994679229844,3.6755030302600336);
\draw[line width=1.2pt,color=ccqqqq] (1.9599994679229844,3.6755030302600336) -- (1.9749994589223827,3.677060027083657);
\draw[line width=1.2pt,color=ccqqqq] (1.9749994589223827,3.677060027083657) -- (1.989999449921781,3.6782773085902782);
\draw[line width=1.2pt,color=ccqqqq] (1.989999449921781,3.6782773085902782) -- (2.0049994409211793,3.6791641449654784);
\draw[line width=1.2pt,color=ccqqqq] (2.0049994409211793,3.6791641449654784) -- (2.019999431920578,3.67972975079452);
\draw[line width=1.2pt,color=ccqqqq] (2.019999431920578,3.67972975079452) -- (2.0349994229199764,3.67998327386511);
\draw[line width=1.2pt,color=ccqqqq] (2.0349994229199764,3.67998327386511) -- (2.049999413919375,3.6799337847634668);
\draw[line width=1.2pt,color=ccqqqq] (2.049999413919375,3.6799337847634668) -- (2.0649994049187734,3.6795902672428022);
\draw[line width=1.2pt,color=ccqqqq] (2.0649994049187734,3.6795902672428022) -- (2.079999395918172,3.678961609341782);
\draw[line width=1.2pt,color=ccqqqq] (2.079999395918172,3.678961609341782) -- (2.0949993869175705,3.6780565952292026);
\draw[line width=1.2pt,color=ccqqqq] (2.0949993869175705,3.6780565952292026) -- (2.109999377916969,3.6768838977500526);
\draw[line width=1.2pt,color=ccqqqq] (2.109999377916969,3.6768838977500526) -- (2.1249993689163675,3.6754520716472623);
\draw[line width=1.2pt,color=ccqqqq] (2.1249993689163675,3.6754520716472623) -- (2.139999359915766,3.6737695474327863);
\draw[line width=1.2pt,color=ccqqqq] (2.139999359915766,3.6737695474327863) -- (2.1549993509151646,3.671844625881218);
\draw[line width=1.2pt,color=ccqqqq] (2.1549993509151646,3.671844625881218) -- (2.169999341914563,3.6696854731188155);
\draw[line width=1.2pt,color=ccqqqq] (2.169999341914563,3.6696854731188155) -- (2.1849993329139616,3.667300116280708);
\draw[line width=1.2pt,color=ccqqqq] (2.1849993329139616,3.667300116280708) -- (2.19999932391336,3.6646964397090462);
\draw[line width=1.2pt,color=ccqqqq] (2.19999932391336,3.6646964397090462) -- (2.2149993149127587,3.6618821816650073);
\draw[line width=1.2pt,color=ccqqqq] (2.2149993149127587,3.6618821816650073) -- (2.2299993059121572,3.658864931527816);
\draw[line width=1.2pt,color=ccqqqq] (2.2299993059121572,3.658864931527816) -- (2.2449992969115558,3.6556521274543092);
\draw[line width=1.2pt,color=ccqqqq] (2.2449992969115558,3.6556521274543092) -- (2.2599992879109543,3.652251054473009);
\draw[line width=1.2pt,color=ccqqqq] (2.2599992879109543,3.652251054473009) -- (2.274999278910353,3.6486688429872105);
\draw[line width=1.2pt,color=ccqqqq] (2.274999278910353,3.6486688429872105) -- (2.2899992699097513,3.644912467662186);
\draw[line width=1.2pt,color=ccqqqq] (2.2899992699097513,3.644912467662186) -- (2.30499926090915,3.6409887466722477);
\draw[line width=1.2pt,color=ccqqqq] (2.30499926090915,3.6409887466722477) -- (2.3199992519085484,3.636904341284143);
\draw[line width=1.2pt,color=ccqqqq] (2.3199992519085484,3.636904341284143) -- (2.334999242907947,3.63266575575398);
\draw[line width=1.2pt,color=ccqqqq] (2.334999242907947,3.63266575575398) -- (2.3499992339073454,3.6282793375156643);
\draw[line width=1.2pt,color=ccqqqq] (2.3499992339073454,3.6282793375156643) -- (2.364999224906744,3.6237512776396343);
\draw[line width=1.2pt,color=ccqqqq] (2.364999224906744,3.6237512776396343) -- (2.3799992159061425,3.619087611541494);
\draw[line width=1.2pt,color=ccqqqq] (2.3799992159061425,3.619087611541494) -- (2.394999206905541,3.6142942199209864);
\draw[line width=1.2pt,color=ccqqqq] (2.394999206905541,3.6142942199209864) -- (2.4099991979049395,3.6093768299125717);
\draw[line width=1.2pt,color=ccqqqq] (2.4099991979049395,3.6093768299125717) -- (2.424999188904338,3.6043410164297276);
\draw[line width=1.2pt,color=ccqqqq] (2.424999188904338,3.6043410164297276) -- (2.4399991799037366,3.5991922036859214);
\draw[line width=1.2pt,color=ccqqqq] (2.4399991799037366,3.5991922036859214) -- (2.454999170903135,3.5939356668760345);
\draw[line width=1.2pt,color=ccqqqq] (2.454999170903135,3.5939356668760345) -- (2.4699991619025337,3.5885765340028244);
\draw[line width=1.2pt,color=ccqqqq] (2.4699991619025337,3.5885765340028244) -- (2.484999152901932,3.5831197878338434);
\draw[line width=1.2pt,color=ccqqqq] (2.484999152901932,3.5831197878338434) -- (2.4999991439013307,3.577570267974994);
\draw[line width=1.2pt,color=ccqqqq] (2.4999991439013307,3.577570267974994) -- (2.5149991349007292,3.5719326730476877);
\draw[line width=1.2pt,color=ccqqqq] (2.5149991349007292,3.5719326730476877) -- (2.5299991259001278,3.566211562957319);
\draw[line width=1.2pt,color=ccqqqq] (2.5299991259001278,3.566211562957319) -- (2.5449991168995263,3.5604113612415054);
\draw[line width=1.2pt,color=ccqqqq] (2.5449991168995263,3.5604113612415054) -- (2.559999107898925,3.554536357487227);
\draw[line width=1.2pt,color=ccqqqq] (2.559999107898925,3.554536357487227) -- (2.5749990988983233,3.5485907098067146);
\draw[line width=1.2pt,color=ccqqqq] (2.5749990988983233,3.5485907098067146) -- (2.589999089897722,3.5425784473625663);
\draw[line width=1.2pt,color=ccqqqq] (2.589999089897722,3.5425784473625663) -- (2.6049990808971204,3.536503472933213);
\draw[line width=1.2pt,color=ccqqqq] (2.6049990808971204,3.536503472933213) -- (2.619999071896519,3.5303695655104743);
\draw[line width=1.2pt,color=ccqqqq] (2.619999071896519,3.5303695655104743) -- (2.6349990628959175,3.5241803829214957);
\draw[line width=1.2pt,color=ccqqqq] (2.6349990628959175,3.5241803829214957) -- (2.649999053895316,3.5179394644679594);
\draw[line width=1.2pt,color=ccqqqq] (2.649999053895316,3.5179394644679594) -- (2.6649990448947145,3.5116502335759407);
\draw[line width=1.2pt,color=ccqqqq] (2.6649990448947145,3.5116502335759407) -- (2.679999035894113,3.505316000450331);
\draw[line width=1.2pt,color=ccqqqq] (2.679999035894113,3.505316000450331) -- (2.6949990268935116,3.498939964728204);
\draw[line width=1.2pt,color=ccqqqq] (2.6949990268935116,3.498939964728204) -- (2.70999901789291,3.4925252181259596);
\draw[line width=1.2pt,color=ccqqqq] (2.70999901789291,3.4925252181259596) -- (2.7249990088923086,3.486074747075528);
\draw[line width=1.2pt,color=ccqqqq] (2.7249990088923086,3.486074747075528) -- (2.739998999891707,3.4795914353452986);
\draw[line width=1.2pt,color=ccqqqq] (2.739998999891707,3.4795914353452986) -- (2.7549989908911057,3.473078066641858);
\draw[line width=1.2pt,color=ccqqqq] (2.7549989908911057,3.473078066641858) -- (2.769998981890504,3.4665373271889486);
\draw[line width=1.2pt,color=ccqqqq] (2.769998981890504,3.4665373271889486) -- (2.7849989728899027,3.4599718082804394);
\draw[line width=1.2pt,color=ccqqqq] (2.7849989728899027,3.4599718082804394) -- (2.7999989638893013,3.4533840088043886);
\draw[line width=1.2pt,color=ccqqqq] (2.7999989638893013,3.4533840088043886) -- (2.8149989548887,3.446776337735608);
\draw[line width=1.2pt,color=ccqqqq] (2.8149989548887,3.446776337735608) -- (2.8299989458880983,3.4401511165944116);
\draw[line width=1.2pt,color=ccqqqq] (2.8299989458880983,3.4401511165944116) -- (2.844998936887497,3.4335105818694815);
\draw[line width=1.2pt,color=ccqqqq] (2.844998936887497,3.4335105818694815) -- (2.8599989278868954,3.426856887403072);
\draw[line width=1.2pt,color=ccqqqq] (2.8599989278868954,3.426856887403072) -- (2.874998918886294,3.420192106736957);
\draw[line width=1.2pt,color=ccqqqq] (2.874998918886294,3.420192106736957) -- (2.8899989098856924,3.413518235417782);
\draw[line width=1.2pt,color=ccqqqq] (2.8899989098856924,3.413518235417782) -- (2.904998900885091,3.406837193260649);
\draw[line width=1.2pt,color=ccqqqq] (2.904998900885091,3.406837193260649) -- (2.9199988918844895,3.4001508265699796);
\draw[line width=1.2pt,color=ccqqqq] (2.9199988918844895,3.4001508265699796) -- (2.934998882883888,3.3934609103168505);
\draw[line width=1.2pt,color=ccqqqq] (2.934998882883888,3.3934609103168505) -- (2.9499988738832865,3.3867691502721655);
\draw[line width=1.2pt,color=ccqqqq] (2.9499988738832865,3.3867691502721655) -- (2.964998864882685,3.380077185095181);
\draw[line width=1.2pt,color=ccqqqq] (2.964998864882685,3.380077185095181) -- (2.9799988558820836,3.373386588377019);
\draw[line width=1.2pt,color=ccqqqq] (2.9799988558820836,3.373386588377019) -- (2.994998846881482,3.3666988706389547);
\draw[line width=1.2pt,color=ccqqqq] (2.994998846881482,3.3666988706389547) -- (3.0099988378808806,3.3600154812853495);
\draw[line width=1.2pt,color=ccqqqq] (3.0099988378808806,3.3600154812853495) -- (3.024998828880279,3.353337810511233);
\draw[line width=1.2pt,color=ccqqqq] (3.024998828880279,3.353337810511233) -- (3.0399988198796777,3.3466671911646135);
\draw[line width=1.2pt,color=ccqqqq] (3.0399988198796777,3.3466671911646135) -- (3.054998810879076,3.3400049005636916);
\draw[line width=1.2pt,color=ccqqqq] (3.054998810879076,3.3400049005636916) -- (3.0699988018784747,3.3333521622692404);
\draw[line width=1.2pt,color=ccqqqq] (3.0699988018784747,3.3333521622692404) -- (3.0849987928778733,3.3267101478124625);
\draw[line width=1.2pt,color=ccqqqq] (3.0849987928778733,3.3267101478124625) -- (3.099998783877272,3.32007997837873);
\draw[line width=1.2pt,color=ccqqqq] (3.099998783877272,3.32007997837873) -- (3.1149987748766703,3.31346272644764);
\draw[line width=1.2pt,color=ccqqqq] (3.1149987748766703,3.31346272644764) -- (3.129998765876069,3.306859417389906);
\draw[line width=1.2pt,color=ccqqqq] (3.129998765876069,3.306859417389906) -- (3.1449987568754674,3.3002710310216115);
\draw[line width=1.2pt,color=ccqqqq] (3.1449987568754674,3.3002710310216115) -- (3.159998747874866,3.2936985031164294);
\draw[line width=1.2pt,color=ccqqqq] (3.159998747874866,3.2936985031164294) -- (3.1749987388742644,3.2871427268764277);
\draw[line width=1.2pt,color=ccqqqq] (3.1749987388742644,3.2871427268764277) -- (3.189998729873663,3.2806045543621094);
\draw[line width=1.2pt,color=ccqqqq] (3.189998729873663,3.2806045543621094) -- (3.2049987208730615,3.274084797882382);
\draw[line width=1.2pt,color=ccqqqq] (3.2049987208730615,3.274084797882382) -- (3.21999871187246,3.2675842313451566);
\draw[line width=1.2pt,color=ccqqqq] (3.21999871187246,3.2675842313451566) -- (3.2349987028718585,3.2611035915692996);
\draw[line width=1.2pt,color=ccqqqq] (3.2349987028718585,3.2611035915692996) -- (3.249998693871257,3.2546435795586843);
\draw[line width=1.2pt,color=ccqqqq] (3.249998693871257,3.2546435795586843) -- (3.2649986848706556,3.2482048617391013);
\draw[line width=1.2pt,color=ccqqqq] (3.2649986848706556,3.2482048617391013) -- (3.279998675870054,3.2417880711587816);
\draw[line width=1.2pt,color=ccqqqq] (3.279998675870054,3.2417880711587816) -- (3.2949986668694526,3.235393808653318);
\draw[line width=1.2pt,color=ccqqqq] (3.2949986668694526,3.235393808653318) -- (3.309998657868851,3.2290226439757594);
\draw[line width=1.2pt,color=ccqqqq] (3.309998657868851,3.2290226439757594) -- (3.3249986488682497,3.222675116892658);
\draw[line width=1.2pt,color=ccqqqq] (3.3249986488682497,3.222675116892658) -- (3.3399986398676482,3.2163517382468565);
\draw[line width=1.2pt,color=ccqqqq] (3.3399986398676482,3.2163517382468565) -- (3.3549986308670467,3.210052990987796);
\draw[line width=1.2pt,color=ccqqqq] (3.3549986308670467,3.210052990987796) -- (3.3699986218664453,3.203779331170124);
\draw[line width=1.2pt,color=ccqqqq] (3.3699986218664453,3.203779331170124) -- (3.384998612865844,3.197531188921384);
\draw[line width=1.2pt,color=ccqqqq] (3.384998612865844,3.197531188921384) -- (3.3999986038652423,3.191308969379552);
\draw[line width=1.2pt,color=ccqqqq] (3.3999986038652423,3.191308969379552) -- (3.414998594864641,3.185113053601185);
\draw[line width=1.2pt,color=ccqqqq] (3.414998594864641,3.185113053601185) -- (3.4299985858640394,3.178943799440948);
\draw[line width=1.2pt,color=ccqqqq] (3.4299985858640394,3.178943799440948) -- (3.444998576863438,3.172801542403257);
\draw[line width=1.2pt,color=ccqqqq] (3.444998576863438,3.172801542403257) -- (3.4599985678628364,3.1666865964667865);
\draw[line width=1.2pt,color=ccqqqq] (3.4599985678628364,3.1666865964667865) -- (3.474998558862235,3.1605992548825634);
\draw[line width=1.2pt,color=ccqqqq] (3.474998558862235,3.1605992548825634) -- (3.4899985498616335,3.1545397909463766);
\draw[line width=1.2pt,color=ccqqqq] (3.4899985498616335,3.1545397909463766) -- (3.504998540861032,3.1485084587462007);
\draw[line width=1.2pt,color=ccqqqq] (3.504998540861032,3.1485084587462007) -- (3.5199985318604305,3.142505493885332);
\draw[line width=1.2pt,color=ccqqqq] (3.5199985318604305,3.142505493885332) -- (3.534998522859829,3.1365311141819294);
\draw[line width=1.2pt,color=ccqqqq] (3.534998522859829,3.1365311141819294) -- (3.5499985138592276,3.1305855203456128);
\draw[line width=1.2pt,color=ccqqqq] (3.5499985138592276,3.1305855203456128) -- (3.564998504858626,3.1246688966317953);
\draw[line width=1.2pt,color=ccqqqq] (3.564998504858626,3.1246688966317953) -- (3.5799984958580247,3.1187814114743877);
\draw[line width=1.2pt,color=ccqqqq] (3.5799984958580247,3.1187814114743877) -- (3.594998486857423,3.1129232180975017);
\draw[line width=1.2pt,color=ccqqqq] (3.594998486857423,3.1129232180975017) -- (3.6099984778568217,3.107094455106779);
\draw[line width=1.2pt,color=ccqqqq] (3.6099984778568217,3.107094455106779) -- (3.6249984688562202,3.101295247060942);
\draw[line width=1.2pt,color=ccqqqq] (3.6249984688562202,3.101295247060942) -- (3.6399984598556188,3.0955257050241682);
\draw[line width=1.2pt,color=ccqqqq] (3.6399984598556188,3.0955257050241682) -- (3.6549984508550173,3.089785927099854);
\draw[line width=1.2pt,color=ccqqqq] (3.6549984508550173,3.089785927099854) -- (3.669998441854416,3.084075998946337);
\draw[line width=1.2pt,color=ccqqqq] (3.669998441854416,3.084075998946337) -- (3.6849984328538143,3.078395994275132);
\draw[line width=1.2pt,color=ccqqqq] (3.6849984328538143,3.078395994275132) -- (3.699998423853213,3.072745975332202);
\draw[line width=1.2pt,color=ccqqqq] (3.699998423853213,3.072745975332202) -- (3.7149984148526114,3.0671259933628114);
\draw[line width=1.2pt,color=ccqqqq] (3.7149984148526114,3.0671259933628114) -- (3.72999840585201,3.0615360890604433);
\draw[line width=1.2pt,color=ccqqqq] (3.72999840585201,3.0615360890604433) -- (3.7449983968514085,3.0559762930002985);
\draw[line width=1.2pt,color=ccqqqq] (3.7449983968514085,3.0559762930002985) -- (3.759998387850807,3.0504466260578527);
\draw[line width=1.2pt,color=ccqqqq] (3.759998387850807,3.0504466260578527) -- (3.7749983788502055,3.0449470998129398);
\draw[line width=1.2pt,color=ccqqqq] (3.7749983788502055,3.0449470998129398) -- (3.789998369849604,3.0394777169398237);
\draw[line width=1.2pt,color=ccqqqq] (3.789998369849604,3.0394777169398237) -- (3.8049983608490026,3.0340384715837);
\draw[line width=1.2pt,color=ccqqqq] (3.8049983608490026,3.0340384715837) -- (3.819998351848401,3.028629349724064);
\draw[line width=1.2pt,color=ccqqqq] (3.819998351848401,3.028629349724064) -- (3.8349983428477996,3.023250329525368);
\draw[line width=1.2pt,color=ccqqqq] (3.8349983428477996,3.023250329525368) -- (3.849998333847198,3.017901381675367);
\draw[line width=1.2pt,color=ccqqqq] (3.849998333847198,3.017901381675367) -- (3.8649983248465967,3.0125824697115644);
\draw[line width=1.2pt,color=ccqqqq] (3.8649983248465967,3.0125824697115644) -- (3.879998315845995,3.0072935503361298);
\draw[line width=1.2pt,color=ccqqqq] (3.879998315845995,3.0072935503361298) -- (3.8949983068453937,3.002034573719675);
\draw[line width=1.2pt,color=ccqqqq] (3.8949983068453937,3.002034573719675) -- (3.9099982978447922,2.9968054837942466);
\draw[line width=1.2pt,color=ccqqqq] (3.9099982978447922,2.9968054837942466) -- (3.9249982888441908,2.9916062185358863);
\draw[line width=1.2pt,color=ccqqqq] (3.9249982888441908,2.9916062185358863) -- (3.9399982798435893,2.9864367102371077);
\draw[line width=1.2pt,color=ccqqqq] (3.9399982798435893,2.9864367102371077) -- (3.954998270842988,2.9812968857696145);
\draw[line width=1.2pt,color=ccqqqq] (3.954998270842988,2.9812968857696145) -- (3.9699982618423864,2.9761866668375925);
\draw[line width=1.2pt,color=ccqqqq] (3.9699982618423864,2.9761866668375925) -- (3.984998252841785,2.971105970221873);
\draw[line width=1.2pt,color=ccqqqq] (3.984998252841785,2.971105970221873) -- (3.9999982438411834,2.96605470801529);
\draw[line width=1.2pt,color=ccqqqq] (3.9999982438411834,2.96605470801529) -- (4.014998234840582,2.961032787849504);
\draw[line width=1.2pt,color=ccqqqq] (4.014998234840582,2.961032787849504) -- (4.0299982258399805,2.956040113113594);
\draw[line width=1.2pt,color=ccqqqq] (4.0299982258399805,2.956040113113594) -- (4.044998216839379,2.951076583164686);
\draw[line width=1.2pt,color=ccqqqq] (4.044998216839379,2.951076583164686) -- (4.0599982078387775,2.9461420935308844);
\draw[line width=1.2pt,color=ccqqqq] (4.0599982078387775,2.9461420935308844) -- (4.074998198838176,2.941236536106767);
\draw[line width=1.2pt,color=ccqqqq] (4.074998198838176,2.941236536106767) -- (4.089998189837575,2.9363597993416963);
\draw[line width=1.2pt,color=ccqqqq] (4.089998189837575,2.9363597993416963) -- (4.104998180836973,2.9315117684211858);
\draw[line width=1.2pt,color=ccqqqq] (4.104998180836973,2.9315117684211858) -- (4.119998171836372,2.926692325441559);
\draw[line width=1.2pt,color=ccqqqq] (4.119998171836372,2.926692325441559) -- (4.13499816283577,2.9219013495781283);
\draw[line width=1.2pt,color=ccqqqq] (4.13499816283577,2.9219013495781283) -- (4.149998153835169,2.9171387172471146);
\draw[line width=1.2pt,color=ccqqqq] (4.149998153835169,2.9171387172471146) -- (4.164998144834567,2.9124043022615167);
\draw[line width=1.2pt,color=ccqqqq] (4.164998144834567,2.9124043022615167) -- (4.179998135833966,2.9076979759811463);
\draw[line width=1.2pt,color=ccqqqq] (4.179998135833966,2.9076979759811463) -- (4.194998126833364,2.9030196074570163);
\draw[line width=1.2pt,color=ccqqqq] (4.194998126833364,2.9030196074570163) -- (4.209998117832763,2.8983690635702857);
\draw[line width=1.2pt,color=ccqqqq] (4.209998117832763,2.8983690635702857) -- (4.224998108832161,2.8937462091659416);
\draw[line width=1.2pt,color=ccqqqq] (4.224998108832161,2.8937462091659416) -- (4.23999809983156,2.8891509071814045);
\draw[line width=1.2pt,color=ccqqqq] (4.23999809983156,2.8891509071814045) -- (4.254998090830958,2.884583018770224);
\draw[line width=1.2pt,color=ccqqqq] (4.254998090830958,2.884583018770224) -- (4.269998081830357,2.880042403421046);
\draw[line width=1.2pt,color=ccqqqq] (4.269998081830357,2.880042403421046) -- (4.284998072829755,2.875528919072);
\draw[line width=1.2pt,color=ccqqqq] (4.284998072829755,2.875528919072) -- (4.299998063829154,2.8710424222206816);
\draw[line width=1.2pt,color=ccqqqq] (4.299998063829154,2.8710424222206816) -- (4.3149980548285525,2.866582768029865);
\draw[line width=1.2pt,color=ccqqqq] (4.3149980548285525,2.866582768029865) -- (4.329998045827951,2.8621498104291123);
\draw[line width=1.2pt,color=ccqqqq] (4.329998045827951,2.8621498104291123) -- (4.3449980368273495,2.8577434022124075);
\draw[line width=1.2pt,color=ccqqqq] (4.3449980368273495,2.8577434022124075) -- (4.359998027826748,2.853363395131959);
\draw[line width=1.2pt,color=ccqqqq] (4.359998027826748,2.853363395131959) -- (4.374998018826147,2.849009639988309);
\draw[line width=1.2pt,color=ccqqqq] (4.374998018826147,2.849009639988309) -- (4.389998009825545,2.8446819867168713);
\draw[line width=1.2pt,color=ccqqqq] (4.389998009825545,2.8446819867168713) -- (4.404998000824944,2.840380284471026);
\draw[line width=1.2pt,color=ccqqqq] (4.404998000824944,2.840380284471026) -- (4.419997991824342,2.836104381701892);
\draw[line width=1.2pt,color=ccqqqq] (4.419997991824342,2.836104381701892) -- (4.434997982823741,2.831854126234897);
\draw[line width=1.2pt,color=ccqqqq] (4.434997982823741,2.831854126234897) -- (4.449997973823139,2.8276293653432454);
\draw[line width=1.2pt,color=ccqqqq] (4.449997973823139,2.8276293653432454) -- (4.464997964822538,2.8234299458184147);
\draw[line width=1.2pt,color=ccqqqq] (4.464997964822538,2.8234299458184147) -- (4.479997955821936,2.8192557140377614);
\draw[line width=1.2pt,color=ccqqqq] (4.479997955821936,2.8192557140377614) -- (4.494997946821335,2.8151065160293562);
\draw[line width=1.2pt,color=ccqqqq] (4.494997946821335,2.8151065160293562) -- (4.509997937820733,2.8109821975341402);
\draw[line width=1.2pt,color=ccqqqq] (4.509997937820733,2.8109821975341402) -- (4.524997928820132,2.806882604065495);
\draw[line width=1.2pt,color=ccqqqq] (4.524997928820132,2.806882604065495) -- (4.53999791981953,2.802807580966321);
\draw[line width=1.2pt,color=ccqqqq] (4.53999791981953,2.802807580966321) -- (4.554997910818929,2.798756973463715);
\draw[line width=1.2pt,color=ccqqqq] (4.554997910818929,2.798756973463715) -- (4.569997901818327,2.7947306267213268);
\draw[line width=1.2pt,color=ccqqqq] (4.569997901818327,2.7947306267213268) -- (4.584997892817726,2.7907283858894854);
\draw[line width=1.2pt,color=ccqqqq] (4.584997892817726,2.7907283858894854) -- (4.5999978838171245,2.7867500961531606);
\draw[line width=1.2pt,color=ccqqqq] (4.5999978838171245,2.7867500961531606) -- (4.614997874816523,2.7827956027778606);
\draw[line width=1.2pt,color=ccqqqq] (4.614997874816523,2.7827956027778606) -- (4.6299978658159215,2.7788647511535127);
\draw[line width=1.2pt,color=ccqqqq] (4.6299978658159215,2.7788647511535127) -- (4.64499785681532,2.774957386836423);
\draw[line width=1.2pt,color=ccqqqq] (4.64499785681532,2.774957386836423) -- (4.659997847814719,2.7710733555893663);
\draw[line width=1.2pt,color=ccqqqq] (4.659997847814719,2.7710733555893663) -- (4.674997838814117,2.7672125034198887);
\draw[line width=1.2pt,color=ccqqqq] (4.674997838814117,2.7672125034198887) -- (4.689997829813516,2.763374676616874);
\draw[line width=1.2pt,color=ccqqqq] (4.689997829813516,2.763374676616874) -- (4.704997820812914,2.759559721785447);
\draw[line width=1.2pt,color=ccqqqq] (4.704997820812914,2.759559721785447) -- (4.719997811812313,2.755767485880269);
\draw[line width=1.2pt,color=ccqqqq] (4.719997811812313,2.755767485880269) -- (4.734997802811711,2.751997816237284);
\draw[line width=1.2pt,color=ccqqqq] (4.734997802811711,2.751997816237284) -- (4.74999779381111,2.7482505606039753);
\draw[line width=1.2pt,color=ccqqqq] (4.74999779381111,2.7482505606039753) -- (4.764997784810508,2.7445255671681834);
\draw[line width=1.2pt,color=ccqqqq] (4.764997784810508,2.7445255671681834) -- (4.779997775809907,2.740822684585538);
\draw[line width=1.2pt,color=ccqqqq] (4.779997775809907,2.740822684585538) -- (4.794997766809305,2.7371417620055576);
\draw[line width=1.2pt,color=ccqqqq] (4.794997766809305,2.7371417620055576) -- (4.809997757808704,2.7334826490964668);
\draw[line width=1.2pt,color=ccqqqq] (4.809997757808704,2.7334826490964668) -- (4.824997748808102,2.729845196068766);
\draw[line width=1.2pt,color=ccqqqq] (4.824997748808102,2.729845196068766) -- (4.839997739807501,2.7262292536976247);
\draw[line width=1.2pt,color=ccqqqq] (4.839997739807501,2.7262292536976247) -- (4.8549977308068994,2.7226346733441105);
\draw[line width=1.2pt,color=ccqqqq] (4.8549977308068994,2.7226346733441105) -- (4.869997721806298,2.7190613069753256);
\draw[line width=1.2pt,color=ccqqqq] (4.869997721806298,2.7190613069753256) -- (4.8849977128056965,2.7155090071834733);
\draw[line width=1.2pt,color=ccqqqq] (4.8849977128056965,2.7155090071834733) -- (4.899997703805095,2.7119776272038982);
\draw[line width=1.2pt,color=ccqqqq] (4.899997703805095,2.7119776272038982) -- (4.9149976948044936,2.708467020932147);
\draw[line width=1.2pt,color=ccqqqq] (4.9149976948044936,2.708467020932147) -- (4.929997685803892,2.7049770429400724);
\draw[line width=1.2pt,color=ccqqqq] (4.929997685803892,2.7049770429400724) -- (4.944997676803291,2.70150754849103);
\draw[line width=1.2pt,color=ccqqqq] (4.944997676803291,2.70150754849103) -- (4.959997667802689,2.6980583935541915);
\draw[line width=1.2pt,color=ccqqqq] (4.959997667802689,2.6980583935541915) -- (4.974997658802088,2.694629434818016);
\draw[line width=1.2pt,color=ccqqqq] (4.974997658802088,2.694629434818016) -- (4.989997649801486,2.6912205297029033);
\draw[line width=1.2pt,color=ccqqqq] (4.989997649801486,2.6912205297029033) -- (5.004997640800885,2.6878315363730674);
\draw[line width=1.2pt,color=ccqqqq] (5.004997640800885,2.6878315363730674) -- (5.019997631800283,2.684462313747659);
\draw[line width=1.2pt,color=ccqqqq] (5.019997631800283,2.684462313747659) -- (5.034997622799682,2.681112721511159);
\draw[line width=1.2pt,color=ccqqqq] (5.034997622799682,2.681112721511159) -- (5.04999761379908,2.6777826201230788);
\draw[line width=1.2pt,color=ccqqqq] (5.04999761379908,2.6777826201230788) -- (5.064997604798479,2.6744718708269932);
\draw[line width=1.2pt,color=ccqqqq] (5.064997604798479,2.6744718708269932) -- (5.079997595797877,2.6711803356589243);
\draw[line width=1.2pt,color=ccqqqq] (5.079997595797877,2.6711803356589243) -- (5.094997586797276,2.667907877455115);
\draw[line width=1.2pt,color=ccqqqq] (5.094997586797276,2.667907877455115) -- (5.109997577796674,2.6646543598592003);
\draw[line width=1.2pt,color=ccqqqq] (5.109997577796674,2.6646543598592003) -- (5.124997568796073,2.661419647328815);
\draw[line width=1.2pt,color=ccqqqq] (5.124997568796073,2.661419647328815) -- (5.1399975597954715,2.6582036051416464);
\draw[line width=1.2pt,color=ccqqqq] (5.1399975597954715,2.6582036051416464) -- (5.15499755079487,2.655006099400965);
\draw[line width=1.2pt,color=ccqqqq] (5.15499755079487,2.655006099400965) -- (5.1699975417942685,2.6518269970406485);
\draw[line width=1.2pt,color=ccqqqq] (5.1699975417942685,2.6518269970406485) -- (5.184997532793667,2.6486661658297175);
\draw[line width=1.2pt,color=ccqqqq] (5.184997532793667,2.6486661658297175) -- (5.199997523793066,2.64552347437641);
\draw[line width=1.2pt,color=ccqqqq] (5.199997523793066,2.64552347437641) -- (5.214997514792464,2.642398792131803);
\draw[line width=1.2pt,color=ccqqqq] (5.214997514792464,2.642398792131803) -- (5.229997505791863,2.6392919893930102);
\draw[line width=1.2pt,color=ccqqqq] (5.229997505791863,2.6392919893930102) -- (5.244997496791261,2.6362029373059643);
\draw[line width=1.2pt,color=ccqqqq] (5.244997496791261,2.6362029373059643) -- (5.25999748779066,2.633131507867805);
\draw[line width=1.2pt,color=ccqqqq] (5.25999748779066,2.633131507867805) -- (5.274997478790058,2.6300775739288897);
\draw[line width=1.2pt,color=ccqqqq] (5.274997478790058,2.6300775739288897) -- (5.289997469789457,2.6270410091944427);
\draw[line width=1.2pt,color=ccqqqq] (5.289997469789457,2.6270410091944427) -- (5.304997460788855,2.6240216882258505);
\draw[line width=1.2pt,color=ccqqqq] (5.304997460788855,2.6240216882258505) -- (5.319997451788254,2.6210194864416314);
\draw[line width=1.2pt,color=ccqqqq] (5.319997451788254,2.6210194864416314) -- (5.334997442787652,2.618034280118081);
\draw[line width=1.2pt,color=ccqqqq] (5.334997442787652,2.618034280118081) -- (5.349997433787051,2.6150659463896138);
\draw[line width=1.2pt,color=ccqqqq] (5.349997433787051,2.6150659463896138) -- (5.364997424786449,2.612114363248814);
\draw[line width=1.2pt,color=ccqqqq] (5.364997424786449,2.612114363248814) -- (5.379997415785848,2.6091794095462024);
\draw[line width=1.2pt,color=ccqqqq] (5.379997415785848,2.6091794095462024) -- (5.394997406785246,2.60626096498974);
\draw[line width=1.2pt,color=ccqqqq] (5.394997406785246,2.60626096498974) -- (5.409997397784645,2.6033589101440713);
\draw[line width=1.2pt,color=ccqqqq] (5.409997397784645,2.6033589101440713) -- (5.4249973887840435,2.600473126429529);
\draw[line width=1.2pt,color=ccqqqq] (5.4249973887840435,2.600473126429529) -- (5.439997379783442,2.5976034961208967);
\draw[line width=1.2pt,color=ccqqqq] (5.439997379783442,2.5976034961208967) -- (5.4549973707828405,2.5947499023459577);
\draw[line width=1.2pt,color=ccqqqq] (5.4549973707828405,2.5947499023459577) -- (5.469997361782239,2.591912229083822);
\draw[line width=1.2pt,color=ccqqqq] (5.469997361782239,2.591912229083822) -- (5.484997352781638,2.5890903611630547);
\draw[line width=1.2pt,color=ccqqqq] (5.484997352781638,2.5890903611630547) -- (5.499997343781036,2.586284184259613);
\draw[line width=1.2pt,color=ccqqqq] (5.499997343781036,2.586284184259613) -- (5.514997334780435,2.5834935848945895);
\draw[line width=1.2pt,color=ccqqqq] (5.514997334780435,2.5834935848945895) -- (5.529997325779833,2.5807184504317897);
\draw[line width=1.2pt,color=ccqqqq] (5.529997325779833,2.5807184504317897) -- (5.544997316779232,2.5779586690751333);
\draw[line width=1.2pt,color=ccqqqq] (5.544997316779232,2.5779586690751333) -- (5.55999730777863,2.5752141298659015);
\draw[line width=1.2pt,color=ccqqqq] (5.55999730777863,2.5752141298659015) -- (5.574997298778029,2.5724847226798295);
\draw[line width=1.2pt,color=ccqqqq] (5.574997298778029,2.5724847226798295) -- (5.589997289777427,2.569770338224056);
\draw[line width=1.2pt,color=ccqqqq] (5.589997289777427,2.569770338224056) -- (5.604997280776826,2.567070868033936);
\draw[line width=1.2pt,color=ccqqqq] (5.604997280776826,2.567070868033936) -- (5.619997271776224,2.564386204469721);
\draw[line width=1.2pt,color=ccqqqq] (5.619997271776224,2.564386204469721) -- (5.634997262775623,2.5617162407131193);
\draw[line width=1.2pt,color=ccqqqq] (5.634997262775623,2.5617162407131193) -- (5.649997253775021,2.5590608707637377);
\draw[line width=1.2pt,color=ccqqqq] (5.649997253775021,2.5590608707637377) -- (5.66499724477442,2.556419989435412);
\draw[line width=1.2pt,color=ccqqqq] (5.66499724477442,2.556419989435412) -- (5.679997235773818,2.553793492352434);
\draw[line width=1.2pt,color=ccqqqq] (5.679997235773818,2.553793492352434) -- (5.694997226773217,2.5511812759456802);
\draw[line width=1.2pt,color=ccqqqq] (5.694997226773217,2.5511812759456802) -- (5.7099972177726155,2.548583237448644);
\draw[line width=1.2pt,color=ccqqqq] (5.7099972177726155,2.548583237448644) -- (5.724997208772014,2.5459992748933855);
\draw[line width=1.2pt,color=ccqqqq] (5.724997208772014,2.5459992748933855) -- (5.7399971997714125,2.543429287106392);
\draw[line width=1.2pt,color=ccqqqq] (5.7399971997714125,2.543429287106392) -- (5.754997190770811,2.540873173704367);
\draw[line width=1.2pt,color=ccqqqq] (5.754997190770811,2.540873173704367) -- (5.76999718177021,2.5383308350899405);
\draw[line width=1.2pt,color=ccqqqq] (5.76999718177021,2.5383308350899405) -- (5.784997172769608,2.5358021724473154);
\draw[line width=1.2pt,color=ccqqqq] (5.784997172769608,2.5358021724473154) -- (5.799997163769007,2.5332870877378477);
\draw[line width=1.2pt,color=ccqqqq] (5.799997163769007,2.5332870877378477) -- (5.814997154768405,2.5307854836955657);
\draw[line width=1.2pt,color=ccqqqq] (5.814997154768405,2.5307854836955657) -- (5.829997145767804,2.528297263822637);
\draw[line width=1.2pt,color=ccqqqq] (5.829997145767804,2.528297263822637) -- (5.844997136767202,2.52582233238478);
\draw[line width=1.2pt,color=ccqqqq] (5.844997136767202,2.52582233238478) -- (5.859997127766601,2.523360594406631);
\draw[line width=1.2pt,color=ccqqqq] (5.859997127766601,2.523360594406631) -- (5.874997118765999,2.5209119556670614);
\draw[line width=1.2pt,color=ccqqqq] (5.874997118765999,2.5209119556670614) -- (5.889997109765398,2.518476322694462);
\draw[line width=1.2pt,color=ccqqqq] (5.889997109765398,2.518476322694462) -- (5.904997100764796,2.516053602761983);
\draw[line width=1.2pt,color=ccqqqq] (5.904997100764796,2.516053602761983) -- (5.919997091764195,2.5136437038827406);
\draw[line width=1.2pt,color=ccqqqq] (5.919997091764195,2.5136437038827406) -- (5.934997082763593,2.5112465348049966);
\draw[line width=1.2pt,color=ccqqqq] (5.934997082763593,2.5112465348049966) -- (5.949997073762992,2.5088620050073045);
\draw[line width=1.2pt,color=ccqqqq] (5.949997073762992,2.5088620050073045) -- (5.96499706476239,2.506490024693634);
\draw[line width=1.2pt,color=ccqqqq] (5.96499706476239,2.506490024693634) -- (5.979997055761789,2.5041305047884705);
\draw[line width=1.2pt,color=ccqqqq] (5.979997055761789,2.5041305047884705) -- (5.9949970467611875,2.5017833569318952);
\draw[line width=1.2pt,color=ccqqqq] (5.9949970467611875,2.5017833569318952) -- (6.009997037760586,2.499448493474649);
\draw[line width=1.2pt,color=ccqqqq] (6.009997037760586,2.499448493474649) -- (6.0249970287599846,2.4971258274731767);
\draw [color=ffqqqq](6.1,2.7) node[anchor=north west] {$\mathscr{C}_f$};
\draw [domain=-0.5:6.68] plot(\x,{(-2.8-0.*\x)/-1.});
\draw [->,color=qqwuqq] (0.,2.8) -- (1.1010433617642879,2.8);
\draw [->,color=qqwuqq] (1.1010433617642879,2.8) -- (1.1010433617642879,0.);
\draw [color=qqwuqq] (0.,2.8)-- (1.1010433617642879,2.8);
\draw [->,color=qqwuqq] (4.550385209665318,2.7999999999997076) -- (4.550385209665318,0.);
\draw [color=qqwuqq](0.02,3.52) node[anchor=north west] {$b$};
\begin{scriptsize}
\draw [color=xdxdff] (0.,2.8)-- ++(-2.5pt,0 pt) -- ++(5.0pt,0 pt) ++(-2.5pt,-2.5pt) -- ++(0 pt,5.0pt);
\draw[color=black] (-4.14,2.64) node {$g$};
\end{scriptsize}
\end{tikzpicture}

\MtV{https://www.geogebra.org/m/AavNWuQD}{Lecture d'antécédents}
\end{minipage}
\end{Mt}

\begin{Rq}[Résolution graphique d'équation]\index{Résolution graphique!Equation}
Soit $f$ une fonction et $k$ un nombre réel. On note $\mathscr{C}$ la courbe représentative de $f$ dans un repère et $D_k$ la droite d'équation $y=k$ (parallèle à l’axe des abscisses).
Les solutions de l'équation $f(x)=k$ sont les abscisses des points d'intersection entre $C$ et $D_k$.

$\circledast$ Résoudre graphiquement $f(x)=m$ revient à déterminer les antécédents de $m$ par $f$.
\end{Rq}





\mini{
\EPC{1}{FEA-64}{Représenter.}
}{
\EPC{1}{FEA-65}{Représenter. Calculer.}
}

\mini{
\EPC{0}{FEA-56}{Représenter. Communiquer.}

\EPC{1}{FEA-58}{Représenter. Communiquer.}
}{
\EPC{1}{FEA-57}{Représenter.}

\EPC{0}{FEA-68}{Approfondissement.}
}

\begin{DTL}

\textbf{Exercice 1}


La valeur $g$ de la pesanteur terrestre (exprimée en N.kg$^{-1}$) est une fonction $g$ de l'altitude $h$ (exprimée en m) : $$g(h)=g_0\frac{R^2}{(r+h)^2}$$  où $g_0 \approx 9,81$ et R est le rayon de la terre moyen environ $6,37 \times 10^6$m.
\begin{enumerate}
\item Quelle est la valeur de $g$ à Lyon, dont l'altitude est 169 m ?
\item Quelle est la valeur de $g$ au sommet de l'Everest ?
\end{enumerate}

\textbf{Exercice 2}


Au cours de ses vacances, Vincent effectue une promenade en vélo. Le graphique ci-dessous indique la distance parcourue en fonction du temps.

\definecolor{ffqqqq}{rgb}{1.,0.,0.}
\definecolor{cqcqcq}{rgb}{0.7529411764705882,0.7529411764705882,0.7529411764705882}
\begin{tikzpicture}[line cap=round,line join=round,>=triangle 45,x=1.0cm,y=0.5cm]
\draw [color=cqcqcq,, xstep=0.5cm,ystep=2.0cm] (-0.22619172798194417,-1.537877030448518) grid (14.818271111514635,23.171115311838655);
\draw[->,color=black] (-0.22619172798194417,0.) -- (14.818271111514635,0.);
\foreach \x in {,0.5,1.,1.5,2.,2.5,3.,3.5,4.,4.5,5.,5.5,6.,6.5,7.,7.5,8.,8.5,9.,9.5,10.,10.5,11.,11.5,12.,12.5,13.,13.5,14.,14.5}
\draw[shift={(\x,0)},color=black] (0pt,2pt) -- (0pt,-2pt) node[below] {\footnotesize $\x$};
\draw[->,color=black] (0.,-1.537877030448518) -- (0.,23.171115311838655);
\foreach \y in {,2.,4.,6.,8.,10.,12.,14.,16.,18.,20.,22.}
\draw[shift={(0,\y)},color=black] (2pt,0pt) -- (-2pt,0pt) node[left] {\footnotesize $\y$};
\draw[color=black] (0pt,-10pt) node[right] {\footnotesize $0$};
\clip(-0.22619172798194417,-1.537877030448518) rectangle (14.818271111514635,23.171115311838655);
\draw [color=ffqqqq] (9.,0.)-- (10.5,14.);
\draw [color=ffqqqq] (10.5,14.)-- (11.,14.);
\draw [color=ffqqqq] (11.,14.)-- (11.502035262490187,22.049556406661356);
\draw [color=ffqqqq] (11.502035262490187,22.049556406661356)-- (12.535054657867978,22.00618692565458);
\draw [color=ffqqqq] (12.535054657867978,22.00618692565458)-- (13.,14.);
\draw [color=ffqqqq] (13.,14.)-- (13.410494823442376,18.016194673031002);
\draw [color=ffqqqq] (13.410494823442376,18.016194673031002)-- (14.47853182544314,0.);
\end{tikzpicture}

\begin{enumerate}
\item Sur quelles périodes de temps Vincent s'éloigne-il de sa maison ?
\item A 10h30, à quelle distance de sa maison se trouve-t-il ?
\item Que se passe-t-il entre 9h30 et 10h00 ?
\item Quelle est sa vitesse moyenne entre 10h et 11h ?
\item Quelle est sa vitesse moyenne entre 12h et 12h30 ? Et entre 14h et 14h30 ?
\end{enumerate}
\end{DTL}}% Résolution d'équation, d'inéquation, courbes
%%##########\impress{\impressionProf}{

\section{Les fonctions}


\begin{description}
\item[Public cible :] Enseignant de mathématique
\item[Durée :] les durées sont notées sur chaque fiche de séance.
\end{description}

\subsection{Compétences du cycle 5}

Les 6 compétences mathématiques, 

\begin{description}
\item[•] chercher, expérimenter – en particulier à l'aide d'outils logiciels ;
\item[•] modéliser, faire une simulation, valider ou invalider un modèle ;
\item[•] représenter, choisir un cadre (numérique, algébrique, géométrique …), changer de registre ;
\item[•] raisonner, démontrer, trouver des résultats partiels et les mettre en perspective ;
\item[•] calculer, appliquer des techniques et mettre en œuvre des algorithmes ;
\item[•] communiquer un résultat par oral ou par écrit, expliquer oralement une démarche.
\end{description}


\subsection{Objectifs de la séquence}

l'objectif des séquences d'analyse est de rendre les élèves capables d'étudier :

\begin{description}
\item[•]  un problème se ramenant à une équation du type $f(x)=k$ et de le résoudre dans le cas où la fonction est donnée (définie par une courbe, un tableau de données, une formule) et aussi lorsque toute autonomie est laissée pour associer au problème divers aspects d’une fonction ;
\item[•]  un problème d'optimisation ou un problème du type $f(x)< k$ et de le résoudre, selon les cas, en exploitant les potentialités de logiciels, graphiquement ou algébriquement, toute autonomie pouvant être laissée pour associer au problème une fonction.
\end{description}


\subsection{Contenus en lien avec le BO}

\begin{tabularx}{\textwidth}{|X|X|X|}
\hline 
CONTENUS & CAPACITÉS ATTENDUES & COMMENTAIRES \\ 
\hline
 
Étude qualitative de
fonctions 
& 
• Décrire, avec un vocabulaire
adapté ou un tableau de
variations, le comportement
d’une fonction définie par une
courbe.

• Dessiner une représentation
graphique compatible avec un
tableau de variations. 


& 
Les élèves doivent distinguer les
courbes pour lesquelles l’information
sur les variations est exhaustive, de
celles obtenues sur un écran graphique. \\ 

 & Lorsque le sens de variation
est donné, par une phrase ou
un tableau de variations :

• comparer les images de deux
nombres d’un intervalle ;

• déterminer tous les nombres
dont l’image est supérieure (ou
inférieure) à une image
donnée. & Les définitions formelles d’une fonction
croissante, d’une fonction décroissante,
sont progressivement dégagées. Leur
maîtrise est un objectif de fin d’année.

$\diamond$ Même si les logiciels traceurs de
courbes permettent d’obtenir rapidement
la représentation graphique d’une
fonction définie par une formule
algébrique, il est intéressant, notamment
pour les fonctions définies par
morceaux, de faire écrire aux élèves un
algorithme de tracé de courbe.

$\leftrightarrows$ Étude des signaux périodiques en
physique. \\ 
\hline 
\end{tabularx} } 
%\impress{\impressionEleve}{\begin{titre}[Étude qualitative de fonctions]

\Titre{Description d'un comportement}{4}
\end{titre}


\begin{CpsCol}
\textbf{Variations de fonctions}
\begin{description}
\item[$\square$] Déterminer graphiquement les extremums d’une fonction sur un intervalle.
\item[$\square$] Exploiter un logiciel de géométrie dynamique ou de calcul formel, la calculatrice ou Python pour décrire les variations d’une fonction donnée par une formule.
\end{description}
\end{CpsCol}

\Rec{1}{VF-0-N}

\begin{DefT}{Tableau de variations}\index{Tableau de variations}
Un tableau de variations d'une fonction $f$ est un tableau dans lequel l'étude synthétise
\begin{description}
\item[•] le domaine de définition de $f$ explicitement,
\item[•] les variations de $f$ à l'aide de flèches,
\item[•] les maximum et minimum.
\end{description} 

\begin{tikzpicture}
\tkzTabInit[lgt=2,espcl=3]{ $x$ / 1,$f$  / 2}
{ $0$ ,$3$,$5$}
\tkzTabVar{+/$3$,-/$1$,+/$2$ }
\end{tikzpicture}
\end{DefT}




\mini{
\EPC{1}{VF-2}{Représenter.}

\EPC{1}{VF-5}{Représenter. Raisonner.}
}{
\EPC{1}{VF-3}{Représenter.}
}

\begin{Rq}
La courbe représentative d'une fonction $f$ et le tableau de variations de $f$ correspondent.
\end{Rq}


\begin{DefT}{Variations - Approche visuelle}
Dans un tableau de variations,
\begin{description}
\item[•] Une fonction croissante sur un intervalle est représentée par une flèche qui monte sur cet intervalle.
\item[•] Une fonction décroissante sur un intervalle est représentée par une flèche qui descend sur cet intervalle.
\end{description}
\end{DefT}



\begin{DefT}{Visualisation d'une fonction croissante ou décroissante}\index{Représentation graphique!Fonction croissante, décroissante}
\begin{description}[leftmargin=*]
\item[•] Lorsque, sur un intervalle $I$, la courbe représentative d'une fonction monte sans portion horizontale, on dit que la fonction est \textbf{strictement croissante sur $I$}. Si la courbe représentative possède une portion horizontale, on dit que la fonction est \textbf{croissante sur $I$}.
\item[•] Lorsque, sur un intervalle $I$, la courbe représentative d'une fonction descend sans portion horizontale, on dit que la fonction est \textbf{strictement décroissante sur $I$}. Si la courbe représentative possède une portion horizontale, on dit que la fonction est \textbf{décroissante sur $I$}.
\end{description}
\end{DefT}




\begin{DefT}{Maximum, minimum}
On dit que
\begin{description}[leftmargin=*]
\item[•] $M$ est le maximum de $f$ sur son domaine de définition si pour tout réel $x$ de $I$, $f(x) \leq M$. 
\item[•] $m$ est le minimum de $f$ sur son domaine de définition si pour tout réel $x$ de $I$, $f(x) \geq m$. 
\end{description} 
On dit que $f$ est bornée  sur $I$ lorsque $f$ accepte un maximum \textbf{et} un minimum sur $I$.
\end{DefT}


 \EPC{1}{VF-3bis}{Communiquer.}

\begin{DefT}{Fonction croissante, décroissante sur $I$ - Approche comparatiste}
On dit que
\begin{description}[leftmargin=*]
\item[•] une fonction $f$ est \textbf{croissante sur $I$}, lorsque pour tout nombre $a$ et $b$ de $I$, les images de $a$ et de $b$ sont rangées dans le même ordre que $a$ et $b$.
\item[•] une fonction $f$ est \textbf{décroissante sur $I$}, lorsque pour tout nombre $a$ et $b$ de $I$,  les images de $a$ et de $b$ sont rangées dans l'ordre inverse de $a$ et $b$.
\end{description} 
\end{DefT}

 

\begin{DefT}{Fonction croissante, décroissante sur $I$  - Approche analytique }
\begin{description}[leftmargin=*]
\item[•] une fonction $f$ est \textbf{croissante sur $I$}, lorsque pour tout nombre $a$ et $b$ de $I$ tels que $a \leq b$, $f(a) \leq f(b)$.
\item[•] une fonction $f$ est \textbf{strictement croissante sur $I$}, lorsque pour tout nombre $a$ et $b$ de $I$ tels que $a \leq b$, $f(a) < f(b)$.
\item[•] une fonction $f$ est décroissante sur $I$, lorsque pour tout nombre $a$ et $b$ de $I$ tels que $a \leq b$, $f(a) \geq f(b)$.
\item[•] une fonction $f$ est décroissante sur $I$, lorsque pour tout nombre $a$ et $b$ de $I$ tels que $a \leq b$, $f(a) > f(b)$.
\end{description} 
\end{DefT}

\EPC{1}{VF-4}{Raisonner. Communiquer.}
 
\EPC{0}{VF-29}{Raisonner. Représenter. Communiquer. }






\begin{minipage}{0.47\linewidth}
\EPC{1}{VF-12}{Raisonner. Représenter.}
\end{minipage}
\hfill
\begin{minipage}{0.47\linewidth}
\EPC{0}{VF-13}{Raisonner. Communiquer.}
\end{minipage}


\begin{minipage}{0.5\linewidth}
\EPC{1}{VF-27}{Représenter. Raisonner. }
\end{minipage}
\hfill
\begin{minipage}{0.47\linewidth}
\EPC{1}{VF-7}{Raisonner. Communiquer.}
\end{minipage}

 


\EPC{0}{VF-28}{Raisonner. Communiquer. }



 


%\EPC{1}{VF-28_cor}{Raisonner. Communiquer. }

%\EPC{1}{VF-29_cor}{Raisonner. Représenter. Communiquer. }} %  séance 1  
%%##########\impress{\impressionEleve}{\begin{titreTice}[Étude qualitative de fonctions]

\Titre{Modélisation avec Geogebra}{1}
\end{titreTice}


\begin{CpsCol}
\textbf{Variations de fonctions}
\begin{description}
\item[$\square$] Utiliser un logiciel de géométrie dynamique pour conjecturer une situation.
\end{description}
\end{CpsCol}


\Rec{1}{VF-14}
} % séance 6
%%##########\impress{\impressionEleve}{\begin{seance}[Étude qualitative de fonctions]

\Titre{Comparaison d'images}{1}
\end{seance}


\begin{CpsCol}
\textbf{Variations de fonctions}
\begin{description}
\item[$\square$] Comparer deux images sur un intervalle donné
\item[$\square$] Déterminer tous les nombres dont l'image est supérieure (inférieure) à une image donnée
\end{description}
\end{CpsCol}


\EPC{1}{VF-26}{Raisonner. Communiquer.}

\EPC{1}{VF-12}{Raisonner. Représenter.}


\EPC{1}{VF-4}{Raisonner. Communiquer.}


\begin{seance}[Étude quantitative de fonctions]

\Titre{Applications}{1}
\end{seance}



\EPC{1}{VF-13}{Raisonner. Communiquer.}

\EPC{1}{VF-27}{Représenter. Raisonner. }

\EPC{1}{VF-7}{Raisonner. Communiquer.}


\begin{titreDTL}[Étude quantitative de fonctions]

\Titre{S'auto évaluer}{1}
\end{titreDTL}


\EPC{1}{VF-28}{Raisonner. Communiquer. }

\EPC{1}{VF-29}{Raisonner. Représenter. Communiquer. }


\begin{titreDTL}[Étude quantitative de fonctions]

\Titre{S'auto évaluer. Corrigé}{1}
\end{titreDTL}


\EPC{1}{VF-28_cor}{Raisonner. Communiquer. }

\EPC{1}{VF-29_cor}{Raisonner. Représenter. Communiquer. }}%   Auto évaluation
%%##########\impress{\impressionEleve}{\begin{synthesecours}[Étude qualitative de fonctions]

\TitreSansTemps{Synthèse}
\end{synthesecours}


\begin{DefT}{Domaine de définition}
Soit $f$ une fonction. \\
Le domaine de définition de $f$ est l'ensemble des valeurs de la variable (souvent appelée $x$) qui admettent une image par $f$.
\end{DefT}


\begin{Rq}
Un intervalle est un ensemble de nombres. Les valeurs extrêmes de l'intervalle sont appelées \textbf{bornes de l'intervalle}. Le domaine de définition est donc un intervalle.
\end{Rq}


\begin{DefT}{Courbe}
Soit $f$ une fonction définie sur un intervalle I. \\
On appelle courbe de la fonction $f$, aussi appelée représentation graphique de $f$, l'ensemble des points $M$ du plan dont les coordonnées sont $(x,f(x))$, où $x$ parcourt le domaine de définition de $f$.
\end{DefT}

\begin{DefT}{Tableau de variations}\index{Tableau de variations}
Un tableau de variations d'une fonction $f$ est un tableau dans lequel l'étude synthétise
\begin{description}
\item[•] le domaine de définition de $f$ explicitement,
\item[•] les variations de $f$ à l'aide de flèches,
\item[•] les maximum et minimum.
\end{description} 

\begin{tikzpicture}
\tkzTabInit[lgt=2,espcl=3]{ $x$ / 1,$f$  / 2}
{ $0$ ,$3$,$5$}
\tkzTabVar{+/$3$,-/$1$,+/$2$ }
\end{tikzpicture}
\end{DefT}


\begin{Rq}
La courbe représentative d'une fonction $f$ et le tableau de variations de $f$ correspondent.
\end{Rq}


\begin{DefT}{Variations - Approche visuelle}
Dans un tableau de variations,
\begin{description}
\item[•] Une fonction croissante sur un intervalle est représentée par une flèche qui monte sur cet intervalle.
\item[•] Une fonction décroissante sur un intervalle est représentée par une flèche qui descend sur cet intervalle.
\end{description}
\end{DefT}



\begin{DefT}{Visualisation d'une fonction croissante ou décroissante}\index{Représentation graphique!Fonction croissante, décroissante}
\begin{description}[leftmargin=*]
\item[•] Lorsque, sur un intervalle $I$, la courbe représentative d'une fonction monte sans portion horizontale, on dit que la fonction est \textbf{strictement croissante sur $I$}. Si la courbe représentative possède une portion horizontale, on dit que la fonction est \textbf{croissante sur $I$}.
\item[•] Lorsque, sur un intervalle $I$, la courbe représentative d'une fonction descend sans portion horizontale, on dit que la fonction est \textbf{strictement décroissante sur $I$}. Si la courbe représentative possède une portion horizontale, on dit que la fonction est \textbf{décroissante sur $I$}.
\end{description}
\end{DefT}




\begin{DefT}{Maximum, minimum}
On dit que
\begin{description}[leftmargin=*]
\item[•] $M$ est le maximum de $f$ sur son domaine de définition si pour tout réel $x$ de $I$, $f(x) \leq M$. 
\item[•] $m$ est le minimum de $f$ sur son domaine de définition si pour tout réel $x$ de $I$, $f(x) \geq m$. 
\end{description} 
On dit que $f$ est bornée  sur $I$ lorsque $f$ accepte un maximum \textbf{et} un minimum sur $I$.
\end{DefT}


 

\begin{DefT}{Fonction croissante, décroissante sur $I$ - Approche comparatiste}
On dit que
\begin{description}[leftmargin=*]
\item[•] une fonction $f$ est \textbf{croissante sur $I$}, lorsque pour tout nombre $a$ et $b$ de $I$, les images de $a$ et de $b$ sont rangées dans le même ordre que $a$ et $b$.
\item[•] une fonction $f$ est \textbf{décroissante sur $I$}, lorsque pour tout nombre $a$ et $b$ de $I$,  les images de $a$ et de $b$ sont rangées dans l'ordre inverse de $a$ et $b$.
\end{description} 
\end{DefT}

 

\begin{DefT}{Fonction croissante, décroissante sur $I$  - Approche analytique }
\begin{description}[leftmargin=*]
\item[•] une fonction $f$ est \textbf{croissante sur $I$}, lorsque pour tout nombre $a$ et $b$ de $I$ tels que $a \leq b$, $f(a) \leq f(b)$.
\item[•] une fonction $f$ est \textbf{strictement croissante sur $I$}, lorsque pour tout nombre $a$ et $b$ de $I$ tels que $a \leq b$, $f(a) < f(b)$.
\item[•] une fonction $f$ est décroissante sur $I$, lorsque pour tout nombre $a$ et $b$ de $I$ tels que $a \leq b$, $f(a) \geq f(b)$.
\item[•] une fonction $f$ est décroissante sur $I$, lorsque pour tout nombre $a$ et $b$ de $I$ tels que $a \leq b$, $f(a) > f(b)$.
\end{description} 
\end{DefT}


} % séance 6
%%##########\impress{\impressionProf}{\input{CHAPITRES/VF-F0-prof}} 
%\impress{\impressionProf}{\begin{titreTice}[Généralités sur les fonctions]

\Titre{Trace active}{0}
\end{titreTice}

\vspace{0.2cm}

Très souvent la géométrie est un terrain privilégié pour introduire une fonction.
Géogébra permet de créer une figure dynamique et de mettre en évidence une courbe comme trace ou lieu de point, image d'un point mobile sur un objet (cercle, droite, segment,...)

\vspace{0.2cm}

\begin{minipage}{0.48\linewidth}
\textbf{Problème :} \textit{extrait du livre Hyperbole Seconde – Nathan 2012 }


Dans un repère d'origine $O$, $\mathscr{C}$ est le demi-cercle ci-contre de centre $O$ et de rayon 4. $A$ est le point de coordonnées (0;5). Pour tout réel $x$ de $[-4;4]$, on pose $f(x)=AM$ où $M$ est le point de $\mathscr{C}$ d'abscisse $x$.
Déterminer le tableau de variation de la fonction $f$.
\end{minipage}
\hfill
\begin{minipage}{0.48\linewidth}
\definecolor{uuuuuu}{rgb}{0.26666666666666666,0.26666666666666666,0.26666666666666666}
\definecolor{ffqqqq}{rgb}{1.,0.,0.}
\definecolor{qqqqff}{rgb}{0.,0.,1.}
\definecolor{xdxdff}{rgb}{0.49019607843137253,0.49019607843137253,1.}
\begin{tikzpicture}[line cap=round,line join=round,>=triangle 45,x=1.0cm,y=1.0cm]
\draw[->,color=black] (-4.28,0.) -- (4.5,0.);
\foreach \x in {-4.,-3.,-2.,-1.,1.,2.,3.,4.}
\draw[shift={(\x,0)},color=black] (0pt,2pt) -- (0pt,-2pt) node[below] {\footnotesize $\x$};
\draw[->,color=black] (0.,-0.58) -- (0.,5.44);
\foreach \y in {,1.,2.,3.,4.,5.}
\draw[shift={(0,\y)},color=black] (2pt,0pt) -- (-2pt,0pt);
\draw[color=black] (0pt,-10pt) node[right] {\footnotesize $0$};
\clip(-4.28,-0.58) rectangle (4.5,5.44);
\draw [shift={(0.,0.)}] plot[domain=0.:3.141592653589793,variable=\t]({1.*4.*cos(\t r)+0.*4.*sin(\t r)},{0.*4.*cos(\t r)+1.*4.*sin(\t r)});
\draw [color=ffqqqq] (-2.4141512871263906,3.18933748024037)-- (0.,5.);
\draw [color=ffqqqq] (-2.4141512871263906,3.18933748024037)-- (-2.4141512871263906,0.);
\draw (-2.54,0.12) node[anchor=north west] {$x$};
\begin{scriptsize}
\draw [fill=xdxdff] (-4.,0.) circle (2.5pt);
\draw[color=xdxdff] (-3.86,0.36) node {$B$};
\draw [fill=qqqqff] (4.,0.) circle (2.5pt);
\draw[color=qqqqff] (4.14,0.36) node {$C$};
\draw [fill=xdxdff] (0.,5.) circle (2.5pt);
\draw[color=xdxdff] (0.14,5.36) node {$A$};
\draw [fill=xdxdff] (-2.4141512871263906,3.18933748024037) circle (2.5pt);
\draw[color=xdxdff] (-2.86,3.4) node {$M$};
\draw [fill=uuuuuu] (-2.4141512871263906,0.) circle (1.5pt);
\end{scriptsize}
\end{tikzpicture}

\end{minipage}

\vspace{0.2cm}

\textbf{Étape 1} : Dans la fenêtre graphique 1, on construit la figure avec le point $M$ sur l'objet cercle.

\includegraphics[scale=0.5]{VF-F2-0.jpg}

Pour tracer cette figure, on utilise les icônes
\begin{description}
    \item[•] \includegraphics[scale=0.5]{VF-F2-20.jpg} 
    \item[•] \includegraphics[scale=0.5]{VF-F2-21.jpg}  pour placer le point M sur le demi cercle
\end{description}       
        
Créer le segment AM : dans la fenêtre algèbre la longueur AM est notée $a$.

\vspace{0.2cm}

\textbf{Étape 2 :}  Ouvrir la fenêtre graphique 2

\includegraphics[scale=0.5]{VF-F2-1.jpg} 

\vspace{0.2cm}

\textbf{Etape 3 :} Rendre la fenêtré graphique 2 active. Au besoin, cliquez simplement sur la fenêtre graphique 2. Graphique 2 se met en gras et Graphique 1 en normal.

\vspace{0.2cm}

Dans la barre de saisie, créer le point $N=(x(M),a)$. 
Attention entre les 2 coordonnées du point $M$, il faut écrire une virgule et non un point-virgule sinon le point $M$ est placé selon ses coordonnées polaires.
$x(M)$ est l'instruction Géogébra qui donne l'abscisse du point $M$.

\vspace{0.2cm}

\textbf{Étape 4 :}

\vspace{0.2cm}

\begin{minipage}{0.68\linewidth}
Cliquez droit sur le point $N$ ou cliquez le point $N$ dans la fenêtre algèbre. Une fenêtre contextuelle apparaît et propose les propriétés du point $N$.
Ouvrir les propriétés de $N$.
Dans l'onglet « Basique », cochez « afficher la trace ».
Puis fermez la fenêtre contextuelle.
\end{minipage}
\hfill
\begin{minipage}{0.28\linewidth}

\includegraphics[scale=0.5]{VF-F2-5.jpg} 

\end{minipage}
\vspace{0.2cm}

\textbf{Étape 5 }: Trace du point $N$

\vspace{0.2cm}

A l'aide de l’icône  \includegraphics[scale=0.5]{VF-F2-4.jpg}         déplacer le point $M$ dans la fenêtre Graphique 1. La trace du point $N$ apparaît sur la fenêtre Graphique 2.

On peut automatiser la situation en animant le point M dans la fenêtre contextuelle du point $M$, Dans l'onglet « Basique », cochez «animer ». et mettre en relief la position de M en affichant sa trace active.

\includegraphics[scale=0.5]{VF-F2-6.jpg} }% TICE %Géogébra. Des figures pour conjecturer. Geogebra tube, partager. La double fenêtre.
%
%
%\chapter{Fonctions affines}
%
%\begin{titre}[Problème du premier degré]

\Titre{Fonctions affines}{3}
\end{titre}


\begin{CpsCol}
\textbf{Variations de fonctions}
\begin{description}
\item[$\square$] Construire une fonction affine pour résoudre un problème
\end{description}
\end{CpsCol}

\Rec{1}{FPD-0}


\begin{DefT}{Fonctions affines}\index{Fonctions!Affine}
Une \textbf{fonction affine} $f$ est une fonction définie sur $\R$ de la forme $f(x)=ax+b$, avec $a\in \R^*$ et $b \in \R$.
\end{DefT}



\begin{Rq}\index{Fonctions!Linéaire}
Une fonction linéaire est une fonction affine particulière ; lorsque $b=0$.
\end{Rq}


\begin{DefT}{Représentation graphique} \index{Fonctions affines! Représentation graphique}
La \textbf{représentation graphique} d'une fonction affine $f$ de la forme $f(x)=ax+b$, avec $a\in \R^*$ et $b \in \R$ est la droite d'équation $y=ax+b$. 
\end{DefT}


\mini{
\EPCN{Modéliser. Représenter. Calculer }


\begin{multicols}{2}
Sur la figure, les droites $(DE)$ et $(BC)$ sont parallèles. $AD=3$, $DB=2$, $AE=4$ et $DE=x$. Soit $f$ la fonction qui à $x$ associe la longueur $BC$. Soit $g$ la fonction qui à $x$ associe le périmètre $AED$ et  $h$ la fonction qui à $x$ associe le périmètre du trapèze $BDEC$.

\begin{center}
\definecolor{xdxdff}{rgb}{0.49019607843137253,0.49019607843137253,1.}
\definecolor{qqqqff}{rgb}{0.,0.,1.}
\begin{tikzpicture}[line cap=round,line join=round,>=triangle 45,x=0.5084745762711864cm,y=0.5084745762711864cm]
\clip(-1.06,-0.36) rectangle (4.92,5.54);
\draw [color=qqqqff] (0.8,5.)-- (0.,0.);
\draw [color=qqqqff] (0.,0.)-- (4.,0.);
\draw [color=qqqqff] (4.,0.)-- (0.8,5.);
\draw [color=qqqqff] (0.3183775351014041,1.9898595943837754)-- (2.72,2.);
\draw (0.2,4.04) node[anchor=north west] {$3$};
\draw (-0.34,1.7) node[anchor=north west] {$2$};
\draw (2.06,3.94) node[anchor=north west] {$4$};
\draw (1.14,2.74) node[anchor=north west] {$x$};
\begin{scriptsize}
\draw [color=qqqqff] (0.8,5.)-- ++(-2.5pt,0 pt) -- ++(5.0pt,0 pt) ++(-2.5pt,-2.5pt) -- ++(0 pt,5.0pt);
\draw[color=qqqqff] (0.94,5.36) node {$A$};
\draw [color=qqqqff] (0.,0.)-- ++(-2.5pt,0 pt) -- ++(5.0pt,0 pt) ++(-2.5pt,-2.5pt) -- ++(0 pt,5.0pt);
\draw[color=qqqqff] (-0.32,0.36) node {$B$};
\draw [color=qqqqff] (4.,0.)-- ++(-2.5pt,0 pt) -- ++(5.0pt,0 pt) ++(-2.5pt,-2.5pt) -- ++(0 pt,5.0pt);
\draw[color=qqqqff] (4.14,0.36) node {$C$};
\draw [color=xdxdff] (0.3183775351014041,1.9898595943837754)-- ++(-2.5pt,0 pt) -- ++(5.0pt,0 pt) ++(-2.5pt,-2.5pt) -- ++(0 pt,5.0pt);
\draw[color=xdxdff] (0.,2.32) node {$D$};
\draw [color=xdxdff] (2.72,2.)-- ++(-2.5pt,0 pt) -- ++(5.0pt,0 pt) ++(-2.5pt,-2.5pt) -- ++(0 pt,5.0pt);
\draw[color=xdxdff] (2.86,2.36) node {$E$};
\end{scriptsize}
\end{tikzpicture}
\end{center}

\end{multicols}

\begin{enumerate}
\item Exprimer $f(x)$, $g(x)$ et $h(x)$ en fonction de $x$. Ces fonctions sont-elles affines ? Linéaires ?
\item En donner des représentations graphiques sur $[1;7]$.
\item En déduire graphiquement la valeur de $x$ pour que $ADE$ et $BDEC$ aient le même périmètre.
\item Retrouver ce résultat par le calcul.
\end{enumerate}


\EPCN{Calculer }


Résoudre dans $\R$ l'équation $4x+2=3-2x$.

Résoudre dans $\R$ l'équation $2(x+3)+1=3-(x-6)$.
}{
\EPCN{Calculer }


\begin{enumerate}
\item Déterminer une fonction affine $f$ telle que $f(0)=5$ et $f(2)=3$.
\item Déterminer une fonction affine $f$ telle que $f(-2)=1$ et $f(3)=6$.
\end{enumerate}


\EPCN{Modéliser. Communiquer. Calculer }


Un cycliste monte une cote de 24 km à la vitesse moyenne de $12 km.h^{-1}$, puis la descend à la vitesse moyenne de $36 km.h^{-1}$.
\begin{enumerate}
\item Soit $d$ la distance parcourue par le cycliste en fonction de la durée $t$ en heures. Écrire $d(t)$ pour $0<t<2$. Pour $t>2$, montrer que $d(t)=24+36(t-2)$.
\item Donner une représentation graphique de $d$
\item Sur le même dessin, donner la représentation de la fonction $h$ qui à $t$ associe la distance qui reste à parcourir en fonction de $t$ pour revenir au départ.
\end{enumerate}
}








% La fonction affine + rep. graphique
%\begin{titre}[Problème du premier degré]

\Titre{Variations et signe de $ax+b$}{2}
\end{titre}


\begin{CpsCol}
\textbf{Variations de fonctions}
\begin{description}
\item[$\square$] Donner le sens de variation d'une fonction affine
\item[$\square$] Donner le tableau de signe de $ax + b$ pour des valeurs numériques données de $a$ et $b$
\item[$\square$] Résoudre une équation, une inéquation produit ou quotient, à l'aide d'un tableau de signes.
\end{description}
\end{CpsCol}

\Rec{1}{FPD-4}


\begin{Pp}\index{Fonctions affines! Variations}
Soit $f$ une fonction affine de la forme $f(x)=ax+b$, avec $a\in \R^*$ et $b \in \R$.
\begin{description}
\item[•] Si $a >0$ alors $f$ est strictement croissante sur $\R$.
\item[•] Si $a <0$ alors $f$ est strictement décroissante sur $\R$.
\end{description} 
\end{Pp}

\ROC

\Rec{1}{FPD-5}

\paragraphe{Les inéquations produits nul}

\begin{Def}
$a, b, c, d$ sont quatre nombres réels connus.

Une \textit{in}équation-produit-nul est une des inéquations suivantes :
\begin{description}
\item[•] $(ax+b)(cx+d) \leq  0$
\item[•] $(ax+b)(cx+d) < 0$
\item[•] $(ax+b)(cx+d) \geq 0$
\item[•] $(ax+b)(cx+d) > 0$
\end{description}
\end{Def}

\begin{Rq}
Tout inéquation qui se factorise en une de ces 4 inéquations peut se résoudre avec les méthodes ci-dessous. 
\end{Rq}

\begin{Mt}
Une \textit{in}équation-produit-nul peut se résoudre par un tableau de signe.
\end{Mt}


On souhaite résoudre $(3x+2)(-4x+1) \leq  0$.

\begin{Mt}
\begin{enumerate}
\item On étudie le signe de $3x+2$. On pourra résoudre $3x+2 \leq  0$.
\item On étudie le signe de $-4x+1$. On pourra résoudre $-4x+1 \leq  0$.
\item On place dans un tableau des signes des expressions.
\item On conclut.
\item On vérifie sur GGB ou sur sa calculatrice le résultat.
\end{enumerate}
\end{Mt}

\paragraphe{Résolution}

\begin{enumerate}
\item $3x+2 \leq  0 \Longleftrightarrow 3x \leq  2  \Longleftrightarrow x \leq \frac{2}{3} $
\item $-4x+1 \leq  0 \Longleftrightarrow -4x \leq -1  \Longleftrightarrow x \geq \frac{1}{4} $. On pense à la propriété : Lorsqu'on multiplie ou l'on divise par un nombre négatif, on change le sens de l'ordre.
\item 
\begin{tabular}{|c|ccccccc|}
\hline 
$x$ & $-\infty$ &   & $\frac{1}{4}$ &   & $\frac{2}{3}$ &  & $+\infty$ \\ 
\hline 
signe de $3x+2$ &   & $-$ &  & $-$ & 0 & + &  \\ 
\hline 
signe de $-4x+1$ &   & $+$ & 0 & $-$ & & $-$ &  \\ 
\hline 
signe de $(3x+2)(-4x+1)$ &   &$-$  & 0 &  $+$ & 0 & $-$&  \\ 
\hline 
\end{tabular} 

\item Comme on souhaite  $(3x+2)(-4x+1) \leq  0$, on a alors $S = \left]- \infty; \frac{1}{4} \right] \cup \left[ \frac{2}{3} ; +\infty  \right[$


\item 

\begin{tikzpicture}[line cap=round,line join=round,>=triangle 45,x=3.220064724919094cm,y=1.0cm]
\begin{axis}[
x=3.220064724919094cm,y=1.0cm,
axis lines=middle,
ymajorgrids=true,
xmajorgrids=true,
xmin=-0.9095477386934674,
xmax=0.6432160804020101,
ymin=-1.6799999999999997,
ymax=3.0599999999999996,
xtick={-0.5,0.0,...,0.5},
ytick={-1.0,0.0,...,3.0},]
\clip(-0.9095477386934674,-1.68) rectangle (0.6432160804020101,3.06);
\draw [samples=50,rotate around={-180.:(-0.20833333333333334,2.5208333333333335)},xshift=-0.6708468176914779cm,yshift=2.5208333333333335cm,line width=2.pt,domain=-0.9166666666666666:0.9166666666666666)] plot (\x,{(\x)^2/2/0.041666666666666664});
\end{axis}
\end{tikzpicture}

\end{enumerate}% Signe de ax+b et tableau de signe
%\begin{titre}[Problème du premier degré]

\Titre{Tableau de signes}{2}
\end{titre}


\begin{CpsCol}
\begin{description}
\item[$\square$] Donner le tableau de signe de $(ax + b)(cx+d)$ ou $\frac{ax + b}{cx+d}$
\item[$\square$] Résoudre une inéquation
\end{description}
\end{CpsCol}


 
\EPC{1}{FPD-19}{Représenter.}
 
\EPCB{1}{FPD-20}{Calculer.}
 


\mini{
\Rec{1}{FPD-10}
}{
\Rec{1}{FPD-11}
}

\mini{
\AD{1}{FPD-15}

\AD{1}{FPD-16}
}{
\AD{1}{FPD-13}

\AD{1}{FPD-14}
}

% Résoudre une équation se ramenant au premier degré
%\begin{titreTice}[Problème du premier degré]

\Titre{Déterminer une fonction affine}{1}
\end{titreTice}


\begin{CpsCol}
\begin{description}
\item[$\square$] Résoudre un système, éventuellement avec sa calculatrice
\end{description}
\end{CpsCol}


\Rec{1}{FPD-18}

\begin{Mt}
Toute équation affine est de la forme $f(x)=ax+b, a\neq0$. ICi, $x$ et $f(x)$ sont connus, on cherche $a$ et $b$.

On se retrouve donc à résoudre un système de deux inconnues à 2 équations. Les deux équations déduites sont : 
\begin{enumerate}
\item $f(4)=2 \Longleftrightarrow 4a+b=2$
\item $f(3)=-1 \Longleftrightarrow 3a+b=-1$
\end{enumerate}
On utilise alors la fonction système de la calculatrice.
\end{Mt}

\subsection*{A la main}

$\left\lbrace \begin{tabular}{c}
$f(4)=2$ \\ 
$f(3)=-1$ \\
\end{tabular} \right. \Longleftrightarrow \left\lbrace \begin{tabular}{c}
$4a+b=2$ \\ 
$3a+b=-1$ \\
\end{tabular} \right.$

En opérant une soustraction des 2 lignes, on obtient 

$\Longleftrightarrow \left\lbrace \begin{tabular}{c}
$4a+b=2$ \\ 
$a=3$ \\
\end{tabular} \right.$

Par substitution,

$\Longleftrightarrow \left\lbrace \begin{tabular}{c}
$4\times 3 +b=2$ \\ 
a=3 \\
\end{tabular} \right. \Longleftrightarrow \left\lbrace \begin{tabular}{c}
$b=2-12$ \\ 
$a=3$ \\
\end{tabular} \right. \Longleftrightarrow \left\lbrace \begin{tabular}{c}
$b=-10$ \\ 
$a=3$ \\
\end{tabular} \right.$

Donc \fbox{$f(x)=3x-10$}



\subsection*{Avec Casio}

\begin{enumerate}
\item Aller dans le menu EQUA
\item Sélectionner F1 : Simultaneous
\item Écrire le nombre d'incinnues : ici 2.
\item On remplace les "0" par les nombres connus. La première colonnes est complétée par les coefficients de $a$, la deuxième par les coefficient de $b$ et la troisième par les constantes.

\begin{tabular}{ccc}
4 & 1 & 2 \\ 
3 & 1 & $-1$ \\ 
\end{tabular} 

\item Appuyer sur \touche{EXE}
\end{enumerate}

\newpage

\subsection*{Avec TI}

\begin{enumerate}
\item Cliquer sur \touche{Shift} \touche{MATRIX} et sélectionner la matrice \texttt{A}
\item Sélectionner \texttt{EDIT} avec les touches de direction

\item Remplacer $ 1 \times 1$ par $2 \times 2$. (2 lignes $\times$ 2 colonnes).
\item Écrire les coefficients de $a$ et de $b$. Voici la matrice \texttt{A}.

\begin{tabular}{cc}
 4 & 1 \\ 
3 & 1 \\ 
\end{tabular} 

\item Cliquer sur \touche{Shift} \touche{MATRIX} et sélectionner la \texttt{B} avec les touches de direction
\item Sélectionner \texttt{EDIT} avec les touches de direction

\item Remplacer $ 1 \times 1$ par $2 \times 1$. (2 lignes $\times$ 1 colonne).
\item Écrire les coefficients de $a$ et de $b$. Voici la matrice \texttt{B}.

\begin{tabular}{c}
2  \\ 
$-1$ \\ 
\end{tabular} 

\item \textbf{A ce stade là, les 2 matrices \texttt{A} et \texttt{B} sont renseignées.}

\item Appuyer sur  \touche{Shift} \touche{QUIT}

\item Cliquer sur \touche{Shift} \touche{MATRIX} \touche{EXE} \touche{$x^{-1}$} \touche{$\times$} \touche{Shift} \touche{MATRIX} \texttt{B} , sur l'écran, il doit s'afficher : [A]*[B]$^{-1}$.

\item Appuyer sur \touche{EXE}

\end{enumerate}




% Déterminer une équation de fonction affine - TICE
%
%
%\chapter{Fonctions de référence}
% 
%\impress{\impressionEleve}{\begin{titre}[Fonctions de référence]

\Titre{La fonction Carré}{4}
\end{titre}


\begin{CpsCol}
\begin{description}
\item[$\square$] Connaitre la fonction Carré : définition et courbes représentative.
\item[$\square$] Pour la fonction carré, résoudre graphiquement ou algébriquement une équation ou une inéquation du type $f(x) = k$, $f(x) < k$.
\item[$\square$] Étudier la parité d'une fonction dans des cas simples.
\end{description}
\end{CpsCol}

\Rec{1}{FR-4}


\begin{DefT}{Fonction Carré}\index{Fonctions!Carré}
La \textbf{fonction Carré} $f$ est la fonction définie sur $\R$ par $f(x)=x^2$.
\end{DefT}


\begin{DefT}{Représentation graphique} \index{Fonction Carré! Représentation graphique}\index{Parabole| see Fonction Carré! Représentation graphique}
La \textbf{représentation graphique} de la fonction Carré s'appelle une \textbf{parabole} et son équation est $y=x^2$. 
\end{DefT}



\begin{Pp}[Variations]
\begin{minipage}{0.48\linewidth}
La fonction Carré est strictement décroissante sur $\R^-$ et strictement croissante sur $\R^+$. 

La parabole d'équation $y=x^2$ est symétrique par rapport à l'axe des ordonnées.
\end{minipage}
\hfill
\begin{minipage}{0.48\linewidth}
\begin{tikzpicture}
\tkzTabInit[lgt=1,espcl=2]{ $x$ / 1,$f $ / 2}
{ $-\infty$ , $0$ ,$+\infty$}
\tkzTabVar{+/$ $,-/$0$,+/$ $ }
\end{tikzpicture}
\end{minipage}
\end{Pp}

\ROC

\newpage



\begin{minipage}{0.48\linewidth}
\EPC{1}{FR-0}{Représenter. Calculer}

\EPC{1}{FR-7}{Raisonner}

\EPC{1}{FR-2}{Représenter. Raisonner}

\EPC{1}{FR-3}{Représenter. Calculer. Raisonner}


\end{minipage}
\hfill
\begin{minipage}{0.48\linewidth}
\EPC{1}{FR-10}{Approfondissement : Raisonner. Calculer}

%\EPC{1}{FR-17}{Approfondissement : Raisonner. Calculer.}
\EPC{1}{FR-3}{Représenter. Calculer. Raisonner}

\EPCP{1}{FR-5}{Représenter. Raisonner.}

\end{minipage}






\begin{DefT}{Intervalle centré}
On dit qu'un intervalle $I$ est centré en 0 lorsque pour tout $x \in I$, $-x \in I$.
\end{DefT}


\begin{DefT}{Parité}
Soit $f$ une fonction définie sur un intervalle centré en 0. \\
On dit que $f$ est \textbf{paire} lorsque $f(x)=f(-x)$. \\
On dit que $f$ est \textbf{impaire} lorsque $f(x)=-f(-x)$. \\
\end{DefT}

\begin{ThT}{Parité}
Soit $f$ une fonction définie sur un intervalle centré en 0. \\
On dit que $f$ est \textbf{paire} si et seulement si sa courbe est symétrique par rapport à l'axe des ordonnées. \\
On dit que $f$ est \textbf{impaire} si et seulement si sa courbe est symétrique par rapport à l'origine du repère. 
\end{ThT}


\begin{minipage}{0.48\linewidth}
\begin{center}
$f$ est paire. 
\end{center}
\definecolor{ccqqqq}{rgb}{0.8,0.,0.}
\begin{tikzpicture}[line cap=round,line join=round,>=triangle 45,x=1.0cm,y=1.0cm]
\begin{axis}[
x=1.0cm,y=1.0cm,
axis lines=middle,
ymajorgrids=true,
xmajorgrids=true,
xmin=-2.303607619423333,
xmax=2.201540617377629,
ymin=-1.2009001594075965,
ymax=3.2524654815659826,
xtick={-2.0,-1.0,...,2.0},
ytick={-1.0,0.0,...,3.0},]
\clip(-2.303607619423333,-1.2009001594075965) rectangle (2.201540617377629,3.2524654815659826);
\draw[line width=2.pt,color=ccqqqq] (-1.9999966694488627,2.999986677806543) -- (-1.9999966694488627,2.999986677806543);
\draw[line width=2.pt,color=ccqqqq] (-1.9999966694488627,2.999986677806543) -- (-1.9899966850870758,2.9600868066575505);
\draw[line width=2.pt,color=ccqqqq] (-1.9899966850870758,2.9600868066575505) -- (-1.979996700725289,2.92038693488303);
\draw[line width=2.pt,color=ccqqqq] (-1.979996700725289,2.92038693488303) -- (-1.9699967163635022,2.8808870624829805);
\draw[line width=2.pt,color=ccqqqq] (-1.9699967163635022,2.8808870624829805) -- (-1.9599967320017153,2.8415871894574036);
\draw[line width=2.pt,color=ccqqqq] (-1.9599967320017153,2.8415871894574036) -- (-1.9499967476399285,2.802487315806299);
\draw[line width=2.pt,color=ccqqqq] (-1.9499967476399285,2.802487315806299) -- (-1.9399967632781416,2.763587441529666);
\draw[line width=2.pt,color=ccqqqq] (-1.9399967632781416,2.763587441529666) -- (-1.9299967789163548,2.7248875666275048);
\draw[line width=2.pt,color=ccqqqq] (-1.9299967789163548,2.7248875666275048) -- (-1.919996794554568,2.686387691099816);
\draw[line width=2.pt,color=ccqqqq] (-1.919996794554568,2.686387691099816) -- (-1.9099968101927811,2.6480878149465985);
\draw[line width=2.pt,color=ccqqqq] (-1.9099968101927811,2.6480878149465985) -- (-1.8999968258309943,2.6099879381678535);
\draw[line width=2.pt,color=ccqqqq] (-1.8999968258309943,2.6099879381678535) -- (-1.8899968414692074,2.5720880607635803);
\draw[line width=2.pt,color=ccqqqq] (-1.8899968414692074,2.5720880607635803) -- (-1.8799968571074206,2.534388182733779);
\draw[line width=2.pt,color=ccqqqq] (-1.8799968571074206,2.534388182733779) -- (-1.8699968727456338,2.4968883040784497);
\draw[line width=2.pt,color=ccqqqq] (-1.8699968727456338,2.4968883040784497) -- (-1.859996888383847,2.4595884247975928);
\draw[line width=2.pt,color=ccqqqq] (-1.859996888383847,2.4595884247975928) -- (-1.84999690402206,2.4224885448912072);
\draw[line width=2.pt,color=ccqqqq] (-1.84999690402206,2.4224885448912072) -- (-1.8399969196602732,2.385588664359294);
\draw[line width=2.pt,color=ccqqqq] (-1.8399969196602732,2.385588664359294) -- (-1.8299969352984864,2.348888783201853);
\draw[line width=2.pt,color=ccqqqq] (-1.8299969352984864,2.348888783201853) -- (-1.8199969509366996,2.312388901418883);
\draw[line width=2.pt,color=ccqqqq] (-1.8199969509366996,2.312388901418883) -- (-1.8099969665749127,2.2760890190103855);
\draw[line width=2.pt,color=ccqqqq] (-1.8099969665749127,2.2760890190103855) -- (-1.7999969822131259,2.2399891359763604);
\draw[line width=2.pt,color=ccqqqq] (-1.7999969822131259,2.2399891359763604) -- (-1.789996997851339,2.2040892523168067);
\draw[line width=2.pt,color=ccqqqq] (-1.789996997851339,2.2040892523168067) -- (-1.7799970134895522,2.168389368031725);
\draw[line width=2.pt,color=ccqqqq] (-1.7799970134895522,2.168389368031725) -- (-1.7699970291277654,2.1328894831211156);
\draw[line width=2.pt,color=ccqqqq] (-1.7699970291277654,2.1328894831211156) -- (-1.7599970447659785,2.0975895975849776);
\draw[line width=2.pt,color=ccqqqq] (-1.7599970447659785,2.0975895975849776) -- (-1.7499970604041917,2.062489711423312);
\draw[line width=2.pt,color=ccqqqq] (-1.7499970604041917,2.062489711423312) -- (-1.7399970760424048,2.0275898246361184);
\draw[line width=2.pt,color=ccqqqq] (-1.7399970760424048,2.0275898246361184) -- (-1.729997091680618,1.9928899372233966);
\draw[line width=2.pt,color=ccqqqq] (-1.729997091680618,1.9928899372233966) -- (-1.7199971073188312,1.9583900491851467);
\draw[line width=2.pt,color=ccqqqq] (-1.7199971073188312,1.9583900491851467) -- (-1.7099971229570443,1.9240901605213687);
\draw[line width=2.pt,color=ccqqqq] (-1.7099971229570443,1.9240901605213687) -- (-1.6999971385952575,1.889990271232063);
\draw[line width=2.pt,color=ccqqqq] (-1.6999971385952575,1.889990271232063) -- (-1.6899971542334706,1.856090381317229);
\draw[line width=2.pt,color=ccqqqq] (-1.6899971542334706,1.856090381317229) -- (-1.6799971698716838,1.822390490776867);
\draw[line width=2.pt,color=ccqqqq] (-1.6799971698716838,1.822390490776867) -- (-1.669997185509897,1.7888905996109772);
\draw[line width=2.pt,color=ccqqqq] (-1.669997185509897,1.7888905996109772) -- (-1.6599972011481101,1.7555907078195592);
\draw[line width=2.pt,color=ccqqqq] (-1.6599972011481101,1.7555907078195592) -- (-1.6499972167863233,1.722490815402613);
\draw[line width=2.pt,color=ccqqqq] (-1.6499972167863233,1.722490815402613) -- (-1.6399972324245364,1.6895909223601389);
\draw[line width=2.pt,color=ccqqqq] (-1.6399972324245364,1.6895909223601389) -- (-1.6299972480627496,1.656891028692137);
\draw[line width=2.pt,color=ccqqqq] (-1.6299972480627496,1.656891028692137) -- (-1.6199972637009628,1.6243911343986066);
\draw[line width=2.pt,color=ccqqqq] (-1.6199972637009628,1.6243911343986066) -- (-1.609997279339176,1.5920912394795486);
\draw[line width=2.pt,color=ccqqqq] (-1.609997279339176,1.5920912394795486) -- (-1.599997294977389,1.559991343934962);
\draw[line width=2.pt,color=ccqqqq] (-1.599997294977389,1.559991343934962) -- (-1.5899973106156022,1.5280914477648477);
\draw[line width=2.pt,color=ccqqqq] (-1.5899973106156022,1.5280914477648477) -- (-1.5799973262538154,1.4963915509692054);
\draw[line width=2.pt,color=ccqqqq] (-1.5799973262538154,1.4963915509692054) -- (-1.5699973418920286,1.4648916535480354);
\draw[line width=2.pt,color=ccqqqq] (-1.5699973418920286,1.4648916535480354) -- (-1.5599973575302417,1.4335917555013369);
\draw[line width=2.pt,color=ccqqqq] (-1.5599973575302417,1.4335917555013369) -- (-1.5499973731684549,1.4024918568291103);
\draw[line width=2.pt,color=ccqqqq] (-1.5499973731684549,1.4024918568291103) -- (-1.539997388806668,1.371591957531356);
\draw[line width=2.pt,color=ccqqqq] (-1.539997388806668,1.371591957531356) -- (-1.5299974044448812,1.3408920576080732);
\draw[line width=2.pt,color=ccqqqq] (-1.5299974044448812,1.3408920576080732) -- (-1.5199974200830944,1.3103921570592627);
\draw[line width=2.pt,color=ccqqqq] (-1.5199974200830944,1.3103921570592627) -- (-1.5099974357213075,1.2800922558849241);
\draw[line width=2.pt,color=ccqqqq] (-1.5099974357213075,1.2800922558849241) -- (-1.4999974513595207,1.2499923540850575);
\draw[line width=2.pt,color=ccqqqq] (-1.4999974513595207,1.2499923540850575) -- (-1.4899974669977338,1.2200924516596627);
\draw[line width=2.pt,color=ccqqqq] (-1.4899974669977338,1.2200924516596627) -- (-1.479997482635947,1.1903925486087403);
\draw[line width=2.pt,color=ccqqqq] (-1.479997482635947,1.1903925486087403) -- (-1.4699974982741602,1.1608926449322894);
\draw[line width=2.pt,color=ccqqqq] (-1.4699974982741602,1.1608926449322894) -- (-1.4599975139123733,1.1315927406303108);
\draw[line width=2.pt,color=ccqqqq] (-1.4599975139123733,1.1315927406303108) -- (-1.4499975295505865,1.102492835702804);
\draw[line width=2.pt,color=ccqqqq] (-1.4499975295505865,1.102492835702804) -- (-1.4399975451887996,1.0735929301497689);
\draw[line width=2.pt,color=ccqqqq] (-1.4399975451887996,1.0735929301497689) -- (-1.4299975608270128,1.044893023971206);
\draw[line width=2.pt,color=ccqqqq] (-1.4299975608270128,1.044893023971206) -- (-1.419997576465226,1.0163931171671154);
\draw[line width=2.pt,color=ccqqqq] (-1.419997576465226,1.0163931171671154) -- (-1.4099975921034391,0.9880932097374964);
\draw[line width=2.pt,color=ccqqqq] (-1.4099975921034391,0.9880932097374964) -- (-1.3999976077416523,0.9599933016823492);
\draw[line width=2.pt,color=ccqqqq] (-1.3999976077416523,0.9599933016823492) -- (-1.3899976233798654,0.9320933930016742);
\draw[line width=2.pt,color=ccqqqq] (-1.3899976233798654,0.9320933930016742) -- (-1.3799976390180786,0.9043934836954712);
\draw[line width=2.pt,color=ccqqqq] (-1.3799976390180786,0.9043934836954712) -- (-1.3699976546562918,0.87689357376374);
\draw[line width=2.pt,color=ccqqqq] (-1.3699976546562918,0.87689357376374) -- (-1.359997670294505,0.8495936632064809);
\draw[line width=2.pt,color=ccqqqq] (-1.359997670294505,0.8495936632064809) -- (-1.349997685932718,0.8224937520236937);
\draw[line width=2.pt,color=ccqqqq] (-1.349997685932718,0.8224937520236937) -- (-1.3399977015709312,0.7955938402153784);
\draw[line width=2.pt,color=ccqqqq] (-1.3399977015709312,0.7955938402153784) -- (-1.3299977172091444,0.7688939277815352);
\draw[line width=2.pt,color=ccqqqq] (-1.3299977172091444,0.7688939277815352) -- (-1.3199977328473576,0.7423940147221639);
\draw[line width=2.pt,color=ccqqqq] (-1.3199977328473576,0.7423940147221639) -- (-1.3099977484855707,0.7160941010372646);
\draw[line width=2.pt,color=ccqqqq] (-1.3099977484855707,0.7160941010372646) -- (-1.2999977641237839,0.6899941867268371);
\draw[line width=2.pt,color=ccqqqq] (-1.2999977641237839,0.6899941867268371) -- (-1.289997779761997,0.6640942717908818);
\draw[line width=2.pt,color=ccqqqq] (-1.289997779761997,0.6640942717908818) -- (-1.2799977954002102,0.6383943562293983);
\draw[line width=2.pt,color=ccqqqq] (-1.2799977954002102,0.6383943562293983) -- (-1.2699978110384234,0.6128944400423868);
\draw[line width=2.pt,color=ccqqqq] (-1.2699978110384234,0.6128944400423868) -- (-1.2599978266766365,0.5875945232298474);
\draw[line width=2.pt,color=ccqqqq] (-1.2599978266766365,0.5875945232298474) -- (-1.2499978423148497,0.5624946057917797);
\draw[line width=2.pt,color=ccqqqq] (-1.2499978423148497,0.5624946057917797) -- (-1.2399978579530628,0.5375946877281841);
\draw[line width=2.pt,color=ccqqqq] (-1.2399978579530628,0.5375946877281841) -- (-1.229997873591276,0.5128947690390606);
\draw[line width=2.pt,color=ccqqqq] (-1.229997873591276,0.5128947690390606) -- (-1.2199978892294892,0.48839484972440883);
\draw[line width=2.pt,color=ccqqqq] (-1.2199978892294892,0.48839484972440883) -- (-1.2099979048677023,0.46409492978422917);
\draw[line width=2.pt,color=ccqqqq] (-1.2099979048677023,0.46409492978422917) -- (-1.1999979205059155,0.4399950092185214);
\draw[line width=2.pt,color=ccqqqq] (-1.1999979205059155,0.4399950092185214) -- (-1.1899979361441286,0.41609508802728556);
\draw[line width=2.pt,color=ccqqqq] (-1.1899979361441286,0.41609508802728556) -- (-1.1799979517823418,0.39239516621052184);
\draw[line width=2.pt,color=ccqqqq] (-1.1799979517823418,0.39239516621052184) -- (-1.169997967420555,0.36889524376823);
\draw[line width=2.pt,color=ccqqqq] (-1.169997967420555,0.36889524376823) -- (-1.1599979830587681,0.3455953207004101);
\draw[line width=2.pt,color=ccqqqq] (-1.1599979830587681,0.3455953207004101) -- (-1.1499979986969813,0.3224953970070621);
\draw[line width=2.pt,color=ccqqqq] (-1.1499979986969813,0.3224953970070621) -- (-1.1399980143351944,0.2995954726881862);
\draw[line width=2.pt,color=ccqqqq] (-1.1399980143351944,0.2995954726881862) -- (-1.1299980299734076,0.2768955477437822);
\draw[line width=2.pt,color=ccqqqq] (-1.1299980299734076,0.2768955477437822) -- (-1.1199980456116208,0.25439562217385014);
\draw[line width=2.pt,color=ccqqqq] (-1.1199980456116208,0.25439562217385014) -- (-1.109998061249834,0.23209569597838997);
\draw[line width=2.pt,color=ccqqqq] (-1.109998061249834,0.23209569597838997) -- (-1.099998076888047,0.20999576915740192);
\draw[line width=2.pt,color=ccqqqq] (-1.099998076888047,0.20999576915740192) -- (-1.0899980925262602,0.18809584171088578);
\draw[line width=2.pt,color=ccqqqq] (-1.0899980925262602,0.18809584171088578) -- (-1.0799981081644734,0.16639591363884154);
\draw[line width=2.pt,color=ccqqqq] (-1.0799981081644734,0.16639591363884154) -- (-1.0699981238026866,0.14489598494126943);
\draw[line width=2.pt,color=ccqqqq] (-1.0699981238026866,0.14489598494126943) -- (-1.0599981394408997,0.123596055618169);
\draw[line width=2.pt,color=ccqqqq] (-1.0599981394408997,0.123596055618169) -- (-1.0499981550791129,0.1024961256695407);
\draw[line width=2.pt,color=ccqqqq] (-1.0499981550791129,0.1024961256695407) -- (-1.039998170717326,0.08159619509538452);
\draw[line width=2.pt,color=ccqqqq] (-1.039998170717326,0.08159619509538452) -- (-1.0299981863555392,0.06089626389570002);
\draw[line width=2.pt,color=ccqqqq] (-1.0299981863555392,0.06089626389570002) -- (-1.0199982019937524,0.04039633207048765);
\draw[line width=2.pt,color=ccqqqq] (-1.0199982019937524,0.04039633207048765) -- (-1.0099982176319655,0.020096399619747185);
\draw[line width=2.pt,color=ccqqqq] (-1.0099982176319655,0.020096399619747185) -- (-0.9999982332701788,0.0);
\draw[line width=2.pt,color=ccqqqq] (-0.9999982332701788,0.0) -- (-0.989998248908392,-0.019903467158317367);
\draw[line width=2.pt,color=ccqqqq] (-0.989998248908392,-0.019903467158317367) -- (-0.9799982645466053,-0.03960340148564179);
\draw[line width=2.pt,color=ccqqqq] (-0.9799982645466053,-0.03960340148564179) -- (-0.9699982801848186,-0.05910333643849419);
\draw[line width=2.pt,color=ccqqqq] (-0.9699982801848186,-0.05910333643849419) -- (-0.9599982958230319,-0.07840327201687458);
\draw[line width=2.pt,color=ccqqqq] (-0.9599982958230319,-0.07840327201687458) -- (-0.9499983114612451,-0.09750320822078307);
\draw[line width=2.pt,color=ccqqqq] (-0.9499983114612451,-0.09750320822078307) -- (-0.9399983270994584,-0.11640314505021965);
\draw[line width=2.pt,color=ccqqqq] (-0.9399983270994584,-0.11640314505021965) -- (-0.9299983427376717,-0.1351030825051842);
\draw[line width=2.pt,color=ccqqqq] (-0.9299983427376717,-0.1351030825051842) -- (-0.919998358375885,-0.15360302058567676);
\draw[line width=2.pt,color=ccqqqq] (-0.919998358375885,-0.15360302058567676) -- (-0.9099983740140982,-0.1719029592916974);
\draw[line width=2.pt,color=ccqqqq] (-0.9099983740140982,-0.1719029592916974) -- (-0.8999983896523115,-0.19000289862324604);
\draw[line width=2.pt,color=ccqqqq] (-0.8999983896523115,-0.19000289862324604) -- (-0.8899984052905248,-0.20790283858032277);
\draw[line width=2.pt,color=ccqqqq] (-0.8899984052905248,-0.20790283858032277) -- (-0.879998420928738,-0.2256027791629276);
\draw[line width=2.pt,color=ccqqqq] (-0.879998420928738,-0.2256027791629276) -- (-0.8699984365669513,-0.2431027203710604);
\draw[line width=2.pt,color=ccqqqq] (-0.8699984365669513,-0.2431027203710604) -- (-0.8599984522051646,-0.2604026622047213);
\draw[line width=2.pt,color=ccqqqq] (-0.8599984522051646,-0.2604026622047213) -- (-0.8499984678433778,-0.2775026046639102);
\draw[line width=2.pt,color=ccqqqq] (-0.8499984678433778,-0.2775026046639102) -- (-0.8399984834815911,-0.29440254774862706);
\draw[line width=2.pt,color=ccqqqq] (-0.8399984834815911,-0.29440254774862706) -- (-0.8299984991198044,-0.31110249145887203);
\draw[line width=2.pt,color=ccqqqq] (-0.8299984991198044,-0.31110249145887203) -- (-0.8199985147580177,-0.3276024357946451);
\draw[line width=2.pt,color=ccqqqq] (-0.8199985147580177,-0.3276024357946451) -- (-0.8099985303962309,-0.34390238075594615);
\draw[line width=2.pt,color=ccqqqq] (-0.8099985303962309,-0.34390238075594615) -- (-0.7999985460344442,-0.3600023263427753);
\draw[line width=2.pt,color=ccqqqq] (-0.7999985460344442,-0.3600023263427753) -- (-0.7899985616726575,-0.3759022725551324);
\draw[line width=2.pt,color=ccqqqq] (-0.7899985616726575,-0.3759022725551324) -- (-0.7799985773108707,-0.39160221939301765);
\draw[line width=2.pt,color=ccqqqq] (-0.7799985773108707,-0.39160221939301765) -- (-0.769998592949084,-0.40710216685643086);
\draw[line width=2.pt,color=ccqqqq] (-0.769998592949084,-0.40710216685643086) -- (-0.7599986085872973,-0.42240211494537205);
\draw[line width=2.pt,color=ccqqqq] (-0.7599986085872973,-0.42240211494537205) -- (-0.7499986242255106,-0.43750206365984146);
\draw[line width=2.pt,color=ccqqqq] (-0.7499986242255106,-0.43750206365984146) -- (-0.7399986398637238,-0.45240201299983873);
\draw[line width=2.pt,color=ccqqqq] (-0.7399986398637238,-0.45240201299983873) -- (-0.7299986555019371,-0.4671019629653642);
\draw[line width=2.pt,color=ccqqqq] (-0.7299986555019371,-0.4671019629653642) -- (-0.7199986711401504,-0.48160191355641757);
\draw[line width=2.pt,color=ccqqqq] (-0.7199986711401504,-0.48160191355641757) -- (-0.7099986867783636,-0.495901864772999);
\draw[line width=2.pt,color=ccqqqq] (-0.7099986867783636,-0.495901864772999) -- (-0.6999987024165769,-0.5100018166151086);
\draw[line width=2.pt,color=ccqqqq] (-0.6999987024165769,-0.5100018166151086) -- (-0.6899987180547902,-0.5239017690827461);
\draw[line width=2.pt,color=ccqqqq] (-0.6899987180547902,-0.5239017690827461) -- (-0.6799987336930035,-0.5376017221759117);
\draw[line width=2.pt,color=ccqqqq] (-0.6799987336930035,-0.5376017221759117) -- (-0.6699987493312167,-0.5511016758946055);
\draw[line width=2.pt,color=ccqqqq] (-0.6699987493312167,-0.5511016758946055) -- (-0.65999876496943,-0.5644016302388271);
\draw[line width=2.pt,color=ccqqqq] (-0.65999876496943,-0.5644016302388271) -- (-0.6499987806076433,-0.5775015852085769);
\draw[line width=2.pt,color=ccqqqq] (-0.6499987806076433,-0.5775015852085769) -- (-0.6399987962458565,-0.5904015408038545);
\draw[line width=2.pt,color=ccqqqq] (-0.6399987962458565,-0.5904015408038545) -- (-0.6299988118840698,-0.6031014970246604);
\draw[line width=2.pt,color=ccqqqq] (-0.6299988118840698,-0.6031014970246604) -- (-0.6199988275222831,-0.6156014538709943);
\draw[line width=2.pt,color=ccqqqq] (-0.6199988275222831,-0.6156014538709943) -- (-0.6099988431604964,-0.6279014113428562);
\draw[line width=2.pt,color=ccqqqq] (-0.6099988431604964,-0.6279014113428562) -- (-0.5999988587987096,-0.6400013694402461);
\draw[line width=2.pt,color=ccqqqq] (-0.5999988587987096,-0.6400013694402461) -- (-0.5899988744369229,-0.651901328163164);
\draw[line width=2.pt,color=ccqqqq] (-0.5899988744369229,-0.651901328163164) -- (-0.5799988900751362,-0.6636012875116102);
\draw[line width=2.pt,color=ccqqqq] (-0.5799988900751362,-0.6636012875116102) -- (-0.5699989057133494,-0.6751012474855842);
\draw[line width=2.pt,color=ccqqqq] (-0.5699989057133494,-0.6751012474855842) -- (-0.5599989213515627,-0.6864012080850863);
\draw[line width=2.pt,color=ccqqqq] (-0.5599989213515627,-0.6864012080850863) -- (-0.549998936989776,-0.6975011693101164);
\draw[line width=2.pt,color=ccqqqq] (-0.549998936989776,-0.6975011693101164) -- (-0.5399989526279892,-0.7084011311606746);
\draw[line width=2.pt,color=ccqqqq] (-0.5399989526279892,-0.7084011311606746) -- (-0.5299989682662025,-0.7191010936367608);
\draw[line width=2.pt,color=ccqqqq] (-0.5299989682662025,-0.7191010936367608) -- (-0.5199989839044158,-0.729601056738375);
\draw[line width=2.pt,color=ccqqqq] (-0.5199989839044158,-0.729601056738375) -- (-0.5099989995426291,-0.7399010204655174);
\draw[line width=2.pt,color=ccqqqq] (-0.5099989995426291,-0.7399010204655174) -- (-0.4999990151808423,-0.7500009848181879);
\draw[line width=2.pt,color=ccqqqq] (-0.4999990151808423,-0.7500009848181879) -- (-0.4899990308190555,-0.7599009497963863);
\draw[line width=2.pt,color=ccqqqq] (-0.4899990308190555,-0.7599009497963863) -- (-0.4799990464572687,-0.7696009154001128);
\draw[line width=2.pt,color=ccqqqq] (-0.4799990464572687,-0.7696009154001128) -- (-0.4699990620954819,-0.7791008816293673);
\draw[line width=2.pt,color=ccqqqq] (-0.4699990620954819,-0.7791008816293673) -- (-0.45999907773369514,-0.7884008484841499);
\draw[line width=2.pt,color=ccqqqq] (-0.45999907773369514,-0.7884008484841499) -- (-0.44999909337190835,-0.7975008159644605);
\draw[line width=2.pt,color=ccqqqq] (-0.44999909337190835,-0.7975008159644605) -- (-0.43999910901012157,-0.8064007840702991);
\draw[line width=2.pt,color=ccqqqq] (-0.43999910901012157,-0.8064007840702991) -- (-0.4299991246483348,-0.8151007528016658);
\draw[line width=2.pt,color=ccqqqq] (-0.4299991246483348,-0.8151007528016658) -- (-0.419999140286548,-0.8236007221585606);
\draw[line width=2.pt,color=ccqqqq] (-0.419999140286548,-0.8236007221585606) -- (-0.4099991559247612,-0.8319006921409833);
\draw[line width=2.pt,color=ccqqqq] (-0.4099991559247612,-0.8319006921409833) -- (-0.39999917156297443,-0.8400006627489341);
\draw[line width=2.pt,color=ccqqqq] (-0.39999917156297443,-0.8400006627489341) -- (-0.38999918720118765,-0.847900633982413);
\draw[line width=2.pt,color=ccqqqq] (-0.38999918720118765,-0.847900633982413) -- (-0.37999920283940086,-0.8556006058414198);
\draw[line width=2.pt,color=ccqqqq] (-0.37999920283940086,-0.8556006058414198) -- (-0.3699992184776141,-0.8631005783259548);
\draw[line width=2.pt,color=ccqqqq] (-0.3699992184776141,-0.8631005783259548) -- (-0.3599992341158273,-0.8704005514360178);
\draw[line width=2.pt,color=ccqqqq] (-0.3599992341158273,-0.8704005514360178) -- (-0.3499992497540405,-0.8775005251716088);
\draw[line width=2.pt,color=ccqqqq] (-0.3499992497540405,-0.8775005251716088) -- (-0.3399992653922537,-0.8844004995327278);
\draw[line width=2.pt,color=ccqqqq] (-0.3399992653922537,-0.8844004995327278) -- (-0.32999928103046694,-0.8911004745193749);
\draw[line width=2.pt,color=ccqqqq] (-0.32999928103046694,-0.8911004745193749) -- (-0.31999929666868016,-0.89760045013155);
\draw[line width=2.pt,color=ccqqqq] (-0.31999929666868016,-0.89760045013155) -- (-0.30999931230689337,-0.9039004263692532);
\draw[line width=2.pt,color=ccqqqq] (-0.30999931230689337,-0.9039004263692532) -- (-0.2999993279451066,-0.9100004032324844);
\draw[line width=2.pt,color=ccqqqq] (-0.2999993279451066,-0.9100004032324844) -- (-0.2899993435833198,-0.9159003807212436);
\draw[line width=2.pt,color=ccqqqq] (-0.2899993435833198,-0.9159003807212436) -- (-0.279999359221533,-0.9216003588355309);
\draw[line width=2.pt,color=ccqqqq] (-0.279999359221533,-0.9216003588355309) -- (-0.26999937485974623,-0.9271003375753463);
\draw[line width=2.pt,color=ccqqqq] (-0.26999937485974623,-0.9271003375753463) -- (-0.25999939049795945,-0.9324003169406896);
\draw[line width=2.pt,color=ccqqqq] (-0.25999939049795945,-0.9324003169406896) -- (-0.24999940613617266,-0.937500296931561);
\draw[line width=2.pt,color=ccqqqq] (-0.24999940613617266,-0.937500296931561) -- (-0.23999942177438588,-0.9424002775479604);
\draw[line width=2.pt,color=ccqqqq] (-0.23999942177438588,-0.9424002775479604) -- (-0.2299994374125991,-0.9471002587898879);
\draw[line width=2.pt,color=ccqqqq] (-0.2299994374125991,-0.9471002587898879) -- (-0.2199994530508123,-0.9516002406573434);
\draw[line width=2.pt,color=ccqqqq] (-0.2199994530508123,-0.9516002406573434) -- (-0.20999946868902553,-0.955900223150327);
\draw[line width=2.pt,color=ccqqqq] (-0.20999946868902553,-0.955900223150327) -- (-0.19999948432723874,-0.9600002062688386);
\draw[line width=2.pt,color=ccqqqq] (-0.19999948432723874,-0.9600002062688386) -- (-0.18999949996545196,-0.9639001900128782);
\draw[line width=2.pt,color=ccqqqq] (-0.18999949996545196,-0.9639001900128782) -- (-0.17999951560366517,-0.9676001743824459);
\draw[line width=2.pt,color=ccqqqq] (-0.17999951560366517,-0.9676001743824459) -- (-0.1699995312418784,-0.9711001593775416);
\draw[line width=2.pt,color=ccqqqq] (-0.1699995312418784,-0.9711001593775416) -- (-0.1599995468800916,-0.9744001449981654);
\draw[line width=2.pt,color=ccqqqq] (-0.1599995468800916,-0.9744001449981654) -- (-0.14999956251830482,-0.9775001312443171);
\draw[line width=2.pt,color=ccqqqq] (-0.14999956251830482,-0.9775001312443171) -- (-0.13999957815651803,-0.980400118115997);
\draw[line width=2.pt,color=ccqqqq] (-0.13999957815651803,-0.980400118115997) -- (-0.12999959379473125,-0.9831001056132048);
\draw[line width=2.pt,color=ccqqqq] (-0.12999959379473125,-0.9831001056132048) -- (-0.11999960943294448,-0.9856000937359408);
\draw[line width=2.pt,color=ccqqqq] (-0.11999960943294448,-0.9856000937359408) -- (-0.10999962507115771,-0.9879000824842047);
\draw[line width=2.pt,color=ccqqqq] (-0.10999962507115771,-0.9879000824842047) -- (-0.09999964070937094,-0.9900000718579968);
\draw[line width=2.pt,color=ccqqqq] (-0.09999964070937094,-0.9900000718579968) -- (-0.08999965634758417,-0.9919000618573167);
\draw[line width=2.pt,color=ccqqqq] (-0.08999965634758417,-0.9919000618573167) -- (-0.0799996719857974,-0.9936000524821649);
\draw[line width=2.pt,color=ccqqqq] (-0.0799996719857974,-0.9936000524821649) -- (-0.06999968762401063,-0.9951000437325409);
\draw[line width=2.pt,color=ccqqqq] (-0.06999968762401063,-0.9951000437325409) -- (-0.059999703262223855,-0.996400035608445);
\draw[line width=2.pt,color=ccqqqq] (-0.059999703262223855,-0.996400035608445) -- (-0.049999718900437085,-0.9975000281098773);
\draw[line width=2.pt,color=ccqqqq] (-0.049999718900437085,-0.9975000281098773) -- (-0.039999734538650314,-0.9984000212368375);
\draw[line width=2.pt,color=ccqqqq] (-0.039999734538650314,-0.9984000212368375) -- (-0.029999750176863543,-0.9991000149893258);
\draw[line width=2.pt,color=ccqqqq] (-0.029999750176863543,-0.9991000149893258) -- (-0.019999765815076773,-0.9996000093673421);
\draw[line width=2.pt,color=ccqqqq] (-0.019999765815076773,-0.9996000093673421) -- (-0.00999978145329,-0.9999000043708864);
\draw[line width=2.pt,color=ccqqqq] (-0.00999978145329,-0.9999000043708864) -- (0.0,-0.9999999999999588);
\draw[line width=2.pt,color=ccqqqq] (0.0,-0.9999999999999588) -- (0.010000187270283544,-0.9998999962545593);
\draw[line width=2.pt,color=ccqqqq] (0.010000187270283544,-0.9998999962545593) -- (0.020000171632070317,-0.9995999931346877);
\draw[line width=2.pt,color=ccqqqq] (0.020000171632070317,-0.9995999931346877) -- (0.030000155993857087,-0.9990999906403443);
\draw[line width=2.pt,color=ccqqqq] (0.030000155993857087,-0.9990999906403443) -- (0.04000014035564386,-0.9983999887715288);
\draw[line width=2.pt,color=ccqqqq] (0.04000014035564386,-0.9983999887715288) -- (0.05000012471743063,-0.9974999875282414);
\draw[line width=2.pt,color=ccqqqq] (0.05000012471743063,-0.9974999875282414) -- (0.0600001090792174,-0.996399986910482);
\draw[line width=2.pt,color=ccqqqq] (0.0600001090792174,-0.996399986910482) -- (0.07000009344100418,-0.9950999869182506);
\draw[line width=2.pt,color=ccqqqq] (0.07000009344100418,-0.9950999869182506) -- (0.08000007780279095,-0.9935999875515474);
\draw[line width=2.pt,color=ccqqqq] (0.08000007780279095,-0.9935999875515474) -- (0.09000006216457772,-0.9918999888103721);
\draw[line width=2.pt,color=ccqqqq] (0.09000006216457772,-0.9918999888103721) -- (0.10000004652636449,-0.9899999906947249);
\draw[line width=2.pt,color=ccqqqq] (0.10000004652636449,-0.9899999906947249) -- (0.11000003088815126,-0.9878999932046058);
\draw[line width=2.pt,color=ccqqqq] (0.11000003088815126,-0.9878999932046058) -- (0.12000001524993803,-0.9855999963400146);
\draw[line width=2.pt,color=ccqqqq] (0.12000001524993803,-0.9855999963400146) -- (0.12999999961172481,-0.9831000001009516);
\draw[line width=2.pt,color=ccqqqq] (0.12999999961172481,-0.9831000001009516) -- (0.1399999839735116,-0.9804000044874165);
\draw[line width=2.pt,color=ccqqqq] (0.1399999839735116,-0.9804000044874165) -- (0.14999996833529838,-0.9775000094994095);
\draw[line width=2.pt,color=ccqqqq] (0.14999996833529838,-0.9775000094994095) -- (0.15999995269708517,-0.9744000151369305);
\draw[line width=2.pt,color=ccqqqq] (0.15999995269708517,-0.9744000151369305) -- (0.16999993705887195,-0.9711000213999795);
\draw[line width=2.pt,color=ccqqqq] (0.16999993705887195,-0.9711000213999795) -- (0.17999992142065874,-0.9676000282885566);
\draw[line width=2.pt,color=ccqqqq] (0.17999992142065874,-0.9676000282885566) -- (0.18999990578244552,-0.9639000358026618);
\draw[line width=2.pt,color=ccqqqq] (0.18999990578244552,-0.9639000358026618) -- (0.1999998901442323,-0.960000043942295);
\draw[line width=2.pt,color=ccqqqq] (0.1999998901442323,-0.960000043942295) -- (0.2099998745060191,-0.9559000527074563);
\draw[line width=2.pt,color=ccqqqq] (0.2099998745060191,-0.9559000527074563) -- (0.21999985886780588,-0.9516000620981455);
\draw[line width=2.pt,color=ccqqqq] (0.21999985886780588,-0.9516000620981455) -- (0.22999984322959266,-0.9471000721143628);
\draw[line width=2.pt,color=ccqqqq] (0.22999984322959266,-0.9471000721143628) -- (0.23999982759137944,-0.9424000827561081);
\draw[line width=2.pt,color=ccqqqq] (0.23999982759137944,-0.9424000827561081) -- (0.24999981195316623,-0.9375000940233815);
\draw[line width=2.pt,color=ccqqqq] (0.24999981195316623,-0.9375000940233815) -- (0.259999796314953,-0.9324001059161829);
\draw[line width=2.pt,color=ccqqqq] (0.259999796314953,-0.9324001059161829) -- (0.2699997806767398,-0.9271001184345125);
\draw[line width=2.pt,color=ccqqqq] (0.2699997806767398,-0.9271001184345125) -- (0.2799997650385266,-0.92160013157837);
\draw[line width=2.pt,color=ccqqqq] (0.2799997650385266,-0.92160013157837) -- (0.28999974940031337,-0.9159001453477554);
\draw[line width=2.pt,color=ccqqqq] (0.28999974940031337,-0.9159001453477554) -- (0.29999973376210015,-0.910000159742669);
\draw[line width=2.pt,color=ccqqqq] (0.29999973376210015,-0.910000159742669) -- (0.30999971812388694,-0.9039001747631107);
\draw[line width=2.pt,color=ccqqqq] (0.30999971812388694,-0.9039001747631107) -- (0.3199997024856737,-0.8976001904090803);
\draw[line width=2.pt,color=ccqqqq] (0.3199997024856737,-0.8976001904090803) -- (0.3299996868474605,-0.891100206680578);
\draw[line width=2.pt,color=ccqqqq] (0.3299996868474605,-0.891100206680578) -- (0.3399996712092473,-0.8844002235776037);
\draw[line width=2.pt,color=ccqqqq] (0.3399996712092473,-0.8844002235776037) -- (0.3499996555710341,-0.8775002411001576);
\draw[line width=2.pt,color=ccqqqq] (0.3499996555710341,-0.8775002411001576) -- (0.35999963993282086,-0.8704002592482394);
\draw[line width=2.pt,color=ccqqqq] (0.35999963993282086,-0.8704002592482394) -- (0.36999962429460764,-0.8631002780218492);
\draw[line width=2.pt,color=ccqqqq] (0.36999962429460764,-0.8631002780218492) -- (0.3799996086563944,-0.8556002974209871);
\draw[line width=2.pt,color=ccqqqq] (0.3799996086563944,-0.8556002974209871) -- (0.3899995930181812,-0.847900317445653);
\draw[line width=2.pt,color=ccqqqq] (0.3899995930181812,-0.847900317445653) -- (0.399999577379968,-0.840000338095847);
\draw[line width=2.pt,color=ccqqqq] (0.399999577379968,-0.840000338095847) -- (0.4099995617417548,-0.831900359371569);
\draw[line width=2.pt,color=ccqqqq] (0.4099995617417548,-0.831900359371569) -- (0.41999954610354157,-0.8236003812728191);
\draw[line width=2.pt,color=ccqqqq] (0.41999954610354157,-0.8236003812728191) -- (0.42999953046532835,-0.8151004037995971);
\draw[line width=2.pt,color=ccqqqq] (0.42999953046532835,-0.8151004037995971) -- (0.43999951482711513,-0.8064004269519033);
\draw[line width=2.pt,color=ccqqqq] (0.43999951482711513,-0.8064004269519033) -- (0.4499994991889019,-0.7975004507297374);
\draw[line width=2.pt,color=ccqqqq] (0.4499994991889019,-0.7975004507297374) -- (0.4599994835506887,-0.7884004751330996);
\draw[line width=2.pt,color=ccqqqq] (0.4599994835506887,-0.7884004751330996) -- (0.4699994679124755,-0.77910050016199);
\draw[line width=2.pt,color=ccqqqq] (0.4699994679124755,-0.77910050016199) -- (0.4799994522742623,-0.7696005258164083);
\draw[line width=2.pt,color=ccqqqq] (0.4799994522742623,-0.7696005258164083) -- (0.48999943663604906,-0.7599005520963545);
\draw[line width=2.pt,color=ccqqqq] (0.48999943663604906,-0.7599005520963545) -- (0.49999942099783584,-0.7500005790018289);
\draw[line width=2.pt,color=ccqqqq] (0.49999942099783584,-0.7500005790018289) -- (0.5099994053596226,-0.7399006065328313);
\draw[line width=2.pt,color=ccqqqq] (0.5099994053596226,-0.7399006065328313) -- (0.5199993897214094,-0.7296006346893618);
\draw[line width=2.pt,color=ccqqqq] (0.5199993897214094,-0.7296006346893618) -- (0.5299993740831961,-0.7191006634714203);
\draw[line width=2.pt,color=ccqqqq] (0.5299993740831961,-0.7191006634714203) -- (0.5399993584449828,-0.708400692879007);
\draw[line width=2.pt,color=ccqqqq] (0.5399993584449828,-0.708400692879007) -- (0.5499993428067695,-0.6975007229121216);
\draw[line width=2.pt,color=ccqqqq] (0.5499993428067695,-0.6975007229121216) -- (0.5599993271685563,-0.6864007535707644);
\draw[line width=2.pt,color=ccqqqq] (0.5599993271685563,-0.6864007535707644) -- (0.569999311530343,-0.675100784854935);
\draw[line width=2.pt,color=ccqqqq] (0.569999311530343,-0.675100784854935) -- (0.5799992958921297,-0.6636008167646337);
\draw[line width=2.pt,color=ccqqqq] (0.5799992958921297,-0.6636008167646337) -- (0.5899992802539165,-0.6519008492998606);
\draw[line width=2.pt,color=ccqqqq] (0.5899992802539165,-0.6519008492998606) -- (0.5999992646157032,-0.6400008824606154);
\draw[line width=2.pt,color=ccqqqq] (0.5999992646157032,-0.6400008824606154) -- (0.6099992489774899,-0.6279009162468983);
\draw[line width=2.pt,color=ccqqqq] (0.6099992489774899,-0.6279009162468983) -- (0.6199992333392766,-0.6156009506587092);
\draw[line width=2.pt,color=ccqqqq] (0.6199992333392766,-0.6156009506587092) -- (0.6299992177010634,-0.6031009856960481);
\draw[line width=2.pt,color=ccqqqq] (0.6299992177010634,-0.6031009856960481) -- (0.6399992020628501,-0.5904010213589151);
\draw[line width=2.pt,color=ccqqqq] (0.6399992020628501,-0.5904010213589151) -- (0.6499991864246368,-0.5775010576473102);
\draw[line width=2.pt,color=ccqqqq] (0.6499991864246368,-0.5775010576473102) -- (0.6599991707864236,-0.5644010945612333);
\draw[line width=2.pt,color=ccqqqq] (0.6599991707864236,-0.5644010945612333) -- (0.6699991551482103,-0.5511011321006845);
\draw[line width=2.pt,color=ccqqqq] (0.6699991551482103,-0.5511011321006845) -- (0.679999139509997,-0.5376011702656636);
\draw[line width=2.pt,color=ccqqqq] (0.679999139509997,-0.5376011702656636) -- (0.6899991238717837,-0.5239012090561708);
\draw[line width=2.pt,color=ccqqqq] (0.6899991238717837,-0.5239012090561708) -- (0.6999991082335705,-0.5100012484722061);
\draw[line width=2.pt,color=ccqqqq] (0.6999991082335705,-0.5100012484722061) -- (0.7099990925953572,-0.49590128851376936);
\draw[line width=2.pt,color=ccqqqq] (0.7099990925953572,-0.49590128851376936) -- (0.7199990769571439,-0.48160132918086074);
\draw[line width=2.pt,color=ccqqqq] (0.7199990769571439,-0.48160132918086074) -- (0.7299990613189307,-0.4671013704734801);
\draw[line width=2.pt,color=ccqqqq] (0.7299990613189307,-0.4671013704734801) -- (0.7399990456807174,-0.45240141239162757);
\draw[line width=2.pt,color=ccqqqq] (0.7399990456807174,-0.45240141239162757) -- (0.7499990300425041,-0.437501454935303);
\draw[line width=2.pt,color=ccqqqq] (0.7499990300425041,-0.437501454935303) -- (0.7599990144042909,-0.42240149810450656);
\draw[line width=2.pt,color=ccqqqq] (0.7599990144042909,-0.42240149810450656) -- (0.7699989987660776,-0.4071015418992381);
\draw[line width=2.pt,color=ccqqqq] (0.7699989987660776,-0.4071015418992381) -- (0.7799989831278643,-0.3916015863194976);
\draw[line width=2.pt,color=ccqqqq] (0.7799989831278643,-0.3916015863194976) -- (0.789998967489651,-0.3759016313652853);
\draw[line width=2.pt,color=ccqqqq] (0.789998967489651,-0.3759016313652853) -- (0.7999989518514378,-0.3600016770366009);
\draw[line width=2.pt,color=ccqqqq] (0.7999989518514378,-0.3600016770366009) -- (0.8099989362132245,-0.3439017233334447);
\draw[line width=2.pt,color=ccqqqq] (0.8099989362132245,-0.3439017233334447) -- (0.8199989205750112,-0.3276017702558164);
\draw[line width=2.pt,color=ccqqqq] (0.8199989205750112,-0.3276017702558164) -- (0.829998904936798,-0.31110181780371626);
\draw[line width=2.pt,color=ccqqqq] (0.829998904936798,-0.31110181780371626) -- (0.8399988892985847,-0.29440186597714413);
\draw[line width=2.pt,color=ccqqqq] (0.8399988892985847,-0.29440186597714413) -- (0.8499988736603714,-0.2775019147761);
\draw[line width=2.pt,color=ccqqqq] (0.8499988736603714,-0.2775019147761) -- (0.8599988580221581,-0.2604019642005839);
\draw[line width=2.pt,color=ccqqqq] (0.8599988580221581,-0.2604019642005839) -- (0.8699988423839449,-0.24310201425059585);
\draw[line width=2.pt,color=ccqqqq] (0.8699988423839449,-0.24310201425059585) -- (0.8799988267457316,-0.22560206492613588);
\draw[line width=2.pt,color=ccqqqq] (0.8799988267457316,-0.22560206492613588) -- (0.8899988111075183,-0.2079021162272039);
\draw[line width=2.pt,color=ccqqqq] (0.8899988111075183,-0.2079021162272039) -- (0.8999987954693051,-0.1900021681538);
\draw[line width=2.pt,color=ccqqqq] (0.8999987954693051,-0.1900021681538) -- (0.9099987798310918,-0.1719022207059241);
\draw[line width=2.pt,color=ccqqqq] (0.9099987798310918,-0.1719022207059241) -- (0.9199987641928785,-0.15360227388357628);
\draw[line width=2.pt,color=ccqqqq] (0.9199987641928785,-0.15360227388357628) -- (0.9299987485546652,-0.13510232768675656);
\draw[line width=2.pt,color=ccqqqq] (0.9299987485546652,-0.13510232768675656) -- (0.939998732916452,-0.11640238211546483);
\draw[line width=2.pt,color=ccqqqq] (0.939998732916452,-0.11640238211546483) -- (0.9499987172782387,-0.09750243716970108);
\draw[line width=2.pt,color=ccqqqq] (0.9499987172782387,-0.09750243716970108) -- (0.9599987016400254,-0.07840249284946543);
\draw[line width=2.pt,color=ccqqqq] (0.9599987016400254,-0.07840249284946543) -- (0.9699986860018122,-0.059102549154757766);
\draw[line width=2.pt,color=ccqqqq] (0.9699986860018122,-0.059102549154757766) -- (0.9799986703635989,-0.039602606085578196);
\draw[line width=2.pt,color=ccqqqq] (0.9799986703635989,-0.039602606085578196) -- (0.9899986547253856,-0.01990266364192672);
\draw[line width=2.pt,color=ccqqqq] (0.9899986547253856,-0.01990266364192672) -- (0.9999986390871723,0.0);
\draw[line width=2.pt,color=ccqqqq] (0.9999986390871723,0.0) -- (1.009998623448959,0.020097219368792274);
\draw[line width=2.pt,color=ccqqqq] (1.009998623448959,0.020097219368792274) -- (1.019998607810746,0.040397159935859905);
\draw[line width=2.pt,color=ccqqqq] (1.019998607810746,0.040397159935859905) -- (1.0299985921725328,0.06089709987739944);
\draw[line width=2.pt,color=ccqqqq] (1.0299985921725328,0.06089709987739944) -- (1.0399985765343196,0.0815970391934111);
\draw[line width=2.pt,color=ccqqqq] (1.0399985765343196,0.0815970391934111) -- (1.0499985608961064,0.10249697788389445);
\draw[line width=2.pt,color=ccqqqq] (1.0499985608961064,0.10249697788389445) -- (1.0599985452578933,0.12359691594884992);
\draw[line width=2.pt,color=ccqqqq] (1.0599985452578933,0.12359691594884992) -- (1.0699985296196801,0.1448968533882775);
\draw[line width=2.pt,color=ccqqqq] (1.0699985296196801,0.1448968533882775) -- (1.079998513981467,0.1663967902021768);
\draw[line width=2.pt,color=ccqqqq] (1.079998513981467,0.1663967902021768) -- (1.0899984983432538,0.1880967263905482);
\draw[line width=2.pt,color=ccqqqq] (1.0899984983432538,0.1880967263905482) -- (1.0999984827050406,0.2099966619533915);
\draw[line width=2.pt,color=ccqqqq] (1.0999984827050406,0.2099966619533915) -- (1.1099984670668275,0.23209659689070694);
\draw[line width=2.pt,color=ccqqqq] (1.1099984670668275,0.23209659689070694) -- (1.1199984514286143,0.25439653120249406);
\draw[line width=2.pt,color=ccqqqq] (1.1199984514286143,0.25439653120249406) -- (1.1299984357904012,0.2768964648887533);
\draw[line width=2.pt,color=ccqqqq] (1.1299984357904012,0.2768964648887533) -- (1.139998420152188,0.29959639794948445);
\draw[line width=2.pt,color=ccqqqq] (1.139998420152188,0.29959639794948445) -- (1.1499984045139748,0.3224963303846877);
\draw[line width=2.pt,color=ccqqqq] (1.1499984045139748,0.3224963303846877) -- (1.1599983888757617,0.3455962621943629);
\draw[line width=2.pt,color=ccqqqq] (1.1599983888757617,0.3455962621943629) -- (1.1699983732375485,0.36889619337851);
\draw[line width=2.pt,color=ccqqqq] (1.1699983732375485,0.36889619337851) -- (1.1799983575993354,0.39239612393712897);
\draw[line width=2.pt,color=ccqqqq] (1.1799983575993354,0.39239612393712897) -- (1.1899983419611222,0.41609605387021986);
\draw[line width=2.pt,color=ccqqqq] (1.1899983419611222,0.41609605387021986) -- (1.199998326322909,0.4399959831777829);
\draw[line width=2.pt,color=ccqqqq] (1.199998326322909,0.4399959831777829) -- (1.2099983106846959,0.4640959118598178);
\draw[line width=2.pt,color=ccqqqq] (1.2099983106846959,0.4640959118598178) -- (1.2199982950464827,0.4883958399163246);
\draw[line width=2.pt,color=ccqqqq] (1.2199982950464827,0.4883958399163246) -- (1.2299982794082696,0.5128957673473036);
\draw[line width=2.pt,color=ccqqqq] (1.2299982794082696,0.5128957673473036) -- (1.2399982637700564,0.5375956941527544);
\draw[line width=2.pt,color=ccqqqq] (1.2399982637700564,0.5375956941527544) -- (1.2499982481318432,0.5624956203326772);
\draw[line width=2.pt,color=ccqqqq] (1.2499982481318432,0.5624956203326772) -- (1.25999823249363,0.5875955458870719);
\draw[line width=2.pt,color=ccqqqq] (1.25999823249363,0.5875955458870719) -- (1.269998216855417,0.6128954708159386);
\draw[line width=2.pt,color=ccqqqq] (1.269998216855417,0.6128954708159386) -- (1.2799982012172038,0.6383953951192773);
\draw[line width=2.pt,color=ccqqqq] (1.2799982012172038,0.6383953951192773) -- (1.2899981855789906,0.6640953187970879);
\draw[line width=2.pt,color=ccqqqq] (1.2899981855789906,0.6640953187970879) -- (1.2999981699407774,0.6899952418493704);
\draw[line width=2.pt,color=ccqqqq] (1.2999981699407774,0.6899952418493704) -- (1.3099981543025643,0.7160951642761251);
\draw[line width=2.pt,color=ccqqqq] (1.3099981543025643,0.7160951642761251) -- (1.3199981386643511,0.7423950860773516);
\draw[line width=2.pt,color=ccqqqq] (1.3199981386643511,0.7423950860773516) -- (1.329998123026138,0.76889500725305);
\draw[line width=2.pt,color=ccqqqq] (1.329998123026138,0.76889500725305) -- (1.3399981073879248,0.7955949278032204);
\draw[line width=2.pt,color=ccqqqq] (1.3399981073879248,0.7955949278032204) -- (1.3499980917497116,0.8224948477278629);
\draw[line width=2.pt,color=ccqqqq] (1.3499980917497116,0.8224948477278629) -- (1.3599980761114985,0.8495947670269772);
\draw[line width=2.pt,color=ccqqqq] (1.3599980761114985,0.8495947670269772) -- (1.3699980604732853,0.8768946857005635);
\draw[line width=2.pt,color=ccqqqq] (1.3699980604732853,0.8768946857005635) -- (1.3799980448350722,0.9043946037486219);
\draw[line width=2.pt,color=ccqqqq] (1.3799980448350722,0.9043946037486219) -- (1.389998029196859,0.932094521171152);
\draw[line width=2.pt,color=ccqqqq] (1.389998029196859,0.932094521171152) -- (1.3999980135586458,0.9599944379681542);
\draw[line width=2.pt,color=ccqqqq] (1.3999980135586458,0.9599944379681542) -- (1.4099979979204327,0.9880943541396285);
\draw[line width=2.pt,color=ccqqqq] (1.4099979979204327,0.9880943541396285) -- (1.4199979822822195,1.0163942696855748);
\draw[line width=2.pt,color=ccqqqq] (1.4199979822822195,1.0163942696855748) -- (1.4299979666440064,1.044894184605993);
\draw[line width=2.pt,color=ccqqqq] (1.4299979666440064,1.044894184605993) -- (1.4399979510057932,1.073594098900883);
\draw[line width=2.pt,color=ccqqqq] (1.4399979510057932,1.073594098900883) -- (1.44999793536758,1.102494012570245);
\draw[line width=2.pt,color=ccqqqq] (1.44999793536758,1.102494012570245) -- (1.4599979197293669,1.1315939256140788);
\draw[line width=2.pt,color=ccqqqq] (1.4599979197293669,1.1315939256140788) -- (1.4699979040911537,1.1608938380323846);
\draw[line width=2.pt,color=ccqqqq] (1.4699979040911537,1.1608938380323846) -- (1.4799978884529406,1.1903937498251627);
\draw[line width=2.pt,color=ccqqqq] (1.4799978884529406,1.1903937498251627) -- (1.4899978728147274,1.2200936609924127);
\draw[line width=2.pt,color=ccqqqq] (1.4899978728147274,1.2200936609924127) -- (1.4999978571765142,1.2499935715341346);
\draw[line width=2.pt,color=ccqqqq] (1.4999978571765142,1.2499935715341346) -- (1.509997841538301,1.2800934814503284);
\draw[line width=2.pt,color=ccqqqq] (1.509997841538301,1.2800934814503284) -- (1.519997825900088,1.3103933907409941);
\draw[line width=2.pt,color=ccqqqq] (1.519997825900088,1.3103933907409941) -- (1.5299978102618748,1.3408932994061318);
\draw[line width=2.pt,color=ccqqqq] (1.5299978102618748,1.3408932994061318) -- (1.5399977946236616,1.3715932074457413);
\draw[line width=2.pt,color=ccqqqq] (1.5399977946236616,1.3715932074457413) -- (1.5499977789854484,1.4024931148598232);
\draw[line width=2.pt,color=ccqqqq] (1.5499977789854484,1.4024931148598232) -- (1.5599977633472353,1.4335930216483765);
\draw[line width=2.pt,color=ccqqqq] (1.5599977633472353,1.4335930216483765) -- (1.5699977477090221,1.4648929278114022);
\draw[line width=2.pt,color=ccqqqq] (1.5699977477090221,1.4648929278114022) -- (1.579997732070809,1.4963928333488998);
\draw[line width=2.pt,color=ccqqqq] (1.579997732070809,1.4963928333488998) -- (1.5899977164325958,1.5280927382608693);
\draw[line width=2.pt,color=ccqqqq] (1.5899977164325958,1.5280927382608693) -- (1.5999977007943826,1.5599926425473107);
\draw[line width=2.pt,color=ccqqqq] (1.5999977007943826,1.5599926425473107) -- (1.6099976851561695,1.592092546208224);
\draw[line width=2.pt,color=ccqqqq] (1.6099976851561695,1.592092546208224) -- (1.6199976695179563,1.6243924492436097);
\draw[line width=2.pt,color=ccqqqq] (1.6199976695179563,1.6243924492436097) -- (1.6299976538797432,1.6568923516534668);
\draw[line width=2.pt,color=ccqqqq] (1.6299976538797432,1.6568923516534668) -- (1.63999763824153,1.6895922534377963);
\draw[line width=2.pt,color=ccqqqq] (1.63999763824153,1.6895922534377963) -- (1.6499976226033168,1.7224921545965977);
\draw[line width=2.pt,color=ccqqqq] (1.6499976226033168,1.7224921545965977) -- (1.6599976069651037,1.755592055129871);
\draw[line width=2.pt,color=ccqqqq] (1.6599976069651037,1.755592055129871) -- (1.6699975913268905,1.788891955037616);
\draw[line width=2.pt,color=ccqqqq] (1.6699975913268905,1.788891955037616) -- (1.6799975756886774,1.8223918543198332);
\draw[line width=2.pt,color=ccqqqq] (1.6799975756886774,1.8223918543198332) -- (1.6899975600504642,1.8560917529765222);
\draw[line width=2.pt,color=ccqqqq] (1.6899975600504642,1.8560917529765222) -- (1.699997544412251,1.8899916510076835);
\draw[line width=2.pt,color=ccqqqq] (1.699997544412251,1.8899916510076835) -- (1.7099975287740379,1.9240915484133163);
\draw[line width=2.pt,color=ccqqqq] (1.7099975287740379,1.9240915484133163) -- (1.7199975131358247,1.9583914451934215);
\draw[line width=2.pt,color=ccqqqq] (1.7199975131358247,1.9583914451934215) -- (1.7299974974976116,1.9928913413479985);
\draw[line width=2.pt,color=ccqqqq] (1.7299974974976116,1.9928913413479985) -- (1.7399974818593984,2.0275912368770475);
\draw[line width=2.pt,color=ccqqqq] (1.7399974818593984,2.0275912368770475) -- (1.7499974662211852,2.0624911317805683);
\draw[line width=2.pt,color=ccqqqq] (1.7499974662211852,2.0624911317805683) -- (1.759997450582972,2.097591026058561);
\draw[line width=2.pt,color=ccqqqq] (1.759997450582972,2.097591026058561) -- (1.769997434944759,2.132890919711026);
\draw[line width=2.pt,color=ccqqqq] (1.769997434944759,2.132890919711026) -- (1.7799974193065458,2.1683908127379627);
\draw[line width=2.pt,color=ccqqqq] (1.7799974193065458,2.1683908127379627) -- (1.7899974036683326,2.2040907051393717);
\draw[line width=2.pt,color=ccqqqq] (1.7899974036683326,2.2040907051393717) -- (1.7999973880301194,2.2399905969152525);
\draw[line width=2.pt,color=ccqqqq] (1.7999973880301194,2.2399905969152525) -- (1.8099973723919063,2.276090488065605);
\draw[line width=2.pt,color=ccqqqq] (1.8099973723919063,2.276090488065605) -- (1.8199973567536931,2.31239037859043);
\draw[line width=2.pt,color=ccqqqq] (1.8199973567536931,2.31239037859043) -- (1.82999734111548,2.3488902684897264);
\draw[line width=2.pt,color=ccqqqq] (1.82999734111548,2.3488902684897264) -- (1.8399973254772668,2.385590157763495);
\draw[line width=2.pt,color=ccqqqq] (1.8399973254772668,2.385590157763495) -- (1.8499973098390536,2.4224900464117356);
\draw[line width=2.pt,color=ccqqqq] (1.8499973098390536,2.4224900464117356) -- (1.8599972942008405,2.459589934434448);
\draw[line width=2.pt,color=ccqqqq] (1.8599972942008405,2.459589934434448) -- (1.8699972785626273,2.4968898218316324);
\draw[line width=2.pt,color=ccqqqq] (1.8699972785626273,2.4968898218316324) -- (1.8799972629244142,2.534389708603289);
\draw[line width=2.pt,color=ccqqqq] (1.8799972629244142,2.534389708603289) -- (1.889997247286201,2.5720895947494173);
\draw[line width=2.pt,color=ccqqqq] (1.889997247286201,2.5720895947494173) -- (1.8999972316479878,2.6099894802700176);
\draw[line width=2.pt,color=ccqqqq] (1.8999972316479878,2.6099894802700176) -- (1.9099972160097747,2.64808936516509);
\draw[line width=2.pt,color=ccqqqq] (1.9099972160097747,2.64808936516509) -- (1.9199972003715615,2.686389249434634);
\draw[line width=2.pt,color=ccqqqq] (1.9199972003715615,2.686389249434634) -- (1.9299971847333484,2.7248891330786504);
\draw[line width=2.pt,color=ccqqqq] (1.9299971847333484,2.7248891330786504) -- (1.9399971690951352,2.763589016097139);
\draw[line width=2.pt,color=ccqqqq] (1.9399971690951352,2.763589016097139) -- (1.949997153456922,2.8024888984900986);
\draw[line width=2.pt,color=ccqqqq] (1.949997153456922,2.8024888984900986) -- (1.9599971378187089,2.841588780257531);
\draw[line width=2.pt,color=ccqqqq] (1.9599971378187089,2.841588780257531) -- (1.9699971221804957,2.880888661399435);
\draw[line width=2.pt,color=ccqqqq] (1.9699971221804957,2.880888661399435) -- (1.9799971065422826,2.920388541915811);
\draw[line width=2.pt,color=ccqqqq] (1.9799971065422826,2.920388541915811) -- (1.9899970909040694,2.960088421806659);
\begin{scriptsize}
\draw[color=ccqqqq] (-1.7857744887565556,2.881351678151518) node {$f$};
\end{scriptsize}
\end{axis}
\end{tikzpicture}
\end{minipage}
\hfill
\begin{minipage}{0.48\linewidth}
\begin{center}
$f$ est impaire. 
\end{center}

\definecolor{ccqqqq}{rgb}{0.8,0.,0.}
\begin{tikzpicture}[line cap=round,line join=round,>=triangle 45,x=1.0cm,y=1.0cm]
\begin{axis}[
x=1.0cm,y=1.0cm,
axis lines=middle,
ymajorgrids=true,
xmajorgrids=true,
xmin=-1.45781350600093,
xmax=1.7182296954219702,
ymin=-1.4425556593053876,
ymax=1.3882659109258797,
xtick={-1.0,0.0,...,1.0},
ytick={-1.0,0.0,...,1.0},]
\clip(-1.45781350600093,-1.4425556593053876) rectangle (1.7182296954219702,1.3882659109258797);
\draw[line width=2.pt,color=ccqqqq] (-0.9999995826886289,-0.9999987480664091) -- (-0.9999995826886289,-0.9999987480664091);
\draw[line width=2.pt,color=ccqqqq] (-0.9999995826886289,-0.9999987480664091) -- (-0.9949995902075939,-0.9850736578863206);
\draw[line width=2.pt,color=ccqqqq] (-0.9949995902075939,-0.9850736578863206) -- (-0.9899995977265589,-0.9702978171958817);
\draw[line width=2.pt,color=ccqqqq] (-0.9899995977265589,-0.9702978171958817) -- (-0.9849996052455239,-0.9556704759984757);
\draw[line width=2.pt,color=ccqqqq] (-0.9849996052455239,-0.9556704759984757) -- (-0.9799996127644889,-0.9411908842974862);
\draw[line width=2.pt,color=ccqqqq] (-0.9799996127644889,-0.9411908842974862) -- (-0.9749996202834539,-0.9268582920962968);
\draw[line width=2.pt,color=ccqqqq] (-0.9749996202834539,-0.9268582920962968) -- (-0.9699996278024189,-0.9126719493982909);
\draw[line width=2.pt,color=ccqqqq] (-0.9699996278024189,-0.9126719493982909) -- (-0.9649996353213839,-0.8986311062068522);
\draw[line width=2.pt,color=ccqqqq] (-0.9649996353213839,-0.8986311062068522) -- (-0.9599996428403489,-0.884735012525364);
\draw[line width=2.pt,color=ccqqqq] (-0.9599996428403489,-0.884735012525364) -- (-0.9549996503593139,-0.87098291835721);
\draw[line width=2.pt,color=ccqqqq] (-0.9549996503593139,-0.87098291835721) -- (-0.9499996578782789,-0.8573740737057737);
\draw[line width=2.pt,color=ccqqqq] (-0.9499996578782789,-0.8573740737057737) -- (-0.9449996653972439,-0.8439077285744386);
\draw[line width=2.pt,color=ccqqqq] (-0.9449996653972439,-0.8439077285744386) -- (-0.9399996729162089,-0.8305831329665883);
\draw[line width=2.pt,color=ccqqqq] (-0.9399996729162089,-0.8305831329665883) -- (-0.9349996804351739,-0.8173995368856062);
\draw[line width=2.pt,color=ccqqqq] (-0.9349996804351739,-0.8173995368856062) -- (-0.9299996879541389,-0.8043561903348758);
\draw[line width=2.pt,color=ccqqqq] (-0.9299996879541389,-0.8043561903348758) -- (-0.9249996954731039,-0.7914523433177809);
\draw[line width=2.pt,color=ccqqqq] (-0.9249996954731039,-0.7914523433177809) -- (-0.9199997029920689,-0.7786872458377049);
\draw[line width=2.pt,color=ccqqqq] (-0.9199997029920689,-0.7786872458377049) -- (-0.9149997105110339,-0.7660601478980311);
\draw[line width=2.pt,color=ccqqqq] (-0.9149997105110339,-0.7660601478980311) -- (-0.9099997180299989,-0.7535702995021434);
\draw[line width=2.pt,color=ccqqqq] (-0.9099997180299989,-0.7535702995021434) -- (-0.9049997255489639,-0.741216950653425);
\draw[line width=2.pt,color=ccqqqq] (-0.9049997255489639,-0.741216950653425) -- (-0.8999997330679289,-0.7289993513552596);
\draw[line width=2.pt,color=ccqqqq] (-0.8999997330679289,-0.7289993513552596) -- (-0.8949997405868939,-0.7169167516110307);
\draw[line width=2.pt,color=ccqqqq] (-0.8949997405868939,-0.7169167516110307) -- (-0.8899997481058589,-0.704968401424122);
\draw[line width=2.pt,color=ccqqqq] (-0.8899997481058589,-0.704968401424122) -- (-0.8849997556248239,-0.6931535507979167);
\draw[line width=2.pt,color=ccqqqq] (-0.8849997556248239,-0.6931535507979167) -- (-0.8799997631437889,-0.6814714497357985);
\draw[line width=2.pt,color=ccqqqq] (-0.8799997631437889,-0.6814714497357985) -- (-0.8749997706627539,-0.669921348241151);
\draw[line width=2.pt,color=ccqqqq] (-0.8749997706627539,-0.669921348241151) -- (-0.8699997781817189,-0.6585024963173576);
\draw[line width=2.pt,color=ccqqqq] (-0.8699997781817189,-0.6585024963173576) -- (-0.8649997857006839,-0.6472141439678019);
\draw[line width=2.pt,color=ccqqqq] (-0.8649997857006839,-0.6472141439678019) -- (-0.8599997932196489,-0.6360555411958674);
\draw[line width=2.pt,color=ccqqqq] (-0.8599997932196489,-0.6360555411958674) -- (-0.8549998007386139,-0.6250259380049376);
\draw[line width=2.pt,color=ccqqqq] (-0.8549998007386139,-0.6250259380049376) -- (-0.8499998082575789,-0.6141245843983961);
\draw[line width=2.pt,color=ccqqqq] (-0.8499998082575789,-0.6141245843983961) -- (-0.8449998157765439,-0.6033507303796264);
\draw[line width=2.pt,color=ccqqqq] (-0.8449998157765439,-0.6033507303796264) -- (-0.8399998232955089,-0.592703625952012);
\draw[line width=2.pt,color=ccqqqq] (-0.8399998232955089,-0.592703625952012) -- (-0.8349998308144739,-0.5821825211189364);
\draw[line width=2.pt,color=ccqqqq] (-0.8349998308144739,-0.5821825211189364) -- (-0.8299998383334389,-0.5717866658837834);
\draw[line width=2.pt,color=ccqqqq] (-0.8299998383334389,-0.5717866658837834) -- (-0.8249998458524039,-0.5615153102499361);
\draw[line width=2.pt,color=ccqqqq] (-0.8249998458524039,-0.5615153102499361) -- (-0.8199998533713689,-0.5513677042207783);
\draw[line width=2.pt,color=ccqqqq] (-0.8199998533713689,-0.5513677042207783) -- (-0.8149998608903339,-0.5413430977996935);
\draw[line width=2.pt,color=ccqqqq] (-0.8149998608903339,-0.5413430977996935) -- (-0.8099998684092989,-0.5314407409900652);
\draw[line width=2.pt,color=ccqqqq] (-0.8099998684092989,-0.5314407409900652) -- (-0.8049998759282639,-0.521659883795277);
\draw[line width=2.pt,color=ccqqqq] (-0.8049998759282639,-0.521659883795277) -- (-0.7999998834472289,-0.5119997762187122);
\draw[line width=2.pt,color=ccqqqq] (-0.7999998834472289,-0.5119997762187122) -- (-0.794999890966194,-0.5024596682637545);
\draw[line width=2.pt,color=ccqqqq] (-0.794999890966194,-0.5024596682637545) -- (-0.789999898485159,-0.4930388099337875);
\draw[line width=2.pt,color=ccqqqq] (-0.789999898485159,-0.4930388099337875) -- (-0.784999906004124,-0.48373645123219466);
\draw[line width=2.pt,color=ccqqqq] (-0.784999906004124,-0.48373645123219466) -- (-0.779999913523089,-0.47455184216235946);
\draw[line width=2.pt,color=ccqqqq] (-0.779999913523089,-0.47455184216235946) -- (-0.774999921042054,-0.4654842327276655);
\draw[line width=2.pt,color=ccqqqq] (-0.774999921042054,-0.4654842327276655) -- (-0.769999928561019,-0.4565328729314962);
\draw[line width=2.pt,color=ccqqqq] (-0.769999928561019,-0.4565328729314962) -- (-0.764999936079984,-0.4476970127772352);
\draw[line width=2.pt,color=ccqqqq] (-0.764999936079984,-0.4476970127772352) -- (-0.759999943598949,-0.438975902268266);
\draw[line width=2.pt,color=ccqqqq] (-0.759999943598949,-0.438975902268266) -- (-0.754999951117914,-0.43036879140797213);
\draw[line width=2.pt,color=ccqqqq] (-0.754999951117914,-0.43036879140797213) -- (-0.749999958636879,-0.4218749301997371);
\draw[line width=2.pt,color=ccqqqq] (-0.749999958636879,-0.4218749301997371) -- (-0.744999966155844,-0.4134935686469444);
\draw[line width=2.pt,color=ccqqqq] (-0.744999966155844,-0.4134935686469444) -- (-0.739999973674809,-0.40522395675297773);
\draw[line width=2.pt,color=ccqqqq] (-0.739999973674809,-0.40522395675297773) -- (-0.734999981193774,-0.3970653445212204);
\draw[line width=2.pt,color=ccqqqq] (-0.734999981193774,-0.3970653445212204) -- (-0.729999988712739,-0.3890169819550561);
\draw[line width=2.pt,color=ccqqqq] (-0.729999988712739,-0.3890169819550561) -- (-0.724999996231704,-0.3810781190578682);
\draw[line width=2.pt,color=ccqqqq] (-0.724999996231704,-0.3810781190578682) -- (-0.720000003750669,-0.3732480058330404);
\draw[line width=2.pt,color=ccqqqq] (-0.720000003750669,-0.3732480058330404) -- (-0.715000011269634,-0.36552589228395616);
\draw[line width=2.pt,color=ccqqqq] (-0.715000011269634,-0.36552589228395616) -- (-0.710000018788599,-0.35791102841399897);
\draw[line width=2.pt,color=ccqqqq] (-0.710000018788599,-0.35791102841399897) -- (-0.705000026307564,-0.3504026642265524);
\draw[line width=2.pt,color=ccqqqq] (-0.705000026307564,-0.3504026642265524) -- (-0.700000033826529,-0.343000049725);
\draw[line width=2.pt,color=ccqqqq] (-0.700000033826529,-0.343000049725) -- (-0.695000041345494,-0.3357024349127253);
\draw[line width=2.pt,color=ccqqqq] (-0.695000041345494,-0.3357024349127253) -- (-0.690000048864459,-0.32850906979311173);
\draw[line width=2.pt,color=ccqqqq] (-0.690000048864459,-0.32850906979311173) -- (-0.685000056383424,-0.3214192043695429);
\draw[line width=2.pt,color=ccqqqq] (-0.685000056383424,-0.3214192043695429) -- (-0.680000063902389,-0.31443208864540234);
\draw[line width=2.pt,color=ccqqqq] (-0.680000063902389,-0.31443208864540234) -- (-0.675000071421354,-0.30754697262407354);
\draw[line width=2.pt,color=ccqqqq] (-0.675000071421354,-0.30754697262407354) -- (-0.670000078940319,-0.3007631063089401);
\draw[line width=2.pt,color=ccqqqq] (-0.670000078940319,-0.3007631063089401) -- (-0.665000086459284,-0.2940797397033855);
\draw[line width=2.pt,color=ccqqqq] (-0.665000086459284,-0.2940797397033855) -- (-0.660000093978249,-0.2874961228107933);
\draw[line width=2.pt,color=ccqqqq] (-0.660000093978249,-0.2874961228107933) -- (-0.655000101497214,-0.28101150563454697);
\draw[line width=2.pt,color=ccqqqq] (-0.655000101497214,-0.28101150563454697) -- (-0.650000109016179,-0.27462513817803);
\draw[line width=2.pt,color=ccqqqq] (-0.650000109016179,-0.27462513817803) -- (-0.645000116535144,-0.2683362704446261);
\draw[line width=2.pt,color=ccqqqq] (-0.645000116535144,-0.2683362704446261) -- (-0.640000124054109,-0.26214415243771866);
\draw[line width=2.pt,color=ccqqqq] (-0.640000124054109,-0.26214415243771866) -- (-0.635000131573074,-0.2560480341606913);
\draw[line width=2.pt,color=ccqqqq] (-0.635000131573074,-0.2560480341606913) -- (-0.630000139092039,-0.2500471656169274);
\draw[line width=2.pt,color=ccqqqq] (-0.630000139092039,-0.2500471656169274) -- (-0.625000146611004,-0.24414079680981063);
\draw[line width=2.pt,color=ccqqqq] (-0.625000146611004,-0.24414079680981063) -- (-0.620000154129969,-0.23832817774272444);
\draw[line width=2.pt,color=ccqqqq] (-0.620000154129969,-0.23832817774272444) -- (-0.615000161648934,-0.2326085584190524);
\draw[line width=2.pt,color=ccqqqq] (-0.615000161648934,-0.2326085584190524) -- (-0.610000169167899,-0.22698118884217802);
\draw[line width=2.pt,color=ccqqqq] (-0.610000169167899,-0.22698118884217802) -- (-0.605000176686864,-0.22144531901548486);
\draw[line width=2.pt,color=ccqqqq] (-0.605000176686864,-0.22144531901548486) -- (-0.600000184205829,-0.21600019894235642);
\draw[line width=2.pt,color=ccqqqq] (-0.600000184205829,-0.21600019894235642) -- (-0.595000191724794,-0.2106450786261762);
\draw[line width=2.pt,color=ccqqqq] (-0.595000191724794,-0.2106450786261762) -- (-0.590000199243759,-0.2053792080703278);
\draw[line width=2.pt,color=ccqqqq] (-0.590000199243759,-0.2053792080703278) -- (-0.585000206762724,-0.2002018372781947);
\draw[line width=2.pt,color=ccqqqq] (-0.585000206762724,-0.2002018372781947) -- (-0.580000214281689,-0.19511221625316044);
\draw[line width=2.pt,color=ccqqqq] (-0.580000214281689,-0.19511221625316044) -- (-0.575000221800654,-0.19010959499860858);
\draw[line width=2.pt,color=ccqqqq] (-0.575000221800654,-0.19010959499860858) -- (-0.570000229319619,-0.18519322351792258);
\draw[line width=2.pt,color=ccqqqq] (-0.570000229319619,-0.18519322351792258) -- (-0.565000236838584,-0.18036235181448604);
\draw[line width=2.pt,color=ccqqqq] (-0.565000236838584,-0.18036235181448604) -- (-0.560000244357549,-0.17561622989168243);
\draw[line width=2.pt,color=ccqqqq] (-0.560000244357549,-0.17561622989168243) -- (-0.555000251876514,-0.17095410775289532);
\draw[line width=2.pt,color=ccqqqq] (-0.555000251876514,-0.17095410775289532) -- (-0.550000259395479,-0.16637523540150825);
\draw[line width=2.pt,color=ccqqqq] (-0.550000259395479,-0.16637523540150825) -- (-0.545000266914444,-0.1618788628409047);
\draw[line width=2.pt,color=ccqqqq] (-0.545000266914444,-0.1618788628409047) -- (-0.540000274433409,-0.15746424007446824);
\draw[line width=2.pt,color=ccqqqq] (-0.540000274433409,-0.15746424007446824) -- (-0.535000281952374,-0.15313061710558237);
\draw[line width=2.pt,color=ccqqqq] (-0.535000281952374,-0.15313061710558237) -- (-0.530000289471339,-0.14887724393763063);
\draw[line width=2.pt,color=ccqqqq] (-0.530000289471339,-0.14887724393763063) -- (-0.525000296990304,-0.14470337057399657);
\draw[line width=2.pt,color=ccqqqq] (-0.525000296990304,-0.14470337057399657) -- (-0.520000304509269,-0.1406082470180637);
\draw[line width=2.pt,color=ccqqqq] (-0.520000304509269,-0.1406082470180637) -- (-0.515000312028234,-0.13659112327321554);
\draw[line width=2.pt,color=ccqqqq] (-0.515000312028234,-0.13659112327321554) -- (-0.510000319547199,-0.13265124934283565);
\draw[line width=2.pt,color=ccqqqq] (-0.510000319547199,-0.13265124934283565) -- (-0.505000327066164,-0.12878787523030752);
\draw[line width=2.pt,color=ccqqqq] (-0.505000327066164,-0.12878787523030752) -- (-0.500000334585129,-0.1250002509390147);
\draw[line width=2.pt,color=ccqqqq] (-0.500000334585129,-0.1250002509390147) -- (-0.49500034210409405,-0.12128762647234073);
\draw[line width=2.pt,color=ccqqqq] (-0.49500034210409405,-0.12128762647234073) -- (-0.49000034962305905,-0.11764925183366912);
\draw[line width=2.pt,color=ccqqqq] (-0.49000034962305905,-0.11764925183366912) -- (-0.48500035714202405,-0.11408437702638341);
\draw[line width=2.pt,color=ccqqqq] (-0.48500035714202405,-0.11408437702638341) -- (-0.48000036466098905,-0.11059225205386712);
\draw[line width=2.pt,color=ccqqqq] (-0.48000036466098905,-0.11059225205386712) -- (-0.47500037217995406,-0.10717212691950379);
\draw[line width=2.pt,color=ccqqqq] (-0.47500037217995406,-0.10717212691950379) -- (-0.47000037969891906,-0.10382325162667694);
\draw[line width=2.pt,color=ccqqqq] (-0.47000037969891906,-0.10382325162667694) -- (-0.46500038721788406,-0.1005448761787701);
\draw[line width=2.pt,color=ccqqqq] (-0.46500038721788406,-0.1005448761787701) -- (-0.46000039473684906,-0.09733625057916681);
\draw[line width=2.pt,color=ccqqqq] (-0.46000039473684906,-0.09733625057916681) -- (-0.45500040225581406,-0.09419662483125059);
\draw[line width=2.pt,color=ccqqqq] (-0.45500040225581406,-0.09419662483125059) -- (-0.45000040977477906,-0.09112524893840497);
\draw[line width=2.pt,color=ccqqqq] (-0.45000040977477906,-0.09112524893840497) -- (-0.44500041729374407,-0.08812137290401348);
\draw[line width=2.pt,color=ccqqqq] (-0.44500041729374407,-0.08812137290401348) -- (-0.44000042481270907,-0.08518424673145965);
\draw[line width=2.pt,color=ccqqqq] (-0.44000042481270907,-0.08518424673145965) -- (-0.43500043233167407,-0.082313120424127);
\draw[line width=2.pt,color=ccqqqq] (-0.43500043233167407,-0.082313120424127) -- (-0.43000043985063907,-0.07950724398539907);
\draw[line width=2.pt,color=ccqqqq] (-0.43000043985063907,-0.07950724398539907) -- (-0.4250004473696041,-0.07676586741865939);
\draw[line width=2.pt,color=ccqqqq] (-0.4250004473696041,-0.07676586741865939) -- (-0.4200004548885691,-0.07408824072729148);
\draw[line width=2.pt,color=ccqqqq] (-0.4200004548885691,-0.07408824072729148) -- (-0.4150004624075341,-0.07147361391467888);
\draw[line width=2.pt,color=ccqqqq] (-0.4150004624075341,-0.07147361391467888) -- (-0.4100004699264991,-0.0689212369842051);
\draw[line width=2.pt,color=ccqqqq] (-0.4100004699264991,-0.0689212369842051) -- (-0.4050004774454641,-0.0664303599392537);
\draw[line width=2.pt,color=ccqqqq] (-0.4050004774454641,-0.0664303599392537) -- (-0.4000004849644291,-0.06400023278320818);
\draw[line width=2.pt,color=ccqqqq] (-0.4000004849644291,-0.06400023278320818) -- (-0.3950004924833941,-0.0616301055194521);
\draw[line width=2.pt,color=ccqqqq] (-0.3950004924833941,-0.0616301055194521) -- (-0.3900005000023591,-0.059319228151368954);
\draw[line width=2.pt,color=ccqqqq] (-0.3900005000023591,-0.059319228151368954) -- (-0.3850005075213241,-0.05706685068234229);
\draw[line width=2.pt,color=ccqqqq] (-0.3850005075213241,-0.05706685068234229) -- (-0.3800005150402891,-0.05487222311575564);
\draw[line width=2.pt,color=ccqqqq] (-0.3800005150402891,-0.05487222311575564) -- (-0.3750005225592541,-0.05273459545499252);
\draw[line width=2.pt,color=ccqqqq] (-0.3750005225592541,-0.05273459545499252) -- (-0.3700005300782191,-0.05065321770343647);
\draw[line width=2.pt,color=ccqqqq] (-0.3700005300782191,-0.05065321770343647) -- (-0.3650005375971841,-0.04862733986447102);
\draw[line width=2.pt,color=ccqqqq] (-0.3650005375971841,-0.04862733986447102) -- (-0.3600005451161491,-0.04665621194147969);
\draw[line width=2.pt,color=ccqqqq] (-0.3600005451161491,-0.04665621194147969) -- (-0.3550005526351141,-0.04473908393784602);
\draw[line width=2.pt,color=ccqqqq] (-0.3550005526351141,-0.04473908393784602) -- (-0.3500005601540791,-0.04287520585695353);
\draw[line width=2.pt,color=ccqqqq] (-0.3500005601540791,-0.04287520585695353) -- (-0.3450005676730441,-0.04106382770218575);
\draw[line width=2.pt,color=ccqqqq] (-0.3450005676730441,-0.04106382770218575) -- (-0.3400005751920091,-0.03930419947692622);
\draw[line width=2.pt,color=ccqqqq] (-0.3400005751920091,-0.03930419947692622) -- (-0.3350005827109741,-0.03759557118455845);
\draw[line width=2.pt,color=ccqqqq] (-0.3350005827109741,-0.03759557118455845) -- (-0.3300005902299391,-0.035937192828466);
\draw[line width=2.pt,color=ccqqqq] (-0.3300005902299391,-0.035937192828466) -- (-0.3250005977489041,-0.03432831441203236);
\draw[line width=2.pt,color=ccqqqq] (-0.3250005977489041,-0.03432831441203236) -- (-0.3200006052678691,-0.03276818593864109);
\draw[line width=2.pt,color=ccqqqq] (-0.3200006052678691,-0.03276818593864109) -- (-0.3150006127868341,-0.0312560574116757);
\draw[line width=2.pt,color=ccqqqq] (-0.3150006127868341,-0.0312560574116757) -- (-0.3100006203057991,-0.02979117883451973);
\draw[line width=2.pt,color=ccqqqq] (-0.3100006203057991,-0.02979117883451973) -- (-0.3050006278247641,-0.028372800210556704);
\draw[line width=2.pt,color=ccqqqq] (-0.3050006278247641,-0.028372800210556704) -- (-0.3000006353437291,-0.027000171543170158);
\draw[line width=2.pt,color=ccqqqq] (-0.3000006353437291,-0.027000171543170158) -- (-0.2950006428626941,-0.025672542835743613);
\draw[line width=2.pt,color=ccqqqq] (-0.2950006428626941,-0.025672542835743613) -- (-0.2900006503816591,-0.024389164091660604);
\draw[line width=2.pt,color=ccqqqq] (-0.2900006503816591,-0.024389164091660604) -- (-0.2850006579006241,-0.023149285314304654);
\draw[line width=2.pt,color=ccqqqq] (-0.2850006579006241,-0.023149285314304654) -- (-0.2800006654195891,-0.0219521565070593);
\draw[line width=2.pt,color=ccqqqq] (-0.2800006654195891,-0.0219521565070593) -- (-0.2750006729385541,-0.020797027673308065);
\draw[line width=2.pt,color=ccqqqq] (-0.2750006729385541,-0.020797027673308065) -- (-0.2700006804575191,-0.01968314881643448);
\draw[line width=2.pt,color=ccqqqq] (-0.2700006804575191,-0.01968314881643448) -- (-0.26500068797648413,-0.018609769939822076);
\draw[line width=2.pt,color=ccqqqq] (-0.26500068797648413,-0.018609769939822076) -- (-0.26000069549544913,-0.01757614104685438);
\draw[line width=2.pt,color=ccqqqq] (-0.26000069549544913,-0.01757614104685438) -- (-0.25500070301441413,-0.01658151214091492);
\draw[line width=2.pt,color=ccqqqq] (-0.25500070301441413,-0.01658151214091492) -- (-0.25000071053337913,-0.015625133225387233);
\draw[line width=2.pt,color=ccqqqq] (-0.25000071053337913,-0.015625133225387233) -- (-0.24500071805234416,-0.014706254303654841);
\draw[line width=2.pt,color=ccqqqq] (-0.24500071805234416,-0.014706254303654841) -- (-0.2400007255713092,-0.013824125379101276);
\draw[line width=2.pt,color=ccqqqq] (-0.2400007255713092,-0.013824125379101276) -- (-0.23500073309027422,-0.012977996455110065);
\draw[line width=2.pt,color=ccqqqq] (-0.23500073309027422,-0.012977996455110065) -- (-0.23000074060923925,-0.012167117535064735);
\draw[line width=2.pt,color=ccqqqq] (-0.23000074060923925,-0.012167117535064735) -- (-0.22500074812820428,-0.011390738622348821);
\draw[line width=2.pt,color=ccqqqq] (-0.22500074812820428,-0.011390738622348821) -- (-0.2200007556471693,-0.010648109720345847);
\draw[line width=2.pt,color=ccqqqq] (-0.2200007556471693,-0.010648109720345847) -- (-0.21500076316613434,-0.009938480832439343);
\draw[line width=2.pt,color=ccqqqq] (-0.21500076316613434,-0.009938480832439343) -- (-0.21000077068509937,-0.009261101962012838);
\draw[line width=2.pt,color=ccqqqq] (-0.21000077068509937,-0.009261101962012838) -- (-0.2050007782040644,-0.008615223112449865);
\draw[line width=2.pt,color=ccqqqq] (-0.2050007782040644,-0.008615223112449865) -- (-0.20000078572302943,-0.008000094287133948);
\draw[line width=2.pt,color=ccqqqq] (-0.20000078572302943,-0.008000094287133948) -- (-0.19500079324199446,-0.00741496548944862);
\draw[line width=2.pt,color=ccqqqq] (-0.19500079324199446,-0.00741496548944862) -- (-0.1900008007609595,-0.006859086722777408);
\draw[line width=2.pt,color=ccqqqq] (-0.1900008007609595,-0.006859086722777408) -- (-0.18500080827992452,-0.006331707990503841);
\draw[line width=2.pt,color=ccqqqq] (-0.18500080827992452,-0.006331707990503841) -- (-0.18000081579888955,-0.005832079296011449);
\draw[line width=2.pt,color=ccqqqq] (-0.18000081579888955,-0.005832079296011449) -- (-0.17500082331785458,-0.005359450642683762);
\draw[line width=2.pt,color=ccqqqq] (-0.17500082331785458,-0.005359450642683762) -- (-0.1700008308368196,-0.004913072033904308);
\draw[line width=2.pt,color=ccqqqq] (-0.1700008308368196,-0.004913072033904308) -- (-0.16500083835578463,-0.004492193473056617);
\draw[line width=2.pt,color=ccqqqq] (-0.16500083835578463,-0.004492193473056617) -- (-0.16000084587474966,-0.004096064963524217);
\draw[line width=2.pt,color=ccqqqq] (-0.16000084587474966,-0.004096064963524217) -- (-0.1550008533937147,-0.003723936508690638);
\draw[line width=2.pt,color=ccqqqq] (-0.1550008533937147,-0.003723936508690638) -- (-0.15000086091267972,-0.003375058111939409);
\draw[line width=2.pt,color=ccqqqq] (-0.15000086091267972,-0.003375058111939409) -- (-0.14500086843164475,-0.003048679776654059);
\draw[line width=2.pt,color=ccqqqq] (-0.14500086843164475,-0.003048679776654059) -- (-0.14000087595060978,-0.0027440515062181173);
\draw[line width=2.pt,color=ccqqqq] (-0.14000087595060978,-0.0027440515062181173) -- (-0.1350008834695748,-0.0024604233040151136);
\draw[line width=2.pt,color=ccqqqq] (-0.1350008834695748,-0.0024604233040151136) -- (-0.13000089098853984,-0.002197045173428576);
\draw[line width=2.pt,color=ccqqqq] (-0.13000089098853984,-0.002197045173428576) -- (-0.12500089850750487,-0.0019531671178420348);
\draw[line width=2.pt,color=ccqqqq] (-0.12500089850750487,-0.0019531671178420348) -- (-0.1200009060264699,-0.0017280391406390187);
\draw[line width=2.pt,color=ccqqqq] (-0.1200009060264699,-0.0017280391406390187) -- (-0.11500091354543493,-0.0015209112452030567);
\draw[line width=2.pt,color=ccqqqq] (-0.11500091354543493,-0.0015209112452030567) -- (-0.11000092106439996,-0.001331033434917678);
\draw[line width=2.pt,color=ccqqqq] (-0.11000092106439996,-0.001331033434917678) -- (-0.10500092858336499,-0.0011576557131664118);
\draw[line width=2.pt,color=ccqqqq] (-0.10500092858336499,-0.0011576557131664118) -- (-0.10000093610233002,-0.0010000280833327875);
\draw[line width=2.pt,color=ccqqqq] (-0.10000093610233002,-0.0010000280833327875) -- (-0.09500094362129505,0.0);
\draw[line width=2.pt,color=ccqqqq] (-0.09500094362129505,0.0) -- (-0.09000095114026008,0.0);
\draw[line width=2.pt,color=ccqqqq] (-0.09000095114026008,0.0) -- (-0.0850009586592251,0.0);
\draw[line width=2.pt,color=ccqqqq] (-0.0850009586592251,0.0) -- (-0.08000096617819014,0.0);
\draw[line width=2.pt,color=ccqqqq] (-0.08000096617819014,0.0) -- (-0.07500097369715517,0.0);
\draw[line width=2.pt,color=ccqqqq] (-0.07500097369715517,0.0) -- (-0.0700009812161202,0.0);
\draw[line width=2.pt,color=ccqqqq] (-0.0700009812161202,0.0) -- (-0.06500098873508522,0.0);
\draw[line width=2.pt,color=ccqqqq] (-0.06500098873508522,0.0) -- (-0.060000996254050254,0.0);
\draw[line width=2.pt,color=ccqqqq] (-0.060000996254050254,0.0) -- (-0.05500100377301528,0.0);
\draw[line width=2.pt,color=ccqqqq] (-0.05500100377301528,0.0) -- (-0.05000101129198031,0.0);
\draw[line width=2.pt,color=ccqqqq] (-0.05000101129198031,0.0) -- (-0.04500101881094534,0.0);
\draw[line width=2.pt,color=ccqqqq] (-0.04500101881094534,0.0) -- (-0.04000102632991037,0.0);
\draw[line width=2.pt,color=ccqqqq] (-0.04000102632991037,0.0) -- (-0.0350010338488754,0.0);
\draw[line width=2.pt,color=ccqqqq] (-0.0350010338488754,0.0) -- (-0.030001041367840427,0.0);
\draw[line width=2.pt,color=ccqqqq] (-0.030001041367840427,0.0) -- (-0.025001048886805453,0.0);
\draw[line width=2.pt,color=ccqqqq] (-0.025001048886805453,0.0) -- (-0.02000105640577048,0.0);
\draw[line width=2.pt,color=ccqqqq] (-0.02000105640577048,0.0) -- (-0.015001063924735505,0.0);
\draw[line width=2.pt,color=ccqqqq] (-0.015001063924735505,0.0) -- (-0.010001071443700531,0.0);
\draw[line width=2.pt,color=ccqqqq] (-0.010001071443700531,0.0) -- (-0.005001078962665558,0.0);
\draw[line width=2.pt,color=ccqqqq] (-0.005001078962665558,0.0) -- (0.0,0.0);
\draw[line width=2.pt,color=ccqqqq] (0.0,0.0) -- (0.004998905999404388,0.0);
\draw[line width=2.pt,color=ccqqqq] (0.004998905999404388,0.0) -- (0.009998898480439361,0.0);
\draw[line width=2.pt,color=ccqqqq] (0.009998898480439361,0.0) -- (0.014998890961474335,0.0);
\draw[line width=2.pt,color=ccqqqq] (0.014998890961474335,0.0) -- (0.01999888344250931,0.0);
\draw[line width=2.pt,color=ccqqqq] (0.01999888344250931,0.0) -- (0.024998875923544283,0.0);
\draw[line width=2.pt,color=ccqqqq] (0.024998875923544283,0.0) -- (0.029998868404579257,0.0);
\draw[line width=2.pt,color=ccqqqq] (0.029998868404579257,0.0) -- (0.03499886088561423,0.0);
\draw[line width=2.pt,color=ccqqqq] (0.03499886088561423,0.0) -- (0.0399988533666492,0.0);
\draw[line width=2.pt,color=ccqqqq] (0.0399988533666492,0.0) -- (0.04499884584768417,0.0);
\draw[line width=2.pt,color=ccqqqq] (0.04499884584768417,0.0) -- (0.04999883832871914,0.0);
\draw[line width=2.pt,color=ccqqqq] (0.04999883832871914,0.0) -- (0.05499883080975411,0.0);
\draw[line width=2.pt,color=ccqqqq] (0.05499883080975411,0.0) -- (0.05999882329078908,0.0);
\draw[line width=2.pt,color=ccqqqq] (0.05999882329078908,0.0) -- (0.06499881577182405,0.0);
\draw[line width=2.pt,color=ccqqqq] (0.06499881577182405,0.0) -- (0.06999880825285902,0.0);
\draw[line width=2.pt,color=ccqqqq] (0.06999880825285902,0.0) -- (0.07499880073389399,0.0);
\draw[line width=2.pt,color=ccqqqq] (0.07499880073389399,0.0) -- (0.07999879321492896,0.0);
\draw[line width=2.pt,color=ccqqqq] (0.07999879321492896,0.0) -- (0.08499878569596393,0.0);
\draw[line width=2.pt,color=ccqqqq] (0.08499878569596393,0.0) -- (0.0899987781769989,0.0);
\draw[line width=2.pt,color=ccqqqq] (0.0899987781769989,0.0) -- (0.09499877065803387,0.0);
\draw[line width=2.pt,color=ccqqqq] (0.09499877065803387,0.0) -- (0.09999876313906884,0.0);
\draw[line width=2.pt,color=ccqqqq] (0.09999876313906884,0.0) -- (0.10499875562010381,0.0011575838426227034);
\draw[line width=2.pt,color=ccqqqq] (0.10499875562010381,0.0011575838426227034) -- (0.10999874810113879,0.0013309545565885286);
\draw[line width=2.pt,color=ccqqqq] (0.10999874810113879,0.0013309545565885286) -- (0.11499874058217376,0.0015208250331449577);
\draw[line width=2.pt,color=ccqqqq] (0.11499874058217376,0.0015208250331449577) -- (0.11999873306320873,0.0017279452689084613);
\draw[line width=2.pt,color=ccqqqq] (0.11999873306320873,0.0017279452689084613) -- (0.1249987255442437,0.0019530652604955103);
\draw[line width=2.pt,color=ccqqqq] (0.1249987255442437,0.0019530652604955103) -- (0.12999871802527868,0.002196935004522576);
\draw[line width=2.pt,color=ccqqqq] (0.12999871802527868,0.002196935004522576) -- (0.13499871050631365,0.002460304497606128);
\draw[line width=2.pt,color=ccqqqq] (0.13499871050631365,0.002460304497606128) -- (0.13999870298734862,0.0027439237363626385);
\draw[line width=2.pt,color=ccqqqq] (0.13999870298734862,0.0027439237363626385) -- (0.1449986954683836,0.003048542717408577);
\draw[line width=2.pt,color=ccqqqq] (0.1449986954683836,0.003048542717408577) -- (0.14999868794941856,0.003374911437360415);
\draw[line width=2.pt,color=ccqqqq] (0.14999868794941856,0.003374911437360415) -- (0.15499868043045353,0.003723779892834624);
\draw[line width=2.pt,color=ccqqqq] (0.15499868043045353,0.003723779892834624) -- (0.1599986729114885,0.004095898080447673);
\draw[line width=2.pt,color=ccqqqq] (0.1599986729114885,0.004095898080447673) -- (0.16499866539252347,0.004492015996816035);
\draw[line width=2.pt,color=ccqqqq] (0.16499866539252347,0.004492015996816035) -- (0.16999865787355845,0.004912883638556179);
\draw[line width=2.pt,color=ccqqqq] (0.16999865787355845,0.004912883638556179) -- (0.17499865035459342,0.005359251002284578);
\draw[line width=2.pt,color=ccqqqq] (0.17499865035459342,0.005359251002284578) -- (0.1799986428356284,0.0058318680846177);
\draw[line width=2.pt,color=ccqqqq] (0.1799986428356284,0.0058318680846177) -- (0.18499863531666336,0.006331484882172018);
\draw[line width=2.pt,color=ccqqqq] (0.18499863531666336,0.006331484882172018) -- (0.18999862779769833,0.006858851391564002);
\draw[line width=2.pt,color=ccqqqq] (0.18999862779769833,0.006858851391564002) -- (0.1949986202787333,0.0074147176094101225);
\draw[line width=2.pt,color=ccqqqq] (0.1949986202787333,0.0074147176094101225) -- (0.19999861275976827,0.00799983353232685);
\draw[line width=2.pt,color=ccqqqq] (0.19999861275976827,0.00799983353232685) -- (0.20499860524080324,0.008614949156930658);
\draw[line width=2.pt,color=ccqqqq] (0.20499860524080324,0.008614949156930658) -- (0.2099985977218382,0.009260814479838014);
\draw[line width=2.pt,color=ccqqqq] (0.2099985977218382,0.009260814479838014) -- (0.21499859020287318,0.009938179497665392);
\draw[line width=2.pt,color=ccqqqq] (0.21499859020287318,0.009938179497665392) -- (0.21999858268390815,0.010647794207029259);
\draw[line width=2.pt,color=ccqqqq] (0.21999858268390815,0.010647794207029259) -- (0.22499857516494312,0.011390408604546088);
\draw[line width=2.pt,color=ccqqqq] (0.22499857516494312,0.011390408604546088) -- (0.2299985676459781,0.01216677268683235);
\draw[line width=2.pt,color=ccqqqq] (0.2299985676459781,0.01216677268683235) -- (0.23499856012701306,0.012977636450504516);
\draw[line width=2.pt,color=ccqqqq] (0.23499856012701306,0.012977636450504516) -- (0.23999855260804803,0.013823749892179056);
\draw[line width=2.pt,color=ccqqqq] (0.23999855260804803,0.013823749892179056) -- (0.244998545089083,0.014705863008472441);
\draw[line width=2.pt,color=ccqqqq] (0.244998545089083,0.014705863008472441) -- (0.24999853757011797,0.015624725796001143);
\draw[line width=2.pt,color=ccqqqq] (0.24999853757011797,0.015624725796001143) -- (0.25499853005115297,0.016581088251381635);
\draw[line width=2.pt,color=ccqqqq] (0.25499853005115297,0.016581088251381635) -- (0.25999852253218797,0.017575700371230386);
\draw[line width=2.pt,color=ccqqqq] (0.25999852253218797,0.017575700371230386) -- (0.26499851501322297,0.018609312152163868);
\draw[line width=2.pt,color=ccqqqq] (0.26499851501322297,0.018609312152163868) -- (0.26999850749425797,0.019682673590798547);
\draw[line width=2.pt,color=ccqqqq] (0.26999850749425797,0.019682673590798547) -- (0.27499849997529296,0.020796534683750898);
\draw[line width=2.pt,color=ccqqqq] (0.27499849997529296,0.020796534683750898) -- (0.27999849245632796,0.021951645427637393);
\draw[line width=2.pt,color=ccqqqq] (0.27999849245632796,0.021951645427637393) -- (0.28499848493736296,0.023148755819074496);
\draw[line width=2.pt,color=ccqqqq] (0.28499848493736296,0.023148755819074496) -- (0.28999847741839796,0.024388615854678684);
\draw[line width=2.pt,color=ccqqqq] (0.28999847741839796,0.024388615854678684) -- (0.29499846989943296,0.025671975531066423);
\draw[line width=2.pt,color=ccqqqq] (0.29499846989943296,0.025671975531066423) -- (0.29999846238046796,0.02699958484485419);
\draw[line width=2.pt,color=ccqqqq] (0.29999846238046796,0.02699958484485419) -- (0.30499845486150295,0.028372193792658453);
\draw[line width=2.pt,color=ccqqqq] (0.30499845486150295,0.028372193792658453) -- (0.30999844734253795,0.02979055237109568);
\draw[line width=2.pt,color=ccqqqq] (0.30999844734253795,0.02979055237109568) -- (0.31499843982357295,0.03125541057678235);
\draw[line width=2.pt,color=ccqqqq] (0.31499843982357295,0.03125541057678235) -- (0.31999843230460795,0.03276751840633492);
\draw[line width=2.pt,color=ccqqqq] (0.31999843230460795,0.03276751840633492) -- (0.32499842478564295,0.03432762585636987);
\draw[line width=2.pt,color=ccqqqq] (0.32499842478564295,0.03432762585636987) -- (0.32999841726667795,0.035936482923503675);
\draw[line width=2.pt,color=ccqqqq] (0.32999841726667795,0.035936482923503675) -- (0.33499840974771294,0.037594839604352795);
\draw[line width=2.pt,color=ccqqqq] (0.33499840974771294,0.037594839604352795) -- (0.33999840222874794,0.039303445895533716);
\draw[line width=2.pt,color=ccqqqq] (0.33999840222874794,0.039303445895533716) -- (0.34499839470978294,0.04106305179366289);
\draw[line width=2.pt,color=ccqqqq] (0.34499839470978294,0.04106305179366289) -- (0.34999838719081794,0.0428744072953568);
\draw[line width=2.pt,color=ccqqqq] (0.34999838719081794,0.0428744072953568) -- (0.35499837967185294,0.044738262397231915);
\draw[line width=2.pt,color=ccqqqq] (0.35499837967185294,0.044738262397231915) -- (0.35999837215288794,0.046655367095904704);
\draw[line width=2.pt,color=ccqqqq] (0.35999837215288794,0.046655367095904704) -- (0.36499836463392293,0.048626471387991636);
\draw[line width=2.pt,color=ccqqqq] (0.36499836463392293,0.048626471387991636) -- (0.36999835711495793,0.05065232527010919);
\draw[line width=2.pt,color=ccqqqq] (0.36999835711495793,0.05065232527010919) -- (0.37499834959599293,0.05273367873887382);
\draw[line width=2.pt,color=ccqqqq] (0.37499834959599293,0.05273367873887382) -- (0.37999834207702793,0.05487128179090202);
\draw[line width=2.pt,color=ccqqqq] (0.37999834207702793,0.05487128179090202) -- (0.3849983345580629,0.05706588442281025);
\draw[line width=2.pt,color=ccqqqq] (0.3849983345580629,0.05706588442281025) -- (0.3899983270390979,0.05931823663121497);
\draw[line width=2.pt,color=ccqqqq] (0.3899983270390979,0.05931823663121497) -- (0.3949983195201329,0.061629088412732666);
\draw[line width=2.pt,color=ccqqqq] (0.3949983195201329,0.061629088412732666) -- (0.3999983120011679,0.06399918976397981);
\draw[line width=2.pt,color=ccqqqq] (0.3999983120011679,0.06399918976397981) -- (0.4049983044822029,0.06642929068157286);
\draw[line width=2.pt,color=ccqqqq] (0.4049983044822029,0.06642929068157286) -- (0.4099982969632379,0.06892014116212829);
\draw[line width=2.pt,color=ccqqqq] (0.4099982969632379,0.06892014116212829) -- (0.4149982894442729,0.07147249120226258);
\draw[line width=2.pt,color=ccqqqq] (0.4149982894442729,0.07147249120226258) -- (0.4199982819253079,0.07408709079859219);
\draw[line width=2.pt,color=ccqqqq] (0.4199982819253079,0.07408709079859219) -- (0.4249982744063429,0.0767646899477336);
\draw[line width=2.pt,color=ccqqqq] (0.4249982744063429,0.0767646899477336) -- (0.4299982668873779,0.07950603864630326);
\draw[line width=2.pt,color=ccqqqq] (0.4299982668873779,0.07950603864630326) -- (0.4349982593684129,0.08231188689091767);
\draw[line width=2.pt,color=ccqqqq] (0.4349982593684129,0.08231188689091767) -- (0.4399982518494479,0.0851829846781933);
\draw[line width=2.pt,color=ccqqqq] (0.4399982518494479,0.0851829846781933) -- (0.4449982443304829,0.0881200820047466);
\draw[line width=2.pt,color=ccqqqq] (0.4449982443304829,0.0881200820047466) -- (0.4499982368115179,0.09112392886719405);
\draw[line width=2.pt,color=ccqqqq] (0.4499982368115179,0.09112392886719405) -- (0.4549982292925529,0.09419527526215211);
\draw[line width=2.pt,color=ccqqqq] (0.4549982292925529,0.09419527526215211) -- (0.4599982217735879,0.09733487118623728);
\draw[line width=2.pt,color=ccqqqq] (0.4599982217735879,0.09733487118623728) -- (0.4649982142546229,0.100543466636066);
\draw[line width=2.pt,color=ccqqqq] (0.4649982142546229,0.100543466636066) -- (0.4699982067356579,0.10382181160825475);
\draw[line width=2.pt,color=ccqqqq] (0.4699982067356579,0.10382181160825475) -- (0.4749981992166929,0.10717065609942002);
\draw[line width=2.pt,color=ccqqqq] (0.4749981992166929,0.10717065609942002) -- (0.4799981916977279,0.11059075010617825);
\draw[line width=2.pt,color=ccqqqq] (0.4799981916977279,0.11059075010617825) -- (0.4849981841787629,0.11408284362514594);
\draw[line width=2.pt,color=ccqqqq] (0.4849981841787629,0.11408284362514594) -- (0.4899981766597979,0.11764768665293954);
\draw[line width=2.pt,color=ccqqqq] (0.4899981766597979,0.11764768665293954) -- (0.4949981691408329,0.12128602918617552);
\draw[line width=2.pt,color=ccqqqq] (0.4949981691408329,0.12128602918617552) -- (0.4999981616218679,0.12499862122147036);
\draw[line width=2.pt,color=ccqqqq] (0.4999981616218679,0.12499862122147036) -- (0.5049981541029028,0.1287862127554405);
\draw[line width=2.pt,color=ccqqqq] (0.5049981541029028,0.1287862127554405) -- (0.5099981465839378,0.13264955378470247);
\draw[line width=2.pt,color=ccqqqq] (0.5099981465839378,0.13264955378470247) -- (0.5149981390649728,0.1365893943058727);
\draw[line width=2.pt,color=ccqqqq] (0.5149981390649728,0.1365893943058727) -- (0.5199981315460078,0.1406064843155677);
\draw[line width=2.pt,color=ccqqqq] (0.5199981315460078,0.1406064843155677) -- (0.5249981240270428,0.14470157381040388);
\draw[line width=2.pt,color=ccqqqq] (0.5249981240270428,0.14470157381040388) -- (0.5299981165080778,0.14887541278699776);
\draw[line width=2.pt,color=ccqqqq] (0.5299981165080778,0.14887541278699776) -- (0.5349981089891128,0.1531287512419658);
\draw[line width=2.pt,color=ccqqqq] (0.5349981089891128,0.1531287512419658) -- (0.5399981014701478,0.15746233917192445);
\draw[line width=2.pt,color=ccqqqq] (0.5399981014701478,0.15746233917192445) -- (0.5449980939511828,0.1618769265734902);
\draw[line width=2.pt,color=ccqqqq] (0.5449980939511828,0.1618769265734902) -- (0.5499980864322178,0.16637326344327955);
\draw[line width=2.pt,color=ccqqqq] (0.5499980864322178,0.16637326344327955) -- (0.5549980789132528,0.1709520997779089);
\draw[line width=2.pt,color=ccqqqq] (0.5549980789132528,0.1709520997779089) -- (0.5599980713942878,0.17561418557399475);
\draw[line width=2.pt,color=ccqqqq] (0.5599980713942878,0.17561418557399475) -- (0.5649980638753228,0.18036027082815362);
\draw[line width=2.pt,color=ccqqqq] (0.5649980638753228,0.18036027082815362) -- (0.5699980563563578,0.1851911055370019);
\draw[line width=2.pt,color=ccqqqq] (0.5699980563563578,0.1851911055370019) -- (0.5749980488373928,0.19010743969715613);
\draw[line width=2.pt,color=ccqqqq] (0.5749980488373928,0.19010743969715613) -- (0.5799980413184278,0.19511002330523272);
\draw[line width=2.pt,color=ccqqqq] (0.5799980413184278,0.19511002330523272) -- (0.5849980337994628,0.2001996063578482);
\draw[line width=2.pt,color=ccqqqq] (0.5849980337994628,0.2001996063578482) -- (0.5899980262804978,0.205376938851619);
\draw[line width=2.pt,color=ccqqqq] (0.5899980262804978,0.205376938851619) -- (0.5949980187615328,0.2106427707831616);
\draw[line width=2.pt,color=ccqqqq] (0.5949980187615328,0.2106427707831616) -- (0.5999980112425678,0.2159978521490925);
\draw[line width=2.pt,color=ccqqqq] (0.5999980112425678,0.2159978521490925) -- (0.6049980037236028,0.22144293294602813);
\draw[line width=2.pt,color=ccqqqq] (0.6049980037236028,0.22144293294602813) -- (0.6099979962046378,0.22697876317058496);
\draw[line width=2.pt,color=ccqqqq] (0.6099979962046378,0.22697876317058496) -- (0.6149979886856728,0.2326060928193795);
\draw[line width=2.pt,color=ccqqqq] (0.6149979886856728,0.2326060928193795) -- (0.6199979811667078,0.2383256718890282);
\draw[line width=2.pt,color=ccqqqq] (0.6199979811667078,0.2383256718890282) -- (0.6249979736477428,0.2441382503761475);
\draw[line width=2.pt,color=ccqqqq] (0.6249979736477428,0.2441382503761475) -- (0.6299979661287778,0.25004457827735393);
\draw[line width=2.pt,color=ccqqqq] (0.6299979661287778,0.25004457827735393) -- (0.6349979586098128,0.2560454055892639);
\draw[line width=2.pt,color=ccqqqq] (0.6349979586098128,0.2560454055892639) -- (0.6399979510908478,0.262141482308494);
\draw[line width=2.pt,color=ccqqqq] (0.6399979510908478,0.262141482308494) -- (0.6449979435718828,0.2683335584316605);
\draw[line width=2.pt,color=ccqqqq] (0.6449979435718828,0.2683335584316605) -- (0.6499979360529178,0.27462238395538);
\draw[line width=2.pt,color=ccqqqq] (0.6499979360529178,0.27462238395538) -- (0.6549979285339528,0.281008708876269);
\draw[line width=2.pt,color=ccqqqq] (0.6549979285339528,0.281008708876269) -- (0.6599979210149878,0.2874932831909439);
\draw[line width=2.pt,color=ccqqqq] (0.6599979210149878,0.2874932831909439) -- (0.6649979134960228,0.29407685689602125);
\draw[line width=2.pt,color=ccqqqq] (0.6649979134960228,0.29407685689602125) -- (0.6699979059770578,0.3007601799881174);
\draw[line width=2.pt,color=ccqqqq] (0.6699979059770578,0.3007601799881174) -- (0.6749978984580928,0.30754400246384894);
\draw[line width=2.pt,color=ccqqqq] (0.6749978984580928,0.30754400246384894) -- (0.6799978909391278,0.31442907431983225);
\draw[line width=2.pt,color=ccqqqq] (0.6799978909391278,0.31442907431983225) -- (0.6849978834201628,0.32141614555268383);
\draw[line width=2.pt,color=ccqqqq] (0.6849978834201628,0.32141614555268383) -- (0.6899978759011978,0.32850596615902017);
\draw[line width=2.pt,color=ccqqqq] (0.6899978759011978,0.32850596615902017) -- (0.6949978683822328,0.33569928613545774);
\draw[line width=2.pt,color=ccqqqq] (0.6949978683822328,0.33569928613545774) -- (0.6999978608632678,0.342996855478613);
\draw[line width=2.pt,color=ccqqqq] (0.6999978608632678,0.342996855478613) -- (0.7049978533443028,0.35039942418510245);
\draw[line width=2.pt,color=ccqqqq] (0.7049978533443028,0.35039942418510245) -- (0.7099978458253378,0.3579077422515425);
\draw[line width=2.pt,color=ccqqqq] (0.7099978458253378,0.3579077422515425) -- (0.7149978383063728,0.36552255967454966);
\draw[line width=2.pt,color=ccqqqq] (0.7149978383063728,0.36552255967454966) -- (0.7199978307874078,0.37324462645074036);
\draw[line width=2.pt,color=ccqqqq] (0.7199978307874078,0.37324462645074036) -- (0.7249978232684428,0.38107469257673116);
\draw[line width=2.pt,color=ccqqqq] (0.7249978232684428,0.38107469257673116) -- (0.7299978157494778,0.3890135080491385);
\draw[line width=2.pt,color=ccqqqq] (0.7299978157494778,0.3890135080491385) -- (0.7349978082305128,0.39706182286457875);
\draw[line width=2.pt,color=ccqqqq] (0.7349978082305128,0.39706182286457875) -- (0.7399978007115477,0.4052203870196685);
\draw[line width=2.pt,color=ccqqqq] (0.7399978007115477,0.4052203870196685) -- (0.7449977931925827,0.41348995051102416);
\draw[line width=2.pt,color=ccqqqq] (0.7449977931925827,0.41348995051102416) -- (0.7499977856736177,0.4218712633352622);
\draw[line width=2.pt,color=ccqqqq] (0.7499977856736177,0.4218712633352622) -- (0.7549977781546527,0.43036507548899916);
\draw[line width=2.pt,color=ccqqqq] (0.7549977781546527,0.43036507548899916) -- (0.7599977706356877,0.43897213696885146);
\draw[line width=2.pt,color=ccqqqq] (0.7599977706356877,0.43897213696885146) -- (0.7649977631167227,0.44769319777143557);
\draw[line width=2.pt,color=ccqqqq] (0.7649977631167227,0.44769319777143557) -- (0.7699977555977577,0.4565290078933679);
\draw[line width=2.pt,color=ccqqqq] (0.7699977555977577,0.4565290078933679) -- (0.7749977480787927,0.46548031733126505);
\draw[line width=2.pt,color=ccqqqq] (0.7749977480787927,0.46548031733126505) -- (0.7799977405598277,0.4745478760817434);
\draw[line width=2.pt,color=ccqqqq] (0.7799977405598277,0.4745478760817434) -- (0.7849977330408627,0.4837324341414195);
\draw[line width=2.pt,color=ccqqqq] (0.7849977330408627,0.4837324341414195) -- (0.7899977255218977,0.49303474150690973);
\draw[line width=2.pt,color=ccqqqq] (0.7899977255218977,0.49303474150690973) -- (0.7949977180029327,0.5024555481748306);
\draw[line width=2.pt,color=ccqqqq] (0.7949977180029327,0.5024555481748306) -- (0.7999977104839677,0.5119956041417986);
\draw[line width=2.pt,color=ccqqqq] (0.7999977104839677,0.5119956041417986) -- (0.8049977029650027,0.5216556594044301);
\draw[line width=2.pt,color=ccqqqq] (0.8049977029650027,0.5216556594044301) -- (0.8099976954460377,0.5314364639593417);
\draw[line width=2.pt,color=ccqqqq] (0.8099976954460377,0.5314364639593417) -- (0.8149976879270727,0.5413387678031498);
\draw[line width=2.pt,color=ccqqqq] (0.8149976879270727,0.5413387678031498) -- (0.8199976804081077,0.551363320932471);
\draw[line width=2.pt,color=ccqqqq] (0.8199976804081077,0.551363320932471) -- (0.8249976728891427,0.5615108733439215);
\draw[line width=2.pt,color=ccqqqq] (0.8249976728891427,0.5615108733439215) -- (0.8299976653701777,0.571782175034118);
\draw[line width=2.pt,color=ccqqqq] (0.8299976653701777,0.571782175034118) -- (0.8349976578512127,0.5821779759996769);
\draw[line width=2.pt,color=ccqqqq] (0.8349976578512127,0.5821779759996769) -- (0.8399976503322477,0.5926990262372147);
\draw[line width=2.pt,color=ccqqqq] (0.8399976503322477,0.5926990262372147) -- (0.8449976428132827,0.6033460757433479);
\draw[line width=2.pt,color=ccqqqq] (0.8449976428132827,0.6033460757433479) -- (0.8499976352943177,0.6141198745146927);
\draw[line width=2.pt,color=ccqqqq] (0.8499976352943177,0.6141198745146927) -- (0.8549976277753527,0.625021172547866);
\draw[line width=2.pt,color=ccqqqq] (0.8549976277753527,0.625021172547866) -- (0.8599976202563877,0.636050719839484);
\draw[line width=2.pt,color=ccqqqq] (0.8599976202563877,0.636050719839484) -- (0.8649976127374227,0.6472092663861633);
\draw[line width=2.pt,color=ccqqqq] (0.8649976127374227,0.6472092663861633) -- (0.8699976052184577,0.6584975621845202);
\draw[line width=2.pt,color=ccqqqq] (0.8699976052184577,0.6584975621845202) -- (0.8749975976994927,0.6699163572311713);
\draw[line width=2.pt,color=ccqqqq] (0.8749975976994927,0.6699163572311713) -- (0.8799975901805277,0.6814664015227331);
\draw[line width=2.pt,color=ccqqqq] (0.8799975901805277,0.6814664015227331) -- (0.8849975826615627,0.6931484450558218);
\draw[line width=2.pt,color=ccqqqq] (0.8849975826615627,0.6931484450558218) -- (0.8899975751425977,0.7049632378270543);
\draw[line width=2.pt,color=ccqqqq] (0.8899975751425977,0.7049632378270543) -- (0.8949975676236327,0.7169115298330468);
\draw[line width=2.pt,color=ccqqqq] (0.8949975676236327,0.7169115298330468) -- (0.8999975601046677,0.7289940710704158);
\draw[line width=2.pt,color=ccqqqq] (0.8999975601046677,0.7289940710704158) -- (0.9049975525857027,0.7412116115357779);
\draw[line width=2.pt,color=ccqqqq] (0.9049975525857027,0.7412116115357779) -- (0.9099975450667377,0.7535649012257493);
\draw[line width=2.pt,color=ccqqqq] (0.9099975450667377,0.7535649012257493) -- (0.9149975375477727,0.7660546901369467);
\draw[line width=2.pt,color=ccqqqq] (0.9149975375477727,0.7660546901369467) -- (0.9199975300288077,0.7786817282659866);
\draw[line width=2.pt,color=ccqqqq] (0.9199975300288077,0.7786817282659866) -- (0.9249975225098427,0.7914467656094852);
\draw[line width=2.pt,color=ccqqqq] (0.9249975225098427,0.7914467656094852) -- (0.9299975149908777,0.8043505521640593);
\draw[line width=2.pt,color=ccqqqq] (0.9299975149908777,0.8043505521640593) -- (0.9349975074719127,0.8173938379263251);
\draw[line width=2.pt,color=ccqqqq] (0.9349975074719127,0.8173938379263251) -- (0.9399974999529477,0.8305773728928993);
\draw[line width=2.pt,color=ccqqqq] (0.9399974999529477,0.8305773728928993) -- (0.9449974924339827,0.8439019070603982);
\draw[line width=2.pt,color=ccqqqq] (0.9449974924339827,0.8439019070603982) -- (0.9499974849150177,0.8573681904254384);
\draw[line width=2.pt,color=ccqqqq] (0.9499974849150177,0.8573681904254384) -- (0.9549974773960527,0.8709769729846363);
\draw[line width=2.pt,color=ccqqqq] (0.9549974773960527,0.8709769729846363) -- (0.9599974698770877,0.8847290047346084);
\draw[line width=2.pt,color=ccqqqq] (0.9599974698770877,0.8847290047346084) -- (0.9649974623581227,0.898625035671971);
\draw[line width=2.pt,color=ccqqqq] (0.9649974623581227,0.898625035671971) -- (0.9699974548391577,0.9126658157933408);
\draw[line width=2.pt,color=ccqqqq] (0.9699974548391577,0.9126658157933408) -- (0.9749974473201927,0.9268520950953343);
\draw[line width=2.pt,color=ccqqqq] (0.9749974473201927,0.9268520950953343) -- (0.9799974398012277,0.9411846235745677);
\draw[line width=2.pt,color=ccqqqq] (0.9799974398012277,0.9411846235745677) -- (0.9849974322822627,0.9556641512276577);
\draw[line width=2.pt,color=ccqqqq] (0.9849974322822627,0.9556641512276577) -- (0.9899974247632977,0.9702914280512207);
\draw[line width=2.pt,color=ccqqqq] (0.9899974247632977,0.9702914280512207) -- (0.9949974172443327,0.9850672040418731);
\begin{scriptsize}
\draw[color=ccqqqq] (-0.732847123067442,-0.9678752130775836) node {$f$};
\end{scriptsize}
\end{axis}
\end{tikzpicture}



\end{minipage}

\begin{Ex}
Soit $f$ définie sur $[-4;4]$ par $f(x) = x^2+1$.
\begin{enumerate}
\item $[-4;4]$ est un intervalle centré en 0.
\item $f(-x)=(-x)^2+1=x^2+1=f(x)$
\end{enumerate}
Donc $f$ est paire sur $[-4;4]$.
\end{Ex}

\begin{AVDM}

Ouvrir Geogebra et taper dans la barre de saisie, \texttt{Fonction( -x $\wedge$ 2+1,-4,4)}. Observer l'axe de symétrie.
\end{AVDM}


\begin{AVDM}

Ouvrir Geogebra et taper dans la barre de saisie, \texttt{Fonction( x $\wedge$ 3+x,-2,2)}. Observer l'axe de symétrie.
\end{AVDM}

\begin{Att}
Soit $f$ définie sur $[-3;4]$ par $f(x) = x^2+1$ a son domaine de définition qui n'est pas centré en 0. Donc $f$ n'est ni paire ni impaire.
\end{Att}

\begin{Log}
On dit que la condition "\textit{l'intervalle est centré en 0}" est une condition \textbf{nécessaire} mais non suffisante.
\end{Log}


\EPC{1}{FEA-105}{Calculer. Communiquer.}

\EPC{0}{FEA-106}{Calculer. Communiquer.}





}% La fonction Carré + rep. graphique
%\impress{\impressionEleve}{\begin{titre}[Fonctions de référence]

\Titre{La fonction Cube}{3}
\end{titre}

\begin{CpsCol}
\textbf{Variations de fonctions}
\begin{description}
\item[$\square$] Connaitre la fonction Cube : définition et courbes représentative.
\item[$\square$] Pour la fonction Cube, résoudre graphiquement ou algébriquement une équation ou une inéquation du type $f(x) = k$, $f(x) < k$.
\item[$\square$] Étudier la parité d'une fonction dans des cas simples.
\end{description}
\end{CpsCol}




\begin{DefT}{Fonction Cube}\index{Fonctions!Cube}
La \textbf{fonction Cube} $f$ est la fonction définie sur $\R$ par $f(x)=x^3$.
\end{DefT}

 
\begin{DefT}{Racine cubique}\index{Fonctions!Racine cubique}
La \textbf{racine cubique} de $x$ est le \textit{nombre réel} $a$ tel que $a^3=x$. On note : $a = \sqrt[3]{x}$.
\end{DefT}

\begin{Ex}
\begin{description}
\item[•] La  racine cubique  de $8$ est égale à 2, on écrit que $2 = \sqrt[3]{8}$ ou $2^3 = 8$.
\item[•] La  racine cubique  de $-27$ est égale à $-3$, on écrit que $-3 = \sqrt[3]{-27}$ ou $(-3)^3 = -27$.
\end{description}
\end{Ex}



\begin{Pp}[Variations]
\begin{minipage}{0.48\linewidth}
La fonction Cube est strictement  croissante sur $\R$. La fonction Cube est impaire. Sa courbe représentative est symétrique par rapport à l'origine du repère.
\end{minipage}
\hfill
\begin{minipage}{0.48\linewidth}
\begin{tikzpicture}
\tkzTabInit[lgt=1,espcl=2]{ $x$ / 1,$f $ / 2}
{ $-\infty$ , $+\infty$}
\tkzTabVar{-/$-\infty$ , +/$+\infty$ }
\end{tikzpicture}
\end{minipage}
\end{Pp}

\ROC

\mini{
\EPC{1}{FR-45}{Chercher.}
 
\EPC{1}{FR-46}{Raisonner.}
}{
\EPC{1}{FR-48}{Chercher.}
}


\mini{
\EPC{1}{FR-47}{Représenter.}

\EPC{1}{FR-50}{Représenter. Raisonner. Calculer}
}{
\EPC{1}{FR-49}{Chercher.}

\EPCC{1}{FR-51}{Calculer.}
}





}% La fonction Cube + rep. graphique
%\impress{\impressionEleve}{\begin{titre}[Fonctions de référence]

\Titre{La fonction Inverse}{4}
\end{titre}



\begin{CpsCol}
\textbf{Variations de fonctions}
\begin{description}
\item[$\square$] Connaitre la fonction Inverse : définition et courbes représentative.
\item[$\square$] Pour la fonction Inverse, résoudre graphiquement ou algébriquement une équation ou une inéquation du type $f(x) = k$, $f(x) < k$.
\item[$\square$] Étudier la parité d'une fonction dans des cas simples.
\end{description}
\end{CpsCol}




\begin{DefT}{Fonction Inverse}\index{Fonctions!Inverse}
La \textbf{fonction Inverse} $f$ est la fonction définie sur $\R^*$ par $f(x)=\frac{1}{x}$.
\end{DefT}


\begin{DefT}{Représentation graphique} \index{Fonction Inverse! Représentation graphique}\index{Hyperbole| see Fonction Inverse! Représentation graphique}
La \textbf{représentation graphique} de la fonction Inverse s'appelle une \textbf{hyperbole} et son équation est $y=\frac{1}{x}$. 
\end{DefT}



\begin{Pp}[Variations]
\begin{minipage}{0.48\linewidth}
La fonction Inverse est strictement décroissante sur $\R^*_-$ et strictement décroissante sur $\R^*_+$. 

L'hyperbole d'équation $y=\frac{1}{x}$ est symétrique par rapport à l'origine du repère.
\end{minipage}
\hfill
\begin{minipage}{0.48\linewidth}
\begin{tikzpicture}
\tkzTabInit[lgt=1,espcl=2]{ $x$ / 1,$f $ / 2}
{ $-\infty$ , $0$ ,$+\infty$}
\tkzTabVar{+/$0$ , -D+ /$ $/$ $ , -/$0$}
\end{tikzpicture}
\end{minipage}
\end{Pp}

\ROC

\mini{
\EPCB{1}{FR-6}{Raisonner.}
 
\EPCB{1}{FR-8}{Raisonner.}

\EPCC{1}{FR-11}{Représenter. Raisonner.}
}{
\EPCB{1}{FR-12}{Représenter. Raisonner.}

\EPCB{1}{FR-19}{Raisonner. Calculer.}
}


\mini{
%\EPCB{1}{FR-13}{Représenter. Raisonner. Calculer}

\EPCB{1}{FR-14}{Représenter. Raisonner. Calculer}

%\EPCB{1}{FR-21}{Représenter. Raisonner. Calculer}
\EPCB{1}{FR-18}{Représenter. Raisonner. Calculer}
}{


\EPCC{1}{FR-16}{Représenter. Raisonner. Calculer}

\EPCC{1}{FR-20}{Représenter. Raisonner. Calculer}
}





}% La fonction Inverse + rep. graphique
%\impress{\impressionEleve}{\begin{titre}[Fonctions de référence]

\Titre{La fonction Racine carrée}{2}
\end{titre}


 
\begin{CpsCol}
\textbf{Variations de fonctions}
\begin{description}
\item[$\square$] Connaitre la fonction Racine carrée : définition et courbes représentative.
\item[$\square$] Pour la fonction Racine carrée, résoudre graphiquement ou algébriquement une équation ou une inéquation du type $f(x) = k$, $f(x) < k$.
\end{description}
\end{CpsCol}




\begin{DefT}{Fonction Racine carrée}\index{Fonctions!Racine carrée}
La \textbf{fonction Racine carrée} $f$ est la fonction définie sur $\R^+$ par $f(x)=\sqrt x$.
\end{DefT}

 



\begin{Pp}[Variations]
\begin{minipage}{0.48\linewidth}
La fonction Racine carrée est strictement  croissante sur $\R^+$.  
\end{minipage}
\hfill
\begin{minipage}{0.48\linewidth}
\begin{tikzpicture}
\tkzTabInit[lgt=1,espcl=2]{ $x$ / 1,$f $ / 2}
{ $0$ , $+\infty$}
\tkzTabVar{-/$0$ , +/$+\infty$ }
\end{tikzpicture}
\end{minipage}
\end{Pp}

\ROC

\mini{
\EPC{1}{FR-56}{Calculer}

\EPC{1}{FR-52}{Raisonner.}

\EPC{1}{FR-54}{Calculer.}
}{
\EPC{1}{FR-53}{Chercher.}

\EPC{1}{FR-55}{Représenter.}

\EPC{1}{FR-51}{Chercher.} 
}
}% La fonction Racine Carrée + rep. graphique
%
% \chapter{Techniques de résolution}
%
%\impress{\impressionEleve}{\begin{titreTice}[Fonctions de référence]

\Titre{Technique de résolution}{0}
\end{titreTice}


\begin{CpsCol}
\textbf{Variations de fonctions}
\begin{description}
\item[$\square$] Résoudre une équation ou une inéquation avec la fonction Carré
\item[$\square$] Résoudre une équation ou une inéquation avec la fonction Inverse
\end{description}
\end{CpsCol}


Le but de cette activité est de connaitre les grands principes de calcul algébrique avec la fonction Inverse et la fonction Carré. 
\subsection*{La fonction Carré}

\subsubsection*{Résolution d'équation du type $ax^2=k$,$k,a \in \R$}

\subsubsection*{Un exemple : Résolution de $x^2=5$}
$x^2=5 \Longleftrightarrow x^2-5 = 0 \Longleftrightarrow \left(x-\sqrt{5} \right)\left(x+\sqrt{5} \right) = 0 \Longleftrightarrow x=\sqrt{5}$ ou $x+\sqrt{5} =0$. \fbox{$S=\left\lbrace \sqrt{5};- \sqrt{5} \right\rbrace $}
\subsubsection*{Applications}
\begin{list}{•}{}
\item $x^2-12 = 0$
\item $x^2-16 = 0$
\item $5x^2-10 = 0$
\end{list}


\subsubsection*{Cas général}

$x^2=k \Longleftrightarrow x^2-k = 0 \Longleftrightarrow \left(x-\sqrt{k} \right)\left(x+\sqrt{k} \right) = 0 \Longleftrightarrow x=\sqrt{k}$ ou $x+\sqrt{k} =0$. \fbox{$S=\left\lbrace \sqrt{k};- \sqrt{k} \right\rbrace $}


\subsection*{Résolution d'inéquation du type $ax^2\leq k$ ou $ax^2\geq k$  ,$k, a \in \R$}

\subsubsection*{Un exemple}
$x^2 \geq 10 \Longleftrightarrow x^2-10 \geq 0 \Longleftrightarrow \left(x-\sqrt{10} \right)\left(x+\sqrt{10} \right) \geq 0$. 

On utilise alors un tableau de signe.

\begin{tabular}{|c|ccccccc|}
\hline 
$x$ & $-\infty$ & & $-\sqrt{10}$ &  & $\sqrt{10}$ &  & $+\infty$ \\ 
\hline 
$x-\sqrt{10}$ & & $-$ &  & $-$ & 0 & + &  \\ 
\hline 
$x+\sqrt{10}$ &  & $-$ & 0 & + &  & + &  \\ 
\hline 
$(x+\sqrt{10})(x-\sqrt{10})$ &  & + &  & $-$ &  & + &  \\ 
\hline 
\end{tabular} 
\fbox{$S=\left] -\infty;\sqrt{10} \right] \cup - \left[\sqrt{10};+\infty \right[ $}

\subsubsection*{Applications}
\begin{list}{•}{}
\item $3x^2-12 > 0$
\item $x^2-16 \leq 0$
\item $9x^2 - 25  \geq 0$
\end{list}

\subsection*{La fonction Inverse}

\subsubsection*{Résolution d'équation du type $\frac{a}{x}=b$,$a$ non nul, $x \neq 0$}

\subsubsection*{Un exemple : Résolution de $\frac{3}{x}=7$ , $x \neq 0$}

$\frac{3}{x}=7 \Longleftrightarrow 3=7x \Longleftrightarrow  x=\frac{3}{7}$. \fbox{$S=\left\lbrace \frac{3}{7} \right\rbrace $}

\subsubsection*{Applications}
\begin{list}{•}{}
\item $\frac{5}{x}=7$
\item $\frac{7}{x+2}=4$
\item $\frac{6}{1-x}=5$
\end{list}


\subsubsection*{Cas général : Résolution de $\frac{a}{x}=b$ , $x \neq 0$ et $a \neq 0$ }

$\frac{a}{x}=b \Longleftrightarrow a=bx $. 

\begin{list}{•}{}
\item Si $b =0$, S= $\oslash$
\item Sinon S=$\left\lbrace \frac{a}{b} \right\rbrace $
\end{list}

\subsubsection*{Résolution d'inéquation du type $\frac{a}{x} \leq b$ ou $\frac{a}{x} \geq b$ , $x \neq 0$}

$\frac{a}{x} \leq b \Longleftrightarrow \frac{a}{x} - b \leq 0 \Longleftrightarrow \frac{a-bx}{x} \leq 0  $.

On utilise alors un tableau de signe dans lequel on étudie les lignes de $a-bx$ et $x$.


}% Technique de résolution
%
%\impress{\impressionEleve}{\begin{titreTice}[Fonctions de référence]

\Titre{Algorithme de dichotomie}{0}
\end{titreTice}


\begin{CpsCol}
\textbf{Variations de fonctions}
\begin{description}
\item[$\square$] Connaitre un algorithme de dichotomie
\end{description}
\end{CpsCol}

\begin{Ety}\index{Algorithme de dichotomie}
\textbf{Dichotomie} : Du grec ancien dikhotomia : « division en deux parties »
\end{Ety}


On souhaite résoudre l'équation $x^2=x+1$ avec $x>0$. On considère l'algorithme suivant.


\begin{algobox}
\Variables
\Ligne a EST\_DU\_TYPE NOMBRE
\Ligne b EST\_DU\_TYPE NOMBRE
\Ligne m EST\_DU\_TYPE NOMBRE
\DebutAlgo
\Ligne a PREND\_LA\_VALEUR 1
\Ligne b PREND\_LA\_VALEUR 2
\Tantque{(b-a>0.01)}
\DebutTantQue
\Ligne m PREND\_LA\_VALEUR (a+b)/2
\Si{(m*m<m+1)}
\DebutSi
\Ligne a PREND\_LA\_VALEUR m
\FinSi
\Sinon
\DebutSinon
\Ligne b PREND\_LA\_VALEUR m
\FinSinon
\FinTantQue
\Ligne AFFICHER a
\Ligne AFFICHER b
\FinAlgo

\end{algobox}

On décide maintenant de créer ce tableau à l'aide d'un tableur.

\begin{enumerate}
\item  Compléter les 2 premières lignes à la mains. 

\item  Créer le tableau ci-dessous dans un tableur.

\begin{tabular}{|c|c|c|c|c|c|c|c|}
\hline 
\rowcolor{gray} & A & B & C & D & E & F & G\\ 
\hline 
\cellcolor{gray} 1 & $a$ & $b$ & $b-a$ & $m=\frac{a+b}{2}$ & $m^2$ & $m+1$ & Condition $m^2<m+1$\\ 
\hline 
\cellcolor{gray}2 & 0 & 10 &  &  &  & &  \\ 
\hline 
\cellcolor{gray}3 &  &  &  &  &  & &  \\
\hline 
\cellcolor{gray}4 &  &  &  &  &  & &  \\
\hline 
\cellcolor{gray}5 &  &  &  &  &  & &  \\
\hline 
\end{tabular} 


\item  Que faut-il écrire dans la cellule $C2$ ? Justifier.
\item  Compléter de même les cellule $D2$, $E2$ et $F2$.
\item  Quelles valeurs faut-il écrire dans les cellule $A3$ et $B3$ ? Justifier.

Dans la cellule $A3$ et la cellule $B3$, on doit inscrire une formule contenant un $SI$.

\textit{Dans un module d'aide de Open office, on a relevé :
La formule $=SI(A2=’3’ ;A3 ;A4)$ signifie : si la valeur de la cellule $A2$ est égale à 3 alors
choisir la cellule $A3$ sinon choisir la cellule $A4$.}

En transformant cette formule donnée, écrire la formule adéquate dans la cellule $A3$ puis une
formule dans la cellule $B3$.

\item  Compléter le tableau en tirant les cellule vers le bas jusqu'à ce que $b – a < 0,01$.
\item  Retrouver la valeur solution.
\item  Programmer cet algorithme en Python.
\end{enumerate}

}% Algorithme de dichotomie
%%##########\impress{\impressionEleve}{\begin{titre}[Fonctions de référence]

\Titre{Les fonctions du second degré}{4}
\end{titre}


\begin{CpsCol}
\textbf{Variations de fonctions}
\begin{description}
\item[$\square$] Connaitre les variations des fonctions du second degré
\item[$\square$] Déterminer l'axe de symétrie d'une courbe d'une fonction du second degré
\item[$\square$] Déterminer l'extremum d'une fonction du second degré
\end{description}
\end{CpsCol}

\mini{
\EPCB{1}{FR-22}{Chercher. Représenter. Modéliser}

\EPCB{1}{FR-25}{Calculer}
}{
\EPCB{1}{FR-23}{Calculer. Communiquer}

\EPCB{1}{FR-24}{Calculer. Représenter}
}

\begin{DefT}{Fonction du second degré} \index{Fonctions ! Second degré}
Une \textbf{fonction polynôme du second degré} est une fonction définie sur $\R$ par $f(x) = ax^2+bx+c$, où $a \in \R^*$ et $b$ et $c$ deux réels.
\end{DefT}


\EPCC{1}{FPD-12}{Chercher. Représenter. Modéliser}


\begin{Pp}[Forme canonique] \index{Fonctions ! Forme canonique}
Soit $f$ une fonction polynôme du second degré définie sur $R$ par $f(x) = ax^2+bx+c$, où $a \in \R^*$ et $b$ et $c$ deux réels.

$f$ peut s'écrire sous la forme $f(x)=a(x-\alpha)^2+ \beta$. Cette forme est la \textbf{forme canonique} de $f$.

Par changement de variable, toute fonction du second degré a une courbe d'équation $Y=aX^2$.

\begin{description}
\item[•] $\beta$ est le minimum ou le maximum de $f$ selon le signe de $a$.
\item[•] La droite d'équation $x=\alpha$ est l'axe de symétrie de la parabole.
\end{description}
\end{Pp}

\begin{Rq}
La forme canonique d'une fonction du second degré permet de :

\begin{description}
\item[•] déterminer l' \textit{extremum}.
\item[•] obtenir les coordonnées du sommet.
\item[•] étudier les variations.
\end{description}
\end{Rq}


\Rec{1}{FR-27}

\begin{Pp}
Soit $f$ une fonction polynôme du second degré définie sur $R$ par $f(x) = ax^2+bx+c$, où $a \in \R^*$ et $b$ et $c$ deux réels.
\begin{description}
\item[•] Si $a>0$, $f$ est décroissante sur $\left]-\infty; \frac{-b}{2a} \right]$ puis croissante sur $\left[\frac{-b}{2a};+\infty \right[$.
\item[•] Si $a<0$, $f$ est croissante sur $\left]-\infty; \frac{-b}{2a} \right]$ puis décroissante sur $\left[\frac{-b}{2a};+\infty \right[$.
\end{description}
\end{Pp}

\ROC


\mini{
\Exo{1}{FR-30}

\Exo{1}{FR-40}
}{
\Exo{1}{FR-31}
}

\vspace{0.4cm}

\Exo{1}{FR-28}

\Exo{1}{FR-29}

\mini{
\App{1}{FR-36}
}{
\Exo{1}{FR-44}
}

\App{1}{FR-37}

\App{1}{FR-41}

\begin{DTL}

On souhaite construire un terrain de jeu
rectangulaire sur le sable avec une élastique
rouge de 32 m.
On souhaite déterminer les dimensions du
rectangle pour que son aire soit maximale.
On note $AB = x$ en m et $AD = d (x)$ en cm.
\begin{enumerate}
\item Montrer que $d (x) = 16 - x$ et vérifier que $x$ appartient à l'intervalle $[ 0;16 ]$.
\item  On note $S(x)$ l’aire en $cm^2$ du rectangle $ABCD$. Montrer que $S(x) = -x^2 + 16x$, pour tout $x \in [0;16]$.
\item  \begin{enumerate}
		\item Démontrer que, pour tout $x$ de $[ 0 ; 16 ]$, $S(x)= - ( x - 8)^2 + 64$.
		\item  Déterminer les variations de $S$ sur $[0 ; 16]$.
		\item  Reproduire et compléter le tableau de valeurs suivant :
		\begin{tabular}{|c|c|c|c|c|c|c|c|c|}
		\hline 
		$x$ & 0 & 1 & 2 & 4 & 8 & 10 & 15 & 16 \\ 
		\hline 
		$S(x)$ &  &  & 28 &  & & 60 & 15 &  \\ 
		\hline 
		\end{tabular}		
		\item  Dans un repère orthogonal , tracer la courbe représentative de la fonction $S$.
On prendra 1 cm pour 2 unités en abscisses et 1 cm pour 8 unités en ordonnées .
 		\end{enumerate}
\item Pour quelle valeur de $x$ l’aire du rectangle $ABCD$ est-elle maximale ? Que vaut cette aire et quelle est alors la nature de $ABCD$ ? 		
 \item 
\begin{enumerate}
		\item Établir le tableau de signes de l’expression $( 12 -x )( x - 4 )$ , pour $x \in [0;16]$.
		\item Vérifier que l'inéquation $-x^2 + 16x \geq 48$ équivaut à $( 12 - x )( x - 4 )\geq0 $. 
		\item  En déduire les valeurs de x pour lesquelles l'aire de $ABCD$ est supérieure ou égale à 48 $cm^2$ .
 \end{enumerate} 			
 \end{enumerate}
\end{DTL}
}% Les fonctions du second degré => Discriminant
%%##########\impress{\impressionEleve}{\begin{titre}[Généralités sur les fonctions]

\Titre{Optimisation}{4}
\end{titre}


\begin{CpsCol}
\textbf{Variations de fonctions}
\begin{description}
\item[$\square$] Déterminer l'extremum d'une fonction
\item[$\square$] Déterminer l'optimum d'une situation donnée
\end{description}
\end{CpsCol}


\mini{
\App{1}{VF-20}

\App{1}{VF-21}

\App{1}{VF-23}
}{
\App{1}{VF-22}

\PO{1}{VF-24}
}

\begin{DefT}{Extremum} \index{Extremum}
Un extremum est soit un maximum, soit un minimum.
\end{DefT}


\begin{DefT}{Majorant} \index{Extremum}
On dit qu'une fonction $f$ est majorée sur $I$ par $M$ lorsque pour tout réel $x \in I$, $f(x) \leq M$. $M$ est appelé le \textbf{majorant}.
\end{DefT}

\begin{DefT}{Minorant} \index{Minorant}
On dit qu'une fonction $f$ est majorée sur $I$ par $M$ lorsque pour tout réel $x \in I$, $f(x) \geq M$. $M$ est appelé le \textbf{minorant}.
\end{DefT}


\begin{DefT}{Fonction bornée} \index{Fonction!Bornée}
Une fonction est dite bornée sur un intervalle $I$ lorsqu'elle admet sur cet intervalle $I$ un majorant et un minorant.
\end{DefT}

\AD{1}{VF-25}






}% Optimisation
%%##########\impress{\impressionEleve}{\begin{titre}[Fonctions de référence]

\Titre{Les fonctions homographiques}{4}
\end{titre}


\begin{CpsCol}
\textbf{Variations de fonctions}
\begin{description}
\item[$\square$] Connaitre les variations des fonctions homographiques
\item[$\square$] Déterminer l'ensemble de définition d'une fonction homographique
\item[$\square$] Transformer des fonctions rationnelles simples
\end{description}
\end{CpsCol}


\Rec{1}{FR-35}


\begin{DefT}{Ensemble de définition} \index{Ensemble de définition}
L'ensemble de définition d'une fonction $f$ est l'ensemble de tous les réels qui possède une image par $f$.
\end{DefT}


\begin{DefT}{Fonction homographique} \index{Fonctions ! Homographique}
Une \textbf{fonction homographique} est une fonction définie sur $\R-\left\lbrace \frac{-d}{c} \right\rbrace$ par $f(x) = \frac{ax+b}{cx+d}$, où $a$, $b$, $c$ et $d$ des réels et $c \neq 0$.
\end{DefT}


\Rec{1}{FR-33}

\Exo{1}{FR-34}

\App{1}{FR-38}


\App{1}{FR-39}


}% Les fonctions homographiques
%
%
% 
% 
%%%##########\begin{titre}[Fonctions de référence]

\Titre{Fonctions de référence}{2}
\end{titre}


\subsection*{Matériel et mise en œuvre}

Pour chaque fonction de référence (carré, inverse, racine carrée, cube, affine : définitions et courbes représentatives) :

\begin{description}
 \item[•] Tableau de signes
  \item[•] Tableau de valeurs
 \item[•] Tableau de variations   
  \item[•] Courbe
  \item[•] Définition
 \end{description} 

Les élèves doivent reformer les fonctions de référence.  


\subsection*{Matériel}
 
 
\subsubsection*{Les définitions}


\begin{DefN}
On appelle \emph{fonction Inverse} la fonction définie sur $\mathbb{R}^*=]-\infty;0[\cup]0;+\infty[$ par $f : x\mapsto \dfrac{1}{x}$ ou $f(x)=\dfrac{1}{x}$.
\end{DefN}

\vspace{0.4cm}

\begin{DefN}
On appelle \emph{fonction Carré} la fonction définie sur $\mathbb{R}=]-\infty;+\infty[$ par $f : x\mapsto  x^2$ ou $f(x)=x^2$. 
\end{DefN}

\vspace{0.4cm} 

\begin{DefN}
On appelle \emph{fonction Cube} la fonction définie sur $\mathbb{R}=]-\infty;+\infty[$ par $f : x\mapsto  x^3$ ou $f(x)=x^3$. 
\end{DefN}

\vspace{0.4cm} 

\begin{DefN}
On appelle \emph{fonction Racine Carrée} la fonction définie sur $\mathbb{R^+}=[0;+\infty[$ par $f : x\mapsto  \sqrt{x}$ ou $f(x)=\sqrt{x}$. 
\end{DefN}

\vspace{0.4cm} 

\begin{DefN}
On appelle \emph{fonction affine} la fonction définie sur $\mathbb{R}=]-\infty;+\infty[$ par $f(x)=  ax+b$. 
\end{DefN} 

 \newpage
 
\subsubsection*{Les tableaux de signes} 

 \begin{tikzpicture}
   \tkzTabInit{$x$ / 1 , $f(x)$ / 1}{$-\infty$, $x_0$, $+\infty$}
   \tkzTabLine{,$-$, z, $+$, }
\end{tikzpicture}


\vspace{0.4cm}

\begin{tikzpicture}
   \tkzTabInit{$x$ / 1 , $f(x)$ / 1}{$-\infty$, $0$, $+\infty$}
   \tkzTabLine{,$+$, z, $+$, }
\end{tikzpicture}

\vspace{0.4cm}

\begin{tikzpicture}
   \tkzTabInit{$x$ / 1 , $f(x)$ / 1}{$-\infty$, $0$, $+\infty$}
   \tkzTabLine{,$-$, z, $+$, }
\end{tikzpicture}
\vspace{0.4cm}

\begin{tikzpicture}
   \tkzTabInit{$x$ / 1 , $f(x)$ / 1}{$-\infty$, $-2$, $+\infty$}
   \tkzTabLine{,$-$, z, $+$, }
\end{tikzpicture}

\vspace{0.4cm}

\begin{tikzpicture}
   \tkzTabInit{$x$ / 1 , $f(x)$ / 1}{$-\infty$, $0$, $+\infty$}
   \tkzTabLine{,$-$, d, $+$, }
\end{tikzpicture}

\vspace{0.4cm}

\begin{tikzpicture}
   \tkzTabInit{$x$ / 1 , $f(x)$ / 1}{$-\infty$, $0$, $+\infty$}
   \tkzTabLine{,$+$, d, $+$, }
\end{tikzpicture}

\vspace{0.4cm}

\begin{tikzpicture}
   \tkzTabInit{$x$ / 1 , $f(x)$ / 1}{$-\infty$, $0$, $+\infty$}
   \tkzTabLine{,$-$, z, $+$, }
\end{tikzpicture}

\vspace{0.4cm}

\begin{tikzpicture}
   \tkzTabInit{$x$ / 1 , $f(x)$ / 1}{  $0$, $+\infty$}
   \tkzTabLine{$0$,   $+$, }
\end{tikzpicture}

\vspace{0.4cm}

\begin{tikzpicture}
   \tkzTabInit{$x$ / 1 , $f(x)$ / 1}{  $0$, $+\infty$}
   \tkzTabLine{$0$,   $-$, }
\end{tikzpicture}
 \newpage
 
\subsubsection*{Les tableaux de valeurs} 

\begin{tabular}{|c|c|c|c|c|c|c|c|}
\hline 
$x$ & 0 & $\frac{1}{4}$ & 1 & 2,25 & 4 & 9 \\ 
\hline 
$f(x)$ & 0 & 0,5  & 1 & 1,5  & 2 & 3\\ 
\hline 
\end{tabular} 

\vspace{0.4cm}

\begin{tabular}{|c|c|c|c|c|c|c|c|c|c|c|c|c|c|c|c|}
\hline 
$x$ & $-3$ & $-\frac{5}{2}$ & $-2$ & $-\frac{3}{2}$ & $-1$ & $-0,5$ &  0 & $\frac{1}{2}$ & 1 & $\frac{3}{2}$ & 2 & 2,5 & 3 \\ 
\hline 
$f(x)$ & 9 & 6,25 & 4 & 2,25 & 1 & $\frac{1}{4}$ & 0 & 0,25 & 1 & 2,25 & 4 & 6,25 & 9 \\ 
\hline 
\end{tabular} 

\vspace{0.4cm}

\begin{tabular}{|c|c|c|c|c|c|c|c|c|c|c|c|c|c|c|c|}
\hline 
$x$ & $-3$ & $-\frac{5}{2}$ & $-2$ & $-\frac{3}{2}$ & $-1$ & $-0,5$ &  0 & $\frac{1}{2}$ & 1 & $\frac{3}{2}$ & 2 & 2,5 & 3 \\ 
\hline 
$f(x)$ & $-27$ & $-15,625$ & $-8$ & $-3,375$ & 1 & $\frac{1}{8}$ & 0 & 0,125 & 1 & 3,375 & 8 & 15,625  & 27 \\ 
\hline 
\end{tabular} 

\vspace{0.4cm}

\begin{tabular}{|c|c|c|c|c|c|c|c|c|c|c|c|c|c|c|c|}
\hline 
$x$ & $-5$ & $-4$ & $-2$ &   $-1$ & $-0,5$ &  0 & $\frac{1}{2}$ & 1 & 2 &   4 & 5 \\ 
\hline 
$f(x)$ & $-\frac{1}{5}$ & $-0,25$ &   $-0,5$ & $-1$ & $-0,2$ & 0 & 0,2  & 1 & 0,5 &  0,25  & 0,2 \\ 
\hline 
\end{tabular}

\vspace{0.4cm}

\begin{tabular}{|c|c|c|c|c|c|c|c|c|c|c|c|c|c|c|c|}
\hline 
$x$ & $-5$ & $-4$ & $-2$ &   $-1$ & $-0,5$ &  0 & $\frac{1}{2}$ & 1 & 2 &   4 & 5 \\ 
\hline 
$f(x)$ & $-\frac{1}{5}$ & $-0,25$ &   $-0,5$ & $-1$ & $-2$ & 0 & 2 & 1 & 0,5 &  0,25  & 0,2 \\ 
\hline 
\end{tabular}

\vspace{0.4cm} 

\begin{tabular}{|c|c|c|c|c|c|c|c|c|c|c|c|c|c|c|c|}
\hline 
$x$ & $-10$ &   $-5$ &  0 & 5 & 10 \\ 
\hline 
$f(x)$ & $-1$ &   $0$ &  1 & 2 & 3  \\ 
\hline 
\end{tabular}


\subsubsection*{Propriétés des variations} 

\begin{ThN}
La fonction \emph{Cube} est strictement croissante sur $\R$. 
\end{ThN}

\vspace{0.4cm} 

\begin{ThN}
La fonction \emph{Inverse} est décroissante sur $]-\infty;0[$ et  décroissante sur $]0;+\infty[$. 
\end{ThN}

\vspace{0.4cm}

\begin{ThN}
La fonction \emph{Carré} est strictement décroissante sur $\R^-$ et strictement croissante sur $\R^+$. 
\end{ThN}

\vspace{0.4cm} 

\begin{ThN}
La fonction \emph{Racine Carrée} est strictement croissante sur $\R+$. 
\end{ThN}

\vspace{0.4cm} 
 
 \begin{ThN}
La fonction \emph{affine}, $f:x \mapsto ax +b$ définie sur $\R$, est strictement croissante sur $\R$ lorsque $a>0$, strictement décroissante sur $\R$ lorsque $a<0$. 
\end{ThN}

 \newpage
 
\subsubsection*{Les tableaux de variations} 

\begin{tikzpicture}
   \tkzTabInit{$x$ / 1 , $f(x)$ / 2}{$-\infty$,   $+\infty$}
   \tkzTabVar{+/ $+\infty$, -/ $-\infty$}
\end{tikzpicture}



\vspace{0.4cm}

\begin{tikzpicture}
   \tkzTabInit{$x$ / 1 , $f(x)$ / 2}{$-\infty$,   $+\infty$}
   \tkzTabVar{-/ $+\infty$, +/ $-\infty$}
\end{tikzpicture}



\vspace{0.4cm}

\begin{tikzpicture}
   \tkzTabInit{$x$ / 1 , $f(x)$ / 2}{$-\infty$,   $+\infty$}
   \tkzTabVar{-/ $-\infty$, +/ $+\infty$}
\end{tikzpicture}

\vspace{0.4cm}

\begin{tikzpicture}
   \tkzTabInit{$x$ / 1 , $f(x)$ / 2}{$-\infty$, 0 ,  $+\infty$}
   \tkzTabVar{-/ $+\infty$,  +/0 ,  -/ $+\infty$}
\end{tikzpicture}

\vspace{0.4cm}

\begin{tikzpicture}
   \tkzTabInit{$x$ / 1 , $f(x)$ / 2}{$-\infty$, 0 ,  $+\infty$}
   \tkzTabVar{+/ $+\infty$,  -/0 ,  +/ $+\infty$}
\end{tikzpicture}

\vspace{0.4cm}

\begin{tikzpicture}
\tkzTabInit[lgt=1,espcl=2]{ $x$ / 1,$f $ / 2}
{ $-\infty$ , $0$ ,$+\infty$}
\tkzTabVar{+/$0$ , -D+ /$ $/$ $ , -/$0$}
\end{tikzpicture}
\vspace{0.4cm}

\begin{tikzpicture}
   \tkzTabInit{$x$ / 1 , $f(x)$ / 2}{$-\infty$,   $+\infty$}
   \tkzTabVar{-/ $-\infty$, +/ $+\infty$}
\end{tikzpicture}

\vspace{0.4cm}

\begin{tikzpicture}
   \tkzTabInit{$x$ / 1 , $f(x)$ / 2}{0,   $+\infty$}
   \tkzTabVar{-/ $-\infty$, +/ $+\infty$}
\end{tikzpicture}

  

 
 
 
 
 
 
\subsubsection*{Les courbes}

\begin{tikzpicture}[line cap=round,line join=round,>=triangle 45,x=1.0cm,y=1.0cm]
\begin{axis}[
x=1.0cm,y=1.0cm,
axis lines=middle,
ymajorgrids=true,
xmajorgrids=true,
xmin=-7.66,
xmax=4.12,
ymin=-2.38,
ymax=2.7,
xtick={-7.0,-6.0,...,4.0},
ytick={-2.0,-1.0,...,2.0},]
\clip(-7.66,-2.38) rectangle (4.12,2.7);
\draw [line width=2.pt,domain=-7.66:4.12] plot(\x,{(--1.--0.2*\x)/1.});
\end{axis}
\end{tikzpicture}



\begin{tikzpicture}[line cap=round,line join=round,>=triangle 45,x=1.0cm,y=1.0cm]
\begin{axis}[
x=1.0cm,y=1.0cm,
axis lines=middle,
ymajorgrids=true,
xmajorgrids=true,
xmin=-2.98,
xmax=3.54,
ymin=-0.8999999999999996,
ymax=6.659999999999997,
xtick={-2.0,-1.0,...,3.0},
ytick={-0.0,1.0,...,6.0},]
\clip(-2.98,-0.9) rectangle (3.54,6.66);
\draw [samples=50,rotate around={0.:(0.,0.)},xshift=0.cm,yshift=0.cm,line width=2.pt,domain=-4.0:4.0)] plot (\x,{(\x)^2/2/0.5});
\end{axis}
\end{tikzpicture}

 
\begin{tikzpicture}[line cap=round,line join=round,>=triangle 45,x=1.0cm,y=1.0cm]
\begin{axis}[
x=1.0cm,y=1.0cm,
axis lines=middle,
ymajorgrids=true,
xmajorgrids=true,
xmin=-0.640000000000001,
xmax=9.520000000000007,
ymin=-0.6600000000000019,
ymax=4.239999999999996,
xtick={-0.0,1.0,...,9.0},
ytick={-0.0,1.0,...,4.0},]
\clip(-0.64,-0.66) rectangle (9.52,4.24);
\draw[line width=2.pt,color=black,smooth,samples=100,domain=4.079999953157344E-8:9.520000000000007] plot(\x,{sqrt((\x))});
\end{axis}
\end{tikzpicture}


 
\begin{center}\begin{tikzpicture}[x=.8cm,y=.8cm,>=stealth]
\draw[line width=.1pt] (-5,-5) grid[step=1] (5,5);
\draw[line width=1pt] (0,0) node[below left, fill=white]{$O$} (1,-2pt) node[below, fill=white]{$1$} --(1,2pt) (-2pt,1) node[left, fill=white]{$1$} --(2pt,1);
\draw[color=red] (2.5,6.25) node[left,fill=white]{$\mathcal{C}_f$};
\draw[line width=1pt,->] (-5,0)--(5,0);
\draw[line width=1pt,->] (0,-5)--(0,5);
\draw[line width=1pt,color=red] plot[smooth, samples=100, domain=-5:-0.19] (\x,{1/\x});
\draw[line width=1pt,color=red] plot[smooth, samples=100, domain=0.19:5] (\x,{1/\x});
\end{tikzpicture}\end{center}


 
\begin{tikzpicture}[line cap=round,line join=round,>=triangle 45,x=1.0cm,y=0.38800705467372043cm]
\begin{axis}[
x=1.0cm,y=0.38800705467372043cm,
axis lines=middle,
ymajorgrids=true,
xmajorgrids=true,
xmin=-2.896551724137932,
xmax=2.8965517241379306,
ymin=-10.09090909090913,
ymax=15.681818181818203,
xtick={-2.0,-1.0,...,2.0},
ytick={-10.0,-5.0,...,15.0},]
\clip(-2.896551724137932,-10.09090909090913) rectangle (2.8965517241379306,15.681818181818203);
\draw[line width=2.pt,color=black,smooth,samples=100,domain=-2.896551724137932:2.8965517241379306] plot(\x,{(\x)^(3.0)});
 
\end{axis}
\end{tikzpicture}



% La fonction Carré + rep. graphique
%
%
%
%%%%%%%%%%%%%%%%%%%%%%%%%%%%%%%%%%%%%%%%%%%%%%%%%%%%%%%%%%%%%%%%%%%%%%%%%%%%%%%%%%%%%%%%%%%%%%%%%%%%%%%%%%%%%%%%%%%%%%%%%%%%%%%%%%%%%%%%%%%%%%%%%%
%%%%%%%%%%%%%%%%%%%%%%%%%%%%%%%%%%%%%%%%%%%%%%%%%%%%%%%%%%%%%%%%%%%%%%%%%%%%%%%%%%%%%%%%%%%%%%%%%%%%%%%%%%%%%%%%%%%%%%%%%%%%%%%%%%%%%%%%%%%%%%%
%%%%%%%%%%%%%%%%%%%%%%%%%%%%%%%%%%%%%%%%%%%%%%%%%%%  Organisation et gestion de données   %%%%%%%%%%%%%%%%%%%%%%%%%%%%%%%%%%%%%%%%%%%%%%%%%%%%%%%
%%%%%%%%%%%%%%%%%%%%%%%%%%%%%%%%%%%%%%%%%%%%%%%%%%%%%%%%%%%%%%%%%%%%%%%%%%%%%%%%%%%%%%%%%%%%%%%%%%%%%%%%%%%%%%%%%%%%%%%%%%%%%%%%%%%%%%%%%%%%%%%%%
%%%%%%%%%%%%%%%%%%%%%%%%%%%%%%%%%%%%%%%%%%%%%%%%%%%%%%%%%%%%%%%%%%%%%%%%%%%%%%%%%%%%%%%%%%%%%%%%%%%%%%%%%%%%%%%%%%%%%%%%%%%%%%%%%%%%%%%%%%
%%
% \part{Statistiques et probabilités}
%%
% \chapter{Informations chiffrées}
% 
% \impress{\impressionEleve}{\begin{titre}[Informations chiffrées]

\Titre{Proportions et pourcentages}{4}
\end{titre}


\begin{CpsCol}
\begin{description}
\item[$\square$] Exploiter la relation entre effectifs, proportions et pourcentages.
\item[$\square$] Traiter des situations simples mettant en jeu des pourcentages de pourcentages.
\end{description}
\end{CpsCol}

 

\begin{DefT}{Proportion}\index{Proportion}
Soient $E$ un ensemble de référence non vide et $n_E$ le nombre d'élément de $E$.
Soient $A$ une partie de $E$ (un sous ensemble de $E$) et $n_A$ le nombre d'élément de $A$
On appelle \textbf{proportion} $p$ de $A$ dans $E$ le réel défini par $p=\frac{n_A}{n_E}$ 
\end{DefT}

\begin{Ex}
Lors d'une élection, on recense 1648 électeurs. Pourtant, seuls 1236 se sont déplacés aux urnes. La proportion d'électeurs votants est donc : $p=\frac{1236}{1648}=0,75=\frac{75}{100}=75\%$ 
\end{Ex}



\begin{Th}
Soit $p$ le réel défini par $p=\frac{n_A}{n_E}$. Alors $n_A=p \times n_E$.

Et pour $p\neq 0$, $n_E=\frac{n_A}{p}$.
\end{Th}


\begin{Th}
A tout ensemble $A$ contenu dans un ensemble $E$ non vide, on a : $0 \leq p \leq 1$.
\end{Th}


%Exercices



\mini{
\EPC{1}{IC-8}{Calculer}

\EPC{0}{IC-11}{Calculer}
}{
\EPC{1}{IC-12}{Calculer.}

\EPC{0}{IC-19}{Calculer.}
}






\begin{ThT}{Pourcentage de pourcentage} \index{Pourcentage!Pourcentage de pourcentage}
Soit $F$ un ensemble non vide de référence, $E$ une partie non vide de $F$ et $A$ une partie de $E$.\\
Si $p_1$ est la proportion de $A$ dans $E$ et si $p_2$ est la proportion de $E$ dans $F$ alors la proportion de $A$ dans $F$ est $p=p_1 \times p_2$.
\end{ThT}

\begin{Ex}
\textit{Lors d'une élection, on recense 1648 électeurs inscrits sur les listes électorales mais seuls 1236 se sont déplacés aux urnes. Le candidat  sortant   a obtenu 40\% des voix des votants. Quel est la proportion des voix par rapport aux inscrits ?  }

La proportion de votants est $p=\frac{1236}{1648}=0,75=\frac{75}{100}=75\%$.

Parmi les votants, le candidat sortant a obtenu 36\% des voix des votants donc $p = 0,4 \times 0,75 = 0,3$.

Ainsi, le candidat sortant a recueilli 30\% des voix des inscrits.
\end{Ex}


\mini{
\EPC{1}{IC-15}{Calculer.}

\EPC{1}{IC-20}{Calculer.}

\EPC{1}{IC-14}{Représenter.}

\EPC{0}{IC-23}{Raisonner.}
}{

\EPC{0}{IC-13}{Représenter.}

\EPCP{1}{IC-21}{Chercher.}

\EPC{0}{IC-22}{Représenter.}
}



\EPC{1}{IC-17}{Représenter.}

\EPC{1}{IC-18}{Représenter.}
 }% Proportion
% \impress{\impressionEleve}{\begin{titre}[Informations chiffrées]

\Titre{Taux d'évolution}{4}
\end{titre}


\begin{CpsCol}
\begin{description}
\item[$\square$] Exploiter la relation entre deux valeurs successives et leur taux d’évolution.
\end{description}
\end{CpsCol}

\EPC{1}{IC-41}{Modéliser. Calculer.}

\begin{DefT}{Variation absolue, relative}\index{Variation absolue}\index{Variation relative, Taux d'évolution}
On considère une quantité qui varie et on appelle $V_i$ sa valeur initiale et $V_f$ sa valeur finale. 
La \textbf{variation absolue} $\Delta V$ est donnée par : $\Delta V = V_f - V_i$.

La \textbf{variation relative} ou \textbf{taux d'évolution $t$} est donnée par : $t = \frac{V_f - V_i}{V_i}$.
\end{DefT}

\begin{Ex}
Paul place 110 euros sur son compte. Il se rend compte 15 jours plus tard qu'il possède 121 euros sur son compte.

La variation absolue est $\Delta V = 121 - 110 = +11$ euros.

La variation relative est $t = \frac{121 - 110}{110}=\frac{11}{110}= +10\%$.
\end{Ex}


%%%%%%% Exercices



\begin{DefT}{Taux d'évolution}\index{Taux d'évolution}

 Soit $t$ le \textbf{taux d'évolution} qui permet à une quantité de passer de $V_i$ à $V_f$.
 On a alors : $V_f = (1+t) \times V_i$.
\end{DefT}

 
\begin{DefT}{Coefficient multiplicateur}
 
Le réel $1+t$ est appelé \textbf{coefficient multiplicateur} associé au taux $t$. On peut le noter $CM$. Avec cette notation, on a alors  $V_f = CM \times V_i$ où $CM =1+t$. 
\end{DefT}

\EPC{1}{IC-2}{Calculer.}


\begin{minipage}{0.58\linewidth}
\begin{Th} 

Soit $CM$ le coefficient multiplicateur et $t$ le taux d'évolution, $t =CM-1$. 
\end{Th}
\end{minipage}
\hfill
\begin{minipage}{0.38\linewidth}

\begin{Rq} 
Pour $V_i \neq 0$, $CM=\frac{V_f}{V_i}$ . 
\end{Rq}
\end{minipage}


\begin{Th} 
\begin{description}
\item[•] Dans le cas d'une baisse, le taux d'évolution est négatif et le $CM$ est compris entre 0 et 1.
\item[•] Dans le cas d'une hausse, le taux d'évolution est positif et le $CM$ est supérieur à 1.
\end{description}

 
\end{Th}

%Exercices




\mini{
\EPC{1}{IC-27}{Calculer.}  

\EPC{1}{IC-28}{Raisonner.}

\EPC{1}{IC-29}{Modéliser. Calculer.}

\EPC{0}{IC-24}{Modéliser. Calculer.} 

}{
\EPC{1}{IC-25}{Modéliser. Calculer.} 

\EPC{0}{IC-26}{Raisonner. Calculer.} 

\EPC{1}{IC-30}{Raisonner. Calculer.}


\EPC{1}{IC-31}{Modéliser. Calculer.} 
}


 

\EPC{0}{IC-1}{Raisonner. Modéliser. Calculer.}}% Taux d'évolutions
% \impress{\impressionEleve}{\begin{titre}[Informations chiffrées]

\Titre{Évolutions successives}{4}
\end{titre}


\begin{CpsCol}
\begin{description}
\item[$\square$] Exploiter la relation entre deux valeurs successives et leur taux d'évolution.
\item[$\square$] Calculer le taux d'évolution global à partir des taux d'évolution successifs. Calculer un taux d'évolution réciproque.
\end{description}
\end{CpsCol}

 

\begin{DefT}{Évolutions successives}\index{Évolutions! successives}
Lorsqu'une quantité subit des évolutions successives $t_1$, $t_2$, $t_3$, ...de sa valeur initiale, cette quantité subit une évolution globale $t$.
\end{DefT}

\begin{Ex}
 Chaque année, le prix de l'électricité en France augmente d'un certain pourcentage. Le prix de l'électricité a augmenté de 5,9\% le 1er juin 2019 et il avait augmenté de  2,8\% le 1er juin 2018.
\end{Ex}

\begin{ThT}{Taux global d'évolution}
Le taux global d'évolution correspondant à deux évolutions successives de taux respectifs $t_1$ et $t_2$  est le réel $t$ tel que $1+t = \left(1+t_1\right)\left(1+ t_2 \right)$.
\end{ThT}


\begin{Ex}
Dans l'exemple précédent, le prix de l'électricité a augmenté de 5,9\% le 1er juin 2019 et augmenté de  2,8\% le 1er juin 2018.

$t_1=5,9\%=\frac{5,9}{100}$ et $t_2=2,8\%=\frac{2,8}{100}$

donc $1+t = \left(1+\frac{2,8}{100}\right)\left(1+ \frac{5,9}{100} \right)$.
 
$1+t = 1,028 \times 1,059 \approx 1,088$ donc $t = 1,088 - 1 \approx 0,088$ ou $t \approx \frac{8,8}{100}$ ou aussi, $t\approx 8,8\%$.
\end{Ex}




\begin{Rq}
On peut faire correspondre un coefficient multiplicateur global, $CM_g$, au taux global d'évolution tel que $$CM = \left(1+t_1\right)\left(1+t_2\right)$$
\end{Rq}


\begin{Ex}
Un objet A coute 75 euros initialement en 2015. En 2016, cet objet a augmenté de 5\% et son prix a baissé en 2017 de 4\%. Le coefficient multiplicateur global est donc $CM = \left(1+\frac{5}{100}\right)\left(1-\frac{4}{100}\right)=1,05 \times 0,96 = 1,008$.\\
Le taux de pourcentage est donc égal à $0,008 \times 100 = 0,8$. Sur les 2 années, l'objet a augmenté de 0,8\%. 
\end{Ex}


\begin{Dem}
Soit deux taux respectifs $t_1$ et $t_2$. On appelle $CM_1 = 1+t_1$ et $CM_2 = 1+t_2$.\\
Soit $P_0$ la valeur initiale. Après la première évolution, $P_1=CM_1 \times P_0 = \left(1+t_1\right)\times P_0 $\\
Après la seconde évolution, $P_2=CM_2 \times P_1 = CM_2 \times CM_1 \times P_0 =\left(1+t_2\right)\times \left(1+t_1 \right)\times P_0 $.\\
Après les 2 évolutions, $P_2=\left(1+t_2\right)\times \left(1+t_1\right)\times P_0 $. Donc Le coefficient multiplicateur global est $CM = \left(1+t_1\right)\left(1+t_2\right)$.
\end{Dem}

\begin{ThT}{Taux global d'évolution pour $n$ évolutions}
Le taux global d'évolution correspondant à $n$ évolutions successives de taux respectifs $t_1$, $t_2$, $t_3$, ..., $t_n$ est le réel $t$ tel que $$1+t = \left(1+t_1\right)\left(1+t_2\right)\left(1+t_3\right)\cdots \left(1+t_n\right)$$
\end{ThT}

\begin{Rq}
On peut faire correspondre un coefficient multiplicateur global, $CM_g$, au taux global d'évolution tel que $$CM = \left(1+t_1\right) \left(1+t_2\right)\left(1+t_3\right)\cdots \left(1+t_n\right)$$
\end{Rq}

\begin{Ex}
Un article subit les évolutions suivantes lors des 3 dernières années : 3\% en 2015, $-1$\% en 2016 et 2\% en 2017. Quel est le taux global d'évolution sur ces 3 années ?

Le taux d'évolution global est : $CM_g=(1+0,03)(1-0,01)(1+0,02)=1,03 \times 0,99 \times 1,02 = 1,04$ arrondi à 0,01 près.

On peut donc dire que sur les 3 dernières années, l'article a augmenté de 4\%.
\end{Ex}



\mini{
\EPC{1}{IC-34}{Chercher.}

%\EPC{1}{IC-16}{Chercher.}
\EPC{1}{IC-36}{Chercher.}


\EPC{0}{IC-5}{Chercher.}
}{
\EPC{1}{IC-33}{Chercher.}

\EPC{1}{IC-3}{Chercher.}
}


\EPC{0}{IC-9}{Chercher.}




\begin{DefT}{Évolution réciproque}\index{Évolution! réciproque}
Une quantité non nulle $V_i$ subit une évolution de taux $t$ et devient égale à une quantité $V_f$. 

Le \textbf{taux réciproque} de $t$ est le taux d'évolution $t'$ qui permet de passer de $V_f$ à $V_i$.
\end{DefT}


\begin{Ex}
\begin{minipage}{0.7\linewidth}
 
Un objet A coute 50 euros (valeur initiale). Une baisse de 20\% fait passer son prix à 40 euros(valeur finale). Le taux d'évolution $t$ est donc égal à $-20$\%.  

Le taux réciproque $t'$ est le taux qui fait passer de 40 euros (la valeur du prix finale) à 50 euros (la valeur du prix initiale). Donc $t'=\frac{50}{40}=1,25$ donc $t' =+25\%$.
\end{minipage}
\begin{minipage}{0.4\linewidth}
\includegraphics[scale=0.5]{taux_reciproque.jpg} 
\end{minipage}

\end{Ex}


\begin{ThT}{Taux d'évolution réciproque}
\begin{minipage}{0.74\linewidth}
 Le coefficient multiplicateur réciproque $CM'$ associe à l'évolution réciproque $t'$ est l'inverse du coefficient multiplicateur non nul $CM$ associé à l'évolution de départ $t$. 
 
 On a : $CM' = \frac{1}{CM}$ c'est à dire : $CM \times CM' = 1$ 

\end{minipage}
\hfill
\begin{minipage}{0.25\linewidth}
\includegraphics[scale=0.5]{taux_reciproque2.png} 
\end{minipage}
\end{ThT}

\ROC





\begin{Rq}
Le taux réciproque $t'$ n'est pas égal au taux initial $t'$.
\end{Rq}
 


\mini{
\EPC{1}{IC-39}{Chercher.}

\EPC{1}{IC-7}{Chercher.}

\EPCP{1}{IC-40}{Chercher.}
}{
\EPC{1}{IC-35}{Chercher.}

\EPC{0}{IC-37}{Chercher.}
 
\EPC{1}{IC-38}{Chercher.} 
}
 
 }% Evolutions successives
%
%
%%
% \chapter{Statistiques}
%%
%%Il y a 3 sortes de mensonges : les mensonges, les sacrés mensonges et les statistiques.
%%
%%\hfill{Mark Twain}
%%
%\impress{\impressionEleve}{\begin{titre}[Les statistiques]

{\color{bleu3}{\LARGE Utilisation d'un tableur} \hfill{Niveau 1}}
\end{titre}



\begin{CpsCol}
\textbf{Interpréter, représenter et traiter des données}
\begin{description}
\item[$\square$] Calculer une fréquence
\item[$\square$] Utiliser un tableur pour effectuer des calcul
\end{description}
\end{CpsCol}

\begin{Rec}

Le travail de statistiques est souvent fastidieux, on a toujours recours à un tableur lorsque les données sont trop importantes. Une feuille de calcul d'un tableur se présente comme cela, en cellules. La cellule rouge est la cellule C3.

L'intérêt d'un tableur est le calcul automatique des caractéristiques demandées : moyenne, fréquence, effectif cumulés,... Pour cela, il faut utiliser des formules mathématiques.

\vspace{0.4cm}

\begin{tabular}{|c|m{2cm}|m{2cm}|m{2cm}|m{2cm}|m{2cm}|}
\hline 
\rowcolor{gray} & A & B & C & D &...\\ 
\hline 
\cellcolor{gray}1 & &4 &  & &  \\ 
\hline 
\cellcolor{gray}2 & & & 5 & &   \\ 
\hline 
\cellcolor{gray}3 & & & \cellcolor{red} & &  \\ 
\hline 
\cellcolor{gray}... & & &  & & \\ 
\hline 
\end{tabular} 

\subsubsection*{Quelques formules : Utilisation du signe = avant toute formule}

\begin{description}
\item[=a+b] calcule la somme de  $a$ et de $b$. 
\item[=SOMME(plage)] calcule la somme des cellules dans une plage rectangulaire.
\item[=NB.SI(plage;critère] renvoie parmi les cellules de la plage celle qui vérifie le critères.
\item[=a/b] calcule le quotient de  $a$ par $b$. $b$ doit être non nul.
\end{description}

\subsubsection*{Utilisations}
\begin{description}[leftmargin=*]
\item Pour calculer 4+5 dans la cellule D3, on tapera dans la cellule D3, \textbf{=B1+C2}. Lorsque une des valeurs des cellules B1 ou C2 change, la somme change alors. 
\item Pour calculer 4/5 dans la cellule B3, on tapera dans la cellule DB3, \textbf{=B1/C2}. 
\end{description}

\subsection*{Application concrète}

Voici les notes obtenues à un contrôle sur 10 par une classe de quatrième :\\
0 -- 1 -- 2 -- 2 -- 3 -- 3 -- 3 -- 3 -- 4 -- 4 -- 5 -- 5 -- 5 -- 6 -- 6 -- 6 -- 7 -- 7 -- 8 -- 8 -- 8 -- 9 -- 9 -- 10 -- 10.

\begin{enumerate}
\item Créer un tableau sur une feuille de calcul et complète le tableau ci-dessous avec les formules adéquates.
\item Combien d'élèves ont obtenu moins de 5 ?
\item Quel est le pourcentage d'élèves qui ont obtenu 8 ?
\item Quel est le pourcentage d'élèves qui ont obtenu au moins 8 ?
\item Change un 3 et par un 8 dans la série statistique. Remarque le changement de résultats.
\end{enumerate}

\begin{tabular}{|p{4.2cm}*{11}{|p{5mm}}|p{8mm}|}
\hline
Note&0 & 1 & 2& 3 & 4 & 5 & 6&7 &8 &9 &10&Total \\
\hline
Effectifs&1 & 1 & 2& &&&& & &&& \\
\hline
Effectifs cumulés&1 & 2 & 4& &  & & & && & & \\
\hline
Fréquences &0,04 & 0,04 & 0,08 & &  & & & & & & & \\
\hline
Fréquences cumulées &0,04  & 0,08  & 0,16  &  &  & & & & & & & \\
\hline
\end{tabular}




\subsubsection*{Effectifs -- Effectifs cumulés}
	Pour chaque note, {\em l'effectif} est le nombre d'élèves ayant eu cette note.	Par exemple (dans le tableau), 1 élève a eu 0 ;	2 élèves ont eu 1\dots\\ 
	Pour chaque note, {\em l'effectif cumulé} est le nombre d'élèves ayant eu cette note ou une note inférieure. Pour le calculer, il suffit à chaque fois de cumuler les effectifs.\\
	Par exemple (dans le tableau), 1 élève a eu 0 ;	$1+2=3$ élèves ont eu 1 au plus ; $3+4=7$ élèves ont eu 2 au plus\dots
	
\subsubsection*{Fréquences -- Fréquences cumulées}
	Pour chaque note, {\em la fréquence} exprime la proportion d'élèves. Par exemple, sur les 20 élèves, 4 élèves ont eu une note de 2.
	La proportion est de 4 sur 20 ou $4\over20$ que l'on exprime en \%\ par le calcul $100\times{4\over20}=20$.\\
	Comme pour les effectifs cumulés, les {\em fréquences cumulées}	sont obtenues en cumulant les fréquences.
\end{Rec}

\begin{DefT}{Fréquence}
La population $P$ étudiée a un effectif total égal à $N$.
La fréquence $f$ d'un sous ensemble de la population, appelé $A$, d'effectif $n$ est le quotient de cet effectif sur l'effectif total $N$. On écrit alors $f_A=\frac{n}{N}$.
\end{DefT}


\begin{Ex}
Une classe de Quatrième compte 25 élèves dont 12 filles.\\
La population est l'ensemble des élèves de la classe. L'effectif de la population est égal à 25, c'est l'effectif total : $N=25$.\\
Pour calculer la fréquence des filles dans cette classe, on détermine l'effectif de ce sous ensemble : les filles. $n=12$\\
Donc $f_{filles}=\frac{12}{25}$.
\end{Ex}

\begin{AD}

Lors du concours Algoréa, les 12\% meilleurs élèves sont qualifiés pour le tour suivant. 
\begin{description}
\item[•] En sixième, Mathis est arrivée $682^{ième}$ sur 6800 participants.
\item[•] En cinquième, Pol est arrivé $524^{ième}$ sur 5200 participants.
\item[•] En quatrième, Luisa est arrivée $855^{ième}$ sur 8200 participants.
\item[•] En troisième, Rafaela est arrivée $423^{ième}$ sur 4500 participants.
\end{description}
Quel élèves ont été sélectionnés pour le tour suivant ?

\end{AD}


\begin{AD}

Lors de la première édition de la Course aux Nombres, les 24 élèves de la classe de Cinquième 2 en Colombie ont obtenu les résultats suivants. Les notes sont évaluées sur 30 points.


\begin{enumerate}
\item Reproduire la feuille de calcul comme indiqué ci dessous.

\begin{tabular}{|c|c|c|c|c|c|c|c|c|c|}
\hline 
\rowcolor{gray} & A & B & C & D & E & F & G & H & I \\ 
\hline 
\cellcolor{gray} 1 & 11 & 16 & 22 & 20 & 26 &  &  & Nombres d'élèves total &  \\ 
\hline 
\cellcolor{gray}2& 26 & 18 & 26 & 19& 16 &  &  & Nombre d'élèves dont la nombre est égale à 22 &  \\ 
\hline 
\cellcolor{gray}3 & 17 &27 & 18 & 16 & 22 &  & & Fréquence d'élèves dont la note est égale à 22 &  \\ 
\hline 
\cellcolor{gray}4 & 16 & 19 & 11 & 21 & 17 & &  & Nombre d'élèves dont la note est égale à 19 &  \\ 
\hline 
\cellcolor{gray}5 & 22 & 15 & 22 & 23 &  &  &  & Fréquence d'élèves dont la note est égale à 19 &  \\ 
\hline 
\end{tabular} 

\item  
\begin{enumerate}
\item  Déterminer dans la cellule I1 le nombre d'élèves participants à ce jeu. 
Rappel : Pour déterminer le nombre de cases remplies d'un tableau, on utilise la syntaxe : =NB(A1:E5). 
\item  Déterminer dans la cellule I2 le nombre d'élèves dont la note est égale à 22 ? 
Rappel : Pour déterminer le nombre de cases remplies avec la valeur 22, on utilise la syntaxe : =NB.SI(A1:E5;22). 
\item  Calcule dans la cellule I3 la fréquence des élèves ayant obtenu 22.
\end{enumerate}
\item  
\begin{enumerate}
\item Détermine la moyenne de cette classe. On pourra utiliser la syntaxe "=MOYENNE(A1:E5)".
\item Explique par une phrase le calcul du tableur pour donner la moyenne.
\item 
\begin{enumerate}
\item Complète le tableau ci-dessous.

\begin{tabular}{|c|c|c|c|c|c|c|}
\hline 
Notes & $[0;5[$ &  $[5;10[$  &  $[10;15[$  &  $[15;20[$  &  $[20;25[$  &  $[25;30]$  \\ 
\hline 
Fréquence &  &  &  &  &  &  \\ 
\hline 
\end{tabular} 

\item Calcule la moyenne avec ce regroupement.	
\end{enumerate}
\item Création d'un diagramme avec un tableur
\begin{enumerate}
\item  Sélectionne la plage de données A1:E5 puis clique sur l'icône graphique
\item  Quel est le problème de la plage de données A1:E5 ?
\end{enumerate}

\end{enumerate}
\end{enumerate}




\end{AD}


\begin{autoeval}
\begin{tabular}{p{12cm}p{0.5cm}p{0.5cm}p{0.5cm}p{1cm}}
\textbf{Compétences visées} &  M I & MF & MF  & TBM \vcomp \\ 
Calculer une fréquence & $\square$ & $\square$  & $\square$ & $\square$ \vcomp \\ 
Utiliser un tableur pour effectuer des calcul & $\square$ & $\square$ & $\square$ & $\square$ \vcomp \\ 
\end{tabular}
{\footnotesize MI : maitrise insuffisante ; MF = Maitrise fragile ; MS = Maitrise satisfaisante ; TBM = Très bonne maitrise}
 
\end{autoeval}}% Effectifs cumulés, fréquences cumulées 
%%##########\impress{\impressionEleve}{\input{CHAPITRES/Stat-F0_cor}}% Effectifs cumulés, fréquences cumulées 
% \impress{\impressionEleve}{\begin{titre}[Les statistiques]

{\color{bleu3}{\LARGE Lecture et représentation de données} \hfill{Niveau 2}}
\end{titre}



\begin{CpsCol}
\textbf{Interpréter, représenter et traiter des données}
\begin{description}
\item[$\square$] Lire des données sur différents supports
\end{description}
\end{CpsCol}

\begin{AD}

Les notes d'une classe de Seconde sont représentées par le diagramme à bâtons ci-dessous.

\begin{center}
\definecolor{ffqqqq}{rgb}{1.,0.,0.}
\definecolor{cqcqcq}{rgb}{0.7529411764705882,0.7529411764705882,0.7529411764705882}
\begin{tikzpicture}[line cap=round,line join=round,>=triangle 45,x=1.0cm,y=1.0cm]
\draw [color=cqcqcq,, xstep=1.0cm,ystep=1.0cm] (-0.76,-1.1175) grid (11.64,7.2625);
\draw[->,color=black] (-0.76,0.) -- (11.64,0.);
\foreach \x in {,1.,2.,3.,4.,5.,6.,7.,8.,9.,10.,11.}
\draw[shift={(\x,0)},color=black] (0pt,2pt) -- (0pt,-2pt) node[below] {\footnotesize $\x$};
\draw[->,color=black] (0.,-1.1175) -- (0.,7.2625);
\foreach \y in {-1.,1.,2.,3.,4.,5.,6.,7.}
\draw[shift={(0,\y)},color=black] (2pt,0pt) -- (-2pt,0pt) node[left] {\footnotesize $\y$};
\draw[color=black] (0pt,-10pt) node[right] {\footnotesize $0$};
\clip(-0.76,-1.1175) rectangle (11.64,7.2625);
\draw [line width=2.pt,color=ffqqqq] (1.,0.)-- (1.,2.);
\draw [line width=1.2pt,color=ffqqqq] (2.,0.)-- (2.,3.);
\draw [line width=1.2pt,color=ffqqqq] (3.,0.)-- (3.,2.);
\draw [line width=1.2pt,color=ffqqqq] (4.,0.)-- (4.,4.);
\draw [line width=1.2pt,color=ffqqqq] (5.,0.)-- (5.,6.);
\draw [line width=1.2pt,color=ffqqqq] (6.,0.)-- (6.,5.);
\draw [line width=1.2pt,color=ffqqqq] (7.,0.)-- (7.,5.);
\draw [line width=1.2pt,color=ffqqqq] (8.,0.)-- (8.,4.);
\draw [line width=1.2pt,color=ffqqqq] (9.,0.)-- (9.,2.);
\draw [line width=1.2pt,color=ffqqqq] (10.,0.)-- (10.,1.);
\draw (10.08,-0.4975) node[anchor=north west] {note sur 10};
\draw (0.12,6.5425) node[anchor=north west] {Effectif};
\end{tikzpicture}
\end{center}

\begin{enumerate}
\item Quelle est la note plus plus souvent obtenue ? Cette valeur s'appelle \textbf{le mode}.
\item Combien d'élèves composent cette classe ?
\end{enumerate}

\end{AD}

\begin{AD}

Évolution de la population chinoise

\begin{center}
\begin{tabular}{|c|c|c|}
\hline 
\rowcolor{bleu1!25!white} & 1950 & 2009  \\ 
\hline 
\cellcolor{bleu1!25!white}  Nombre d'habitants &  \np{554760} millions & 1,3 milliards  \\ 
\hline 
\cellcolor{bleu1!25!white} Espérance de vie & 39 & 73 \\ 
\hline 
\cellcolor{bleu1!25!white}IDH ( indicateur de développement humain) & 0,55 (1980) & 0,772 \\ 
\hline 
\cellcolor{bleu1!25!white}Part de personnes sachant lire et écrire (en \%) & H : 79\% et F :54,4\% &  H :94,1\% et F :82,1\%  \\ 
\hline 
\cellcolor{bleu1!25!white}Mortalité infantile (pour 1000 naissances) & 195 & Garçons(17), Filles(24) \\ 
\hline 
\cellcolor{bleu1!25!white}Fécondité & 6,6 & 1,7 \\ 
\hline 
\end{tabular} 
\end{center}

\begin{enumerate}
\item Quelle est l'évolution du nombre d'habitants en Chine entre  1950 et 2009 ?
\item Combien de femmes savent lire en 1950 ? en 2009 ? 
\item Quelle est l'évolution de l'espérance de vie en Chine entre 1950 et 2009 ?
\item Dans un village de \np{1280} habitants comprenant 45\% d'hommes, combien d'habitants savent lire ?
\end{enumerate}


\end{AD}


\begin{AD}

\begin{minipage}{0.5\linewidth}
Le graphique ci-dessous montre les résultats à un contrôle de sciences obtenus par deux groupes d'élèves, désignés par « Groupe A » et « Groupe B ». 
La note moyenne pour le Groupe A est de 62,0 et de 64,5 pour le Groupe B. Les élèves réussissent ce contrôle lorsque leur note est de 50 points ou davantage. Sur la base de ce graphique, le professeur conclut que le Groupe B a mieux réussi ce contrôle que le Groupe A. Les élèves du Groupe A ne sont pas d'accord avec le professeur. Ils essaient de le convaincre que le Groupe B n'a pas nécessairement mieux réussi. \hfill{{\scriptsize D'après PISA 2009}}
\end{minipage}
\begin{minipage}{0.5\linewidth}
\begin{center}
\includegraphics[scale=0.6]{Stat-13.eps} 
\end{center}
\end{minipage}
En vous servant du graphique, donnez un argument mathématique que les élèves du Groupe A pourraient utiliser. 


\end{AD}

\begin{AD}

On a représenté sur le diagramme suivant les vols du mois de février d’une compagnie aérienne.  
\begin{center}
\definecolor{xdxdff}{rgb}{0.49019607843137253,0.49019607843137253,1.}
\definecolor{ffdxqq}{rgb}{1.,0.8431372549019608,0.}
\definecolor{ffffqq}{rgb}{1.,1.,0.}
\definecolor{ffxfqq}{rgb}{1.,0.4980392156862745,0.}
\definecolor{ffqqqq}{rgb}{1.,0.,0.}
\definecolor{xfqqff}{rgb}{0.4980392156862745,0.,1.}
\definecolor{qqqqff}{rgb}{0.,0.,1.}
\begin{tikzpicture}[line cap=round,line join=round,>=triangle 45,x=1.0cm,y=1.0cm]
\clip(-0.096,-0.204) rectangle (18.304,8.336);
\draw[color=xfqqff,fill=xfqqff,fill opacity=1.0] (5.,3.5757359312880714) -- (5.424264068711929,3.5757359312880714) -- (5.424264068711929,4.) -- (5.,4.) -- cycle; 
\draw [shift={(5.,4.)},color=ffqqqq,fill=ffqqqq,fill opacity=1.0] (0,0) -- (0.:0.6) arc (0.:30.488940499830935:0.6) -- cycle;
\draw [shift={(5.,4.)},color=ffxfqq,fill=ffxfqq,fill opacity=0.95] (0,0) -- (30.488940499830935:0.6) arc (30.488940499830935:59.44286434307001:0.6) -- cycle;
\draw [shift={(5.,4.)},fill=black,fill opacity=1.0] (0,0) -- (59.44286434307001:0.6) arc (59.44286434307001:90.:0.6) -- cycle;
\draw [shift={(5.,4.)},color=qqqqff,fill=qqqqff,fill opacity=0.44]  plot[domain=1.5707963267948966:4.71238898038469,variable=\t]({1.*4.*cos(\t r)+0.*4.*sin(\t r)},{0.*4.*cos(\t r)+1.*4.*sin(\t r)});
\draw [shift={(5.,4.)},color=xfqqff,fill=xfqqff,fill opacity=0.66]  (0,0) --  plot[domain=-1.5707963267948966:0.,variable=\t]({1.*4.*cos(\t r)+0.*4.*sin(\t r)},{0.*4.*cos(\t r)+1.*4.*sin(\t r)}) -- cycle ;
\draw [shift={(5.,4.)},color=ffqqqq,fill=ffqqqq,fill opacity=0.6]  (0,0) --  plot[domain=0.:0.5321323971666955,variable=\t]({1.*4.*cos(\t r)+0.*4.*sin(\t r)},{0.*4.*cos(\t r)+1.*4.*sin(\t r)}) -- cycle ;
\draw [shift={(5.,4.)},color=ffxfqq,fill=ffxfqq,fill opacity=0.4]  (0,0) --  plot[domain=0.5321323971666955:1.037473699602908,variable=\t]({1.*3.973415659102379*cos(\t r)+0.*3.973415659102379*sin(\t r)},{0.*3.973415659102379*cos(\t r)+1.*3.973415659102379*sin(\t r)}) -- cycle ;
\draw [shift={(5.,4.)},color=ffffqq,fill=ffffqq,fill opacity=0.5]  (0,0) --  plot[domain=1.037473699602908:1.5707963267948966,variable=\t]({1.*4.*cos(\t r)+0.*4.*sin(\t r)},{0.*4.*cos(\t r)+1.*4.*sin(\t r)}) -- cycle ;
\draw (10.644,6.776) node[anchor=north west] {Vols vers l'Asie};
\draw (10.664,6.176) node[anchor=north west] {Vols vers l'Afrique};
\draw (10.664,5.616) node[anchor=north west] {Vols vers l'Amérique};
\draw (10.704,5.056) node[anchor=north west] {Vols vers l'Europe};
\draw (10.724,4.436) node[anchor=north west] {Vols vers la France};
\begin{scriptsize}
\draw [fill=ffffqq] (10.304,6.636) circle (2.5pt);
\draw [fill=ffdxqq] (10.304,6.036) circle (2.5pt);
\draw [fill=ffqqqq] (10.324,5.436) circle (2.5pt);
\draw [fill=xfqqff] (10.344,4.836) circle (2.5pt);
\draw [fill=xdxdff] (10.364,4.236) circle (2.5pt);
\end{scriptsize}
\end{tikzpicture}
\end{center}

\begin{minipage}{8cm}
Dans chaque cas, quelle fréquence représentent les vols vers la France, l’Europe et l’Asie. 
\end{minipage}
\begin{minipage}{1cm}
$~~$
\end{minipage}
\begin{minipage}{8cm}
Au mois de février, cette compagnie a affrété 576 vols. Calculer le nombre de vols vers la France, l’Europe et l’Asie. 
\end{minipage}

\end{AD}



\begin{autoeval}
\begin{tabular}{p{12cm}p{0.5cm}p{0.5cm}p{0.5cm}p{1cm}}
\textbf{Compétences visées} &  M I & MF & MF  & TBM \vcomp \\ 
Lire des données sur différents supports & $\square$ & $\square$  & $\square$ & $\square$ \vcomp \\ 
\end{tabular}
{\footnotesize MI : maitrise insuffisante ; MF = Maitrise fragile ; MS = Maitrise satisfaisante ; TBM = Très bonne maitrise}
 
\end{autoeval}}% Calcul de moyenne, écart type 
%% \impress{\impressionEleve}{\input{CHAPITRES/Stat-F1_cor}}% Calcul de moyenne, de médiane  
%  \impress{\impressionEleve}{\input{CHAPITRES/Stat-F1_bis}}% Calcul de médiane, de quartiles 
%%\impress{\impressionEleve}{\begin{titre}[Les statistiques]

{\color{bleu3}{\LARGE Calcul de moyenne} \hfill{Niveau 2}}
\end{titre}



\begin{CpsCol}
\textbf{Interpréter, représenter et traiter des données}
\begin{description}
\item[$\square$] Calculer une moyenne
\item[$\square$] Interpréter une caractéristique de position
\end{description}
\end{CpsCol}

\begin{Rec}

Si 2/5 des habitants d'un pays ont au moins 50 ans et 1/3 des habitants de ce pays ont moins de 20 ans, est-il possible que l'âge moyen de la population soit de 40 ans ?

\hfill{D’après le cadre d’évaluation «PISA 2006»:}
 
\hfill{items libérés Mathématiques OCDE/DEPP – janvier 2011}


\end{Rec}

\begin{DefT}{moyenne}
La \textbf{moyenne} d'une série statistique est le quotient de la somme de TOUTES les valeurs par l'effectif total de cette série.
\end{DefT}

\begin{Ex}
Voici les notes obtenues par Aurélie en Mathématiques au cours de l'année. 
\begin{itemize}
\item	1er trimestre :		10 – 9 – 11 – 12 – 11,5 – 14 – 12
\item	2ème trimestre :	9,5 – 11 – 12,5 – 8 – 13 – 18
\item	3ème trimestre :	8 – 9 – 14 – 12 – 10 – 13 – 11,5
\end{itemize}
Calculons sa moyenne annuelle :
		$m1 =\frac{10+9+11+\ldots{}+11,5}{20}=\frac{229}{20}=11,45$ 
\end{Ex}
		

\begin{AD}

\begin{multicols}{2}
Voici ci-contre la répartition par âge des membres d'un club d'échec.
Quel est l'age moyen des adhérents ?

\includegraphics[scale=0.35]{Stat-cours1.eps} 
\end{multicols}	
\end{AD}


 


\begin{tabular}{|c|c|c|c|c|c|}
\hline 
heures & $0\leq h < 4$ & $4 \leq h < 8$ & $8 \leq h < 12$ & $12 \leq h < 16$ & $16 \leq h$ \\ 
\hline 
températures & 12 & 18 & 22 & 30 & 26 \\ 
\hline 
\end{tabular} 



\subsection*{Moyenne de moyennes}
\begin{Ex}
		
Avec l'exemple précédent :
\begin{itemize}
\item 		1er trimestre :	  la moyenne d'Aurélie est $11,36$ ;
\item		2ème trimestre : la moyenne d'Aurélie est $12$ ;
\item		3ème trimestre : la moyenne d'Aurélie est $11,07$.
\end{itemize}
Calculons la moyenne de ses moyennes trimestrielles :
		$m2 =\frac{11,36+12+11,07}{3}=\frac{34,43}{3}$ donc  		$m2\approx 11,48$
\end{Ex}
				
		
		
\begin{Rq}	
Cette moyenne est rarement égale à la moyenne de la série.

La moyenne est toujours comprise entre la plus petite valeur et la plus grande valeur de la série statistique.
\end{Rq}



\begin{AD}

Voici les notes d'un élève de 4\ieme\ en mathématiques.
\medskip
\begin{description}
 \item [1\ier\ trimestre] 15\kern5mm 10\kern5mm 8\kern5mm 13\kern5mm 10
\item[2\ieme\ trimestre] 13\kern5mm 9\kern5mm 7\kern5mm 14\kern5mm 13\kern5mm 16
\item[3\ieme\ trimestre] 12\kern5mm 15\kern5mm 17\kern5mm 14\kern5mm 12
\end{description}
\begin{enumerate}
  \item Calcule sa moyenne pour chacun des trois trimestres.
  \item Calcule la moyenne des moyennes des trois trimestres.
  \item Calcule la moyenne de l'ensemble de ses notes sur son année de 4\ieme.\\Que remarque-t-on ?
\end{enumerate}
\end{AD}

\begin{AD}

On a trié 4 paquets de 40 M\&M's et on a obtenu les résultats suivants : 


\begin{center}
 \includegraphics[scale=0.5]{mms1.jpg}
 \end{center} 


\begin{enumerate}
\item Calcule le nombre de bonbons bleus dans ces 4 paquets.
\item Combien en moyenne a-t-on de M\&M's bleus par paquets ? 
\end{enumerate}


\end{AD}

\begin{Exo}

\begin{minipage}[top]{10cm}
L'entreprise est à la recherche de qualifications de plus en plus élevées pour faire face au développement de technologies en constante évolution et pour une bonne compréhension des consignes de travail. Lors de sa scolarité, un jeune doit développer de l'intérêt et de la curiosité, si utiles pour réussir ensuite sa vie professionnelle. Face au nombre, en baisse mais encore inquiétant, de sorties du système scolaire sans qualification, il paraît intéressant d'étudier ce phénomène du point de vue européen à la lumière des mathématiques. En France, 13\% des jeunes de 18 à 24 ans qui ne poursuivent pas d'études ni de formation n'ont ni CAP, ni BEP, ni bac et sont sortants précoces.
\end{minipage}
\begin{minipage}{6cm}
\includegraphics[scale=0.5]{stat12.jpg} 
\end{minipage}

\begin{enumerate}
\item Calculer la moyenne des sorties précoces en Europe à l'aide des données du tableau. Que remarquez-vous ? Justifier.
\item Tracer une représentation graphique de ce tableau sur tableur ou sur une feuille.
\item Compléter le tableau suivant et construire l'histogramme de cette série.

\begin{tabular}{|c|c|c|c|c|c|c|}
\hline 
Sorties précoces en 2007 & [0;5[ & [5;10[ & [10;15[ & [15;20[ & [20;25[ & [25;30] \\ 
\hline 
Nombre de pays européens &  &  &  &  &  &  \\ 
\hline 
\end{tabular} 

\item Calculer la moyenne des sorties précoces en Europe à l'aide de la question 3. Comparer le résultat avec la question 1.  
\end{enumerate}




  
\end{Exo}



\begin{DefT}{moyenne pondérée}
La \textbf{moyenne pondérée} d’une série statistique est le quotient de la somme
de TOUTES les valeurs multipliées chacune par leur coefficients
par la somme de ces coefficients.
\end{DefT}

\begin{AD}

Pierre, Jean et Alain ont passé un examen comportant quatre
disciplines. Pour être reçu, il faut atteindre 10 de moyenne.
\begin{enumerate}
\item Calculer la moyenne, sans coefficient, des trois candidats.
\begin{center}
  \begin{tabular}{|c|c|c|c|c|}
\cline{2-5}
\multicolumn{1}{c|}{}&Français&Mathématiques&Anglais&Technologie\\
\hline
Pierre&15&9&11&7\\
\hline
Jean&10&11&12&9\\
\hline
Alain&7&14&13&8\\
\hline
  \end{tabular}
\end{center}
\item Pour cet examen, le français, les mathématiques, l'anglais et la
technologie ont respectivement pour coefficient 6; 4; 2 et 5.\par
Calculer la moyenne pondérée de chaque candidat et dire s'ils sont reçus
ou non.
\end{enumerate}
\end{AD}

\begin{AD}

Lors d'un test d'endurance, plusieurs élèves ont eu 12 minutes pour
parcourir la plus grande distance possible. Voici les résultats des
élèves :
2230 -- 2450 -- 1890 -- 1850 -- 2650 -- 2630 -- 2110 -- 2250 -- 2180 --
1980 -- 2000 -- 2850 -- 1950 -- 2920 -- 1975 -- 1910 -- 1860 -- 1930 --
2010 -- 2400 -- 2650 -- 2320 -- 2190 -- 2730 -- 2120 -- 2380 -- 2220.

\begin{enumerate}
\item
Calcule la moyenne des distances parcourues.

\item
On veut calculer la moyenne approximative des distances parcourues.
Pour cela, dénombrer le nombre d'élèves dans chacun des intervalles
suivants :\\
\hskip 0pt plus 500pt minus 500pt\begin{tabular}{|*{6}{c|}}
\hline
$[1800 ; 2000 [$ & $[2000 ; 2200 [$ &
$[2200 ; 2400 [$ & $[2400 ; 2600 [$ &
$[2600 ; 2800 [$ & $[2800 ; 3000 [$ \\
\hline
&&
 &&& \\
\hline
\end{tabular}\hskip 0pt plus 500 pt minus 500 pt\strut
\item
Calculer une moyenne approchée en remplaçant la distance de chaque élève
	par le début de chaque intervalle (1800 pour le premier, 2000 pour
	le deuxième,\dots) en pensant à remplacer chaque série de nombres
	identiques par une multiplication.
	
\item Reprendre la question précédente en 
	remplaçant la distance de chaque élève
	par le milieu de chaque intervalle (1900 pour le premier, 2100 pour
	le deuxième,\dots)
	
\item Reprendre la question précédente en 
	remplaçant la distance de chaque élève
	par la fin de chaque intervalle (2000 pour le premier, 2200 pour
	le deuxième,\dots)
	
\end{enumerate}
\end{AD}

\begin{AD}

Le professeur d'EPS a relevé les performances ci-dessous :
\begin{center}
  \begin{tabularx}{0.85\linewidth}{|l|X|X|l|X|X|}
    \hline
\multicolumn{1}{|c|}{Noms}&Temps au 80~m (s)&Hauteur du saut (cm)&\multicolumn{1}{c|}{Noms}&Temps au 80~m (s)&Hauteur du saut (cm)\\
\hline
Charles&\multicolumn{1}{c|}{13}&\multicolumn{1}{c|}{110}&Gérald&\multicolumn{1}{c|}{13,6}&\multicolumn{1}{c|}{115}\\
Bruno&\multicolumn{1}{c|}{12,5}&\multicolumn{1}{c|}{120}&Nicolas&\multicolumn{1}{c|}{13,9}&\multicolumn{1}{c|}{110}\\
Sylvie&\multicolumn{1}{c|}{15}&\multicolumn{1}{c|}{100}&Florence&\multicolumn{1}{c|}{14,7}&\multicolumn{1}{c|}{110}\\
Brice&\multicolumn{1}{c|}{13,2}&\multicolumn{1}{c|}{125}&Daniel&\multicolumn{1}{c|}{13}&\multicolumn{1}{c|}{125}\\
Carine&\multicolumn{1}{c|}{15,4}&\multicolumn{1}{c|}{100}&Viviane&\multicolumn{1}{c|}{16}&\multicolumn{1}{c|}{95}\\
Léon&\multicolumn{1}{c|}{12}&\multicolumn{1}{c|}{135}&Barbara&\multicolumn{1}{c|}{15,1}&\multicolumn{1}{c|}{105}\\
Christian&\multicolumn{1}{c|}{12,6}&\multicolumn{1}{c|}{130}&Jeanne&\multicolumn{1}{c|}{14,9}&\multicolumn{1}{c|}{110}\\
\'Elisabeth&\multicolumn{1}{c|}{15,4}&\multicolumn{1}{c|}{95}&Lucie&\multicolumn{1}{c|}{15,4}&\multicolumn{1}{c|}{100}\\
Aude&\multicolumn{1}{c|}{14,9}&\multicolumn{1}{c|}{110}&Odile&\multicolumn{1}{c|}{14,2}&\multicolumn{1}{c|}{85}\\
Cécile&\multicolumn{1}{c|}{16,2}&\multicolumn{1}{c|}{85}&Alice&\multicolumn{1}{c|}{15,6}&\multicolumn{1}{c|}{105}\\
Clément&\multicolumn{1}{c|}{12}&\multicolumn{1}{c|}{140}&Gaël&\multicolumn{1}{c|}{13,5}&\multicolumn{1}{c|}{125}\\
Cathy&\multicolumn{1}{c|}{15,8}&\multicolumn{1}{c|}{100}&Pierre&\multicolumn{1}{c|}{12,3}&\multicolumn{1}{c|}{135}\\
Delphine&\multicolumn{1}{c|}{15}&\multicolumn{1}{c|}{105}&Armand&\multicolumn{1}{c|}{12,8}&\multicolumn{1}{c|}{130}\\
Jacques&\multicolumn{1}{c|}{13,1}&\multicolumn{1}{c|}{135}&Jean&\multicolumn{1}{c|}{13,1}&\multicolumn{1}{c|}{115}\\
André&\multicolumn{1}{c|}{13,9}&\multicolumn{1}{c|}{120}&David&\multicolumn{1}{c|}{12,5}&\multicolumn{1}{c|}{135}\\
\hline
  \end{tabularx}
\end{center}
\begin{enumerate}
  \item
    \begin{enumerate}
    \item Quel est le temps moyen mis pour effectuer le 80~m ?
    \item Combien d'élèves ont un temps supérieur au temps moyen ?
    \end{enumerate}
  \item
    \begin{enumerate}
    \item Quelle est la hauteur moyenne d'un saut ?
    \item Combien d'élèves ont une hauteur inférieure à la hauteur
      moyenne ?
    \end{enumerate}
\end{enumerate}

\end{AD}

\subsection*{Utilisation d'un tableur pour le calcul de la moyenne}

Le tableur est un outil performant pour calculer la moyenne.


\begin{Ex}
Sur la feuille de classeur ci-dessous,
on a calculé la longueur moyenne des prénoms des 684 élèves du collège.\\
		Les longueurs des prénoms sont rangées de la cellule C4 à la cellule C687.\\
		Ces cellules se suivent, on dit qu'elles sont contiguës.\\
		Dans la formule, on utilise alors le symbole \og  deux – points \fg{}.\\
		Dans la zone de saisie des fonctions, on a tapé : =MOYENNE(C4:C687).\\
		
\includegraphics[scale=0.5]{Stat-cours2.jpg} 		
\end{Ex}

\begin{AD}

Lors de la première édition de la Course aux Nombres, les 24 élèves de la classe de Cinquième 2 en Colombie ont obtenu les résultats suivants. Les notes sont évaluées sur 30 points.


\begin{enumerate}
\item Reproduire la feuille de calcul comme indiqué ci dessous.

\begin{tabular}{|c|c|c|c|c|c|c|c|c|c|}
\hline 
\rowcolor{gray} & A & B & C & D & E & F & G & H & I \\ 
\hline 
\cellcolor{gray} 1 & 11 & 16 & 22 & 20 & 26 &  &  & Nombres d'élèves total &  \\ 
\hline 
\cellcolor{gray}2& 26 & 18 & 26 & 19& 16 &  &  & Nombre d'élèves dont la nombre est égale à 22 &  \\ 
\hline 
\cellcolor{gray}3 & 17 &27 & 18 & 16 & 22 &  & & Fréquence d'élèves dont la note est égale à 22 &  \\ 
\hline 
\cellcolor{gray}4 & 16 & 19 & 11 & 21 & 17 & &  & Nombre d'élèves dont la note est égale à 19 &  \\ 
\hline 
\cellcolor{gray}5 & 22 & 15 & 22 & 23 &  &  &  & Fréquence d'élèves dont la note est égale à 19 &  \\ 
\hline 
\end{tabular} 

\item  
\begin{enumerate}
\item  Déterminer dans la cellule I1 le nombre d'élèves participants à ce jeu. 
Rappel : Pour déterminer le nombre de cases remplies d'un tableau, on utilise la syntaxe : =NB(A1:E5). 
\item  Déterminer dans la cellule I2 le nombre d'élèves dont la note est égale à 22 ? 
Rappel : Pour déterminer le nombre de cases remplies avec la valeur 22, on utilise la syntaxe : =NB.SI(A1:E5;22). 
\item  Calcule dans la cellule I3 la fréquence des élèves ayant obtenu 22.
\end{enumerate}
\item  
\begin{enumerate}
\item Détermine la moyenne de cette classe. On pourra utiliser la syntaxe "=MOYENNE(A1:E5)".
\item Explique par une phrase le calcul du tableur pour donner la moyenne.
\item 
\begin{enumerate}
\item Complète le tableau ci-dessous.

\begin{tabular}{|c|c|c|c|c|c|c|}
\hline 
Notes & $[0;5[$ &  $[5;10[$  &  $[10;15[$  &  $[15;20[$  &  $[20;25[$  &  $[25;30]$  \\ 
\hline 
Fréquence &  &  &  &  &  &  \\ 
\hline 
\end{tabular} 

\item Calcule la moyenne avec ce regroupement.	
\end{enumerate}
\item Création d'un diagramme avec un tableur
\begin{enumerate}
\item  Sélectionne la plage de données A1:E5 puis clique sur l'icône graphique
\item  Quel est le problème de la plage de données A1:E5 ?
\end{enumerate}

\end{enumerate}
\end{enumerate}




\end{AD}



\begin{AD}

\begin{enumerate}
\item Explique le fonctionnement de cet algorithme :

\begin{tabular}{|p{10cm}|}
\hline 
\#1. Somme = 0 \\ 
\#2. Moyenne = 0\\ 
\#3. Répéter 4 fois\\ 
\#4. \hspace{1cm} Demander\_une\_valeur x et attendre\\ 
\#5. \hspace{1cm} Lire réponse\\ 
\#6. \hspace{1cm} Somme prend\_la\_valeur Somme +  réponse\\ 
\#7. Moyenne = Somme / 4\\ 
\#8. Afficher Moyenne\\ 
\hline 
\end{tabular} 

\item A l'aide du logiciel Scratch, traduis cet algorithme en programme. 

\end{enumerate}

\end{AD}


\begin{autotest}
\begin{enumerate}
\item On donne les nombres suivants : $$4~~ - ~~ 16 ~~  - ~~ 8 ~~ - ~~ 13 ~~ - ~~ 15 $$
La moyenne est $$a. 11~~ ~~ ~~ b. 11,5 ~~  ~~ ~~ c. 11,2 ~~ ~~ ~~ d. 12$$

\item Pour calculer la moyenne des cellules A5 à B5, avec un tableur, on tape la formule
$$a. MOYENNE(A5:B5)~~ ~~ ~~ b. =MOYENNE(A5:B5) ~~  ~~ ~~ c. MOYENNE(A5;B5) ~~ ~~ ~~ d. =MOYENNE(A5;B5)$$

\item Voici le diagramme à bâtons des notes d'un devoir. 

\definecolor{qqqqff}{rgb}{0.,0.,1.}
\definecolor{cqcqcq}{rgb}{0.7529411764705882,0.7529411764705882,0.7529411764705882}
\begin{tikzpicture}[line cap=round,line join=round,>=triangle 45,x=1.0cm,y=1.0cm]
\draw [color=cqcqcq,, xstep=1.0cm,ystep=1.0cm] (-0.72,-1.1) grid (10.62,7.46);
\draw[->,color=black] (-0.72,0.) -- (10.62,0.);
\foreach \x in {,1.,2.,3.,4.,5.,6.,7.,8.,9.,10.}
\draw[shift={(\x,0)},color=black] (0pt,2pt) -- (0pt,-2pt) node[below] {\footnotesize $\x$};
\draw[->,color=black] (0.,-1.1) -- (0.,7.46);
\foreach \y in {-1.,1.,2.,3.,4.,5.,6.,7.}
\draw[shift={(0,\y)},color=black] (2pt,0pt) -- (-2pt,0pt) node[left] {\footnotesize $\y$};
\draw[color=black] (0pt,-10pt) node[right] {\footnotesize $0$};
\clip(-0.72,-1.1) rectangle (10.62,7.46);
\draw [color=qqqqff] (1.,0.)-- (1.,1.);
\draw [color=qqqqff] (3.,0.)-- (3.,2.);
\draw [color=qqqqff] (4.,0.)-- (4.,3.);
\draw [color=qqqqff] (5.,4.)-- (5.02,-0.14);
\draw [color=qqqqff] (6.,0.)-- (6.,6.);
\draw [color=qqqqff] (7.,6.)-- (7.,0.);
\draw [color=qqqqff] (8.,4.)-- (8.,0.);
\draw [color=qqqqff] (10.,1.)-- (10.,0.);
\draw (9.44,-0.5) node[anchor=north west] {notes};
\draw (0.16,6.42) node[anchor=north west] {Effectifs};
\end{tikzpicture}

La moyenne est $$a. \approx 5,9 ~~ ~~ ~~ b. \approx 6,5 ~~  ~~ ~~ c. \approx 1,7 ~~ ~~ ~~ d. \approx 5,5$$

\item Pierre n'a retrouvé que 7 contrôles où il a obtenu chaque fois 8 sur 10. Il a perdu une copie mais il sait que sa moyenne est de 7,75. Quelle est la note de la copie perdue ?

\item Grégory espère obtenir au Bac 17/20 en Maths, 14/20 en Lettres et 17/20 en Langues. Le coefficient de l'épreuve des maths est 7, celui de l'épreuve des lettres est 3 et celui de l'épreuve des langues est 2. Peut il espérer obtenir la mention "Très bien" ?

\item Le tableau donne le nombre de voitures qui passent dans un virage dangereux où la vitesse est limitée à 60 km/h. 

\begin{tabular}{|c|c|c|c|c|}
\hline 
Vitesse & [30;40[ & [40;50[ & [50;60[ & [60;70[ \\ 
\hline 
Effectif & 15 & 60 & 80 & 25 \\ 
\hline 
\end{tabular} 

Peut on dire les conducteurs respectent la limitation de vitesse ?

\end{enumerate}
\end{autotest}



\begin{autoeval}
\begin{tabular}{p{12cm}p{0.5cm}p{0.5cm}p{0.5cm}p{1cm}}
\textbf{Compétences visées} &  M I & MF & MF  & TBM \vcomp \\ 
Calculer une moyenne & $\square$ & $\square$  & $\square$ & $\square$ \vcomp \\ 
Interpréter une caractéristique de position & $\square$ & $\square$ & $\square$ & $\square$ \vcomp \\ 
\end{tabular}
{\footnotesize MI : maitrise insuffisante ; MF = Maitrise fragile ; MS = Maitrise satisfaisante ; TBM = Très bonne maitrise}
 
\end{autoeval}}% Représentation de statistiques en histogramme
%%\impress{\impressionEleve}{\begin{titreDTL}[Statistiques descriptives]

\Titre{S'auto évaluer}{1}
\end{titreDTL}

\begin{minipage}{0.48\linewidth}
\EPC{1}{stat-24}{Chercher. Calculer.}
\end{minipage}
\hfill
\begin{minipage}{0.48\linewidth}
\EPC{1}{stat-25}{Chercher. Calculer.}
\end{minipage}

\EPC{1}{stat-26}{Représenter. Calculer.}

\EPC{1}{stat-27}{Chercher. Communiquer.}

}% Auto évaluation
% \impress{\impressionEleve}{\input{CHAPITRES/Stat-calculatrice}}% 
% \impress{\impressionEleve}{\input{CHAPITRES/Stat-tableur}}% 
%%
% \chapter{Probabilités}
%%
%%  \impress{\impressionEleve}{\begin{titre}[Probabilités]

\Titre{Mettre en ouvre une simulation}{1}
\end{titre}


\begin{CpsCol}
\begin{description}
\item[$\square$] Simuler une situation avec Python
\end{description}
\end{CpsCol}



Python requiert des bibliothèques pour chaque utilisation particulière ( tkinter, math, ...). Pour travailler sur des probabilités et simuler avec la fonction random, il est nécessaire d'appeler la fonction random en début de programme avec  $$ \text{\color{orange}import \color{black} random}$$

A l'inverse de la majorité des langages de programmation, Python a une syntaxe très simple. Il n'y a pas de point virgule ou accolade pour délimiter les blocs. Python utilise le système d'indentation. Il est donc impératif de bien placer les blocs par rapport à leur indentation.


\subsubsection*{Simulation du lancé d'un dé}

\color{orange} import \color{black} random

x = random.randint(1,6)

\color{purple} print \color{black}(x)



\subsubsection*{Afficher une simulation de 50 lancers d'un dé à 6 faces}


\color{orange} import \color{black} random

\color{orange} for \color{black} i \color{orange} in \color{purple}range\color{black}(50):

\hspace{0.4cm}     x = random.randint(1,6)
 
\hspace{0.4cm}   \color{purple} print \color{black}(x)



\subsubsection*{Applications}

La probabilité d'obtenir une face lorsque la pièce n'est pas non truquée est égale à $\frac{1}{2}$. Ce résultat semble évident. On peut le simuler de la manière suivante.


\subsection*{$n$ Lancers d'une pièce. L'approche fréquentiste de Bayès}

On souhaite connaitre la fréquence de sortie de la face "Pile" lors de $n$ lancers d'une pièce. 

On note 0 la face "Face" et 1 la face "Pile".

Écrire un programme pour une valeur variable $n$.

\begin{enumerate}
\item Tester le programme pour $n=10$
\item Tester le programme pour $n=1000$
\item Tester le programme pour $n=1000000$
\item Conclure
\end{enumerate}

%\vspace{0.4cm}
%On pourra tester ce programme.
%
%\vspace{0.4cm}
%
%
%
%\color{orange} import \color{black} random
%
%n = \color{purple}int\color{black}(\color{purple}input\color{black}(\color{black}"Entrer le nombre de lancer "\color{black}))
%
%k = 0
%
%\color{orange} for \color{black} i \color{orange} in \color{purple}range\color{black}(n):
%
%\hspace{0.4cm}     x = random.randint(0,1)
%
%\hspace{0.4cm}   \color{orange} if \color{black} x = 1 :
%
%\hspace{0.8cm}   k=k+1
%
%f=k/n
%
%\color{purple} print \color{black}(f)


}% Mettre en ouvre une simulation
%%   \impress{\impressionEleve}{\begin{titre}[Probabilités]

\Titre{Modélisation de situations}{4}
\end{titre}


\begin{CpsCol}
\begin{description}
\item[$\square$] Construire un arbre pondéré, un diagramme, un tableau
\end{description}
\end{CpsCol}


\mini{
\AD{1}{Prob-5}

\AD{1}{Prob-6}

\AD{1}{Prob-7}
}{
\AD{1}{Prob-8}

\PO{1}{Prob-19}

\PO{1}{Prob-10}

\PO{1}{Prob-23}
}


}% Modéliser une situation
%  \impress{\impressionEleve}{\begin{titre}[Probabilités]

\Titre{Probabilité, loi de probabilité}{4}
\end{titre}


\begin{CpsCol}
\begin{description}
\item[$\square$] Utiliser des modèles théoriques de référence (dé, pièce équilibrée, tirage au sort avec
équiprobabilité dans une population) en comprenant que les probabilités sont définies
a priori.
\item[$\square$] Construire un modèle à partir de fréquences observées, en distinguant nettement
modèle et réalité.
\item[$\square$] Calculer des probabilités dans des cas simples : expérience aléatoire à deux ou trois
épreuves.
\end{description}
\end{CpsCol}

\Rec{1}{Prob-20}

\begin{DefT}{Épreuve ou expérience aléatoire}\index{Épreuve}\index{Expérience aléatoire}
On appelle \textbf{épreuve} ou \textbf{expérience aléatoire}, une situation soumise au hasard.

On appelle \textbf{issue} le résultat d'une épreuve, notée $\omega$. 

L'ensemble des issues est l'\textbf{univers}, noté $\Omega$.\index{Univers} \index{Issue}
\end{DefT}

\begin{Ex}
On lance un dé cubique dont les faces sont numérotées de 1 à 6 et on observe le numéro de la face obtenue. Cette expérience a 6 issues possibles dont une issue est le 6. L'univers est $\Omega = \lbrace 1;2;3;4;5;6 \rbrace$.
\end{Ex}



\mini{
\EPC{1}{Prob-6}{Modéliser. Représenter.}
}{
\EPC{0}{Prob-5}{Modéliser. Représenter.}
}


\begin{DefT}{Loi de probabilités}\index{Loi de probabilités}
Définir la loi de probabilité pour une expérience aléatoire dont l'univers est $\Omega = \lbrace x_1; x_2; ... ; x_n \rbrace$ consiste à déterminer pour chacune des issues $x_i$ un nombre $p_i$, $i$ variant de 1 à $n$, appelée \textbf{probabilité}, tel que $p_1+p_2+...p_n=1$
\end{DefT}

\begin{Ex}
On lance un dé cubique équilibré dont les faces sont numérotées de 1 à 6 et on observe le numéro de la face obtenue.   Donc $p_1+p_2+p_3+p_4+p_5+p_6=1$. Comme le dé est équilibré, $p_i=\frac{1}{6}$.
\end{Ex}


\begin{Pp}
En répétant un grand nombre de fois une expérience aléatoire, la fréquence de chaque issue se stabilise autour d'une valeur. Cette valeur est la probabilité de l'issue.
\end{Pp}



\mini{
\EPC{1}{Prob-31}{Communiquer.}
 
\EPC{1}{Prob-26}{Communiquer.}

\EPC{1}{Prob-28}{Modéliser.}

\EPC{1}{Prob-27}{Communiquer.}
}{
\EPC{1}{Prob-32}{Modéliser. }

\EPC{1}{Prob-8}{Chercher.}

\EPC{0}{Prob-18}{Chercher.}

\EPC{0}{Prob-29}{Modéliser.}
}





\EPCP{1}{Prob-25}{Modéliser. Représenter.}

%\EPCP{1}{Prob-0}{Modéliser. Représenter.}

 }% Equiprobabilité + simulation
%    \impress{\impressionEleve}{\begin{titre}[Probabilités]

\Titre{Notion d'événement}{6}
\end{titre}


\begin{CpsCol}
\begin{description}
\item[$\square$] Déterminer l'intersection de deux événements
\item[$\square$] Déterminer la réunion de deux événements
\item[$\square$] Utiliser la relation fondamentale
\end{description}
\end{CpsCol}

 
\begin{DefT}{Événement}\index{Événement}
On appelle \textbf{événement} un sous-ensemble de l'univers, c'est à dire un ensemble qui réunit toutes les issues favorables à une action déterminée. Un événement est une partie de l'univers.
\end{DefT}


\begin{Rq}
Une issue $x_i$ réalise un événement $A$ lorsque $x_i$ est un élément de $A$.
\end{Rq}

\begin{DefT}{Événement certain, Événement impossible} \index{Événements!Certain}\index{Événements!impossible}
Un \textbf{événement certain} est un événement qui est réalisé par toutes ses issues.

Un \textbf{événement impossible} est un événement qui n'est réalisé par aucune de ses issues.
\end{DefT}

\begin{Ex}
Lorsqu'on lance un dé cubique équilibré dont les faces sont numérotées de 1 à 6, l'événement obtenir un 7 est un événement impossible. 

L'événement "obtenir un nombre compris entre 1 et 6" est un événement certain. En effet, quelque soit la face obtenue, le nombre est compris entre 1 et 6.
\end{Ex}


\begin{DefT}{Événements incompatibles ou disjoints} \index{Événements!Incompatibles}\index{Événements!Disjoints see Événements!Incompatibles}
Deux événements A et B sont dits \textbf{incompatibles} ou \textbf{disjoints} lorsqu'ils ne peuvent se réaliser en même temps,
ou encore lorsque $A \cap B = \oslash$.
\end{DefT}

\begin{Ex}
$A = \left\lbrace 1 ; 2 \right\rbrace $ et $C = \left\lbrace  3 ; 5 \right\rbrace$ sont disjoints.
\end{Ex}


\begin{DefT}{Événements contraires} \index{Événements!Contraires}
Soit $\Omega$ un univers fini, $A$ et$ B$ deux événement inclus dans $\Omega$.
$A$ et $B$ sont deux événements \textbf{contraires} lorsque $A \cap B = \oslash$ et $A \cup B = \Omega$. On note $B=\overline{A}$ ou $A=\overline{B}$.
\end{DefT}

\begin{Ex}
Soit $\Omega =  \left\lbrace 1, 2, 3, 4, 5, 6 \right\rbrace $. $A = \left\lbrace1 ; 2\right\rbrace $ et $B = \left\lbrace 3 ; 4 ; 5 ; 6 \right\rbrace $ sont contraires et on note $A=\overline{B}$.
\end{Ex}


\begin{DefT}{Événement élémentaire} \index{Événement!élémentaire}
Un \textbf{événement élémentaire} est un événement qui ne contient qu'une seule issue.
\end{DefT}

\begin{Ex}
Lorsqu'on lance un dé cubique équilibré dont les faces sont numérotées de 1 à 6, l'événement obtenir un nombre pair plus petit que 3 est un événement élémentaire. L'événement ne contient que l'issue favorable 2. 
\end{Ex}

\mini{
\Fl{1}{Prob-30bis} 
}{
\EPC{1}{Prob-15}{Modéliser. Calculer.} 
}



\begin{DefT}{Probabilité d'un événement} \index{Probabilité d'un événement}
Soit $\Omega$ l'univers lié à une expérience aléatoire.
A chaque partie $B$ de $\Omega$, on fait correspondre un nombre compris entre 0 et 1, appelé \textbf{probabilité} de cet
événement $B$ tel que :
\begin{description}
\item[•] La somme des probabilités des événements élémentaires qui composent $\Omega$ est égale à 1.
\item[•] La probabilité de $B$ est la somme des probabilités des événements élémentaires qui composent $B$.
\item[•] La probabilité de l'événement impossible est 0. On note $p(B)$ la probabilité de l'événement $B$.
\end{description}
\end{DefT}


\begin{Pp}[Relation fondamentale]
Soit $\Omega$ l'univers lié à une expérience aléatoire et $A$ et $B$ deux événements de cet univers.
$$p(A \cup B) = p(A) + p(B) – p(A \cap B)$$
Si $A$ et $B$ sont incompatibles alors $p(A \cup B) = p(A) + p(B)$.
\end{Pp}

\begin{Pp}
Soit $A$ un événement de $\Omega$ et $B$ son événement contraire. $p (A)+ p (B)=1$.
\end{Pp}
 

\mini{
\EPC{1}{Prob-21}{Modéliser. Calculer.} 
}{
\EPC{0}{Prob-14}{Chercher.}
}
 
\mini{
\EPC{1}{Prob-7}{Modéliser. Calculer.} 


\EPC{1}{Prob-22}{Chercher.}

\EPC{0}{Prob-3}{Chercher.}

\EPC{1}{Prob-38}{Modéliser. }
}{
\EPC{1}{Prob-37}{Modéliser. }

\EPC{1}{Prob-41}{Modéliser. }

\EPC{1}{Prob-40}{Modéliser. }

 \EPC{1}{Prob-39}{Modéliser. }
}

\begin{Pp}[Équiprobabilité]
Lorsque tous les événements élémentaires ont la même probabilité, on dit qu'il y a équiprobabilité des
issues.
Dans ce cas, si l'univers $\Omega$ est composé de $n$ éventualités $\omega_i$ : $p(\omega_i)=\frac{1}{n}$

La probabilité d'un événement composé de $k$ éventualités est égale à $p(A)=\frac{k}{n}$
\end{Pp}






\mini{

\EPC{1}{Prob-33}{Modéliser. }

\EPC{1}{Prob-34}{Modéliser. }




\EPC{1}{Prob-35}{Modéliser. }

}{



\EPC{1}{Prob-36}{Modéliser. }

\EPC{1}{Prob-10}{Chercher.}
%\EPC{0}{Prob-30}{Modéliser. Calculer.} 
}


%\EPC{1}{Prob-13}{Chercher.}
 
 }% Equiprobabilité + simulation
%%########## \impress{\impressionEleve}{\begin{titre}[Probabilités - 1]

\Titre{Probabilité d'un événement}{2}
\end{titre}


\begin{CpsCol}
\begin{description}
\item[$\square$] Déterminer l'intersection de deux événements
\item[$\square$] Déterminer la réunion de deux événements
\item[$\square$] Utiliser la relation fondamentale
\item[$\square$] Déterminer la probabilité dans des situations équiprobables
\end{description}
\end{CpsCol}
 
 
\EPCN{Corrigé} 


On jette une pièce équilibrée 40 fois.

 

\begin{center}
\begin{lstlisting}
import random
n=int(input("Quel est le nombre de lancers ? "))
piece=["pile","face"]
proba_piece=[0]*2
for i in range (n) :
    alea = random.randint(0,1)
    proba_piece[alea]=proba_piece[alea]+1
    
for j in range (2) :
    print("On a alors p(",piece[j],") = ",proba_piece[j]/n)
\end{lstlisting}
\end{center}
 
 

 
\EPCN{Modéliser. Raisonner.} 


En informatique, un octet est une suite de 8 chiffres tous égaux à 1 ou 0. Par exemple : 10101100.

On note A : "les 2 premiers chiffres sont des 1" et B : "le dernier chiffre est 0".
\begin{enumerate}
\item Déterminer les octets vérifiant $A\cap B$.
\item  Déterminer les octets vérifiant $A\cup B$.
\end{enumerate}

\EPCN{Modéliser. Raisonner. Communiquer} 


Morgan est perchiste dans une station de sports d'hiver. Il observe qu'une personne sur 3 est un snowboarder et les autres des skieurs. Les trois quarts des snowboarders ont moins de 20 ans alors que la moitié des skieurs ont 20 ans ou plus.

On appelle :
 \begin{description}
  \item S l'événement "La personne est un snowboarder".
  \item K l'événement "La personne est un skieur".
  \item V: "la personne a moins de 20 ans".
  \end{description} 

\begin{enumerate}
\item Construire un arbre qui modélise cette situation.

\item Quelle est la probabilité que la prochaine personne qui se présente à la remontée mécanique soit un skieur de 20 ans ou plus ?
\item Quelle est la probabilité que la prochaine personne qui se présente à la remontée mécanique ait moins de 20 ans ? 
\end{enumerate}











\EPCN{Modéliser. Représenter. Communiquer}  


Dans sa penderie, Guillaume a deux pantalons, un noir et un blanc, deux vestes, une noire et une blanche, et trois chemises, deux blanches et une noire. Il prend au hasard un pantalon, une chemise et une veste.
\begin{enumerate}
\item A l'aide d'un arbre de dénombrement, déterminer le nombre de façon différentes de s'habiller.
\item
\begin{enumerate}
\item Calculer la probabilité de l'événement $A$ : "Il est habillé tout en noir".
\item Calculer la probabilité de l'événement $B$ : "Il est habillé tout en blanc".
\item $A$ et $B$ sont-ils contraires ?
Calculer la probabilité de l'événement $C$ : "Il une veste et un pantalon de couleurs différentes".
\end{enumerate}
\end{enumerate}
 
\EPCN{Modéliser. Communiquer}  



Une étude montre qu'un adolescent sur trois possède une télévision dans sa chambre et un sur cinq un ordinateur. 60\% des adolescents n'ont ni l'un ni l'autre dans leur chambre. 

\begin{enumerate}
\item Quel est l'événement contraire de "un adolescent n'a ni ordinateur, ni télévision dans sa chambre" ?
\item On appelle les événements :
\begin{description}
\item[•] O : "l'adolescent a un ordinateur dans sa chambre".
\item[•] T : "l'adolescent a un téléviseur dans sa chambre".
\end{description}
\item Déterminer à l'aide de phrases les événements : $O \cup T$,  $O \cap T$.
\item Déterminer la part d'adolescents ayant à la fois une télévision et un ordinateur dans leur chambre.
\end{enumerate}

 
 
 
\EPCN{Modéliser. Raisonner} 


Le chevalier de Méré (1607-1684) prétend qu'il fallait lancer quatre fois le dé cubique pour avoir plus de chance d'obtenir un 6 que de ne pas en obtenir.
\begin{enumerate}
\item Que pensez vous de cette affirmation ? Vous pourrez vous appuyer sur une simulation à l'aide d'un tableur ou d'un programme.
\item Essayons de modéliser.
	\begin{enumerate}
		\item Construire un arbre de dénombrement représentant les 4 lancers.  
		\item Déterminer la probabilité de ne pas obtenir de 6.
		\item En déduire la probabilité d'obtenir un 6.	
		\item Que dire alors à l'affirmation du chevalier de Méré ?				
	\end{enumerate}
\end{enumerate}
 
 
\EPCN{Modéliser. Représenter. Communiquer}  


Dans une classe de 35 élèves, 12 élèves font du judo, 25 font de la musique et 5 ne font ni judo ni musique. On rencontre un élève au hasard de cette classe.
\begin{enumerate}
\item Quelle est la probabilité qu'il pratique au moins une de ces deux activités ?
\item Quelle est la probabilité pour qu'il fasse à la fois du judo et de la musique ?
\end{enumerate}


\EPCN{Raisonner} 


$A$ et $B$ sont deux événements incompatibles tels que $p(A)=0,4$ et $p(B)=0,5$.

\begin{enumerate}
\item Calculer $p(A \cup B)$.
\item Calculer $p(\overline{A})$ et $p(\overline{B})$
\item $\overline{A}$ et $\overline{B}$ sont-ils incompatibles ?
\end{enumerate}

On utilisera la formule $p(A\cup B)= p(A) + p(B) - p(A \cap B)$ dans le cas où $A$ et $B$ sont incompatibles.


\EPCN{Communiquer. Calculer.} 


On dispose d'un dé pipé dont les faces sont numérotées de 1 à 6. Une étude statistique conduit à l'estimation suivante :
\begin{description}
\item[•] les faces de 1 à 4 ont la même probabilité p de sortie.
\item[•] la face 5 a une probabilité de sortie égale à 0,1.
\item[•] la face 6 a une probabilité de sortie égale à 0,3.
\end{description}
\begin{enumerate}
\item Déterminer la probabilité de sortie de chaque face.
\item Déterminer la probabilité d'obtenir une face paire.
\end{enumerate}





}% Equiprobabilité + simulation
%%########## \impress{\impressionEleve}{\begin{titre}[Probabilités - 2]

\Titre{Probabilité d'un événement}{2}
\end{titre}


\begin{CpsCol}
\begin{description}
\item[$\square$] Déterminer l'intersection de deux événements
\item[$\square$] Déterminer la réunion de deux événements
\item[$\square$] Utiliser la relation fondamentale
\item[$\square$] Déterminer la probabilité dans des situations équiprobables
\end{description}
\end{CpsCol}
 
 
\EPCN{Corrigé} 


On jette une pièce équilibrée 40 fois.

 

\begin{center}
\begin{lstlisting}
import random
n=int(input("Quel est le nombre de lancers ? "))
piece=["pile","face"]
proba_piece=[0]*2
for i in range (n) :
    alea = random.randint(0,1)
    proba_piece[alea]=proba_piece[alea]+1
    
for j in range (2) :
    print("On a alors p(",piece[j],") = ",proba_piece[j]/n)
\end{lstlisting}
\end{center}
 
 

 
\EPCN{Modéliser. Raisonner.} 


En informatique, un octet est une suite de 8 chiffres tous égaux à 1 ou 0. Par exemple : 10101100.

On note A : "les 2 premiers chiffres sont des 1" et B : "le dernier chiffre est 0".
\begin{enumerate}
\item Déterminer les octets vérifiant $A\cap B$.
\item  Déterminer les octets vérifiant $A\cup B$.
\end{enumerate}

\EPCN{Modéliser. Représenter. Communiquer}  


Dans sa penderie, Guillaume a deux pantalons, un noir et un blanc, deux vestes, une noire et une blanche, et trois chemises, deux blanches et une noire. Il prend au hasard un pantalon, une chemise et une veste.
\begin{enumerate}
\item A l'aide d'un arbre de dénombrement, déterminer le nombre de façon différentes de s'habiller.
\item
\begin{enumerate}
\item Calculer la probabilité de l'événement $A$ : "Il est habillé tout en noir".
\item Calculer la probabilité de l'événement $B$ : "Il est habillé tout en blanc".
\item $A$ et $B$ sont-ils contraires ?
Calculer la probabilité de l'événement $C$ : "Il une veste et un pantalon de couleurs différentes".
\end{enumerate}
\end{enumerate}
 
\EPCN{Modéliser. Représenter. Communiquer}  


Dans une classe de 35 élèves, 12 élèves font du judo, 25 font de la musique et 5 ne font ni judo ni musique. On rencontre un élève au hasard de cette classe.
\begin{enumerate}
\item Quelle est la probabilité qu'il pratique au moins une de ces deux activités ?
\item Quelle est la probabilité pour qu'il fasse à la fois du judo et de la musique ?
\end{enumerate}

 
 

\EPCN{Raisonner} 


$A$ et $B$ sont deux événements incompatibles tels que $p(A)=0,4$ et $p(B)=0,5$.

\begin{enumerate}
\item Calculer $p(A \cup B)$.
\item $\overline{A}$ et $\overline{B}$ sont-ils incompatibles ?
\end{enumerate}

\EPCN{Modéliser. Raisonner. Communiquer} 


Je lance un dé et je m'intéresse à l'événement $I$:"Obtenir un nombre inférieur strictement à 3".
\begin{enumerate}
\item Quelle est \textit{a priori} la probabilité de l'événement $I$ ?
\item Est ce toujours cette valeur ?
\end{enumerate}




\EPCN{Communiquer. Calculer.} 


On dispose d'un dé pipé dont les faces sont numérotées de 1 à 6. Une étude statistique conduit à l'estimation suivante :
\begin{description}
\item[•] les faces de 1 à 4 ont la même probabilité p de sortie.
\item[•] la face 5 a une probabilité de sortie égale à 0,1.
\item[•] la face 6 a une probabilité de sortie égale à 0,3.
\end{description}
\begin{enumerate}
\item Déterminer la probabilité de sortie de chaque face.
\item Déterminer la probabilité d'obtenir une face paire.
\end{enumerate}

\EPCN{Modéliser. Communiquer}  


Un jeu de de 32 cartes n'est pas truqué. On titre aléatoirement une carte.
\begin{enumerate}
\item Déterminer la probabilité de tirer un trèfle.
\item Déterminer la probabilité de tirer un roi.
\item Déterminer la probabilité de tirer le roi de cœur.
\end{enumerate}





\EPCNA{Modéliser. Raisonner} 


Le chevalier de Méré (1607-1684) prétend qu'il fallait lancer quatre fois le dé cubique pour avoir plus de chance d'obtenir un 6 que de ne pas en obtenir.
\begin{enumerate}
\item Que pensez vous de cette affirmation ? Vous pourrez vous appuyer sur une simulation à l'aide d'un tableur ou d'un programme.
\item Essayons de modéliser.
	\begin{enumerate}
		\item Construire un arbre de dénombrement représentant les 4 lancers.  
		\item Déterminer la probabilité de ne pas obtenir de 6.
		\item En déduire la probabilité d'obtenir un 6.	
		\item Que dire alors à l'affirmation du chevalier de Méré ?				
	\end{enumerate}
\end{enumerate}

\EPCNA{Raisonner} 


$A$ et $B$ sont deux événements tels que :
 $ p(A)=0,2$,  $p(B)=0,5$ , $p(A \cap B) = 0,1$.

\begin{enumerate}
\item Déterminer les probabilités des événements suivants :

	\begin{enumerate}
	\item $p(A \cup B)$	
	\item $p(A \cap \overline{B})$
	\item $p(\overline{A} \cap B)$
	\item $p(\overline{A} \cup B)$
	\item $p(\overline{A} \cap \overline{B})$
	\end{enumerate}

\item Identifier parmi les événements étudiés dans la question 1, l'événement contraire de $A \cup B$
\end{enumerate}







}% Equiprobabilité + simulation
%
%%########## \impress{\impressionEleve}{\begin{titre}[Probabilités]

\Titre{Probabilité d'un événement}{1}
\end{titre}

\begin{CpsCol}
\begin{description}
\item[$\square$] Déterminer l'intersection de deux événements
\item[$\square$] Déterminer la réunion de deux événements
\item[$\square$] Utiliser la relation fondamentale
\item[$\square$] Déterminer la probabilité dans des situations équiprobables
\end{description}
\end{CpsCol}


\begin{DefT}{Épreuve}\index{Épreuve}
On appelle \textbf{épreuve} ou expérience aléatoire, une situation soumise au hasard.
On appelle \textbf{issue} le résultat d'une épreuve, notée $\omega$. L'ensemble des issues est l'\textbf{univers} noté $\Omega$.\index{Univers} \index{Issue}
\end{DefT}

\begin{Ex}
\textit{Trois boules indiscernables au toucher et numérotées de 1 à 3 sont dans une urne. On tire une boule après l'autre.}

L'\textbf{épreuve} est : le tirage des trois boules. Les boules indiscernables au toucher, l'expérience est soumise au hasard.

Une \textbf{issue} est : (3,2,1)

L'univers est : $\Omega = \lbrace (1,2,3),(1,3,2),(2,3,1),(2,1,3),(3,2,1),(3,1,2)\rbrace$. Lorsque l'univers ne peut être donnée dans son ensemble, on peut données quelques issues.

\end{Ex}


\begin{DefT}{Événement}
On appelle événement un sous-ensemble de l'univers, c'est à dire un ensemble qui réunit toutes les issues favorables à une action déterminée.

\textbf{Un évènement est un ensemble.}
\end{DefT}


\begin{Ex}

Dans l'exemple précédent, on s'intéresse à l'événement A:"Obtenir un nombre qui se termine par 2".

$\Omega = \lbrace (1,3,2),(3,1,2)\rbrace$
\end{Ex}

\begin{DefT}{Événement élémentaire}
On appelle événement élémentaire, un événement qui ne continent qu'une seule issue.

\end{DefT}


\begin{Ex}

Dans l'exemple précédent, on s'intéresse à l'événement B:"Obtenir un nombre qui se termine par 2 et qui commence par 3".

$B = \lbrace(3,1,2)\rbrace$
\end{Ex}


\begin{Att}

L'événement $A$ n'est pas un événement élémentaire.
\end{Att}



\begin{DefT}{Événements incompatibles ou disjoints} \index{Événements!Incompatibles}\index{Événements!Disjoints see Événements!Incompatibles}
Deux événements A et B sont dits \textbf{incompatibles} ou \textbf{disjoints} lorsqu'ils ne peuvent se réaliser en même temps,
ou encore lorsque $A \cap B = \oslash$.
\end{DefT}

\begin{Ex}
$A = \left\lbrace 1 ; 2 \right\rbrace $ et $C = \left\lbrace  3 ; 5 \right\rbrace$ sont disjoints.
\end{Ex}


\begin{DefT}{Événements contraires} \index{Événements!Contraires}
Soit $\Omega$ un univers fini, $A$ et$ B$ deux événement inclus dans $\Omega$.
$A$ et $B$ sont deux événements \textbf{contraires} lorsque $A \cap B = \oslash$ et $A \cup B = \Omega$.
\end{DefT}

\begin{Ex}
Soit $\Omega =  \left\lbrace 1, 2, 3, 4, 5, 6 \right\rbrace $. $A = \left\lbrace1 ; 2\right\rbrace $ et $D = \left\lbrace 3 ; 4 ; 5 ; 6 \right\rbrace $ sont contraires.
\end{Ex}



\begin{DefT}{Probabilité d'un événement} \index{Probabilité d'un événement}
Soit $\Omega$ l'univers lié à une expérience aléatoire.
A chaque partie $B$ de $\Omega$, on fait correspondre un nombre compris entre 0 et 1, appelé \textbf{probabilité} de cet
événement $B$ tel que :
\begin{description}
\item[1] La somme des probabilités des événements élémentaires qui composent $\Omega$ est égale à 1.
\item[2] La probabilité de $B$ est la somme des probabilités des événements élémentaires qui composent $B$. On note $p(B)$ la probabilité de l'événement $B$.
\item[3] La probabilité d'un événement impossible est 0. 
\end{description}
\end{DefT}

\begin{Ex}

\textit{On dispose d'un dé pipé dont les faces sont numérotées de 1 à 6. Une étude statistique conduit à
l'estimation suivante :}
\begin{description}
\item[•]  \textit{les faces de 1 et 2 ont la même probabilité p de sortie.}
\item[•]  \textit{les faces de 3 à 5 ont la même probabilité 0,1 de sortie.}
\item[•]  \textit{la face 6 à une probabilité de sortie égale à 0,3.}
\end{description}
\textit{Déterminer la probabilité de sortie des faces 1 et 2.}

Soit $p_i$ la probabilité de la face numérotée $i$. Chaque face est une issue. L'événement "obtenir la face $i$" est un événement élémentaire puisque seule la face $i$ est une issue de cet événement.

\begin{description}
\item[1] On peut donc écrire $p_1+p_2+p_3+p_4+p_5+p_6=1$
\item[2] On s'intéresse à l'événement $B$ :" obtenir un face paire". $B=\left\lbrace 2 ; 4 ; 6 \right\rbrace $ donc $p(B)=p_2+p_4+p_6$
\item[3] Un événement impossible est $I$:"obtenir la face 7" donc $p(I)=0$.
\end{description}
\end{Ex}


\begin{Pp}[Relation fondamentale]
Soit $\Omega$ l'univers lié à une expérience aléatoire et $A$ et $B$ deux événements de cet univers.
$$p(A \cup B) = p(A) + p(B) – p(A \cap B)$$
Si $A$ et $B$ sont incompatibles alors $p(A \cup B) = p(A) + p(B)$.
\end{Pp}



\begin{Ex}

Les événements $A$ et $B$ sont tels que $p(A)=0,3$; $p(B)=0,5$ et $ p(A \cup B)=0,2$. Calculer $p(A \cup B)$.

Or, $p(A \cup B) = p(A) + p(B) – p(A \cap B) = 0,3 + 0,5 - 0,2 = 0,6$
\end{Ex}

\begin{Ex}

Les événements $A$ et $B$ sont tels que $p(A)=0,6$; $p(B)=0,7$ et $ p(A \cup B)=0,9$. Calculer $p(A \cap B)$.

Or, $p(A \cup B) = p(A) + p(B) – p(A \cap B) \Longleftrightarrow 0,9 = 0,6 + 0,7 - p(A \cap B) \Longleftrightarrow 0,9 = 1,3 - p(A \cap B) \Longleftrightarrow p(A \cap B)  = 1,3 - 0,9 = 0,4  $
\end{Ex}



\begin{Pp}
Soit $A$ un événement de $\Omega$ et $B$ son événement contraire. $p (A)+ p (B)=1$. $B$ est alors noté $\overline{A}$
\end{Pp}

\begin{Ex}
Dans un paquet de bonbons M\&M's, il y des bonbons bleus, rouges, verts, marrons, jaunes et oranges. On intéresse à l'événement $A$:" Obtenir un bonbon bleu". On sait que $p(A)=0,2$.

$\overline{A}$ est l'événement :"Obtenir un bonbon d'une autre couleur que bleu". Or $p(A)+p\left(\overline{A}\right)=1$ donc $p\left(\overline{A}\right)=0,8$.
\end{Ex}



\begin{Pp}[Équiprobabilité]
Lorsque tous les événements élémentaires ont la même probabilité, on dit qu'il y a équiprobabilité des
issues.
Dans ce cas, si l'univers $\Omega$ est composé de $n$ éventualités $\omega_i$ : $p(\omega_i)=\frac{1}{n}$

La probabilité d'un événement composé de $k$ éventualités est égale à $p(A)=\frac{k}{n}$
\end{Pp}


\begin{Ex}
\textit{On tire une carte dans un jeu de 32 cartes. On considère l'expérience soumise au hasard et les carte neuves.}
\begin{enumerate}
\item \textit{Quelle est la probabilité d'obtenir un Roi ?}
\item \textit{Quelle est la probabilité d'obtenir un Trèfle ?}
\item \textit{Quelle est la probabilité d'obtenir le Roi de Trèfle ?}
\item \textit{Quelle est la probabilité d'obtenir un Roi ou un Trèfle ?}
\end{enumerate}

\begin{enumerate}
\item Il y a 4 rois sur les 32 cartes.$p(R)=\frac{4}{32}=\frac{1}{8}$
\item Il y a 8 Trèfle sur les 32 cartes.$p(T)=\frac{8}{32}=\frac{1}{4}$
\item Il y a un seul roi de trèfle sur les 32 cartes. $p(R \cap)=\frac{1}{32}$
\item $p(R \cup T)=p(R)+p(T)-p(R \cap T) = \frac{4}{32} + \frac{8}{32} - \frac{1}{32} =\frac{11}{32}$
\end{enumerate}
\end{Ex}


\begin{Ex}
\textit{Trois boules indiscernables au toucher et numérotées de 1 à 3 sont dans une urne. On tire une boule après l'autre.}

\textit{L'univers est : $\Omega = \lbrace (1,2,3),(1,3,2),(2,3,1),(2,1,3),(3,2,1),(3,1,2)\rbrace$.}

\textit{On appelle :}
\begin{description}
\item $A$:"le nombre commence par 1"
\item $B$:"le nombre est impair"
\end{description}

\begin{enumerate}
\item \textit{Décrire par une phrase $A \cap B$}.

$A \cap B$ est l'événement "Obtenir un nombre impair qui commence par 1".

\item \emph{Calculer} $p(A)$ \textit{et}  $p(B)$

Par dénombrement de l'univers $\Omega$, $p(A) = \frac{2}{6}= \frac{1}{3}$.

Par dénombrement de l'univers $\Omega$, $p(B) = \frac{3}{6}= \frac{1}{2}$.
\item \textit{Calculer} $p(A \cap B)$.

Par dénombrement de l'univers $\Omega$, $p(A\cap B) = \frac{1}{6}$.

\item \textit{En déduire }$p(A \cup B)$. 

$p(A \cup B) = p(A)+p(B)-p(A\cap B) =  \frac{2}{6} +  \frac{3}{6} -  \frac{1}{6} =  \frac{4}{6}=  \frac{2}{3}$


On peut vérifier par dénombrement.
\end{enumerate}
\end{Ex}


}% Equiprobabilité + simulation
%%########## \impress{\impressionEleve}{
\begin{His}


L'histoire des probabilités a commencé avec celle du hasard et notamment des jeux de hasard. Bien que quelques calculs de probabilité soient apparus dans des applications précises au Moyen Âge, ce n'est qu'au XVIIe siècle que la théorie des probabilités prend vraiment ses débuts. Elles évoluent sans vrai formalisme pendant deux siècles autour du célèbre problème des partis, de problèmes d'urnes ou d'autres problèmes issus de jeux. Apparaît alors la théorie classique des probabilités basée sur la théorie de la mesure et la théorie de l'intégration. Cette théorie s'est depuis lors diversifiée dans de nombreuses applications.

\vspace{0.4cm}
Les discussions entre scientifiques, la publication des ouvrages et leur transmission étant difficiles à certaines époques, certaines questions historiques restent difficiles à résoudre ; c'est le cas de la paternité de la théorie des probabilités.

\vspace{0.4cm}

\begin{wrapfigure}[11]{r}{3.6cm}
\vspace{-7mm}
\includegraphics[scale=0.5]{image_chapitres/pascal.jpg}
\unnumberedcaption{Blaise \textsc{Pascal}}
\end{wrapfigure}

Le véritable début de la théorie des probabilités date de la correspondance entre Pierre de \textsc{Fermat} et Blaise \textsc{Pascal} en 1654 au sujet d'une désormais célèbre question posée par Antoine \textsc{Gombaud} (dit \textsc{chevalier de Méré}) : le problème des partis ou problèmes des points. 
\vspace{0.4cm}
«\textit{ Il avait pour objet de déterminer la proportion suivant laquelle l'enjeu doit être partagé entre les joueurs lorsqu'ils conviennent de ne point achever la partie, et qu'il leur reste à prendre pour la gagner, des nombres de points inégaux. Pascal en donna le premier la solution, mais pour le cas de deux joueurs seulement ; il fut ensuite résolu pour Fermat, dans le cas général d'un nombre quelconque de joueurs.} » (\textsc{Poisson}).
\vspace{0.4cm}
À la suite d'un séjour à Paris en 1655, Christian \textsc{Huygens} prend connaissance de cette discussion à l'Académie Parisienne et publie en 1657 le premier traité sur la théorie probabiliste : \textit{De ratiociniis in ludo aleae} (raisonnements sur les jeux de dés). C'est dans une lettre adressée à Frans van \textsc{Shooten}, qui a traduit son traité en latin dans \textit{Mathematische Oeffeningen}, qu'il attribue la paternité de la théorie des probabilités à \textsc{Pascal} et \textsc{Fermat} :

\vspace{0.4cm}

    « \textit{Il faut savoir d'ailleurs qu'il y a un certain temps que quelques-uns des plus Célèbres Mathématiciens de toute la France se sont occupés de ce genre de calcul, afin que personne ne m'attribue l'honneur de la première Invention qui ne m'appartient pas. }»

\vspace{0.4cm}

Puisqu'il fallait un certain délai entre l'écriture, la publication des œuvres et la diffusion de ces dernières, la paternité de la théorie des probabilités n'est pas unanime. Si la date de publication compte, c'est à \textsc{Huygens} que revient l'honneur d'être appelé le père de la théorie des probabilités, cependant si la date d'écrit compte, c'est à Jérôme \textsc{Cardan} que revient ce « titre ». Cependant la mauvaise réputation de \textsc{Cardan} a fait pencher la paternité sur \textsc{Pascal} et \textsc{Fermat}. \textsc{Leibniz} (1646-1716) ne cite que \textsc{Pascal}, \textsc{Fermat} et \textsc{Huygens} ; \textsc{Montmort} (1678-1719) cite \textsc{Cardan} mais d'une manière restrictive ; \textsc{Montucla} (1725-1799), \textsc{Laplace} (1749-1827) et \textsc{Poisson} (1781-1840) ne citent que \textsc{Pascal} et \textsc{Fermat}.

 
\PESP{https://fr.wikipedia.org/wiki/Histoire\_des_probabilit\%C3\%A9s}
 
\end{His}
% Equiprobabilité + simulation
%
%\chapter{Échantillon et simulation}
%
%\impress{\impressionEleve}{\begin{titre}[Probabilités]

\Titre{Intervalle de fluctuation}{4}
\end{titre}


\begin{CpsCol}
\begin{description}
\item[$\square$] Mettre en œuvre une simulation
\item[$\square$] Exploiter et faire une analyse critique d'un résultat d'échatillonnage
\end{description}
\end{CpsCol}


\Rec{1}{ES-0}

\begin{DefT}{Échantillon}\index{Échantillon}
Lorsqu'on répète $n$ fois, de façon identique et indépendante, une même expérience aléatoire, on obtient une série
de $n$ résultats que l'on appelle \textbf{échantillon de taille $n$}.
\end{DefT}

\begin{Rq}
\begin{enumerate}
\item Pour obtenir un échantillon à l'aide d'un tirage, celui ci doit s'effectuer avec remise pour que la
proportion du caractère ne change pas.
\item Si la taille de l'échantillon est négligeable devant l'effectif total, on peut assimiler un tirage sans
remise à un tirage avec remise.
\end{enumerate}
\end{Rq}



\begin{Ex}
\begin{enumerate}
\item  On lance 5 fois un même dé et on note les 5 faces obtenues. On dit que l'on a un échantillon de
taille 5.
\item On choisit 10 élèves dans le lycée. On dit que l'on a un échantillon de taille 10 des élèves du lycée, car le nombre de
lycéen choisi par rapport au nombre total de lycéen est négligeable. Par contre, si on choisit 10 élèves
dans la classe, on n'a plus un échantillon de la classe.
\end{enumerate}
\end{Ex}

\begin{DefT}{Fluctuation d'échantillonnage}\index{Fluctuation d'échantillonnage}
Lorsqu'on effectue plusieurs échantillons de même taille, la fréquence observée $f$ du caractère
varie : on l'appelle la \textbf{fluctuation d'échantillonnage}.
\end{DefT}


\begin{DefT}{Intervalle de fluctuation}\index{Intervalle de fluctuation}
Pour un grand nombre de tirages d'échantillon, l'intervalle centré en $p$, qui contient au moins
95\% des fréquences observées, $f$ est appelé \textbf{intervalle de fluctuation} de la fréquence $f$ au seuil de 95\%
des échantillons.

$$I_f=\left[ p - \frac{1}{\sqrt{n}}; p + \frac{1}{\sqrt{n}} \right]$$
\end{DefT}

\begin{Ex}
On jette une pièce de monnaie équilibrée, $p=0,5$ et on réalise 200 échantillons de taille 100 et pour chaque
échantillon, on note la fréquence d'apparition du coté pile.
Au moins 190 fréquences observées appartiennent à l'intervalle de fluctuation $[0,4 ; 0,6]$.
\end{Ex}

\mini{
\EPC{1}{ES-2}{Modéliser.}

\EPC{1}{ES-4}{Modéliser.}

\EPC{1}{ES-5}{Modéliser.}
}{
\EPC{1}{ES-8}{Modéliser.}

\CR{1}{ES-3}{Communiquer.}
}

\mini{
\EPC{1}{ES-9}{Modéliser.}

\EPC{1}{ES-6}{Modéliser.}

\EPC{1}{ES-10}{Modéliser.}

\EPC{1}{ES-13}{Modéliser.}
}{
\EPC{1}{ES-7}{Modéliser.}

\CR{1}{ES-11}{Communiquer.}

\EPCP{1}{ES-14}{Représenter.}
}


}% Intervalle de fluctuation + Prise de decision
%%%##########\impress{\impressionEleve}{\begin{titre}[Probabilités]

\Titre{Estimation}{1}
\end{titre}


\begin{CpsCol}
\begin{description}
\item[$\square$] Estimer une proportion
\end{description}
\end{CpsCol}


\Rec{1}{ES-11}

\begin{DefT}{Estimation}\index{Estimation}
L'intervalle $I_c=\left[ f - \frac{1}{\sqrt{n}}; f + \frac{1}{\sqrt{n}} \right]$ est appelé intervalle de confiance au seuil de 0,95.
\end{DefT}

\Exo{1}{ES-15}

\CR{1}{ES-12}


}%  Intervalle de confiance + Prise de decision
%%%
%%%
%%%
%%%
%%%%%%%%%%%%%%%%%%%%%%%%%%%%%%%%%%%%%%%%%%%%%%%%%%%%%%%%%%%%%%%%%%%%%%%%%%%%%%%%%%%%%%%%%%%%%%%%%%%%%%%%%%%%%%%%%%%%%%%%%%%%%%%%%%%%%%%%%%%%%%%%%%%
%%%%%%%%%%%%%%%%%%%%%%%%%%%%%%%%%%%%%%%%%%%%%%%%%%%%%%%%%%%%%%%%%%%%%%%%%%%%%%%%%%%%%%%%%%%%%%%%%%%%%%%%%%%%%%%%%%%%%%%%%%%%%%%%%%%%%%%%%%%%%%%%%%%
%%%%%%%%%%%%%%%%%%%%%%%%%%%%%%%%%%%%%%%%%%%%%%%%%%%%%%%%%%%%%  Espace et géométrie   %%%%%%%%%%%%%%%%%%%%%%%%%%%%%%%%%%%%%%%%%%%%%%%%%%%%%%%%%%%%%%
%%%%%%%%%%%%%%%%%%%%%%%%%%%%%%%%%%%%%%%%%%%%%%%%%%%%%%%%%%%%%%%%%%%%%%%%%%%%%%%%%%%%%%%%%%%%%%%%%%%%%%%%%%%%%%%%%%%%%%%%%%%%%%%%%%%%%%%%%%%%%%%%%%%
%%%%%%%%%%%%%%%%%%%%%%%%%%%%%%%%%%%%%%%%%%%%%%%%%%%%%%%%%%%%%%%%%%%%%%%%%%%%%%%%%%%%%%%%%%%%%%%%%%%%%%%%%%%%%%%%%%%%%%%%%%%%%%%%%%%%%%%%%%%%%%%%%%%
%
%\part{Géométrie}
%
%%\chapter{Espace}
%%%\intro{pavage}{Dans les arts orientaux, les formes jouent un rôle essentiel. La composition de \textbf{translation}, \textbf{rotation}, \textbf{symétrie} de ces formes illumine certains des plus beaux palais du Moyen orient. 
%%%
%%%Escher a produit plus de 150 dessins en couleurs, dans lesquels s'imbriquaient des créatures qui rampaient, nageaient ou planaient, emplissant tout le plan. Sans entrer dans les détails mathématiques, notons que les œuvres d'Escher présentent souvent des transformations géométriques connues, telles la translation, la rotation, la réflexion ou l'homothétie.  }
%%%
%%##########\impress{\impressionEleve}{\begin{titreTice}[Géométrie dans l'espace]

\Titre{Construction et visualisation}{4}
\end{titreTice}


\begin{CpsCol}
\textbf{Construction et visualisation}
\begin{description}
\item[$\square$] Construire avec un logiciel dynamique un solide usuel
\end{description}
\end{CpsCol}

\subsection*{Construction d'un cube }

\begin{enumerate}
\item Ouvrir une fenêtre 2D et une fenêtre 3D
\item Placer 2 points A et B dans la fenêtre 2D 
\item A l'aide de l'icône \textbf{Polygone régulier} \includegraphics[scale=0.4]{poly.jpg}, construire un carré ABCD.
\item Sélectionner la fenêtre 3D. Une barre, d'icônes propre 3D s'initialise à la place de la barre d'icônes 2D. 

\includegraphics[scale=0.5]{outils-3D.jpg} 

\item Utiliser l'icône \includegraphics[scale=0.5]{cube.jpg} pour extruder le cube. Cliquer sur les points $A$ et $B$ dans la fenêtre 3D. Et voilà le cube.

On peut placer directement 2 points dans la \textbf{fenêtre 3D }et créer un cube mais si l'on souhaite un cube "posé", il est préférable d'utiliser cette méthode.

\end{enumerate}

\subsection*{Construction d'un plan }

Un plan passe par 3 points non alignés.

 A l'aide de l'icône \textbf{Plan passant par 3 points} \includegraphics[scale=0.4]{plan.jpg}, construire le plan ABE.
 
\textbf{Attention}, les points induits de la construction du solide construit précédemment ne sont pas sélectionnables \mbox{directement}. Il faut les sélectionner dans la fenêtre \textbf{Algèbre}.













}
%%##########\impress{\impressionEleve}{\begin{titre}[Géométrie dans l'espace]

\Titre{Calculs et positions relatives}{8}
\end{titre}


\begin{CpsCol}
\textbf{Positions relatives}
\begin{description}
\item[$\square$] Construire un patron d'un solide usuel (parallélépipède rectangle, pyramides, cônes de révolution)
\item[$\square$] Construire avec un logiciel dynamique un solide usuel
\item[$\square$] Représenter en perspective cavalière
\item[$\square$] Visualiser la position relative d'une droite et d'un plan, de deux plans
\item[$\square$] Effectuer des calculs de longueurs, de volumes, d'aires
\end{description}
\end{CpsCol}


\EPCB{1}{GE-0}{Chercher. Représenter.}


\begin{DefT}{Perspective cavalière}\index{Perspective cavalière}
La représentation de solide de l'espace en perspective cavalière respecte les règles suivantes  :
\begin{description}
\item[•] la figure vue de face à les dimensions grandeur réelle (sans changer sa forme : un carré est un carré, un rectangle est un rectangle....) 
\item[•] deux droites parallèles sont représentées par 2 droites parallèles
\item[•] des points alignés sont représentés par des points alignés
\item[•] le milieu d'un segment est représenté par le milieu du segment dessiné
\item[•] les éléments non visibles sont dessinés en pointillés, les éléments visibles en traits pleins
\end{description}
\end{DefT}

\mini{
\EPCC{1}{GE-1}{Chercher. Représenter.}
}{
\EPCC{1}{GE-2}{Chercher. Représenter.}
}

\begin{Reg}[ d'incidence]
\begin{description}
\item[R1] Dans chaque plan de l'espace,on peut appliquer tous les théorèmes de géométrie plane (théorèmes de Pythagore, de Thalès,...).
\item[R2]  Par deux points distincts de l'espace, il passe une unique droite.
\item[R3]  Par trois points non alignes A, B et C de l'espace, il passe un unique plan, noté (ABC)
\item[R4]   Si deux points distincts A et B de l'espace appartiennent à un plan P, alors la droite (AB) est contenue dans le plan P.
\end{description}
\end{Reg}

\mini{
\EPCB{1}{GE-7}{Chercher. Représenter. Calculer.}


}{
\EPCB{1}{GE-9}{Chercher. Représenter. Calculer.}

\EPCB{1}{GE-10}{Chercher. Représenter. Calculer.}
}

\EPCB{1}{GE-8}{Chercher. Représenter. Modéliser. Calculer.}

\begin{Reg}[Positions relatives de deux droites]
\begin{description}
\item[R5]   Deux droites de l'espace sont soit coplanaires, soit non coplanaires.
\end{description}


\begin{tabular}{|>{\centering\arraybackslash}p{3.75cm}|>{\centering\arraybackslash}p{3.75cm}|>{\centering\arraybackslash}p{3.75cm}|>{\centering\arraybackslash}p{3.75cm}|}
\hline 
\multicolumn{3}{|c|}{Droites coplanaires (dans un même plan)} & Droites non coplanaires  \\ 
\hline 
d et d' sécantes & \multicolumn{2}{c|}{d et d' parallèles} &  \\ 
\hline 
\definecolor{ffqqqq}{rgb}{1.,0.,0.}
\definecolor{qqqqff}{rgb}{0.,0.,1.}
\begin{tikzpicture}[line cap=round,line join=round,>=triangle 45,x=0.7cm,y=0.7cm].0cm]
\clip(-8.22,-3.62) rectangle (-2.46,1.82);
\draw [color=qqqqff] (-7.,-3.)-- (-4.,-3.);
\draw [color=qqqqff] (-4.,-3.)-- (-3.,-2.);
\draw [dash pattern=on 2pt off 2pt,color=qqqqff] (-3.,-2.)-- (-6.,-2.);
\draw [dash pattern=on 2pt off 2pt,color=qqqqff] (-6.,-2.)-- (-7.,-3.);
\draw [color=qqqqff] (-7.,0.)-- (-4.,0.);
\draw [color=qqqqff] (-4.,0.)-- (-3.,1.);
\draw [color=qqqqff] (-3.,1.)-- (-6.,1.);
\draw [color=qqqqff] (-6.,1.)-- (-7.,0.);
\draw [color=qqqqff] (-7.,0.)-- (-7.,-3.);
\draw [color=qqqqff] (-4.,0.)-- (-4.,-3.);
\draw [color=qqqqff] (-3.,1.)-- (-3.,-2.);
\draw [dash pattern=on 2pt off 2pt,color=qqqqff] (-6.,1.)-- (-6.,-2.);
\draw [color=ffqqqq,domain=-8.22:-2.66] plot(\x,{(-9.-0.*\x)/3.});
\draw [dash pattern=on 2pt off 2pt, color=ffqqqq,domain=-8.22:-2.46] plot(\x,{(-5.25--0.75*\x)/3.75});
\draw [color=ffqqqq,domain=-8.22:-6.86] plot(\x,{(-5.25--0.75*\x)/3.75});
\draw [color=ffqqqq,domain=-3.02:-2.66] plot(\x,{(-5.25--0.75*\x)/3.75});
\begin{scriptsize}
\draw [color=qqqqff] (-7.,-3.)-- ++(-2.5pt,0 pt) -- ++(5.0pt,0 pt) ++(-2.5pt,-2.5pt) -- ++(0 pt,5.0pt);
\draw[color=qqqqff] (-7.34,-2.76) node {$A$};
\draw [color=qqqqff] (-4.,-3.)-- ++(-2.5pt,0 pt) -- ++(5.0pt,0 pt) ++(-2.5pt,-2.5pt) -- ++(0 pt,5.0pt);
\draw[color=qqqqff] (-3.8,-3.1) node {$B$};
\draw [color=qqqqff] (-3.,-2.)-- ++(-2.5pt,0 pt) -- ++(5.0pt,0 pt) ++(-2.5pt,-2.5pt) -- ++(0 pt,5.0pt);
\draw[color=qqqqff] (-2.86,-1.64) node {$C$};
\draw [color=qqqqff] (-6.,-2.)-- ++(-2.5pt,0 pt) -- ++(5.0pt,0 pt) ++(-2.5pt,-2.5pt) -- ++(0 pt,5.0pt);
\draw[color=qqqqff] (-5.78,-1.7) node {$D$};
\draw [color=qqqqff] (-7.,0.)-- ++(-2.5pt,0 pt) -- ++(5.0pt,0 pt) ++(-2.5pt,-2.5pt) -- ++(0 pt,5.0pt);
\draw[color=qqqqff] (-7.24,0.26) node {$E$};
\draw [color=qqqqff] (-4.,0.)-- ++(-2.5pt,0 pt) -- ++(5.0pt,0 pt) ++(-2.5pt,-2.5pt) -- ++(0 pt,5.0pt);
\draw[color=qqqqff] (-3.82,-0.2) node {$F$};
\draw [color=qqqqff] (-3.,1.)-- ++(-2.5pt,0 pt) -- ++(5.0pt,0 pt) ++(-2.5pt,-2.5pt) -- ++(0 pt,5.0pt);
\draw[color=qqqqff] (-2.86,1.36) node {$G$};
\draw [color=qqqqff] (-6.,1.)-- ++(-2.5pt,0 pt) -- ++(5.0pt,0 pt) ++(-2.5pt,-2.5pt) -- ++(0 pt,5.0pt);
\draw[color=qqqqff] (-5.86,1.36) node {$H$};
\draw[color=ffqqqq] (-17.32,-2.68) node {$s$};
\draw [fill=black] (-6.75,-2.75) circle (1pt);
\draw[color=black] (-6.6,-2.4) node {$I$};
\end{scriptsize}
\end{tikzpicture}
  & 
  
\definecolor{ffqqqq}{rgb}{1.,0.,0.}
\definecolor{qqqqff}{rgb}{0.,0.,1.}
\begin{tikzpicture}[line cap=round,line join=round,>=triangle 45,x=0.7cm,y=0.7cm]
\clip(-7.84,-3.56) rectangle (-1.86,2.08);
\draw [color=qqqqff] (-7.,-3.)-- (-4.,-3.);
\draw [color=qqqqff] (-4.,-3.)-- (-3.,-2.);
\draw [dash pattern=on 2pt off 2pt,color=qqqqff] (-3.,-2.)-- (-6.,-2.);
\draw [dash pattern=on 2pt off 2pt,color=qqqqff] (-6.,-2.)-- (-7.,-3.);
\draw [color=qqqqff] (-7.,0.)-- (-4.,0.);
\draw [color=qqqqff] (-4.,0.)-- (-3.,1.);
\draw [color=qqqqff] (-3.,1.)-- (-6.,1.);
\draw [color=qqqqff] (-6.,1.)-- (-7.,0.);
\draw [color=qqqqff] (-7.,0.)-- (-7.,-3.);
\draw [color=qqqqff] (-4.,0.)-- (-4.,-3.);
\draw [color=qqqqff] (-3.,1.)-- (-3.,-2.);
\draw [dash pattern=on 2pt off 2pt,color=qqqqff] (-6.,1.)-- (-6.,-2.);
\draw [color=ffqqqq,domain=-7.84:-2.36] plot(\x,{(-9.-0.*\x)/3.});
\draw [color=ffqqqq,domain=-7.84:-2.36] plot(\x,{(--3.-0.*\x)/3.});
\begin{scriptsize}
\draw [color=qqqqff] (-7.,-3.)-- ++(-2.5pt,0 pt) -- ++(5.0pt,0 pt) ++(-2.5pt,-2.5pt) -- ++(0 pt,5.0pt);
\draw[color=qqqqff] (-7.34,-2.76) node {$A$};
\draw [color=qqqqff] (-4.,-3.)-- ++(-2.5pt,0 pt) -- ++(5.0pt,0 pt) ++(-2.5pt,-2.5pt) -- ++(0 pt,5.0pt);
\draw[color=qqqqff] (-3.8,-3.1) node {$B$};
\draw [color=qqqqff] (-3.,-2.)-- ++(-2.5pt,0 pt) -- ++(5.0pt,0 pt) ++(-2.5pt,-2.5pt) -- ++(0 pt,5.0pt);
\draw[color=qqqqff] (-2.86,-1.64) node {$C$};
\draw [color=qqqqff] (-6.,-2.)-- ++(-2.5pt,0 pt) -- ++(5.0pt,0 pt) ++(-2.5pt,-2.5pt) -- ++(0 pt,5.0pt);
\draw[color=qqqqff] (-5.78,-1.7) node {$D$};
\draw [color=qqqqff] (-7.,0.)-- ++(-2.5pt,0 pt) -- ++(5.0pt,0 pt) ++(-2.5pt,-2.5pt) -- ++(0 pt,5.0pt);
\draw[color=qqqqff] (-7.24,0.26) node {$E$};
\draw [color=qqqqff] (-4.,0.)-- ++(-2.5pt,0 pt) -- ++(5.0pt,0 pt) ++(-2.5pt,-2.5pt) -- ++(0 pt,5.0pt);
\draw[color=qqqqff] (-3.82,-0.2) node {$F$};
\draw [color=qqqqff] (-3.,1.)-- ++(-2.5pt,0 pt) -- ++(5.0pt,0 pt) ++(-2.5pt,-2.5pt) -- ++(0 pt,5.0pt);
\draw[color=qqqqff] (-2.86,1.36) node {$G$};
\draw [color=qqqqff] (-6.,1.)-- ++(-2.5pt,0 pt) -- ++(5.0pt,0 pt) ++(-2.5pt,-2.5pt) -- ++(0 pt,5.0pt);
\draw[color=qqqqff] (-5.86,1.36) node {$H$};
\draw[color=ffqqqq] (-17.32,-2.68) node {$s$};
\draw[color=ffqqqq] (-17.32,0.84) node {$t$};
\end{scriptsize}
\end{tikzpicture}  
  
  
  & 

\definecolor{ffqqqq}{rgb}{1.,0.,0.}
\definecolor{qqqqff}{rgb}{0.,0.,1.}
\begin{tikzpicture}[line cap=round,line join=round,>=triangle 45,x=0.7cm,y=0.7cm]
\clip(-8.28,-3.7) rectangle (-2.06,1.64);
\draw [color=qqqqff] (-7.,-3.)-- (-4.,-3.);
\draw [color=qqqqff] (-4.,-3.)-- (-3.,-2.);
\draw [dash pattern=on 2pt off 2pt,color=qqqqff] (-3.,-2.)-- (-6.,-2.);
\draw [dash pattern=on 2pt off 2pt,color=qqqqff] (-6.,-2.)-- (-7.,-3.);
\draw [color=qqqqff] (-7.,0.)-- (-4.,0.);
\draw [color=qqqqff] (-4.,0.)-- (-3.,1.);
\draw [color=qqqqff] (-3.,1.)-- (-6.,1.);
\draw [color=qqqqff] (-6.,1.)-- (-7.,0.);
\draw [color=qqqqff] (-7.,0.)-- (-7.,-3.);
\draw [color=qqqqff] (-4.,0.)-- (-4.,-3.);
\draw [color=qqqqff] (-3.,1.)-- (-3.,-2.);
\draw [dash pattern=on 2pt off 2pt,color=qqqqff] (-6.,1.)-- (-6.,-2.);
\draw [color=ffqqqq,domain=-8.28:-2.66] plot(\x,{(-9.-0.*\x)/3.});
\draw [color=ffqqqq,domain=-8.28:-2.66] plot(\x,{(-9.-0.*\x)/3.});
\begin{scriptsize}
\draw [color=qqqqff] (-7.,-3.)-- ++(-2.5pt,0 pt) -- ++(5.0pt,0 pt) ++(-2.5pt,-2.5pt) -- ++(0 pt,5.0pt);
\draw[color=qqqqff] (-7.34,-2.76) node {$A$};
\draw [color=qqqqff] (-4.,-3.)-- ++(-2.5pt,0 pt) -- ++(5.0pt,0 pt) ++(-2.5pt,-2.5pt) -- ++(0 pt,5.0pt);
\draw[color=qqqqff] (-3.8,-3.1) node {$B$};
\draw [color=qqqqff] (-3.,-2.)-- ++(-2.5pt,0 pt) -- ++(5.0pt,0 pt) ++(-2.5pt,-2.5pt) -- ++(0 pt,5.0pt);
\draw[color=qqqqff] (-2.86,-1.64) node {$C$};
\draw [color=qqqqff] (-6.,-2.)-- ++(-2.5pt,0 pt) -- ++(5.0pt,0 pt) ++(-2.5pt,-2.5pt) -- ++(0 pt,5.0pt);
\draw[color=qqqqff] (-5.78,-1.7) node {$D$};
\draw [color=qqqqff] (-7.,0.)-- ++(-2.5pt,0 pt) -- ++(5.0pt,0 pt) ++(-2.5pt,-2.5pt) -- ++(0 pt,5.0pt);
\draw[color=qqqqff] (-7.24,0.26) node {$E$};
\draw [color=qqqqff] (-4.,0.)-- ++(-2.5pt,0 pt) -- ++(5.0pt,0 pt) ++(-2.5pt,-2.5pt) -- ++(0 pt,5.0pt);
\draw[color=qqqqff] (-3.82,-0.2) node {$F$};
\draw [color=qqqqff] (-3.,1.)-- ++(-2.5pt,0 pt) -- ++(5.0pt,0 pt) ++(-2.5pt,-2.5pt) -- ++(0 pt,5.0pt);
\draw[color=qqqqff] (-2.86,1.36) node {$G$};
\draw [color=qqqqff] (-6.,1.)-- ++(-2.5pt,0 pt) -- ++(5.0pt,0 pt) ++(-2.5pt,-2.5pt) -- ++(0 pt,5.0pt);
\draw[color=qqqqff] (-5.86,1.36) node {$H$};
\draw[color=ffqqqq] (-17.32,-2.68) node {$s$};
\draw[color=ffqqqq] (-17.32,-2.68) node {$t$};
\end{scriptsize}
\end{tikzpicture} 
  
  & 
  
  \definecolor{qqwuqq}{rgb}{0.,0.39215686274509803,0.}
\definecolor{ffqqqq}{rgb}{1.,0.,0.}
\definecolor{qqqqff}{rgb}{0.,0.,1.}
\begin{tikzpicture}[line cap=round,line join=round,>=triangle 45,x=0.7cm,y=0.7cm]
\clip(-7.92,-3.64) rectangle (-1.86,1.66);
\draw [color=qqqqff] (-7.,-3.)-- (-4.,-3.);
\draw [color=qqqqff] (-4.,-3.)-- (-3.,-2.);
\draw [dash pattern=on 1pt off 1pt,color=qqqqff] (-3.,-2.)-- (-6.,-2.);
\draw [dash pattern=on 1pt off 1pt,color=qqqqff] (-6.,-2.)-- (-7.,-3.);
\draw [color=qqqqff] (-7.,0.)-- (-4.,0.);
\draw [color=qqqqff] (-4.,0.)-- (-3.,1.);
\draw [color=qqqqff] (-3.,1.)-- (-6.,1.);
\draw [color=qqqqff] (-6.,1.)-- (-7.,0.);
\draw [color=qqqqff] (-7.,0.)-- (-7.,-3.);
\draw [color=qqqqff] (-4.,0.)-- (-4.,-3.);
\draw [color=qqqqff] (-3.,1.)-- (-3.,-2.);
\draw [dash pattern=on 1pt off 1pt,color=qqqqff] (-6.,1.)-- (-6.,-2.);
\draw [color=ffqqqq,domain=-7.92:-2.4 ] plot(\x,{(-9.-0.*\x)/3.});
\draw [color=ffqqqq,domain=-7.92:-2.4 ] plot(\x,{(-9.-0.*\x)/3.});
\draw [color=qqwuqq,domain=-7.92:-6.5] plot(\x,{(-8.26-1.54*\x)/-3.64});
\draw [color=qqwuqq,domain=-2.85:-2.4] plot(\x,{(-8.26-1.54*\x)/-3.64});
\draw [dash pattern=on 2pt off 2pt,color=qqwuqq,domain=-6.8:-2.85] plot(\x,{(-8.26-1.54*\x)/-3.64});
\begin{scriptsize}
\draw [color=qqqqff] (-7.,-3.)-- ++(-2.5pt,0 pt) -- ++(5.0pt,0 pt) ++(-2.5pt,-2.5pt) -- ++(0 pt,5.0pt);
\draw[color=qqqqff] (-7.34,-2.76) node {$A$};
\draw [color=qqqqff] (-4.,-3.)-- ++(-2.5pt,0 pt) -- ++(5.0pt,0 pt) ++(-2.5pt,-2.5pt) -- ++(0 pt,5.0pt);
\draw[color=qqqqff] (-3.8,-3.1) node {$B$};
\draw [color=qqqqff] (-3.,-2.)-- ++(-2.5pt,0 pt) -- ++(5.0pt,0 pt) ++(-2.5pt,-2.5pt) -- ++(0 pt,5.0pt);
\draw[color=qqqqff] (-2.86,-1.64) node {$C$};
\draw [color=qqqqff] (-6.,-2.)-- ++(-2.5pt,0 pt) -- ++(5.0pt,0 pt) ++(-2.5pt,-2.5pt) -- ++(0 pt,5.0pt);
\draw[color=qqqqff] (-5.78,-1.7) node {$D$};
\draw [color=qqqqff] (-7.,0.)-- ++(-2.5pt,0 pt) -- ++(5.0pt,0 pt) ++(-2.5pt,-2.5pt) -- ++(0 pt,5.0pt);
\draw[color=qqqqff] (-7.24,0.26) node {$E$};
\draw [color=qqqqff] (-4.,0.)-- ++(-2.5pt,0 pt) -- ++(5.0pt,0 pt) ++(-2.5pt,-2.5pt) -- ++(0 pt,5.0pt);
\draw[color=qqqqff] (-3.82,-0.2) node {$F$};
\draw [color=qqqqff] (-3.,1.)-- ++(-2.5pt,0 pt) -- ++(5.0pt,0 pt) ++(-2.5pt,-2.5pt) -- ++(0 pt,5.0pt);
\draw[color=qqqqff] (-2.86,1.36) node {$G$};
\draw [color=qqqqff] (-6.,1.)-- ++(-2.5pt,0 pt) -- ++(5.0pt,0 pt) ++(-2.5pt,-2.5pt) -- ++(0 pt,5.0pt);
\draw[color=qqqqff] (-5.86,1.36) node {$H$};
\draw[color=ffqqqq] (-17.32,-2.68) node {$s$};
\draw[color=ffqqqq] (-17.32,-2.68) node {$t$};
\draw [color=qqqqff] (-6.64,-0.54)-- ++(-2.5pt,0 pt) -- ++(5.0pt,0 pt) ++(-2.5pt,-2.5pt) -- ++(0 pt,5.0pt);
\draw[color=qqwuqq] (-17.32,-4.72) node {$a$};
\end{scriptsize}
\end{tikzpicture}
  
    \\ 
\hline 
d et d' ont un point d'intersection A & d et d' strictement parallèles ( pas de point d'intersection) & d et d' confondues & Aucun plan ne contient les deux droites d et d' ( pas de point d'intersection) \\ 
\hline 
\end{tabular} 
\end{Reg}

\begin{Reg}[Positions relatives d'une droite et d'un plan]
\begin{description}
\item[R6] Une droite $d$ et un plan $\mathscr{P}$ de l'espace sont soit sécants en un point, soit parallèles.
\end{description}
\end{Reg}

$\triangleright$ Représenter les 3 cas
\begin{description}
\item[•] $d$ et $\mathscr{P}$  ont un point d'intersection $B$. 
\item[•] $d$ et $\mathscr{P}$  sont strictement parallèles.
\item[•] $d$ est contenue dans $\mathscr{P}$.
\end{description}

\EPCB{1}{GE-12}{ Représenter. }

\begin{Reg}[Positions relatives de deux plans]
\begin{description}
\item[R7] Deux plans de l'espace sont soit sécants en une droite, soit parallèles.
\end{description}
\end{Reg}

$\triangleright$ Représenter les 3 cas
\begin{description}
\item[•] $\mathscr{P}$ et $\mathscr{P}'$  sont sécants en $d$.
\item[•] $d$ et $\mathscr{P}$  sont strictement parallèles.
\item[•] $\mathscr{P}$ et $\mathscr{P}'$  sont confondus.
\end{description}


\EPCN{ Utiliser un logiciel de géométrie dynamique. Geogebra}

\begin{enumerate}
\item Construction d'un cube  

\begin{enumerate}
\item Ouvrir une fenêtre 2D et une fenêtre 3D
\item Placer 2 points A et B dans la fenêtre 2D 
\item A l'aide de l'icône \textbf{Polygone régulier} \includegraphics[scale=0.4]{poly.jpg}, construire un carré ABCD.
\item Sélectionner la fenêtre 3D. Une barre, d'icônes propre 3D s'initialise à la place de la barre d'icônes 2D. 

\includegraphics[scale=0.5]{outils-3D.jpg} 

\item Utiliser l'icône \includegraphics[scale=0.5]{cube.jpg} pour extruder le cube. Cliquer sur les points $A$ et $B$ dans la fenêtre 3D. Et voilà le cube.

On peut placer directement 2 points dans la \textbf{fenêtre 3D }et créer un cube mais si l'on souhaite un cube "posé", il est préférable d'utiliser cette méthode.

\end{enumerate}

\item  Construction d'un plan  

Un plan passe par 3 points non alignés.

 A l'aide de l'icône \textbf{Plan passant par 3 points} \includegraphics[scale=0.4]{plan.jpg}, construire le plan ABE.
 
\textbf{Attention}, les points induits de la construction du solide construit précédemment ne sont pas sélectionnables \mbox{directement}. Il faut les sélectionner dans la fenêtre \textbf{Algèbre}.

\end{enumerate}


\mini{
\EPCB{1}{GE-3}{ Représenter. Raisonner. Communiquer. }

\EPCB{1}{GE-4}{ Représenter. Raisonner. Communiquer.}

\EPCB{1}{GE-17}{ Modéliser. Calculer. }
}{
\EPCB{1}{GE-5}{ Représenter. Raisonner. Communiquer.}

\EPCB{1}{GE-11}{ Représenter. Raisonner. Communiquer.}

\EPCB{1}{GE-16}{ Représenter. Calculer. Communiquer.}

\EPCNA{Pour aller plus loin}

\url{http://lycee-valin.fr/maths/exercices_en_ligne/espace.html}

}
} % Patron, solide et perspective
%%##########\impress{\impressionEleve}{\begin{titre}[Géométrie dans l'espace]

\Titre{Parallélisme}{4}
\end{titre}


\begin{CpsCol}
\textbf{Parallélisme}
\begin{description}
\item[$\square$] Déterminer l'intersection d'une droite et d'un plan
\item[$\square$] Déterminer l'intersection de deux plans
\end{description}
\end{CpsCol}


\begin{Reg}[Parallélisme entre droites]
\begin{description}
\item[P1] Deux droites parallèles à une même droite sont parallèles entre elles.
	$$ \text{ Si } d// d' \text{ et } d'// d" \text{ alors } d// d''$$
\item[P2]   Si deux droites sont parallèles, alors tout plan qui coupe l'une, coupe l'autre.
\end{description}
\end{Reg}


\begin{Reg}[Parallélisme entre plans]
\begin{description}
\item[P3]   Deux plans parallèles à un même plan sont parallèles entre eux.
	$$\text{ Si } \mathscr{P} // \mathscr{P}' \text{ et }  \mathscr{P}'// \mathscr{Q} \text{ alors } \mathscr{P} //\mathscr{Q}$$

\item[P4]  Si deux droites sécantes $d$ et $d'$ d'un plan $\mathscr{P}$   sont parallèles à deux droites sécantes $\Delta$ et $\Delta '$ d'un plan $\mathscr{Q}$ alors $\mathscr{P}$  et $\mathscr{Q}$ sont parallèles.

\definecolor{ffqqqq}{rgb}{1.,0.,0.}
\definecolor{qqqqff}{rgb}{0.,0.,1.}
\definecolor{qqwuqq}{rgb}{0.,0.39215686274509803,0.}
\begin{tikzpicture}[line cap=round,line join=round,>=triangle 45,x=0.5cm,y=0.5cm]
\clip(4.28,-2.3) rectangle (15.16,3.98);
\fill[color=qqwuqq,fill=qqwuqq,fill opacity=0.11] (5.,1.) -- (11.,1.) -- (15.,3.) -- (9.,3.) -- cycle;
\fill[color=qqwuqq,fill=qqwuqq,fill opacity=0.11] (9.,0.) -- (15.,0.) -- (11.,-2.) -- (5.,-2.) -- cycle;
\draw [color=qqqqff] (6.12,1.26)-- (13.22,2.62);
\draw [color=qqqqff] (6.2,-1.72)-- (13.18,-0.34);
\draw [color=ffqqqq] (8.46,2.5)-- (12.,2.);
\draw [color=ffqqqq] (8.78,-0.44)-- (12.,-1.);
\end{tikzpicture}


\item[P5]  Si deux plans $\mathscr{P}$  et $\mathscr{P}'$ sont parallèles alors tout plan qui coupe $\mathscr{P}$  coupe $\mathscr{P'}$ et les droites d'intersection $d$ et $d'$ sont parallèles.

\AV{https://www.geogebra.org/m/Hj7trpR8}{Visualisation}
\end{description}
\end{Reg}


\AD{1}{GE-6}

\Exo{1}{GE-13}



\begin{Reg}[Parallélisme entre droite et plan]
\begin{description}
\item[P6] Si deux plans $\mathscr{P}$ et $\mathscr{P}'$ sont parallèles et si une droite $d$ est parallèle à $\mathscr{P}$ alors $d$ est parallèle à $\mathscr{P}'$.
\item[P7]  Si deux droites $d$ et $d'$ sont parallèles et si $d$ est contenue dans un plan $\mathscr{P}$, alors $d'$ est parallèle à $\mathscr{P}$.
\item[P8]  Si deux plans $\mathscr{P}$ et $\mathscr{P}'$ sont sécants selon une droite $\Delta$ et si une droite $d$ est parallèle à $\mathscr{P}$ et à $\mathscr{P}'$ alors d est parallèle à $\Delta$.

\AV{https://www.geogebra.org/m/yCxHKp2T}{Visualisation}

\end{description}
\end{Reg}



\AD{1}{GE-14}

\Exo{1}{GE-15}



\begin{ThT}{Théorème du toit}
Si 
$ \left\lbrace  \begin{tabular}{l}
$d$ et $d'$ sont des droites parallèles \\ 
 $\mathscr{P}$  est un plan qui contient $d$ et  $\mathscr{P}'$ un plan qui contient $d'$ \\ 
 $\mathscr{P}$  et  $\mathscr{P}'$ sont sécants selon une droite $\Delta$ \\ 
\end{tabular} 
 \right\rbrace  $
alors $\Delta$  est parallèle à $d$ et à $d'$.

\AV{https://www.geogebra.org/m/KQJzHRS3}{Visualisation}
\end{ThT}


\AD{1}{GE-18}




} % Intersections droites et plans
%
%
%\chapter{Géométrie euclidienne}
%%\intro{origami}{De simples pliages se redressent pour donner forme à des êtres, des animaux, des objets. La conception de pliage utilise la géométrie plane dans tous les sens. Les \textbf{triangles} \textit{scalène}, \textit{rectangle}, \textit{isocèle}, \textit{équilatéral} sont présents dans tous les origamis et participent à la création de ces solides nés de feuilles planes. }
%
%\begin{titre}[Géométrie euclidienne]

\Titre{Situations géométriques}{4}
\end{titre}


\begin{CpsCol}
\textbf{Résoudre des problèmes de géométrie dans le plan}
\begin{description}
\item[$\square$] Utiliser le théorème de Thalès, de Pythagore
\item[$\square$] Utiliser les propriétés des symétries axiales ou centrale
\item[$\square$] Calculer des longueurs, des angles, des aires et des volumes.
\end{description}
\end{CpsCol}



\begin{minipage}{0.47\linewidth}

\Exo{1}{GVA-1}

\Exo{1}{GVA-3}

\PO{1}{GVA-6}

\Exo{1}{GVA-4}
\end{minipage}
\hfill
\begin{minipage}{0.47\linewidth}

\Exo{1}{GVA-2}

\Exo{1}{GVA-5}

\Exo{1}{GVA-8}


\PO{1}{GVA-7}
\end{minipage}


\begin{minipage}[t]{0.47\linewidth}
\Exo{1}{GVA-11}

\Exo{1}{GVA-12}
\end{minipage}
\hfill
\begin{minipage}[t]{0.47\linewidth}

\PO{1}{GVA-14}

\begin{ROC}

Démontrer que pour tout réel $\alpha$, $\cos^2(\alpha) + \sin^2(\alpha) = 1$ dans un triangle rectangle.
\end{ROC}

\end{minipage}


 % Transformations du collège
%\impress{\impressionEleve}{\begin{titre}[Géométrie vectorielle et analytique]

\Titre{Projection orthogonale}{3}
\end{titre}

\begin{CpsCol}
\textbf{Démontrer avec les vecteurs}
\begin{description}
\item[$\square$] Résoudre des problèmes de géométrie plane sur des figures simples ou complexes
(triangles, quadrilatères, cercles).
\item[$\square$] Calculer des longueurs, des angles, des aires et des volumes.
\item[$\square$] Traiter de problèmes d’optimisation. 
\end{description}
\end{CpsCol}



\begin{DefT}{Projection orthogonale}\index{Projection orthogonale}
Soit $d$ une droite du plan et $A$ un point extérieur à $d$. La projection orthogonale est l'application du plan qui  au point $A$ associe le point $A'$ de la droite $d$ tel que la droite $(AA')$ est perpendiculaire à $d$ en $A'$. 

\definecolor{xdxdff}{rgb}{0.49019607843137253,0.49019607843137253,1.}
\definecolor{ududff}{rgb}{0.30196078431372547,0.30196078431372547,1.}
\begin{tikzpicture}[line cap=round,line join=round,>=triangle 45,x=0.5cm,y=0.5cm]
\clip(0.831296100794201,-0.7348563902439222) rectangle (17.486323662311374,5.80145736585363);
\draw[line width=2.pt,fill=black,fill opacity=0.10000000149011612] (9.190921775722733,1.1160512934995357) -- (9.06194825746555,1.7700717388119918) -- (8.407927812153094,1.6410982205548086) -- (8.536901330410277,0.9870777752423525) -- cycle; 
\draw [line width=2.pt,domain=0.831296100794201:17.486323662311374] plot(\x,{(-17.201085830991378--4.87081073170731*\x)/24.699720119533016});
\draw [line width=2.pt,dash pattern=on 3pt off 3pt,domain=0.831296100794201:17.486323662311374] plot(\x,{(-215.6669425698831--24.699720119533016*\x)/-4.87081073170731});
\draw (2.119703893288926,1.0563449756097338) node[anchor=north west] {$d$};
\begin{scriptsize}
\draw [fill=ududff] (-5.453619960155669,-1.7718677073170916) circle (2.5pt);
\draw[color=ududff] (-5.155086447260557,-1.1119514146341656) node {$A_1$};
\draw [fill=ududff] (19.246100159377345,3.098943024390219) circle (2.5pt);
\draw[color=ududff] (19.466072221510586,3.6802978536585105) node {$B$};
\draw[color=black] (-13.43546335756202,-2.793166731707334) node {$f$};
\draw [color=xdxdff] (8.536901330410277,0.9870777752423525)-- ++(-2.5pt,0 pt) -- ++(5.0pt,0 pt) ++(-2.5pt,-2.5pt) -- ++(0 pt,5.0pt);
\draw[color=xdxdff] (8.02752499058181,0.5378393170731492) node {$A'$};
\draw[color=black] (6.267748493515845,10.405159121951185) node {$g$};
\draw [color=xdxdff] (7.748641917797854,4.984315185099545)-- ++(-2.5pt,0 pt) -- ++(5.0pt,0 pt) ++(-2.5pt,-2.5pt) -- ++(0 pt,5.0pt);
\draw[color=xdxdff] (7.964675829972312,5.565772975609728) node {$A$};
\end{scriptsize}
\end{tikzpicture}


\end{DefT}



\begin{Rq}
Tout point du plan a un unique projeté orthogonal sur une droite $d$.

Tout point $M$ de la droite $d$ est le projeté orthogonal d'une infinité de point du plan qui appartiennent à la droite passant par $M$ et perpendiculaire à $M$.

\definecolor{xdxdff}{rgb}{0.49019607843137253,0.49019607843137253,1.}
\definecolor{ududff}{rgb}{0.30196078431372547,0.30196078431372547,1.}
\begin{tikzpicture}[line cap=round,line join=round,>=triangle 45,x=0.5cm,y=0.5cm]
\clip(2.2454022145079233,-3.468795317073187) rectangle (14.406714792445936,7.749781658536554);
\draw[line width=2.pt,fill=black,fill opacity=0.10000000149011612] (9.190921775722733,1.1160512934995357) -- (9.06194825746555,1.7700717388119918) -- (8.407927812153094,1.6410982205548086) -- (8.536901330410277,0.9870777752423525) -- cycle; 
\draw [line width=2.pt,domain=2.2454022145079233:14.406714792445936] plot(\x,{(-17.201085830991378--4.87081073170731*\x)/24.699720119533016});
\draw [line width=2.pt,dash pattern=on 6pt off 6pt,domain=2.2454022145079233:14.406714792445936] plot(\x,{(-215.6669425698831--24.699720119533016*\x)/-4.87081073170731});
\draw (2.119703893288926,1.0563449756097338) node[anchor=north west] {$d$};
\begin{scriptsize}
\draw [fill=ududff] (-5.453619960155669,-1.7718677073170916) circle (2.5pt);
\draw[color=ududff] (-5.155086447260557,-1.1119514146341656) node {$A_1$};
\draw [fill=ududff] (19.246100159377345,3.098943024390219) circle (2.5pt);
\draw[color=ududff] (19.466072221510586,3.6802978536585105) node {$B$};
\draw[color=black] (-13.43546335756202,-2.793166731707334) node {$f$};
\draw [color=xdxdff] (8.536901330410277,0.9870777752423525)-- ++(-2.5pt,0 pt) -- ++(5.0pt,0 pt) ++(-2.5pt,-2.5pt) -- ++(0 pt,5.0pt);
\draw[color=xdxdff] (7.9332512496675625,0.5378393170731492) node {$M$};
\draw[color=black] (6.267748493515845,10.405159121951185) node {$g$};
\draw [color=xdxdff] (7.748641917797854,4.984315185099545)-- ++(-2.5pt,0 pt) -- ++(5.0pt,0 pt) ++(-2.5pt,-2.5pt) -- ++(0 pt,5.0pt);
\draw[color=xdxdff] (7.964675829972312,5.565772975609728) node {$A$};
\draw [color=xdxdff] (9.048175607643872,-1.6055771743290161)-- ++(-2.5pt,0 pt) -- ++(5.0pt,0 pt) ++(-2.5pt,-2.5pt) -- ++(0 pt,5.0pt);
\draw[color=xdxdff] (9.253083622467036,-1.0333899512195315) node {$C$};
\draw [color=xdxdff] (9.229394525056087,-2.524532310751975)-- ++(-2.5pt,0 pt) -- ++(5.0pt,0 pt) ++(-2.5pt,-2.5pt) -- ++(0 pt,5.0pt);
\draw[color=xdxdff] (9.441631104295533,-1.9447029268292864) node {$D$};
\draw [color=xdxdff] (7.474252237411067,6.375736178964803)-- ++(-2.5pt,0 pt) -- ++(5.0pt,0 pt) ++(-2.5pt,-2.5pt) -- ++(0 pt,5.0pt);
\draw[color=xdxdff] (7.6818546072295675,6.948454731707288) node {$E$};
\end{scriptsize}
\end{tikzpicture}

\end{Rq}

\EPCG{1}{GVA-63}{Représenter. Calculer.}


\EPCG{1}{GVA-64}{Raisonner. Représenter. Calculer.}


\EPCG{1}{GVA-59}{Représenter. Calculer.}



\EPCG{1}{GVA-59}{Représenter}

 

\EPCG{1}{GVA-60}{Représenter}

\App{1}{GVA-13}
 

\EPCG{1}{GVA-61}{Modéliser. Représenter. Calculer.}


\begin{ROC}

Le projeté orthogonal du point M sur une droite $\Delta$ est le point de la droite $\Delta$ le plus
proche du point M.
\end{ROC}


\begin{Approfondissement}

\begin{enumerate}
\item Démontrer que les trois médiatrices d'un triangle quelconque ABC sont concourantes.
\item Démontrer que les trois hauteurs d'un triangle quelconque ABC sont concourantes.
\end{enumerate}

\end{Approfondissement}


 } % Projection orthogonale.
%
%
%\chapter{Géométrie vectorielle et analytique}
%%\impress{\impressionEleve}{\begin{titre}[Géométrie vectorielle et analytique]

\Titre{Repérage}{1}
\end{titre}

\begin{CpsCol}
\textbf{Se repérer dans le plan}
\begin{description}
\item[$\square$] Repérer un point dans le plan
\end{description}
\end{CpsCol}


\Rec{1}{GVA-21}

\begin{Rq}
Il existe plusieurs systèmes de repérages selon l'utilité et le travail demandé : 
\begin{description}
 \item[•] repère cartésien
 \item[•] repère polaire
 \item[•] repère sphérique
 \item[•] repère équatorial
 \end{description} 
\end{Rq}


\begin{DefT}{Repère cartésien}\index{Repère cartésien}
On appelle \textbf{repère cartésien du plan} deux axes gradués \textbf{orientés et sécants} en un point nommé l'origine du repère.
\end{DefT}


\begin{DefT}{Repère orthonormé}\index{Repère orthonormé}
Un repère est dit \textbf{orthonormé} lorsque les axes sont perpendiculaires et les unités sur les axes égales à 1.
\end{DefT}} % Repérage plan
%
%%%%%%%%%%%%%%%%%%%%%%%%%%%%%%%%%%%%%%%%%%%%%%%%   A décommenter une fois 
%\impress{\impressionEleve}{\begin{titre}[Géométrie vectorielle et analytique]

\Titre{Notions de vecteur}{4}
\end{titre}

\begin{CpsCol}
\textbf{Connaitre la notion de vecteur}
\begin{description}
\item[$\square$] Représenter géométriquement des vecteurs.
\item[$\square$] Représenter un vecteur dont on connaît les coordonnées. Lire les coordonnées d’un
vecteur.
\end{description}
\end{CpsCol}



\begin{DefT}{Vecteur}\index{Vecteurs}
On dit que le vecteur $\overrightarrow{AB}$ est le représentant du déplacement de A vers B. Le vecteur $\overrightarrow{AB}$ possède donc une \textbf{direction}, un \textbf{sens}, une longueur appelée \textbf{norme}. 
Ces 3 données caractérisent un vecteur.

La translation de vecteur $\overrightarrow{AB}$ est le déplacement de A vers B.
\end{DefT}


 
 
\begin{ThT}{Vecteurs égaux} \index{Vecteurs!Égaux} 
\begin{minipage}{0.48\linewidth}
Deux vecteurs $\overrightarrow{AB}$ et $\overrightarrow{CD}$ sont égaux lorsque ils ont la même direction (leurs supports sont parallèles), le même sens, la même norme $(AB=CD)$.

On écrit $\overrightarrow{AB}=\overrightarrow{CD}$. 

Le quadrilatère $ABDC$ est donc un parallélogramme.
\end{minipage}
\hfill
\begin{minipage}{0.48\linewidth}
\definecolor{ffqqqq}{rgb}{1.,0.,0.}
\definecolor{qqqqff}{rgb}{0.,0.,1.}
\begin{tikzpicture}[line cap=round,line join=round,>=triangle 45,x=1.0cm,y=1.0cm]
\clip(-2.46,-0.22) rectangle (4.58,2.42);
\draw [->,color=ffqqqq] (-2.,0.) -- (1.,2.);
\draw [color=ffqqqq](-0.82,1.46) node[anchor=north west] {$\vec{u}$};
\draw [->,color=ffqqqq] (1.,0.) -- (4.,2.);
\draw [color=ffqqqq](2.06,1.38) node[anchor=north west] {$\vec{v}$};
\begin{scriptsize}
\draw [fill=qqqqff] (-2.,0.) circle (0.5pt);
\draw[color=qqqqff] (-2.24,0.) node {$A$};
\draw [fill=qqqqff] (1.,2.) circle (0.5pt);
\draw[color=qqqqff] (1.14,2.2) node {$B$};
\draw [fill=qqqqff] (1.,0.) circle (0.5pt);
\draw[color=qqqqff] (0.7,-0) node {$C$};
\draw [fill=qqqqff] (4.,2.) circle (0.5pt);
\draw[color=qqqqff] (4.14,2.2) node {$D$};
\end{scriptsize}
\end{tikzpicture}
\end{minipage}
\end{ThT}


\begin{DefT}{Vecteur nul}\index{Vecteurs!Nul}
On dit que le vecteur $\overrightarrow{AB}$ est nul lorsque $A$ et $B$ sont confondus. On note $\overrightarrow{AB}=\overrightarrow{0}$
\end{DefT}




\mini{

\EPC{1}{GVA-32}{Représenter. Raisonner.}

\EPC{1}{GVA-33}{Représenter. Raisonner.}

\EPC{1}{GVA-35}{Représenter. Raisonner.}
}{
\EPC{1}{GVA-34}{Représenter. Calculer.}

\begin{DefT}{Vecteurs opposés}\index{Vecteurs!Opposés}
Deux vecteurs sont dits opposés lorsqu'ils ont la même direction, la même norme et sont de sens opposés. $\overrightarrow{AB}$ et  $\overrightarrow{BA}$ sont opposés. On écrit : $\overrightarrow{AB} = -\overrightarrow{BA}$.
\end{DefT}
}


\EPC{1}{GVA-40}{Représenter}

\begin{DefT}{Base et repère du plan}\index{Base du plan}\index{Repère du plan}
Deux vecteurs non colinéaires forment une \textbf{base} du plan. On le note $\left(\vec{u};\vec{v}\right)$.  \\ 
Deux vecteurs non colinéaires et un point du plan forment un \textbf{repère} du plan. On la note \Oij. Un repère permet de repérer des points dans le plan. Les coordonnées des points sont dépendantes de l'origine $O$.
\end{DefT}




\begin{ThT}{Coordonnées de vecteurs}\index{Vecteurs!Coordonnées}

Deux vecteurs égaux ont des coordonnées égales.
$\overrightarrow{AB}(x,y)$ et $\overrightarrow{CD}(x',y')$.

$\overrightarrow{AB}=\overrightarrow{CD} \Longleftrightarrow x=x'$ et $y=y'$.

Dans un repère on donne deux points $A(x_A; y_A)$ et $B(x_B;y_B)$. Les coordonnées du vecteur $\overrightarrow{AB}$ sont $(x_B – x_A; y_B – y_A)$.
\end{ThT}


\mini{
\EPC{1}{GVA-37}{Représenter}

\EPC{0}{GVA-46}{Représenter. Calculer}
}{
\EPC{1}{GVA-44}{Représenter}

\EPC{0}{GVA-36}{Représenter. Calculer}
}
} % Vecteurs, égalité, Chasles, construction
%\impress{\impressionEleve}{\begin{titre}[Géométrie vectorielle et analytique]

\Titre{Milieu de segment}{1}
\end{titre}


\begin{CpsCol}
\textbf{Utiliser des nombres pour calculer et résoudre des problèmes}
\begin{description}
\item[$\square$] Calculer les coordonnées du milieu d'un segment
\end{description}
\end{CpsCol}


\Rec{1}{GVA-10}



\begin{Th} \index{Coordonnées du milieu}
\Oij est un repère du plan, $A$ et $B$ deux points de coordonnées $(x_A;y_A)$ et  $(x_B;y_B)$. 

Alors le milieu $M$ du segment $[AB]$ a pour coordonnées $$x_M=\frac{x_A+x_B}{2} ~~ et ~~ y_M=\frac{y_A+y_B}{2}$$

\end{Th}

\mini{
\AD{1}{GVA-15}

\Exo{1}{GVA-20}
}{
\EPC{0}{GVA-18}{Calculer.}
}} % Milieu
%\impress{\impressionEleve}{\begin{titre}[Géométrie vectorielle et analytique]

\Titre{Distance entre deux points}{2}
\end{titre}

\begin{CpsCol}
\textbf{Utiliser des nombres pour calculer et résoudre des problèmes}
\begin{description}
\item[$\square$] Calculer la distance entre deux points connaissant les coordonnées des points
\end{description}
\end{CpsCol}



\begin{Th}\index{Distance entre deux points}
\Oij est un repère orthonormal du plan, $A$ et $B$ deux points de coordonnées $(x_A;y_A)$ et  $(x_B;y_B)$. 

Alors la distance $AB$, la longueur du segment $[AB]$, est donnée par  $$d(A,B)=\sqrt{(x_B-x_A)^2+(y_B-y_A)^2}$$
\end{Th}

\mini{
\EPC{1}{GVA-16}{Calculer.}

\EPC{1}{GVA-4}{Raisonner. Représenter. Calculer.}

\EPC{0}{GVA-54}{Raisonner. Représenter. Calculer.}

\EPC{0}{GVA-17}{Raisonner. Représenter. Calculer.}

}{

\EPC{1}{GVA-56}{Modéliser. Représenter. Calculer.}

\EPC{0}{GVA-57}{Raisonner. Calculer.}

\EPC{1}{GVA-19}{Raisonner. Représenter. Calculer.}
}
\newpage
\EPCP{1}{GVA-0}{Raisonner.} } % Distance et coordonnées
%\impress{\impressionEleve}{\begin{titre}[Géométrie vectorielle et analytique]

\Titre{Opérations de vecteur}{4}
\end{titre}

\begin{CpsCol}
\textbf{Calculer avec les vecteurs}
\begin{description}
\item[$\square$] Calculer les coordonnées d’une somme de vecteurs, d’un produit d’un vecteur par un
nombre réel.
\item[$\square$] Multiplier un vecteur par un réel
\end{description}
\end{CpsCol}




\begin{DefT}{Somme de deux vecteurs}\index{Vecteurs!Somme}
\begin{minipage}{0.48\linewidth}
Soit $\overrightarrow{u}$ et $\overrightarrow{v}$ deux vecteurs quelconques.

On appelle somme de vecteurs $\overrightarrow{u}$ et $\overrightarrow{v}$, notée $\overrightarrow{u}+\overrightarrow{v}$, le vecteurs $\overrightarrow{w}$ qui résulte de la translation suivant $\overrightarrow{u}$ puis de la translation suivant $\overrightarrow{v}$.
\end{minipage}
\hfill
\begin{minipage}{0.48\linewidth}
\definecolor{qqqqff}{rgb}{0.,0.,1.}
\definecolor{ffqqqq}{rgb}{1.,0.,0.}
\begin{tikzpicture}[line cap=round,line join=round,>=triangle 45,x=1.0cm,y=1.0cm]
\clip(-2.08,-1.32) rectangle (4.12,2.3);
\draw [->,color=ffqqqq] (-2.,-1.) -- (1.,2.);
\draw [->,color=qqqqff] (1.,2.) -- (4.,0.);
\draw [->] (-2.,-1.) -- (4.,0.);
\draw [color=ffqqqq](-0.82,1.26) node[anchor=north west] {$\vec{u}$};
\draw [color=qqqqff](2.64,1.76) node[anchor=north west] {$\vec{v}$};
\draw (0.96,-0.36) node[anchor=north west] {$\vec{w}$};
\end{tikzpicture}
\end{minipage}
\end{DefT}


\begin{Nt}
$\overrightarrow{w}=\overrightarrow{u}+\overrightarrow{v}$
\end{Nt}





\begin{DefT}{Relation de Chasles}\index{Relation de Chasles}
Soit $A$, $B$ et $C$trois points quelconques.

$\overrightarrow{AC}=\overrightarrow{AB}+\overrightarrow{BC}$
\end{DefT}

\mini{
\AD{1}{GVA-22}
}{
\EPC{0}{GVA-50}{Raisonner. Calculer}
}

\begin{Th}
Soit $A$, $B$ et $C$trois points quelconques.\\
Le quadrilatère $ABDC$ est un parallélogramme si et seulement si $\overrightarrow{AD}=\overrightarrow{AB}+\overrightarrow{AC}$.
\end{Th}

\ROC

\begin{Th}

Soit $\overrightarrow{u}$ de coordonnées $(x,y)$ et $\overrightarrow{v}$ de coordonnées $(x',y')$.

Le vecteur  $\overrightarrow{u}+\overrightarrow{v}$ a pour coordonnées $(x+x',y+y')$.
\end{Th}


\mini{
\EPC{1}{GVA-31}{Raisonner. Calculer}

\EPC{1}{GVA-47}{Raisonner. Calculer}
}{
\EPC{1}{GVA-49}{Calculer}
}


\begin{DefT}{Différence de deux vecteurs}\index{Vecteurs!Différence}
\begin{minipage}{0.48\linewidth}
Soit $\overrightarrow{u}$ et $\overrightarrow{v}$ deux vecteurs quelconques.

On appelle \textbf{différence de vecteurs} $\overrightarrow{u}$ et $\overrightarrow{v}$, notée $\overrightarrow{u}-\overrightarrow{v}$, le vecteurs $\overrightarrow{w}$ qui résulte de la translation suivant $\overrightarrow{u}$ puis de la translation suivant $-\overrightarrow{v}$.

\end{minipage}
\hfill
\begin{minipage}{0.48\linewidth}
\definecolor{qqqqff}{rgb}{0.,0.,1.}
\definecolor{qqwuqq}{rgb}{0.,0.39215686274509803,0.}
\definecolor{ffqqqq}{rgb}{1.,0.,0.}
\begin{tikzpicture}[line cap=round,line join=round,>=triangle 45,x=1.0cm,y=1.0cm]
\clip(-2.3,-2.3) rectangle (6.24,2.2);
\draw [->,color=ffqqqq] (-2.,-1.) -- (1.,2.);
\draw [color=ffqqqq](0.42,0.94) node[anchor=north west] {$\vec{u}$};
\draw [->,color=qqwuqq] (6.,1.) -- (4.,2.);
\draw [color=qqwuqq](5.06,2.18) node[anchor=north west] {$\vec{v}$};
\draw [->,color=ffqqqq] (-1.,-2.) -- (2.,1.);
\draw [->,color=qqwuqq] (2.,1.) -- (4.,0.);
\draw [color=qqwuqq](2.64,1.26) node[anchor=north west] {$-\vec{v}$};
\draw [->,color=qqqqff] (-1.,-2.) -- (4.,0.);
\draw [color=qqqqff](1.52,-0.78) node[anchor=north west] {$\vec{u}-\vec{v}$};
\end{tikzpicture}
\end{minipage}
\end{DefT}

\begin{Th}

Soit $\overrightarrow{u}$ de coordonnées $(x,y)$ et $\overrightarrow{v}$ de coordonnées $(x',y')$.

Le vecteur  $\overrightarrow{u}-\overrightarrow{v}$ a pour coordonnées $(x-x',y-y')$.
\end{Th}

\mini{
\AD{1}{GVA-38}
}{
\Exo{1}{GVA-52}

\PO{1}{GVA-41}
}

\begin{DefT}{Produit d'un vecteur par un réel}\index{Vecteurs!Produit d'un vecteur par un réel}
\begin{minipage}{0.48\linewidth}
Soit $\overrightarrow{u}$ un vecteur quelconque et $k$ un réel non nul.

On appelle \textbf{produit du vecteur} $\overrightarrow{u}$ par le réel $k$, le vecteur $k\overrightarrow{u}$ :
\begin{description}
\item[•] de même direction que $\overrightarrow{u}$
\item[•] 
\begin{description}
\item[•] de même sens que $\overrightarrow{u}$ si $k$ est positif
\item[•] de sens opposé à $\overrightarrow{u}$ si $k$ est négatif
\end{description}
\item[•] de norme
\begin{description}
\item[•] $k \times \Vert \overrightarrow{u} \Vert$ si $k$ est positif
\item[•] $-k \times \Vert \overrightarrow{u} \Vert$ si $k$ est négatif
\end{description}

\end{description}

\end{minipage}
\hfill
\begin{minipage}{0.48\linewidth}

\definecolor{ffxfqq}{rgb}{1.,0.4980392156862745,0.}
\definecolor{qqqqff}{rgb}{0.,0.,1.}
\definecolor{ffqqqq}{rgb}{1.,0.,0.}
\begin{tikzpicture}[line cap=round,line join=round,>=triangle 45,x=1.0cm,y=1.0cm]
\clip(-2.1,-1.26) rectangle (4.52,2.16);
\draw [->,color=ffqqqq] (-2.,-1.) -- (1.,2.);
\draw [color=ffqqqq](0.08,1.24) node[anchor=north west] {$\vec{u}$};
\draw [->,color=qqqqff] (1.,0.) -- (2.,1.);
\draw [->,color=ffxfqq] (4.,1.) -- (2.,-1.);
\draw [color=qqqqff](0.34,0.24) node[anchor=north west] {$k>0$};
\draw [color=ffxfqq](2.64,-0.38) node[anchor=north west] {$k<0$};
\draw (2.46,1.7) node[anchor=north west] {$k\vec{u}$};
\end{tikzpicture}

\end{minipage}
\end{DefT}


\begin{Rq}
$k=0$ ou $\vec{u}=\vec{0}$ si et seulement $k\vec{u}=\vec{0}$
\end{Rq}

\mini{
\AD{1}{GVA-23}

\Exo{1}{GVA-48}
}{
\Exo{1}{GVA-24}
}

\begin{Th}
Soit $\overrightarrow{AB}$ de coordonnées $(x,y)$ et $k$ un réel.

Le vecteur  $k\overrightarrow{AB}$ a pour coordonnées $(kx,ky)$.
\end{Th}


\mini{
\AD{1}{GVA-25}
}{
\AD{1}{GVA-38}
}

} % Somme de vecteurs, Multiplication d'un vecteur par un réel
%%%########## \impress{\impressionEleve}{\begin{titre}[Géométrie vectorielle et analytique]

\Titre{Opérations de vecteur}{6}
\end{titre}

 

 
\AD{1}{GVA-22_cor}
 
\AD{1}{GVA-50_cor}
 

 
\AD{1}{GVA-31_cor}

\AD{1}{GVA-47_cor}
 
\AD{1}{GVA-49_cor}
 



 
\AD{1}{GVA-38_cor}
 
\Exo{1}{GVA-52_cor}

\PO{1}{GVA-41_cor}
 


\newpage
 
\AD{1}{GVA-23_cor}

\newpage

\Exo{1}{GVA-48_cor}

 \newpage
 
\Exo{1}{GVA-24_cor}
 

\newpage
 
\AD{1}{GVA-25_cor}



}
%\impress{\impressionEleve}{\begin{titre}[Géométrie vectorielle et analytique]

\Titre{Colinéarité de vecteurs}{4}
\end{titre}

\begin{CpsCol}
\textbf{Démontrer avec les vecteurs}
\begin{description}
\item[$\square$] Caractériser alignement et parallélisme par la colinéarité de vecteurs.
\end{description}
\end{CpsCol}



\begin{DefT}{Colinéarité de 2 vecteurs}\index{Vecteurs!Colinéarité}
Deux vecteurs $\overrightarrow{u}$ et $\overrightarrow{v}$ sont dits colinéaires lorsqu'ils ont la même direction, c'est à ire lorsqu'il existe un réel $k$ tel que $\overrightarrow{u}=k\overrightarrow{v}$.
\end{DefT}



\begin{Rq}
Le vecteur nul est colinéaire à tout autre vecteur du plan.
\end{Rq}


\begin{Th}
\begin{enumerate}
\item $A$, $B$, $C$ et $D$ sont 4 points distincts du plan, les droites $(AB)$ et $(CD)$ sont parallèles lorsque les vecteurs $\overrightarrow{AB}$ et $\overrightarrow{CD}$ sont colinéaires.
\item $A$, $B$ et $C$ sont 3 points distincts du plan  sont alignés lorsque les vecteurs $\overrightarrow{AB}$ et $\overrightarrow{AC}$ sont colinéaires.
\end{enumerate}
\end{Th}

\begin{ThT}{Condition de colinéarité}\index{Vecteurs!Condition de colinéarité}
Soit $\overrightarrow{u}$ de coordonnées $(x,y)$ et $\overrightarrow{v}$ de coordonnées $(x',y')$.

Les vecteurs  $\overrightarrow{u}$ et $\overrightarrow{v}$sont colinéaires si et seulement si $xy'-x'y=0$.

Le nombre réel $xy'-x'y$ est appelé \textbf{déterminant} de $\overrightarrow{u}$ et de $\overrightarrow{v}$. On note : $\text{dét}\left(\overrightarrow{u},\overrightarrow{v}\right)= xy'-x'y$.
\end{ThT}

\begin{Nt}
$\text{dét}\left(\overrightarrow{u},\overrightarrow{v}\right)=$ \begin{tabular}{|cc|}
$x$ & $x'$ \\  
$y$ & $y'$ \\ 
\end{tabular} $=xy'-x'y$
\end{Nt}

\Rec{1}{GVA-39}

\mini{
\EPC{1}{GVA-26}{Représenter. Calculer}

\EPC{0}{GVA-28}{Représenter. Calculer}

\EPC{1}{GVA-29}{Représenter. Calculer}

\EPC{0}{GVA-42}{Représenter. Calculer}
}{
\EPC{1}{GVA-27}{Représenter. Calculer}

\EPC{0}{GVA-30}{Représenter. Calculer}

\EPC{1}{GVA-43}{Représenter. Calculer}

\EPC{0}{GVA-51}{Représenter. Calculer}
}

\EPCP{1}{GVA-58}{Modéliser. Calculer}} % Colinéarité
% 
%
%\chapter{Équations de droite}
%%%##########\intro{origami}{De simples pliages se redressent pour donner forme à des êtres, des animaux, des objets. La conception de pliage utilise la géométrie plane dans tous les sens. Les \textbf{triangles} \textit{scalène}, \textit{rectangle}, \textit{isocèle}, \textit{équilatéral} sont présents dans tous les origamis et participent à la création de ces solides nés de feuilles planes. }
%
%\impress{\impressionEleve}{\input{CHAPITRES/ED-F0}} % Caractérisation
%\impress{\impressionEleve}{\begin{titre}[Géométrie vectorielle et analytique]

\Titre{Droites parallèles, droites sécantes}{4}
\end{titre}



\begin{CpsCol}
\textbf{Résoudre des problèmes de géométrie dans le plan}
\begin{description}
\item[$\square$] Déterminer si deux droites sont parallèles ou sécantes.
\item[$\square$] Résoudre un système de deux équations linéaires à deux inconnues, déterminer le point d'intersection de deux droites sécantes.
\end{description}
\end{CpsCol}





\begin{Th}\index{Droites!Parallèles}
Dans un repère, deux droites $d$ et $d_1$ d'équation respectives $y=mxp+p$ et $y=m'x+p'$ sont parallèles si et seulement si leurs coefficients directeurs sont égaux, $m=m'$.
\end{Th}


\begin{Rq}
Deux droites sont sécantes si et seulement si leurs coefficients directeurs sont différents.
\end{Rq}


\begin{minipage}{0.47\linewidth}

\EPC{0}{ED-4}{Calculer.}

\EPC{1}{ED-5}{Calculer.}

\EPC{1}{ED-6}{Calculer.}

\EPC{0}{ED-17}{Modéliser. Calculer.}
\end{minipage}
\hfill
\begin{minipage}{0.47\linewidth}


\EPC{1}{ED-7}{Représenter. Calculer.}

\EPC{0}{ED-8}{Représenter. Calculer.}
\end{minipage}



} % Alignement, parallélisme, intersection.
%%%%%%%%%%%%%%%%%%%%%%%%%%%%%%%%%%%%%%%%%%%%%%%%   
%
%%\chapter{Trigonométrie}
%%\intro{origami}{De simples pliages se redressent pour donner forme à des êtres, des animaux, des objets. La conception de pliage utilise la géométrie plane dans tous les sens. Les \textbf{triangles} \textit{scalène}, \textit{rectangle}, \textit{isocèle}, \textit{équilatéral} sont présents dans tous les origamis et participent à la création de ces solides nés de feuilles planes. }
%%
%%##########\impress{\impressionEleve}{\begin{titre}[Trigonométrie]

\Titre{Le cercle trigonométrique}{4}
\end{titre}


\begin{CpsCol}
\textbf{Enroulement autour du cercle}
\begin{description}
\item[$\square$] Connaitre le cercle trigonométrique
\item[$\square$] Faire le lien entre radian et degré
\end{description}
\end{CpsCol}


\Rec{1}{Trig-4}




\begin{DefT}{Cercle trigonométrique}\index{Cercle trigonométrique}
On appelle cercle trigonométrique tout cercle
\begin{description}
\item[•] de rayon 1,
\item[•] orienté dans le sens opposé aux aiguilles d'une montre,
\end{description} 
Un tel cercle est dit orienté dans le sens direct.\index{Sens direct}
\end{DefT}




\begin{DefT}{Enroulement de la droite}\index{Enroulement de la droite}
On obtient les correspondances suivantes.

\begin{tabular}{|c|c|c|c|c|c|c|c|}
\hline 
Abscisse du point $N$ & $-\frac{\pi}{2}$ & $0$ & $\frac{\pi}{2}$ & $\pi$ & $\frac{\pi}{6}$ & $\frac{\pi}{4}$ & $\frac{\pi}{3}$ \\ 
\hline 
Mesure de l'angle $\widehat{AOM}$ & $-90$ & 0 & 90 & 180 & 30 & 45 & 60 \\ 
\hline 
\end{tabular} 
\end{DefT}




\mini{
\Exo{1}{Trig-0}

\Exo{1}{Trig-1}
}{
\Exo{1}{Trig-2}

\Exo{1}{Trig-3}
}

\mini{
\Exo{1}{Trig-5}

\PO{1}{Trig-8}
}{
\Exo{1}{Trig-6}
}



\begin{DefT}{Mesure principale}\index{Mesure principale}
On appelle mesure principale de l'angle $\left(\overrightarrow{OA};\overrightarrow{OM} \right)$ la mesure de cet angle compris entre $]-\pi;\pi]$
\end{DefT}

\Exo{1}{Trig-7}
} % Le cercle trigonométrique
%%##########\impress{\impressionEleve}{Trig-F1}} % Cosinus et sinus
%
%\part{Informatique}
%
%\chapter{Algorithme et Programmation}
%
% 

\begin{titre}[Algorithmique et Python]

\Titre{Affectation de variables}{4}
\end{titre}



%%%%%%%%%%%%%%%%%%%%%%%%%%%%%%%%%%%%%%%%%%%%%%%
%%%%		 Corps du document
%%%%%%%%%%%%%%%%%%%%%%%%%%%%%%%%%%%%%%%%%%%%%%%

Une variable (zone mémoire étiquetée) sert à stocker une information qui peut être sous la forme d’un nombre, une phrase,
une liste de nombres, une liste de mots...

L'affectation des variables dans Python se fait avec le symbole =, dont le nouvel usage, non symétrique, doit être explicité
aux élèves. Ainsi \texttt{a=1} correspond à l'instruction stocker la valeur 1 dans la variable $a$. Autrement dit, la variable
a prend la valeur 1, ce que l'on écrit de façon synthétique $a \longleftarrow 1$.

Le nom d'une variable ne peut ni être un nombre, ni certains mots réservés (commandes Python, qui prennent une coloration
différente quand on les écrit, comme \texttt{def}, \texttt{pass}, \texttt{lambda}, etc). Il est fortement recommandé de donner des noms explicites à tous les objets que l'on crée.

Il est possible de s'entrainer en ligne sur \texttt{https://repl.it/} après une inscription gratuite.
 
\begin{DefT}{L'affectation}
Pour créer une variable, il suffit de l'écrire. Python gère son type dynamiquement. Pour affecter une valeur à cette variable ($x \longleftarrow a$), on écrit $x=a$. 
\end{DefT} 
 
 
\begin{DefT}{La fonction input}
Il est souvent pratique de donner des valeurs  à calculer, des chaines de caractères à comparer à un programme.

La fonction de  \texttt{ \textbf{input}} permet d'assigner  à une variable une valeur entrée par l'utilisateur.
\end{DefT}
 

\begin{minipage}{0.3\linewidth}

\begin{Cod}
\begin{description}
\item[] \texttt{x= input("Entrer un nombre entier") }
\item[] \texttt{print(x+2)}
\end{description}
\end{Cod}
\end{minipage}
\begin{minipage}{0.3\linewidth}
 
\begin{Cod}
\begin{description}
\item[] \texttt{x=int(input("Entrer un nombre entier"))}
\item[] \texttt{print(x+2)}
\end{description}
\end{Cod}
\end{minipage}
\begin{minipage}{0.3\linewidth}
 
\begin{Cod}
\begin{description}
\item[] \texttt{x= input("Entrer un nombre entier") }
\item[] \texttt{print(x+" 2")}
\end{description}
\end{Cod}
\end{minipage}
 

\begin{Rq}
La fonction \texttt{print} peut parfois avoir une utilisation délicate lorsque qu'on souhaite afficher du texte et des variables dans le même message. Il pratique d'utiliser la fonction \texttt{format}.
\end{Rq}

\begin{Ex}

Il faut écrire la chaine de caractères comme on voudrait la voir afficher et remplacer les variables par \texttt{$\lbrace\rbrace$}. Les variables sont écrites comme paramètres de la fonction \texttt{format}.
\begin{description}
\item[] \texttt{x= int(input("Entrer un nombre entier")) }
\item[] \texttt{y= int(input("Entrer un nombre entier")) }
\item[] \texttt{somme = x + y }
\item[] \texttt{print("$\lbrace\rbrace$ + $\lbrace\rbrace$ = $\lbrace\rbrace$".format(x,y,somme))}
\end{description}
\end{Ex}

\begin{Rq}
Par défaut, la fonction \texttt{input} renvoie une chaine de caractères (\texttt{str}). Pour "forcer" le typage entier à la fonction \texttt{input}, on attribue la fonction \texttt{int} à la fonction input. On peut aussi forcer avec \texttt{float} pour les réels.
\end{Rq}

\begin{ExC}{Simuler la somme obtenue par lancer de 3 dés.}

\texttt{import random}
 
\texttt{de1 = random.randint(1,6)}

\texttt{de2 = random.randint(1,6)}

\texttt{de3 = random.randint(1,6)}

\texttt{somme = de1 + de2 + de3}
 
\texttt{print(somme)}

\end{ExC}

\begin{Rq}
La bibliothèque \texttt{\textbf{random}} propose de la création de nombres aléatoires.
\end{Rq}






\begin{Rq}
La bibliothèque NumPy (http://www.numpy.org/) permet d’effectuer des calculs numériques avec Python. Elle introduit une gestion facilitée des tableaux de nombres.

Il faut au départ importer le package numpy avec l’instruction suivante : \texttt{import numpy as np}

Les fonctions \texttt{fct} numpy seront appelées par \texttt{np.fct}
\end{Rq}


\begin{Cod}
 \lstinputlisting{statistiques.py}
\end{Cod}


 {\Large Les types de variable} 


\paragraphe {Les entiers : le type  \texttt{int}}


Ils supportent les opérations usuelles (+, $-$,$ \times$, **(exponentiation), \texttt{abs()}(valeur absolue)...), mais aussi // (quotient entier de la division euclidienne), \% (reste de la division euclidienne).
Les priorités opératoires sont conformes aux standards mathématiques.

\paragraphe{Les nombres flottants : le type \texttt{float}}

Ils supportent la plupart des opérations usuelles (y compris la division euclidienne qui demande à être étudiée). Il faut
savoir que l'on peut convertir des nombres (ou autres objets) d'un type à un autre.

\paragraphe{ Les nombres flottants : le type \texttt{bool}}

Ils ne peuvent prendre que deux valeurs : \texttt{False} ou \texttt{True}
\texttt{False} a pour valeur 0 et \texttt{True} a pour valeur 1. On peut donc faire des calculs avec les booléens : \texttt{False} * \texttt{True} donne 0.
Ils sont générés par les opérateurs dits booléens, comme la comparaison (<, >, <=, >=;) le test d’égalité (==), le test
de différence (! =)  qui peuvent être combinés avec les opérateurs logiques \texttt{not}, \texttt{or} et \texttt{and}. Par exemple, \texttt{A or B} est vraie si au moins une des deux propriétés est vraie.

\paragraphe{Les n-uplets : le type \texttt{tuple}}

Le mot "tuple" vient des suffixes anglais, comme n-uplet vient des suffixes français des mots triplet, quadruplet, etc.
Les tuple contiennent des éléments qui peuvent être de type quelconque, éventuellement de types différents. Ils sont
délimités par des parenthèses ( ) et les éléments sont séparés par une virgule.

Chaque élément possède un indice : le premier élément porte l’indice 0, le deuxième porte l’indice 1 ...

On peut repérer un élément en commençant par la fin : le dernier porte l’indice $-1$, l’avant dernier porte l’indice $-2$...

Un tuple est un objet non mutable : on ne peut ni modifier la valeur d’un élément, ni ajouter ou supprimer des éléments.
En revanche, on peut concaténer 2 tuple (mettre bout-à-bout les contenus), compter les occurrences d’un élément, ou le
nombre d’éléments du tuple, tester l’appartenance d’un élément au tuple ...

 \paragraphe{Les chaînes de caractères : le type \texttt{string}}

Une chaîne de caractère est donnée entre guillemets (’simples’, "doubles" ou ”’triples”’). Les caractères peuvent être des
lettres, des nombres, un espace...

Pour définir une chaîne de caractères, on utilise :
\begin{description}
\item[•] soit les apostrophes : ’Il a dit : "bonjour"’
\item[•] soit les guillemets : "SNT, c’est génial!"
\item[•] soit des triples guillemets qui permettent de mettre tout ce qu’on veut dans la chaîne de caractères par exemple
”’SNT, c’est pour tous ;)”’.
\end{description}
  
Toutes les opérations vues sur les tuple sont aussi valables sur les chaînes de caractères (y compris le tri, qui renvoie là
encore une liste de caractères dans l’ordre alphabétique, les signes de ponctuation en premier).



\begin{Rq}
Il est possible de forcer le typage d'une variable, d'une expression avec \texttt{str()} ,  \texttt{int()} 
\end{Rq}



\begin{ExD}

Créer un programme qui demande votre nom, votre prénom et votre age et qui affiche les valeurs renseignées.
\end{ExD}

\begin{ExD}

Déterminer de l’aire d’un triangle avec la formule $A =\frac12 ab \sin \widehat{C}$.
\end{ExD}

\begin{Rqs}
\begin{description}
\item[•] La fonction \texttt{sinus} s'obtient avec la bibliothèque \texttt{numpy}. Elle prend comme paramètre un réel en radian. Il conviendra de convertir la valeur de l'angle de degré en radian.
\item[•] Pour accéder au sinus et à la valeur de $\pi$, on écrit :

\begin{lstlisting}
import numpy as np
....  
np.pi
pn.sin()
\end{lstlisting}
\end{description}

Mais ce n'est pas la seule façon évidement. La bibliothèque \texttt{math} convient aussi.

\end{Rqs}

\begin{ExD}

Calculer les coordonnées d'un vecteur  connaissant les coordonnées de deux points.
\end{ExD}


%%%%%%%%%%%%%%%%%%%%%%%%%%%%%%%%%%%%%%%%%%%%%%%%%%%%%%%%%%%%%%%%%%%%%%%%%%%%%%%%%%%%%%%%%%%%%%%%%%%%%%%%%%%%
%%%%%%%%%%%%%%%%%%%%%%%%%%%%%%%%%%%%%%%%%%%%%%%%%%%%%%%%%%%%%%%%%%%%%%%%%%%%%%%%%%%%%%%%%%%%%%%%%%%%%%%%%%%%
%%%%%%%%%%%%%%%%%%%%%%%%%%%%        Nouvelle page
%%%%%%%%%%%%%%%%%%%%%%%%%%%%%%%%%%%%%%%%%%%%%%%%%%%%%%%%%%%%%%%%%%%%%%%%%%%%%%%%%%%%%%%%%%%%%%%%%%%%%%%%%%%%
%%%%%%%%%%%%%%%%%%%%%%%%%%%%%%%%%%%%%%%%%%%%%%%%%%%%%%%%%%%%%%%%%%%%%%%%%%%%%%%%%%%%%%%%%%%%%%%%%%%%%%%%%%%%

\newpage

\begin{titre}[Algorithmique et Python]

\Titre{Les fonctions 1}{4}
\end{titre}

\begin{DefT}{Syntaxe}
Pour définir une fonction, on utilise le mot clé \texttt{\textbf{def}} suivi du nom à la fonction. On peut lui passer des paramètres mais ce n'est pas obligatoire. 
Dans le cas où une variable est renvoyée, ce code se nomme une fonction et on renvoie la variable avec l'instruction \texttt{\textbf{return}}. Dans le cas où une variable n'est est pas renvoyée, ce code se nomme une procédure.

Il suffit ensuite d'appeler la fonction par son nom dans le script : \texttt{nomdelafonction()}.
\end{DefT}

Le bloc d’instructions (suivi ou non de l’instruction return) s’appelle le corps de la fonction. Il doit être obligatoirement
indenté (c’est à dire "décalé", d’une tabulation, souvent égale à 4 espaces) et la borne de l’indentation marque la borne
de la définition de fonction. L’instruction \texttt{\textbf{return}} (ou \texttt{\textbf{return if ...}}) qui veut dire renvoyer (ou renvoyer si...) est une instruction de sortie de la fonction. Toutes les instructions écrites après ne sont pas prises en compte. Certaines fonctions ne renvoient pas de résultat, servent seulement à effectuer une partie du programme.




\begin{ExC}{Écrire une fonction qui calcule la somme de deux nombres}
 
\begin{lstlisting}
def somme(x,y):
    s=x+y
    return s
 
a=int(input('entre un nombre :'))
b=int(input('entre un nombre :'))    
print(a,'+',b,'=', somme(a,b))
\end{lstlisting}
\end{ExC}

\begin{ExC}{Écrire une fonction qui calcule la moyenne de deux nombres}
 
\begin{lstlisting}
def moyenne(a,b):
    m=(a+b)/2
    return m
 
a=int(input('entre un nombre :'))
b=int(input('entre un nombre :'))    
print(a,'et',b,' ont pour moyenne', moyenne(a,b))
\end{lstlisting}
\end{ExC}

\begin{DefT}{Variable globale et locale}
  Une variable locale est définie dans la fonction. 
    Une variable globale est définie en dehors de toute fonction. Dans l'exemple précédent, s, x et y sont locales, a et  b sont globales.
  \begin{description}
  
  \item[• Règle 1 :]   Une variable locale n'existe que dans la fonction.
  
   \item[• Règle 2 :] Une variable globale peut être utilisée mais ne peut pas être modifiée directement à l'intérieur d'une fonction.
  
   \item[• Règle 3 :] pour pouvoir modifier une variable globale $x$ dans une fonction il suffit d'écrire : global $x$.
\end{description}
\end{DefT}

\begin{ExD} 

Créer une fonction qui calcule l'aire d'un disque et le périmètre du cercle périmètre associé au rayon $r$. Coder ensuite un programme qui propose l'aire et le rayon connaissance $r$.
\end{ExD}



\begin{ExD} 

Créer un programme qui demande 2 nombres et une opération (addition, soustraction, multiplication et division)  à un utilisateur puis qui renvoie le résultat de l'opération. 
\end{ExD}

%%%%%%%%%%%%%%%%%%%%%%%%%%%%%%%%%%%%%%%%%%%%%%%%%%%%%%%%%%%%%%%%%%%%%%%%%%%%%%%%%%%%%%%%%%%%%%%%%%%%%%%%%%%%
%%%%%%%%%%%%%%%%%%%%%%%%%%%%%%%%%%%%%%%%%%%%%%%%%%%%%%%%%%%%%%%%%%%%%%%%%%%%%%%%%%%%%%%%%%%%%%%%%%%%%%%%%%%%
%%%%%%%%%%%%%%%%%%%%%%%%%%%%        Nouvelle page
%%%%%%%%%%%%%%%%%%%%%%%%%%%%%%%%%%%%%%%%%%%%%%%%%%%%%%%%%%%%%%%%%%%%%%%%%%%%%%%%%%%%%%%%%%%%%%%%%%%%%%%%%%%%
%%%%%%%%%%%%%%%%%%%%%%%%%%%%%%%%%%%%%%%%%%%%%%%%%%%%%%%%%%%%%%%%%%%%%%%%%%%%%%%%%%%%%%%%%%%%%%%%%%%%%%%%%%%%
\newpage

\begin{titre}[Algorithmique et Python]

\Titre{Le test}{4}
\end{titre}

 


\begin{DefT}{Test conditionnel}
Un  \textbf{test}   est une instruction qui ouvre le choix parmi deux actions suivant le résultat d'un test.

Sa structure est : \textbf{Si} test vérifié \textbf{alors} Action 1 \textbf{sinon} Action 2.
\end{DefT}



\begin{minipage}[t]{0.31\linewidth}
\begin{Ex}
\begin{description}
\item Donner $x$
\item[Test :] $x<12$
\item[Action 1 :] Bonjour
\item[Action 2 :] Bon après midi
\end{description}
\end{Ex}
\end{minipage}
\hfill
\begin{minipage}[t]{0.31\linewidth}
\begin{Syn}
\texttt{ Donner $x$}

\texttt{ \textbf{Si} $x<12$ \textbf{Alors}}

\texttt{ \hspace{0.5cm}	 Bonjour }

\texttt{ \textbf{Sinon}}

\texttt{\hspace{0.5cm}	Bon après midi }
   	
\texttt{\textbf{FinSi}}

\end{Syn}
\end{minipage}
\hfill
\begin{minipage}[t]{0.31\linewidth}

\begin{Cod}
\begin{description}
\item[] $x$=int(input("nombre ?"))
\item[] if  $x<12$ :
\item[] \hspace{0.5cm} print("Bonjour")
\item[] else :
\item[] \hspace{0.5cm} print("Bon après midi")
\end{description}
\end{Cod}
\end{minipage}


\begin{Rq}
Pour signifier le bloc de test, Python utilise l'indentation. Si l'indentation n'est pas conforme, Python renvoie une erreur. 
Pour chainer plusieurs tests, Python utilise la syntaxe : \texttt{if} ...  \texttt{elif} ... \texttt{else}
\end{Rq}

\begin{minipage}{0.5\linewidth}
\begin{DefT}{le ==}
Pour tester la véracité d'une égalité, on utilise le "==". 
\end{DefT}
\end{minipage}
\begin{minipage}{0.5\linewidth}
\begin{Cod}
\texttt{ x = input("Taper oui ou non ? ")} 

\texttt{ if x == "oui":}

 \hspace{0.5cm} \texttt{print("vous avez tapé oui")}
  
\texttt{ else :}  \texttt{ print("vous avez tapé non") } 
\end{Cod}
\end{minipage}


\begin{ExC}{Soit $I=[a;b]$ un intervalle donné. Le réel $x_0$ appartient-il à $I$ ?}

 

\begin{minipage}{0.42\linewidth}

 
\textbf{Algorithme }

\texttt{ Entrer $a$ et $b$  } 

\texttt{ Entrer $x_0$  }

\texttt{ Si $a<= x_0 $ et $x_0 <= b$}

$ \quad \quad $  Afficher $x_0$ appartient à l'intervalle $I$ 

\texttt{ Sinon}

$ \quad \quad $   Afficher $x_0$ n'appartient pas à l'intervalle $I$ 
 
\end{minipage}
\begin{minipage}{0.7\linewidth}
 
\textbf{Python}

 
\texttt{ a = float(input("Entrer la borne inférieure"))}
 
\texttt{ b = float(input("Entrer la borne supérieure"))}
 
\texttt{ x = float(input("Entrer le nombre"))} 

\texttt{ if a <= x and x <=b :  } 

\hspace{0.4cm}	\texttt{ print(x," appartient à [",a,";",b,"]" ) } 

\texttt{ else :} 

\hspace{0.4cm}	\texttt{ print(x," n'appartient pas à [",a,";",b,"]" ) } 
\end{minipage}

\end{ExC}


\begin{ExD}

Déterminer la nature d'un triangle connaissant les longueurs des 3 cotés.
\end{ExD}

\begin{ExD}

Déterminer l'alignement de 3 points dont on connait les coordonnées.
\end{ExD}

\begin{ExD}

Déterminer une équation de droite passant par deux points dont on connait les coordonnées.
\end{ExD}
%%%%%%%%%%%%%%%%%%%%%%%%%%%%%%%%%%%%%%%%%%%%%%%%%%%%%%%%%%%%%%%%%%%%%%%%%%%%%%%%%%%%%%%%%%%%%%%%%%%%%%%%%%%%
%%%%%%%%%%%%%%%%%%%%%%%%%%%%%%%%%%%%%%%%%%%%%%%%%%%%%%%%%%%%%%%%%%%%%%%%%%%%%%%%%%%%%%%%%%%%%%%%%%%%%%%%%%%%
%%%%%%%%%%%%%%%%%%%%%%%%%%%%        Nouvelle page
%%%%%%%%%%%%%%%%%%%%%%%%%%%%%%%%%%%%%%%%%%%%%%%%%%%%%%%%%%%%%%%%%%%%%%%%%%%%%%%%%%%%%%%%%%%%%%%%%%%%%%%%%%%%
%%%%%%%%%%%%%%%%%%%%%%%%%%%%%%%%%%%%%%%%%%%%%%%%%%%%%%%%%%%%%%%%%%%%%%%%%%%%%%%%%%%%%%%%%%%%%%%%%%%%%%%%%%%%
\newpage

\begin{titre}[Algorithmique et Python]

\Titre{Les boucles}{4}
\end{titre}
 

\begin{DefT}{Boucle finie}
Une boucle finie est une itération dont on connait le nombre de répétition \textit{a priori}. 
\end{DefT}

\begin{ExC}{Calculer la somme des n+1 premiers entiers}
\begin{lstlisting}
def somme(n):
	S= 0
	for i in range (n):
	S += i
	return(S)
\end{lstlisting}
\end{ExC}



 
\begin{minipage}[t]{0.49\linewidth}
L'écriture algorithmique  de $n+1$ itérations
\begin{algobox}
\Pour{$i$}{0}{$n$}
\DebutPour
\Ligne Action
\FinPour
\end{algobox}

\end{minipage}
\hfill\vrule\hfill
\begin{minipage}[t]{0.49\linewidth}
La programmation en Python de $n$ itérations. $i$ varie de 0 à $n$ inclus.
\begin{lstlisting}
for i in range(n+1) :
	action
\end{lstlisting}
\end{minipage}


\begin{minipage}{0.25\linewidth}
\begin{Cod}
\begin{lstlisting}
for i in range (10) : 
    print(3*i)
\end{lstlisting}
\end{Cod}
\end{minipage}
\begin{minipage}{0.28\linewidth}
\begin{Cod}
\begin{lstlisting}
mot="salut"
for caractere in mot : 
   print(caractere)
\end{lstlisting}
\end{Cod}
\end{minipage}
\begin{minipage}{0.25\linewidth}
\begin{Cod}
\begin{lstlisting}
x = 2 
while x < 10 : 
	x = x + 2
	print(x)
\end{lstlisting}
\end{Cod}
\end{minipage}
\begin{minipage}{0.25\linewidth}
\begin{Cod}
\begin{lstlisting}
x = 2
while x < 10 :
	x = x + 2
print(x)
\end{lstlisting}
\end{Cod}
\end{minipage}
 
 
 
 
 
 
 
 

\begin{ExC}{Simuler toutes les possibilités de faces obtenues lors d'un lancer de 3 dés.}
 
\begin{lstlisting}
for i in range (1,7):
    for j in range (1,7):
        for k in range (1,7):
            print(i,'+',j,'+',k,'=',i+j+k)
\end{lstlisting}                
\end{ExC}     
    

 



\begin{ExD}

Construire un tableau de valeurs connaissant une fonction dans un intervalle donné.
\end{ExD}

\begin{ExD}

Simuler 100 lancés d'un dé équilibré cubique et calculer la fréquence d'apparition d'un nombre pair. 
\end{ExD}


 


\begin{ExD}

On lance deux dés équilibrés cubiques et on s'intéresse à la fréquence d'obtenir une somme supérieure à 10. 

Simuler $N$ échantillons de taille $n$ et calculer dans chacun des cas la proportion où l'écart entre $p$ et $f$ est inférieur ou égal à $\frac{1}{n}$.    
\end{ExD}


\begin{ExD}

Enrichir le programme de devinette, de manière à ce que :
\begin{description}
\item[•] le joueur propose des nombres jusqu’à ce qu’il trouve ; le nombre d’essais sera affiché en fin de partie
\item[•] le joueur a un nombre de tentatives limité ; le nombre d’essais sera affiché en fin de partie
\item[•] la durée de la partie est limitée dans le temps.
\end{description}

Les programmes attendus comporteront une fonction (au moins ), par exemple correspondant au jeu "simple" précédemment
programmé. Dans la dernière option, on pourra utiliser la fonction time.time().
\end{ExD}

\begin{Rq}
La fonction \texttt{time.time()} renvoie la durée écoulée, en secondes, depuis une date référence prise comme origine des
temps (appelée "The Epoch"). Trouverez-vous la date de "The Epoch", en n’utilisant que Python ?
\end{Rq}

\newpage

\paragraphe{Boucle avec arrêt}


\begin{minipage}{0.5\linewidth}
\begin{DefT}{Boucle à condition d'arrêt}
Une boucle à condition d'arrêt est une itération qui va s'arrêter dès que la condition n'est plus vraie (\color{orange}True\color{black} passe à \color{orange}False\color{black}). 

On utilise un booléen (voir ci-contre).
\end{DefT}
\end{minipage}
\begin{minipage}{0.5\linewidth}
\begin{DefT}{Booléen}
Un booléen est une variable qui ne prend que 2 valeurs : 
\begin{description}
\item[•] \color{orange}True\color{black} \quad ou \quad \color{orange}False \color{black}.
\item[•] 1 ou 0.
\end{description}
\end{DefT}
\end{minipage}

\begin{Syn}
\begin{minipage}[t]{0.49\linewidth}
\begin{algobox}
\Tantque{b-a<4}
\DebutTantQue
\Ligne action
\FinTantQue
\end{algobox}

\end{minipage}
\hfill\vrule\hfill
\begin{minipage}[t]{0.49\linewidth}
\begin{lstlisting}
While b-a<4:
	action
\end{lstlisting}
\end{minipage}
\end{Syn}

\begin{Att}
La boucle peut devenir infinie et faire "planter" le programme si la condition d'arrêt n'est jamais validée.
\end{Att}


\begin{ExC}{Créer un tableau de valeurs}
 \lstinputlisting{liste.py}
\end{ExC}



\begin{ExC}{Déterminer par balayage un encadrement de $\sqrt{2}$ d’amplitude inférieure ou égale à $10^{-n}$. }

\vspace{0.4cm}

\textbf{Résolution mathématique et algorithmique}

$\sqrt{2}$ est une solution de l'équation $x^2-2=0$. 

\vspace{0.4cm}

\begin{minipage}{0.3\linewidth}

\begin{tabular}{|l|}
\hline 
Algorithme \\ 
\hline 
Entrer le pas  \\ 
$x \leftarrow 0$ \\
Affecter $f(x) \leftarrow x*x-2 $ \\
Tant que $f(x)$ < 0 \\
$ \quad \quad	x \leftarrow x +step $ \\
Afficher  $x ,"< x_0 < ",x + $pas\\
\hline 
\end{tabular} 
  
 
\end{minipage}
\begin{minipage}{0.5\linewidth}
\begin{tabular}{|c|}
\hline 
Python \\ 
\hline 
\begin{lstlisting}
step = round(float(input("Entrer le pas")),2)
print(step)
x = 0
f = x*x - 2
while f < 0 : # Parcours de boucle
    f = x*x - 2
    x += step
print(f, round(x-2*step,2),"< x_0 < ",round(x-step,2) ) 
\end{lstlisting}\\
\hline 
\end{tabular} 
\end{minipage}

\end{ExC}



 
 
 
\begin{ExD} 

Calculer la moyenne arithmétique de $n$ valeurs pondérées données par l'utilisateur.
\end{ExD}




\begin{ExD}

Déterminer la première puissance d’un nombre positif donné supérieure ou inférieure à une valeur donnée.
\end{ExD}



\begin{ExD}

Déterminer les nombres premiers inférieur à 100. 
\end{ExD}

%%%%%%%%%%%%%%%%%%%%%%%%%%%%%%%%%%%%%%%%%%%%%%%%%%%%%%%%%%%%%%%%%%%%%%%%%%%%%%%%%%%%%%%%%%%%%%%%%%%%%%%%%%%%
%%%%%%%%%%%%%%%%%%%%%%%%%%%%%%%%%%%%%%%%%%%%%%%%%%%%%%%%%%%%%%%%%%%%%%%%%%%%%%%%%%%%%%%%%%%%%%%%%%%%%%%%%%%%
%%%%%%%%%%%%%%%%%%%%%%%%%%%%        Nouvelle page
%%%%%%%%%%%%%%%%%%%%%%%%%%%%%%%%%%%%%%%%%%%%%%%%%%%%%%%%%%%%%%%%%%%%%%%%%%%%%%%%%%%%%%%%%%%%%%%%%%%%%%%%%%%%
%%%%%%%%%%%%%%%%%%%%%%%%%%%%%%%%%%%%%%%%%%%%%%%%%%%%%%%%%%%%%%%%%%%%%%%%%%%%%%%%%%%%%%%%%%%%%%%%%%%%%%%%%%%%

\newpage

\begin{titre}[Algorithmique et Python]

\Titre{Les listes}{4}
\end{titre}


\begin{DefT}{Les listes}
Le type liste est un type composé. C'est une suite ordonnée d'objets qui n'ont pas forcément le même type, elle est donc hétérogène. 

Une liste constituée des éléments $e_1$,$e_2$,....,$e_n$  s'écrit [$e_1$,$e_2$,....,$e_n$]. Les $e_i$ peuvent  être des listes eux mêmes ce qui permet par exemple de créer une matrice.

Une liste vide est une liste ne contenant aucun objet, on la note [ ].
\end{DefT}

\begin{tabular}{|c|c|c|}
\hline 
 ALGORITHMIE & PYTHON & Les méthodes existantes des listes \vplus \\ 
\hline 
Longueur(A) & len(A) & Renvoie le nombre d'éléments \vplus\\ 
\hline 
A[i] & A[i] & Renvoie le i-ème élément de la liste. \vplus\\ 
\hline 
A[i] $\longleftarrow$ k & A[i]=k & Le i-ème élément de A prend la valeur k \vplus\\ 
\hline 
Tranche (x,y) & A[x:y] & Tranche de la liste qui commence à l'index x et s'arrête avant l'index y. \vplus\\ 
\hline 
Supprime(k) & del(A[k]) & Supprime la valeur de la liste à l'index k. \vplus \\ 
\hline 
Tri(A) & A.sort() & trie une liste \vplus\\ 
\hline 
Inverse(A) & A.reverse()  & Inverse l'ordre des éléments dans une liste. \vplus\\ 
\hline 
Index(x)& A.index(x) & Donne l'index de x dans la liste.\vplus\\ 
\hline 
Ajouter (k,A) & A.append(k) & Ajoute la valeur k à la fin de la liste A. \vplus\\ 
\hline 
Supprime(k,A) & A.remove(k) & Recherche la 1ere valeur de k dans la liste A et la supprime. \vplus \\ 
\hline 
\end{tabular} 

\begin{ExC}{Créer une liste vide et insérer la suite 10 premiers nombres pairs.}
\begin{description}
\item  \texttt{nbre\_pairs = []} 
\item  \texttt{for i in range(10):}
\item  $\quad \quad $ \texttt{nbre\_pairs.append(2*i)}
\item  \texttt{print(nbre\_pairs)}
\end{description}
\end{ExC}

\begin{ExC}{Construire une liste qui simule un paquet de bonbons M\&M's. Il y a environ 42 bonbons dans un paquet.}

\begin{lstlisting}
import random
n=int(input("Quel est le nombre de bonbons dans le paquet ? "))
couleur=["jaune","bleu","rouge","orange","vert","marron"]
proba_couleur=[0]*6
for i in range (n) :
    alea = random.randint(0,5)
    proba_couleur[alea]=proba_couleur[alea]+1
    
for j in range (6) :
    print("On a alors p(",couleur[j],") = ",proba_couleur[j]/n)
\end{lstlisting}
 \end{ExC}
 

\begin{Rq}
Il est possible de déclarer une liste en utilisant [0]*6 plutot que [0,0,0,0,0,0].
\end{Rq}


\paragraphe{Remarques sur la copie de listes}  

Pour copier une liste, on ne peut pas simplement écrire $L2=L1$, sinon toute modification de l’une
entraîne une modification de l’autre. Les listes $L1$ et $L2$ sont liées car elles pointent vers la même zone mémoire.
 

Une première solution pour effectuer une copie peut être d'utiliser le \textit{slicing} (en indiquant [:]), qui renvoie une nouvelle liste :

\begin{Cod}
\begin{description}
\item[] L1=[1, 2, 3]
\item[] L2=L1[:] 
\item[] L2[3][0]=0
\item[] \texttt{print}(L1 , L2)
\end{description}
\end{Cod}

\begin{Rq} 
On peut aussi définir une nouvelle liste "en compréhension" en remplaçant L2=L1[:] par L2 = [i for i in L1].
\end{Rq}

Ces solutions ne sont satisfaisantes que si l’on ne manipule que des listes de premier niveau ne comportant aucune
sous-liste. Pour contourner cette difficultés des listes imbriquées, il faut utiliser une méthode qui n’est pas dans la bibliothèque standard. On importe le module \texttt{copy}, puis on utilise la méthode \texttt{deepcopy}() :


\begin{Cod}
\begin{description}
\item[] \texttt{import copy}
\item[] L1=[1,2,3,[4,5,6]] 
\item[] L2=\texttt{copy.deepcopy}(L1)
\item[] L2[3][1] = 0
\item[] \texttt{print}(L1 , L2)
\end{description}
\end{Cod}



\begin{ExD} 

Écrire un programme qui demande 4 nombres à un utilisateur et qui les insère dans une liste de 4 nombres puis qui ajoute à la liste le nombre impair suivant du dernier nombre entré et enfin qui classe les nombres par ordre croissant.
\end{ExD}

\begin{ExD} 

Calculer la moyenne arithmétique de $n$ valeurs pondérées données par l'utilisateur.
\end{ExD}

\begin{ExD} 

Calculer la médiane de $n$ valeurs données par l'utilisateur.
\end{ExD}

\newpage
%%%%%%%%%%%%%%%%%%%%%%%%%%%%%%%%%%%%%%%%%%%%%%%%%%%%%%%%%%%%%%%%%%%%%%%%%%%%%%%%%%%%%%%%%%%%%%%%%%%%%%%%%%%%
%%%%%%%%%%%%%%%%%%%%%%%%%%%%%%%%%%%%%%%%%%%%%%%%%%%%%%%%%%%%%%%%%%%%%%%%%%%%%%%%%%%%%%%%%%%%%%%%%%%%%%%%%%%%
%%%%%%%%%%%%%%%%%%%%%%%%%%%%        Nouvelle page
%%%%%%%%%%%%%%%%%%%%%%%%%%%%%%%%%%%%%%%%%%%%%%%%%%%%%%%%%%%%%%%%%%%%%%%%%%%%%%%%%%%%%%%%%%%%%%%%%%%%%%%%%%%%
%%%%%%%%%%%%%%%%%%%%%%%%%%%%%%%%%%%%%%%%%%%%%%%%%%%%%%%%%%%%%%%%%%%%%%%%%%%%%%%%%%%%%%%%%%%%%%%%%%%%%%%%%%%%



\begin{titre}[Algorithmique et Python]

\Titre{Les fonctions 2}{4}
\end{titre}
 
 
\begin{ExD} 

Expliquer la fonction \texttt{creaListe} suivante.

\begin{lstlisting}
from random import *
n=int(input("Entrer la longueur de la liste "))
m=int(input("la valeur maximale stricte des nombres de la liste "))
def creaListe(n,m):
    listeAleatoire=[ ]
    for i in range(n):
        listeAleatoire.append(randint(0,m))
    return listeAleatoire
    
print(creaListe(n,m))

\end{lstlisting}

\end{ExD}

\begin{ExD} 

Créer un programme qui demande à l'utilisateur 10 nombres et qui va lui donner la médiane, la moyenne, l'écart inter quartile et l'écart type.

\end{ExD} 


\paragraphe{Sans numpy}

Il faut revenir à la définition de l'écart type. A savoir : $\sigma = \sqrt{\frac{1}{n}\sum(x-\overline{x})^2} = \sqrt{\overline{x^2}-\overline{x}^2}$.

Ce n'est pas aussi rapide qu'avec numpy. Mais pédagogiquement cela peut valoir le détour.

D'autant que "Lire et comprendre une fonction écrite en Python renvoyant la moyenne m, l’écart type (...)" est explicitement mentionné dans le programme en page 13.
  
 \begin{Cod}
 \lstinputlisting{ecart-type-brut.py}
 \end{Cod}

 

%
% \printindex
 


\end{document}



