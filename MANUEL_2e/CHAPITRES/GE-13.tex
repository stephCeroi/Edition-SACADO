
$ABCDEFGH$ est un cube.

\begin{center}
\definecolor{ffxfqq}{rgb}{1.,0.4980392156862745,0.}
\definecolor{zzttqq}{rgb}{0.6,0.2,0.}
\definecolor{qqqqff}{rgb}{0.,0.,1.}
\begin{tikzpicture}[line cap=round,line join=round,>=triangle 45,x=1.0cm,y=1.0cm]
\clip(4.,-5.5) rectangle (9.78,-0.4);
\fill[color=zzttqq,fill=zzttqq,fill opacity=0.1] (6.,-1.) -- (8.,-2.) -- (9.,-4.) -- cycle;
\fill[color=zzttqq,fill=zzttqq,fill opacity=0.1] (5.,-2.) -- (6.,-4.) -- (8.,-5.) -- cycle;
\draw [color=qqqqff] (5.,-5.)-- (8.,-5.);
\draw [color=qqqqff] (8.,-5.)-- (9.,-4.);
\draw [dash pattern=on 2pt off 2pt,color=qqqqff] (9.,-4.)-- (6.,-4.);
\draw [dash pattern=on 2pt off 2pt,color=qqqqff] (6.,-4.)-- (5.,-5.);
\draw [color=qqqqff] (5.,-2.)-- (8.,-2.);
\draw [color=qqqqff] (8.,-2.)-- (9.,-1.);
\draw [color=qqqqff] (9.,-1.)-- (6.,-1.);
\draw [color=qqqqff] (6.,-1.)-- (5.,-2.);
\draw [color=qqqqff] (5.,-2.)-- (5.,-5.);
\draw [color=qqqqff] (8.,-2.)-- (8.,-5.);
\draw [color=qqqqff] (9.,-1.)-- (9.,-4.);
\draw [dash pattern=on 2pt off 2pt,color=qqqqff] (6.,-1.)-- (6.,-4.);
\draw [color=ffxfqq] (6.,-1.)-- (8.,-2.);
\draw [color=ffxfqq] (8.,-2.)-- (9.,-4.);
\draw [dash pattern=on 2pt off 2pt,color=ffxfqq] (9.,-4.)-- (6.,-1.);
\draw [dash pattern=on 2pt off 2pt,color=zzttqq] (5.,-2.)-- (6.,-4.);
\draw [dash pattern=on 2pt off 2pt,color=zzttqq] (6.,-4.)-- (8.,-5.);
\draw [color=zzttqq] (8.,-5.)-- (5.,-2.);
\begin{scriptsize}
\draw [color=qqqqff] (5.,-5.)-- ++(-1.0pt,0 pt) -- ++(2.0pt,0 pt) ++(-1.0pt,-1.0pt) -- ++(0 pt,2.0pt);
\draw[color=qqqqff] (4.6,-4.9) node {$A$};
\draw [color=qqqqff] (8.,-5.)-- ++(-1.0pt,0 pt) -- ++(2.0pt,0 pt) ++(-1.0pt,-1.0pt) -- ++(0 pt,2.0pt);
\draw[color=qqqqff] (8.14,-4.76) node {$B$};
\draw [color=qqqqff] (9.,-4.)-- ++(-1.0pt,0 pt) -- ++(2.0pt,0 pt) ++(-1.0pt,-1.0pt) -- ++(0 pt,2.0pt);
\draw[color=qqqqff] (9.14,-3.76) node {$C$};
\draw [color=qqqqff] (6.,-4.)-- ++(-1.0pt,0 pt) -- ++(2.0pt,0 pt) ++(-1.0pt,-1.0pt) -- ++(0 pt,2.0pt);
\draw[color=qqqqff] (6.14,-3.76) node {$D$};
\draw [color=qqqqff] (5.,-2.)-- ++(-1.0pt,0 pt) -- ++(2.0pt,0 pt) ++(-1.0pt,-1.0pt) -- ++(0 pt,2.0pt);
\draw[color=qqqqff] (4.86,-1.74) node {$E$};
\draw [color=qqqqff] (8.,-2.)-- ++(-1.0pt,0 pt) -- ++(2.0pt,0 pt) ++(-1.0pt,-1.0pt) -- ++(0 pt,2.0pt);
\draw[color=qqqqff] (8.3,-1.96) node {$F$};
\draw [color=qqqqff] (9.,-1.)-- ++(-1.0pt,0 pt) -- ++(2.0pt,0 pt) ++(-1.0pt,-1.0pt) -- ++(0 pt,2.0pt);
\draw[color=qqqqff] (9.14,-0.76) node {$G$};
\draw [color=qqqqff] (6.,-1.)-- ++(-1.0pt,0 pt) -- ++(2.0pt,0 pt) ++(-1.0pt,-1.0pt) -- ++(0 pt,2.0pt);
\draw[color=qqqqff] (6.14,-0.76) node {$H$};
\draw[color=zzttqq] (5.3,-2.92) node {$b_1$};
\end{scriptsize}
\end{tikzpicture}
\end{center}

\begin{enumerate}
\item
\begin{enumerate}
\item A l'aide des propriétés du cube, justifier que $EFCD$ est un parallélogramme.
\item En déduire que la droite $(FC)$ est parallèle au plan $(EBD)$.
\end{enumerate}
\item
\begin{enumerate}
\item Montrer de même que la droite $(FH)$ est parallèle au plan $(EBD)$.
\item En déduire la position relative des plan $(FCH)$ et $(EBD)$.
\end{enumerate}
\end{enumerate}