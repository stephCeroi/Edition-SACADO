
\begin{description}[leftmargin=*]
\item Le salaire moyen d'une petite entreprise de onze salariés est 1200 €. Calculer le salaire d'un employé supplémentaire sachant que le salaire moyen a augmenté de 2\%.

Soit $S$ la somme des salaires actuels et $\overline{s}$ la moyenne de ces salaires.

$\overline{s} = \frac{S}{11}$ en remplaçant,  $1200 = \frac{S}{11}$ donc $S = 1200 \times 11 = \np{13200}$

Un employé supplémentaire arrive avec un salaire $s_n$ et le salaire moyen augmente de 2\%, soit $\overline{S_n}$, le nouveau salaire moyen, $S_n = 1200 \ times 1,02 = \np {1224}$.

Alors, $1224 = \frac{13200+s_n}{12}$.

Il reste donc à trouver $s_n$.

$1224 = \frac{13100+s_n}{12}$

$1224 \times 12 = \np{13200}+s_n$

$1224 \times 12 - \np{13200}= s_n$


$s_n = 1224 \times 12 - \np{13100}= 1488$

\item Une classe est composée de 15 garçons et 10 filles. La moyenne des notes de cette classe est de 10,3. Si la moyenne des filles est de 10,8, quelle est celle des garçons ?

Soit $\overline{n}$ la moyenne des notes, $\overline{f}$ la moyenne des notes des filles et $\overline{g}$ la moyenne des notes des garçons.


$\overline{n} = \frac{10 \times \overline{f}+ 15 \times \overline{g}}{25}$

$10,3 = \frac{10 \times 10,8 + 15 \times \overline{g}}{25}$

$10,3 \times 25 = 10 \times 10,8 + 15 \times \overline{g}$

$257,5 = 108  + 15 \times \overline{g}$

$257,5 - 108  = 15 \times \overline{g}$

$149,5  = 15 \times \overline{g}$


$\frac{149,5}{15} = \overline{g}$

$9,96 \approx \overline{g}$



\end{description}