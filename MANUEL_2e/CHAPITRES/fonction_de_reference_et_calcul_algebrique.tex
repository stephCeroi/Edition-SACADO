\documentclass[10pt]{article}

\input{../../../latex_preambule_style/preambule}
\input{../../../latex_preambule_style/styleCourslycee}
\input{../../../latex_preambule_style/styleExercices}
%\input{../../latex_preambule_style/styleCahier}
\input{../../../latex_preambule_style/bas_de_page_seconde}
\input{../../../latex_preambule_style/algobox}



%%%%%%%%%%%%%%%  Affichage ou impression  %%%%%%%%%%%%%%%%%%
\newcommand{\impress}[2]{
\ifthenelse{\equal{#1}{1}}  %   1 imprime / affiche  -----    0 n'affiche pas
{%condition vraie
#2
}% fin condition vraie
{%condition fausse
}% fin condition fausse
} % fin de la procédure
%%%%%%%%%%%%%%%  Affichage ou impression  %%%%%%%%%%%%%%%%%%



%%%%%%%%%%%%%%%  Indentation  %%%%%%%%%%%%%%%%%%
\parindent=0pt
%%%%%%%%%%%%%%%%%%%%%%%%%%%%%%%%%%%%%%%%%%%%%%%%
\begin{document}

\begin{titre}[Les fonctions de référence]

\Titre{Calculs numériques et algébriques}{3}
\end{titre}


\begin{CpsCol}
\begin{description}
\item[$\square$] Si $a$ et $b$ sont des réels, $(a+b)^2=a^2+2ab+b^2$.
\item[$\square$] Si $a$ et $b$ sont des réels, $(a-b)^2=a^2-2ab+b^2$.
\item[$\square$] Si $a$ et $b$ sont des réels strictement positifs, $\sqrt{a \times b} = \sqrt{a} \times \sqrt{b}$.
\item[$\square$] Étudier la position relative des courbes d’équation $y = x$, $y = x^2$, $y = x^3$, pour $x \geq 0$.
\item[$\square$] Étudier la parité d’une fonction dans des cas simples.
\end{description}
\end{CpsCol}



\EPCN{Calculer}

\begin{enumerate}
\item Développer $(a+b)(a+b)$ où $a$ et $b$ sont deux réels.
\item En déduire la forme développée de $(a+b)^2$. \fbox{Encadrer la formule dans votre cours.}
\item Dans la suite de l'exercice, $x$ est un nombre réel. Utiliser la question précédente pour factoriser les expressions suivantes : 
\begin{enumerate}
\item $A(x)=x^2+2x+1$.
\item $B(x)=x^2+6x+9$.
\item $C(x)=x^2+10x+25$.
\item $D(x)=x^2+20x+100$.
\end{enumerate}
\item Factoriser $H(x)=4x^2+12x+9$ 
\end{enumerate}
 

\EPCN{Calculer}

\begin{enumerate}
\item Développer $(a-b)(a-b)$ où $a$ et $b$ sont deux réels.
\item En déduire la forme développée de $(a-b)^2$. \fbox{Encadrer la formule dans votre cours.}
\item Dans la suite de l'exercice, $x$ est un nombre réel. Utiliser la question précédente pour factoriser les expressions suivantes : 
\begin{enumerate}
\item $A(x)=x^2-2x+1$.
\item $B(x)=x^2-6x+9$.
\item $C(x)=x^2-10x+25$.
\item $D(x)=x^2-20x+100$.
\end{enumerate}
\item Factoriser $H(x)=4x^2-12x+9$ 
\end{enumerate}


\begin{Rq}

$(a+b)^2$ est une forme factorisée puisque $(a+b)^2=(a+b)(a+b)$.

$(a-b)^2$ est une forme factorisée puisque $(a-b)^2=(a-b)(a-b)$.
\end{Rq}

 
\begin{minipage}{0.55\linewidth}
\EPC{1}{FEA-26}{Raisonner. Calculer. }
\end{minipage}
\begin{minipage}{0.45\linewidth}
\EPC{1}{FEA-86}{Représenter. Calculer. }
\end{minipage}




\EPCN{Calculer.}

On pose $x = \sqrt{3}$ et $y=\sqrt{2}$. Calculer

\begin{enumerate}
\begin{minipage}{0.25\linewidth}
\item $A = x^4-y$
\end{minipage}
\begin{minipage}{0.25\linewidth}
\item $B = 2x^2+2x+3$
\end{minipage}
\begin{minipage}{0.25\linewidth}
\item $C = (x+2)(x-4)$
\end{minipage}
\begin{minipage}{0.25\linewidth}
\item $D = x^3 \times y^3$
\end{minipage}
\end{enumerate}



\EPCN{Calculer.}

Le but de cet exercice est de simplifier les écritures numériques.

\begin{Ex}
$\sqrt{12} = \sqrt{4 \times 3} = \sqrt{4} \times \sqrt{3}= 2 \sqrt{3}$

Remarque : $\sqrt{12} = \sqrt{6 \times 2}$ mais ce produit n'aboutit pas.
\end{Ex}
 

\begin{enumerate}
\item Écrire sous la forme $a\sqrt{b}$ les nombres suivants :

$A = \sqrt{8}$ , $B=\sqrt{27}$ , $C=\sqrt{20}$  , $D=\sqrt{18}$  , $E=\sqrt{12}$  , $F=\sqrt{72}$
\item Simplifie les écritures :

$A=\frac{\sqrt{20}}{4}$ , $B = \frac{2-\sqrt{8}}{4}$ , $C=\frac{3-\sqrt{27}}{3}$ , $D=\sqrt{18}\sqrt{2}$  , $E=\sqrt{12}+\sqrt{45}$  , $F=\sqrt{18} -\sqrt{72} +\sqrt{32}$

\end{enumerate}


\EPC{1}{FEA-51}{Calculer.}

\begin{minipage}{0.45\linewidth}
\EPCN{Calculer}

Soit $x \in \R-\left\lbrace 0;-1\right\rbrace $.  Calculer $\frac{1}{x+1}-\frac{1}{x}$

 \end{minipage}
\begin{minipage}{0.55\linewidth}

\EPCN{Raisonner. Calculer.}

Soit $a$ et $b$ deux réels non nuls. Est-il vrai que $\frac{1}{\frac{1}{a}+\frac{1}{b}}=a+b$ ?
 \end{minipage}



\EPCN{Calculer.} 

On donne $E = \frac{2}{3}+\frac{17}{2} \times \frac{4}{3}$ et $F = \frac{\sqrt 6 \times \sqrt 3\times \sqrt{16} }{\sqrt 2}  $ 
 
Démontrer que les nombres E et F sont égaux.



\EPCN{Représenter. Calculer.}

Soit $x \geq 0$.
\begin{enumerate}
\item Étudier le signe de $x^3-x^2$.
\item Étudier le signe de $x^2-x$.
\item En déduire la position relative des courbes d’équation $y = x$, $y = x^2$, $y = x^3$, pour $x \geq 0$.
\end{enumerate}
\textit{Aide : Penser à factoriser. On pourra utiliser sa calculatrice ou Geogebra pour visualiser les courbes.}


\EPCN{Représenter. Calculer.}

\begin{enumerate}
\item 
\begin{enumerate}
\item Soit $f$ la fonction Carré. Calculer $f(-x)$. Comparer $f(-x)$ et $f(x)$. On dit que la fonction est paire.
\item Soit $g$ la fonction Cube. Calculer $g(-x)$. Comparer $g(-x)$ et $g(x)$. On dit que la fonction est impaire.
\item Soit $i$ la fonction Inverse. Calculer $i(-x)$. Comparer $i(-x)$ et $i(x)$. Que dire de la fonction inverse ?
\item La fonction Racine carrée est-elle paire ? impaire ? justifier.
\end{enumerate}
\item La fonction $h$ définie sur $\R$ par $h(x)=x^2+2$ est-elle paire ? impaire ? justifier.
\item La fonction $k$ définie sur $\R$ par $k(x)=x^3+x-2$ est-elle paire ? impaire ? justifier.
\item La fonction $u$ définie sur $\R$ par $u(x)=x^3+x^2-2$ est-elle paire ? impaire ? justifier.
\end{enumerate}
 
\newpage

\begin{titre}[Les fonctions de référence]

\Titre{Calculs algébriques}{1}
\end{titre}

\begin{CpsCol}
\begin{description}
\item[$\square$] Résoudre algébriquement une équation ou une inéquation du type $f(x) = k$, $f(x) < k$.
\end{description}
\end{CpsCol}


 
\begin{minipage}{0.55\linewidth}
\EPCN{Calculer }

Résoudre les équations suivantes dans $\R$.

\begin{enumerate}
\item $(x+3)(x-7)=0$
\item $x^3-x=0$
\item $\left( x-\sqrt{3} \right)(5-2x)=0$
\end{enumerate}

 \end{minipage}
\begin{minipage}{0.55\linewidth}
\EPCN{Calculer }

Résoudre les inéquations suivantes dans $\R$.

\begin{enumerate}
\item $2x+1>3$
\item $-5x + \frac{1}{2} \geq \frac{5}{2}$
\end{enumerate}
 \end{minipage}

\EPCN{Représenter. Raisonner. Calculer }

On considère un triangle $ABC$ et un nombre réel $x$. On a $AB=x+1$, $BC=4$ et $CA=15$.

\begin{enumerate}
\item Montrer que $x+1 \leq 19$ et $ x+5 > 15$
\item Donner le plus grand intervalle de $\R$ auquel appartient $x$.
\end{enumerate}




\EPCN{ Représenter. Calculer. }

On souhaite démontrer que $x^3-8=0$ n'admet qu'une unique solution $x=2$ sur $\R$.

\begin{enumerate}
\item Démontrer que pour tout réel $x$, $x^3-8=(x-2)(x^2+2x+4)$.
\item
\begin{enumerate}
\item Démontrer que pour tout réel $x$, $x+2x+4=(x+1)^2+3$.
\item En déduire que pour tout réel $x$, $x+2x+4>0$.
\end{enumerate}
\item En déduire que $x^3-8=0$ n'admet qu'une unique solution $x=2$ sur $\R$. 
\end{enumerate}



\EPCN{ Représenter. Calculer. }

\begin{enumerate}
\item Résoudre, $\sqrt{x^2-3}=1$.
\item Résoudre, $\sqrt{x-3}=x+1$. 
\item Résoudre, $\frac{3}{x}=4$. 
\item Résoudre, $\frac{-1}{x}=x+2$. 
\item Résoudre, $\sqrt{x}-x=0$. 
\end{enumerate}




\EPCN{ Raisonner. Calculer. }

Soit $k$ un nombre réel. On considère l'équation suivante d'inconnue $x$ : $k^2x+7=x-2k$.


\begin{enumerate}
\item Résoudre cette équation dans $\R$ en fonction de $k$.
\item Pour quelles valeurs de $k$ n'existe-t-il pas de solution ?
\item A quel plus petit ensemble de nombres appartient $k$ lorsque 0 est une solution de l'équation ?
\end{enumerate}
%
%
%
%
%
%\begin{minipage}{0.56\linewidth}
%
%
%\EPC{1}{FR-45}{Chercher.}
%
%
Ranger \textbf{sans calculatrice} par ordre croissant les nombres suivants :

 $\sqrt{3}$; $\sqrt{\frac{5}{3}}$; $\sqrt{\pi}$; $ \sqrt{3,8}$ ; $ \sqrt{0,1287}$

 
%
%Justifier votre démarche dans chaque cas.
%
%\EPC{1}{FR-47}{Représenter.}
%\end{minipage}
%\hfill
%\begin{minipage}{0.42\linewidth}
%
%\EPC{1}{FR-57}{Représenter.} 
%\end{minipage}
%
%
%\EPC{1}{FPD-0}{Chercher. Représenter.}
%
%\EPC{1}{FPD-2}{Chercher. Représenter.}
%
%\EPC{1}{FR-56}{Calculer}
%
%\EPC{1}{FR-51}{Chercher.}
%
%\EPC{1}{FR-54}{Calculer.}
% 
%\EPC{1}{FR-6}{Raisonner. Communiquer.}
%
%\begin{minipage}{0.49\linewidth}
%\EPC{1}{FR-55}{Représenter.} 
%\end{minipage}
%\hfill
%\begin{minipage}{0.49\linewidth}
%\EPC{1}{FR-62}{Représenter.} 
%\end{minipage}
%
%\EPC{1}{FR-63}{Représenter.} 
%
%\EPC{1}{FR-61}{Chercher. Raisonner.}
% 
%
%\EPCC{1}{FR-58}{Raisonner. Communiquer.}
%
%\EPCC{1}{FR-59}{Raisonner. Communiquer.}
%
%\EPCC{1}{FR-60}{Raisonner. Communiquer.}


%\EPC{1}{stat-35}{Chercher. Modéliser.}
%
%\ligne{30}
% 
%\multido{\i=0+1}{87}{ \AD{1}{FEA-\i} }
\end{document}
