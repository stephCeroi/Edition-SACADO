
Soit la fonction $f$, définie sur $\R$ par $f (x) = x^2 - 4x +1$.
\begin{enumerate}
\item Démontrer que pour tout réel $x$, $f(x)=(x-2)^2-3$.
\item  Déterminer alors les variations de $f$ sur $[2;+\infty     [$  puis sur $]-\infty;2]$.
\item  Dresser le tableau de variations de $f$ sur $\R$
\item A l'aide d'un traceur, tracer la représentation graphique de $f$.
\item Par quelle transformation la représentation graphique de $f$ est-elle l'image de celle de la fonction carré ?
\item  Résoudre graphiquement l'équation $f(x) = 6$. Résoudre par calcul cette même équation.
\item  Grâce à la calculatrice, donner une valeur approchée à 0,1 près des solutions de l'équation $f(x)=0$.
 \end{enumerate}