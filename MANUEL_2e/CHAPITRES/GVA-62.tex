
On considère un triangle quelconque $ABC$ tel que $a=BC$ , $b=AC$, et $c=AB$.
\begin{enumerate}
\item  Prouver que les points de la bissectrice ($d$) de l'angle en $A$ dans $ABC$ sont équidistants des droites $(AB)$ et $(AC)$.
\item On note J le pied de la bissectrice issue de A dans le triangle ABC.
Démontrer l'égalité : $\frac{JB}{JC}=\frac{AB}{AC}$
\item En déduire que : $b \overrightarrow{JB} + c \overrightarrow{JC} = \vec{0}$. 
\item On considère le point P défini par : $a \overrightarrow{PA} + b \overrightarrow{PB} + c \overrightarrow{PC} = \vec{0}$. 
Démontrer que $A, P,$ et $J$ sont alignés. En déduire que les trois bissectrices de $ABC$ sont concourantes.
\end{enumerate}