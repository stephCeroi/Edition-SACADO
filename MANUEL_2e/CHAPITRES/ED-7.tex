
On se place dans un repère \Oij du plan (on complétera une figure au fur et à mesure des questions). Soient $A(-1;3)$ et $B(5;1)$ deux points du plan.
\begin{enumerate}
\item Déterminer une équation de la droite .
\item Placer le point $C\left(\frac{3}{2};2 \right)$. Le point appartient-il à la droite $(AB)$ ?
\item  Déterminer une équation de la droite $\Delta$ parallèle à la droite $d:y=3x-\frac{5}{2}$ passant par le point $E(-1;1)$.
\item  Déterminer le nombre de solution puis résoudre dans $\R$ chacun des systèmes suivants:
\begin{enumerate}
\item $\left\lbrace \begin{tabular}{c}
$x+3y=8$ \\  
$6x-2y=5$ \\ 
\end{tabular}  \right. $
\item $\left\lbrace \begin{tabular}{c}
$3x-y=-4$ \\  
$6x-2y=5$ \\ 
\end{tabular}  \right. $

 \end{enumerate}		
\item  Que peut on en déduire ?
\end{enumerate}
