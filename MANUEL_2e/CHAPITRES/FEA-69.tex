
La valeur $g$ de la pesanteur terrestre (exprimée en N.kg$^{-1}$) est une fonction $g$ de l'altitude $h$ (exprimée en m) : $$g(h)=g_0\frac{R^2}{(r+h)^2}$$  où $g_0 \approx 9,81$ et R est le rayon de la terre moyen environ $6,37 \times 10^6$m.
\begin{enumerate}
\item Quelle est la valeur de $g$ à Lyon, dont l'altitude est 169 m ?
\item Quelle est la valeur de $g$ au sommet de l'Everest ?
\end{enumerate}