\begin{titre}[Géométrie vectorielle et analytique]

\Titre{Projection orthogonale}{3}
\end{titre}

\begin{CpsCol}
\textbf{Démontrer avec les vecteurs}
\begin{description}
\item[$\square$] Résoudre des problèmes de géométrie plane sur des figures simples ou complexes
(triangles, quadrilatères, cercles).
\item[$\square$] Calculer des longueurs, des angles, des aires et des volumes.
\item[$\square$] Traiter de problèmes d’optimisation. 
\end{description}
\end{CpsCol}



\begin{DefT}{Projection orthogonale}\index{Projection orthogonale}
Soit $d$ une droite du plan et $A$ un point extérieur à $d$. La projection orthogonale est l'application du plan qui  au point $A$ associe le point $A'$ de la droite $d$ tel que la droite $(AA')$ est perpendiculaire à $d$ en $A'$. 

\definecolor{xdxdff}{rgb}{0.49019607843137253,0.49019607843137253,1.}
\definecolor{ududff}{rgb}{0.30196078431372547,0.30196078431372547,1.}
\begin{tikzpicture}[line cap=round,line join=round,>=triangle 45,x=0.5cm,y=0.5cm]
\clip(0.831296100794201,-0.7348563902439222) rectangle (17.486323662311374,5.80145736585363);
\draw[line width=2.pt,fill=black,fill opacity=0.10000000149011612] (9.190921775722733,1.1160512934995357) -- (9.06194825746555,1.7700717388119918) -- (8.407927812153094,1.6410982205548086) -- (8.536901330410277,0.9870777752423525) -- cycle; 
\draw [line width=2.pt,domain=0.831296100794201:17.486323662311374] plot(\x,{(-17.201085830991378--4.87081073170731*\x)/24.699720119533016});
\draw [line width=2.pt,dash pattern=on 3pt off 3pt,domain=0.831296100794201:17.486323662311374] plot(\x,{(-215.6669425698831--24.699720119533016*\x)/-4.87081073170731});
\draw (2.119703893288926,1.0563449756097338) node[anchor=north west] {$d$};
\begin{scriptsize}
\draw [fill=ududff] (-5.453619960155669,-1.7718677073170916) circle (2.5pt);
\draw[color=ududff] (-5.155086447260557,-1.1119514146341656) node {$A_1$};
\draw [fill=ududff] (19.246100159377345,3.098943024390219) circle (2.5pt);
\draw[color=ududff] (19.466072221510586,3.6802978536585105) node {$B$};
\draw[color=black] (-13.43546335756202,-2.793166731707334) node {$f$};
\draw [color=xdxdff] (8.536901330410277,0.9870777752423525)-- ++(-2.5pt,0 pt) -- ++(5.0pt,0 pt) ++(-2.5pt,-2.5pt) -- ++(0 pt,5.0pt);
\draw[color=xdxdff] (8.02752499058181,0.5378393170731492) node {$A'$};
\draw[color=black] (6.267748493515845,10.405159121951185) node {$g$};
\draw [color=xdxdff] (7.748641917797854,4.984315185099545)-- ++(-2.5pt,0 pt) -- ++(5.0pt,0 pt) ++(-2.5pt,-2.5pt) -- ++(0 pt,5.0pt);
\draw[color=xdxdff] (7.964675829972312,5.565772975609728) node {$A$};
\end{scriptsize}
\end{tikzpicture}


\end{DefT}



\begin{Rq}
Tout point du plan a un unique projeté orthogonal sur une droite $d$.

Tout point $M$ de la droite $d$ est le projeté orthogonal d'une infinité de point du plan qui appartiennent à la droite passant par $M$ et perpendiculaire à $M$.

\definecolor{xdxdff}{rgb}{0.49019607843137253,0.49019607843137253,1.}
\definecolor{ududff}{rgb}{0.30196078431372547,0.30196078431372547,1.}
\begin{tikzpicture}[line cap=round,line join=round,>=triangle 45,x=0.5cm,y=0.5cm]
\clip(2.2454022145079233,-3.468795317073187) rectangle (14.406714792445936,7.749781658536554);
\draw[line width=2.pt,fill=black,fill opacity=0.10000000149011612] (9.190921775722733,1.1160512934995357) -- (9.06194825746555,1.7700717388119918) -- (8.407927812153094,1.6410982205548086) -- (8.536901330410277,0.9870777752423525) -- cycle; 
\draw [line width=2.pt,domain=2.2454022145079233:14.406714792445936] plot(\x,{(-17.201085830991378--4.87081073170731*\x)/24.699720119533016});
\draw [line width=2.pt,dash pattern=on 6pt off 6pt,domain=2.2454022145079233:14.406714792445936] plot(\x,{(-215.6669425698831--24.699720119533016*\x)/-4.87081073170731});
\draw (2.119703893288926,1.0563449756097338) node[anchor=north west] {$d$};
\begin{scriptsize}
\draw [fill=ududff] (-5.453619960155669,-1.7718677073170916) circle (2.5pt);
\draw[color=ududff] (-5.155086447260557,-1.1119514146341656) node {$A_1$};
\draw [fill=ududff] (19.246100159377345,3.098943024390219) circle (2.5pt);
\draw[color=ududff] (19.466072221510586,3.6802978536585105) node {$B$};
\draw[color=black] (-13.43546335756202,-2.793166731707334) node {$f$};
\draw [color=xdxdff] (8.536901330410277,0.9870777752423525)-- ++(-2.5pt,0 pt) -- ++(5.0pt,0 pt) ++(-2.5pt,-2.5pt) -- ++(0 pt,5.0pt);
\draw[color=xdxdff] (7.9332512496675625,0.5378393170731492) node {$M$};
\draw[color=black] (6.267748493515845,10.405159121951185) node {$g$};
\draw [color=xdxdff] (7.748641917797854,4.984315185099545)-- ++(-2.5pt,0 pt) -- ++(5.0pt,0 pt) ++(-2.5pt,-2.5pt) -- ++(0 pt,5.0pt);
\draw[color=xdxdff] (7.964675829972312,5.565772975609728) node {$A$};
\draw [color=xdxdff] (9.048175607643872,-1.6055771743290161)-- ++(-2.5pt,0 pt) -- ++(5.0pt,0 pt) ++(-2.5pt,-2.5pt) -- ++(0 pt,5.0pt);
\draw[color=xdxdff] (9.253083622467036,-1.0333899512195315) node {$C$};
\draw [color=xdxdff] (9.229394525056087,-2.524532310751975)-- ++(-2.5pt,0 pt) -- ++(5.0pt,0 pt) ++(-2.5pt,-2.5pt) -- ++(0 pt,5.0pt);
\draw[color=xdxdff] (9.441631104295533,-1.9447029268292864) node {$D$};
\draw [color=xdxdff] (7.474252237411067,6.375736178964803)-- ++(-2.5pt,0 pt) -- ++(5.0pt,0 pt) ++(-2.5pt,-2.5pt) -- ++(0 pt,5.0pt);
\draw[color=xdxdff] (7.6818546072295675,6.948454731707288) node {$E$};
\end{scriptsize}
\end{tikzpicture}

\end{Rq}

\EPCG{1}{GVA-63}{Représenter. Calculer.}


\EPCG{1}{GVA-64}{Raisonner. Représenter. Calculer.}


\EPCG{1}{GVA-59}{Représenter. Calculer.}



\EPCG{1}{GVA-59}{Représenter}

 

\EPCG{1}{GVA-60}{Représenter}

\App{1}{GVA-13}
 

\EPCG{1}{GVA-61}{Modéliser. Représenter. Calculer.}


\begin{ROC}

Le projeté orthogonal du point M sur une droite $\Delta$ est le point de la droite $\Delta$ le plus
proche du point M.
\end{ROC}


\begin{Approfondissement}

\begin{enumerate}
\item Démontrer que les trois médiatrices d'un triangle quelconque ABC sont concourantes.
\item Démontrer que les trois hauteurs d'un triangle quelconque ABC sont concourantes.
\end{enumerate}

\end{Approfondissement}


 