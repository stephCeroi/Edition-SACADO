
\paragraph{Parie A}

\begin{enumerate}
\item Tracer une droite graduée d'unité 3cm.
\item Placer les points $M$ tel que $\Vert M \Vert = 1$. On notera $M_1$ et $M_2$.
\item Placer les points $A$ tel que $\Vert A \Vert = \frac{2}{3}$. On notera $A_1$ et $A_2$.
\end{enumerate}

Remarque : Dans ce type de question, on ne spécifie pas les 2 points ! C'est à l'élève de savoir ce qu'il doit faire. Ce type d'exercice se pose alors comme :

\paragraph{Parie B}

\begin{enumerate}
\item Tracer une droite graduée d'unité 2cm.
\item Placer le point $M$ tel que $\Vert M \Vert = 2$.
\item Placer le point $M$ tel que $\Vert M \Vert = \frac{3}{2}$.
\end{enumerate}