
Gaston Lagaffe fait tomber son téléphone portable dans le port de son village le 28 aout. Il rentre chez lui, sèche son portable au sèche-cheveux, le met dans un sac de riz. Après 3 jours, ledit téléphone ne démarre pas. Il souhaite acheter un téléphone sous 3 jours. Son choix se porte sur un Mouette 6X.

Sur internet, il lit une promotion proposée sur ce téléphone dans certains magasins, Darty, Fnac, Carrefour, rueducommerce.com, etc. : 30 euros remboursés si l'achat est effectué avant le 17 septembre. Comme M. Lagaffe vit à l'étranger, il bénéficie de la détaxe à l'exportation. Quasi tous les magasins, sauf rueducommerce.com, passe par un organisme de détaxe et le remboursement de la taxe s'élève à 12\% du montant de la facture si ce montant dépasse 172,50 euros.

Le Mouette 6X est proposé au prix de 249 euros TTC dans ces grandes enseignes mais n'est disponible avant le 28 aout qu'à la Fnac. 

Chez PriceTelephone, un petit détaillant, le même téléphone est proposé à 232 euros, avec une détaxe à l'exportation de 20\% mais sans la remise de 30 euros.

Dans quel magasin, Gaston Lagaffe doit-il acheter son Mouette 6X ? Justifier clairement la réponse.