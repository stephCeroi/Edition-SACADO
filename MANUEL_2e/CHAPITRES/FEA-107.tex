
Les vétérinaires donnent parfois le tableau de correspondance entre l'age des chats $c$ en mois et l'age humain en années $H(c)$ : 

\begin{tabular}{|c|c|c|c|c|c|c|}
\hline 
Age du chat (en mois) & 0,5 & 1 & 2 & 6 & 12 & 16 \\ 
\hline 
Age humain (en année) & 10 & 18 & 26 & 42 & 70 & 94 \\ 
\hline 
\end{tabular} 


\begin{enumerate}
\item Tracer dans un repère orthonormé, les points issus du tableau de valeurs avec comme unité : 1cm pour 1 an en abscisse et 1cm pour 10 ans en ordonnée.
\item On peut modéliser ce tableau par la fonction $K$ suivante : $K(c)=\frac{5c(c+1)^3}{c^3+1}$. Construire un tableau de valeurs à la calculatrice sur l'intervalle $[0;16]$ : Cette modélisation est-elle raisonnable en comparaison avec le tableau donné ?
\item Calculer et interpréter $K(14)$.
\item A l'aide du tableau de valeurs, estimer l'age d'un chat qui aurait l'age de 50 ans en age humain.
\end{enumerate}

 