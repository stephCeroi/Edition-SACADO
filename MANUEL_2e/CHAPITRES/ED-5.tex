
Soient $A(2\sqrt{2};-3)$  ; $B(\sqrt{2}-1;1)$  ;  $C(1;-\sqrt{2})$  et  $D(-\sqrt{2}-\frac{1}{2};\sqrt{2}+2)$    dans un repère \Oij.
\begin{enumerate}
\item Les droites $(AB)$ et $(CD)$ sont-elles parallèles?
\item Dans ce même repère, on considère les points $E(500;-1498)$, $F(0,05;1,85)$  et $G \left( -\frac{4}{3};2 \right)$. Les points $E$, $F$ et $G$ sont-ils alignés ?
\end{enumerate}