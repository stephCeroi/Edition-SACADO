
Le programme ci-dessous, écrit en Python, renvoie une valeur $y$ pour tout entier naturel $n$.

\begin{lstlisting}
def Carre(n):
	i=n
	y=0
	if i == 0:
		return y
	else :
		while i>0:
			y=y+n
			i=i-1
		return y

print(Carre(3))
\end{lstlisting}

\begin{enumerate}
\item Dans Edupython, taper le code et regarder le résultat.
\item Expliquer pourquoi cet algorithme donne, sans effectuer une multiplication, le carré de l'entier $n$. On pourra rajouter dans la boucle \texttt{while} la ligne \texttt{print(y)}.
\item A quoi sert le test \texttt{if} dans ce programme ?
\item Modifier ce programme pour qu'il renvoie le cube d'un entier $n$ sans calculer de produit.
\end{enumerate}