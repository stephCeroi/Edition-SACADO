\begin{titre}[Fonctions et expressions algébriques]

\Titre{Distance entre deux nombres réels}{4}
\end{titre}


\begin{CpsCol}
\textbf{Utiliser des nombres pour calculer et résoudre des problèmes}
\begin{description}
\item[$\square$] Notation $\vert a\vert$. Distance entre deux nombres réels
\item[$\square$] Représentation de l'intervalle $[a-r;a+r]$ puis caractérisation par la condition $\vert x-	a\vert \leq r$.
\end{description}
\end{CpsCol}


\begin{DefT}{Valeur absolue}
Soit $M$ un point d'abscisse $a$ sur la droite graduée d'origine $O$ d'abscisse 0 et $a$ un réel.\\
On note $\vert a\vert$ la distance de $M$ à $O$.\\
L'écriture $\vert a\vert$ est appelée \textbf{valeur absolue} de $a$. on peut alors écrite : $OM = d(O,M)=\vert a\vert$
\end{DefT}


\begin{tikzpicture}[line cap=round,line join=round,>=triangle 45,x=1.0cm,y=1.0cm]
\begin{axis}[
x=1.0cm,y=1.0cm,
axis lines=middle,
ymajorgrids=true,
xmajorgrids=true,
xmin=-8.0,
xmax=8.0,
ymin=-0.5,
ymax=0.5,
xtick={-8.0,-7.0,...,8.0},
ytick={-0.0,1.0,...,0.0},]
\clip(-8.,-0.5) rectangle (8.,0.5);
\end{axis}
\end{tikzpicture}

\begin{Rq}
\begin{enumerate}
\item La valeur absolue d'un nombre se caractérise par la distance de ce nombre à 0.
\item La valeur absolue d'un nombre est positive.
\end{enumerate}
\end{Rq}


\begin{Ex}
Sur cet exemple, $OA_1 = OA_2 = 5$. On peut écrire que $\Vert -5 \Vert = \Vert 5 \Vert = 5$. 

\begin{tikzpicture}[line cap=round,line join=round,>=triangle 45,x=1.0cm,y=1.0cm]
\begin{axis}[
x=1.0cm,y=1.0cm,
axis lines=middle,
ymajorgrids=true,
xmajorgrids=true,
xmin=-5.479999999999998,
xmax=6.080000000000007,
ymin=-1.0199999999999991,
ymax=0.8400000000000013,
xtick={-5.0,-4.0,...,6.0},
ytick={-1.0,0.0,...,0.0},]
\clip(-5.48,-1.02) rectangle (6.08,0.84);
\begin{scriptsize}
\draw [color=black] (-5.,0.)-- ++(-2.5pt,0 pt) -- ++(5.0pt,0 pt) ++(-2.5pt,-2.5pt) -- ++(0 pt,5.0pt);
\draw[color=black] (-4.81,0.42) node {$A_2$};
\draw [color=black] (5.,0.)-- ++(-2.5pt,0 pt) -- ++(5.0pt,0 pt) ++(-2.5pt,-2.5pt) -- ++(0 pt,5.0pt);
\draw[color=black] (5.19,0.42) node {$A_1$};
\draw [color=black] (0.,0.)-- ++(-2.5pt,0 pt) -- ++(5.0pt,0 pt) ++(-2.5pt,-2.5pt) -- ++(0 pt,5.0pt);
\draw[color=black] (0.14,0.37) node {$O$};
\end{scriptsize}
\end{axis}
\end{tikzpicture}
\end{Ex}

\AD{1}{FEA-78}



\AD{1}{FEA-79}

