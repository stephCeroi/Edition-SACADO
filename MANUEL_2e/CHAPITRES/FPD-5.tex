
Le but de cette activité est de déterminer le signe de l'expression $ax + b$ en fonction des coefficients $a$ et $b$.

\begin{multicols}{2}
\textbf{1. Étude de cas particuliers}

\textbf{Cas 1 :} Soit $f(x)=2 x+4$ la fonction définie sur $\R$.
\begin{enumerate}
\item Tracer la représentation graphique de $f$ dans un repère orthonormé.
\item  Déterminer graphiquement la valeur de $x$ pour laquelle $f(x)=0$.
\item  Déterminer graphiquement les valeurs de $x$ pour lesquelles $f(x)>0$.
\item  En déduire l'intervalle sur lequel $f(x)<0$.
\item  Retrouver ce résultat par le calcul.
\item  Recopier et compléter le tableau suivant :
\begin{center}
\begin{tabular}{|c|p{1cm}ccc>{\raggedleft\arraybackslash}p{1cm}|}
\hline 
$x$ & $-\infty$ &  & $\cdots$  &  & $+\infty$ \\ 
\hline 
$f(x)$ &  &  & 0 &  &  \\ 
\hline 
\end{tabular} 
\end{center}

\end{enumerate}



\textbf{Cas 2 :}
Reprendre l'étude précédente avec la fonction définie sur $\R$ par $f(x)=-x+3$.
Quel constat peut on formuler ?


\vspace{0.4cm}

\textbf{2. Cas général}


Soit $f(x)=ax+b$ la fonction définie sur $\R$.
\begin{enumerate}
\item Déterminer la valeur de $x$ pour laquelle $f(x)=0$.
\item Déterminer les valeurs de $x$ pour lesquelles $f(x)>0$. 
\item En déduire l'intervalle sur lequel $f(x)<0$.
\end{enumerate}

\vspace{0.4cm}

\textbf{Bilan}

Déterminer, en utilisant les cas ci-dessus, le signe de l'expression $ax + b$ en fonction du signe de $a$.

Recopier et compléter le tableau suivant :

\begin{center}
\begin{tabular}{|c|p{1cm}ccc>{\raggedleft\arraybackslash}p{1cm}|}
\hline 
$x$ & $-\infty$ &  & $\cdots$ &  & $+\infty$ \vplus \\ 
\hline 
$f(x)$ &  &  & 0 &  &  \\ 
\hline 
\end{tabular} 
\end{center}

\end{multicols}