
Dans un repère orthonormal, on donne les points $ A (2, 0)$,$ B (6, 0)$, $C (0, 3) $et $D ( 0, 5)$.
La perpendiculaire à $(BC)$ passant par $A$ coupe $(BC)$ en $H$, la perpendiculaire à $(BC)$ passant par $D$ coupe $(BC)$ en $K$.

\begin{enumerate}
\item Évaluer de deux façons différentes l'aire du triangle $ABC$, puis celle du triangle $BCD$.
\item En déduire que $A$ et $D$ sont équidistants de la droite $(BC)$.
\end{enumerate}

