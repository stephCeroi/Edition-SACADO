
Dans un jeu de 32 cartes neuves, on tire au hasard une carte. On note sa couleur et sa valeur.
\begin{enumerate}
\item Donner une issue possible. Combien dénombre-t-on d'issues ?
\item  On s'intéresse à l'événement $R$ : « Obtenir un roi ».
\begin{enumerate}
\item  Quelles sont les issues favorables à cet événement $R$ ?
\item Déterminer la probabilité $p(R)$.
\end{enumerate}




\item  On s'intéresse à l'événement $C$ : « Obtenir un cœur ».
\begin{enumerate}
\item Quelles sont les issues favorables à cet événement $C$ ?
\item Déterminer la probabilité $p(C)$.
\end{enumerate}
\item  On s'intéresse à l'événement $C$ et $R$, encore noté $C \cap R$.
\begin{enumerate}
\item A quel événement correspond $C \cap R$ ?
\item Déterminer les issue(s) favorable(s) à $C \cap R$. 
\item Déterminer la probabilité $p(C \cap R)$.
\end{enumerate}
\item Compléter par des nombres le diagramme suivant :


\psset{xunit=0.10583941605839398cm,yunit=1.0cm,algebraic=true,dimen=middle,dotstyle=o,dotsize=5pt 0,linewidth=0.8pt,arrowsize=3pt 2,arrowinset=0.25}
\begin{pspicture*}(7.448275862068974,0.26341463414633937)(83.03448275862083,3.970731707317073)
\rput{-0.13263123496249735}(34.275862068961914,2.0439024390243983){\psellipse[linecolor=blue](0,0)(18.987731085671538,0.917147820532316)}
\rput{-0.03229185560099871}(57.310344827588104,1.9902439024390228){\psellipse[linecolor=red](0,0)(17.331314006628332,0.8522393823435648)}
\rput{-0.2951691355065136}(45.37931034482864,2.1024390243902378){\psellipse(0,0)(35.0748282165448,1.6721293037164233)}
\rput[tl](18.48275862068968,2.5365853658536577){$\blue{R}$}
\rput[tl](66.89655172413804,2.478048780487804){\red{C}}
\rput[tl](44.27586206896559,3.746341463414634){$\Omega$}
\end{pspicture*}




\end{enumerate}