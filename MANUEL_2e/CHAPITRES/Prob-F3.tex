\begin{titre}[Probabilités]

\Titre{Notion d'événement}{6}
\end{titre}


\begin{CpsCol}
\begin{description}
\item[$\square$] Déterminer l'intersection de deux événements
\item[$\square$] Déterminer la réunion de deux événements
\item[$\square$] Utiliser la relation fondamentale
\end{description}
\end{CpsCol}

 
\begin{DefT}{Événement}\index{Événement}
On appelle \textbf{événement} un sous-ensemble de l'univers, c'est à dire un ensemble qui réunit toutes les issues favorables à une action déterminée. Un événement est une partie de l'univers.
\end{DefT}


\begin{Rq}
Une issue $x_i$ réalise un événement $A$ lorsque $x_i$ est un élément de $A$.
\end{Rq}

\begin{DefT}{Événement certain, Événement impossible} \index{Événements!Certain}\index{Événements!impossible}
Un \textbf{événement certain} est un événement qui est réalisé par toutes ses issues.

Un \textbf{événement impossible} est un événement qui n'est réalisé par aucune de ses issues.
\end{DefT}

\begin{Ex}
Lorsqu'on lance un dé cubique équilibré dont les faces sont numérotées de 1 à 6, l'événement obtenir un 7 est un événement impossible. 

L'événement "obtenir un nombre compris entre 1 et 6" est un événement certain. En effet, quelque soit la face obtenue, le nombre est compris entre 1 et 6.
\end{Ex}


\begin{DefT}{Événements incompatibles ou disjoints} \index{Événements!Incompatibles}\index{Événements!Disjoints see Événements!Incompatibles}
Deux événements A et B sont dits \textbf{incompatibles} ou \textbf{disjoints} lorsqu'ils ne peuvent se réaliser en même temps,
ou encore lorsque $A \cap B = \oslash$.
\end{DefT}

\begin{Ex}
$A = \left\lbrace 1 ; 2 \right\rbrace $ et $C = \left\lbrace  3 ; 5 \right\rbrace$ sont disjoints.
\end{Ex}


\begin{DefT}{Événements contraires} \index{Événements!Contraires}
Soit $\Omega$ un univers fini, $A$ et$ B$ deux événement inclus dans $\Omega$.
$A$ et $B$ sont deux événements \textbf{contraires} lorsque $A \cap B = \oslash$ et $A \cup B = \Omega$. On note $B=\overline{A}$ ou $A=\overline{B}$.
\end{DefT}

\begin{Ex}
Soit $\Omega =  \left\lbrace 1, 2, 3, 4, 5, 6 \right\rbrace $. $A = \left\lbrace1 ; 2\right\rbrace $ et $B = \left\lbrace 3 ; 4 ; 5 ; 6 \right\rbrace $ sont contraires et on note $A=\overline{B}$.
\end{Ex}


\begin{DefT}{Événement élémentaire} \index{Événement!élémentaire}
Un \textbf{événement élémentaire} est un événement qui ne contient qu'une seule issue.
\end{DefT}

\begin{Ex}
Lorsqu'on lance un dé cubique équilibré dont les faces sont numérotées de 1 à 6, l'événement obtenir un nombre pair plus petit que 3 est un événement élémentaire. L'événement ne contient que l'issue favorable 2. 
\end{Ex}

\mini{
\Fl{1}{Prob-30bis} 
}{
\EPC{1}{Prob-15}{Modéliser. Calculer.} 
}



\begin{DefT}{Probabilité d'un événement} \index{Probabilité d'un événement}
Soit $\Omega$ l'univers lié à une expérience aléatoire.
A chaque partie $B$ de $\Omega$, on fait correspondre un nombre compris entre 0 et 1, appelé \textbf{probabilité} de cet
événement $B$ tel que :
\begin{description}
\item[•] La somme des probabilités des événements élémentaires qui composent $\Omega$ est égale à 1.
\item[•] La probabilité de $B$ est la somme des probabilités des événements élémentaires qui composent $B$.
\item[•] La probabilité de l'événement impossible est 0. On note $p(B)$ la probabilité de l'événement $B$.
\end{description}
\end{DefT}


\begin{Pp}[Relation fondamentale]
Soit $\Omega$ l'univers lié à une expérience aléatoire et $A$ et $B$ deux événements de cet univers.
$$p(A \cup B) = p(A) + p(B) – p(A \cap B)$$
Si $A$ et $B$ sont incompatibles alors $p(A \cup B) = p(A) + p(B)$.
\end{Pp}

\begin{Pp}
Soit $A$ un événement de $\Omega$ et $B$ son événement contraire. $p (A)+ p (B)=1$.
\end{Pp}
 

\mini{
\EPC{1}{Prob-21}{Modéliser. Calculer.} 
}{
\EPC{0}{Prob-14}{Chercher.}
}
 
\mini{
\EPC{1}{Prob-7}{Modéliser. Calculer.} 


\EPC{1}{Prob-22}{Chercher.}

\EPC{0}{Prob-3}{Chercher.}

\EPC{1}{Prob-38}{Modéliser. }
}{
\EPC{1}{Prob-37}{Modéliser. }

\EPC{1}{Prob-41}{Modéliser. }

\EPC{1}{Prob-40}{Modéliser. }

 \EPC{1}{Prob-39}{Modéliser. }
}

\begin{Pp}[Équiprobabilité]
Lorsque tous les événements élémentaires ont la même probabilité, on dit qu'il y a équiprobabilité des
issues.
Dans ce cas, si l'univers $\Omega$ est composé de $n$ éventualités $\omega_i$ : $p(\omega_i)=\frac{1}{n}$

La probabilité d'un événement composé de $k$ éventualités est égale à $p(A)=\frac{k}{n}$
\end{Pp}






\mini{

\EPC{1}{Prob-33}{Modéliser. }

\EPC{1}{Prob-34}{Modéliser. }




\EPC{1}{Prob-35}{Modéliser. }

}{



\EPC{1}{Prob-36}{Modéliser. }

\EPC{1}{Prob-10}{Chercher.}
%\EPC{0}{Prob-30}{Modéliser. Calculer.} 
}


%\EPC{1}{Prob-13}{Chercher.}
 
 