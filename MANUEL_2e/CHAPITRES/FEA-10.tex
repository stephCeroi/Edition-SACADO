
On choisit un nombre entier naturel $A$. 

Si $A$ est pair, on le divise par 2 ; 

si $A$ est impair, on calcule $3n+1$ puis on recommence avec le nombre obtenu et ainsi de suite. 

\vspace{0.5cm}

On construit ainsi la suite $(u_n)$ telle que :

$\left\lbrace \begin{tabular}{cc}
$u_{n+1}=\frac{u_n}{2}$ & si $u_n$ est pair \\ 
$u_{n+1}=3u_n+1$ & si $u_n$ est pair est impair\\  
\end{tabular} \right. $

\vspace{0.5cm}


On définit alors :
\begin{description}
\item[Le temps de vol :] c'est le plus petit indice $n$ tel que $u_n = 1$. Il est de 17 pour la suite de Syracuse 15 et de 46 pour la suite de Syracuse 127.
\item[l'altitude maximale  :] c'est la valeur maximale de la suite. Elle est de 160 pour la suite de Syracuse 15 et de 4372 pour la suite de Syracuse 127.
\end{description}

\vspace{0.5cm}

\begin{Ex}
Vous avez choisi le nombre : 12\\
6.0 - 3.0 - 10.0 - 5.0 - 16.0 - 8.0 - 4.0 - 2.0 - 1.0\\
Le temps de vol est \textbf{9} et l'altitude est \textbf{16}.
\end{Ex}



\vspace{0.5cm}

Programmer : 
\begin{enumerate}
\item le temps de vol.
\item l'altitude maximale.
\end{enumerate}




\href{https://fr.wikipedia.org/wiki/Conjecture_de_Syracuse}{>> En savoir plus}

