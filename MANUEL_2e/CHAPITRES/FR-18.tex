
La fonction $f:x \mapsto \left(\frac{1}{x+2}-3 \right)^2$ se décompose de la façon suivante :

$f:x\mapsto x+2 \mapsto \frac{1}{x+2} \mapsto \frac{1}{x+2}-3 \mapsto \left(\frac{1}{x+2}-3 \right)^2$


Décomposer, comme montré dans l'exemple, les fonctions suivantes à l'aide des fonctions affine, Carré et Inverse.
\begin{enumerate}
\item $g(x)=\left(\frac{1}{x}-1 \right)^2$
\item $h(x)=\frac{1}{x^2+5}+3$
\end{enumerate}