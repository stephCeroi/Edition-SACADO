
La racine carrée du nombre $a$, notée $\sqrt a$ est le nombre positif qui est solution de l'équation $x^2 = a$ ou aussi, $x^2-a=0$. On cherche donc à résoudre l'équation $x^2-a=0$ à l'aide d'un algorithme pour déterminer une valeur approchée de $\sqrt a$ avec la précision $n$ souhaitée.

Soit $f(x)=x^2-a$. 

$f(0)=-a$ avec $a$ positif donc $f(0) \leq 0$. On calcule $f(0+kn)$ où $n$ est la précision et $k$ un entier naturel. 

A la main, on a alors 

\begin{tabular}{|c|c|c|c|c|c|c|c|}
\hline 
$f(0)$ & $f(0+0,1)$ & $f(0+2\times 0,1)$ & $f(0+3\times0,1)$ & $f(0+4\times0,1)$ & $f(0,5)$ & $f(0,6)$ & $f(0,7)$ \\ 
\hline 
 &  &  &  &  &  &  &  \\
\hline 
\end{tabular} 


\begin{verbatim}
p = int(input("Quelle précision ?"))
a = int(input("De quel nombre souhaitez vous la racine carrée ?"))
precision = 10**(-p)
v = -a
i=0
while v < 0 :
    v= i**2 - a
    if v >0:
        print(i-precision ,"<= a <=", i)
        break
    i=i+precision
\end{verbatim}

