
\textit{Lors d'un congrès international, 76  personnes parlent anglais, ce qui correspond à 80\% des participants. 60\% de l'ensemble des participants sont des femmes et parmi elles, 49  parlent anglais.}

\begin{enumerate}
\item \textit{Calculer le nombre de personnes présentes au congrès.}

Soit $N$ le nombre de personnes au congrès. $\frac{80}{100}N= 76$

$\frac{76 \times 100}{80}= N$

$95= N$. Il y a donc 95 personnes au congrès.

\item \textit{Construire un tableau à double entrée donnant la répartition des effectifs.}

\begin{tabular}{|c|c|c|c|}
\hline 
  & $Anglais$ & $\overline{Anglais}$ & Total \\ 
\hline 
Homme  & 27 & 11 & 38 \\ 
\hline 
Femme & 49 & 8 & 57 \\ 
\hline 
Total & 76 & 19 & 95 \\ 
\hline 
\end{tabular} 


\item \textit{La proportion de personnes parlant anglais est-elle plus importante chez les hommes ou les femmes ?}

Pour répondre à cette question, on peut la voir sous deux angles.

\begin{enumerate}
\item La proportion par rapport à toutes les personnes du congrès.
\begin{enumerate}
\item Soit $p_H$ la proportion des hommes qui parlent anglais. $p_H=\frac{27}{95}$
\item Soit $p_F$ la proportion des hommes qui parlent anglais. $p_F=\frac{49}{95}$
\item Comme  $p_F=\frac{49}{95} > \frac{27}{95}=P_H$, la proportion de femmes est plus importante.
\end{enumerate}

 \item La proportion par rapport à chaque genre.
\begin{enumerate}
\item Soit $p_H$ la proportion des hommes qui parlent anglais. $p_H=\frac{27}{38}$
\item Soit $p_F$ la proportion des hommes qui parlent anglais. $p_F=\frac{49}{57}$
\item On doit donc comparer Comme  $p_F=\frac{49}{57}$ et $\frac{27}{38}=P_H$. \textbf{Pour comparer deux fraction, on les soustrait}.

$p_F - p_H=\frac{49}{57} - \frac{27}{38} =\frac{49 \times 38}{38\times 57} - \frac{27\times 57}{57\times 38}$

$p_F - p_H=\frac{1862}{2166} - \frac{1539}{2166} =\frac{323}{2166}$ donc $p_F - p_H >0$ donc la proportion de femmes est plus importante.


\end{enumerate}


\end{enumerate}
\end{enumerate} 