
Au 1\ier juin un producteur de pomme de terre peut en récolter 1200 kg et les vendre 1 € le kilogramme. S'il attend, chaque jour sa récolte augmente de 60 kg mais le prix baisse de 0,02 € par kilogramme et par jour.
\begin{enumerate}
\item
	\begin{enumerate}
		\item Combien gagnera-t-il s'il vend toute sa récolte le 1\ier juin ?
		\item Combien gagnera-t-il s'il attend 1 jour pour vendre toute sa récolte ? S'il attend 5 jours ?	S'il attend 10 jours ?
		\item On suppose que le producteur attend $n$ jours ($0 \leq n \leq 50$) avant de vendre sa récolte. Indiquer en fonction de $n$			\begin{enumerate}
			\item le nombre de kg à vendre;
			\item le prix du kilogramme;		
		\end{enumerate}
		Montrer que le prix de vente est $P(n)=1,2(-n^2+30n+1000)$
	\end{enumerate}
\item A l'aide d'un traceur, dessiner la courbe de $f$ définie sur $[0;50]$ par $f(x)=1,2(-x^2+30x+1000)$.
\item Conjecturer la date à laquelle $P$ sera maximal.
\item Démontrer que $P(n)=-1,2[(n-15)^2-\np{1225}]$. En déduire le résultat précédent.
\end{enumerate}