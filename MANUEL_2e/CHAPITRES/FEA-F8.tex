\begin{titre}[Arithmétique]

\Titre{Multiples, diviseurs, nombres premiers}{4}
\end{titre}

\begin{CpsCol}
\begin{description}
\item[$\square$] Modéliser et résoudre des problèmes mobilisant les notions de multiple, de diviseur,
de nombre pair, de nombre impair, de nombre premier.
\item[$\square$] Pour des entiers a et b donnés, déterminer le plus grand multiple de a inférieur ou
égal à b.
\item[$\square$] Déterminer si un entier naturel est premier.
\end{description}
\end{CpsCol}

\begin{DefT}{Diviseur}\index{Diviseur}
Soient $a$ et $b$ deux nombres relatifs. $a$ est un diviseur de $b$ lorsqu'il existe un entier $k$ tel que $b = k  \times a$. Les 3 assertions sont équivalentes :
\begin{description}
\item[•] $b$ est un multiple de $a$.
\item[•] $a$ est un diviseur de $b$.
\item[•] $b$ est un divisible par $a$.
\end{description}
\end{DefT}

\begin{Ex}
3 est un diviseur de 36. En effet, $36 = 3 \times 12$ et $12 \in \Z$.
\end{Ex}

\EPC{1}{FEA-17}{Raisonner. Calculer. } 

\begin{DefT}{Fraction irréductible}\index{Fraction irréductible}
Une fraction est irréductible lorsqu'il n'est pas possible de la simplifier, autrement dit lorsque le numérateur et le dénominateur n'ont pas de diviseur commun.
\end{DefT}

\begin{ROC}

Démontrer que $\frac{1}{3}$ n'est pas décimal.
\end{ROC}





\begin{DefT}{Nombre pair et impair}\index{Nombre pair et impair}
Soit $a$ un entier. 
\begin{description}
\item[•] $a$ est pair lorsqu'il existe un entier $n$ tel que $a = 2\times n$.
\item[•]  $a$ est impair lorsqu'il existe un entier $n$ tel que $a = 2\times n +1$.
\end{description}
\end{DefT}


\begin{Th}
 
Soit $a$ un entier ($a \in \Z)$. $a^2$ est impair si et seulement si $a$ est impair.
\end{Th}

\EPCN{Raisonner. }

Démontrer ce résultat.

\begin{Th} 

Soit $a$ un entier ($a \in \Z)$. Si $b$ et $b'$ sont deux multiples de $a$ alors $b+b'$ est un multiple de $a$.
\end{Th}

\EPCN{Raisonner. }

Démontrer ce résultat.



\begin{DefT}{Nombre premier}
Un entier naturel est dit nombre premier lorsqu'il possède exactement deux diviseurs : 1 et lui-même.
\end{DefT}


\begin{Rq} 
Un nombre premier est supérieur ou égal à 2. 1 n'est pas premier, il ne possède qu'un seul diviseur.
\end{Rq}


\mini{
\EPCN{Raisonner.}

Soit $a$ un entier impair. Démontrer que $a^3$ est un nombre impair.


\EPCN{Raisonner.}

Une crèche dispose de 60 dalles carrées en mousse. Elle souhaite les placer de manière à former un rectangle.

\begin{enumerate}
\item Quelles sont les dimensions possibles de ce rectangle ?
\item Quel est celui qui le plus grand périmètre ?
\end{enumerate}


\EPCN{Chercher.}

La conjecture de Goldbach affirme que "tout nombre pair supérieur ou égal à 4 est la somme de deux nombres premiers".

\begin{enumerate}
\item Vérifier cette conjecture pour tout nombre pairs de l'intervalle [10;20].
\item Trouver tous les nombres premiers $p$ et $p'$ tels que $p+p'=100$.
\item Déterminer une algorithme pour déterminer $p$ et $p'$.
\end{enumerate}


\EPCP{1}{FEA-102}{Modéliser. Calculer.}

\begin{ROC}

On cherche à démontrer que $\sqrt2$ n'est pas rationnel.
On raisonne par l'absurde. On suppose $\sqrt{2}$ s'écrit sous forme d'une fraction irréductible $\frac{a}{b}$, $a$ et $b$ deux entiers naturels non nuls.

\begin{enumerate}
\item Démontrer que $2b^2=a^2$.
\item En déduire que $a^2$ est pair donc $a$ est pair.
\item Démontrer alors que $b^2$ est pair. Conclure.
\end{enumerate}



\end{ROC}




}{

\EPCN{Chercher.}

On considère un nombre premier $p \geq 6$. 
\begin{enumerate}
\item Démontrer que $p=6q+1$ ou $p=6q+5$, avec $q$ un entier naturel non nul.
\item On suppose que $p=6q+1$.
  \begin{enumerate}
  \item Justifier que $p^2=12q(3q+1)+1$. 
  \item Quelle est la parité de $q(3q+1)$ ?
  \item En déduire qu'il existe un entier naturel $k$ tel que $p^2 = 24k +1$.
  \end{enumerate}
\item On suppose que $p=6q+5$. Démontrer qu'il existe un entier naturel $k$ tel que $p^2 = 24k +1$.
 \end{enumerate}

\EPCN{Chercher.}

On considère un entier naturel $n$.
  \begin{enumerate}
  \item Démontrer que $n^2+n$ est pair. 
  \item Démontrer que le chiffre des unités de $n^2+n$ n'est jamais égal à 4 ou 8.
  \end{enumerate}
 
\EPCP{1}{FEA-101}{Modéliser. Calculer.}
}



