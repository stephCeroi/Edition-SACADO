\begin{titre}[Programmation en Python]

\Titre{Itérations}{2}
\end{titre}


\begin{CpsCol}
\textbf{Python}
\begin{description}
\item[$\square$] Utiliser une boucle
\item[$\square$] Connaitre la différence d'utilisation des deux types de boucle
\end{description}
\end{CpsCol}

\begin{PC}

Créer un programme qui demande à l'utilisateur un nombre entier naturel $n$ et qui calcule la somme des entiers naturels non nuls inférieurs où égal à $n$. 
\end{PC}


\begin{DefT}{Itération. Boucle}
Une \index{Itération}itération est un procédé qui répète une même action. Une boucle\index{boucle} est une itération. 
\end{DefT}

\begin{Rq}
Il existe deux types de boucles :
\begin{description}
\item[•] Lorsque le nombre d'itérations est \textit{a priori} connue
\item[•] Lorsque le nombre d'itérations est inconnue
\end{description}
\end{Rq}

\subsection*{La boucle Pour}

\begin{Syn}
\begin{minipage}[t]{0.49\linewidth}
L'écriture algorithmique  de $n+1$ itérations
\begin{algobox}
\Pour{$i$}{0}{$n$}
\DebutPour
\Ligne Action
\FinPour
\end{algobox}

\end{minipage}
\hfill\vrule\hfill
\begin{minipage}[t]{0.49\linewidth}
La programmation en Python de $n$ itérations. $i$ varie de 0 à $n$ inclus.
\begin{lstlisting}
for i in range(n+1) :
	action
\end{lstlisting}
\end{minipage}
\end{Syn}


\begin{minipage}{0.48\linewidth}
\begin{Ex}
Calculer la somme des 5 premiers nombres entiers naturels non nuls.
\begin{lstlisting}
somme = 0
for i in range(1,6) :
	somme = somme + i
print(somme)
\end{lstlisting}
\end{Ex}
\end{minipage}
\hfill
\begin{minipage}{0.48\linewidth}
\begin{Cod}
\begin{lstlisting}
somme = 0
for i in range(1,6) :
	somme = somme + i
print(somme)
\end{lstlisting}
\end{Cod}
Que se passe-t-il si \texttt{\textbf{print}(somme)} est dans la boucle ?
\end{minipage}

\begin{Rq}
L'instruction \texttt{for i in range(1,6)} calcule de 1 jusqu'à 5 inclus. 
\end{Rq}

\AD{1}{Prog-6}

\AD{1}{Prog-9}

\AD{1}{Prog-7}

\subsection*{La boucle Tant que}

\begin{Syn}
\begin{minipage}[t]{0.49\linewidth}
L'écriture algorithmique de la boucle "TANT QUE"
\begin{algobox}
\Tantque{condition}
\DebutTantQue
\Ligne action1
\FinTantQue
\end{algobox}
\end{minipage}
\hfill\vrule\hfill
\begin{minipage}[t]{0.49\linewidth}
La programmation en Python
\begin{lstlisting}
while condition :
     action 
\end{lstlisting}
\end{minipage}
\end{Syn}

\begin{minipage}{0.68\linewidth}
\begin{Ex}
Déterminer l'entier naturel $n_0$ à partir duquel la somme des premiers entiers naturels inférieur ou égal à $n_0$ est strictement supérieure à 15.

\begin{tabular}{|c|c|c|c|}
\hline 
Etapes & somme  & i &  condition \\ 
\hline 
0 & 0 & 1 & Vraie \\ 
\hline 
1 & 1 & 2 & Vraie \\ 
\hline 
2 & 3 & 3 & Vraie \\ 
\hline 
4 & 6 & 4 & Vraie \\ 
\hline 
5 & 10 & 5 & Vraie \\ 
\hline
6 & 15 & 6 & Faux \\ 
\hline
\end{tabular} 
\end{Ex}
\end{minipage}
\hfill
\begin{minipage}{0.3\linewidth}
\begin{Cod}
\begin{lstlisting}
somme = 0
i=1
while somme < 15 :	
  somme = somme + i
  i=i+1
print(i)
\end{lstlisting}
\end{Cod}
\end{minipage}

\begin{Att}
Il faut être vigilant dans la position des lignes de code dans la boucle. 
\end{Att}

\AD{1}{Prog-8}


