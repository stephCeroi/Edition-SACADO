
Soit $f$ la fonction définie par $f(x)= 2-\frac{1}{x - 1}$.
\begin{enumerate}
\item Déterminer l'ensemble de définition $\mathscr{D}$ de $f$ .
\item Démontrer que $f$ est une fonction homographique.
\item Dresser le tableau de variations de $f$ sur $\mathscr{D}$.
\item Résoudre $f(x)=3$.
\item Résoudre $f(x)>0$.
\item Déterminer les coordonnées du centre de symétrie de la courbe de $f$.
\item Laquelle de ces courbes représentent la fonction $f$ ?

\begin{minipage}{0.3\linewidth}
\definecolor{ffqqqq}{rgb}{1.,0.,0.}
\definecolor{cqcqcq}{rgb}{0.7529411764705882,0.7529411764705882,0.7529411764705882}
\begin{tikzpicture}[line cap=round,line join=round,>=triangle 45,x=0.7cm,y=0.7cm]
\draw [color=cqcqcq,, xstep=0.7cm,ystep=0.7cm] (-4.218018493909193,-1.6141256626061267) grid (2.7429054105363053,3.546239258556136);
\draw[->,color=black] (-4.218018493909193,0.) -- (2.7429054105363053,0.);
\foreach \x in {-4.,-3.,-2.,-1.,1.,2.}
\draw[shift={(\x,0)},color=black] (0pt,2pt) -- (0pt,-2pt) node[below] {\footnotesize $\x$};
\draw[->,color=black] (0.,-1.6141256626061267) -- (0.,3.546239258556136);
\foreach \y in {-1.,1.,2.,3.}
\draw[shift={(0,\y)},color=black] (2pt,0pt) -- (-2pt,0pt) node[left] {\footnotesize $\y$};
\draw[color=black] (0pt,-10pt) node[right] {\footnotesize $0$};
\clip(-4.218018493909193,-1.6141256626061267) rectangle (2.7429054105363053,3.546239258556136);
\draw[color=ffqqqq,smooth,samples=100,domain=-4.218018493909193:-1.001] plot(\x,{2.0-1.0/((\x)+1.0)});
\draw[color=ffqqqq,smooth,samples=100,domain=-0.99:2.7429054105363053] plot(\x,{2.0-1.0/((\x)+1.0)});
\end{tikzpicture}
\end{minipage}
\hfill
\begin{minipage}{0.3\linewidth}
\definecolor{ffqqqq}{rgb}{1.,0.,0.}
\definecolor{cqcqcq}{rgb}{0.7529411764705882,0.7529411764705882,0.7529411764705882}
\begin{tikzpicture}[line cap=round,line join=round,>=triangle 45,x=0.7069837151037939cm,y=0.7069837151037939cm]
\draw [color=cqcqcq,, xstep=0.7069837151037939cm,ystep=0.7069837151037939cm] (-2.6773340030585895,-1.8925626187839466) grid (4.394964683858037,4.233050417128092);
\draw[->,color=black] (-2.6773340030585895,0.) -- (4.394964683858037,0.);
\foreach \x in {-2.,-1.,1.,2.,3.,4.}
\draw[shift={(\x,0)},color=black] (0pt,2pt) -- (0pt,-2pt) node[below] {\footnotesize $\x$};
\draw[->,color=black] (0.,-1.8925626187839466) -- (0.,4.233050417128092);
\foreach \y in {-1.,1.,2.,3.,4.}
\draw[shift={(0,\y)},color=black] (2pt,0pt) -- (-2pt,0pt) node[left] {\footnotesize $\y$};
\draw[color=black] (0pt,-10pt) node[right] {\footnotesize $0$};
\clip(-2.6773340030585895,-1.8925626187839466) rectangle (4.394964683858037,4.233050417128092);
\draw[color=ffqqqq,smooth,samples=100,domain=-2.6773340030585895:4.394964683858037] plot(\x,{2.0-1.0/((\x)-1.0)});
\end{tikzpicture}
\end{minipage}
\hfill
\begin{minipage}{0.3\linewidth}
\definecolor{ffqqqq}{rgb}{1.,0.,0.}
\definecolor{cqcqcq}{rgb}{0.7529411764705882,0.7529411764705882,0.7529411764705882}
\begin{tikzpicture}[line cap=round,line join=round,>=triangle 45,x=0.7cm,y=0.7cm]
\draw [color=cqcqcq,, xstep=0.7cm,ystep=0.7cm] (-4.218018493909193,-1.6141256626061267) grid (2.7429054105363053,3.546239258556136);
\draw[->,color=black] (-4.218018493909193,0.) -- (2.7429054105363053,0.);
\foreach \x in {-4.,-3.,-2.,-1.,1.,2.}
\draw[shift={(\x,0)},color=black] (0pt,2pt) -- (0pt,-2pt) node[below] {\footnotesize $\x$};
\draw[->,color=black] (0.,-1.6141256626061267) -- (0.,3.546239258556136);
\foreach \y in {-1.,1.,2.,3.}
\draw[shift={(0,\y)},color=black] (2pt,0pt) -- (-2pt,0pt) node[left] {\footnotesize $\y$};
\draw[color=black] (0pt,-10pt) node[right] {\footnotesize $0$};
\clip(-4.218018493909193,-1.6141256626061267) rectangle (2.7429054105363053,3.546239258556136);
\draw[color=ffqqqq,smooth,samples=100,domain=-4.218018493909193:2.7429054105363053] plot(\x,{2.0+1.0/((\x)+1.0)});
\end{tikzpicture}
\end{minipage}

\end{enumerate}