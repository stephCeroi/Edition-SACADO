
\begin{multicols}{2}
Sur la figure, les droites $(DE)$ et $(BC)$ sont parallèles. $AD=3$, $DB=2$, $AE=4$ et $DE=x$. Soit $f$ la fonction qui à $x$ associe la longueur $BC$. Soit $g$ la fonction qui à $x$ associe le périmètre $AED$ et  $h$ la fonction qui à $x$ associe le périmètre du trapèze $BDEC$.

\begin{center}
\definecolor{xdxdff}{rgb}{0.49019607843137253,0.49019607843137253,1.}
\definecolor{qqqqff}{rgb}{0.,0.,1.}
\begin{tikzpicture}[line cap=round,line join=round,>=triangle 45,x=0.5084745762711864cm,y=0.5084745762711864cm]
\clip(-1.06,-0.36) rectangle (4.92,5.54);
\draw [color=qqqqff] (0.8,5.)-- (0.,0.);
\draw [color=qqqqff] (0.,0.)-- (4.,0.);
\draw [color=qqqqff] (4.,0.)-- (0.8,5.);
\draw [color=qqqqff] (0.3183775351014041,1.9898595943837754)-- (2.72,2.);
\draw (0.2,4.04) node[anchor=north west] {$3$};
\draw (-0.34,1.7) node[anchor=north west] {$2$};
\draw (2.06,3.94) node[anchor=north west] {$4$};
\draw (1.14,2.74) node[anchor=north west] {$x$};
\begin{scriptsize}
\draw [color=qqqqff] (0.8,5.)-- ++(-2.5pt,0 pt) -- ++(5.0pt,0 pt) ++(-2.5pt,-2.5pt) -- ++(0 pt,5.0pt);
\draw[color=qqqqff] (0.94,5.36) node {$A$};
\draw [color=qqqqff] (0.,0.)-- ++(-2.5pt,0 pt) -- ++(5.0pt,0 pt) ++(-2.5pt,-2.5pt) -- ++(0 pt,5.0pt);
\draw[color=qqqqff] (-0.32,0.36) node {$B$};
\draw [color=qqqqff] (4.,0.)-- ++(-2.5pt,0 pt) -- ++(5.0pt,0 pt) ++(-2.5pt,-2.5pt) -- ++(0 pt,5.0pt);
\draw[color=qqqqff] (4.14,0.36) node {$C$};
\draw [color=xdxdff] (0.3183775351014041,1.9898595943837754)-- ++(-2.5pt,0 pt) -- ++(5.0pt,0 pt) ++(-2.5pt,-2.5pt) -- ++(0 pt,5.0pt);
\draw[color=xdxdff] (0.,2.32) node {$D$};
\draw [color=xdxdff] (2.72,2.)-- ++(-2.5pt,0 pt) -- ++(5.0pt,0 pt) ++(-2.5pt,-2.5pt) -- ++(0 pt,5.0pt);
\draw[color=xdxdff] (2.86,2.36) node {$E$};
\end{scriptsize}
\end{tikzpicture}
\end{center}

\end{multicols}

\begin{enumerate}
\item Exprimer $f(x)$, $g(x)$ et $h(x)$ en fonction de $x$. Ces fonctions sont-elles affines ? Linéaires ?
\item En donner des représentations graphiques sur $[1;7]$.
\item En déduire graphiquement la valeur de $x$ pour que $ADE$ et $BDEC$ aient le même périmètre.
\item Retrouver ce résultat par le calcul.
\end{enumerate}