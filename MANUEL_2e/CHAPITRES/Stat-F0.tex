\begin{titre}[Statistiques descriptives]

\Titre{Effectifs cumulés}{2}
\end{titre}

Il y a 3 sortes de mensonges : les mensonges, les sacrés mensonges et les statistiques.

\hfill{Mark Twain}

\begin{CpsCol}
\textbf{Utiliser des nombres pour calculer et résoudre des problèmes}
\begin{description}
\item[$\square$] Calculer les effectifs cumulés
\item[$\square$] Calculer les fréquences cumulées
\item[$\square$] Passer des effectifs aux fréquences
\end{description}
\end{CpsCol}


\EPC{1}{stat-2}{Représenter. Communiquer}

\begin{DefT}{Effectifs cumulés}
Les effectifs cumulés croissants donnent les effectifs des valeurs inférieurs à chaque valeur du caractère.

Les effectifs cumulés décroissants donnent les effectifs des valeurs supérieurs à chaque valeur du caractère.
\end{DefT}

\begin{Rq}
On représente les effectifs cumulés dans un tableau pour faciliter la lecture. Ce tableau est le tableau des effectifs cumulés, croissants ou décroissants.
\end{Rq}

\begin{DefT}{Courbe des effectifs cumulés}
La courbe des effectifs cumulés croissants relie le nuage de points qui correspondent aux effectifs cumulées croissants.

La courbe des fréquences cumulées croissantes relie le nuage de points qui correspondent aux fréquences cumulées croissantes.

De même avec les effectifs cumulés décroissants et les fréquences cumulées décroissantes.
\end{DefT}


\EPC{1}{stat-5}{Représenter. Communiquer}




\EPC{1}{stat-6}{Représenter. Calculer. Communiquer}

\EPC{1}{stat-4}{Représenter. Calculer}


%\CR{1}{stat-10}



