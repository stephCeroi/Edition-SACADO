
Le tableau suivant indique la population (en millions d’habitants) et la densité de population (en $hab/km^2$) des pays du Proche-Orient.

\begin{tabular}{|c|c|c|}
\hline 
Pays & Population (Millions) & Densité ($hab/km^2$) \\ 
\hline 
Arabie Saoudite & 20,9 & 9,7 \\ 
\hline 
Barhein & 0,7 & 700 \\ 
\hline 
E.A.U & 2,8 & 33,3 \\ 
\hline 
Égypte & 66,9 & 66,8 \\ 
\hline 
Iran & 66,2 & 40,1 \\ 
\hline 
Irak & 22,5 & 51,8 \\ 
\hline 
Israël & 6,1 & 290,4 \\ 
\hline 
Jordanie & 4,7 & 47,9 \\ 
\hline 
Koweït & 2,1 & 116,6 \\ 
\hline 
Liban & 4,1 & 410 \\ 
\hline 
Oman & 2,5 & 11,7 \\ 
\hline 
Qatar & 0,5 & 45,5 \\ 
\hline 
Syrie & 16 & 86,4 \\ 
\hline 
Yemen & 16,4 & 31 \\ 
\hline 
\end{tabular} 

\begin{enumerate}
\item On considère la série statistique des populations.
\begin{enumerate}
\item calculer la moyenne des populations.

En utilisant la calculatrice, MENU > STAT , la moyenne est égale à 16,6.

\item calculer la moyenne élaguée de la densité du Qatar et de l'Égypte.

En utilisant la calculatrice, MENU > STAT , la moyenne est égale à 13,75.

\item  Calculer la différence entre cette moyenne élaguée et la moyenne initiale.

Soit $d$ la différence : $d = 16,6 - 13,75 = 2,85$
\end{enumerate}
\item Pour la série des densités,

\begin{enumerate}
\item  calculer la médiane, le premier et le troisième quartile de la série des densités.

En utilisant la calculatrice, MENU > STAT , la médiane est égale à 138,65.


\item  calculer la médiane élaguée de la densité du Bahreïn et et de l'Arabie Saoudite, ainsi que la médiane de cette nouvelle série.

En utilisant la calculatrice, MENU > STAT , la médiane est égale à 138,65.

\item  Calculer la différence entre cette médiane et la médiane initiale.

Soit $d_1$ la différence : $d_1 = 138,65 - 138,65 = 0 $

\end{enumerate}
\item Quels commentaires vous inspirent ces résultats.


La médiane n'est pas sujette aux série élaguées alors que la moyenne est très sensible aux valeurs extrêmes.
\end{enumerate}
