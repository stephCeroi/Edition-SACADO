\begin{titre}[Probabilités]

\Titre{Mettre en ouvre une simulation}{1}
\end{titre}


\begin{CpsCol}
\begin{description}
\item[$\square$] Simuler une situation avec Python
\end{description}
\end{CpsCol}



Python requiert des bibliothèques pour chaque utilisation particulière ( tkinter, math, ...). Pour travailler sur des probabilités et simuler avec la fonction random, il est nécessaire d'appeler la fonction random en début de programme avec  $$ \text{\color{orange}import \color{black} random}$$

A l'inverse de la majorité des langages de programmation, Python a une syntaxe très simple. Il n'y a pas de point virgule ou accolade pour délimiter les blocs. Python utilise le système d'indentation. Il est donc impératif de bien placer les blocs par rapport à leur indentation.


\subsubsection*{Simulation du lancé d'un dé}

\color{orange} import \color{black} random

x = random.randint(1,6)

\color{purple} print \color{black}(x)



\subsubsection*{Afficher une simulation de 50 lancers d'un dé à 6 faces}


\color{orange} import \color{black} random

\color{orange} for \color{black} i \color{orange} in \color{purple}range\color{black}(50):

\hspace{0.4cm}     x = random.randint(1,6)
 
\hspace{0.4cm}   \color{purple} print \color{black}(x)



\subsubsection*{Applications}

La probabilité d'obtenir une face lorsque la pièce n'est pas non truquée est égale à $\frac{1}{2}$. Ce résultat semble évident. On peut le simuler de la manière suivante.


\subsection*{$n$ Lancers d'une pièce. L'approche fréquentiste de Bayès}

On souhaite connaitre la fréquence de sortie de la face "Pile" lors de $n$ lancers d'une pièce. 

On note 0 la face "Face" et 1 la face "Pile".

Écrire un programme pour une valeur variable $n$.

\begin{enumerate}
\item Tester le programme pour $n=10$
\item Tester le programme pour $n=1000$
\item Tester le programme pour $n=1000000$
\item Conclure
\end{enumerate}

%\vspace{0.4cm}
%On pourra tester ce programme.
%
%\vspace{0.4cm}
%
%
%
%\color{orange} import \color{black} random
%
%n = \color{purple}int\color{black}(\color{purple}input\color{black}(\color{black}"Entrer le nombre de lancer "\color{black}))
%
%k = 0
%
%\color{orange} for \color{black} i \color{orange} in \color{purple}range\color{black}(n):
%
%\hspace{0.4cm}     x = random.randint(0,1)
%
%\hspace{0.4cm}   \color{orange} if \color{black} x = 1 :
%
%\hspace{0.8cm}   k=k+1
%
%f=k/n
%
%\color{purple} print \color{black}(f)


