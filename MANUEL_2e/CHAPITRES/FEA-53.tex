
La loi de Planck décrit la distribution de l'énergie $W(\lambda)$ rayonnée en fonction de la température $T$ du corps noir. Selon la loi de Planck, à une température $T$ donnée, l'énergie $W(\lambda)$ passe par un maximum $W_{max}$ pour une longueur d'onde $\lambda_{max}$. 

La loi de Wien décrit la relation liant la longueur d'onde  $\lambda_{max}$, correspondant au pic d'émission lumineuse du corps noir, et la température $T$ (exprimée en kelvin). On retient généralement, en exprimant la longueur d'onde en mètre et la température en kelvin : $$\lambda_{max} = \frac{hc}{\np{4,49651} kT}$$ où :

\begin{description}
\item[•] $h=\np{6,626 070 040} \times 10^{-34} J.s$,
\item[•] $k=\np{1,38064852} \times 10^{-23} J.K^{-1}$
\item[•] $c=299 792 458 m/s$
\end{description}

 
\begin{enumerate}
\item Donner l'expression de $\lambda_{max}$ en fonction de la température $T$ uniquement.
\item Quelle est la longueur d'onde maximale pour une température de 273 K.
\end{enumerate}