\begin{titre}[Géométrie vectorielle et analytique]

\Titre{Distance entre deux points}{2}
\end{titre}

\begin{CpsCol}
\textbf{Utiliser des nombres pour calculer et résoudre des problèmes}
\begin{description}
\item[$\square$] Calculer la distance entre deux points connaissant les coordonnées des points
\end{description}
\end{CpsCol}



\begin{Th}\index{Distance entre deux points}
\Oij est un repère orthonormal du plan, $A$ et $B$ deux points de coordonnées $(x_A;y_A)$ et  $(x_B;y_B)$. 

Alors la distance $AB$, la longueur du segment $[AB]$, est donnée par  $$d(A,B)=\sqrt{(x_B-x_A)^2+(y_B-y_A)^2}$$
\end{Th}

\mini{
\EPC{1}{GVA-16}{Calculer.}

\EPC{1}{GVA-4}{Raisonner. Représenter. Calculer.}

\EPC{0}{GVA-54}{Raisonner. Représenter. Calculer.}

\EPC{0}{GVA-17}{Raisonner. Représenter. Calculer.}

}{

\EPC{1}{GVA-56}{Modéliser. Représenter. Calculer.}

\EPC{0}{GVA-57}{Raisonner. Calculer.}

\EPC{1}{GVA-19}{Raisonner. Représenter. Calculer.}
}
\newpage
\EPCP{1}{GVA-0}{Raisonner.} 