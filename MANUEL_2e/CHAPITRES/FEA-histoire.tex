
\begin{His}

\begin{wrapfigure}[11]{r}{3.6cm}
\vspace{-7mm}
\includegraphics[scale=0.5]{image_chapitres/Cantor.jpg}
\unnumberedcaption{\textsc{G. F. L. P. Cantor}} 
\end{wrapfigure}

\textbf{Georg Ferdinand Ludwig Philipp Cantor} (3 mars 1845, Saint-Pétersbourg – 6 janvier 1918, Halle) est un mathématicien allemand, connu pour être le créateur de la théorie des ensembles. Il établit l'importance de la bijection entre les ensembles, définit les ensembles infinis et les ensembles bien ordonnés. Il prouva également que les nombres réels sont « plus nombreux » que les entiers naturels. En fait, le théorème de Cantor implique l'existence d'une « infinité d'infinis ». Il définit les nombres cardinaux, les nombres ordinaux et leur arithmétique. Le travail de Cantor est d'un grand intérêt philosophique (ce dont il était parfaitement conscient) et a donné lieu à maintes interprétations et à maints débats.

\vspace{0.4cm}

Cantor a été confronté à la résistance de la part des mathématiciens de son époque, en particulier Kronecker. Poincaré, bien qu'il connût et appréciât les travaux de Cantor, avait de profondes réserves sur son maniement de l'infini en tant que totalité achevée. Les accès de dépressions récurrents du mathématicien, de 1884 à la fin de sa vie, ont été parfois attribués à l'attitude hostile de certains de ses contemporains, mais ces accès sont souvent à présent interprétés comme des manifestations d'un probable trouble bipolaire.

\vspace{0.4cm}

Au XXIe siècle, la valeur des travaux de Cantor n'est pas discutée par la majorité des mathématiciens qui y voient un changement de paradigme, à l'exception d'une partie du courant constructiviste qui s'inscrit à la suite de Kronecker. Dans le but de contrer les détracteurs de Cantor, David Hilbert a affirmé : « Nul ne doit nous exclure du Paradis que Cantor a créé ».

\vspace{0.4cm}

Cantor fut le fondateur de la théorie des ensembles, à partir de 1874. Avant lui, le concept d'ensemble était plutôt basique, et avait été utilisé implicitement depuis les débuts des mathématiques, depuis Aristote. Personne n'avait compris que cette théorie avait des éléments non implicites. Avant Cantor, il n'y avait en fait que les ensembles finis (qui sont aisés à comprendre) et les ensembles infinis (qui étaient plutôt sujets à discussion philosophique). En prouvant qu'il y a une infinité de tailles d'ensembles infinis, Cantor a établi que les bases de cette théorie étaient non-triviales. La théorie des ensembles joue ainsi le rôle d'une théorie fondatrice pour les mathématiques modernes, parce qu'elle interprète des propositions relatives à des objets mathématiques (par exemple, nombres et fonctions) provenant de toutes les disciplines des mathématiques (comme l'algèbre, l'analyse et la topologie) en une seule théorie, et fournit un ensemble standard d'axiomes pour les prouver ou les infirmer. Les concepts de base de celle-ci sont aujourd'hui utilisés dans toutes les disciplines des mathématiques.

\vspace{0.4cm}

Dans une de ses premières publications, Cantor prouve que l'ensemble des nombres réels contient plus de nombres que l'ensemble des entiers naturels ; ce qui montre, pour la première fois, qu'il existe des ensembles infinis de tailles différentes. Il fut aussi le premier à apprécier l'importance des correspondances un pour un (les bijections) dans la théorie des ensembles. Il utilisa ce concept pour définir les ensembles finis et infinis, subdivisant ces derniers en ensembles dénombrables et non dénombrables.

\PESP{https://fr.wikipedia.org/wiki/Georg\_Cantor}
\end{His}