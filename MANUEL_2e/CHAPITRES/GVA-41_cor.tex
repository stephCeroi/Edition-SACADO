
Dans un repère \Oij, on donne les points $A(-3;2)$, $B(-2;5)$ et $C(3;-1)$.
Calculer les coordonnées du centre de gravité $G$ du triangle $ABC$. 

\vspace{0.5cm}
Dans un problème ouvert, il faut organiser sa pensée et mettre en place une stratégie de résolution. En voici une.
\vspace{0.5cm}

\begin{enumerate}
\item Le centre de gravité est l'intersection des médianes. Soit $I$ le milieu du segment $[AB]$, $G$ vérifie : $\overrightarrow{AG} = \frac{2}{3}\overrightarrow{AI}$

\item \textbf{Commençons par calculer les coordonnées de $I$}. 

$x_I =  \frac{x_B+x_C}{2} = \frac{-2+3}{2}= \frac{1}{2}$ 

$y_I =  \frac{y_B+y_C}{2}= \frac{5-1}{2}= 2$


$I \left(\frac{1}{2} ; 2 \right)$ 

\item \textbf{Traduire en termes de coordonnées l'égalité : $\overrightarrow{AG} = \frac{2}{3}\overrightarrow{AI}$.}


Soit $G$ le point du plan de coordonnées $(x_G;y_G)$.


$\overrightarrow{AG}(x_G-x_A;y_G-y_A)$ soit $\overrightarrow{AG}(x_G+3;y_G-2)$  et $\overrightarrow{AI}(x_I-x_A;y_I-yA)$ soit  $\overrightarrow{AI}\left(\frac{1}{2}+3;2-2 \right)$, $\overrightarrow{AI}\left(\frac{7}{2};0\right)$

\begin{tabular}{c}
$x_G+3 = \frac{2}{3} \times \frac{7}{2} $\\ 
$y_G-2 = \frac{2}{3} \times 0 $ \\ 
\end{tabular} 

 
\item \textbf{Résolution.}


$\Longleftrightarrow$
\begin{tabular}{c}
$x_G  = \frac{7}{3} -3$\\ 
$y_G = 2$ \\ 
\end{tabular} 
$\Longleftrightarrow$
\begin{tabular}{c}
$x_G  = \frac{-2}{3}$\\ 
$y_G = 2$ \\ 
\end{tabular}



\end{enumerate}
