\begin{titre}[Calculs numériques]

\Titre{Les racines carrées}{3}
\end{titre}

\begin{CpsCol}
\begin{description}
\item[$\square$] Manipuler les racines carrées
\end{description}
\end{CpsCol}

\begin{DefT}{La racine Carrée}
 Soit $a$ un nombre positif, la racine Carrée de $a$ notée, $\sqrt{a}$, est l'unique nombre positif tel que $\sqrt{a}^2=a$.
\end{DefT}  


\begin{ThT}{Les racines carrées\index{Racines carrées}}
Pour tous nombres $a$ et $b$ positifs, 
\begin{description}
\item $\sqrt{ab}=\sqrt{a}\sqrt{b}$
\item $\sqrt{\frac{a}{b}}=\frac{\sqrt{a}}{\sqrt{b}}$, $b >0$
\end{description}
\end{ThT}

\EPCN{Calculer }

On pose $x = \sqrt{3}$ et $y=\sqrt{2}$. Calculer

\begin{enumerate}
\begin{minipage}{0.5\linewidth}
\item $x^4-y$
\item $2x^2+2x+3$
\end{minipage}
\begin{minipage}{0.5\linewidth}
\item $(x+2)(x-4)$
\item $x^3 \times y^3$
\end{minipage}
\end{enumerate}

\EPCN{Raisonner. Calculer }

\begin{enumerate}
\item Écrire sous la forme $a\sqrt{b}$ les nombres suivants :

$A = \sqrt{8}$ , $B=\sqrt{27}$ , $C=\sqrt{20}$  , $D=\sqrt{18}$  , $E=\sqrt{12}$  , $F=\sqrt{72}$
\item Simplifie les écritures :

$A=\frac{\sqrt{20}}{4}$ , $B = \frac{2-\sqrt{8}}{4}$ , $C=\frac{3-\sqrt{27}}{3}$ , $D=\sqrt{18}\sqrt{2}$  , $E=\sqrt{12}+\sqrt{45}$  , $F=\sqrt{18} -\sqrt{72} +\sqrt{32}$

\end{enumerate}

 

\EPC{0}{FEA-3}{Représenter. Chercher.}
 

 
\EPC{1}{FEA-51}{Calculer }

\EPCN{Raisonner. Représenter. Calculer }

\begin{enumerate}
\item
\begin{enumerate}
\item Tracer un carré $ABCD$ de coté $a$.  
\item Calculer la longueur de la diagonale $AC^2=2a$.
\item En déduire que $AC=a\sqrt{2}$
\end{enumerate}
\item
Applications : 
\begin{enumerate}
\item Quelle est la longueur de la diagonale d'un carré de coté 5 ?
\item Quelle est la longueur de l'hypoténuse d'un triangle isocèle rectangle dont un coté mesure 3cm ?
\end{enumerate}
\end{enumerate}



\EPCN{ Calculer }

On considère un triangle ABC dont les cotés mesurent :$AB = 4\sqrt{3}$, $BC = 2\sqrt{12}$ et $CA= 4\sqrt{6}$. Quelle est la nature de ce triangle ?






\EPCNM{Raisonner. Représenter. Calculer } 

\begin{enumerate}
\item Tracer un triangle équilatéral $ABC$ de coté $a$. $I$ est le milieu de $[AB]$. 
\item Quelle est la longueur $AI$ ?
\item Démontrer que $IC^2=\frac{3a^2}{4}$.
\item Démontrer que la hauteur d'un triangle équilatéral est égale à $\frac{a\sqrt3}{2}$.
\item Quelle est la longueur de la hauteur d'un triangle équilatéral de coté 6 ?
\end{enumerate}



\EPCNM{Raisonner. Représenter. Calculer } 

 
On donne le cube $ABCDEFGH$ suivant de coté de longueur $a$. I est le milieu de $[AB]$. Calculer la longueur $IH$ en fonction de $a$.

\begin{tikzpicture}[line cap=round,line join=round,>=triangle 45,x=1.0cm,y=1.0cm]
\clip(-0.48,-3.68) rectangle (5.76,2.52);
\draw [line width=2.pt] (0.,-3.)-- (4.,-3.);
\draw [line width=2.pt] (4.,-3.)-- (5.,-2.);
\draw [line width=2.pt,dash pattern=on 3pt off 3pt] (5.,-2.)-- (1.,-2.);
\draw [line width=2.pt,dash pattern=on 3pt off 3pt] (1.,-2.)-- (0.,-3.);
\draw [line width=2.pt] (0.,1.)-- (4.,1.);
\draw [line width=2.pt] (4.,1.)-- (5.,2.);
\draw [line width=2.pt] (5.,2.)-- (1.,2.);
\draw [line width=2.pt] (1.,2.)-- (0.,1.);
\draw [line width=2.pt] (0.,1.)-- (0.,-3.);
\draw [line width=2.pt,dash pattern=on 3pt off 3pt] (1.,-2.)-- (1.,2.);
\draw [line width=2.pt] (4.,1.)-- (4.,-3.);
\draw [line width=2.pt] (5.,-2.)-- (5.,2.);
\draw [line width=2.pt,dash pattern=on 3pt off 3pt] (1.,2.)-- (2.,-3.);
\draw [line width=2.pt] (0.,-3.)-- (2.,-3.);
\draw [line width=2.pt] (1.,-2.88) -- (1.,-3.12);
\draw [line width=2.pt] (2.,-3.)-- (4.,-3.);
\draw [line width=2.pt] (3.,-2.88) -- (3.,-3.12);
\begin{scriptsize}
\draw [color=black] (0.,-3.)-- ++(-2.5pt,0 pt) -- ++(5.0pt,0 pt) ++(-2.5pt,-2.5pt) -- ++(0 pt,5.0pt);
\draw[color=black] (-0.3,-2.69) node {$A$};
\draw [color=black] (4.,-3.)-- ++(-2.5pt,0 pt) -- ++(5.0pt,0 pt) ++(-2.5pt,-2.5pt) -- ++(0 pt,5.0pt);
\draw[color=black] (4.18,-3.11) node {$B$};
\draw [color=black] (5.,-2.)-- ++(-2.5pt,0 pt) -- ++(5.0pt,0 pt) ++(-2.5pt,-2.5pt) -- ++(0 pt,5.0pt);
\draw[color=black] (5.14,-1.63) node {$C$};
\draw [color=black] (1.,-2.)-- ++(-2.5pt,0 pt) -- ++(5.0pt,0 pt) ++(-2.5pt,-2.5pt) -- ++(0 pt,5.0pt);
\draw[color=black] (0.72,-1.73) node {$D$};
\draw [color=black] (0.,1.)-- ++(-2.5pt,0 pt) -- ++(5.0pt,0 pt) ++(-2.5pt,-2.5pt) -- ++(0 pt,5.0pt);
\draw[color=black] (-0.16,1.27) node {$E$};
\draw [color=black] (4.,1.)-- ++(-2.5pt,0 pt) -- ++(5.0pt,0 pt) ++(-2.5pt,-2.5pt) -- ++(0 pt,5.0pt);
\draw[color=black] (3.84,1.37) node {$F$};
\draw [color=black] (5.,2.)-- ++(-2.5pt,0 pt) -- ++(5.0pt,0 pt) ++(-2.5pt,-2.5pt) -- ++(0 pt,5.0pt);
\draw[color=black] (5.14,2.37) node {$G$};
\draw [color=black] (1.,2.)-- ++(-2.5pt,0 pt) -- ++(5.0pt,0 pt) ++(-2.5pt,-2.5pt) -- ++(0 pt,5.0pt);
\draw[color=black] (1.14,2.37) node {$H$};
\draw [color=black] (2.,-3.)-- ++(-2.5pt,0 pt) -- ++(5.0pt,0 pt) ++(-2.5pt,-2.5pt) -- ++(0 pt,5.0pt);
\draw[color=black] (2.26,-3.21) node {$I$};
\end{scriptsize}
\end{tikzpicture}



\EPCNA{Raisonner. Représenter. Calculer } 

 
On donne la pyramide à base carrée $ABCD$ de coté de longueur $a$ et de sommet $S$ dont la hauteur est 10 cm.
Pour quelles valeurs de $a$, on a $15 < SA < 20$ ?


\begin{ThT}{Les racines carrées\index{Racines carrées}}
Pour tout nombre $a$, $\sqrt{a^2}= \vert a \vert$
\end{ThT}

\EPCNA{Calculer } 

On donne $E = \frac{2}{3}+\frac{17}{2} \times \frac{4}{3}$ et $F = \frac{\sqrt 6 \times \sqrt 3\times \sqrt{16} }{\sqrt 2}  $ 
 
Démontrer que les nombres E et F sont égaux
