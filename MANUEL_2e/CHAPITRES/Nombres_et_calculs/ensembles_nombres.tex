\chapter{Éléments de géométrie}
{https://sacado.xyz/qcm/parcours_show_course/0/117129}
{


 \begin{CpsCol}
\textbf{Les savoir-faire du parcours}
 \begin{itemize}
 \item \textbf{Utiliser des nombres pour calculer et résoudre des problèmes}
 \item Connaitre les ensembles de nombres
 \item  Identifier l'ensemble d'un nombre
 \end{itemize}
 \end{CpsCol}

}

\begin{pageCours}

\section{Ensembles de nombres}

\begin{DefT}{Ensemble des réels}
L'ensemble de tous les nombres connus en Seconde est appelé ensemble des nombres réels, noté $\R$.
\end{DefT}

\begin{DefT}{Ensemble de nombres}\index{Ensemble de nombres!Réels $\R$}
Les autres ensembles de nombres, inclus dans $\R$.
\begin{enumerate}
\item On appelle \textbf{entiers naturels} les nombres : 0 ; 1 ; 2 ; 3 . . . Leur ensemble est noté $\N$.\index{Ensemble de nombres! Entiers naturels $\N$}
On a donc : $\N =  \lbrace 0 ; 1 ; 2 ; 3 \cdots \rbrace $
\item  On appelle \textbf{entiers relatifs} les nombres entiers naturels et leurs symétriques par rapport à 0. Leur ensemble est noté $\Z$.\index{Ensemble de nombres! Entiers $\Z$}
On a donc : $\Z = \lbrace \cdots -3 ; -2 ; -1 ; 0 ; 1 ; 2 ; 3  \cdots  \rbrace$
\item  On appelle \textbf{nombres rationnels} les nombres de la forme $\frac{a}{b}$, $a$ et $b$ entiers et $b$ non nuls.  Leur ensemble est noté $\Q$. \index{Ensemble de nombres! Rationnels $\Q$}
On a donc : $\Q = \lbrace \cdots \frac{5}{3} ; -\frac{5}{7} ; -\frac{13}{22} \cdots   \rbrace$. Les réels non rationnels sont dit \textbf{irrationnels} : $\sqrt{2}$
\item Par construction, $\N$ est inclus dans $\Z$  est inclus dans $\D$  est inclus dans $\Q$  est inclus dans $\R$. Ces ensembles sont dits "emboités".
\item  On peut représenter l'ensemble des réels sur une droite graduée.
\begin{center}
\begin{tikzpicture}[line cap=round,line join=round,>=triangle 45,x=1.0cm,y=1.0cm]
\draw[->,color=black] (-4.36,0.) -- (10.66,0.);
\foreach \x in {-4.,-3.,-2.,-1.,1.,2.,3.,4.,5.,6.,7.,8.,9.,10.}
\draw[shift={(\x,0)},color=black] (0pt,2pt) -- (0pt,-2pt) node[below] {\footnotesize $\x$};
\draw[color=black] (0pt,-10pt) node[right] {\footnotesize $0$};
\clip(-4.36,-0.5) rectangle (10.66,0.5);
\end{tikzpicture}
 \end{center} 
\end{enumerate}
\end{DefT}


%\begin{Nt}
%
%Soit $M$ un point d'abscisse $a$, sur la droite graduée.\\
%On note $\vert a\vert$ la distance de $M$ à $O$.
%
%Ainsi, on peut établir que $\vert -a \vert = \vert a \vert$. En effet, deux nombres opposés sont à la même distance de l'origine.
%\end{Nt}

 

\begin{DefT}{Appartenance}\index{Ensemble!Appartenance}
Pour symboliser l'appartenance d'un nombre à un ensemble, on utilise le symbole $\in$ : 
$-2 \in \Z \quad \quad \frac{5}{3} \in \Q $ 
 
 $ \not\in $ signifie n'appartient pas : $-5 \not \in \N \quad \quad \pi \not\in \Q$
\end{DefT}


\begin{Nt}

\begin{description}[leftmargin=*]
\item  On appelle $\R^+$ l'ensemble des réels positifs, $\R^-$ l'ensemble des réels négatifs.
\item  On utilise une étoile pour enlever 0 d'un ensemble. $\R^*$ l'ensemble des réels non nuls.
\end{description}

\end{Nt}

\begin{DefT}{Nombres décimaux}
Les \textbf{nombres décimaux} sont des nombres rationnels dont le dénominateur est une puissance de 2, de 5 ou de 10 ou un produit de puissances de ces nombres.
\end{DefT}

\begin{Ex}
\begin{description}[leftmargin=*]
\item  $A=\frac{3}{25}$ est un nombre décimal car $25 = 5^2$, 25 est donc une puissance de $5$.
\item  $B=\frac{13}{20}$ est un nombre décimal car $20 = 2^2 \times 5$, $20$ est donc un produit d'une puissance de $2$ et de $5$.
\end{description}
\end{Ex}

\begin{Att}

$A=\frac{7}{15}$ n'est pas un nombre décimal. $15=3 \times 5$ n'est pas une puissance de $5$ mais un multiple de $5$ !

\end{Att}



\begin{DefT}{Nombre décimal périodique}
Le nombre $a_0,\underline{a_1a_2a_3}$ est un nombre décimal périodique de période $a_1a_2a_3$. Les chiffres $a_1$, $a_2$, $a_3$ se répètent indéfiniment.
\end{DefT}




%%\PESP{https://fr.wikipedia.org/wiki/D\%C3\%A9veloppement\_d\%C3\%A9cimal\_p\%C3\%A9riodique}
%%
%%\AV{https://www.youtube.com/watch?v=N_cDA6tF-40}{Conjecture de Cantor}

\end{pageCours} 

\begin{pageAD} 
 

\Sf{Connaitre les ensembles de nombres}
 
  
\begin{ExoCad}{Représenter.}{1234}{2}{0}{0}{0}{0}
 
 \begin{minipage}{0.48\linewidth}
Placer, dans le plus petit ensemble, les nombres suivants :
$$ -4 ~~ ;~~  \sqrt{36}~~  ; ~~ \frac{5}{7} ~~  ;~~  -\sqrt{28}~~ ;~~  \frac{72}{6} ~~  ; ~~ 2,3 ~~  ; ~~ 0 ;~~ -1,785 $$
\end{minipage}
\hfill
\begin{minipage}{0.48\linewidth}

\definecolor{qqqqff}{rgb}{0.,0.,1.}
\definecolor{qqwuqq}{rgb}{0.,0.39215686274509803,0.}
\definecolor{xfqqff}{rgb}{0.4980392156862745,0.,1.}
\definecolor{ffqqqq}{rgb}{1.,0.,0.}
\begin{tikzpicture}[line cap=round,line join=round,>=triangle 45,x=0.6782617187500004cm,y=0.6782617187500004cm]
\clip(-6.0955652959673365,-0.7955347642346219) rectangle (5.699291731705649,4.717946580732306);
\draw [rotate around={0.:(-2.,2.)},color=ffqqqq] (-2.,2.) ellipse (0.8410791202168588cm and 0.4973682009769633cm);
\draw [rotate around={-0.20389871046679844:(-1.6955307407221418,2.00727956851319)},color=xfqqff] (-1.6955307407221418,2.00727956851319) ellipse (1.661680538260386cm and 0.7868931785368128cm);
\draw [rotate around={-0.7690246825780399:(-0.8815627427186662,1.9688848516262334)},color=qqwuqq] (-0.8815627427186662,1.9688848516262334) ellipse (2.664297761397899cm and 1.2951216094501943cm);
\draw [rotate around={-0.002358769768743201:(-0.14948545367760763,1.9998003125678991)},color=qqqqff] (-0.14948545367760763,1.9998003125678991) ellipse (3.765828366589848cm and 1.832457538495793cm);
\draw [color=ffqqqq](-1.242473081456056,2.3296214542697094) node[anchor=north west] {$\mathbb{N}$};
\draw [color=xfqqff](0.26259982051263214,2.3142635675149267) node[anchor=north west] {$\mathbb{Z}$};
\draw [color=qqwuqq](2.6123564939943598,2.1989056807601443) node[anchor=north west] {$\mathbb{Q}$};
\draw [color=qqqqff](4.946755280721304,2.3142635675149267) node[anchor=north west] {$\mathbb{R}$};
\end{tikzpicture}
\end{minipage}

\end{ExoCad}


  
\begin{ExoCad}{Raisonner. Communiquer.}{1234}{0}{0}{0}{0}{0}

Complète par  $\in$ ou $\notin$.

\begin{enumerate}
\begin{minipage}{0.32\linewidth}
\item $\pi \ldots \ldots \R$
\item $-5 \ldots \ldots \Z$
\item $\dfrac{1}{2} \ldots \ldots \D$
\end{minipage}
\hfill
\begin{minipage}{0.32\linewidth}
\item $-9 \ldots \ldots \N$
\item $\dfrac{1}{2}  \ldots \ldots \R$
\item $\dfrac{7}{3}  \ldots \ldots \Q$
\end{minipage}
\hfill
\begin{minipage}{0.32\linewidth}
\item $\sqrt{4} \ldots \ldots \N$
\item $-1,5 \ldots \ldots \Z$
\item $\dfrac{12}{4}  \ldots \ldots \N$
\end{minipage}
\end{enumerate}
 
\end{ExoCad}
 
 
 
\begin{ExoCad}{Représenter.}{1234}{0}{0}{0}{0}{0}

Dans chaque cas, trouver, lorsque cela est possible, un nombre $x$ qui remplit les critères suivants :
 
\begin{enumerate}[leftmargin=*]
\item $x \not\in D$ et $x \in \R$  \point{1}
\item $x \in \Q$ et $x \not\in \Z$  \point{1}
\item $x \not\in \N$ et $x \in \Z$  \point{1}
\end{enumerate}
 
 \end{ExoCad}
 
\begin{ExoCad}{Raisonner.}{1234}{0}{0}{0}{0}{0}

Les affirmations sont-elles vraies ou fausses ?
\begin{enumerate}[leftmargin=*]
\item La différence de deux nombres entiers naturels est un entier naturel. \point{1}
\item Le quotient de deux nombres décimal est un nombre décimal. \point{1}
\item Le quotient de deux nombres premiers distincts peut être un entier relatif. \point{1}
\item Le quotient de deux nombres premiers distincts peut être un nombre décimal. \point{1}
\end{enumerate} 
 
 \end{ExoCad}


 
\begin{ExoCad}{Raisonner. Représenter.Calculer.}{1234}{0}{0}{0}{0}{0}

On pose $a=4\times \left(\frac{3}{5}\right)^3 \times \frac{20}{3}$. Montrer que $\sqrt{a}$ est un nombre décimal. \point{4}
 
\end{ExoCad}

 
 

   
\begin{ExoCad}{Calculer.}{1234}{0}{0}{0}{0}{0}
 
Démontrer que $0,\underline{9}=1$.   \point{5}
 
\end{ExoCad}


 
\end{pageAD}
 
 \begin{pageCours}
 
 
\section{Opérations d'ensembles de nombres}
 
 

\begin{DefT}{Inclusion}\index{Ensemble!Inclusion}
Un ensemble $A$ est inclus dans un ensemble $B$ lorsque tous les éléments de $A$ sont contenus dans $B$. On note $A \subset B$.
\end{DefT}


\begin{Rq}
\begin{description}
\item[•] Un ensemble \textbf{est inclus dans} un ensemble. $A$ et $B$ deux ensembles : $A \subset B$.
\item[•] Un élément \textbf{appartient à} un ensemble. $x$ un élément et $A$ un ensemble : $x \in B$.
\end{description}
\end{Rq}


\begin{DefT}{Complémentaire}\index{Ensemble!Complémentaire}
Soit $\Omega$ un ensemble contenant un ensemble $A$. On appelle complémentaire de $A$ dans $\Omega$, tous les éléments de $\Omega$ qui n'appartiennent pas à $A$.
\end{DefT}



\begin{Log}\index{Démonstration par l'absurde}\index{Contraposée}
\begin{description}[leftmargin=*]
\item La \textbf{contraposée} d'une implication "si A alors B" est l'implication : "si non(B) alors non(A)". Pour démontrer que "si A alors B" , on peut démonter que : "si non(B) alors non(A)".  En langage symbolique, on écrit  : $A \Longrightarrow B \Longleftrightarrow \rceil B \Longrightarrow \rceil A$ 
\item Une démonstration est appelée \textbf{démonstration par l'absurde} lorsqu'on démontre que la supposition posée \textit{a priori} mène à une absurdité.
\end{description}
\end{Log}

\section{Multiples et diviseurs}

\begin{DefT}{Multiple}\index{Multiple!Nombres}

Soit $a$ un nombre entier. Le nombre $m$ est dit \textbf{multiple} de $a$ s'il existe un entier $k \in \Z$ tel que $m=ka$.
\end{DefT}

\begin{Ex} 

$5 \times 7 = 35$ donc $35$ est un multiple de $5$ et aussi,  $35$ est un multiple de $7$.
\end{Ex}

\begin{DefT}{Diviseur}\index{Diviseur!Nombres}

Soit $n$ un nombre entier. Le nombre $q$ est dit \textbf{diviseur} de $n$ s'il existe un entier $k \in \Z$ tel que $n=dq$.
\end{DefT}

\begin{Ex} 

$48 = 6 \times 8$ donc $8$ est un diviseur de $48$ et aussi,  $6$ est diviseur de $48$.
\end{Ex}

\begin{DefT}{Nombre premier}\index{Nombre premier}

Un \textbf{nombre premier} est un nombre entier supérieur à 2 avec exactement 2 diviseurs : 1 et lui-même. 

\end{DefT}





\end{pageCours} 
\begin{pageAD} 
 

\Sf{Opérer avec les ensembles}
 
  
\begin{ExoCad}{Communiquer.}{1234}{2}{0}{0}{0}{0}

Recopie et complète par $\subset$, $\in$, $\not\subset$ ou $\notin$.

\begin{enumerate}
\begin{minipage}{0.32\linewidth}
\item $\N \ldots \ldots \R$
\item $-5 \ldots \ldots \Z$
\end{minipage}
\hfill
\begin{minipage}{0.32\linewidth}
\item $\left\lbrace 0;1;2 \right\rbrace \ldots \ldots \N$
\item $]-\infty;1] \ldots \ldots \R$
\end{minipage}
\hfill
\begin{minipage}{0.32\linewidth}
\item $\sqrt{3} \ldots \ldots \N$
\item $-1,5 \ldots \ldots \Z$
\end{minipage}
\end{enumerate}
 
\end{ExoCad}

\begin{ExoCad}{Raisonner.}{1234}{0}{0}{0}{0}{0}

Déterminer les complémentaires des ensembles suivants. On écrira : $\Omega \setminus A$ ou $\overline{A}$ puis on décrira l'ensemble clairement, par les éléments de cet ensemble ou une phrase.
\begin{enumerate}
\item $\Omega = \lbrace 1;2;3;4;5;6\rbrace$ et $A = \lbrace 1;2 \rbrace$. $\Omega \setminus A = $\point{1}
\item $\Omega = \lbrace -4;-2;-1;0;1;2;\rbrace$ et $A = \lbrace -2;1;2 \rbrace$.  $\Omega \setminus A = $\point{1}
\item $\Omega = \R$ et $A = \Q$. $\Omega \setminus A = $\point{1}
\item $\Omega = \R$ et $A = \R^-$. $\Omega \setminus A = $\point{1}
\end{enumerate} 
 \end{ExoCad}



\Sf{Utiliser les multiples et les diviseurs}

\begin{ExoCad}{Calculer.}{1234}{0}{0}{0}{0}{0}

Donner la liste des 30 premiers nombres premiers : 

\begin{tabularx}{\textwidth}{|X|X|X|X|X|X|X|X|X|X|}
\hline 
 &   &   &   &   & &  &   &   &   \\  
\hline 
 &   &   &   &   & &  &   &   &   \\  
\hline 
 &   &   &   &   & &  &   &   &   \\  
\hline 
\end{tabularx} 

\end{ExoCad}




\begin{ExoCad}{Calculer.}{1234}{0}{0}{0}{0}{0}


\begin{enumerate}
\item  51 est-il un nombre premier ? Justifier. \point{2}
\item  Décomposer 24 en produit de facteurs premiers.\point{2}
\end{enumerate}

\end{ExoCad}


\begin{ExoCad}{Calculer.}{1234}{0}{0}{0}{0}{0}
Vrai ou faux : Quel que soit l'entier $n$, $2n-1$ est un nombre premier. Justifier.\point{2}
\end{ExoCad}



\begin{ExoCad}{Calculer.}{1234}{0}{0}{0}{0}{0}


 
\end{ExoCad}



 
\end{pageAD}


%%%%%%%%%%%%%%%%%%%%%%%%%%%%%%%%%%%%%%%%%%%%%%%%%%%%%%%%%%%%%%%%%%%
%%%%  Niveau 1
%%%%%%%%%%%%%%%%%%%%%%%%%%%%%%%%%%%%%%%%%%%%%%%%%%%%%%%%%%%%%%%%%%%
\begin{pageExercices} 


 %%%%%%%%%%%%%%%%%%%%%%%%%%%
\begin{ExoCuN}{Représenter.}{2}{0}{0}{0}{0}

Démontrer que $0,\underline{1}$ est un nombre rationnel à préciser.
 
\end{ExoCuN}
 
%%%%%%%%%%%%%%%%%%%%%%%%%%%
\begin{ExoCuN}{Représenter.}{2}{0}{0}{0}{0}
Je suis un nombre à trois chiffres non nuls. Je suis divisible par 94. Changez l'ordre de mes chiffres et je deviens divisible par 49.
Qui suis-je ? 
\end{ExoCuN}


\begin{ExoCdN}{Chercher.communiquer.}{2}{0}{0}{0}{0}
 
Dans chaque cas, trouver, lorsque cela est possible, le nombre $x$ qui remplit les critères suivants :
\begin{enumerate}
\item $x \in \Q$ et $x \not\in \Z$
\item $x \in \R$ et $x \not\in \N$. 
\end{enumerate}
\end{ExoCdN}

\begin{ExoCdN}{Représenter. Raisonner.}{2}{0}{0}{0}{0}

Démontrer que $\sqrt{2}$ est un irrationnel ou encore que $\sqrt{2} \in \R\setminus\Q$

\end{ExoCdN}

%%%%%%%%%%%%%%%%%%%%%%%%%%%%%%%%%%%%%%%%%%%%%%%%%%%%%%%%%%%%%%%%%%%
\begin{ExoCtN}{Représenter.}{2}{1}{0}{0}{0}

\begin{enumerate}
\item Démontrer que $0,\underline{12}$ est un nombre rationnel à préciser.
\item Démontrer que $0,\underline{485}$ est un nombre rationnel à préciser.
\end{enumerate}
\end{ExoCtN}

%%%%%%%%%%%%%%%%%%%%%%%%%%%%%%%%%%%%%%%%%%%%%%%%%%%%%%%%%%%%%%%%%%%
\begin{ExoCtN}{Raisonner.}{1}{0}{0}{0}{0}

\begin{enumerate}
\item Démontrer que tout entier $n$ multiple de $9$ est aussi un multiple de $3$.
\item Démontrer que si $m$ est un multiple de $6$ alors $m$ est aussi un multiple de $3$.
\end{enumerate}
\end{ExoCtN}


\begin{ExoCtN}{Représenter.}{1}{0}{0}{0}{0}

Dans un pays où le système monétaire n’est constitué que de pièces de 3 et de 5, il s’agit d’aider les habitants en créant un programme  qui donne le nombre de pièces nécessaires à tout achat d’un montant entier supérieur ou égal à 8.

\hfill{{\footnotesize Source : d’après PISA, items libérés}}

\end{ExoCtN}


\end{pageExercices}

  
%%%%%%%%%%%%%%%%%%%%%%%%%%%%%%%%%%%%%%%%%%%%%%%%%%%%%%%%%%%%%%%%%%%
%%%%  Niveau 2
%%%%%%%%%%%%%%%%%%%%%%%%%%%%%%%%%%%%%%%%%%%%%%%%%%%%%%%%%%%%%%%%%%%



%\begin{pageParcoursd} 
% 
%%%%%%%%%%%%%%%%%%%%%%%%%%%%%%%%%%%%%%%%%%%%%%%%%%%%%%%%%%%%%%%%%%%%
%
%
% 
%%%%%%%%%%%%%%%%%%%%%%%%%%%%%%%%%%%%%%%%%%%%%%%%%%%%%%%%%%%%%%%%%%%%
%
%
%
%%%%%%%%%%%%%%%%%%%%%%%%%%%%%%%%%%%%%%%%%%%%%%%%%%%%%%%%%%%%%%%%%%%%
%
%
% %%%%%%%%%%%%%%%%%%%%%%%%%%%%%%%%%%%%%%%%%%%%%%%%%%%%%%%%%%%%%%%%%%%
%\begin{ExoCd}{Représenter. Raisonner.}{1234}{2}{0}{0}{0}{0}
%
%
%\end{ExoCd}
% 
%%%%%%%%%%%%%%%%%%%%%%%%%%%%%%%%%%%%%%%%%%%%%%%%%%%%%%%%%%%%%%%%%%%%
%\begin{ExoCd}{Représenter. Raisonner.}{1234}{2}{0}{0}{0}{0}
%
%
%\end{ExoCd}
% 
%\end{pageParcoursd}
%
%%%%%%%%%%%%%%%%%%%%%%%%%%%%%%%%%%%%%%%%%%%%%%%%%%%%%%%%%%%%%%%%%%%%
%%%%%  Niveau 3
%%%%%%%%%%%%%%%%%%%%%%%%%%%%%%%%%%%%%%%%%%%%%%%%%%%%%%%%%%%%%%%%%%%%
%\begin{pageParcourst}
%
%
%
%
%%%%%%%%%%%%%%%%%%%%%%%%%%%%%%%%%%%%%%%%%%%%%%%%%%%%%%%%%%%%%%%%%%%%
%\begin{ExoCt}{Raisonner.}{1234}{2}{0}{0}{0}{0}
% 
%\end{ExoCt}
%
%%%%%%%%%%%%%%%%%%%%%%%%%%%%%%%%%%%%%%%%%%%%%%%%%%%%%%%%%%%%%%%%%%%%
%\begin{ExoCt}{Représenter.}{1234}{2}{0}{0}{0}{0}
%
% 
%
%\end{ExoCt}
%
%%%%%%%%%%%%%%%%%%%%%%%%%%%%%%%%%%%%%%%%%%%%%%%%%%%%%%%%%%%%%%%%%%%%
%\begin{ExoCt}{Représenter.}{1234}{2}{0}{0}{0}{0}
%
% 
%
%\end{ExoCt} 
% 
%\end{pageParcourst}
%
%%%%%%%%%%%%%%%%%%%%%%%%%%%%%%%%%%%%%%%%%%%%%%%%%%%%%%%%%%%%%%%%%%%%
%%%%%  Brouillon
%%%%%%%%%%%%%%%%%%%%%%%%%%%%%%%%%%%%%%%%%%%%%%%%%%%%%%%%%%%%%%%%%%%%


\begin{pageBrouillon} 
 
\ligne{32}



\end{pageBrouillon}

%%%%%%%%%%%%%%%%%%%%%%%%%%%%%%%%%%%%%%%%%%%%%%%%%%%%%%%%%%%%%%%%%%%
%%%%  Auto
%%%%%%%%%%%%%%%%%%%%%%%%%%%%%%%%%%%%%%%%%%%%%%%%%%%%%%%%%%%%%%%%%%%


%%%%%%%%%%%%%%%%%%%%%%%%%%%%%%%%%%%%%%%%%%%%%%%%%%%%%%%%%%%%%%%%%%%
\begin{pageAuto} 


\begin{ExoAuto}{Raisonner.}{1234}{2}{0}{0}{0}{0}

 
%%%%%%%%%%%%%%%%%%%%%%%%%%%%%%%%%%%%%%%%%%%%%%%%%%%%%%%%%%%%%%%%%%%
\end{ExoAuto}

\begin{ExoAuto}{Raisonner.}{1234}{2}{0}{0}{0}{0}
  

\end{ExoAuto}

%%%%%%%%%%%%%%%%%%%%%%%%%%%%%%%%%%%%%%%%%%%%%%%%%%%%%%%%%%%%%%%%%%%
\begin{ExoAuto}{Raisonner.}{1234}{2}{0}{0}{0}{0}

 
 

\end{ExoAuto}

 
%%%%%%%%%%%%%%%%%%%%%%%%%%%%%%%%%%%%%%%%%%%%%%%%%%%%%%%%%%%%%%%%%%%
\begin{ExoAuto}{Raisonner.}{1234}{2}{0}{0}{0}{0}

 
 

\end{ExoAuto}


\end{pageAuto}
