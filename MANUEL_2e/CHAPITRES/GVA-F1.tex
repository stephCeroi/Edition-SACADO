\begin{titre}[Géométrie vectorielle et analytique]

\Titre{Repérage}{1}
\end{titre}

\begin{CpsCol}
\textbf{Se repérer dans le plan}
\begin{description}
\item[$\square$] Repérer un point dans le plan
\end{description}
\end{CpsCol}


\Rec{1}{GVA-21}

\begin{Rq}
Il existe plusieurs systèmes de repérages selon l'utilité et le travail demandé : 
\begin{description}
 \item[•] repère cartésien
 \item[•] repère polaire
 \item[•] repère sphérique
 \item[•] repère équatorial
 \end{description} 
\end{Rq}


\begin{DefT}{Repère cartésien}\index{Repère cartésien}
On appelle \textbf{repère cartésien du plan} deux axes gradués \textbf{orientés et sécants} en un point nommé l'origine du repère.
\end{DefT}


\begin{DefT}{Repère orthonormé}\index{Repère orthonormé}
Un repère est dit \textbf{orthonormé} lorsque les axes sont perpendiculaires et les unités sur les axes égales à 1.
\end{DefT}