
$SABCDT$ est un octoèdre régulier. On admet que les droites $(AC)$, $(BD)$ et $(ST)$ se coupent en $O$.


\definecolor{qqqqff}{rgb}{0.,0.,1.}
\begin{tikzpicture}[line cap=round,line join=round,>=triangle 45,x=1.0cm,y=1.0cm]
\clip(-0.5,-7.46) rectangle (6.72,0.58);
\draw [color=qqqqff] (0.,-4.)-- (4.,-4.);
\draw [color=qqqqff] (4.,-4.)-- (6.,-3.);
\draw [dash pattern=on 2pt off 2pt,color=qqqqff] (6.,-3.)-- (2.,-3.);
\draw [dash pattern=on 2pt off 2pt,color=qqqqff] (2.,-3.)-- (0.,-4.);
\draw [dash pattern=on 2pt off 2pt,color=qqqqff] (0.,-4.)-- (6.,-3.);
\draw [dash pattern=on 2pt off 2pt,color=qqqqff] (2.,-3.)-- (4.,-4.);
\draw [dash pattern=on 2pt off 2pt,color=qqqqff] (2.,-3.)-- (3.,0.);
\draw [dash pattern=on 2pt off 2pt,color=qqqqff] (2.,-3.)-- (3.,-7.);
\draw [color=qqqqff] (4.,-4.)-- (3.,-7.);
\draw [color=qqqqff] (6.,-3.)-- (3.,-7.);
\draw [color=qqqqff] (3.,-7.)-- (0.,-4.);
\draw [color=qqqqff] (0.,-4.)-- (3.,0.);
\draw [color=qqqqff] (3.,0.)-- (4.,-4.);
\draw [color=qqqqff] (3.,0.)-- (6.,-3.);
\draw [dash pattern=on 2pt off 2pt,color=qqqqff] (3.,0.)-- (3.,-7.);
\begin{scriptsize}
\draw [color=qqqqff] (0.,-4.)-- ++(-1.5pt,0 pt) -- ++(3.0pt,0 pt) ++(-1.5pt,-1.5pt) -- ++(0 pt,3.0pt);
\draw[color=qqqqff] (-0.28,-3.72) node {$A$};
\draw [color=qqqqff] (4.,-4.)-- ++(-1.5pt,0 pt) -- ++(3.0pt,0 pt) ++(-1.5pt,-1.5pt) -- ++(0 pt,3.0pt);
\draw[color=qqqqff] (4.14,-3.78) node {$B$};
\draw [color=qqqqff] (6.,-3.)-- ++(-1.5pt,0 pt) -- ++(3.0pt,0 pt) ++(-1.5pt,-1.5pt) -- ++(0 pt,3.0pt);
\draw[color=qqqqff] (6.14,-2.72) node {$C$};
\draw [color=qqqqff] (2.,-3.)-- ++(-1.5pt,0 pt) -- ++(3.0pt,0 pt) ++(-1.5pt,-1.5pt) -- ++(0 pt,3.0pt);
\draw[color=qqqqff] (1.74,-2.78) node {$D$};
\draw [color=qqqqff] (3.,0.)-- ++(-1.5pt,0 pt) -- ++(3.0pt,0 pt) ++(-1.5pt,-1.5pt) -- ++(0 pt,3.0pt);
\draw[color=qqqqff] (3.14,0.28) node {$S$};
\draw [color=qqqqff] (3.,-7.)-- ++(-1.5pt,0 pt) -- ++(3.0pt,0 pt) ++(-1.5pt,-1.5pt) -- ++(0 pt,3.0pt);
\draw[color=qqqqff] (3.22,-7.2) node {$T$};
\


\begin{enumerate}
\item Montrer que la droite $(BC)$ est parallèle au plan $(ADS)$.
\item Montrer que les points S, B, T et D sont coplanaires. En déduire que le quadrilatère $SBTD$ est un losange.
\item En déduire que la droite $(BT)$ est parallèle au plan $(ADS)$.
\item Montrer alors que les plans $(ADS)$ et $(BTC)$ sont parallèles.
\end{enumerate}