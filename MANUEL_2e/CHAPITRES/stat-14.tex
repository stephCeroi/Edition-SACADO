
Lors d'un concours de pèche en mer, le jury a répertorié le nombre de poissons en fonction de leur taille.

\begin{tabular}{|c|c|c|c|c|c|c|c|c|c|}
\hline  
Taille en cm& [10;20[& [20;30[& [30;40[ &[40;50[ &[50;60[& [60;70[& [70;80[& [80;90[& [90;100[\\ 
\hline  
Effectif &10 &7& 29& 25& 15& 12& 5& 6 &5\\ 
\hline 
\end{tabular}

\subsection*{Détermination de la médiane – Construction du tableau}
On souhaite déterminer la médiane de la série étudiée.
\begin{enumerate}
\item Construire, dans une feuille de calcul de tableur, le tableau ci-dessous.

\begin{tabular}{|c|c|c|c|c|c|c|c|c|c|c|}
\hline
\rowcolor{gray}&A &B &C &D& E &F &G &H& I& J\\
\hline
\cellcolor{gray}1& Taille en cm& [10;20[& [20;30[ &[30;40[ &[40;50[& [50;60[& [60;70[ &[70;80[ &[80;90[& [90;100[\\
\hline
\cellcolor{gray}2& Borne inf& 10& 20&&&&&&&\\
\hline
\cellcolor{gray}3& Borne sup &20& 30&&&&&&&\\
\hline
\cellcolor{gray}4& Effectif &10& 7& 29& 25& 15& 12 &5 &6& 5\\
\hline
\cellcolor{gray}5 &&&&&&&&&&\\
\hline
\cellcolor{gray}6 &&&&&&&&&&\\
\hline
\end{tabular}
\item Rechercher dans l'assistant de formule l'intitulé MEDIANE. Pourquoi ne peut-on pas l'utiliser ?
\item Dans la cellule B5, écrire =B2. Quelle valeur s'affiche-t-il ? Que fait le tableur ?
\item Dans la cellule C5, écrire une formule qui calcule B5 + C2. Quel est le résultat ?
\item Écrire dans la cellule A5 : effectifs cumulés croissants. Déterminer par glisser-copier toutes les
valeurs de la ligne 5. Quel est le nombre total de poissons péchés ? Complétez le tableau ci-dessus.
\item Déterminer la classe dans laquelle appartient la taille médiane des poissons.
\item La ligne 6 est la ligne des effectifs cumulés décroissants. Complétez le tableau ci-dessus.

\end{enumerate}


\subsection*{Lecture d'un tableau}

\begin{enumerate}
\item Quelle information donne la cellule E5 ? La cellule G5 ? La cellule I5 ?
\item Quelle information donne la cellule D6 ? La cellule E6 ? La cellule J6 ?
\item Quelle borne est-elle associée aux ECC ? aux ECD ?
\end{enumerate}


\subsection*{Utilisation d'un traceur}


\begin{enumerate}
\item  Tracer, à l'aidez du traceur Géogébra, le polygone des ECC et celui des ECD.
\item  Déterminer graphiquement la médiane (point d’intersection).
Remarque : 4 points suffisent pour déterminer la médiane. Lesquels ?
Autre méthode : Tracer la droite d'équation $y =\frac{\text{effectif total}}{2}$
\item  Où lit-on la médiane ?
\end{enumerate}