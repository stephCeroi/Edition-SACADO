
\begin{His}

{\color{brown}{\Large Représentation des nombres en informatique}}

\bigskip


Un ordinateur ne peut pas stocker ou traiter des nombres
ayant un nombre infini de chiffres. De nos
jours, les microprocesseurs traitent des nombres qui occupent
8 octets, soit 64 bits, et qui sont codés comme en notation
scientifique, mais en base 2 au lieu notre base 10 familière.

Par exemple, pour coder le nombre 11,2, on commence par l'écrire
en notation scientifique base 2 : $11,2=1,4\times 2^3$, 
puis on code la {\em mantisse} : 1,4, et l'{\em exposant } :  3, 
le tout en binaire.

Mais le codage en binaire de la mantisse peut réserver 
de mauvaises surprises : certains nombres simples, par exemple
1,4, ont une infinité de chiffres en binaire, de la même manière que, en base 10, certains nombres comme 1/7, ont une infinité de chiffres. 
Voici l'écriture en binaire de 1,3 :
\[ 1,4=\underline{1,0110~0110~0110~0110~0110~...}_2 \]
 
Du fait que le microprocesseur ne stocke qu'un certain nombre de chiffres
binaires, ce n'est pas réellement le nombre 1,4 qui est stocké, mais
un nombre proche, obtenu en coupant l'écriture 
$\underline{1,0110~0110~0110~0110~0110~...}_2$ à un certain nombre de chiffres.
De ce fait, au lieu de stocker et de calculer avec le nombre 1,4,
le processeur utilise le nombre

\[ \nombre{1,399999999999999911182158029987476766109466552734375} \]

et les calculs utilisant 1,4 seront donc (légèrement) faux, par exemple :

\begin{verbatim}
>>> 3 * 1.4 - 4.2
-8.881784197001252e-16
\end{verbatim}
(Le résultat est donné en  écriture scientifique, 
c'est $\nombre{-8,881784197001252}\times 10^{-16}$)

\medskip

Ces petites erreurs peuvent avoir de graves conséquences lorsqu'elles
s'accumulent. Ainsi, le 25 Février 1991, pendant la guerre du Golfe, un 
missile Patriot a échoué à intercepter un missile Scud, provoquant
la mort de 28 soldats. L'ordinateur du Patriot comptait les $10^e$ de 
seconde, or $1/10$ est aussi un nombre infini en binaire, et donc
ce n'était réellement 1/10 qui etait stocké. Après 100h de fonctionnement,
l'accumulation de cette petite erreur avait provoqué une erreur de 0,34s
entre la valeur calculée et la valeur réelle. C'était suffisant
pour que le Patriot râte sa cible.

\end{His}





