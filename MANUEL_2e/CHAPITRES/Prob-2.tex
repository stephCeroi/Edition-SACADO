
Le chevalier de Méré (1607-1684) prétend qu'il fallait lancer quatre fois le dé cubique pour avoir plus de chance d'obtenir un 6 que de ne pas en obtenir.
\begin{enumerate}
\item Que pensez vous de cette affirmation ? Vous pourrez vous appuyer sur une simulation à l'aide d'un tableur ou d'un programme.
\item Essayons de modéliser.
	\begin{enumerate}
		\item Construire un arbre de dénombrement représentant les 4 lancers.  
		\item Déterminer la probabilité de ne pas obtenir de 6.
		\item En déduire la probabilité d'obtenir un 6.	
		\item Que dire alors à l'affirmation du chevalier de Méré ?				
	\end{enumerate}
\end{enumerate}