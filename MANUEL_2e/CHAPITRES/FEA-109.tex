
Voici un script Python qui à tout entier $x \in [-3;4]$ calcule son image par la fonction $f$ définie par $f(x)=\frac{5x^3+x^2+2}{x-6}$ en arrondissant le résultat au centième.

\begin{lstlisting}
def maFonction(x):
    return round((5*x**3+x**2+2)/(x-6),2)

for x in range(-3,5) :
    y = maFonction(x)
    print(x,y)
 \end{lstlisting}   
    
\begin{enumerate}
\item Ouvrir le logiciel EduPython ou Pyscripter.
\item Taper ce code. Recopier les résultats dans le tableau ci dessous :

\begin{tabular}{|c|c|c|c|c|c|c|c|c|}
\hline 
$x$ & $-3$ & $-2$ & $-1$ & 0 & 1 & 2 & 3 & 4 \\ 
\hline 
$f(x)$ &   &   &   &   &   &   &   &  \\ 
\hline 
\end{tabular} 

\item Quelle est l'utilité de ce code ?

\item Quelle erreur renvoie le code suivant ? Justifier.

\begin{lstlisting}
def maFonction(x):
    return round((5*x**3+x**2+2)/(x-6),2)

for x in range(-3,7) :
    y = maFonction(x)
    print(x,y)
 \end{lstlisting}

\item Modifier le script pour qu'il calcule les images de $x \in [-5;10]$ par la fonction $f$ définie par $f(x)= x^2 +5x - 3 $ en arrondissant le résultat au dixième.

\item  

\begin{enumerate}
\item 
Taper le code suivant et analyser.
\begin{lstlisting}
import matplotlib.pyplot as plt
import numpy as np

x = np.linspace(-5, 5, 100)
image_de_x = x**2 +5*x - 3
plt.plot(x,image_de_x)
plt.show()
\end{lstlisting}
\item Quel est le domaine de définition de $f$ ?
\item Comment se nomme $f(x)$ dans ce script ?
\item  Quelle est l'utilité de ce code ?
\end{enumerate}  


 \end{enumerate}