
\begin{His}

{\large {\color{brown}Fraction égyptienne}}

\vspace{0.4cm}

Une fraction égyptienne est une somme de fractions unitaires, c'est-à-dire de fractions qui ont des numérateurs égaux à un et des dénominateurs entiers positifs, avec ces dénominateurs tous différents les uns des autres.

\vspace{0.2cm}

Tous les nombres rationnels positifs peuvent être écrits sous cette forme et ce, d'une infinité de façons différentes.

\vspace{0.2cm}

En effet, il est trivial d'exprimer toute fraction par une somme de fractions unitaires en se permettant de répéter les termes comme dans l'exemple :
$$\frac{2}{5}=\frac{1}{5}+\frac{1}{5}$$ 

\begin{wrapfigure}[10]{r}{3.6cm}
\vspace{-7mm}
\includegraphics[scale=0.5]{image_chapitres/fibonacci.jpg}
\unnumberedcaption{Leonardo \textbf{\textsc{Fibonacci}}} 
\end{wrapfigure}

Mais si l'on exige que tous les dénominateurs soient distincts, à l'instar des Égyptiens de l'Antiquité, cette représentation est toujours possible grâce à l'identité :
$$\frac{1}{a}=\frac{1}{a+1}+\frac{1}{a(a+1)}$$

que connaissait dès 1202 le grand mathématicien européen du Moyen Âge Leonardo \textsc{Fibonacci}.

\vspace{0.2cm}

Ainsi, en reprenant l'exemple ci-dessus : 2/5 = 1/5 + 1/6 + 1/30. En appliquant le même procédé à chacune des fractions unitaires, 2/5 peut donc s'exprimer comme une multitude de fractions égyptiennes.

\vspace{0.2cm}

On peut démontrer le même résultat en utilisant les séries harmoniques.

\vspace{0.2cm}

Ce type de sommes, utilisé pour exprimer les fractions par les anciens Égyptiens, a continué à faire l'objet d'études lors de la période médiévale. En notation mathématique moderne, les fractions égyptiennes ont été remplacées par les fractions ordinaires et la notation décimale. Néanmoins, les fractions égyptiennes continuent d'être un objet d'étude en théorie des nombres moderne et en mathématiques récréatives, aussi bien que dans les études historiques modernes des mathématiques anciennes.

Dans cet article, nous résumons ce qui est connu à propos des fractions égyptiennes à la fois anciennes et modernes. Pour les détails des sujets traités ici, voir les articles liés.

\vspace{0.4cm}

\PESP{https://fr.wikipedia.org/wiki/Fraction\_\%C3\%A9gyptienne}
\end{His}
