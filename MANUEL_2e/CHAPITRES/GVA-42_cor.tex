
Dans un repère \Oij, on donne les points $M(0;3)$, $N(2;3)$, $P(-9;0)$ et $Q(-1;1)$.
\begin{enumerate}
\item Calculer les coordonnées des points A et B tels que :
 \begin{enumerate}
\item $\overrightarrow{NA}=\frac{1}{2}\overrightarrow{MN}$


$\overrightarrow{NA}(x_A-x_N;y_A-y_N), \overrightarrow{NA}(x_A-2;y_A-3)$

$\overrightarrow{MN}(x_M-x_N;y_M-y_N), \overrightarrow{MN}(2-0;3-3), \overrightarrow{MN}(2;0)$

$\overrightarrow{NA}=\frac{1}{2}\overrightarrow{MN} \Longleftrightarrow  x_A-2 =\frac{1}{2} \times 2 \text{ et } y_A-3 = \frac{1}{2} \times 0 $

$x_A-2 =3 \text{ et } y_A  = 3$  \fbox{$x_A = 3 \text{ et } y_A  = 3  $}

\item $\overrightarrow{MB}=3\overrightarrow{MQ}$


$\overrightarrow{MB}(x_B-x_M;y_B-y_M), \overrightarrow{MB}(x_B-0;y_B-3)$

$\overrightarrow{MQ}(x_Q-x_M;y_Q-y_M), \overrightarrow{MQ}(-1-0;1-3), \overrightarrow{MN}(-1;-2)$

$\overrightarrow{MB}=3\overrightarrow{MQ} \Longleftrightarrow  x_B =3 \times (-1) \text{ et } y_B-3 =3\times (-2) $

\fbox{$x_B =-3 \text{ et } y_B  = -3  $}

\end{enumerate}
\item Calculer les coordonnées des vecteurs $\overrightarrow{PA}$ et $\overrightarrow{PB}$.


$\overrightarrow{PA}(x_A-x_P;y_A-y_P), \overrightarrow{PA}(3-(-9);3-0)$, \fbox{$\overrightarrow{PA}(12;3)$} 


$\overrightarrow{PB}(x_B-x_P;y_B-y_P), \overrightarrow{PB}(-3-(-9);-3-0)$, \fbox{$\overrightarrow{PB}(6;-3)$} 

\item Les droites $(PA)$ et $(PB)$ sont-elles parallèles ? Que peut-on déduire ?

$dét(\overrightarrow{PA}, \overrightarrow{PB}) = 12 \times (-3) - 6 \times 3 = -54 \neq 0$

Les droites $(PA)$ et $(PB)$ ne sont pas parallèles. Elles sont sécantes en $P$.



\end{enumerate}
