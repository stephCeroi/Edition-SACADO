\begin{titreTice}[Statistiques descriptives]

\TitreSansTemps{Utilisation de calculatrice}
\end{titreTice}


\begin{CpsCol}
\begin{description}
\item[$\square$] Savoir utiliser la calculatrice pour manipuler des série statistiques
\end{description}
\end{CpsCol}

\subsection*{Série explicite}

Une étude a été faite sur des œufs déposés dans des nids de petite taille.
Le tableau suivant donne en mm le diamètre des œufs.
$$19,8 ~ - 22,1 - 21,5 - 20,9 - 22 - 22,3 - 21 - 20,3 - 20,9 - 22 - 20,8 ~ – ~ 21,2 ~ – 21 ~ – ~ 20,8 ~ – ~  20,3 ~ - ~ 22$$
On souhaite utiliser la calculatrice pour obtenir la moyenne, la médiane, les quartiles.


\begin{minipage}[t]{0.48\linewidth}
\begin{tabular}{|p{\linewidth}|}
\textbf{CASIO}  \\ 
\hline
> Aller dans le menu \texttt{STAT} \\ 
\hline
> Vider la colonne \menu{DEL-A} avec \touche{F4}  \\ 
\hline
> Valider avec \menub{Yes}  la colonne  avec \touche{F1}  \\ 
\hline
> Entrer dans la première colonne chaque valeur \toucheEXE\\ 
\hline
> Entrer dans \menu{CALC} avec \touche{F2}  \\ 
\hline
> \menu{SET}, vérifier que \texttt{1VAR Xlist : List1}, numéro de
votre liste et \texttt{1Var Freq : 1} , valider avec \toucheEXE \\  
\hline
> Choisir \menub{1Var} \\ 
\hline
\end{tabular} 
\end{minipage}
\hfill
\begin{minipage}[t]{0.48\linewidth}

\begin{tabular}{|p{\linewidth}|} 
\textbf{TI 82 Stats et supérieur} \\ 
\hline
> Appuyer sur la touche \touche{STAT} \\ 
\hline
> Sélectionner  \menu{EDIT} \\ 
\hline
> \texttt{1: Edit...} puis confirmer avec \toucheEnter \\ 
\hline
> Sélectionner \menub{L1} \\  
\hline
> La vider si nécessaire $\Rightarrow$ Sélectionner \menu{EDIT}, \texttt{\textbf{4:ClrList}} \toucheEnter puis taper \toucheS \touche{1} \toucheEnter  \\ 
\hline
> Taper \touche{STAT}, rentrer les valeurs dans \texttt{L1} \\ 
\hline
> Calcul des paramètres : \touche{STAT} sélectionner \menu{CALC} \toucheEnter  \\ 
\hline
> Sélectionner \texttt{1:1Var-Stats} \toucheEnter \toucheEnter \\ 
\hline
\end{tabular} 
\end{minipage}

On lit alors la moyenne : $\overline{x}$, la médiane : $med$, et les quartiles : $Q_1$ et $Q_3$.


\subsection*{Série pondérée}

On veut déterminer les paramètres de la série suivante.

\begin{tabular}{|c|c|c|c|c|c|c|c|c|c|c|} 
\hline
Nids de roitelets &19,8& 20,8 &21 &21,2& 21,5& 21,7& 22 &22,1& 22,2\\
\hline
Effectifs& 6& 7 &9 &10& 11& 8 &7 &5& 4 \\
\hline
\end{tabular} 


\begin{minipage}[t]{0.48\linewidth}
\begin{tabular}{|p{\linewidth}|}
\textbf{CASIO}  \\ 
\hline
> Aller dans le menu \texttt{STAT} \\ 
\hline
> Vider la colonne \menu{DEL-A} avec \touche{F4}  \\ 
\hline
> Valider avec \menub{Yes}  la colonne  avec \touche{F1}  \\ 
\hline
> Entrer dans la première colonne chaque valeur \toucheEXE\\ 
\hline
> Entrer dans \menu{CALC} avec \touche{F2}  \\ 
\hline
> Entrer dans \menu{SET}, vérifier que \texttt{1VAR Xlist : List1} ou numéro de votre liste \\
\hline
> Choisir \texttt{1Var Freq : List2}. Valider avec \toucheEXE \\  
\hline
> Choisir \menub{1Var} \\ 
\hline
\end{tabular} 
\end{minipage}
\hfill
\begin{minipage}[t]{0.48\linewidth}
\begin{tabular}{|p{\linewidth}|} 

\textbf{TI 82 Stats et supérieur} \\ 
\hline
> Appuyer sur la touche \touche{STAT} \\
\hline 
> Sélectionner \texttt{EDIT 1: Edit} puis confirmer avec \toucheEnter \\ 
\hline
> Sélectionner la liste \texttt{List1} (vide) \\ 
\hline
> Rentrer les valeurs dans \texttt{L1} et dans \texttt{L2} \\ 
\hline
> Calcul des paramètres \\
\hline
> \touche{STAT} sélectionner \menub{CALC} \toucheEnter Sélectionner \texttt{1:1Var-Stats} \\
\hline
> \toucheS \touche{1}  \touche{,} \toucheS \touche{2} \toucheEnter  \toucheEnter \\ 
\hline
\end{tabular} 

\end{minipage}

On lit alors la moyenne : $\overline{x}$, la médiane : $med$, et les quartiles : $Q_1$ et $Q_3$.