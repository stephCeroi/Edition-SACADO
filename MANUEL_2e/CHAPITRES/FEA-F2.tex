\begin{titre}[Calcul littéral]

\Titre{Factorisation, développement, identité remarquable}{4}
\end{titre}


\begin{CpsCol}
\begin{description}
\item[$\square$] Développer, factoriser des expressions polynomiales simples.
\item[$\square$] Choisir la forme la plus adaptée (factorisée, développée réduite) d’une expression en
vue de la résolution d’un problème.
\end{description}
\end{CpsCol}

\begin{minipage}{0.5\linewidth}
\begin{ThT}{Les identités remarquables\index{Identités remarquables}}
Pour tous nombres $a$ et $b$, 
\begin{description}
\item $(a+b)^2=a^2+2ab+b^2$
\item $(a-b)^2=a^2-2ab+b^2$
\item $(a-b)(a+b)=a^2-b^2$
\end{description}
\end{ThT}
\end{minipage}
\begin{minipage}{0.5\linewidth}
\begin{Rq}
Dans la théorème ci-contre, le membre de gauche est la forme factorisée\index{identités remarquables!forme factorisée} et dans celui de droite est la forme développée de l'expression.\index{identités remarquables!forme développée}.
\end{Rq}
\end{minipage}

\EPC{1}{FEA-18}{Calculer }

\EPC{0}{FEA-23}{Calculer }

\EPC{1}{FEA-24}{Calculer }

\EPC{1}{FEA-25}{Calculer }

 
\EPC{0}{FEA-26}{Raisonner. Calculer. }

\EPC{1}{FEA-27}{Raisonner. Modéliser. Calculer. }
 
\EPC{1}{FEA-86}{Représenter. Modéliser. Calculer. }

\EPC{1}{FEA-39}{Calculer. }

\EPC{1}{FEA-96}{Raisonner. Calculer. }

\EPC{1}{FEA-97}{Raisonner. Modéliser. Calculer. }

\EPC{1}{FEA-98}{Raisonner. Modéliser. Calculer. }

\EPC{0}{FEA-99}{Raisonner. Modéliser. Calculer. }

\EPC{0}{FEA-100}{Calculer.}

\EPCP{1}{FEA-28}{Raisonner. Modéliser. Calculer. }

\paragraphe{Pour s'entrainer, encore et encore}

\EPC{0}{FEA-20}{Calculer }

\EPC{0}{FEA-25}{Calculer }


\paragraphe{Des défis}

\begin{minipage}{0.48\linewidth}
\PO{1}{FEA-29}
\end{minipage}
\hfill
\begin{minipage}{0.48\linewidth}
\PO{1}{FEA-30}
\end{minipage}
 
 
%\PO{1}{FEA-63}







