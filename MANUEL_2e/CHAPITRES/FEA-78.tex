
\paragraph{Parie A}

\begin{enumerate}
\item Tracer une droite graduée d'unité 3 cm.
\item Placer les points $M$ tel que $\vert OM \vert = 1$. On notera $M_1$ et $M_2$.
\item Placer les points $A$ tel que $\vert OA \vert = \frac{2}{3}$. On notera $A_1$ et $A_2$.
\end{enumerate}

Remarque : Dans ce type de question, on ne spécifie pas les 2 points que l'élève doit trouver ! C'est à l'élève de savoir ce qu'il doit faire. Ce type d'exercice se pose alors comme :

\paragraph{Parie B}

\begin{enumerate}
\item Tracer une droite graduée d'unité 2 cm.
\item Placer le point $A$ d'abscisse $a$ tel que $\vert a \vert = 2$.
\item Placer le point $B$ d'abscisse $b$ tel que $\vert b \vert = \frac{5}{2}$.
\item Placer le point $M$ d'abscisse $x$ tel que $OM = \frac{3}{4}$.
\end{enumerate}