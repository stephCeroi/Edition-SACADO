
\begin{enumerate}

\item Lors d'un lancer d'un dé, quelle est la probabilité d'obtenir une face paire ?

\item Mise en œuvre

\begin{list}{}{}
		\item Dans la cellule $A1$ écrire : Face 2
		\item Dans la cellule $A2$ écrire : Face 4
		\item Dans la cellule $A3$ écrire : Face 6
		\item Dans la cellule $A4$ écrire : Fréquence
		\item Dans la cellule $B5$ écrire : ech.1
		\item Dans la cellule $B6$ écrire : = ALEA.ENTRE.BORNES(1;6). 
		\item Glisser copier la cellule $B6$ jusqu'à la ligne 105 pour obtenir 100 lancers.
		\item Dans la cellule $B1$ écrire : =NB.SI($B6:B105$;2)
		\item Dans la cellule $B2$ écrire : =NB.SI($B6:B105$;4)
		\item Dans la cellule $B3$ écrire : =NB.SI($B6:B105$;6)
		\item Dans la cellule $B4$ faire calculer la somme des cellules $B1+B2+B3$ divisée par 100. 
		\item Donner la signification du résultat de la cellule $B4$.
	\end{list}

La fréquence observée est-elle égale à la probabilité de la question 1 ?De plus, on note une fluctuation des résultats.
\item Faire 50 simulations de ces 100 lancers. On pourra copier alors 50 fois la colonne B jusqu'à la colonne AY. On obtient alors 50 fréquences.
\item Utiliser l'assistant graphique pour illustrer la situation par un nuage de points.
\item Dans quel intervalle se situent au moins 95\% de ces fréquences ?
\end{enumerate}