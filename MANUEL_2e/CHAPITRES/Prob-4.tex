
La répartition des groupes sanguins dans la population française est présentée dans le tableau suivant :



\begin{tabular}{|c|c|c|c|c|c|}
\hline 
 &  & \multicolumn{4}{c|}{Groupe sanguin} \\ 
\hline 
 &  & O & A &B  & AB \\ 
\hline 
Rhésus & Rh+ & 37 & 39 & 7 & 2 \\ 
\hline 
Rhésus & Rh- & 6 & 6 & 2 & 1 \\ 
\hline 
\end{tabular} 
Groupe sanguin
O A B AB
Rhésus
Rh+ 37% 39% 7% 2%
Rh- 6% 6% 2% 1%
L'expérience aléatoire consiste à choisir au hasard une personne dans cette population. On assimile les


L'expérience aléatoire consiste à choisir au hasard une personne dans cette population. On assimile les
probabilités aux fréquences observées.

Quelles est la probabilité de chacun des événements :
\begin{enumerate}
\item A : «La personne est du groupe A»
\item Rh+ : «La personne est de rhésus positif»
\item AB− : «La personne est de groupe AB et de rhésus négatif»
\end{enumerate}
