
Le plan est muni d'un repère \Oij. 
On considère les points : $A(-1;-3), B(7;0), C(4;5)$ et $D(-4;2)$.
\begin{enumerate}
\item Démontrer que $ABCD$ est un parallélogramme.

$\overrightarrow{AB}(7-(-1);0-(-3)),\overrightarrow{AB}(8;3)$ et $\overrightarrow{DC}(4-(-4);5-2),\overrightarrow{DC}(8;3)$

$\overrightarrow{AB} = \overrightarrow{DC}$ donc $ABCD$ est un parallélogramme.



\item  Calculer les coordonnées des points $I$, $J$ et $K$ définis par les relations vectorielles : $$\overrightarrow{AI}=\frac{2}{3}\overrightarrow{AC} ~~ \text{et}~~\overrightarrow{BJ}=\frac{1}{2}\overrightarrow{CB} ~~ \text{et}~~ \overrightarrow{AK}=3\overrightarrow{AD}$$.

\begin{description}
\item[•] $ABCD$ est un parallélogramme donc $\overrightarrow{AC} = \overrightarrow{AB} + \overrightarrow{AD}$, or $\overrightarrow{AD}(-3;5)$ donc $\overrightarrow{AC}(5;8)$ 

$\overrightarrow{AI}=\frac{2}{3}\overrightarrow{AC} \Longleftrightarrow \overrightarrow{AI} \left( \frac{10}{3};\frac{16}{3}\right)$

Comme $A(-1;-3)$ et $\overrightarrow{AI}(x_I-x_A;y_I-y_A)$, on a : \fbox{$I\left( \frac{7}{3};\frac{7}{3}\right)$}


\item[•] $\overrightarrow{BJ}=\frac{1}{2}\overrightarrow{CB}$ et $\overrightarrow{CB}(3;-5)$  

Donc $\overrightarrow{BJ} \left( \frac{3}{2};-\frac{5}{2}\right)$

Comme $B(7;0)$ et $\overrightarrow{BJ}(x_J-7;y_J)$, on a : \fbox{$J\left( \frac{17}{2};-\frac{5}{2}\right)$}

\item[•] $\overrightarrow{AK}=3\overrightarrow{AD}$ et $\overrightarrow{AD}(-3;5)$  

Donc $\overrightarrow{AK} (-9;15)$.

Comme $A(-1;-3)$ et $\overrightarrow{AK}(x_K+1;y_K+3)$, on a : \fbox{$K\left( -10;12\right)$}

\end{description}

\item  Les points $I$, $J$ et $K$ sont-ils alignés ?

$\overrightarrow{IJ}\left(\frac{37}{6};-\frac{29}{6}\right)$ et $\overrightarrow{IK}\left(- \frac{37}{3};-\frac{43}{3}\right)$


$dét\left(\overrightarrow{IJ} ; \overrightarrow{IK}\right)=\frac{37}{6} \times \left(-\frac{43}{3}\right)-\left(-\frac{29}{6}\right)\times\left(- \frac{37}{3}\right) \neq 0 $

$I$, $J$ et $K$ sont-ils alignés 

\end{enumerate}