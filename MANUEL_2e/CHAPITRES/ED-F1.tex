\begin{titre}[Géométrie vectorielle et analytique]

\Titre{Droites parallèles, droites sécantes}{4}
\end{titre}



\begin{CpsCol}
\textbf{Résoudre des problèmes de géométrie dans le plan}
\begin{description}
\item[$\square$] Déterminer si deux droites sont parallèles ou sécantes.
\item[$\square$] Résoudre un système de deux équations linéaires à deux inconnues, déterminer le point d'intersection de deux droites sécantes.
\end{description}
\end{CpsCol}





\begin{Th}\index{Droites!Parallèles}
Dans un repère, deux droites $d$ et $d_1$ d'équation respectives $y=mxp+p$ et $y=m'x+p'$ sont parallèles si et seulement si leurs coefficients directeurs sont égaux, $m=m'$.
\end{Th}


\begin{Rq}
Deux droites sont sécantes si et seulement si leurs coefficients directeurs sont différents.
\end{Rq}


\begin{minipage}{0.47\linewidth}

\EPC{0}{ED-4}{Calculer.}

\EPC{1}{ED-5}{Calculer.}

\EPC{1}{ED-6}{Calculer.}

\EPC{0}{ED-17}{Modéliser. Calculer.}
\end{minipage}
\hfill
\begin{minipage}{0.47\linewidth}


\EPC{1}{ED-7}{Représenter. Calculer.}

\EPC{0}{ED-8}{Représenter. Calculer.}
\end{minipage}



