
Le montant des dépenses (en euros) de chaque client
lors d’une journée de soldes a été relevé et trié dans le
tableau ci-dessous où les fréquences sont exprimées en
pourcentage.

\begin{tabular}{|c|c|c|c|c|c|c|}
\hline 
montant des dépenses (en euros) & [10;30[ & [30;50[ & [50;70[ & [70;90[ & [90;110[ & [110;130[ \\ 
\hline 
Fréquence & 15 & 25 & 10 & 20 & 10 & 20 \\ 
\hline 
\end{tabular} 



\begin{enumerate}
\item Construire le polygone des fréquences cumulées
croissantes.
\item Déterminer par lecture graphique, une approximation
\begin{enumerate}
\item de la médiane ; 
\item du premier quartile ;
\item du troisième quartile.
\end{enumerate}
\item Interpréter ces résultats.
\item Déterminer une approximation de la moyenne.
\end{enumerate}

La lecture graphique est-elle possible ? Interpréter ce résultat.