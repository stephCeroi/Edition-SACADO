\chapter{Les ensembles de nombres et intervalles}
{https://sacado.xyz/qcm/parcours_show_course/0/117129}
{


 \begin{CpsCol}
\textbf{Les savoir-faire du parcours}
 \begin{itemize}
 \item \textbf{Utiliser des nombres pour calculer et résoudre des problèmes}
\item[$\square$] \textbf{Chercher :}  Tester, essayer plusieurs pistes de résolution.
\item[$\square$] \textbf{Représenter :} Produire et utiliser plusieurs représentations des nombres.
\item[$\square$] \textbf{Raisonner :} Mener collectivement une investigation en sachant prendre en compte le point de vue d’autrui.
\item[$\square$] \textbf{Communiquer :} Expliquer à l’oral ou à l’écrit (sa démarche, son raisonnement, un calcul, un protocole de construction géométrique, un algorithme), comprendre les explications d’un autre
et argumenter dans l’échange.
 \end{itemize}
 \end{CpsCol}

}

\begin{pageCours}


 \end{pageCours}

\begin{pageAD}


\begin{ExoCad}{Représenter. Raisonner.}{1234}{2}{0}{0}{0}{0}
 

\begin{tabular}{ccc}

$\frac{2}{10}..........\Z$ & $-\sqrt{25}..........\Z$ & $\frac{\sqrt{3}}{4}..........\Q$ \\ 

$\pi..........\R$  & $-\frac{5}{3}..........\Q$  &  $\sqrt{11}..........\R$ \\ 

\end{tabular} 

\end{ExoCad}


\begin{ExoCad}{Représenter. Raisonner.}{1234}{2}{0}{0}{0}{0}

Recopier et compléter le tableau.

%\begin{tabular}{|c|c|c|}
%\hline 
%Intervalle & Inégalité & Représentation  \vplus \\ 
%\hline 
%$x\in \left[ -5 ; \frac{2}{3}\right]$ & $-5  \leq x \leq  \frac{2}{3} $  &  \vplus \\ 
%\hline 
% & $-1 \leq x <4$ &  \vplus  \\ 
%\hline 
%$x\in \left[ 3 ; 6 \right[ $  &  &  \vplus  \\ 
%\hline 
% &  & \definecolor{ffdxqq}{rgb}{1.,0.8431372549019608,0.}
%\definecolor{ffxfqq}{rgb}{1.,0.4980392156862745,0.}
%\begin{tikzpicture}[line cap=round,line join=round,>=triangle 45,x=1.0cm,y=1.0cm]
%\draw[->,color=black] (-5.174092090680384,0.) -- (2.566282833730012,0.);
%\foreach \x in {-5.,-4.,-3.,-2.,-1.,1.,2.}
%\draw[shift={(\x,0)},color=black] (0pt,2pt) -- (0pt,-2pt) node[below] {\footnotesize $\x$};
%\draw[color=black] (0pt,-10pt) node[right] {\footnotesize $0$};
%\clip(-5.174092090680384,-0.4115875953650586) rectangle (2.566282833730012,0.4791698364123281);
%\draw [line width=2.4pt,color=ffxfqq] (-3.,0.)-- (2.,0.);
%\end{tikzpicture}  \vplus \\ 
%\hline 
%\end{tabular} 

\end{ExoCad}


\begin{ExoCad}{Représenter. Raisonner. Communiquer.}{1234}{2}{0}{0}{0}{0}
 

Déterminer l'ensemble des valeurs de $x$ dans chaque cas.
\begin{enumerate}
\item On jette un dé à 6 face et on regarde la face obtenue. Soit $x$ le numéro de la face. 
\item $[-1,1;3]$ et $[2,9;6]$
\item $x > -4$ et $x \leq 10$
\item $x \leq -3$ et $x \leq 5$
\item $x \leq 5$ ou $x \geq 2$
\end{enumerate}
\end{ExoCad}

\begin{ExoCad}{Représenter. Raisonner. Communiquer.}{1234}{2}{0}{0}{0}{0}
 


On propose dans chaque cas deux ensembles $A$ et $B$. Lequel est inclus dans l'autre ?  

\textit{{\small On pourra représenter chaque intervalle sur une droite graduée.}}

\begin{minipage}{0.48\linewidth}

 
\begin{enumerate}
\item $A = \left[ -\frac{11}{10};\frac{29}{10}\right]$ et $B=\left[-\frac{3}{2};3 \right]$
\item $A =\left[ \frac{1}{2}; +\infty \right[$ et $B=[0,7;0,8]$.
\item $A =[1;2]$ et $B=]1;2[$. 
\end{enumerate}

\end{minipage}
\hfill
\begin{minipage}{0.48\linewidth}
 
\begin{enumerate}

\item

\begin{tikzpicture}[line cap=round,line join=round,>=triangle 45,x=1.0cm,y=1.0cm]
\draw [->,line width=1.pt,domain=0.34:6.36] plot(\x,{(-14.-0.*\x)/7.});
\end{tikzpicture}
\item

\begin{tikzpicture}[line cap=round,line join=round,>=triangle 45,x=1.0cm,y=1.0cm]
\draw [->,line width=1.pt,domain=0.34:6.36] plot(\x,{(-14.-0.*\x)/7.});
\end{tikzpicture}
\item

\begin{tikzpicture}[line cap=round,line join=round,>=triangle 45,x=1.0cm,y=1.0cm]
\draw [->,line width=1.pt,domain=0.34:6.36] plot(\x,{(-14.-0.*\x)/7.});
\end{tikzpicture}
\end{enumerate}

\end{minipage}
 
 \end{ExoCad}
 
\begin{ExoCad}{Représenter. Raisonner. Communiquer.}{1234}{2}{0}{0}{0}{0}


Déterminer les intersections des ensembles suivants. On écrira : $A \cap B = $ où $A$ et $B$ sont les ensembles ci-dessous.
 

\textit{{\small On pourra représenter chaque intervalle sur une droite graduée.}}



\begin{minipage}{0.48\linewidth}

\begin{enumerate}
\item $\Z$ et $\Q$
\item $[-5;2[$ et $[0;7]$
\item $[-1;4]$ et $[-3;-1]$
\item $\N$ et $]-\infty;5]$
\item $[-5;0[$ et $[0;3]$
\end{enumerate}

\end{minipage}
\hfill
\begin{minipage}{0.48\linewidth}
 
\begin{enumerate}
\item

\begin{tikzpicture}[line cap=round,line join=round,>=triangle 45,x=1.0cm,y=1.0cm]
\draw [->,line width=1.pt,domain=0.34:6.36] plot(\x,{(-14.-0.*\x)/7.});
\end{tikzpicture}
\item

\begin{tikzpicture}[line cap=round,line join=round,>=triangle 45,x=1.0cm,y=1.0cm]
\draw [->,line width=1.pt,domain=0.34:6.36] plot(\x,{(-14.-0.*\x)/7.});
\end{tikzpicture}
\item

\begin{tikzpicture}[line cap=round,line join=round,>=triangle 45,x=1.0cm,y=1.0cm]
\draw [->,line width=1.pt,domain=0.34:6.36] plot(\x,{(-14.-0.*\x)/7.});
\end{tikzpicture}
\item

\begin{tikzpicture}[line cap=round,line join=round,>=triangle 45,x=1.0cm,y=1.0cm]
\draw [->,line width=1.pt,domain=0.34:6.36] plot(\x,{(-14.-0.*\x)/7.});
\end{tikzpicture}
\item

\begin{tikzpicture}[line cap=round,line join=round,>=triangle 45,x=1.0cm,y=1.0cm]
\draw [->,line width=1.pt,domain=0.34:6.36] plot(\x,{(-14.-0.*\x)/7.});
\end{tikzpicture}
\end{enumerate}

\end{minipage}
\end{ExoCad}

\begin{ExoCad}{Représenter. Raisonner. Communiquer.}{1234}{2}{0}{0}{0}{0}


Déterminer, dans chaque cas, la réunion des ensembles suivants. On écrira : $A \cup B = $ où $A$ et $B$ sont les ensembles ci-dessous.
\begin{enumerate}
\item $\Q$ et $\R$
\item $\left\lbrace 1;3;5;7  \right\rbrace $ et $\left\lbrace 0;2;4;8  \right\rbrace $
\item $[-3;4]$ et $[2;6]$
\item $[0;+\infty[$ et $]-\infty;5]$
\end{enumerate}
On pourra représenter chaque intervalle sur une droite graduée tracée à main levée.
 \end{ExoCad}

\begin{ExoCad}{Calculer. Représenter. Raisonner. Communiquer.}{1234}{2}{0}{0}{0}{0}

 

\begin{enumerate}
\item On considère le nombre $\frac{19}{11}$.

\begin{enumerate}
\item Donner le développement décimal de $\frac{19}{11}$ avec 8 chiffres significatifs. $\frac{19}{11}$ semble-t-il décimal ?
\item On dit que $\frac{19}{11}$ a une écriture périodique.
Préciser sa période (série de chiffres qui se répète à l'infini dans le développement décimal).
\end{enumerate}
\item On considère le nombre $x=0,13131313....$ dont le développement décimal a pour période 13.
\begin{enumerate}
\item Démontrer que $100x = 13 + x$. 
\item  En déduire une écriture fractionnaire de $x$. Quelle est la nature du nombre $x$ ?
\end{enumerate}
\item Démontrer que $x=3,412412412...$ est un nombre rationnel. 
\item Estimer le résultat avec la calculatrice.
\end{enumerate}
\end{ExoCad}





\begin{ExoCad}{Représenter. Chercher.}{1234}{2}{0}{0}{0}{0}
 
Soit $A$ et $B$ d'abscisse respective $4$ et $-2$.

\begin{enumerate}
\item Le point $I$ est le milieu de $[AB]$. Quelle est l'abscisse du point $I$ ? 
\item Soit $M$ le point d'abscisse $x$ de la droite $(AB)$. Calculer $IM$. 
\item Compléter le tableau suivant.

\begin{tabular}{|c|p{1.5cm}|c|p{1.5cm}|c|p{1.5cm}|}
\hline 
Abscisses de $M$ & $IM$ & Abscisses de $M$ & $IM$ & Abscisses de $M$ & $IM$ \\ 
\hline 
$1$ & & $-6$ &  & $-1$ & \\ 
\hline 
$-2$ & & $3$ &  & $1$ & \\ 
\hline 
$5$ & & $0$ &  & $4$ & \\ 
\hline 
\end{tabular}

\item  A quelle condition le point $M$ appartient-il au segment $[AB]$ ?

\end{enumerate}

On pourra se rendre à la page Se rendre à la page : \url{https://www.geogebra.org/m/jsvqnzbq} pour visualiser la situation. 
\end{ExoCad}
 

\begin{ExoCad}{Chercher. Communiquer.}{1234}{2}{0}{0}{0}{0}

 
\begin{enumerate}
\item Soit $A$ et $B$ d'abscisse respective $3$ et $-1$. Déterminer le rayon de l'intervalle $[AB]$.
\item Représenter le segment $[AB]$ par un intervalle puis par une inégalité.

\item Représenter sur la droite graduée le segment $[AB]$.
 
\end{enumerate}
  
\end{ExoCad}


 
 
\begin{ExoCad}{Représenter. Chercher.}{1234}{2}{0}{0}{0}{0}

On donne les segments $[AB]$ et $[EF]$ représentés ci-dessous.

\definecolor{ttqqqq}{rgb}{0.2,0.,0.}
\definecolor{qqzzff}{rgb}{0.,0.6,1.}
\definecolor{qqzzcc}{rgb}{0.,0.6,0.8}
\begin{tikzpicture}[line cap=round,line join=round,>=triangle 45,x=1.0cm,y=1.0cm]
\begin{axis}[
x=1.0cm,y=1.0cm,
axis lines=middle,
xmin=-2.200000000000002,
xmax=8.480000000000008,
ymin=-0.8000000000000048,
ymax=0.7799999999999957,
xtick={-2.0,-1.0,...,8.0},
ytick={-0.0,1.0,...,0.0},]
\clip(-2.2,-0.8) rectangle (8.48,0.78);
\draw [line width=2.pt,color=qqzzff] (-2.,0.)-- (4.,0.);
\draw [line width=2.pt] (5.,0.)-- (8.,0.);
\begin{scriptsize}
\draw [color=qqzzcc] (-2.,0.)-- ++(-2.5pt,0 pt) -- ++(5.0pt,0 pt) ++(-2.5pt,-2.5pt) -- ++(0 pt,5.0pt);
\draw[color=qqzzcc] (-1.86,0.37) node {$A$};
\draw [color=qqzzcc] (4.,0.)-- ++(-2.5pt,0 pt) -- ++(5.0pt,0 pt) ++(-2.5pt,-2.5pt) -- ++(0 pt,5.0pt);
\draw[color=qqzzcc] (3.96,0.37) node {$B$};
\draw [color=ttqqqq] (5.,0.)-- ++(-2.5pt,0 pt) -- ++(5.0pt,0 pt) ++(-2.5pt,-2.5pt) -- ++(0 pt,5.0pt);
\draw[color=ttqqqq] (5.14,0.37) node {$E$};
\draw [color=ttqqqq] (8.,0.)-- ++(-2.5pt,0 pt) -- ++(5.0pt,0 pt) ++(-2.5pt,-2.5pt) -- ++(0 pt,5.0pt);
\draw[color=ttqqqq] (8.14,0.37) node {$F$};
\end{scriptsize}
\end{axis}
\end{tikzpicture}


\begin{enumerate}
\item 
	\begin{enumerate}
		\item Déterminer le rayon de l'intervalle $[AB]$.
		\item Représenter $[AB]$ par une inégalité.
	\end{enumerate}
\item 
	\begin{enumerate}
		\item Déterminer le rayon de l'intervalle $[EF]$.
		\item Représenter $[EF]$ par une inégalité.
	\end{enumerate}
\end{enumerate}
 
\end{ExoCad}

\begin{ExoCad}{Représenter. Chercher.}{1234}{2}{0}{0}{0}{0}

Soit $x$ un réel.   


\begin{enumerate}
	\item Déterminer puis représenter l'ensemble des points $M$ d'abscisse $x$ tel que $\vert x- 3 \vert \leq 3$.
	\item Déterminer puis représenter l'ensemble des points $M$ d'abscisse $x$ tel que $\vert x+4 \vert \leq 1$.
	\item Déterminer puis représenter l'ensemble des points $M$ d'abscisse $x$ tel que $\vert x+\frac{2}{3} \vert \leq 4$.
\end{enumerate}
 
\end{ExoCad}

\begin{ExoCad}{Représenter. Chercher.}{1234}{2}{0}{0}{0}{0}

Soit $x$ un réel.   


\begin{enumerate}
	\item Déterminer puis représenter l'ensemble des points $M$ d'abscisse $x$ tel que $\vert x - \frac{4}{5} \vert \leq \frac{1}{2}$.
    \item Déterminer puis représenter l'ensemble des points $M$ d'abscisse $x$ tel que $\vert x- \pi \vert \leq 1$.
	\item Déterminer puis représenter l'ensemble des points $M$ d'abscisse $x$ tel que $\vert x - \sqrt{2}  \vert \leq  1$.
	\item Écrire à l'aide d'une double inégalité puis représenter  l'ensemble tel  que $\vert x + \frac{2}{3} \vert \leq 5$.
	\item Écrire à l'aide d'une double inégalité puis représenter  l'ensemble tel  que $\vert x+ \pi \vert \leq 10^{-1}$.	
\end{enumerate}
 
\end{ExoCad}

\begin{ExoCad}{Représenter. Chercher.}{1234}{2}{0}{0}{0}{0}

\begin{enumerate}
\item On s'intéresse à $\frac{3}{7}$.
\begin{enumerate}
	\item 1 est-elle une valeur approchée de $\frac{3}{7}$ à $ 10^{-1}$ près ?
	\item Déterminer une valeur approchée $a$ à $ 10^{-1}$ près de $\frac{3}{7}$.	
\end{enumerate}

\item
On s'intéresse à $\sqrt{10}$.
\begin{enumerate}
	\item Déterminer un encadrement de $\sqrt{10}$ à $ 10^{-2}$ près .
	\item Déterminer à la calculatrice  une valeur approchée de  $\sqrt{10}$.	
\end{enumerate}
\end{enumerate}
  
\end{ExoCad}


\end{pageAD}

%%%%%%%%%%%%%%%%%%%%%%%%%%%%%%%%%%%%%%%%%%%%%%%%%%%%%%%%%%%%%%%%%%%%%%%%%%%%%%%%%%%%%%%%%%%%%%%%%%%%%%%%%%%%%%%%%%%%%%%%%%%
%%%%%%%%%%%%%%%%%%%%%%%%%%%%%%%%%%%%%%%%%%%%%%%%%%%%%%%%%%%%%%%%%%%%%%%%%%%%%%%%%%%%%%%%%%%%%%%%%%%%%%%%%%%%%%%%%%%%%%%%%%%
%%%%%%%%%%%%%%%              pageParcoursu                         %%%%%%%%%%%%%%%%%%%%%%%%%%%%%%%%%%%%%%%%%%%%%%%%%%%%%%%%
%%%%%%%%%%%%%%%%%%%%%%%%%%%%%%%%%%%%%%%%%%%%%%%%%%%%%%%%%%%%%%%%%%%%%%%%%%%%%%%%%%%%%%%%%%%%%%%%%%%%%%%%%%%%%%%%%%%%%%%%%%%
%%%%%%%%%%%%%%%%%%%%%%%%%%%%%%%%%%%%%%%%%%%%%%%%%%%%%%%%%%%%%%%%%%%%%%%%%%%%%%%%%%%%%%%%%%%%%%%%%%%%%%%%%%%%%%%%%%%%%%%%%%%
\begin{pageParcoursu}

\begin{ExoCu}{Représenter. Chercher.}{1234}{2}{0}{0}{0}{0}
 
\end{ExoCu}

\begin{ExoCu}{Représenter. Chercher.}{1234}{2}{0}{0}{0}{0}
 
\end{ExoCu}

\begin{ExoCu}{Représenter. Chercher.}{1234}{2}{0}{0}{0}{0}
 
\end{ExoCu}


\end{pageParcoursu}
%%%%%%%%%%%%%%%%%%%%%%%%%%%%%%%%%%%%%%%%%%%%%%%%%%%%%%%%%%%%%%%%%%%%%%%%%%%%%%%%%%%%%%%%%%%%%%%%%%%%%%%%%%%%%%%%%%%%%%%%%%%
%%%%%%%%%%%%%%%%%%%%%%%%%%%%%%%%%%%%%%%%%%%%%%%%%%%%%%%%%%%%%%%%%%%%%%%%%%%%%%%%%%%%%%%%%%%%%%%%%%%%%%%%%%%%%%%%%%%%%%%%%%%
%%%%%%%%%%%%%%%              pageParcoursd                    %%%%%%%%%%%%%%%%%%%%%%%%%%%%%%%%%%%%%%%%%%%%%%%%%%%%%%%%%%%%%
%%%%%%%%%%%%%%%%%%%%%%%%%%%%%%%%%%%%%%%%%%%%%%%%%%%%%%%%%%%%%%%%%%%%%%%%%%%%%%%%%%%%%%%%%%%%%%%%%%%%%%%%%%%%%%%%%%%%%%%%%%%
%%%%%%%%%%%%%%%%%%%%%%%%%%%%%%%%%%%%%%%%%%%%%%%%%%%%%%%%%%%%%%%%%%%%%%%%%%%%%%%%%%%%%%%%%%%%%%%%%%%%%%%%%%%%%%%%%%%%%%%%%%%

\begin{pageParcoursd}

\begin{ExoCd}{Représenter. Chercher.}{1234}{2}{0}{0}{0}{0}
 
\end{ExoCd}

\begin{ExoCd}{Représenter. Chercher.}{1234}{2}{0}{0}{0}{0}
 
\end{ExoCd}

\begin{ExoCd}{Représenter. Chercher.}{1234}{2}{0}{0}{0}{0}
 
\end{ExoCd}


\end{pageParcoursd}

%%%%%%%%%%%%%%%%%%%%%%%%%%%%%%%%%%%%%%%%%%%%%%%%%%%%%%%%%%%%%%%%%%%%%%%%%%%%%%%%%%%%%%%%%%%%%%%%%%%%%%%%%%%%%%%%%%%%%%%%%%%
%%%%%%%%%%%%%%%%%%%%%%%%%%%%%%%%%%%%%%%%%%%%%%%%%%%%%%%%%%%%%%%%%%%%%%%%%%%%%%%%%%%%%%%%%%%%%%%%%%%%%%%%%%%%%%%%%%%%%%%%%%%
%%%%%%%%%%%%%%%            pageParcourst                      %%%%%%%%%%%%%%%%%%%%%%%%%%%%%%%%%%%%%%%%%%%%%%%%%%%%%%%%%%%%%
%%%%%%%%%%%%%%%%%%%%%%%%%%%%%%%%%%%%%%%%%%%%%%%%%%%%%%%%%%%%%%%%%%%%%%%%%%%%%%%%%%%%%%%%%%%%%%%%%%%%%%%%%%%%%%%%%%%%%%%%%%%
%%%%%%%%%%%%%%%%%%%%%%%%%%%%%%%%%%%%%%%%%%%%%%%%%%%%%%%%%%%%%%%%%%%%%%%%%%%%%%%%%%%%%%%%%%%%%%%%%%%%%%%%%%%%%%%%%%%%%%%%%%%

\begin{pageParcourst}

\begin{ExoCt}{Représenter. Chercher.}{1234}{2}{0}{0}{0}{0}
 
\end{ExoCt}

\begin{ExoCt}{Représenter. Chercher.}{1234}{2}{0}{0}{0}{0}
 
\end{ExoCt}

\begin{ExoCt}{Représenter. Chercher.}{1234}{2}{0}{0}{0}{0}
 
\end{ExoCt}


\end{pageParcourst}
%%%%%%%%%%%%%%%%%%%%%%%%%%%%%%%%%%%%%%%%%%%%%%%%%%%%%%%%%%%%%%%%%%%%%%%%%%%%%%%%%%%%%%%%%%%%%%%%%%%%%%%%%%%%%%%%%%%%%%%%%%%
%%%%%%%%%%%%%%%%%%%%%%%%%%%%%%%%%%%%%%%%%%%%%%%%%%%%%%%%%%%%%%%%%%%%%%%%%%%%%%%%%%%%%%%%%%%%%%%%%%%%%%%%%%%%%%%%%%%%%%%%%%%
%%%%%%%%%%%%%%%              pageAuto                         %%%%%%%%%%%%%%%%%%%%%%%%%%%%%%%%%%%%%%%%%%%%%%%%%%%%%%%%%%%%%
%%%%%%%%%%%%%%%%%%%%%%%%%%%%%%%%%%%%%%%%%%%%%%%%%%%%%%%%%%%%%%%%%%%%%%%%%%%%%%%%%%%%%%%%%%%%%%%%%%%%%%%%%%%%%%%%%%%%%%%%%%%
%%%%%%%%%%%%%%%%%%%%%%%%%%%%%%%%%%%%%%%%%%%%%%%%%%%%%%%%%%%%%%%%%%%%%%%%%%%%%%%%%%%%%%%%%%%%%%%%%%%%%%%%%%%%%%%%%%%%%%%%%%%
\begin{pageAuto}

\begin{ExoAuto}{Représenter. Chercher.}{1234}{2}{0}{0}{0}{0}
 
\end{ExoAuto}

\begin{ExoAuto}{Représenter. Chercher.}{1234}{2}{0}{0}{0}{0}
 
\end{ExoAuto}

\begin{ExoAuto}{Représenter. Chercher.}{1234}{2}{0}{0}{0}{0}
 
\end{ExoAuto}


\end{pageAuto}