 
\textit{Dans une classe de 30 élèves, 12 sont internes. Dans une autre classe de 25 élèves, 11 sont internes.
}

\begin{enumerate}

\item \textit{Calculer la proportion d'internes dans chaque classe.}


Soit $p_A$ la proportion d'internes dans la classe de 30. $p_A = \frac{12}{30}$ 

Soit $p_B$ la proportion d'internes dans la classe de 25. $p_B = \frac{11}{25}$ 

\item \textit{Calculer la proportion d'internes sur ces deux classes réunies.}

Soit $p$ la proportion d'internes dans les deux classes. $p  = \frac{12+11}{30+25}= \frac{23}{55}$ 


\end{enumerate}


