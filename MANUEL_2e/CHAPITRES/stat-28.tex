\subsection*{Série statistique discrète }

On a représenté ci-dessous le diagramme en bâtons des résultats d'un trimestre d'une classe de seconde de 35 élèves.

 
\includegraphics[scale=0.5]{../Archives/DS/DS4-stat_tableau_signe/exo1.eps} 

NB : Dans les questions suivantes, on arrondira, si nécessaire, les résultats à 0,1 près.

\begin{enumerate}
\item Compléter le tableau suivant : \hfill{2 points}

% \usepackage{array} is required
\begin{tabular}{|>{\centering\arraybackslash}p{3cm}|>{\centering\arraybackslash}p{1cm}|>{\centering\arraybackslash}p{1cm}|>{\centering\arraybackslash}p{1cm}|>{\centering\arraybackslash}p{1cm}|>{\centering\arraybackslash}p{1cm}|>{\centering\arraybackslash}p{1cm}|>{\centering\arraybackslash}p{1cm}|}
\hline 
Notes & 3 & 7 &  & 10 & 12 &  & 18  \rule[-7pt]{0pt}{20pt} \\ 
\hline 
Effectifs &  &  & 5 &  &  & 4 &  \rule[-7pt]{0pt}{20pt} \\ 
\hline 
Effectifs cumulés croissants &  &  &  &  &  & &   \rule[-7pt]{0pt}{20pt} \\ 
\hline 
\end{tabular} 




\item  
\begin{enumerate}
\item  Calculer la moyenne de cette série statistique.  
\item  Calculer l'étendue de cette série statistique. 
\end{enumerate}


\item  Calculer le pourcentage d'élèves ayant eu au moins 12 lors de ce trimestre.  
\item  
\begin{enumerate}

\item  Déterminer la médiane de cette série. 
\item  Déterminer le premier quartile.  
\item  Déterminer le troisième quartile.  
\end{enumerate}

\item   En utilisant la question précédente, compléter les affirmations suivantes : Lors de ce trimestre,
\begin{enumerate}
\item  environ 75 \% des élèves de cette classe ont une note supérieure ou égale à …........ 
\item  environ 75 \% des élèves de cette classe ont une note inférieure ou égale …........  
\item  environ la moitié des élèves de cette classe a une note inférieure ou égale à ….......... 
\item  environ …........ \% des élèves de cette classe ont une note comprise entre 8 et 12. 
\end{enumerate}


\end{enumerate}


\subsection*{Regroupement par classe }

Sur le livret scolaire, les notes de la classe sont regroupées en quatre classes données dans le tableau ci-dessous.
 
\begin{enumerate}

\item   Compléter ce tableau ci-dessous.  

\begin{tabular}{|c|c|c|c|c|}
\hline 
Notes & [0;5[ & [5;10[ & [10;15[ & [15;20] \rule[-7pt]{0pt}{20pt} \\ 
\hline 
Effectifs &  & 8 &  &  \rule[-7pt]{0pt}{20pt}  \\ 
\hline 
Effectifs cumulés croissants &  &  &  &  \rule[-7pt]{0pt}{20pt}  \\ 
\hline 
\end{tabular} 

\item  Calculer la moyenne de cette série  statistique. 
\item  Construire le polygone des effectifs cumulés croissants. 

\includegraphics[scale=0.5]{../Archives/DS/DS4-stat_tableau_signe/exo1-1.eps} 
\begin{enumerate}

\item  Déterminer la médiane de cette série. 
\item  Déterminer le premier quartile.  
\item  Déterminer le troisième quartile. 
\end{enumerate}
\end{enumerate}