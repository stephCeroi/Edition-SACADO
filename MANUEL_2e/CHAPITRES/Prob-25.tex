
Python requiert des bibliothèques pour chaque utilisation particulière ( tkinter, math, ...). Pour travailler sur des probabilités et simuler avec la fonction random, il est nécessaire d'appeler la fonction random en début de programme avec \texttt{\color{orange}import \color{black} random}


\begin{minipage}{0.48\linewidth}

\subsubsection*{Simulation du lancé d'un dé}
\begin{lstlisting}
import random
x = random.randint(1,6)
print(x)
\end{lstlisting}
\end{minipage}
\hfill
\begin{minipage}{0.48\linewidth}
\subsubsection*{Afficher une simulation de 50 lancers d'un dé à 6 faces}

\begin{lstlisting}
import random
for  i  in range(50):
	x = random.randint(1,6)
	print(x)
\end{lstlisting}
\end{minipage}

\subsubsection*{Application : $n$ Lancers d'une pièce.}

La probabilité d'obtenir une face lorsque la pièce n'est pas non truquée est égale à $\frac{1}{2}$. Ce résultat semble évident. On peut le simuler par une \textbf{approche fréquentiste de Bayès}.
 
On souhaite connaitre la fréquence de sortie de la face "Pile" lors de $n$ lancers d'une pièce. 

On note 0 la face "Face" et 1 la face "Pile".

Écrire un programme pour une valeur variable $n$ puis tester le programme pour $n=10$, $n=1000$, $n=1000000$ et Conclure
 

%\vspace{0.4cm}
%On pourra tester ce programme.
%
%\vspace{0.4cm}
%
%
%
%\color{orange} import \color{black} random
%
%n = \color{purple}int\color{black}(\color{purple}input\color{black}(\color{black}"Entrer le nombre de lancer "\color{black}))
%
%k = 0
%
%\color{orange} for \color{black} i \color{orange} in \color{purple}range\color{black}(n):
%
%\hspace{0.4cm}     x = random.randint(0,1)
%
%\hspace{0.4cm}   \color{orange} if \color{black} x = 1 :
%
%\hspace{0.8cm}   k=k+1
%
%f=k/n
%
%\color{purple} print \color{black}(f)