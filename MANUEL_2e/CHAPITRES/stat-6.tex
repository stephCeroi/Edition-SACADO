
Un distributeur automatique de café propose des expressos. Une pesée sur 30 expressos a donné les masses
suivantes (en grammes) de café utilisé :

\begin{tabular}{|c|c|c|c|c|c|c|c|c|c|}
\hline 
81& 82& 85& 83 &83 &82 &87 &84& 85 &84\\ 
\hline 
84 &81 &83 &86 &84 &80& 80& 79 &87& 85\\ 
\hline 
81 &82 &85 &87 &79 &80& 86 &89& 83 &89\\ 
\hline 
\end{tabular} 

\begin{enumerate}
\item Reproduire et compléter le tableau ci-dessous par les valeurs approchées au centième des fréquences
et par les fréquences cumulées décroissantes (FCD).

\begin{tabular}{|c|c|c|c|c|}
\hline 
Masse (en g) &[79 ;82[& [82 ;85[ &[85 ;88[ &[88 ;91[\\
\hline 
Fréquence&&&&\\
\hline 
FCD&&&&\\
\hline 
\end{tabular} 

\item  Représenter la courbe des FCD.
\item  Par lecture graphique, recopier et compléter : « 75\% des expressos contiennent plus de $\cdots$g de
café ».
\end{enumerate}