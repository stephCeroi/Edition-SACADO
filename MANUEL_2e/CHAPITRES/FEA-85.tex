

Le balayage est une méthode pour trouver une valeur approchée particulièrement facile à implémenter. Elle consiste à  obtenir un encadrement à $10^{-n}$ près de la valeur de référence, dont on sait qu'elle est comprise entre les deux entiers $a$ et $b$. 

L'algorithme suivant est incomplet. Il détermine par balayage une valeur approchée de $\sqrt{n}$, $n$ connu, avec un pas de 0,1.

$p \longleftarrow 0,1$

$a \longleftarrow 3$ 

Tant que $\vert a- \sqrt{n} \vert > p$

\hspace{1cm} $a= a+p$ 

Afficher $a$


\begin{minipage}{0.5\linewidth}


\textbf{1.} On choisit $n= 12$. Faire fonctionner cet algorithme et compléter le tableau.


\begin{tabular}{|c|p{2cm}|c|}
\hline 
Etape & a & condition \\ 
\hline 
Initialisation & 3 & VRAI \\ 
\hline 
Etape 1 &   &   \\ 
\hline 
Etape 2 &   &   \\ 
\hline 
Etape ...... &   &   \\ 
\hline 
Etape ...... &   &   \\ 
\hline 
Etape ...... &   &   \\ 
\hline 
Etape ...... &   &   \\ 
\hline
\end{tabular}
 
\end{minipage} 
\begin{minipage}{0.5\linewidth}

\textbf{2.} Python est un langage de programmation qui permet d'écrire des algorithmes réalisables par des ordinateurs. L'algorithme précédent s'écrit :

\begin{lstlisting}
from math import *  
p = 0.1 
a = 3 
while abs(  a- sqrt(12)) > p : 
    a= a+p  
print("Une valeur approchee est {}".format($a$)) 
\end{lstlisting}
 
 Tester ce code.
 
\textbf{3.}  Modifier ce code pour déterminer toutes les valeurs  approchées de $\sqrt{12}$ à $10^{-2}$ près.
  
\textbf{4.}  Modifier ce code pour déterminer une valeur approchée de $\sqrt{31}$ à $10^{-3}$ près.
 
\textbf{5.}  Modifier ce code pour déterminer une valeur approchée de $\pi$ à $10^{-10}$ près. Avec la bibliothèque \texttt {math}, $\pi$ s'écrit pi.
\end{minipage}