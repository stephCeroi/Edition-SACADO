
En utilisant la relation de Chasles, compléter les égalités suivantes :
\begin{enumerate}
 \item 
$\overrightarrow{AB}$ =  $\overrightarrow{AC}+\overrightarrow{CB}$ 
   \item  $\overrightarrow{AE}$ =  $\overrightarrow{AC}+\overrightarrow{CE}$ 
   
    \item  $\overrightarrow{DF}$ = $\overrightarrow{DC}+\overrightarrow{CF}$                     
\end{enumerate} 
    
 $ABCD$ est un parallélogramme.   
 
 
 \definecolor{zzttqq}{rgb}{0.6,0.2,0.}
\definecolor{ududff}{rgb}{0.30196078431372547,0.30196078431372547,1.}
\begin{tikzpicture}[line cap=round,line join=round,>=triangle 45,x=1.0cm,y=1.0cm]
\clip(0.2,-1.) rectangle (6.1,3.88);
\draw [line width=2.pt,color=zzttqq] (1.,2.)-- (5.,3.);
\draw [line width=2.pt,color=zzttqq] (5.,3.)-- (5.,1.);
\draw [line width=2.pt,color=zzttqq] (5.,1.)-- (1.,0.);
\draw [line width=2.pt,color=zzttqq] (1.,0.)-- (1.,2.);
\begin{scriptsize}
\draw [fill=ududff] (1.,2.) circle (2.5pt);
\draw[color=ududff] (1.14,2.37) node {$A$};
\draw [fill=ududff] (5.,3.) circle (2.5pt);
\draw[color=ududff] (5.14,3.37) node {$B$};
\draw [fill=ududff] (5.,1.) circle (2.5pt);
\draw[color=ududff] (5.14,1.37) node {$C$};
\draw [fill=ududff] (1.,0.) circle (2.5pt);
\draw[color=ududff] (1.,-0.33) node {$D$};
\end{scriptsize}
\end{tikzpicture}
 
 
 
 
 
 \begin{enumerate}   
  \item  $\overrightarrow{AC}=\overrightarrow{AD}+\overrightarrow{AB}$                   
   \item  $\overrightarrow{AD} =\overrightarrow{BC}$                   
    \item  $\overrightarrow{BD}=\overrightarrow{BA}+\overrightarrow{BC}$                   
  
 \end{enumerate} 