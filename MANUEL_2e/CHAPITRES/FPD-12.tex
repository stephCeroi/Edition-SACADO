
Un pompiste revend le litre d'essence au prix de 1,20 euros, alors que le prix d'achat est pour lui de 0,85 euros. Le pompiste sait alors qu'il peut compter sur une vente journalière de 100 litres. Mais il sait aussi qu'à chaque baisse de 0,01  euros, il vend 100 litres d'essence supplémentaires par jour.

Quel est le prix de vente d'un litre d'essence pour lequel le bénéfice du pompiste est le plus grand ?

Démontrer que le pompiste pourrait vendre à perte s'il ne fait pas attention.

\hfill{\rotatebox{180}{\textit{Aide : Le problème peut se ramener à la forme :} $(-n+35)(n+1)$}}

