 
Écrire un algorithme en langage naturel qui teste si trois points $A(x_A;y_A)$, $B(x_B;y_B)$, $C(x_C;y_C)$ sont alignés. Coder cet algorithme avec Python.

\begin{verbatim}
def determinant(a,b,c,d): # premier vecteur(a,b), second vecteur(c,d)
    det = a*d-b*c
    if det == 0 :
        return "Les points A, B et C sont alignés."
    else :
        return "Les points A, B et C ne sont pas alignés."  


a = float(input("Quelle est l'abscisse du point A ?"))
b = float(input("Quelle est l'ordonnée du point A ?"))

c = float(input("Quelle est l'abscisse du point B ?"))
d = float(input("Quelle est l'ordonnée du point B ?"))

e = float(input("Quelle est l'abscisse du point C ?"))
f = float(input("Quelle est l'ordonnée du point C ?"))

# Coordonnées du premier vecteur
abs1 = c-a #abscisse
ord1 = d-b #ordonnée

# Coordonnées du second vecteur
abs2 = e-a #abscisse
ord2 = f-b #ordonnée

print(determinant(abs1,ord1,abs2,ord2))
\end{verbatim}