
Voici un algorithme écrit en Algobox.

\begin{algobox}
\Variables
\Ligne n EST\_DU\_TYPE NOMBRE
\Ligne q EST\_DU\_TYPE NOMBRE
\DebutAlgo
\Ligne Lire $n$
\Ligne q PREND\_LA\_VALEUR (n+2)*(n+2)
\Ligne q PREND\_LA\_VALEUR q-(n+4)
\Ligne q PREND\_LA\_VALEUR q/(n+3)
\Ligne AFFICHER q
\FinAlgo
\end{algobox}


\begin{enumerate}
\item Tester cet algorithme avec $n=4$ puis$n=7$. 
\item Un élève a saisi $n=-3$. Que se passe-t-il ? Pourquoi ? 
\item Émettre une conjecture sur le résultat fourni par cet algorithme.
\item Démontrer algébriquement cette conjecture. 
\item Programmer cet algorithme.
\end{enumerate}