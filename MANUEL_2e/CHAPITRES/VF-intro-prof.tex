

\section{Les fonctions}


\begin{description}
\item[Public cible :] Enseignant de mathématique
\item[Durée :] les durées sont notées sur chaque fiche de séance.
\end{description}

\subsection{Compétences du cycle 5}

Les 6 compétences mathématiques, 

\begin{description}
\item[•] chercher, expérimenter – en particulier à l'aide d'outils logiciels ;
\item[•] modéliser, faire une simulation, valider ou invalider un modèle ;
\item[•] représenter, choisir un cadre (numérique, algébrique, géométrique …), changer de registre ;
\item[•] raisonner, démontrer, trouver des résultats partiels et les mettre en perspective ;
\item[•] calculer, appliquer des techniques et mettre en œuvre des algorithmes ;
\item[•] communiquer un résultat par oral ou par écrit, expliquer oralement une démarche.
\end{description}


\subsection{Objectifs de la séquence}

l'objectif des séquences d'analyse est de rendre les élèves capables d'étudier :

\begin{description}
\item[•]  un problème se ramenant à une équation du type $f(x)=k$ et de le résoudre dans le cas où la fonction est donnée (définie par une courbe, un tableau de données, une formule) et aussi lorsque toute autonomie est laissée pour associer au problème divers aspects d’une fonction ;
\item[•]  un problème d'optimisation ou un problème du type $f(x)< k$ et de le résoudre, selon les cas, en exploitant les potentialités de logiciels, graphiquement ou algébriquement, toute autonomie pouvant être laissée pour associer au problème une fonction.
\end{description}


\subsection{Contenus en lien avec le BO}

\begin{tabularx}{\textwidth}{|X|X|X|}
\hline 
CONTENUS & CAPACITÉS ATTENDUES & COMMENTAIRES \\ 
\hline
 
Étude qualitative de
fonctions 
& 
• Décrire, avec un vocabulaire
adapté ou un tableau de
variations, le comportement
d’une fonction définie par une
courbe.

• Dessiner une représentation
graphique compatible avec un
tableau de variations. 


& 
Les élèves doivent distinguer les
courbes pour lesquelles l’information
sur les variations est exhaustive, de
celles obtenues sur un écran graphique. \\ 

 & Lorsque le sens de variation
est donné, par une phrase ou
un tableau de variations :

• comparer les images de deux
nombres d’un intervalle ;

• déterminer tous les nombres
dont l’image est supérieure (ou
inférieure) à une image
donnée. & Les définitions formelles d’une fonction
croissante, d’une fonction décroissante,
sont progressivement dégagées. Leur
maîtrise est un objectif de fin d’année.

$\diamond$ Même si les logiciels traceurs de
courbes permettent d’obtenir rapidement
la représentation graphique d’une
fonction définie par une formule
algébrique, il est intéressant, notamment
pour les fonctions définies par
morceaux, de faire écrire aux élèves un
algorithme de tracé de courbe.

$\leftrightarrows$ Étude des signaux périodiques en
physique. \\ 
\hline 
\end{tabularx} 