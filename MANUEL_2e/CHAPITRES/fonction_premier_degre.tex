\documentclass[20pt]{article}

\input{../../../latex_preambule_style/preambule}
\input{../../../latex_preambule_style/styleCourslycee}
\input{../../../latex_preambule_style/styleExercices}
%\input{../../latex_preambule_style/styleCahier}
\input{../../../latex_preambule_style/bas_de_page_seconde}
\input{../../../latex_preambule_style/algobox}



%%%%%%%%%%%%%%%  Affichage ou impression  %%%%%%%%%%%%%%%%%%
\newcommand{\impress}[2]{
\ifthenelse{\equal{#1}{1}}  %   1 imprime / affiche  -----    0 n'affiche pas
{%condition vraie
#2
}% fin condition vraie
{%condition fausse
}% fin condition fausse
} % fin de la procédure
%%%%%%%%%%%%%%%  Affichage ou impression  %%%%%%%%%%%%%%%%%%



%%%%%%%%%%%%%%%  Indentation  %%%%%%%%%%%%%%%%%%
\parindent=0pt
%%%%%%%%%%%%%%%%%%%%%%%%%%%%%%%%%%%%%%%%%%%%%%%%



\begin{document}


\subsection{Dessiner la représentation graphique d'une fonction affine}

\Exe

Soit $f$ définie sur $\R$ par $f(x)= \frac{1}{3}x+2$.

\begin{enumerate}
\item Déterminer les variations de $f$ sur son domaine de définition.
\item Construire la représentation graphique de la fonction de  $f$.
 \end{enumerate}

\Exe

Soit $f$ définie sur $\R$ par $f(x)= -\frac{2}{5}x-1$.

\begin{enumerate}
\item Déterminer les variations de $f$ sur son domaine de définition.
\item Construire la représentation graphique de la fonction $f$.
\end{enumerate}


\subsection{Déterminer une fonction affine connaissant deux points et leurs images}

\Exe 

Soit $f$ une fonction affine telle que $f(4)=2$ et $f(6)=-1$. 
\begin{enumerate}
\item Déterminer les variations de $f$ sans effectuer de calcul.
\item Déterminer la fonction $f$
\end{enumerate}

\Exe

Construire un exercice sur le modèle précédent et donner sa correction.


\subsection{Déterminer le signe de $ax+b$}

\Exe

Soit $f$ la fonction définie sur $\R$ par f(x)= $-3x+2$.
Dresser le tableau de signe de $f$ sur son domaine de définition.

\Exe

Soit $f$ la fonction définie sur $[3;10]$ par f(x)= $\sqrt{5}x-2$.
Dresser le tableau de signe de $f$ sur son domaine de définition.



\subsection{Pour raisonner}

\EPCNA{Raisonner. Chercher}

Est-il possible de déterminer une fonction affine $f$ tel que la fonction $g$ définie par $g(x)=f(f(x))$ soit décroissante ?

\EPCNA{Chercher. Calculer}

En fonction des valeurs de $t$, étudier le signe de $f(x)=(3t^2-t)x+1$

\end{document}
