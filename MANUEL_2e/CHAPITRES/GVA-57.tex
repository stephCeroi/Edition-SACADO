 
Cet exercice propose deux énoncés. De prime abord, lequel de ces deux énoncés vous parait le plus simple ?
Résoudre les deux énoncés.
\begin{enumerate}[leftmargin=*]
\item On considère dans un repère \Oij le point $A(1;3)$ et le cercle $\mathscr{C}$ de centre $A$ et de rayon $\sqrt{10}$.

Donner l'abscisse du point du cercle $\mathscr{C}$ d'ordonnée 2.
 
\item On considère dans un repère \Oij le point $A(1;3)$ et le cercle $\mathscr{C}$ de centre $A$ et de rayon 2.

Donner l'abscisse du point du cercle $\mathscr{C}$ d'ordonnée 4. 
\end{enumerate}
Conclure sur la difficulté des exercices.
