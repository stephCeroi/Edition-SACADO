
Un entomologiste a relevé durant les mois de mai et juin le nombre quotidien de papillon dans un pré.

\begin{tabular}{|c|c|c|c|c|c|c|c|c|c|}
\hline 
Nombre de papillons &8 &10 &12 &14 &16& 18 &20& 22 &24\\
\hline 
Nombre de jours &5& 8 &10& 16& 4& 0& 0& 6& 12\\
\hline 
\end{tabular} 

\begin{enumerate}
\item Représenter cette série statistique par un nuage de points.

\begin{tikzpicture}[line cap=round,line join=round,>=triangle 45,x=0.4840308370044054cm,y=0.4989648033126286cm]
\begin{axis}[
x=0.4840308370044054cm,y=0.4989648033126286cm,
axis lines=middle,
ymajorgrids=true,
xmajorgrids=true,
xmin=-1.4675767918088716,
xmax=29.52218430034129,
ymin=-1.1203319502074895,
ymax=18.921161825726152,
xtick={-0.0,2.0,...,28.0},
ytick={-0.0,2.0,...,18.0},]
\clip(-1.4675767918088716,-1.1203319502074895) rectangle (29.52218430034129,18.921161825726152);
\draw (24.368600682593854,0.7883817427385716) node[anchor=north west] {Nombre de papillons};
\draw (0.10238907849829544,16.97095435684648) node[anchor=north west] {Nombre de jours};
\begin{scriptsize}
\draw [color=black] (8.,5.)-- ++(-2.5pt,0 pt) -- ++(5.0pt,0 pt) ++(-2.5pt,-2.5pt) -- ++(0 pt,5.0pt);
\draw [color=black] (10.,8.)-- ++(-2.5pt,0 pt) -- ++(5.0pt,0 pt) ++(-2.5pt,-2.5pt) -- ++(0 pt,5.0pt);
\draw [color=black] (12.,10.)-- ++(-2.5pt,0 pt) -- ++(5.0pt,0 pt) ++(-2.5pt,-2.5pt) -- ++(0 pt,5.0pt);
\draw [color=black] (14.,16.)-- ++(-2.5pt,0 pt) -- ++(5.0pt,0 pt) ++(-2.5pt,-2.5pt) -- ++(0 pt,5.0pt);
\draw [color=black] (16.,4.)-- ++(-2.5pt,0 pt) -- ++(5.0pt,0 pt) ++(-2.5pt,-2.5pt) -- ++(0 pt,5.0pt);
\draw [color=black] (18.,0.)-- ++(-2.5pt,0 pt) -- ++(5.0pt,0 pt) ++(-2.5pt,-2.5pt) -- ++(0 pt,5.0pt);
\draw [color=black] (20.,0.)-- ++(-2.5pt,0 pt) -- ++(5.0pt,0 pt) ++(-2.5pt,-2.5pt) -- ++(0 pt,5.0pt);
\draw [color=black] (22.,6.)-- ++(-2.5pt,0 pt) -- ++(5.0pt,0 pt) ++(-2.5pt,-2.5pt) -- ++(0 pt,5.0pt);
\draw [color=black] (24.,12.)-- ++(-2.5pt,0 pt) -- ++(5.0pt,0 pt) ++(-2.5pt,-2.5pt) -- ++(0 pt,5.0pt);
\end{scriptsize}
\end{axis}
\end{tikzpicture}


\item  Dresser le tableau des effectifs cumulés croissants.

\begin{tabular}{|c|c|c|c|c|c|c|c|c|c|}
\hline 
Nombre de papillons &8 &10 &12 &14 &16& 18 &20& 22 &24\\
\hline 
Nombre de jours &5& 8 &10& 16& 4& 0& 0& 6& 12\\
\hline 
ECC & 5& 13 &23& 39& 43& 43& 43& 49& 61\\
\hline 
ECD &61& 56 &48& 38& 22& 18& 18& 18& 12\\
\hline

\end{tabular} 


\item  Combien de jours y a-t-il douze papillons au moins dans le pré ?
\vspace{0.5cm}

\textbf{Il y a 48 jours où le nombre de papillons est supérieur à 12.}


\vspace{0.5cm}

Attention ! \textbf{Au moins} 12 signifie \textbf{12 ou plus}. Donc il faut construire les \textbf{effectifs cumulés décroissants} !

\end{enumerate}