\begin{titre}[Problème du premier degré]

\Titre{Variations et signe de $ax+b$}{2}
\end{titre}


\begin{CpsCol}
\textbf{Variations de fonctions}
\begin{description}
\item[$\square$] Donner le sens de variation d'une fonction affine
\item[$\square$] Donner le tableau de signe de $ax + b$ pour des valeurs numériques données de $a$ et $b$
\item[$\square$] Résoudre une équation, une inéquation produit ou quotient, à l'aide d'un tableau de signes.
\end{description}
\end{CpsCol}

\Rec{1}{FPD-4}


\begin{Pp}\index{Fonctions affines! Variations}
Soit $f$ une fonction affine de la forme $f(x)=ax+b$, avec $a\in \R^*$ et $b \in \R$.
\begin{description}
\item[•] Si $a >0$ alors $f$ est strictement croissante sur $\R$.
\item[•] Si $a <0$ alors $f$ est strictement décroissante sur $\R$.
\end{description} 
\end{Pp}

\ROC

\Rec{1}{FPD-5}

\paragraphe{Les inéquations produits nul}

\begin{Def}
$a, b, c, d$ sont quatre nombres réels connus.

Une \textit{in}équation-produit-nul est une des inéquations suivantes :
\begin{description}
\item[•] $(ax+b)(cx+d) \leq  0$
\item[•] $(ax+b)(cx+d) < 0$
\item[•] $(ax+b)(cx+d) \geq 0$
\item[•] $(ax+b)(cx+d) > 0$
\end{description}
\end{Def}

\begin{Rq}
Tout inéquation qui se factorise en une de ces 4 inéquations peut se résoudre avec les méthodes ci-dessous. 
\end{Rq}

\begin{Mt}
Une \textit{in}équation-produit-nul peut se résoudre par un tableau de signe.
\end{Mt}


On souhaite résoudre $(3x+2)(-4x+1) \leq  0$.

\begin{Mt}
\begin{enumerate}
\item On étudie le signe de $3x+2$. On pourra résoudre $3x+2 \leq  0$.
\item On étudie le signe de $-4x+1$. On pourra résoudre $-4x+1 \leq  0$.
\item On place dans un tableau des signes des expressions.
\item On conclut.
\item On vérifie sur GGB ou sur sa calculatrice le résultat.
\end{enumerate}
\end{Mt}

\paragraphe{Résolution}

\begin{enumerate}
\item $3x+2 \leq  0 \Longleftrightarrow 3x \leq  2  \Longleftrightarrow x \leq \frac{2}{3} $
\item $-4x+1 \leq  0 \Longleftrightarrow -4x \leq -1  \Longleftrightarrow x \geq \frac{1}{4} $. On pense à la propriété : Lorsqu'on multiplie ou l'on divise par un nombre négatif, on change le sens de l'ordre.
\item 
\begin{tabular}{|c|ccccccc|}
\hline 
$x$ & $-\infty$ &   & $\frac{1}{4}$ &   & $\frac{2}{3}$ &  & $+\infty$ \\ 
\hline 
signe de $3x+2$ &   & $-$ &  & $-$ & 0 & + &  \\ 
\hline 
signe de $-4x+1$ &   & $+$ & 0 & $-$ & & $-$ &  \\ 
\hline 
signe de $(3x+2)(-4x+1)$ &   &$-$  & 0 &  $+$ & 0 & $-$&  \\ 
\hline 
\end{tabular} 

\item Comme on souhaite  $(3x+2)(-4x+1) \leq  0$, on a alors $S = \left]- \infty; \frac{1}{4} \right] \cup \left[ \frac{2}{3} ; +\infty  \right[$


\item 

\begin{tikzpicture}[line cap=round,line join=round,>=triangle 45,x=3.220064724919094cm,y=1.0cm]
\begin{axis}[
x=3.220064724919094cm,y=1.0cm,
axis lines=middle,
ymajorgrids=true,
xmajorgrids=true,
xmin=-0.9095477386934674,
xmax=0.6432160804020101,
ymin=-1.6799999999999997,
ymax=3.0599999999999996,
xtick={-0.5,0.0,...,0.5},
ytick={-1.0,0.0,...,3.0},]
\clip(-0.9095477386934674,-1.68) rectangle (0.6432160804020101,3.06);
\draw [samples=50,rotate around={-180.:(-0.20833333333333334,2.5208333333333335)},xshift=-0.6708468176914779cm,yshift=2.5208333333333335cm,line width=2.pt,domain=-0.9166666666666666:0.9166666666666666)] plot (\x,{(\x)^2/2/0.041666666666666664});
\end{axis}
\end{tikzpicture}

\end{enumerate}