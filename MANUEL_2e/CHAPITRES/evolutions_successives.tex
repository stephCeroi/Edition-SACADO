\documentclass[10pt]{article}

\input{../../../latex_preambule_style/preambule}
\input{../../../latex_preambule_style/styleCourslycee}
\input{../../../latex_preambule_style/styleExercices}
%\input{../../latex_preambule_style/styleCahier}
\input{../../../latex_preambule_style/bas_de_page_seconde}
\input{../../../latex_preambule_style/algobox}



%%%%%%%%%%%%%%%  Affichage ou impression  %%%%%%%%%%%%%%%%%%
\newcommand{\impress}[2]{
\ifthenelse{\equal{#1}{1}}  %   1 imprime / affiche  -----    0 n'affiche pas
{%condition vraie
#2
}% fin condition vraie
{%condition fausse
}% fin condition fausse
} % fin de la procédure
%%%%%%%%%%%%%%%  Affichage ou impression  %%%%%%%%%%%%%%%%%%



%%%%%%%%%%%%%%%  Indentation  %%%%%%%%%%%%%%%%%%
\parindent=0pt
%%%%%%%%%%%%%%%%%%%%%%%%%%%%%%%%%%%%%%%%%%%%%%%%
\begin{document}

\begin{titre}[Informations chiffrées]

\Titre{Évolutions successives}{4}
\end{titre}


\begin{CpsCol}
\begin{description}
\item[$\square$] Exploiter la relation entre deux valeurs successives et leur taux d'évolution.
\item[$\square$] Calculer le taux d'évolution global à partir des taux d'évolution successifs. 
\item[$\square$] Calculer un taux d'évolution réciproque.
\end{description}
\end{CpsCol}



\begin{Sit}
Le prix d'un produit augmente de $t$\% chaque année pendant $n$ années. Quelle est la hausse globale sur ces $n$ années ?

Vous devez étudier cette situation et l'étayer avec des exemples et essayer de déterminer une formule générale. Par exemple, vous pourrez regarder ce qui se passe pour une augmentation de $3$\% pendant 8 années.
\end{Sit}



\begin{Sit}
Une quantité augmente de $t$\% la première année puis diminue de $t$\% la seconde année. 

Comment varie cette quantité sur ces deux années ?

Vous devez faire une explication claire de votre étude et démontrer votre résultat. 
\end{Sit}



\begin{Sit}
Le prix d'un produit varie de $t_1$\% la première année, de $t_2$\% la deuxième année, de $t_3$\% la troisième année, de $t_4$\% la quatrième année, de $t_5$\% la cinquième année. Quelle est la hausse globale sur ces 5 années ?

Vous devez étudier cette situation et l'étayer avec des exemples et essayer de déterminer une formule générale. 
\end{Sit}



\begin{Sit}
Une quantité $q_1$ augmente de $t$\%. On obtient une nouvelle quantité $q_2$. Quel doit être le taux de pourcentage $t'$ qu'il faut appliquer à la quantité $q_2$ pour retrouver $q_1$ ?

Vous devez étudier cette situation et l'étayer avec des exemples et essayer de déterminer une formule générale. 
\end{Sit}


\begin{Sit}
Le prix d'un produit augmente de $t$\% chaque année pendant $n$ années. Quelle est le taux moyen d'augmentation par année ?


Vous devez étudier cette situation pour $n=2$, puis $n=3$ et l'étayer avec des exemples et essayer de déterminer une formule générale pour $n$ années.  
\end{Sit}

\begin{Rap}

Un taux d'évolution de $t$, exprimé en pourcentage, est associé au coefficient multiplicateur, noté $CM$ égal à : $$CM=1+\frac{t}{100}$$

\end{Rap}

\mini{
\EPC{1}{IC-34}{Chercher. Calculer.}

%\EPC{1}{IC-16}{Chercher.}
\EPC{1}{IC-36}{Chercher. Calculer.}


\EPC{0}{IC-5}{Chercher. Calculer.}
}{
\EPC{1}{IC-33}{Chercher. Calculer.}

\EPC{1}{IC-3}{Chercher. Calculer.}
}


\EPC{0}{IC-9}{Chercher. Communiquer.}



  
 


\mini{
\EPC{1}{IC-39}{Chercher. Calculer.}

\EPC{1}{IC-7}{Chercher. Représenter.}

\EPCP{1}{IC-40bis}{Chercher. Communiquer.}
}{
\EPC{1}{IC-35}{Chercher. Calculer.}

\EPC{0}{IC-37}{Chercher. Calculer.}
 
\EPC{1}{IC-38}{Chercher. Calculer.} 

\EPCP{1}{IC-40}{Chercher. Communiquer.}
}
 
 
%\EPC{1}{stat-35}{Chercher. Modéliser.}
%
%\ligne{30}
% 
%\multido{\i=0+1}{87}{ \AD{1}{FEA-\i} }
\end{document}
