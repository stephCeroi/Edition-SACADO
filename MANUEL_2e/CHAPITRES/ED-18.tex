
\subsubsection{niveau 1}

Soit $EFGH$ un parallélogramme de centre $O$.

$I$ et $J$ sont les milieux respectifs de $[EF]$ et $[FG]$.

Les droites $(HI)$ et $(HJ)$ coupent respectivement $[EG]$ en $M$ et $N$.

\textit{L'objectif est de démontrer que $EM=MN=NG$}

\begin{enumerate}
\item En considérant le repère (E,EF,EH), exprimer les coordonnées des points $E$, $F$, $G$, $H$, $I$ et $J$.
\item 
\begin{enumerate}
\item Déterminer une équation de la droite $(HI)$
\item Quelle est l'abscisse du point $M$ ?
\item En déduire les coordonnées du point $M$.
\end{enumerate}
\item Déterminer les coordonnées du point $N$.
\item Conclure.
\end{enumerate}

\subsubsection{niveau 2}

Soit $EFGH$ un parallélogramme de centre $O$.

$I$ et $J$ sont les milieux respectifs de $[EF]$ et $[FG]$.

Les droites $(HI)$ et $(HJ)$ coupent respectivement $[EG]$ en $M$ et $N$.

\textit{L'objectif est de démontrer que $EM=MN=NG$}

\begin{enumerate}
\item En considérant le repère (E,EF,EH), exprimer les coordonnées des points $E$, $F$, $G$, $H$, $I$ et $J$.
\item 
\begin{enumerate}
\item Déterminer une équation de la droite $(HI)$
\item Quelle est l'abscisse du point $M$ ?
\item En déduire les coordonnées du point $M$.
\end{enumerate}
\item Conclure.
\end{enumerate}


\subsubsection{niveau 3}

Soit $EFGH$ un parallélogramme de centre $O$.

$I$ et $J$ sont les milieux respectifs de $[EF]$ et $[FG]$.

Les droites $(HI)$ et $(HJ)$ coupent respectivement $[EG]$ en $M$ et $N$.

\textit{L'objectif est de démontrer que $EM=MN=NG$}

\begin{enumerate}
\item En considérant le repère (E,EF,EH), exprimer les coordonnées des points $E$, $F$, $G$, $H$, $I$ et $J$.
\item Conclure.
\end{enumerate}


\subsubsection{niveau 4}

Soit $EFGH$ un parallélogramme de centre $O$.

$I$ et $J$ sont les milieux respectifs de $[EF]$ et $[FG]$.

Les droites $(HI)$ et $(HJ)$ coupent respectivement $[EG]$ en $M$ et $N$.

\textit{L'objectif est de démontrer que $EM=MN=NG$}