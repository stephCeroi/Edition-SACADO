\begin{titre}[Informations chiffrées]

\Titre{Proportions et pourcentages}{4}
\end{titre}


\begin{CpsCol}
\begin{description}
\item[$\square$] Exploiter la relation entre effectifs, proportions et pourcentages.
\item[$\square$] Traiter des situations simples mettant en jeu des pourcentages de pourcentages.
\end{description}
\end{CpsCol}

 

\begin{DefT}{Proportion}\index{Proportion}
Soient $E$ un ensemble de référence non vide et $n_E$ le nombre d'élément de $E$.
Soient $A$ une partie de $E$ (un sous ensemble de $E$) et $n_A$ le nombre d'élément de $A$
On appelle \textbf{proportion} $p$ de $A$ dans $E$ le réel défini par $p=\frac{n_A}{n_E}$ 
\end{DefT}

\begin{Ex}
Lors d'une élection, on recense 1648 électeurs. Pourtant, seuls 1236 se sont déplacés aux urnes. La proportion d'électeurs votants est donc : $p=\frac{1236}{1648}=0,75=\frac{75}{100}=75\%$ 
\end{Ex}



\begin{Th}
Soit $p$ le réel défini par $p=\frac{n_A}{n_E}$. Alors $n_A=p \times n_E$.

Et pour $p\neq 0$, $n_E=\frac{n_A}{p}$.
\end{Th}


\begin{Th}
A tout ensemble $A$ contenu dans un ensemble $E$ non vide, on a : $0 \leq p \leq 1$.
\end{Th}


%Exercices



\mini{
\EPC{1}{IC-8}{Calculer}

\EPC{0}{IC-11}{Calculer}
}{
\EPC{1}{IC-12}{Calculer.}

\EPC{0}{IC-19}{Calculer.}
}






\begin{ThT}{Pourcentage de pourcentage} \index{Pourcentage!Pourcentage de pourcentage}
Soit $F$ un ensemble non vide de référence, $E$ une partie non vide de $F$ et $A$ une partie de $E$.\\
Si $p_1$ est la proportion de $A$ dans $E$ et si $p_2$ est la proportion de $E$ dans $F$ alors la proportion de $A$ dans $F$ est $p=p_1 \times p_2$.
\end{ThT}

\begin{Ex}
\textit{Lors d'une élection, on recense 1648 électeurs inscrits sur les listes électorales mais seuls 1236 se sont déplacés aux urnes. Le candidat  sortant   a obtenu 40\% des voix des votants. Quel est la proportion des voix par rapport aux inscrits ?  }

La proportion de votants est $p=\frac{1236}{1648}=0,75=\frac{75}{100}=75\%$.

Parmi les votants, le candidat sortant a obtenu 36\% des voix des votants donc $p = 0,4 \times 0,75 = 0,3$.

Ainsi, le candidat sortant a recueilli 30\% des voix des inscrits.
\end{Ex}


\mini{
\EPC{1}{IC-15}{Calculer.}

\EPC{1}{IC-20}{Calculer.}

\EPC{1}{IC-14}{Représenter.}

\EPC{0}{IC-23}{Raisonner.}
}{

\EPC{0}{IC-13}{Représenter.}

\EPCP{1}{IC-21}{Chercher.}

\EPC{0}{IC-22}{Représenter.}
}



\EPC{1}{IC-17}{Représenter.}

\EPC{1}{IC-18}{Représenter.}
 