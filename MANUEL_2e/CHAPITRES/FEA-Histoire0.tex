 
\begin{titre}[Nombres et calculs]

\TitreSansTemps{L'histoire des irrationnels}
\end{titre}

 




On accorde aux Grecs la découverte de l’incommensurabilité, un phénomène géométrique qu’ils auraient qualifié d’irrationnel dès le départ. Or, le mot «irrationnel» est polysémique. On pense parfois qu’il porte la nature insensée de ces nombres, mais il proviendrait, en fait, du contraire du «ratio» (rapport) qui était à la source des nombres pour les Grecs. 

Pour les mathématiciens de cette période de l’histoire (environ 500 ans avant notre ère), il  semble  qu’il  paraissait  impossible  que  des  mesures  ne  puissent  pas  admettre  une  unité  de mesure commune. Les livres d’histoire des mathématiques décrivent cet épisode comme le  «scandale  des  irrationnelles  (sous-entendu:  des  grandeurs  irrationnelles)».


Pour les pythagoriciens, cette idée d’irrationalité amenée par Hippase de Métaponte, un disciple immédiat de Pythagore (569-475 avant notre ère), semble avoir été scandaleuse, car leur vision du monde venait de s’écrouler (Mankiewicz, 2000; Guedj, 1998; Le Lionnais,  1983, Guillen,  1983;  Kline,  1972;  Desanti,  1967).  Certains  écrits  parlent  même  d’une  «trahison  criminelle envers la doctrine de Pythagore» (Szabo, 1977).  


Davis et Hersh (1982) mentionnent que les Babyloniens avaient trouvé, vers le XVIIe siècle avant notre ère, une excellente approximation de $\sqrt 2$, équivalent à 1,414232963. Toutefois,  il  fallut  attendre  quelques  siècles  avant  d’en  arriver  à  la  démonstration  de  l’irrationalité  de $\sqrt 2$,  identifiée  par  le  mathématicien  anglais  Godfrey  Hardy  (1877-1947)  comme  un  exemple  important  de  la  beauté  des  mathématiques  (Papert,  1980).  

Cette  démonstration par l’absurde était courante à l’époque d’Aristote (384-322 avant notre ère) et elle se retrouve dans un des livres des Éléments d’Euclide (325-265 avant notre ère). Ces écrits comportaient plusieurs définitions importantes, dont celles des segments commensurables et incommensurables, ce qui a grandement aidé à légitimer l’incommensurabilité en géométrie.

Par  la  suite,  l’absence  d’écrits  mathématiques  sur  une  longue  période  à  propos  des  nombres  irrationnels  laisse  penser  qu’il  a  fallu  attendre  plusieurs  années  pour  que  ce  concept subisse une certaine évolution. Il y a tout de même $\pi$ qui a beaucoup intéressé les mathématiciens durant une longue période. L’évolution de la compréhension de ce nombre est assez bien documentée (Sykes, 2000; Tent, 2001).

De  nos  jours,  le  nombre  irrationnel  est  souvent  associé  au  nombre  à  virgule.  Tel  que  mentionné  plus  tôt,  l’arrivée  des  nombres  à  virgule  se  situe  beaucoup  plus  tard  au  cours  de  l’histoire,  soit  en  1585.  Cela  s’est  fait  grâce  à  Simon  Stevin  (1548-1620)  qui  a  réussi  à  imposer  sa  disme,  qui  veut  dire  «dixième».  Il  est  intéressant  de  constater  que  l’écriture  en  base 10 proposée par Stevin tient compte de la notion d’incommensurabilité. Les nombres à virgule ont donc joué un rôle dans l’évolution du concept de nombre irrationnel. Comme le mentionnent Arcavi, Bruckheimer et Ben-Zvi (1987), l’intérêt pour la nature de ces nombres est devenu beaucoup plus répandu à ce moment. Malgré tout, des mathématiciens de cette époque  continuaient  de  banaliser  l’utilité  des  irrationnels  (Ifrah,  1994).  Ceux-ci  hésitaient  même à les considérer comme des nombres.


C’est  Richard  Dedekind  (1831-1916)  qui  viendra  légitimer  les  nombres  irrationnels  en  proposant des définitions et des règles pour leur utilisation. Son raisonnement se réfère à la droite numérique et se base, entre autres, sur le fait que des segments de droites peuvent avoir un rapport incommensurable. En effet, sans les nombres irrationnels, la droite numérique est pleine de «trous» (Dedekind, 1963).

C’est à partir de cette époque que les nombres irrationnels sont devenus de plus en plus acceptés. Plus ils ont été acceptés, plus ils ont joué un rôle majeur en science.La science a un réel besoin des nombres irrationnels. Et cela fait déjà bien plus  d’un  siècle  que  les  scientifiques  ont  noté  qu’un  nombre  croissant  de quantités assez particulières faisait leur entrée dans presque chaque théorie scientifique, signifiant par là leur importance dans les descriptions modernes de l’espace-temps. (Guillen, 1983)

L’étude du contexte historique du nombre irrationnel donne un éclairage à notre étude en dégageant des enjeux de l’enseignement du nombre irrationnel au secondaire. Par ce survol, il apparait possible de réunir des concepts connexes, bien documentés par la recherche, dont celui de rapport (Gheverghese Joseph, 1997; Harrison et al., 1989; Lamon, 1993; Singh, 2000), de  commensurabilité  (Solomon,  1987),  d’incommensurabilité  (Crone,  1955;  Kline,  1972;  Rusnock et Thagard, 1995; Sfard, 1991), de périodicité (Shama, 1998), de mesure (Davydov et Tsvetkovich, 1991; Hiebert, 1984; Jensen et O’Neil, 1981; Sterling,  1998), d’approximation (Hall,  1984;  Kawahara  Lang,  2001;  Menon,  2003;  Montagne  et  Van  Garderen,  2003;  Ronau,  1988;  Siegler  et  Booth,  2004;  Thompson,  1979)  et  de  droite  numérique  (Carr  et  Katterns, 1984; Ernest, 1985; Kennedy, 2000; Kurland, 1990).


 