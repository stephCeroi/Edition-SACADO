

On donne le tableau suivant.

\begin{tabular}{|c|c|c|c|c|}
\hline 
 & A & B & C & Vrai/Faux \\ 
\hline 
FIGURE 1 & $(-1;2)$ & $(6;0,5)$ & $(-1;2,5)$ &  \\ 
\hline 
FIGURE 2 & $(-3;6)$ & $(-1;2)$ & $(-7;1)$ &  \\ 
\hline 
FIGURE 3 & $(3;-2)$ & $(-2;-3)$ & $(-4;3)$ &  \\ 
\hline 
FIGURE 4 & $(-1;-1)$ & $(1;3)$ & $(5;1)$ & \\ 
\hline 
FIGURE 5 & $(-0,5;1,5)$ & $(1,5;1)$ & $(-1;-0,5)$ &  \\ 
\hline 
FIGURE 6 & $(4;3)$ & $(-1;0)$ & $(3;4)$ &  \\ 
\hline 
\end{tabular} 

\vspace{1cm}
 
\begin{minipage}{0.48\linewidth}

Pour chacune des figures, on souhaite dire si le triangle est rectangle ou pas.

\begin{enumerate}
\item En combien de point le triangle peut-il être rectangle ?
\item Compléter l'algorithme ci-dessous.

\begin{tabular}{l}
\hline 
a $\longleftarrow -1$ \#abscisse de A \\
b $\longleftarrow 6$ \#ordonnée de A \\
c $\longleftarrow 2$ \#abscisse de B \\
d $\longleftarrow 0,5$ \#ordonnée de B \\
e $\longleftarrow -1$  \#abscisse de C \\
f $\longleftarrow 2,5$  \#ordonnée de C\\

$ABcarre = ( \ldots - a )^2 + ( d - \ldots )^2$\\
$BCcarre = ( \ldots - \ldots )^2 + ( \ldots - \ldots )^2$\\
$ACcarre = ( \ldots - \ldots )^2 + ( \ldots - \ldots )^2$\\

Si $ \ldots \ldots \ldots \ldots \ldots \ldots \ldots \ldots \ldots \ldots$ = $\ldots \ldots \ldots $\\
alors $ \ldots \ldots \ldots \ldots \ldots \ldots \ldots \ldots \ldots \ldots \ldots \ldots \ldots $\\
sinon Si $ \ldots \ldots \ldots \ldots \ldots \ldots \ldots \ldots \ldots \ldots$ = $\ldots \ldots \ldots $\\
alors $ \ldots \ldots \ldots \ldots \ldots \ldots \ldots \ldots \ldots \ldots \ldots \ldots \ldots $\\
sinon Si $  \ldots \ldots \ldots \ldots \ldots\ldots \ldots \ldots \ldots \ldots $ = $\ldots \ldots \ldots $\\
alors $ \ldots \ldots \ldots \ldots \ldots \ldots \ldots \ldots \ldots \ldots\ldots \ldots \ldots $\\
sinon $ \ldots \ldots \ldots \ldots \ldots \ldots \ldots \ldots \ldots \ldots\ldots \ldots \ldots $\\
\hline 
\end{tabular} 
\end{enumerate}
\end{minipage}
\hfill
\begin{minipage}{0.48\linewidth}
Python est un langage de programmation orienté objet extrêmement puissant et simple dans sa syntaxe. Voici un code Python incomplet.

A l'aide de l'algorithme, compléter ce code.


\begin{tabular}{l}
\hline 
a = float(input("entrer l'abscisse du point A"))\\
b = float(input("entrer l'ordonnée du point A"))\\
c = float(input("entrer l'abscisse du point A"))\\
d = float(input("entrer l'ordonnée du point A"))\\
e = float(input("entrer l'abscisse du point A"))\\
f = float(input("entrer l'ordonnée du point A")) \\

$ABcarre = ( \ldots - a )**2 + ( d - \ldots )**2$\\
$BCcarre = ( \ldots - \ldots )**2 + ( \ldots - \ldots )**2$\\
$ACcarre = ( \ldots - \ldots )**2 + ( \ldots - \ldots )**2$\\


if $  \ldots \ldots \ldots \ldots \ldots\ldots \ldots \ldots \ldots \ldots $ == $ \ldots \ldots \ldots \ldots \ldots\ldots \ldots \ldots $ :\\
\hspace{1cm} print("Le triangle ABC est rectangle en ......")\\
elif $  \ldots \ldots \ldots \ldots \ldots\ldots \ldots \ldots \ldots \ldots $ == $ \ldots \ldots \ldots \ldots \ldots\ldots \ldots \ldots $ :\\
\hspace{1cm} print("Le triangle ABC est rectangle en ......")\\
elif $  \ldots \ldots \ldots \ldots \ldots\ldots \ldots \ldots \ldots \ldots $ == $ \ldots \ldots \ldots \ldots \ldots\ldots \ldots \ldots $ :\\
\hspace{1cm} print("Le triangle ABC est rectangle en ......")\\
else :\\
\hspace{1cm} print("Le triangle ABC n'est pas rectangle.")\\
\hline 
\end{tabular} 

\end{minipage}

\vspace{1cm}

Analyse du code Python
\vspace{0.2cm}

\begin{enumerate}
 
	\item Quelles remarques pouvez-vous faire sur ce code ?
	\item Compléter ce programme en Python.
	\item Ouvrir le logiciel EduPython. Tapez le code en respectant les indentations et exécuter-le en utlisant la flèche verte.
 
\item Quelles améliorations pourrait-on proposer pour ce code ?
\end{enumerate}